\subsection{Taskname}
The DFT module is invoked by specifying:
\begin{verbatim}
  task dft
\end{verbatim}

\subsection{Basis Sets}
The DFT module uses up to three different basis sets, specified 
in the input file as described in the NWCHEM user documentation.
These bases are recognized according to
the following values for the $<$string name$>$ field of the basis input:
\begin{itemize}
\item {\tt "ao basis"}  ({\sl required}) \\
   The basis set for the (Kohn-Sham) molecular orbitals.
\item {\tt "cd basis"} ({\sl recommended}) \\
   The charge density fitting basis set used for the
  evaluation of the Coulomb potential according to the Dunlap scheme
  (B.I.~Dunlap, J.W.D.~Connolly and J.R.~Sabin, J.~Chem.~Phys.~{\bf 71},
  4993 (1979)).  If no basis set is specified,
  the $O(N^4)$ exact Coulomb contribution is computed. 
\item {\tt "xc basis"}  ({\sl optional})\\
   The fitting basis for the evaluation of the
  exchange-correlation potential.  If not specified,
  the XC contribution will be evaluated by numerical quadrature.
  At present, the quadrature is efficient enough so that this option
  is rarely used.
\end{itemize}

\subsection{Spin Multiplicity}

Both {\sl closed-shell} and {\sl open-shell} systems can be studied. 
The spin multiplicity is specified be means of the keyword {\tt MULT}.
(where \verb+mult+ is equal to the number of alpha electrons minus
beta electrons +1).

\begin{verbatim}
   mult <integer mult> 
\end{verbatim}

\subsection{Charge}

The charge is NO LONGER specified in the dft context.  See section 
\ref{sec:charge}.

\subsection{Guess (Initial MO Vectors)}

If a file {\tt movecs} is found in the working directory, the 
guess keyword is ignored, and the initial Kohn-Sham orbitals are read in.
If there is no {\tt movecs} file in the working directory,
two options are recognized for initial orbitals:
\begin{itemize}
\item (Default) start from a density matrix that is the superposition of HF
  atomic solutions:
\begin{verbatim}
   guess atomic
\end{verbatim}
\item start from a one-electron (core) hamiltonian eigenvectors:
\begin{verbatim}
   guess hcore
\end{verbatim}
\end{itemize}

\subsection{SCF Optimization Control}

The default optimization is by iteration of the Kohn-Sham equations
without damping for up to 30 iterations.  The following keywords
control the optimization:
\begin{itemize}
\item Maximum number of SCF (Kohn-Sham) iterations (default 30):
\begin{verbatim}
   itrscf <integer itrscf>
\end{verbatim}
\item Use direct inversion of the iterative subspace (DIIS) optimization
  (P.~Pulay,  Chem.\ Phys.\ Lett.\ {\bf 73}, 393 (1980) and P.~Pulay,
  J.~Comp.~Chem.~{\bf 3}, 566 (1982)):
\begin{verbatim}
   diis
\end{verbatim}
  {\tt nfock} Fock matrices are used to compute the DIIS step
  (Default value is 10):
\begin{verbatim}
   nfock <integer nfock>
\end{verbatim}
\item Density matrix damping. Damp by {\tt damp} percent (default 0\%):
\begin{verbatim}
   damp <integer ndamp>  
\end{verbatim}
\item Fock matrix level shifting (M.F.~Guest and V.R.~Saunders,
  Mol.~Phys.~{\bf 28}, 819 (1974)). The diagonal elements of
the Fock matrix corresponding to virtual orbitals are shifted by 
{\tt  lshift} Hartree (default 0):
\begin{verbatim}
   lshift <real lshift> 
\end{verbatim}
The level shifting procedure may be switched off after {\tt ncyshft} cycles
(default 99):
\begin{verbatim}
   ncyshft <integer ncyshft> 
\end{verbatim}
\end{itemize}

\subsection{SCF Convergence Tolerances}
\begin{itemize}
\item Delta energy convergence tolerance $10^{-{\tt iscfcon}}$(default 7):
\begin{verbatim}
  scfcon <integer iscfcon>
\end{verbatim}
\item Delta density convergence tolerance $10^{-{\tt igcon}}$
  (default -1 [meaning ignore]):
\begin{verbatim}
  gcon <integer igcon>
\end{verbatim}
\item DIIS error norm convergence tolerance $10^{-{\tt idiisoff}}$ 
  (only used with DIIS, default 7):
\begin{verbatim}
  idiisoff <integer idiisoff>
\end{verbatim}
\end{itemize}


\subsection{I/O Operations}

Disk storage of inverted CD and XC matrices may be turned off 
(they are written out by default) with the keyword:
\begin{verbatim}
   noio
\end{verbatim}

\subsection{Exchange and Correlation Functionals}
The default exchange-correlation treatment uses the Slater exchange 
functional and the Vosk-Wilk-Nusair correlation functional.
The following functionals are available:
\paragraph{Exchange}
\begin{itemize}
\item  Slater $\rho^{1/3}$ functional; J.C.~Slater, 
  {\sl Quantum Theory of Molecules and
  Solids, Vol.~4: The Self-Consistent Field for Molecules and Solids}
(McGraw-Hill, New York, 1974):
\begin{verbatim}
   slater
\end{verbatim}
\item Becke gradient-corrected functional; 
  A.D.~Becke, J.~Chem.~Phys.~88, 3098 (1988):
\begin{verbatim}
   becke88
\end{verbatim}
\item Hartree-Fock exact exchange ($O(N^4)$):
\begin{verbatim}
   HFexch
\end{verbatim}
\end{itemize}

\paragraph{Correlation}
%{\bf Correlation}
\begin{itemize}
\item VWN Local density functional; S.J.~Vosko, L.~Wilk and M.~Nusair, 
  Can.~J.~Phys.~{\bf  58}, 1200 (1980); the parameters used in this
  formula are obtained by fitting to the {\bf Ceperley \&
  Alder\footnotemark[1]} Quantum
  MonteCarlo solution of the {\bf homogenous electron gas}.
\begin{verbatim}
   vwn
\end{verbatim}
\item VWN Local density functional; S.J.~Vosko, L.~Wilk and M.~Nusair, 
  Can.~J.~Phys.~{\bf  58}, 1200 (1980); ; the parameters used in this
  formula are obtained by fitting to the {\bf RPA} solution to the
  homogenous electron gas.
\begin{verbatim}
   vwnrpa
\end{verbatim}
\item Perdew \& Wang 1991 Local density functional; the parameters used in this
  formula are obtained by fitting to the {\bf Ceperley \&
  Alder\footnotemark[1]} Quantum
  MonteCarlo solution of the {\bf homogenous electron gas}.
\begin{verbatim}
pw91lda
\end{verbatim}
\item LYP Gradient-corrected functional; 
  C.~Lee, W.~Yang and R.~G.~Parr, Phys.~Rev.~B {\bf 37}, 785 (1988):
\begin{verbatim}
   lyp
\end{verbatim}
\item Perdew86 Gradient-corrected functional; J.~P.~Perdew, Phys.~Rev.~B 
  {\bf33}, 8822 (1986):
\begin{verbatim}
   perdew86
\end{verbatim}
\item Perdew91 Gradient-corrected functional;  J.P.~Perdew,
  J.A.~Chevary, S.H.~Vosko, K.A.~Jackson, M.R.~Pederson, D.J.~Singh and C.~Fiolhais,
Phys. Rev. B {\bf 46}, 6671 (1992).

\begin{verbatim}
   perdew91
\end{verbatim}
\end{itemize}
\footnotetext[1]{D.M.~Ceperley and B.J.~Alder, Phys. Rev. Lett. {\bf 45},
  566 (1980).}

\paragraph{Hybrid Methods}

\begin{itemize}
\item Becke ``{\sl half and half}'' exchange:
  A.D.~Becke, J.~Chem.~Phys.~98, 1372 (1992); the Exchange energy is
  computed as
\begin{eqnarray*}
E_{X} \ = \ \frac{1}{2} E^{\rm HF}_X + \frac{1}{2} E^{\rm Slater}_{X} +
\frac{1}{2} \Delta E^{\rm Becke88}_{X} 
\end{eqnarray*}

\begin{verbatim}
   beckehandh
\end{verbatim}

\item {\sl Adiabatic connection method}:
  A.D.~Becke, J.~Chem.~Phys.~98, 5648 (1993); the Exchange-Correlation
 energy is  computed as
\begin{eqnarray*}
E_{XC} \ &=& \ E^{\rm VWN}_C + a_0 E^{\rm HF}_X + (1-a_0) E^{\rm Slater}_{X} +
a_X \Delta E^{\rm Becke88}_{X} + a_C \Delta E^{Perdew91}_C \\
& &{\rm where } \\
a_0 &=& 0.20, \ a_X = 0.72, \ a_C = 0.81
\end{eqnarray*}



\begin{verbatim}
   acm
\end{verbatim}

\end{itemize}

\subsection{Integral Tolerances}
Three keywords provide control of screening for the evaluation of
the AO Gaussian functions, exchange-correlation (XC) Gaussian fitting 
functions, charge-density (CD) Gaussian fitting functions, 
and two-electron integrals:
\begin{itemize}
\item  A Gaussian orbital basis (AO) function with exponent $\zeta$
and radial factor $e^{-\zeta\cdot r_i^2}$ is 
evaluated  at a point $r_i$ only if 
$\zeta\cdot r_i^2 > {\tt iAOacc}$ (default {\tt iAOacc} value: 20)
\begin{verbatim}
   accAOfunc <integer iAOacc>
\end{verbatim}
\item Similarly, a XC fitting function with exponent $\zeta$
and radial factor $e^{-\zeta\cdot r_i^2}$ is 
evaluated  at a point $r_i$ only if 
$\zeta\cdot r_i^2 > {\tt iXCacc}$ (default {\tt iXCacc} value: 20)
\begin{verbatim}
   accXCfunc <integer iXCacc>
\end{verbatim}
\item As above, for the Coulomb fitting basis.  A  CD fitting function 
with exponent $\zeta$
and radial factor $e^{-\zeta\cdot r_i^2}$ is 
evaluated  at a point $r_i$ only if 
$\zeta\cdot r_i^2 > {\tt itol2e}$
In addition, the same
parameter {\tt itol2e} is used in Schwarz screening for the 
Coulomb integrals: only integrals with estimated
values greater than $10^{(-{\tt itol2e})}$ are evaluated. (Default
{\tt itol2e} value: 15):
\begin{verbatim}
   accCoul <integer itol2e>
\end{verbatim}
\end{itemize}
\subsection{Quadrature Tolerances}
\begin{itemize}
\item Two types of angular grid have been implemented.
  Gauss-Legendre (default):
\begin{verbatim}
   gausleg
\end{verbatim}
or Lebedev (routine supplied by M.~Caus\`a of the University of Torino):
\begin{verbatim}
   lebedev
\end{verbatim}
\item Choices of grid density (default is {\tt medium}), range from 
\begin{verbatim}
   coarse
\end{verbatim}
to
\begin{verbatim}
   xfine
\end{verbatim}
with the following definitions around each center:

  \begin{tabular}[right]{|l|r r r r|} \hline
Keyword & {\tt coarse} & {\tt medium} & {\tt fine} & {\tt xfine} \\ \hline
$N_{radial}$ & 30 & 40 & 60 & 90 \\
$N_{angular}$ & 6 & 8 & 10 & 10 \\  \hline
%$\Delta E/E$ $^a$  & & & &\\ 
%$\Delta \nabla E/|\nabla E|$$^a$  & & & &\\
%\multicolumn{5}{l}{$^a$ Based on calculations of 12-crown-6} \\
%\multicolumn{5}{l}{   with a 6-31g$*$ basis} \\
  \end{tabular}

\item Grid cutoff: for a given center (i.e.~atom) $A$, points that lie
more than {\tt iRmax} bohr from center $A$ are neglected. (Default value: 40):
\begin{verbatim}
   accQrad <integer iRmax>
\end{verbatim}
\end{itemize}

\subsection{Mulliken Population Analysis}
Mulliken analysis of the charge distribution is invoked by the keyword:
\begin{verbatim}
   mulliken
\end{verbatim}




%%% Local Variables: 
%%% mode: latex
%%% TeX-master: t
%%% End: 
