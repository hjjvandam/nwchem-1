\label{sec:dft}
\Large
******
It would be really nice if someone could supply a '25-words-or-less'
description of the DFT module right here.
******
\normalsize

The DFT module is invoked in NWChem 
by specifying the keyword \verb+dft+ on the compound directive,

\begin{verbatim}
  DFT
    ...
  END
\end{verbatim}

The keyword \verb+dft+ tells the code that this is a compound directive
containing input for the DFT module.  Additional subdirectives
may be specified by the user to define a particular
problem.  The \verb+dft+ input will be processed until the
\verb+END+ directive is encountered.  The actual DFT calculation will
be performed when the input encounters a \verb+TASK+ directive of the form,

\begin{verbatim}
  TASK DFT
\end{verbatim}

Refer to the \verb+TASK+ directive description in
Section \ref{sec:task} for a complete list of operations that can be
specified in the DFT module.)  The options that are currently working
in the code are

\Large
****list them here.  (Is there a default operation for DFT, by the way?)
\normalsize

The following 
subsections describe the keywords and
optional subdirectives that can be specified for a \verb+DFT+ calculation
in NWChem.


\subsection{Basis Sets for the DFT Module}
% The DFT module uses up to three different basis sets, specified 
% in the input file as described in the NWCHEM user documentation.
% These bases are recognized according to
% the following values for the $<$string name$>$ field of the basis input:

The DFT module requires at a minimum the basis set for the Kohn-Sham 
molecular orbitals.  This basis set must be in the default basis set named
{\tt "ao basis"}, or it must be assigned to this default name using the
\verb+SET+ directive (see Section \ref{sec:set}).

In addition to the basis set for the Kohn-Sham orbitals, 
the charge density fitting basis set can also be specified in the 
input directives for the DFT module.  This basis set is used for the 
evaluation of the Coulomb potential in the Dunlap scheme\footnote{B.I.~Dunlap, 
J.W.D.~Connolly and J.R.~Sabin, J.~Chem.~Phys.~{\bf 71},  4993 (1979)}.
The charge density fitting basis set must have the name {\tt "cd basis"}.
This can be the actual name of a basis set, or a basis set can be 
assigned this name using the \verb+SET+ directive, as described in
Section \ref{sec:set}.  If this basis set is not defined by input,
the $O(N^4)$ exact Coulomb contribution is computed.

The user also has the option of specifying a third basis set for the 
evaluation of the exchange-correlation potential.  This basis set must
have the name {\tt "xc basis"}.  If this basis set is not specified
by input, the exchange contribution (XC) is evaluated by numerical
quadrature.  In most applications, this approach is efficient enough,
so the {\tt "xc basis"} basis set is not generally required.

For the DFT module, the input options for defining the basis sets in a given
calculation can be summarized as follows;

\begin{itemize}
\item {\tt "ao basis"} -- Kohn-Sham molecular orbitals; required for all 
calculations

%  ({\sl required}) \\
%   The basis set for the (Kohn-Sham) molecular orbitals.
\item {\tt "cd basis"} -- charge density fitting basis set; optional, but
recommended for evaluation of the Coulomb potential

% ({\sl recommended}) \\
%    The charge density fitting basis set used for the
%   evaluation of the Coulomb potential according to the Dunlap scheme
%   (B.I.~Dunlap, J.W.D.~Connolly and J.R.~Sabin, J.~Chem.~Phys.~{\bf 71},
%   4993 (1979)).  If no basis set is specified,
%   the $O(N^4)$ exact Coulomb contribution is computed. 
\item {\tt "xc basis"} -- exchange-correlation (XC) fitting basis set; 
optional, and usually not needed

%  ({\sl optional})\\
%   The fitting basis for the evaluation of the
%  exchange-correlation potential.  If not specified,
%  the XC contribution will be evaluated by numerical quadrature.
%  At present, the quadrature is efficient enough so that this option
%  is rarely used.
\end{itemize}

\subsection{Spin Multiplicity in the DFT Module}

Both {\sl closed-shell} and {\sl open-shell} systems can be studied using
the DFT module.  Specifying the keyword \verb+MULT+ within the \verb+DFT+
directive allows the user to define the spin multiplicity of the system.
The form of the input line is as follows;
% The spin multiplicity is specified be means of the keyword {\tt MULT}.
% (where \verb+mult+ is equal to the number of alpha electrons minus
% beta electrons +1).

\begin{verbatim}
   mult <integer mult> 
\end{verbatim}

When the keyword \verb+MULT+ is specified, the user can define the integer
variable \verb+mult+, where \verb+mult+ is equal to the number of alpha 
electrons minus beta electrons, plus 1.

% **********************
% NOTE: if CHARGE is not an input option, just don't mention it.  If there
% is something really significant about what the DFT modules assumes 
% for the charge, it should be explained at the top of this section, with
% the general description of the DFT module.
%
% \subsection{Charge}
% 
% The charge is NO LONGER specified in the dft context.  See section 
% \ref{sec:charge}.
% **********************

\subsection{Initial MO Vectors for the DFT Module}

% If a file {\tt movecs} is found in the working directory, the 
% guess keyword is ignored, and the initial Kohn-Sham orbitals are read in.
% If there is no {\tt movecs} file in the working directory,
% two options are recognized for initial orbitals:
% \begin{itemize}
% \item (Default) start from a density matrix that is the superposition of HF
%   atomic solutions:
% \begin{verbatim}
%    guess atomic
% \end{verbatim}
% \item start from a one-electron (core) hamiltonian eigenvectors:
% \begin{verbatim}
%    guess hcore
% \end{verbatim}
% \end{itemize}

The initial guess for the molecular orbital vectors for the calculation
in the DFT module is taken from the contents of the file {\tt atomic.movecs}.
This  means that the calculation starts from a density matrix that is
a superposition of Hartree-Fock atomic solutions.

The keyword \verb+GUESS+ can be used within the DFT directive to specify
an alternative starting point for the molecular orbital vectors.  The form
of the keyword is as follows;

\begin{verbatim}
   guess <string input_movecs default atomic>
\end{verbatim}

Currently, only one alternative to the default value of {\tt atomic} is
recognized for the string \verb+input_movecs+.  This is the string {\tt hcore},
which directs the DFT calculation to start from one-electron (core)
hamiltonian eigenvectors.  When this option is invoked, the input line is
as follows;

\begin{verbatim}
   guess hcore
\end{verbatim}

\Large
**Does this mean that the VECTORS directive doesn't work with DFT?
\normalsize

\subsection{Optimization Control in the DFT Module}
\label{opt}

\Large
**What is being optimized here?  Does the SCF have anything to do
with the SCF module described above, or is the name just a coincidence?
***
\normalsize

The default optimization in the DFT module is to iterate on the 
Kohn-Sham (SCF) equations for a specified number of iterations 
without damping.  The.  The keyword that controls this optimization 
is \verb+ITRSCF+, and has the following general form,

% \begin{itemize}
% \item Maximum number of SCF (Kohn-Sham) iterations (default 30):
\begin{verbatim}
   itrscf <integer itrscf default 30>
\end{verbatim}

The damping to be imposed after \verb+itrscf+ iterations is specified
as a percent of ***what?***, using the following keyword,

\begin{verbatim}
   damp <integer ndamp default 0>
\end{verbatim}

As an alternative to this iterative optimization, the user can instead
specify direct inversion\footnote {P.~Pulay,  Chem.\ Phys.\ Lett.\ {\bf 73}, 
393 (1980) and P.~Pulay,  J.~Comp.~Chem.~{\bf 3}, 566 (1982)} of the
iterative subspace.  This is done by specifying the keyword \verb+diis+.

When the keyword \verb+diis+ is specified, the user must also define
the number of Fock matrices to be used in computing the DIIS step.  The
form of the input for this option is as follows,

% \item Use direct inversion of the iterative subspace (DIIS) optimization
%   (P.~Pulay,  Chem.\ Phys.\ Lett.\ {\bf 73}, 393 (1980) and P.~Pulay,
%   J.~Comp.~Chem.~{\bf 3}, 566 (1982)):

\begin{verbatim}
   diis
   nfock <integer nfock default 10>
\end{verbatim}

The user also has the option of level-shifting\footnote {M.F.~Guest and 
V.R.~Saunders, Mol.~Phys.~{\bf 28}, 819 (1974)} the Fock matrix by specifying
the keyword \verb+lshift+.  The form of this directive is as follows,

\begin{verbatim}
   lshift <real lshift default 0> 
\end{verbatim}

Specifying this keyword causes the diagonal elements of the Fock matrix
corresponding to the virtual orbitals to be shifted by \verb+lshift+.
By default, this level-shifting procedure is switched off after 99 iterations.
The user can specify the point at which it is switched off by defining
the keyword {\tt ncyshft}.  The form of the input line is as follows,

\begin{verbatim}
  ncyshft <integer ncyshft default 99> 
\end{verbatim}

% \item Density matrix damping. Damp by {\tt ndamp} percent (default 0\%):
% \begin{verbatim}
%    damp <integer ndamp>  
% \end{verbatim}
% \item Fock matrix level shifting (M.F.~Guest and V.R.~Saunders,
%   Mol.~Phys.~{\bf 28}, 819 (1974)). The diagonal elements of
% the Fock matrix corresponding to virtual orbitals are shifted by 
% {\tt  lshift} Hartree (default 0):
% \begin{verbatim}
%    lshift <real lshift> 
% \end{verbatim}
% The level shifting procedure may be switched off after {\tt ncyshft} cycles
% (default 99):
% \begin{verbatim}
%   ncyshft <integer ncyshft> 
% \end{verbatim}
% \end{itemize}

\subsection{Convergence Tolerances for the SCF in the DFT Module}

\Large
**What is being tested here?  Energy error?  What does the test mean?
I've taken a guess at it, but I may be wrong**
\normalsize

The user has the option of specifying the convergence tolerances for the
change in energy and the change in density in the solution using the DFT
module.  This is done by specifying the keywords {\tt scfcon} and {\tt 
igcon}.  The input line to specify the tolerance for the change in
energy is as follows,

\begin{verbatim}
  scfcon <integer iscfcon default 7>
\end{verbatim}

The input value for the variable \verb+iscfcon+ is used in the code to
define the convergence tolerance as $10^{-{\tt iscfcon}}$.

Similarly, the input line to specify the convergence tolerance for the 
change in density is as follows,

\begin{verbatim}
  gcon <integer igcon default -1>
\end{verbatim}

The input value for the variable \verb+igcon+ is used in the code to
define the convergence tolerance as $10^{-{\tt igcon}}$.  The default
value is -1, which sets the tolerance so large that in effect the 
convergence test on the change in density is ignored.

When the option for direct inversion of the iterative subspace is
specified (by entering the keyword \verb+DIIS+; see Section \ref{opt}),
the user can define the DIIS error norm convergence tolerance.  This
is done by specifying an input line of the form,

\begin{verbatim}
  idiisoff <integer idiisoff default 7>
\end{verbatim}

The error norm convergence tolerance is defined as $10^{-{\tt idiisoff}}$.

% \begin{itemize}
% \item Delta energy convergence tolerance $10^{-{\tt iscfcon}}$(default 7):
% \begin{verbatim}
%   scfcon <integer iscfcon>
% \end{verbatim}
% \item Delta density convergence tolerance $10^{-{\tt igcon}}$
%   (default -1 [meaning ignore]):
% \begin{verbatim}
%   gcon <integer igcon>
% \end{verbatim}
% \item DIIS error norm convergence tolerance $10^{-{\tt idiisoff}}$ 
%   (only used with DIIS, default 7):
% \begin{verbatim}
%   idiisoff <integer idiisoff>
% \end{verbatim}
% \end{itemize}


\subsection{Matrix I/O Operations in the DFT Module}

The inverted charge-density and exchange-correlation matrices
for a DFT calculation are normally written to disk storage.  The user
can prevent this by specifying the keyword \verb+noio+ within the
input for the DFT directive.  The input to exercise this option is
as follows,
% Disk storage of inverted CD and XC matrices may be turned off 
% (they are written out by default) with the keyword:

\begin{verbatim}
   noio
\end{verbatim}

\Large
**What are the consequences of exercising this option?  Are the matrices
needed for something later on?  Will they have to be recomputed at some
point?***
\normalsize

\subsection{Exchange and Correlation Functionals for the DFT Module}

The user has the option of specifying the exchange-correlation treatment 
in the DFT Module.  The default for the exchange functional is the 
Slater $\rho^{1/3}$ functional (from J.C.~Slater, 
{\sl Quantum Theory of Molecules and
Solids, Vol.~4: The Self-Consistent Field for Molecules and Solids}
(McGraw-Hill, New York, 1974)).  The default for the correlation
functional is the Vosk-Wilk-Nusair (VWN) local density functional
(S.J.~Vosko, L.~Wilk and M.~Nusair, 
Can.~J.~Phys.~{\bf  58}, 1200 (1980)).  The parameters used in this
formula are obtained by fitting to the {\bf Ceperley \&
Alder\footnotemark[1]} Quantum
MonteCarlo solution of the {\bf homogenous electron gas}.

These defaults can be invoked explicitly by specifying the following
keywords within the DFT module input directive,

\begin{verbatim}
  slater
  vwn
\end{verbatim}

The following subsections describe the options other than the defaults for
these functionals in the DFT module.
 
\Large
**any restrictions on which exchange functional goes with which correlation
functional?***
\normalsize

% The following functionals are available:
% \paragraph{Exchange}
% \begin{itemize}
% \item  Slater $\rho^{1/3}$ functional; J.C.~Slater, 
%   {\sl Quantum Theory of Molecules and
%   Solids, Vol.~4: The Self-Consistent Field for Molecules and Solids}
% (McGraw-Hill, New York, 1974):
% \begin{verbatim}
%    slater
% \end{verbatim}

\subsubsection{Optional Exchange Functionals}

There are two options in addition to the default for the exchange functional.
These are the Becke gradient-corrected functional (see A.D.~Becke, 
J.~Chem.~Phys.~88, 3098 (1988)), and the Hartree-Fock
exact exchange functional.

The Becke gradient-corrected functional is invoked by specifying the input
line,

% \item Becke gradient-corrected functional; 
%   A.D.~Becke, J.~Chem.~Phys.~88, 3098 (1988):
\begin{verbatim}
   becke88
\end{verbatim}
% \item Hartree-Fock exact exchange ($O(N^4)$):

The Hartree-Fock exact exchange functional, (which is exact to $O(N^4)$),
is invoked by specifying the input line,

% \begin{verbatim}
\begin{verbatim}
   HFexch
\end{verbatim}

% \paragraph{Correlation}
% {\bf Correlation}

\subsubsection{Optional Correlation Functionals}

There are five options in addition to the default correlation functional.
Each one is listed below with the keyword that must be specified by input
to invoke it for the DFT calculation.

\begin{itemize}
% \item VWN Local density functional; S.J.~Vosko, L.~Wilk and M.~Nusair, 
%   Can.~J.~Phys.~{\bf  58}, 1200 (1980); the parameters used in this
%   formula are obtained by fitting to the {\bf Ceperley \&
%   Alder\footnotemark[1]} Quantum
%   MonteCarlo solution of the {\bf homogenous electron gas}.
% \begin{verbatim}
%    vwn
% \end{verbatim}
\item VWN Local density functional; S.J.~Vosko, L.~Wilk and M.~Nusair, 
  Can.~J.~Phys.~{\bf  58}, 1200 (1980); ; the parameters used in this
  formula are obtained by fitting to the {\bf RPA} solution to the
  homogenous electron gas.
\begin{verbatim}
   vwnrpa
\end{verbatim}
\item Perdew \& Wang 1991 Local density functional; the parameters used in this
  formula are obtained by fitting to the {\bf Ceperley \&
  Alder\footnotemark[1]} Quantum
  MonteCarlo solution of the {\bf homogenous electron gas}.
\begin{verbatim}
   pw91lda
\end{verbatim}
\item LYP Gradient-corrected functional; 
  C.~Lee, W.~Yang and R.~G.~Parr, Phys.~Rev.~B {\bf 37}, 785 (1988):
\begin{verbatim}
   lyp
\end{verbatim}
\item Perdew86 Gradient-corrected functional; J.~P.~Perdew, Phys.~Rev.~B 
  {\bf33}, 8822 (1986):
\begin{verbatim}
   perdew86
\end{verbatim}
\item Perdew91 Gradient-corrected functional;  J.P.~Perdew,
  J.A.~Chevary, S.H.~Vosko, K.A.~Jackson, M.R.~Pederson, D.J.~Singh and C.~Fiolhais,
Phys. Rev. B {\bf 46}, 6671 (1992).

\begin{verbatim}
   perdew91
\end{verbatim}
\end{itemize}
\footnotetext[1]{D.M.~Ceperley and B.J.~Alder, Phys. Rev. Lett. {\bf 45},
  566 (1980).}

% \paragraph{Hybrid Methods}
\subsubsection{Hybrid Methods for Determining the Exchange and Correlation 
Functionals}

In addition to the options listed above for the exchange and correlation
functionals, the user has the alternative of specifying a hybrid
method.  This consists of the Becke ``{\sl half and half}'' exchange 
functional (see A.D.~Becke, J.~Chem.~Phys.~98, 1372 (1992)), and a
correlation functional computed by the adiabatic connection method
(see A.D.~Becke, J.~Chem.~Phys.~98, 5648 (1993)).  

This option is invoked
by specifying input lines as follows,

% \begin{itemize}
% \item Becke ``{\sl half and half}'' exchange:
%   A.D.~Becke, J.~Chem.~Phys.~98, 1372 (1992); the Exchange energy is
%   computed as
% \begin{eqnarray*}
% E_{X} \ = \ \frac{1}{2} E^{\rm HF}_X + \frac{1}{2} E^{\rm Slater}_{X} +
% \frac{1}{2} \Delta E^{\rm Becke88}_{X} 
% \end{eqnarray*}

\begin{verbatim}
   beckehandh
   acm
\end{verbatim}

The keyword \verb+beckehandh+ specifies that the exchange energy will be
computed as
\begin{eqnarray*}
E_{X} \ = \ \frac{1}{2} E^{\rm HF}_X + \frac{1}{2} E^{\rm Slater}_{X} +
\frac{1}{2} \Delta E^{\rm Becke88}_{X} 
\end{eqnarray*}

The keyword \verb+acm+ specifies that the exchange-correlation energy
is computed as

% \item {\sl Adiabatic connection method}:
%   A.D.~Becke, J.~Chem.~Phys.~98, 5648 (1993); the Exchange-Correlation
%  energy is  computed as

\begin{eqnarray*}
E_{XC} \ &=& \ E^{\rm VWN}_C + a_0 E^{\rm HF}_X + (1-a_0) E^{\rm Slater}_{X} +
a_X \Delta E^{\rm Becke88}_{X} + a_C \Delta E^{Perdew91}_C \\
& &{\rm where } \\
a_0 &=& 0.20, \ a_X = 0.72, \ a_C = 0.81
\end{eqnarray*}

\Large
**Any suggestions for the user as to when or why this might be a good option
to select?***
\normalsize

% \begin{verbatim}
%    acm
% \end{verbatim}

% \end{itemize}

\subsection{Integral Tolerances in the DFT Module}

%  Three keywords provide control of screening for the evaluation of
% the AO Gaussian functions, exchange-correlation (XC) Gaussian fitting 
% functions, charge-density (CD) Gaussian fitting functions, 
% and two-electron integrals:

The user has the option of controlling screening for the tolerances in
the integral evaluations for the DFT module.  In most applications, the
default values will be adequate for the calculation, but different values
can be specified in the input for the DFT module using the keywords
described below.

The input to define a screening tolerance for evaluation of the AO 
Gaussian functions is specified with the keyword \verb+accAOfunc+, as
follows,

\begin{verbatim}
   accAOfunc <integer iAOacc default 20>
\end{verbatim}

A Gaussian orbital basis (AO) function with exponent $\zeta$
and radial factor $e^{-\zeta\cdot r_i^2}$ is 
evaluated  at a point $r_i$ only if 
$\zeta\cdot r_i^2$ is greater than the value specified for ${\tt iAOacc}$.

The input to define a screening tolerance for evaluation of the exchange-
correlation (XC) Gaussian fitting functions is specified with the
keyword \verb+accXCfunc+, as follows,
 
\begin{verbatim}
   accXCfunc <integer iXCacc default 20>
\end{verbatim}

An exchange-correlation (XC) fitting function with exponent $\zeta$
and radial factor $e^{-\zeta\cdot r_i^2}$ is 
evaluated  at a point $r_i$ only if 
$\zeta\cdot r_i^2$ is greater than the value specified for ${\tt iXCacc}$.

The input to define a screening tolerance for evaluation of the
charge-density (CD) Gaussian fitting functions is specified with the
keyword \verb+accCoul+, as follows,

\begin{verbatim}
   accCoul <integer itol2e default 15>
\end{verbatim}
% \end{itemize}

A  charge-density (CD) fitting function with exponent $\zeta$
and radial factor $e^{-\zeta\cdot r_i^2}$ is evaluated  at a 
point $r_i$ only if $\zeta\cdot r_i^2$ is greater than ${\tt itol2e}$
In addition, the input
parameter {\tt itol2e} is also used to define the tolerance  in Schwarz 
screening for the Coulomb integrals.  Only integrals with estimated
values greater than $10^{(-{\tt itol2e})}$ are evaluated.

% \begin{itemize}
% \item  A Gaussian orbital basis (AO) function with exponent $\zeta$
% and radial factor $e^{-\zeta\cdot r_i^2}$ is 
% evaluated  at a point $r_i$ only if 
% $\zeta\cdot r_i^2 > {\tt iAOacc}$ (default {\tt iAOacc} value: 20)
% \begin{verbatim}
%    accAOfunc <integer iAOacc>
% \end{verbatim}
% \item Similarly, a XC fitting function with exponent $\zeta$
% and radial factor $e^{-\zeta\cdot r_i^2}$ is 
% evaluated  at a point $r_i$ only if 
% $\zeta\cdot r_i^2 > {\tt iXCacc}$ (default {\tt iXCacc} value: 20)
% \begin{verbatim}
%    accXCfunc <integer iXCacc>
% \end{verbatim}
% \item As above, for the Coulomb fitting basis.  A  CD fitting function 
% with exponent $\zeta$
% and radial factor $e^{-\zeta\cdot r_i^2}$ is 
% evaluated  at a point $r_i$ only if 
% $\zeta\cdot r_i^2 > {\tt itol2e}$
% In addition, the same
% parameter {\tt itol2e} is used in Schwarz screening for the 
%  Coulomb integrals: only integrals with estimated
% values greater than $10^{(-{\tt itol2e})}$ are evaluated. (Default
% {\tt itol2e} value: 15):
% \begin{verbatim}
%    accCoul <integer itol2e>
% \end{verbatim}
% \end{itemize}

\subsection{Quadrature Tolerances in the DFT Module}

% \begin{itemize}
% \item 
Two types of angular grid have been implemented in the DFT module.  These
are the Gauss-Legendre grid, and the Lebedev grid\footnote{The subroutine 
for the Lebedev grid was supplied by M.~Caus\`a of the University of Torino.}
The default in the DFT calculation is to select the Gauss-Legendre grid,
but the user can specify this input explicitly using the following input
line,

%  Gauss-Legendre (default):
\begin{verbatim}
   KEYWORD  <string grid_type default gausleg>  \
     <string tag> <string radial_density default medium>  \
                  <string ang_density default medium>     \
   END
\end{verbatim}

By default, the string \verb+grid_type+ is set to \verb+gausleg+, for the
Gauss-Legendre grid.  The Lebedev grid is specified with \verb+grid_type+
entered as \verb+lebedev+.  The default grid density for every atom or
center in the molecule is set to \verb+medium+ by default.  The user has the
option of defining the radial and angular grid density for a given atom 
or center of the molecule, within
the range from coarse to extra-fine.  This is done by specifying the string 
\verb+tag+ that identifies the atom or center in the basis set (see Section
\ref{sec:basis}), and entering \verb+coarse+, \verb+medium+, \verb+fine+,
or \verb+xfine+ for the strings \verb+radial_density+ and \verb+ang_density+.

% or Lebedev (routine supplied by M.~Caus\`a of the University of Torino):
%  \begin{verbatim}
%    lebedev
% \end{verbatim}
% \item Choices of grid density (default is {\tt medium}), range from 
% \begin{verbatim}
%    coarse
% \end{verbatim}
% to
% \begin{verbatim}
%    xfine
% \end{verbatim}

The effect of this input on the grid size for each atom or center is listed
in the following table.

% with the following definitions around each center:

  \begin{tabular}[right]{|l|r r r r|} \hline
Keyword & {\tt coarse} & {\tt medium} & {\tt fine} & {\tt xfine} \\ \hline
$N_{radial}$ & 30 & 40 & 60 & 90 \\
$N_{angular}$ & 6 & 8 & 10 & 10 \\  \hline
%$\Delta E/E$ $^a$  & & & &\\ 
%$\Delta \nabla E/|\nabla E|$$^a$  & & & &\\
%\multicolumn{5}{l}{$^a$ Based on calculations of 12-crown-6} \\
%\multicolumn{5}{l}{   with a 6-31g$*$ basis} \\
  \end{tabular}

\Large
***NOTE: I guessed at the proper form of the input for this feature, based
on the input conventions defined earlier.  Someone needs to check this,
and supply the correct names for the variables and keywords that I've made
up.***
\normalsize

% \item  Grid cutoff: for a given center (i.e.~atom) $A$, points that lie
% more than {\tt iRmax} bohr from center $A$ are neglected. (Default value: 40)

The user also has the option of specifying the grid cut-off for the
DFT calculation, using the keyword \verb+accQrad+.  The input line for
this option is as follows,

\begin{verbatim}
   accQrad <integer iRmax default 40>
\end{verbatim}
% \end{itemize}

The value entered for \verb+iRmax+ is the cutoff distance, in bohr, for grid
points around a given center or atom.  Grid points that lie more than 
\verb+iRmax+ bohr from the center or atom are neglected. 

\subsection{Mulliken Population Analysis in the DFT Module}

Mulliken analysis of the charge distribution is invoked by the keyword:
\begin{verbatim}
   mulliken
\end{verbatim}

\Large
**What is this, and why would the user want to do it?  What is the default?**
\normalsize


%%% Local Variables: 
%%% mode: latex
%%% TeX-master: t
%%% End: 
