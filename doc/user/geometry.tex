\label{sec:geom}

\begin{verbatim}
  GEOMETRY [<string name geometry>] \
           [units <string units bohr>] \
           [bqbq] \
           [print [xyz] || noprint]
    
    [SYMMETRY GROUP <string group_name> [print]]

    <string tag> <real x> <real y> <real z> \
        [charge <real charge>] [mass <real mass>]
    ...
  END
\end{verbatim}

Cartesian geometry specification, as well as Z-matrix-like format is
available.  As mentioned above (section
\ref{sec:arch}), multiple geometries may be stored in the database
provided each is given an independent name.  The default name of
\verb+"geometry"+ is used by most application modules to access the
geometry at which to perform a calculation.  Associating a named
geometry with the required name of \verb+"geometry"+ is described in
section \ref{sec:arch}.  Known names for units are \verb+au+,
\verb+bohr+, or \verb+angstrom+ (the conversion factor used to convert
from Angstr\"{o}m to Bohr is $1.8897265$). By default, the input
module prints any geometry it encounters.  Printing can be disabled
with the \verb+PRINT+ option.  The \verb+XYZ+ qualifier to print
causes the geometry to also be printed in the \verb+XYZ+ format of
XMol.  

\subsection{Cartesian coordinate input}

Each line in the body of the directive specifies the name or tag, and
the coordinates of one center or atom.  The charge associated with the
center is inferred from the atom type.  The charge may be explicitly
specified using the \verb+CHARGE+ keyword.  Default masses may be
overriden by specifying the mass.  The default masses are those of the
highest naturally abundant isotope for the given atom, not the average
mass of the element.  

The tag associated with each center is interpreted as follows:
\begin{itemize}
\item If it begins with \verb+BQ+ (ignoring case) then it is treated
      as a dummy center with default zero charge. Dummy centers may 
      optionally have basis functions or non-zero charge.
\item If it begins with either the symbol or name of an element then
      it is thought to be an atom.  Atoms {\em must} have basis
      functions associated with them and the default charge is the
      atomic number adjusted for the presence of ECPs (see
      \ref{sec:ecp}).  The user provided charges (of all centers,
      atomic and dummy) and the total charge of the system are used to
      determine the number of electrons
\item The tag of a center is used in the \verb+BASIS+ directive to
      associate functions with centers.  All centers with the same tag
      will have the same basis functions.  Atomic centers may have
      standard basis sets sited upon them.
\item When automatic symmetry detection is functional only centers
      with the same tag will be candidates for testing for symmetry
      equivalence.
\end{itemize}

By default NWCHEM does not include the interaction between dummy
centers.  The \verb+BQBQ+ qualifier to the \verb+GEOMETRY+ directive
causes these interactions to be included.

 The use of molecular symmetry in NWCHEM is not yet automated, thus,
the user is responsible for detecting symmetry and specifying the
coordinates of the symmetry unique atoms in a suitable orientation
relative to the rotation axes and symmetry planes.  Since it seems
that only the original authors of the symmetry package seem to
understand the latter, we provide many examples in Appendix
\ref{symexamples}.

\subsection{Z-matrix input}
\label{sec:Z-matrix}
 
If the \verb+GEOMETRY+ directive contains the \verb+zmat+ keyword, the
structure of the molecule is defined by means of its Z-matrix, or
the cartesian coordinates of the atoms, or a mixture of the two. A
blank line terminates the list of the atoms.  Bond lengths,  bond
angles, or torsion angles can be specified by means of numerical
values assigned to variables. The list of variables and  their
assigned numerical  values  follow  the  list  of atoms, and is also
terminated with a blank line.  Then comes the end of the data group
input with a last line 'end'

A second set of variables to be qualified as '{\bf frozen}' may be
specified.  They come  after  the  blank  line  that indicates the end
of the 'variables'. The second set is also terminated by a blank line.
{\bf The 'frozen' specification is presently not active, but included
  to ensure compatibility with other codes. Only cartesian coordinates
  can be kept constant via the 'active atoms' list (see section
  \ref{sec:activeatoms})}

'Ghost' atoms, often used in the assessment of basis set superposition
error for  example,  are  atoms  with their associated basis set, but
for which the nuclear charge is set to zero. Specifying a 'ghost' atom
may be  accomplished appending the '{\tt ghost}' parameter at the end of the
appropriate lines below.

When  two  numerical  values,  separated by a comma, are given for some
variables, they are considered as the initial and final values for the
definition of a Linearized Synchronous Transit pathway.  The
geometries generated by linear interpolation between the initial and
final values.

\begin{enumerate}

   \item $<$atom$>$ [ghost]

    Only  the  name  of the first atom is required. 'ghost' is used
    only when specifying a 'ghost' atom.

   \item $<$atom$>$ $<$i1$>$ $<$blength$>$ [ghost]

    Only a name and a bond distance is required for atom 2.    For
    'ghost', same remark applies as before.

   \item $<$atom$>$ $<$i1$>$ $<$blength$>$ $<$i2$>$ $<$alpha$>$ [ghost]

    Only  a  name, distance, and angle are required for atom 3.  For
    'ghost', same remark applies as before.

   \item $<$atom$>$ $<$i1$>$ $<$blength$>$ $<$i2$>$ $<$alpha$>$ $<$i3$>$ $<$beta$>$ $<$i4$>$ [ghost] %
        
    \begin{itemize}
    \item [$\bullet$]  {\tt atom} is the chemical symbol of this
        atom;  it  can  be followed  
        by numbers if desired. The chemical symbol implies the nuclear
        charge.  
    \item [$\bullet$]  {\tt i1} defines the connectivity of the following bond.  
    \item [$\bullet$]  {\tt blength} is the bond length 'this atom-atom i1'.  
    \item [$\bullet$]  {\tt i2} defines the connectivity of the following angle.  
    \item [$\bullet$]  {\tt alpha} is the angle 'this atom-atom i1-atom i2'.  
    \item [$\bullet$]  {\tt i3} defines the connectivity of the following angle.  
    \item [$\bullet$]  {\tt beta}  is  either  the dihedral angle
'this atom-atom {\tt i1}-atom {\tt i2}-atom 
        {\tt i3}', or perhaps a second bond angle, 'this atom-atom i1-atom {\tt i3}' 
    \item [$\bullet$]  {\tt i4} defines the nature of {\tt beta}.  If
{\tt beta} is a dihedral
        angle,  {\tt i4}=0  ( default ).  If {\tt beta} is a second bond angle,
        then {\tt i4}=+/-1 
        (sign specifies one of two possible directions).  
    \item [$\bullet$]  For 'ghost', same remark applies as before.
    \end{itemize}

  \item Line no.\ 4  is repeated for each remaining atom. A blank line
        indicates the end of the atoms in the molecule.

  \item The use of 'dummy' atoms is possible, by using 'X' or 'BQ' for  the
chemical symbol.

  \item The connectivity i1, i2, i3, may be given as integers,
1, 2, 3, 4, 5, \ldots or as strings which match one of the {\tt atom}s.
In this case, numbers must be added to the {\tt atom} string to
ensure uniqueness.

   \item Symbolic strings may be given in place of  numeric  values  for
{\tt blength}, {\tt alpha},  {\tt beta}.  The  same  string may be repeated.  Any mixture of
    numeric data and symbols may be given.

   \item All symbolic definitions follow the blank line that signals  the
    end  of the Z-matrix input. The list of symbolic definitions ends with a
    blank line, followed by an '{\tt end}' directive. If there are no symbolic
    definitions, and  all  the  bond  lengths, angles, and torsions are
    specified by their numeric values in the Z-matrix data, then the
    end of the Z-matrix data is detected through two blank lines, one
    to indicate the end of the Z-matrix proper, the other to indicate
    the end of the symbolic definitions.

   \item A second set of  symbolic  definitions,  separated  from  the
    first  set through  a blank line, may be included to specify
    {\bf frozen} internal 
    coordinates. A blank line again  defines  the  end  of  the
    list  of  'frozen' internal coordinates.

   \item Note that atoms in the Z-matrix data may be specified via Cartesian
    coordinates expressed in units of {\AA}ngstr{\o}ms, Each line has the
    following form:

     \begin{verbatim}
       <atom> <x> <y> <z> [ghost]
     \end{verbatim}

    with  {\tt atom}  being  the atomic name as before, x, y, z being the
    Cartesian coordinates, and {\tt ghost} used only when specifying a
    'ghost' atom.
\end{enumerate}

\subsubsection*{example}

The following example is the z-matrix for $CH_3CF_3$,


\begin{verbatim}
 geometry zmt
   C 
   C 1 CC 
   H 1 CH1 2 HCH1 
   H 1 CH2 2 HCH2 3 TOR1 +1 
   H 1 CH3 2 HCH3 3 TOR2 -1 
   F 2 CF1 1 CCF1 3 TOR3  0 
   F 2 CF2 1 CCF2 6 FCH1  1 
   F 2 CF3 1 CCF3 6 FCH2 -1

   CC   = 1.4888 
   CH1  = 1.0790 
   CH2  = 1.0789  
   CH3  = 1.0789  
   CF1  = 1.3667 
   CF2  = 1.3669 
   CF3  = 1.3669

   HCH1 = 10428 
   HCH2 = 10474 
   HCH3 = 1047 
   CCF1 = 112.0713 
   CCF2 = 112.0341 
   CCF3 = 112.0340 
   TOR1 = 109.3996 
   TOR2 = 109.3997 
   TOR3 = 180.0000 
   FCH1 = 106.7846 
   FCH2 = 106.7842

 end   
\end{verbatim}

   The  separation  of the symbolic definitions into two groups is what
   makes the internal coordinates of the second group 'frozen'.
   Removal  of  the blank  line between the first and second set of
   symbolic definitions would remove the 'frozen' character of all the
   variables defined in  the  second group.

