\label{sec:geom}

\Large
% comments follow
WARNING -- I have substantially re-written this section, and may have
guessed wrong on the more obscure points.  Someone (preferably the original
author), should read this draft and correct the more egregious 
errors. -- J.M. Cuta
\normalsize

The \verb+GEOMETRY+ directive is used to define specific geometry objects
for a calculation or series of calculations.  The input for this directive
allows the user to define the geometry of the molecule(s) to be considered
in the calculation(s).  NWChem already has a fairly good idea of what most
atoms look like and the sort of molecular structures in which they might be
arranged.  This information is stored in the basis sets, which can be specified
for a given task using the \verb+BASIS+ directive (see Section 
\ref{sec:basis}).  The \verb+GEOMETRY+ directive allows the user to specify
particular elements of the geometry objects, such as the coordinates of 
the individual atoms in a molecule, and the appropriate symmetry group.
It is also possible to define multiple geometry objects for the same
molecule under different names, for use in different tasks of the same 
calculation.

The form of the \verb+GEOMETRY+ directive is as follows;

\begin{verbatim}
  GEOMETRY [<string name default geometry>] [zmat]\
           [units <string units default bohr>] \
           [bqbq] \
           [print [xyz] || noprint]
    
    [SYMMETRY GROUP <string group_name> [print]]

    <string tag> <real x> <real y> <real z> \
        [charge <real charge>] [mass <real mass>] [ghost]
    ...

    <string tag> < list_of_Z-matrix_variales> [ghost]
    ...

    <string tag> <list_of_frozen_variables>
    ...

  END
\end{verbatim}

% Cartesian geometry specification, as well as Z-matrix-like format is
% available.  As mentioned above (section
% \ref{sec:arch}), multiple geometries may be stored in the database
% provided each is given an independent name.  The default name of
% \verb+"geometry"+ is used by most application modules to access the
% geometry at which to perform a calculation.  Associating a named
% geometry with the required name of \verb+"geometry"+ is described in
% section \ref{sec:arch}.  Known names for units are \verb+au+,
% \verb+bohr+, or \verb+angstrom+ (the conversion factor used to convert
% from Angstr\"{o}m to Bohr is $1.8897265$). By default, the input
% module prints any geometry it encounters.  Printing can be disabled
% with the \verb+PRINT+ option.  The \verb+XYZ+ qualifier to print
% causes the geometry to also be printed in the \verb+XYZ+ format of
% XMol.  

This is a multiple directive and may contain many or few lines,
depending on the application.  The input supplied with this directive
consists of three main sections; 

\begin{itemize}
\item basic information to identify the geometry object in the code and
define how it will be processed
\item input to identify the symmetry group for the molecule
\item input to specify the geometric locations of the atoms in the molecule
\end{itemize}

The input options and keywords for each of these sections are
discussed in detail in the following subsections.  

\subsection{Geometry Object Identification Input}

The input for the first four lines of the directive allows the user to name 
the object, specify the geometry input format, decide how to treat dummy
centers in the geometry, and tell the code what to do about printing 
the geometry information.

The default in the code is to specify the geometry using
Cartesian coordinates, but the user has the option of specifying
the geometry for some (or all) of the atoms using a Z-matrix-like format
(see Section \ref{sec:Z-matrix}).
This option is flagged by specifying the keyword \verb+zmat+ on the
first line of the directive.  If this keyword is omitted, the geometry
input must be specified using Cartesian coordinates (see section 
\ref{sec:cart}).

The default \verb+name+ for a geometry object is the string \verb+geometry+,
and most modules in the code look for an object with this name.  
The user can redirect the module to a different geometry object by assigning
the string \verb+geomery+ to \verb+name+ using the \verb+SET+ directive 
(see Section \ref{sec:set}).  The default for the geometry input unit
is \verb+bohr+, and this is the working unit for such distances in the code.
However, the geometric coordinates can be supplied in \verb+au+ or 
\verb+angstrom+ by specifying the appropriate keyword on the directive.
(Note: The conversion factor used in the code to convert from Angstr\"{o}m 
to Bohr is $1.8897265$.)

The default in NWChem is to ignore interactions between dummy centers. 
Specifying the keyword \verb+bqbq+ forces the code to include these 
interactions.  The complimentary keywords \verb+print+ and \verb+noprint+
allow the user to specify the print option for the geometry module, 
independent of any specifications in the top-level \verb+PRINT+ directive.
The \verb+print+ keyword tells the code to print everything for this
module, regardless of other instructions.  In addition, the keyword
\verb+xyz+ specifies that the coordinates will be printed in the XYZ
format of XMol.  If the keyword \verb+noprint+ is specified, the printing
of all geometry output for this module is suppressed.

\subsection{Symmetry Group Input}

The input following the keyword \verb+SYMMETRY GROUP+ is used to specify
the symmetry for the molecule modeled with this geometry object.
There is no default for this input in the current version of NWChem, since
the use of molecular symmetry is not yet automated in the code.
Examples of expected input for the string \verb+group_name+ include
such entries as

\begin{itemize}
\item \verb+c2v+ -- for molecular symmetry C\_{2v}
\item \verb+d2h+ -- for molecular symmetry D\_{2h}
\item \verb+Td+ -- for molecular symmetry T\_{d}
\item \verb+d6h+ -- for molecular symmetry D\_{6h}
\end{itemize}

The user must know the symmetry of the molecule being modeled, and be able
to specify the
coordinates of the symmetry-unique atoms in a suitable orientation
relative to the rotation axes and symmetry planes.  
Appendix \ref{symexamples} lists a number of examples of the \verb+geometry+
directive input for specific molecules having symmetry patterns recognized
by NWChem.

\subsection{Cartesian coordinate input}
\label{sec:cart}

The default in NWChem is to specify the geometry information in Cartesian 
coordinates.  Each atom of the molecule being modeled by the geometry
object must be specified on a line in the directive.  That is, each atom
or center
must be identified on a line of the following form;

\begin{verbatim}

    <string tag> <real x> <real y> <real z> \
        [charge <real charge>] [mass <real mass>] [ghost]

\end{verbatim}

The string \verb+tag+ is the name of the atom or center.  The string is limited to
16 characters, and must correspond to the
name of an atom in one of the basis sets defined for the calculation.
Atoms or centers with the same \verb+tag+ will use the same basis set in
the calculation. 
(See Section \ref{sec:basis} for a discussion of the input for the
\verb+BASIS+ directive).  In most cases, the entry for \verb+tag+ is the
chemical symbol for the element, such as \verb+O+ for oxygen, \verb+H+
for hydrogen, \verb+Fe+ for iron, etc.  Dummy centers 
are differentiated from atoms and centers by giving them a \verb+tag+ that 
begins with the letters \verb+bq+.  An atom must have basis functions
associated with it, and so must all centers.  If the keyword \verb+bqbq+
is specified, dummy centers must also have basis functions.

\Large
\verb+************************+
Shall we tell the users what a dummy center is, and the difference
between a 'center' and an atom?  Or do we assume they were all born
knowing this?  Also might want to explain 'ghost' atoms here...
\verb+************************+
\normalsize

The xyz coordinates of the atom in the molecule, relative to some origin
(***shall we tell them where it is?***) are specified as real numbers 
following the string \verb+tag+.  The user also has the option of 
specifying the charge of the atom (or center) and its mass.  

The default charge for an
atom is its atomic number, adjusted for the presence of ECPs (see Section
\ref{sec:ecp}).  In order to specify a different value for the
charge on a particular atom, the user must enter
the keyword \verb+charge+, then enter a real number for the unit charge
of the atom in the real variable \verb+charge+.  The values specified for the charges of 
all atoms, centers, and
dummy centers are used in conjunction with the 
total charge of the system (as specified with the \verb+CHARGE+ directive; 
see Section 
\ref{sec:charge}) to determine the total number of electrons in the system.

The default mass for an atom is taken to be the mass of its highest naturally
occuring isotope.  If the user wishes to model some other isotope of the
element, its mass must be defined explicitly by specifying the keyword 
\verb+mass+ and entering the appropriate value for \verb+mass+.  

\Large
(***shall
we tell them the units to use for mass?  Or is this something everybody
already knows?***)
\normalsize


%  Each line in the body of the directive specifies the name or tag, and
% the coordinates of one center or atom.  The charge associated with the
% center is inferred from the atom type.  The charge may be explicitly
% specified using the \verb+CHARGE+ keyword.  Default masses may be
% overriden by specifying the mass.  The default masses are those of the
% highest naturally abundant isotope for the given atom, not the average
% mass of the element.  
% 
% The tag associated with each center is interpreted as follows:
% \begin{itemize}
% \item If it begins with \verb+BQ+ (ignoring case) then it is treated
%       as a dummy center with default zero charge. Dummy centers may 
%       optionally have basis functions or non-zero charge.
% \item If it begins with either the symbol or name of an element then
%       it is thought to be an atom.  Atoms {\em must} have basis
%       functions associated with them and the default charge is the
%       atomic number adjusted for the presence of ECPs (see
%       \ref{sec:ecp}).  The user provided charges (of all centers,
%       atomic and dummy) and the total charge of the system are used to
%       determine the number of electrons
% \item The tag of a center is used in the \verb+BASIS+ directive to
%       associate functions with centers.  All centers with the same tag
%       will have the same basis functions.  Atomic centers may have
%       standard basis sets sited upon them.
% \item When automatic symmetry detection is functional only centers
%       with the same tag will be candidates for testing for symmetry
%       equivalence.
% \end{itemize}
% 
% By default NWCHEM does not include the interaction between dummy
% centers.  The \verb+BQBQ+ qualifier to the \verb+GEOMETRY+ directive
% causes these interactions to be included.
% 
%  The use of molecular symmetry in NWCHEM is not yet automated, thus,
% the user is responsible for detecting symmetry and specifying the
% coordinates of the symmetry unique atoms in a suitable orientation
% relative to the rotation axes and symmetry planes.  Since it seems
% that only the original authors of the symmetry package seem to
% understand the latter, we provide many examples in Appendix
% \ref{symexamples}.

\subsection{Z-matrix input}
\label{sec:Z-matrix}

Specifying the keyword \verb+zmat+ on the \verb+GEOMETRY+ directive allows
the user to specify the structure of the molecule by means of its Z-matrix.
However, it is also possible to enter the coordinates of some of the atoms using
cartesian coordinates, even when the keyword \verb+zmat+ has been specified.
The code is able to distinguish between
the two types of input by means of the input following the string \verb+tag+
for the particular atom.  If the Z-matrix input is specified for the atom,
the input consists of pairs numbers that define connectivity indices and 
bond length, bond angle, or torsion angle.  If the
input is in Cartesian coordinates, the input consists of three real numbers
defining the x,y,z coordinates of the atom.  However, when the \verb+zmat+
keyword is specified, any Cartesian coordinate input (i.e., x,y,z, coordinates)
must be specified in {\AA}ngstr{\o}ms, regardless of the entry specified
for \verb+units+.

The keyword \verb+ghost+ is used to denote a 'ghost' atom in the molecular
geometry.  A 'ghost' atom is like an ordinary atom in that it has its 
associated basis set, but it is assumed to have a nuclear charge of zero.
This is a useful property for such things as calculations to assess basis set
superposition error.  

\Large
(***How does this differ from setting the \verb+charge+
to zero in the Cartesian coordinates input option?***)
\normalsize

% If the \verb+GEOMETRY+ directive contains the \verb+zmat+ keyword, the
% structure of the molecule is defined by means of its Z-matrix, or
% the cartesian coordinates of the atoms, or a mixture of the two. A
% blank line terminates the list of the atoms.  Bond lengths,  bond
% angles, or torsion angles can be specified by means of numerical
% values assigned to variables. The list of variables and  their
% assigned numerical  values  follow  the  list  of atoms, and is also
% terminated with a blank line.  Then comes the end of the data group
% input with a last line 'end'
% 
% A second set of variables to be qualified as '{\bf frozen}' may be
% specified.  They come  after  the  blank  line  that indicates the end
% of the 'variables'. The second set is also terminated by a blank line.
% {\bf The 'frozen' specification is presently not active, but included
%   to ensure compatibility with other codes. Only cartesian coordinates
%   can be kept constant via the 'active atoms' list (see section
%   \ref{sec:activeatoms})}
% 
% 'Ghost' atoms, often used in the assessment of basis set superposition
% error for  example,  are  atoms  with their associated basis set, but
% for which the nuclear charge is set to zero. Specifying a 'ghost' atom
% may be  accomplished appending the '{\tt ghost}' parameter at the end of the
% appropriate lines below.
%
% When  two  numerical  values,  separated by a comma, are given for some
% variables, they are considered as the initial and final values for the
% definition of a Linearized Synchronous Transit pathway.  The
% geometries generated by linear interpolation between the initial and
% final values.

The input supplied for the atoms describing the molecule by means of its
Z-matrix requires the following sequential approach.

\Large
\verb+*****************+
Questions we might want to answer at this point;
\begin{itemize}
\item Is there a particular starting point for the Z-matrix of a molecule,
or can the user pick any old atom?
\item Is there a required order in which the atoms of a molecule must be
entered?
\item What about molecules with multiple bonds?
\item What is a ghost?  (Same as a center?  Or something else?)
\end{itemize}
\verb+****************+
\normalsize

\begin{enumerate}

   \item $<$atom$>$ [ghost]

%    Only  the  name  of the first atom is required. 'ghost' is used
%    only when specifying a 'ghost' atom.
    For the first atom in the molecule, the Z-matrix input requires only  
    the name of the atom at it appears in the basis set.  This is usually 
    the chemical symbol for the atom.
    However, if the atom is a 'dummy center', the chemical symbol must
    be preceeded by \verb+x+ or \verb+bq+. (The keyword 'ghost' is required
    only when specifying that the atom is to be defined as a 'ghost' atom.)

   \item $<$atom$>$ $<$i1$>$ $<$blength$>$ [ghost]

%   Only a name and a bond distance is required for atom 2.  For 'ghost',
%   same remark applies as before.
    For the second atom, the Z-matrix input requires the name of the atom
    as it appears in the basis set, 
    plus the connectivity $<$i1$>$ and the bond distance $<$blength$>$ 
    connecting it to another atom. 

    (NOTE: The connectivity for any bond, {\tt i1, i2, i3} \ldots, 
    can be specified as an integer (1, 2, 3, \ldots) or as a string 
    which matches the name specified for one of the atoms in the molecule.
    If the connectivity (such as {\tt i1}) is defined as a string,
    and the same element appears more than once in the molecule, a number
    must be added to the {\tt atom} string of the Z-matrix input for
    subsequent appearances of that element in the molecule, to ensure 
    a unique \verb+tag+ for each occcurance.)

   \item $<$atom$>$ $<$i1$>$ $<$blength$>$ $<$i2$>$ $<$alpha$>$ [ghost]

%    Only  a  name, distance, and angle are required for atom 3.  For
%    'ghost', same remark applies as before.
    For the third atom of the molecule, the Z-matrix input requires the
    name of the atom as it appears in the basis set and the connectivity 
    $<$i1$>$ and bond distance 
    $<$blength$>$ connecting it to atom 2, plus the connectivity $<$i2$>$
    and angle $<$alpha$>$ it makes with the plane of the first two atoms. 

   \item $<$atom$>$ $<$i1$>$ $<$blength$>$ $<$i2$>$ $<$alpha$>$ $<$i3$>$ $<$beta$>$ $<$i4$>$ [ghost] %

    For the fourth atom and all subsequent atoms in the molecule (if it
    has more than four atoms), the Z-matrix input requires the name 
    of the atom as it appears in the basis set,
    plus the connectivities and bond length,
    bond angle, and either the dihedral angle or a second bond angle
    for the atom.
        
    \begin{itemize}
%    \item [$\bullet$]  {\tt atom} is the chemical symbol of this
%        atom;  it  can  be followed  
%        by numbers if desired. The chemical symbol implies the nuclear
%        charge.  
    \item [$\bullet$]  {\tt i1} defines the connectivity of the following bond  
%    \item [$\bullet$]  {\tt blength} is the bond length 'this atom-atom i1'.
    \item [$\bullet$]    {\tt blength} is the bond length between this atom and the atom denoted by {\tt i1}
    \item [$\bullet$]  {\tt i2} defines the connectivity of the following angle
%    \item [$\bullet$]  {\tt alpha} is the angle 'this atom-atom i1-atom i2'.
    \item [$\bullet$]  {\tt alpha} is the angle that the bond with this atom 
makes with the plane of the bond between atom {\tt i1} and atom {\tt i2}  
    \item [$\bullet$]  {\tt i3} defines the connectivity of the following angle
%    \item [$\bullet$]  {\tt beta}  is  either  the dihedral angle
%'this atom-atom {\tt i1}-atom {\tt i2}-atom 
%        {\tt i3}', or perhaps a second bond angle, 'this atom-atom i1-atom {\t%t i3}'
    \item [$\bullet$]  {\tt beta}  is  the dihedral angle between
this atom and the plane containing atom {\tt i1} and atom {\tt i2} and atom 
        {\tt i3}; alternatively, it defines a second bond angle for this atom, between this atom 
and the plane of the two atoms atom {\tt i1} and atom {\tt i3}.  
Which alternative is selected depends on the value specified for {\tt i4}, as explained below. 
%    \item [$\bullet$]  {\tt i4} defines the nature of {\tt beta}.  If
%{\tt beta} is a dihedral
%        angle,  {\tt i4}=0  ( default ).  If {\tt beta} is a second bond angle,
%        then {\tt i4}=+/-1 
%        (sign specifies one of two possible directions).  
%    \item [$\bullet$]  For 'ghost', same remark applies as before.
    \item [$\bullet$]  {\tt i4} defines the nature of the input for {\tt beta}.  If {\tt i4} is zero, 
{\tt beta} is interpreted as a dihedral
        angle.  (This is the default.)  If {\tt i4} is entered as +1 or -1, {\tt beta} 
is interpreted as a second bond angle.  (The sign of {\tt i4} specifies the 
direction of the bond angle ***relative to what?***).
    \end{itemize}

%  \item Line no.\ 4  is repeated for each remaining atom. A blank line
%        indicates the end of the atoms in the molecule.

%  \item The use of 'dummy' atoms is possible, by using 'X' or 'BQ' for  the
%chemical symbol.
%
%  \item The connectivity i1, i2, i3, may be given as integers,
% 1, 2, 3, 4, 5, \ldots or as strings which match one of the {\tt atom}s.
% In this case, numbers must be added to the {\tt atom} string to
% ensure uniqueness.
%
%   \item Symbolic strings may be given in place of  numeric  values  for
%{\tt blength}, {\tt alpha},  {\tt beta}.  The  same  string may be repeated.  Any mixture of
%    numeric data and symbols may be given.
% 
%   \item All symbolic definitions follow the blank line that signals  the
%    end  of the Z-matrix input. The list of symbolic definitions ends with a
%    blank line, followed by an '{\tt end}' directive. If there are no symbolic
%    definitions, and  all  the  bond  lengths, angles, and torsions are
%    specified by their numeric values in the Z-matrix data, then the
%    end of the Z-matrix data is detected through two blank lines, one
%    to indicate the end of the Z-matrix proper, the other to indicate
%    the end of the symbolic definitions.
%
%   \item A second set of  symbolic  definitions,  separated  from  the
%    first  set through  a blank line, may be included to specify
%    {\bf frozen} internal 
%    coordinates. A blank line again  defines  the  end  of  the
%    list  of  'frozen' internal coordinates.
%
%   \item Note that atoms in the Z-matrix data may be specified via Cartesian
%    coordinates expressed in units of {\AA}ngstr{\o}ms, Each line has the
%    following form:
%
%     \begin{verbatim}
%       <atom> <x> <y> <z> [ghost]
%     \end{verbatim}
%
%    with  {\tt atom}  being  the atomic name as before, x, y, z being the
%    Cartesian coordinates, and {\tt ghost} used only when specifying a
%    'ghost' atom.
\end{enumerate}

The Z-matrix variables {\tt blength}, {\tt alpha}, and {\tt beta} can be
entered either as numeric values or symbolic strings, or as a mixture of
the two types for a given atom.   The end of the Z-matrix input for the 
molecule is signalled by a blank line in the directive, and all symbolic 
definitions for the variables must be supplied following that blank line.
The list of symbolic definitions is terminated with another blank line.
If there are no symbolic definitions for the Z-matrix input (that is,  
all of the bond lengths, angles, and torsions are specified by their 
numeric values), then the end of the Z-matrix data is signalled by entering
two blank lines in the directive; one to indicate the end of the Z-matrix 
input, and the other to indicate the end of the symbolic definitions.

The \verb+GEOMETRY+ directive also contains a feature that allows the user
to define symbolic definitions for a set of variables that will be 
qualified as '{\bf frozen}'.  A {\bf frozen} atom is one that will not
have any of its geometric parameters changed in the course of an optimization
or other calculation that would change the geometry of a molecule.  
The ability to actually use this feature
for atoms defined by means of the Z-matrix input does not exist in NWChem
as yet, but the input option has been included to ensure compatibilty with
other codes.  Only atoms that have been specified using Cartesian coordinates
can retain their geometry parameters at the specified input values.  This
is accomplished by means of the optional 'active atoms' list (refer to Section
\ref{activeatoms}).

If the user wishes to specify a set of {\bf frozen} variables, however,
% (we won't ask why...)
they must be specified following the blank line that signals the end of 
the symbolic definitions for the Z-matrix input.  The end of the input
for the {\bf frozen} variables is also signaled with a blank line.

The following example illustrates the Z-matrix input for the molecule
$CH_3CF_3$.  This input uses integer numbers for the connectivities {\tt i1},
{\tt i2}, {\tt i3} \ldots, but has symbolic strings as entries for the Z-matrix
variables {\tt blength}, {\tt alpha}, and {\tt beta}.  It therefore must
also include input lines defining the symbolic strings, following the blank
line that terminates the Z-matrix input.  This example also includes a set
of {\bf frozen} variables, which are entered after the blank line terminating
the definitions of the symbolic strings used by the Z-matrix.


% \subsubsection*{example}

% The following example is the z-matrix for $CH_3CF_3$,

The \verb+GEOMETRY+ directive for this example is as follows;

\begin{verbatim}
 geometry zmat
   C 
   C 1 CC 
   H 1 CH1 2 HCH1 
   H 1 CH2 2 HCH2 3 TOR1 +1 
   H 1 CH3 2 HCH3 3 TOR2 -1 
   F 2 CF1 1 CCF1 3 TOR3  0 
   F 2 CF2 1 CCF2 6 FCH1  1 
   F 2 CF3 1 CCF3 6 FCH2 -1

   CC   = 1.4888 
   CH1  = 1.0790 
   CH2  = 1.0789  
   CH3  = 1.0789  
   CF1  = 1.3667 
   CF2  = 1.3669 
   CF3  = 1.3669

   HCH1 = 10428 
   HCH2 = 10474 
   HCH3 = 1047 
   CCF1 = 112.0713 
   CCF2 = 112.0341 
   CCF3 = 112.0340 
   TOR1 = 109.3996 
   TOR2 = 109.3997 
   TOR3 = 180.0000 
   FCH1 = 106.7846 
   FCH2 = 106.7842

 end   
\end{verbatim}

%   The  separation  of the symbolic definitions into two groups is what
%   makes the internal coordinates of the second group 'frozen'.
%   Removal  of  the blank  line between the first and second set of
%   symbolic definitions would remove the 'frozen' character of all the
%   variables defined in  the  second group.

In this example, the symbolic strings listed after the first blank line
(which signals the end of the Z-matrix input for the molecule) are defined
with the values listed ({\tt CC= 1.4888}, {\tt CH1 = 1.0790}, etc.).  The
second blank line forces the symbolic definitions starting with {\tt HCH1
= 10428} to be {\bf frozen}.
\Large
\verb+***************************+
What does the code actually do with this input, since the {\bf frozen}
feature is not active?  Does it go ahead and use the specified values
for {\tt HCH1} through {\tt FCH2}, and change them as required by the
calculation?  Or are those strings defined some other way, and the
{\bf frozen} assignments ignored?
\verb+***************************+
\normalsize
