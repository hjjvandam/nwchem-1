% $Id: cosmo.tex,v 1.4 2002-02-06 19:38:51 sohirata Exp $
\label{sec:cosmo}

COSMO is the continuum solvation `COnductor-like Screening MOdel'
of A. Klamt and G. Sch\"{u}\"{u}rmann to describe dielectric screening
effects in solvents.

\begin{enumerate}
\item A. Klamt and G. Sch\"{u}\"{u}rmann, J.~Chem.~Soc.~Perkin Trans. 2, 1993,
p799-805.
\end{enumerate}

The NWChem COSMO module implements algorithm for calculation of the
energy for Hartree-Fock (RHF and ROHF) and Kohn-Sham (restricted and unrestricted)
wavefunctions. At the present gradients are calculated by finite
difference of the energy and the code does not work with spherical
basis functions or ECPs.  In the current implementation the code
calculates the gas-phase energy of the system followed by the
solution-phase energy, and returns the electrostatic contribution
to the solvation free energy. The codes does not calculate the
non-electrostatic contributions to the free energy.

Invoking the COSMO solvation model is done by specifying the input
COSMO input block with the input options as follows:

\begin{verbatim}
cosmo
    dielec  <real dielec default 78.4>
    radius  <real atom1>
            <real atom2>
       . . .
            <real atomN>
    rsolv   <real rsolv default 0.00>
    iscren  <integer iscren default 0>
    minbem  <integer minbem default 2>
    maxbem  <integer maxbem default 3>
    ificos  <integer ificos default 0>
    lineq   <integer lineq default 1>
END
\end{verbatim}

\verb+Dielec+ is the value of the dielectric constant of the medium, 
with a default value of 78.4 (the dielectric constant for water).

\verb+Radius+ is an array that specifies the radius of the spheres
associated with each atom and that make up the molecule-shaped cavity.
Default values are Van der Waals radii. Values are in units of angstroms.
The codes uses the following Van der Waals radii by default:

\begin{verbatim}
      data vdwr(103) /
     1   0.80,0.49,0.00,0.00,0.00,1.65,1.55,1.50,1.50,0.00,
     2   2.30,1.70,2.05,2.10,1.85,1.80,1.80,0.00,2.80,2.75,
     3   0.00,0.00,1.20,0.00,0.00,0.00,2.70,0.00,0.00,0.00,
     4   0.00,0.00,0.00,1.90,1.90,0.00,0.00,0.00,0.00,1.55,
     5   0.00,1.64,0.00,0.00,0.00,0.00,0.00,0.00,0.00,0.00,
     6   0.00,0.00,0.00,0.00,0.00,0.00,0.00,0.00,0.00,0.00,
     7   0.00,0.00,0.00,0.00,0.00,0.00,0.00,0.00,0.00,0.00,
     8   0.00,0.00,0.00,0.00,0.00,0.00,0.00,0.00,0.00,0.00,
     9   0.00,0.00,0.00,0.00,0.00,0.00,0.00,0.00,0.00,0.00,
     1   0.00,0.00,0.00,0.00,0.00,0.00,0.00,0.00,0.00,1.65,
     2   0.00,0.00,0.00/
\end{verbatim}

with 0.0 values replaced by 1.80. Other radii can be used as well.
See for examples:

\begin{enumerate}
\item E. V. Stefanovich and T. N. Truong, Chem.~Phys.~Lett. 244, 65 (1995).
\item V. Barone, M. Cossi, and J. Tomasi, J.~Chem.~Phys. 107, 3210 (1997).
\end{enumerate}

\verb+Rsolv+ is a parameter used to define the solvent accessible
surface. See the original reference of Klamt and Schuurmann for a
description. The default value is 0.00 (in angstroms).

\verb+Iscren+ is a flag to define the dielectric charge scaling option.
``{\tt iscren 1}'' implies the original scaling from Klamt and Sch\"{u}\"{u}rmann,
mainly ``$(\epsilon-1)/(\epsilon+1/2)$'', where $\epsilon$ is the dielectric constant.
``{\tt iscren 0}'' implies the modified scaling suggested by Stefanovich and
Truong, mainly ``$(\epsilon-1)/\epsilon$''. Default is to use the modified scaling.
For high dielectric the difference between the scaling is not 
significant.

The next three parameters define the tesselation of the unit sphere.
The approach follows the original proposal by Klamt and Sch\"{u}\"{u}rmann.
A very fine tesselation is generated from \verb+maxbem+ refining 
passes starting from either an octahedron or an icosahedron. The
boundary elements created with the fine tesselation are condensed
down to a coarser tesselation based on \verb+minbem+. The induced
point charges from the polarization of the medium are assigned to
the centers of the coarser tesselation. Default values are
``{\tt minbem 2}'' and ``{\tt maxbem 3}''. The flag \verb+ificos+ serves to
select the original tesselation, ``{\tt ificos 0}'' for an octahedron
(default) and ``{\tt ificos 1}'' for an icoshedron. Starting from an icosahedron
yields a somewhat finer tesselation that converges somewhat faster.
Solvation energies are not really sensitive to this choice for
sufficiently fine tesselations.

The \verb+lineq+ parameter serves to select the numerical algorithm to solve
the linear equations yielding the effective charges that represent
the polarization of the medium. ``{\tt lineq 0}'' selects an iterative method 
(default), ``{\tt lineq 1}'' selects a dense matrix linear equation solver.
For large molecules where the number of effective charges is large,
the codes selects the iterative method.

The following example is for a water molecule in `water', using
the HF/6-31G** level of theory:

\begin{verbatim}
start
echo
 title "h2o"
geometry
o                  .0000000000         .0000000000        -.0486020332
h                  .7545655371         .0000000000         .5243010666
h                 -.7545655371         .0000000000         .5243010666
end
basis segment cartesian
  o library 6-31g**
  h library 6-31g**
end
cosmo
  dielec 78.0
  radius 1.40
         1.16
         1.16
  rsolv  0.50
  lineq  0
end
task scf energy
\end{verbatim}
