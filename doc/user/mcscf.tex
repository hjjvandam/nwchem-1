\label{sec:mcscf}
\Large
*****
It would be really nice if someone could supply a '25 words or less'
description of the Multiconfiguration SCF module right here.********
\normalsize

The Multiconfiguration SCF program is invoked in NWChem 
by specifying the keyword \verb+mcscf+ on the compound directive,

\begin{verbatim}
  MCSCF
    ...
  END
\end{verbatim}

The keyword \verb+mcscf+ tells the code that this is a compound directive,
and additional directives may be specified by the user to define the particular
problem.  The \verb+mcscf+ input will be processed until the
\verb+END+ directive is encountered.  A calculation will be performed 
only when the input encounters a \verb+TASK+ directive of the form,

\begin{verbatim}
  TASK MCSCF
\end{verbatim}

\Large
***Is there a default operation for the mcscf module?  What operations
can be performed in this module?***
\normalsize

% This compound directive provides input for the MCSCF module.
% Currently, only Complete Active Space SCF (CASSCF) wavefunctions are
% supported. The MCSCF module should ideally be invoked from a previously
% converged RHF/ROHF SCF calculation.
% The following input options have the same meaning
% and syntax as with the SCF directive (\ref{SCF}); \verb+maxiter+,
% \verb+thresh+, \verb+tol2e+ and \verb+vectors+.  These options default to
% their corresponding SCF values if available, otherwise the SCF defaults
% listed in section \ref{SCF} are used. The converged MCSCF molecular
% orbitals and ``orbital energies'' are stored in ``canonical''
% representation with the \verb+vectors output+ option.

In the current version of NWChem, only Complete Active Space SCF (CASSCF) 
wavefunctions are supported in the Multiconfiguration SCF module.  Calculations
with this module require data from a converged  RHF/ROHF SCF calculation
(see Section \ref{SCF}).  The results from a previous SCF calculation must be
in the database, or a \verb+TASK+ directive specifying such a calculation
must be included in the input file before the \verb+TASK+ directive for
the MCSCF calculation.

The MCSCF calculation uses the input parameter specifications from the
previous SCF calculation for the directives \verb+vectors+, \verb+maxiter+,
\verb+thresh+ and \verb+tol2e+.  These  input directives 
are repeated here for reference.  (See Section \ref{SCF} for a complete
description of the SCF module input.)


\begin{verbatim}
  VECTORS [input (<string input_movecs default atomic>) || \
                   (project <string basisname> <string filename>)] \
          [swap [alpha|beta] <integer vec1 vec2> ...] \
          [output <string output_movecs default input_movecs>] \
          [lock]

   MAXITER <integer maxiter default 8>

   THRESH  <real thresh default 1.0e-5>

   TOL2E <real tol2e default 1.0e-7>
\end{verbatim}

The converged MCSCF molecular orbital vectors and ``orbital energies'' are 
stored in ``canonical'' representation, in the location specified according
to the option selected after the keyword \verb+output+ on the \verb+VECTORS+
directive in the SCF input.
  
% with the \verb+vectors output+ option.

Additional input for the MCSCF module is required to define active and
inactive orbitals, and to specify the charge in the active space.  The
user also has the option of specifying the type of orbital Hessian used, 
and can modify certain parameters using \verb+SET+ directives.  The following
subsections describe the input directives that can be specified within
the \verb+mcscf+ multiple directive.


% The following three directives; \verb+active+, \verb+actelec+ and
% \verb+multiplicity+ are mandatory and together with the charge specify
% the CASSCF wavefunction. The inactive and secondary orbitals are
% determined by reconciling the number of electrons and basis
% functions. Currently, there is no facility to designate frozen core or
% virtual orbitals. 

% The CI submodule is determinant-based.  Consequently L\"owdin spin
% projection is used for low-spin states.

% \subsection{Inactive}
\subsection{Defining the CASSCF Wavefunction for Multiconfiguration SCF}

The directives \verb+INACTIVE+, \verb+ACTIVE+, \verb+ACTELEC+ and
\verb+MULTIPLICITY+ must be specified for any MCSCF calculation.  The
input parameters supplied with these directives and the charge are used to 
specify the CASSCF wavefunction.

The lowest inactive orbitals are defined by specifying a directive of
the form,

\begin{verbatim}
  inactive <integer inactive>
\end{verbatim}

Orbitals lower than the value entered for the integer \verb+inactive+
are the inactive orbitals.

The active orbitals are specified by entering a directive of the form,

% \subsection{Active}
\begin{verbatim}
  active <integer active>
\end{verbatim}

The integer value entered for \verb+active+ designates the number of active
orbitals following the inactive orbitals (as defined by the entry for
\verb+inactive+ on the \verb+INACTIVE+ directive). 

% \subsection{Active Electrons}

The number of electrons in the active space is defined by entering a
directive of the form,

\begin{verbatim}
  actelec <integer actelec>
\end{verbatim}

The number of electrons,  $N_{a}$, is specified by the value entered
for the integer \verb+actelec+. The inactive and secondary orbitals are
determined by reconciling the number of electrons and basis
functions. Currently, there is no facility to designate frozen core or
virtual orbitals. 



% Sets the number of electrons, $N_{a}$, in the active space.

% \subsection{Multiplicity}

The spin multiplicity of the wavefunction is defined using a directive
of the form,

\begin{verbatim}
  multiplicity <integer multiplicity>
\end{verbatim}

The spin multiplicity of the wavefunction, $(2S + 1)$ is entered as the
value for \verb+multiplicity+ on this input directive. The number
of active alpha and beta electrons is determined by using $M_{s} =
(n_{\alpha} - n_{\beta})/2 = S$ and $N_{a} = n_{\alpha} + n_{\beta}$.

The CI submodule is determinant-based, and for this reason L\"owdin spin
projection is used for low-spin states.

\Large
**Why is this worth mentioning?
\normalsize

% \subsection{Symmetry}

By default, the state symmetry of the wavefunction is ***what?***, which is
represented with a value of 1 for the variable \verb+state_irrep+.  The
user has the option of specifying the state symmetry, with a directive of
the form,

\begin{verbatim}
  symmetry <integer state_irrep default 1>
\end{verbatim}

\Large*** What can the user do here?  What state symmetries might be
desirable?
\normalsize
  
% Sets the desired state symmetry.

% \subsection{Level Shift}

\subsection{Hessian Type in the Multiconfiguration SCF Module}

% \subsection{Hessian Type}

By default, the MCSCF module uses the exact orbital Hessian in the Newton-
Raphson orbital solution.  The user can specify a different choice for the form
of the orbital Hessian using the directive,

\begin{verbatim}
  hessian <string type default ``exact'' >
\end{verbatim}

% This directive selects the type of orbital Hessian used in the
%  Newton-Raphson orbital solution. Possible choices are \verb+onel+ and
% \verb+exact+. 
Entering the value \verb+onel+ for the string \verb+type+ specifies
that Fock operators will be used to
approximate the Hessian with first-order convergence. This will usually
be faster than the default option, (obtained with \verb+type+ specified 
as \verb+exact+), since the exact Hessian which involves a Fock matrix
construction each microiteration. 

If level-shifting has been specified in the preceeding SCF calculation,
the same parameters will be used in the MCSCF calculation.  (See Section
\ref{sec:level} for a description of the level-shifting input for
the SCF calculation.)  The user has the option of specifying the level-shifting
separately for the MCSCF calculation, with a directive of the form,

\begin{verbatim}
  level <real levelshift default 0.1>
\end{verbatim}

The value entered for \verb+levelshift+ is the initial level shift applied 
to the orbital Hessian. 
% Note this also defaults to the SCF level shift if present. 
When this option is used, the specified level shift
remains constant at each macroiteration until the gradient norm is
less than $10^{-2}$~au, at which point it is reduced to zero.

% \subsection{Profile}
% \begin{verbatim}
%   profile <string options>
% \end{verbatim}
% The directive by itself enables general profiling of MCSCF components
% including orbital solution, CI solution and transformation. Additional
% options; \verb+fock+ and \verb+ci+ enable more detailed profiling of
% these components

% \subsection{Hessian Type}
% \begin{verbatim}
%   hessian <string type default ``exact'' >
% \end{verbatim}
% This directive selects the type of orbital Hessian used in the
% Newton-Raphson orbital solution. Possible choices are \verb+onel+ and
% \verb+exact+. The \verb+onel+ option uses only Fock operators to
% approximate the Hessian with first-order convergence. The \verb+exact+
% option selects the exact Hessian which involves a Fock matrix
% construction each microiteration. 

\subsection{Input by SET Directives for the MCSCF Module}

The user has the option of exercising greater control over the MCSCF 
module by specifying values for the variables listed in 
Table~(\ref{MCSCF_variables}).  These variables can be modified using
the \verb+SET+ directive (see Section \ref{sec:set}).  In most instances, the
defaults should suffice, but for some exceptional cases improved performance
may be acheived with different values.

In particular, the variables \verb+ciiterlo+ and \verb+ciiterhi+ which
specify the macroiteration range where the CI wavefunction is optimized,
may need to be changed for some problems. By
default, CI wavefunction optimization is performed at every
macroiteration. Furthermore, unless \verb+ciiterlo+ is set to zero,
the very first CI optimization is {\em always} performed in order to
generate an initial density. Setting \verb+ciiterlo+ to zero
forces an high-spin ROHF calculation. A possible convergence strategy
is to set the \verb+ciiterlo+ to about 5 to ensure a tight
optimization of the inactive orbitals.

\Large
***any other suggestions, such as on the usefulness of the other
variables in the table?
\normalsize

\begin{table}
\caption{MCSCF variables}
\label{MCSCF_variables}
\vspace{.2in}
\begin{tabular}{lrrl}
\hline\hline
Variable                        & Type     & Default          & Synopsis \\
\hline
\verb+mcscf:e2approx+           & logical  &  TRUE            & Use E2(T) extrapolation \\
\verb+mcscf:microci+            & logical  &  TRUE            & CI relaxed within line search \\
\verb+mcscf:conjugacy+          & logical  &  TRUE            & Conjugacy used \\
\verb+mcscf:cgreset+            & integer  &  -1              & Interval between conjugacy resets \\
\verb+mcscf:canonical+          & logical  &  TRUE            & Generate and save canonical orbitals \\
\verb+mcscf:movecs lock+        & logical  &  FALSE           & Lock MO ordering to previously saved \\
\verb+mcscf:citol+              & real     &  1.d-8           & CI convergence tolerance \\
\verb+mcscf:line_search_tol+    & real     &  0.1             & Line search tolerance  \\
\verb+mcscf:ciiterlo+           & integer  &  1               & Macro CI relax low end \\
\verb+mcscf:ciiterhi+           & integer  &  \verb+maxiter+  & Macro CI relax high end \\
\hline\hline
\end{tabular}
\end{table}

% \subsection{Profile}

The directive by itself enables general profiling of MCSCF components
including orbital solution, CI solution and transformation. Additional
options; \verb+fock+ and \verb+ci+ enable more detailed profiling of
these components

% \subsection{Variables and Print Control}
\subsection{Output Control for the Multiconfiguration SCF Module}

The user can obtain general profiling of the MCSCF components, including
the orbital solution, CI solution, and transformation, by specifying a
directive of the form,

\begin{verbatim}
  profile <string options>
\end{verbatim}

Entries \verb+fock+ and \verb+ci+ for the string \verb+options+ produces
additional detailed profiling of these components.


%  Greater control of the MCSCF module is provided by several variables
% listed in the table~(\ref{MCSCF_variables}). In most instances, the
% defaults should suffice. For some exceptional cases, these variables can
% be modified using the \verb+SET+ syntax (see \ref{sec:set}).

% The variables, \verb+ciiterlo+ and \verb+ciiterhi+, specify the
% macroiteration range where the CI wavefunction is optimized. By
% default, CI wavefunction optimization is performed at every
% macroiteration. Furthermore, unless \verb+ciiterlo+ is set to zero,
% the very first CI optimization is {\em always} performed in order to
% generate an initial density. While setting \verb+ciiterlo+ to zero
% forces an high-spin ROHF calculation. A possible convergence strategy
% is to set the \verb+ciiterlo+ to about 5 to ensure a tight
% optimization of the inactive orbitals.

Specific output items can selectively enabled or disabled using the
\verb+print+ control mechanism~(\ref{sec:printcontrol}) with the
available print options listed in table(\ref{MCSCF_print_options}).

\begin{table}
\caption{MCSCF Print Options}
\label{MCSCF_print_options}
\vspace{.2in}
\begin{tabular}{lrl}
\hline\hline
Option                          & Class    &  Synopsis \\
\hline
\verb+ci energy+                & default  &  CI energy eigenvalue \\
\verb+fock energy+              & default  &  Energy derived from Fock matrices \\
\verb+gradient norm+            & default  &  Gradient norm \\
\verb+movecs+                   & default  &  Converged occupied MO vectors \\
\verb+trace energy+             & high     &  Trace Energy \\
\verb+converge info+            & high     &  Convergence data and monitoring \\
\verb+precondition+             & high     &  Orbital preconditioner iterations \\
\verb+microci+                  & high     &  CI iterations in line search \\
\verb+canonical+                & high     &  Canonicalization infomation \\
\verb+new movecs+               & debug    &  MO vectors at each macroiteration \\
\verb+ci guess+                 & debug    &  Initial guess CI vector \\
\verb+density matrix+           & debug    &  One- and Two-particle density matrices \\
\hline\hline
\end{tabular}
\end{table}

