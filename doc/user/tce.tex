%
% $Id: tce.tex,v 1.14 2009-03-05 23:44:54 d3p975 Exp $
%
\label{sec:tce}

\section{Overview}

The Tensor Contraction Engine (TCE) Module of NWChem implements 
a variety of approximations that converge at the exact solutions
of Schr\"{o}dinger equation.  They include configuration interaction theory
through singles, doubles, triples, and quadruples substitutions,
coupled-cluster theory through connected singles, doubles, triples, and quadruples substitutions,
and many-body perturbation theory through fourth order in its 
tensor formulation.  Not only optimized parallel programs of some
of these high-end correlation theories are new, but also the way in
which they have been developed is unique.  The working equations
of all of these methods have been derived completely automatically by
a symbolic manipulation program called a Tensor Contraction Engine (TCE),
and the optimized parallel programs have also been computer-generated by the same program,
which were interfaced to NWChem.  The development of the TCE
program and this portion of the NWChem program has been financially 
supported by the United States Department of Energy, Office of Science,
Office of Basic Energy Science, through the SciDAC program.

The capabilities of the module include:
\begin{itemize}
\item Restricted Hartree--Fock, unrestricted Hartree--Fock, and restricted open-shell
Hartree--Fock references,
\item Restricted KS DFT and unrestricted KS DFT references,
\item Unrestricted configuration interaction theory (CISD, CISDT, and CISDTQ),
\item Unrestricted coupled-cluster theory (LCCD, CCD, LCCSD, CCSD, QCISD, CCSDT, CCSDTQ),
\item Unrestricted iterative many-body perturbation theory [MBPT(2), MBPT(3), MBPT(4)] in its tensor formulation,
\end{itemize}
New capabilities added in the version 4.6 are:
\begin{itemize}
\item Unrestricted coupled-cluster singles and doubles with perturbative connected triples \{CCSD(T), CCSD[T]\},
\item Unrestricted equation-of-motion coupled-cluster theory (EOM-CCSD, EOM-CCSDT, EOM-CCSDTQ) for excitation energies, transition moments and oscillator strengths, and excited-state dipole moments,
\item Unrestricted coupled-cluster theory (CCSD, CCSDT, CCSDTQ) for dipole moments.
\end{itemize}
Version 4.6 and onwards the distributed binary executables do not contain CCSDTQ and its
derivative methods, owing to their large volume.  The source code includes them, so a user
can reinstate them by \verb+setenv CCSDTQ yes+ and recompile TCE module.
The following optimizations have been used in the module:
\begin{itemize}
\item Spin symmetry (spin integration is performed wherever possible within the
unrestricted framework, making the present unrestricted program 
optimal for an open-shell system.  The spin adaption was not performed, 
although in a restricted calculation for a closed-shell system, certain spin blocks of 
integrals and amplitudes are further omitted by symmetry, and
consequently, the present unrestricted CCSD requires only twice
as many operations as a spin-adapted restricted CCSD for a closed-shell system),
\item Point-group symmetry,
\item Index permutation symmetry,
\item Runtime orbital range tiling for memory management,
\item Dynamic load balancing (local index sort and matrix multiplications) parallelism,
\item Multiple parallel I/O schemes including fully in-core algorithm using Global Arrays,
\item Frozen core and virtual approximation.
\item DIIS extrapolation and Jacobi update of excitation amplitudes
\end{itemize}

In addition to changes made in the 4.7 version the most essential component of the 5.0 release 
include:

\begin{itemize}
\item Several variants of active-space CCSDt and EOMCCSDt methods that employ limited set of triply 
excited cluster amplitudes defined by active orbitals.
\item  Ground-state non-iterative CC approaches that account for the effect of triply and/or
quadruply excited connected clusters: the perturbative approaches based on the similarity 
transformed Hamiltonian: CCSD(2), CCSD(2)$_T$, CCSDT(2)$_Q$, the completely and locally renormalized 
methods: CR-CCSD(T), LR-CCSD(T), LR-CCSD(TQ)-1.
\item Excited-state non-iterative corrections due to triples to the EOMCCSD excitation energies:
the completely renormalized EOMCCSD(T) method (CR-EOMCCSD(T)).
\item Improved DIIS solver.
\item New form of the offset tables for files that store cluster amplitudes, recursive intermediates, and one- and two-electron integrals.
\item More efficient storage of two-electron integrals for CC calculations based on the RHF or ROHF references.
\item Improved scalability of the CCSD and CCSD(T) codes.
\end{itemize} 

New features in the 5.1 release included:
\begin{itemize}
\item Dynamic dipole polarizabilities at the CCSD and CCSDT levels using the linear response formalism.
\item Ground- and excited- states the iterative second-order model CC2.
\item Improved algorithms for the 2-e integral transformation.
\item Improvements in time-to-solution and memory use for commonly used methods.
\end{itemize} 

New features in the 5.2 release include:
\begin{itemize}
\item Additional algorithms for the 2-e integral transformation, including efficient and scalable spin-free out-of-core $N^5$ algorithms.
\item Hybrid I/O schemes for both spin-orbital and spin-free calculations which eliminate the memory bottleneck of the 2-e integrals 
in favor of disk storage.  Calculations with nearly 400 basis functions at the CCSD(T) can be performed on workstation using this method.
\item Parallel check-pointing and restart for ground-state (including property) calculations at the CCSD, CCSDT and CCSDTQ levels of theory.
\item Dynamic dipole polarizabilities at the CCSDTQ level using the linear response formalism.
\end{itemize} 

\section{Performance of CI, MBPT, and CC methods}

For reviews or tutorials of these highly-accurate correlation methods, the user is referred to:
\begin{itemize}
\item Trygve Helgaker, Poul Jørgensen and Jeppe Olsen, \textit{Molecular Electronic-Structure Theory}.
\item A. Szabo and N. S. Ostlund, \textit{Modern Quantum Chemistry: Introduction to Advanced Electronic Structure Theory}.
\item B. O. Roos (editor), \textit{Lecture Notes in Quantum Chemistry}.
\end{itemize}

For background on development of the symbolic algebra tools which help create the code used in this model see:
\begin{itemize}
\item S. Hirata, J. Phys. Chem. A {\bf 107,} 9887 (2003).
\item S. Hirata, J. Chem. Phys. {\bf 121}, 51 (2004).
\item S. Hirata, Theor. Chem. Acc. {\bf 116}, 2 (2006).
\end{itemize}

For details of the CC algorithms implemented in 5.0 version, see:
\begin{itemize}
\item S. Hirata, P.-D. Fan, A.A. Auer, M. Nooijen, P. Piecuch, J. Chem. Phys.  {\bf 121}, 12197 (2004).
\item K. Kowalski, S. Hirata, M. W\l och, P. Piecuch, T.L. Windus, J. Chem. Phys. {\bf 123}, 074319 (2005).  
\item K. Kowalski, P. Piecuch, J. Chem. Phys.  {\bf 115}, 643 (2001).  
\item K. Kowalski, P. Piecuch, Chem. Phys. Lett. {\bf 347}, 237 (2001).  
\item P. Piecuch, S.A. Kucharski, and R.J. Bartlett, J. Chem. Phys. {\bf 110}, 6103 (1999).
\item P. Piecuch, N. Oliphant, and L. Adamowicz, J. Chem. Phys.  {\bf 99}, 1875 (1993).
\item N. Oliphant and L. Adamowicz, Int. Rev. Phys. Chem.  {\bf 12}, 339 (1993).
\item P. Piecuch, N. Oliphant, and L. Adamowicz, J. Chem. Phys.  {\bf 99}, 1875 (1993).
\item K. Kowalski, P. Piecuch, J. Chem. Phys. {\bf 120},  1715 (2004).
\item K. Kowalski, J. Chem. Phys. {\bf 123}, 014102 (2005).
\item P. Piecuch, K. Kowalski, I.S.O. Pimienta, M.J. McGuire, Int. Rev. Phys. Chem.  {\bf 21}, 527 (2002).
\item K. Kowalski, P. Piecuch, J. Chem. Phys. {\bf 122}, 074107 (2005)
\item K. Kowalski, W. de Jong, J. Mol. Struct.: THEOCHEM, {\bf 768}, 45 (2006).  
\end{itemize}
and references therein.

For details of the CC methods implemented in 5.1 version, see:
\begin{itemize}
\item O. Christiansen, H. Koch, P. J\o rgensen, Chem. Phys. Lett. {\bf 243}, 409 (1995). 
\item K. Kowalski, J. Chem. Phys. {\bf 125}, 124101 (2006)
\end{itemize}
and references therein.

\section{Algorithms of CI, MBPT, and CC methods}

\subsection{Spin, spatial, and index permutation symmetry}

The TCE thoroughly analyzes the working equation of many-electron theory models and 
automatically generates a program that takes full advantage of these symmetries at the same time.
To do so, the TCE first recognizes the index permutation symmetries among the working equations,
and perform strength reduction and factorization by carefully monitoring the index permutation
symmetries of intermediate tensors.  Accordingly, every input and output tensor (such as 
integrals, excitation amplitudes, residuals) has just two independent but strictly ordered index strings,
and each intermediate tensor has just four independent but strictly ordered index strings.
The operation cost and storage size of tensor contraction is minimized by using the index range 
restriction arising from these index permutation symmetries and also spin and spatial symmetry integration.

\subsection{Runtime orbital range tiling}

To maintain the peak local memory usage at a manageable level, in the beginning of the calculation,
the orbitals are rearranged into tiles (blocks) that contains orbitals with the same spin and spatial
symmetries.  So the tensor contractions in these methods are carried out at the tile level; the spin,
spatial, and index permutation symmetry is employed to reduce the operation and storage cost at the tile 
level also.

\subsection{Dynamic load balancing parallelism}

In a parallel execution, dynamic load balancing of tile-level local tensor index sorting and local 
tensor contraction (matrix multiplication) will be invoked.

\subsection{Parallel I/O schemes}

Each process is assigned a local tensor index sorting and tensor contraction dynamically.  It must first
retrieve the tiles of input tensors, and perform these local operations, and accumulate the output
tensors to the storage.  We have developed a uniform interface for these I/O operations to either
(1) a global file on a global file system, (2) a global memory on a global or distributed memory system,
and (3) semi-replicated files on a distributed file systems.  Some of these operations depend on 
the ParSoft library.

\section{Input syntax}
\label{sec:inputsyntax}

The keyword to invoke the many-electron theories in the module is
\verb+TCE+.  To perform a single-point energy calculation, include
\begin{verbatim}
      TASK TCE ENERGY
\end{verbatim}
in the input file, which may be preceeded by the TCE input block
that details the calculations:
\begin{verbatim}
  TCE
    [(DFT||HF||SCF) default HF=SCF]
    [FREEZE [[core] (atomic || <integer nfzc default 0>)] \
             [virtual <integer nfzv default 0>]]
    [(LCCD||CCD||CCSD||CC2||LR-CCSD||LCCSD||CCSDT||CCSDTA||CCSDTQ|| \
      CCSD(T)||CCSD[T]||CCSD(2)_T||CCSD(2)||CCSDT(2)_Q|| \
      CR-CCSD[T]||CR-CCSD(T)|| \
      LR-CCSD(T)||LR-CCSD(TQ)-1||CREOMSD(T)|| \
      QCISD||CISD||CISDT||CISDTQ|| \
      MBPT2||MBPT3||MBPT4||MP2||MP3||MP4) default CCSD]
    [THRESH <double thresh default 1e-6>]
    [MAXITER <integer maxiter default 100>]
    [PRINT (none||low||medium||high||debug)
      <string list_of_names ...>]
    [IO (fortran||eaf||ga||sf||replicated||dra||ga_eaf) default ga]
    [DIIS <integer diis default 5>]
    [LSHIFT <double lshift default is 0.0d0>]
    [NROOTS <integer nroots default 0>]
    [TARGET <integer target default 1>]
    [TARGETSYM <character targetsym default 'none'>]
    [SYMMETRY]
    [2EORB]
    [2EMET <integer fast2e default 1>]
    [T3A_LVL] 
    [ACTIVE_OA]
    [ACTIVE_OB]
    [ACTIVE_VA]
    [ACTIVE_VB]
    [DIPOLE]
    [TILESIZE <no default (automatically adjusted)>]
    [(NO)FOCK <logical recompf default .true.>]
    [FRAGMENT <default -1 (off)>]
    [DENSMAT <filename>]
  END
\end{verbatim}
Also supported are energy gradient calculation, geometry optimization,
and vibrational frequency (or hessian) calculation, on the basis of
numerical differentiation.  To perform these calculations, use
\begin{verbatim}
      TASK TCE GRADIENT
\end{verbatim}
or
\begin{verbatim}
      TASK TCE OPTIMIZE
\end{verbatim}
or
\begin{verbatim}
      TASK TCE FREQUENCIES
\end{verbatim}

The user may also specify the parameters of reference wave function calculation
in a separate block for either HF (SCF) or DFT, depending on the first keyword
in the above syntax.

Since every keyword except the model has a default value, a minimal input file will be
\begin{verbatim}
  GEOMETRY
  Be 0.0 0.0 0.0
  END

  BASIS
  Be library cc-pVDZ
  END

  TCE
  ccsd
  END

  TASK TCE ENERGY
\end{verbatim}
which performs a CCSD/cc-pVDZ calculation of the Be atom in its
singlet ground state with a spin-restricted HF reference.

\section{Keywords of {\tt TCE} input block}

\subsection{{\tt HF}, {\tt SCF}, or {\tt DFT} --- the reference wave function}

This keyword tells the module
which of the HF (SCF) or DFT module is going to be used for the calculation
of a reference wave function.  The keyword \verb+HF+ and \verb+SCF+ are
one and the same keyword internally, and are default.  When these are used,
the details of the HF (SCF) calculation can be specified in the SCF input
block, whereas if \verb+DFT+ is chosen, DFT input block may be provided.

For instance, RHF-RCCSDT calculation (R standing for spin-restricted) 
can be performed with the following input blocks:
\begin{verbatim}
  SCF
  SINGLET
  RHF
  END

  TCE
  SCF
  CCSDT
  END

  TASK TCE ENERGY
\end{verbatim}
This calculation (and any correlation calculation in the TCE module using a RHF or RDFT
reference for a closed-shell system) skips the storage and computation of all $\beta$ spin
blocks of integrals and excitation amplitudes.  ROHF-UCCSDT (U standing for spin-unrestricted)
for an open-shell doublet system can be requested by
\begin{verbatim}
  SCF
  DOUBLET
  ROHF
  END

  TCE
  SCF
  CCSDT
  END

  TASK TCE ENERGY
\end{verbatim}
and likewise, UHF-UCCSDT for an open-shell doublet system can be specified with
\begin{verbatim}
  SCF
  DOUBLET
  UHF
  END

  TCE
  SCF
  CCSDT
  END

  TASK TCE ENERGY
\end{verbatim}
The operation and storage costs of the last two calculations are identical.  To use the
KS DFT reference wave function for a UCCSD calculation of an open-shell doublet system,
\begin{verbatim}
  DFT
  ODFT
  MULT 2
  END

  TCE
  DFT
  CCSD
  END

  TASK TCE ENERGY
\end{verbatim}
Note that the default model of the DFT module is LDA.

\subsection{{\tt CCSD},{\tt CCSDT},{\tt CCSDTQ},{\tt CISD},{\tt CISDT},{\tt CISDTQ},
{\tt MBPT2},{\tt MBPT3},{\tt MBPT4}, etc.
 --- the correlation models}

These keywords stand for the following models:
\begin{itemize}
\item LCCD: linearized coupled-cluster doubles,
\item CCD: coupled-cluster doubles,
\item LCCSD: linearized coupled-cluster singles \& doubles,
\item CCSD: coupled-cluster singles \& doubles (also EOM-CCSD),
\item LR-CCSD: locally renormalized EOMCCSD method,  
\item CC2: second-order approximate coupled cluster with singles and doubles model
\item CCSDT: coupled-cluster singles, doubles, \& triples (also EOM-CCSDT),
\item CCSDTA: coupled-cluster singles, doubles, \& active triples (also EOM-CCSDT).
Three variants of the active-space CCSDt and EOMCCSDt approaches can be selected based on 
various definitions of triply excited clusters: (1) version I (keyword {\tt T3A\_LVL  1}) uses
the largest set of triply excited amplitudes defined by at least one occupied and one unoccupied 
active spinorbital labels. (2) Version II (keyword {\tt T3A\_LVL  2}) uses
triply excited amplitudes that carry at least two occupied and unoccupied
active spinorbital labels. (3) Version III (keyword {\tt T3A\_LVL  3}) uses
triply excited amplitudes that are defined by active indices only.
Each version requires defining relevant set of occupied active $\alpha$ and $\beta$ spinorbitals
({\tt ACTIVE\_OA} and {\tt ACTIVE\_OB}) as well as active unoccupied $\alpha$ and $\beta$ spinorbitals
({\tt ACTIVE\_VA} and {\tt ACTIVE\_VB}).
\item CCSDTQ: coupled-cluster singles, doubles, triples, \& quadruples (also EOM-CCSDTQ),
\item CCSD(T): CCSD and perturbative connected triples,
\item CCSD[T]: CCSD and perturbative connected triples,
\item CR-CCSD[T]: completely renormalized CCSD[T] method,
\item CR-CCSD(T): completely renormalized CCSD(T) method,
\item CCSD(2)\_T: CCSD and perturbative CCSD(T)$_T$ correction,
\item CCSD(2)\_TQ: CCSD and  perturbative CCSD(2) correction,
\item CCSDT(2)\_Q: CCSDT and perturbative CCSDT(2)$_Q$ correction.
\item LR-CCSD(T): CCSD and perturbative locally renormalized CCSD(T) correction,
\item LR-CCSD(TQ)-1: CCSD and perturbative locally renormalized CCSD(TQ) (LR-CCSD(TQ)-1) correction,
\item CREOMSD(T): EOMCCSD energies and completely renormalized EOMCCSD(T)(IA) correction.
In this option NWCHEM prints two components: (1) total energy of the K-th state 
$E_K=E_K^{\rm EOMCCSD}+\delta_K^{\rm CR-EOMCCSD(T),IA}(T)$ and (2)
the so-called $\delta$-corrected EOMCCSD excitation energy
$\omega_K^{\rm CR-EOMCCSD(T),IA}=\omega_K^{\rm EOMCCSD}+\delta_K^{\rm CR-EOMCCSD(T),IA}(T)$.
\item QCISD: quadratic configuration interaction singles \& doubles,
\item CISD: configuration interaction singles \& doubles,
\item CISDT: configuration interaction singles, doubles, \& triples,
\item CISDTQ: configuration interaction singles, doubles, triples, \& quadruples,
\item MBPT2=MP2: iterative tensor second-order many-body or M\o ller--Plesset perturbation theory,
\item MBPT3=MP3: iterative tensor third-order many-body or M\o ller--Plesset perturbation theory,
\item MBPT4=MP4: iterative tensor fourth-order many-body or M\o ller--Plesset perturbation theory,
\end{itemize}

All of these models are based on spin-orbital expressions of the amplitude and energy equations, 
and designed primarily for spin-unrestricted reference wave functions.  However, for a restricted 
reference wave function of a closed-shell system, some further reduction of operation and storage
cost will be made.  Within the unrestricted framework, all these methods take full advantage
of spin, spatial, and index permutation symmetries to save operation and storage costs at every
stage of the calculation.  Consequently, these computer-generated programs will perform significantly
faster than, for instance, a hand-written spin-adapted CCSD program in NWChem, although the nominal 
operation cost for a spin-adapted CCSD is just one half of that for spin-unrestricted CCSD (in spin-unrestricted
CCSD there are three independent sets of excitation amplitudes, whereas in spin-adapted CCSD there
is only one set, so the nominal operation cost for the latter is one third of that of the former.  For 
a restricted reference wave function of a closed-shell system, all $\beta$ spin block of the excitation
amplitudes and integrals can be trivially mapped to the all $\alpha$ spin block, reducing the ratio
to one half).

While the MBPT (MP) models implemented in the TCE module give identical correlation energies as 
conventional implementation for a canonical HF reference of a closed-shell system, the former are intrinsically
more general and theoretically robust for other less standard reference wave functions and open-shell systems.
This is because the zeroth order of Hamiltonian is chosen to be the full Fock operatior (not just the diagonal
part), and no further approximation was invoked.  So unlike the conventional implementation where the Fock
matrix is assumed to be diagonal and a correlation energy is evaluated in a single analytical formula that involves
orbital energies (or diagonal Fock matrix elements), the present tensor MBPT requires the iterative solution
of amplitude equations and subsequent energy evaluation and is generally more expensive than the former.
For example, the operation cost of many conventional implementation of MBPT(2) scales as the fourth power 
of the system size, but the cost of the present tensor MBPT(2) scales as the fifth power of the system size,
as the latter permits non-canonical HF reference and the former does not (to reinstate the non-canonical HF 
reference in the former makes it also scale as the fifth power of the system size).

\subsection{{\tt THRESH} --- the convergence threshold of iterative solutions of amplitude equations}

This keyword specifies the convergence threshold of iterative solutions of amplitude equations,
and applies to all of the CI, CC, and MBPT models.
The threshold refers to the norm of residual,
namely, the deviation from the amplitude equations.
The default value is \verb+1e-6+.

\subsection{{\tt MAXITER} --- the maximum number of iterations}

It sets the maximum allowed number iterations for the iterative solutions of amplitude equations.
The default value is \verb+100+.

\subsection{{\tt IO} --- parallel I/O scheme}

There are five parallel I/O schemes implemented for all the models, which need to be
wisely chosen for a particular problem and computer architecture. 
\begin{itemize}
\item \verb+fortran+ : Fortran77 direct access,
\item \verb+eaf+ : Exclusive Access File library,
\item \verb+ga+ : Fully incore, Global Array virtual file,
\item \verb+sf+ : Shared File library,
\item \verb+replicated+ : Semi-replicated file on distributed file system with EAF library.
\item \verb+dra+ : Distributed file on distributed file system with DRA library.
\item \verb+ga_eaf+ : Semi-replicated file on distributed file system with EAF library. GA is used 
to speedup the file reconciliation.
\end{itemize}
The GA algorithm, which is default, stores all input (integrals and
excitation amplitudes), output (residuals), and intermediate tensors in the shared memory area
across all nodes by virtue of GA library.  This fully incore algorithm replaces disk I/O by
inter-process communications.  This is a recommended algorithm whenever feasible.  Note that 
the memory management through runtime orbital range tiling described above applies to local
(unshared) memory of each node, which may be separately allocated from the shared memory space
for GA.  So when there is not enough shared memory space (either physically or due to software
limitations, in particular, shmmax setting), the GA algorithm can crash due to an out-of-memory error.
The replicated scheme is the currently the only disk-based algorithm for a genuinely distributed
file system.  This means that each node keeps an identical copy of input tensors and
it holds non-identical overlapping segments of intermediate and output tensors in its local disk.
Whenever data coherency is required, a file reconcilation process will take place to make the intermediate
and output data identical throughout the nodes.  This algorithm, while requiring redundant data space on
local disk, performs reasonably efficiently in parallel.  For sequential execution, this reduces 
to the EAF scheme.  For a global file system, the SF scheme is recommended.  This together with
the Fortran77 direct access scheme does not usually exhibit scalability unless shared files on
the global file system also share the same I/O buffer.  For sequential executions, the 
SF, EAF, and replicated schemes are interchangeable, while the Fortran77 scheme is appreciably
slower.

Two new I/O algorithms \verb+dra+ and \verb+ga_eaf+ combines GA and DRA or EAF based replicated 
algorithm.  In the former, arrays that are not active (e.g., prior $T$ amplitudes used in DIIS
or EOM-CC trial vectors) in GA algorithm will be moved to DRA.  In the latter, the intermediates
that are formed by tensor contractions are initially stored in GA, thereby avoiding the need to
accumulate the fragments of the intermediate scattered in EAFs in the original EAF algorithm.
Once the intermediate is formed completely, then it will be replicated as EAFs.

The spin-free 4-index transformation algorithms are exclusively compatible with the GA I/O scheme,
although out-of-core algorithms for the 4-index transformation are accessible using the
\verb+2emet+ options.  See \ref{subsection:2EMET} for details.

\subsection{{\tt DIIS} --- the convergence acceleration}

It sets the number iterations in which a DIIS extrapolation is performed to accelerate
the convergence of excitation amplitudes.  The default value is 5, which means in every
five iteration, one DIIS extrapolation is performed (and in the rest of the iterations,
Jacobi rotation is used).  When zero or negative value is specified, the DIIS is turned
off.  It is not recommended to perform DIIS every iteration, whereas setting a large 
value for this parameter necessitates a large memory (disk) space to keep the excitation
amplitudes of previous iterations. In 5.0 version we significantly improved the DIIS solver
by re-organizing the itrative process and by introducing the level shift option 
({\tt lshift}) that enable to increase small orbital energy differences used in 
calculating the up-dates for cluster amplitudes. 
Typical values for {\tt lshift} oscillates between 0.3 and 0.5 for CC calculations for 
ground states of multi-configurational character.
Otherwise, the value of {\tt lshift} is by default set equal to 0.


\subsection{{\tt FREEZE} --- the frozen core/virtual approximation}

Some of the lowest-lying core orbitals and/or some of the highest-lying
virtual orbitals may be excluded in the calculations
by this keyword (this does not affect the ground state HF or DFT calculation).
No orbitals are frozen by default.  To exclude the atom-like
core regions altogether, one may request
\begin{verbatim}
  FREEZE atomic
\end{verbatim}
To specify the number of lowest-lying occupied orbitals be excluded, one may use
\begin{verbatim}
  FREEZE 10
\end{verbatim}
which causes 10 lowest-lying occupied orbitals excluded.
This is equivalent to writing
\begin{verbatim}
  FREEZE core 10
\end{verbatim}
To freeze the highest virtual orbitals, use the \verb+virtual+
keyword.  For instance, to freeze the top 5 virtuals
\begin{verbatim}
  FREEZE virtual 5
\end{verbatim}

\subsection{{\tt NROOTS} --- the number of excited states}
 
One can specify the number of excited state roots to be determined.  The default
value is \verb+1+.  It is advised that the users request several more roots than actually
needed, since owing to the nature of the trial vector algorithm, some low-lying
roots can be missed when they do not have sufficient overlap with the initial guess
vectors.

\subsection{{\tt TARGET} and {\tt TARGETSYM} --- the target root and its symmetry}
 
At the moment, the first and second geometrical derivatives of excitation
energies that are needed in force, geometry, and frequency calculations are
obtained by numerical differentiation.  These keywords may be used to specify
which excited state root is being used for the geometrical derivative calculation.
For instance, when \verb+TARGET 3+ and \verb+TARGETSYM a1g+ are included in the
input block, the total energy (ground state energy plus excitation energy)
of the third lowest excited state root (excluding the ground state) transforming as
the irreducible representation \verb+a1g+ will be passed to the module which performs
the derivative calculations.  The default values of these keywords are \verb+1+ and \verb+none+,
respectively.
 
The keyword \verb+TARGETSYM+ is essential in excited state geometry
optimization, since it is very common that the order of excited states changes due to
the geometry changes in the course of optimization.  Without specifying the \verb+TARGETSYM+,
the optimizer could (and would likely) be optimizing the geometry of an excited state that
is different from the one the user had intended to optimize at the starting geometry.
On the other hand, in the frequency calculations, \verb+TARGETSYM+ must be \verb+none+,
since the finite displacements given in the course of frequency calculations will lift
the spatial symmetry of the equilibrium geometry.  When these finite displacements can
alter the order of excited states including the target state, the frequency calculation
is not be feasible.

\subsection{{\tt SYMMETRY} --- restricting the excited state symmetry}
 
By adding this keyword to the input block, the user can request the module to
seek just the roots of the specified irreducible representation as 
\verb+TARGETSYM+.  By default, this option is not set.
\verb+TARGETSYM+ must be specified when \verb+SYMMETRY+ is invoked.

\subsection{{\tt 2EORB} --- alternative storage of two-electron integrals}

In the 5.0 version a new option has been added in order to provide more economical 
way of  storing two-electron integrals used in CC calculations based on the  RHF and ROHF 
references. The {\tt 2EORB} keyword can be used for all CC methods except for those 
using an active-space (CCSDt).  All two-electron integrals are
transformed and subsequently stored in a way which is compatible with assumed tiling scheme. 
The transformation from orbital to spinorbital form of the two-electron integrals is 
performed on-the-fly during execution of the CC module. This option, although slower, allows  
to significantly reduce the memory requirements needed by  the first half of 4-index 
transformation and final file with fully transformed two-electron integrals . 
Savings in the memory requirements on the order of magnitude (and more)
have been observed for large-scale open-shell calculations.

\subsection{{\tt 2EMET} --- alternative storage of two-electron integrals}\label{subsection:2EMET}

In addition to the algorithm implemented in the 5.0 version, several new computation-intensive 
algorithms has been added to the 5.1 version with the purpose of improving scalability and overcoming 
local memory bottleneck of the 5.0 {\tt 2EORB} 4-index transformation. 
In order to give the user a full control over this part of the TCE code several keywords were designed to define the most vital
parameters that determine the perfromance of 4-index transformation. All new keywords must  be used
with the {\tt 2EORB} keyword. The {\tt 2emet} keyword (default value 1 or "{\tt 2emet 1}",
refers to the 5.0 version of the 4-index transformation), 
defines the algorithm to be used. By putting "{\tt 2emet 2}" the TCE code will execute 
the algoritm based on the two step procedure with two intermediate files. In some instances 
this algorithm is characterized by better timings compared to algorithms 3 and 4, although it is 
more memory demanding.  In contrast to algorithms nr 1,3, and 4 this approach can make use of  
a disk to store intermediate files.
For this purpose one should use the keyword {\tt idiskx} ("{\tt idiskx 0}" causes
that all intermediate files are stored on global arrays, while "{\tt idiskx 1}"
tells the code to use a disk to store intermediates; default value of {\tt idiskx } is equal 0).
Algorithm nr 3 ("{\tt 2emet 3}") uses only one intermediate file whereas
algorithm nr 4 ("{\tt 2emet 4}") is a version of algorithm 3 with the option of reducing the 
memory requirements. For example, by using new keyword "{\tt split 4}"  we will reduce the size of 
only  intermediate file by factor of 4 (the {\tt split } keyword can be only used in the context of
algorithm nr 4). 
All new algorithms (i.e. {\tt 2emet 2+}) use the {\tt attilesize} setting to define the size of the atomic tile.
By default {\tt attilesize } is set equal 30. For larger systems the use of larger values of 
{\tt attilesize } is recommended (typically between 40-60).

Additional algorithms for version 5.2 are numbers 5, 6 and 9.  Other values of {\tt 2emet} are not 
supported and refer to methods which do not function properly.  Algorithms 5 and 6 were written as
out-of-core $N^5$ methods ({\tt idiskx 1}) and are the most efficient algorithms at the present time.
The corresponding in-core variants ({\tt idiskx 0}) are available but require excessive memory
with respect to the methods discussed above, although they may be faster if sufficient memory
is available (to get enough memory often requires excessive nodes, which decreases performance
in the later stages of the calculation).  The difference between 5 and 6 is that 5 writes
to a single file (SF or GA) while 6 uses multiple files.  For smaller calculations,
particularly single-node jobs, 5 is faster than 6, but for more than a handful of processors,
algorithm 6 should be used.  The perforance discrepancy depends on the hardware used but 
in algorithm eliminates simulataneous disk access on parallel file systems or memory mutexes for
the in-core case.  For NFS filesystems attached to parallel clusters, no performance differences
have been observed, but for Lustre and PVFS they are signficant.  Using algorithm 5 for large parallel
file systems will make the file system inaccessible to other users, invoking the wrath of 
system administrators.

Algorithm 9 is an out-of-core solution to the memory bottleneck of the 2-e integrals.  In this approach,
the intermediates of the 4-index transformation as well as the MO integrals are stored in an SF file.
As before, this requires a shared file system.  The primary purpose of algorithm 9 is to make the
performance of the NWChem coupled-cluster codes competive with fast serial codes on workstations.
It does not work in parallel for all machines and thus should only be used in serial.
It succeeds in this purpose when corresponding functionality is compared.  A more scalable version of
this algorithm is possible, but the utility is limited since large parallel computers do not
permit the wall times necessary to use an out-of-core method, which is necessarily slower than the
in-core variant.  An extensible solution to these issues using complex heterogeneous I/O is in development.

New to 5.2 is the inclusion of multiple {\tt 2emet} options for the spin-orbital transformations, which
are the default when {\tt 2eorb} is not set and are mandatory for UHF and KS references.  The are currently
three algorithms 1, 2 and 3 available.  The numbering scheme does not correspond in any way to the numbering
scheme for the {\tt 2eorb} case, except that {\tt 2emet 1} corresponds to the default algorithm present in
previous releases, which uses the user-defined I/O scheme.  Algorithm 2 ({\tt 2emet 2}) writes an SF file
for the half-transformed integrals, which is at least an order-of-magnitude larger than the fully-transformed
integrals, but stores the fully-transformed integrals in core.  Thus, once the 4-index transformation is complete,
this algorithm will perform exactly as when algorithm 1 is used.  Unfortuntely, the spin-orbital 2-e 
fully-transformed integrals are still quite large and an algorithm corresponding to {\tt 2eorb}/{\tt 2emet=9}
is available with {\tt 2emet 3}.  Algorithm 3 is also limited in its scalability, but it permits
relatively large UHF-based calculations using single workstations for patient users.

In cases where the user has access to both shared and local filesystems for parallel calculations, 
the {\tt permanent\_dir} setting refers to the location of SF files.  The file system for {\tt scratch\_dir}
will not be used for any of the 4-index transformation algorithms which are compatible with {\tt io=ga}.
 
In the later part of this manual several examples illustrate the use of the newly introduced 
keywords.

\subsection{{\tt DIPOLE} --- the ground- and excited-state dipole moments}

When this is set, the ground-state CC calculation will enter another round 
of iterative step for the so-called $\Lambda$ equation to obtain the one-particle
density matrix and dipole moments.  Likewise, for excited-states (EOM-CC), the 
transition moments and dipole moments will be computed when (and only when) this
option is set.  In the latter case, EOM-CC left hand side solutions will be sought
incurring approximately three times the computational cost of excitation energies 
alone (note that the EOM-CC effective Hamiltonian is not Hermitian and has distinct
left and right eigenvectors).

\subsection{{\tt (NO)FOCK} --- (not) recompute Fock matrix}

The default is \verb+FOCK+ meaning that the Fock matrix will
be reconstructed (as opposed to using the orbital energies as the diagonal part of
Fock).  This is essential in getting correct correlation energies with ROHF or DFT
reference wave functions.  However, currently, this module cannot reconstruct the
Fock matrix when one-component relativistic effects are operative.  So when a user
wishes to run TCE's correlation methods with DK or other relativistic reference,
\verb+NOFOCK+ must be set and orbital energies must be used for the Fock matrix.

\subsection{{\tt PRINT} --- the verbosity}

This keyword changes the level of output verbosity.  One may also
request some particular items in Table \ref{tbl:tce-printable} printed.

\begin{table}[htbp]
\begin{center}
\caption{Printable items in the TCE modules and their default print levels.}
\label{tbl:tce-printable}
\begin{tabular}{lll}
\hline\hline
Item                     & Print Level   & Description \\
\hline 
``time''                 & vary          & CPU and wall times \\
``tile''                 & vary          & Orbital range tiling information \\
``t1''                   & debug         & $T_1$ excitation amplitude dumping \\
``t2''                   & debug         & $T_2$ excitation amplitude dumping \\
``t3''                   & debug         & $T_3$ excitation amplitude dumping \\
``t4''                   & debug         & $T_4$ excitation amplitude dumping \\
``general information''  & default       & General information \\
``correlation information''       & default       & TCE information \\
``mbpt2''                & debug         & Caonical HF MBPT2 test \\
``get\_block''            & debug         & I/O information \\
``put\_block''            & debug         & I/O information \\
``add\_block''            & debug         & I/O information \\
``files''                & debug         & File information \\
``offset''               & debug         & File offset information \\
``ao1e''                 & debug         & AO one-electron integral evaluation \\
``ao2e''                 & debug         & AO two-electron integral evaluation \\
``mo1e''                 & debug         & One-electron integral transformation \\
``mo2e''                 & debug         & Two-electron integral transformation \\
\hline\hline
\end{tabular}
\end{center}
\end{table}

\section{Sample input}

The following is a sample input for a ROHF-UCCSD energy calculation of a water radical cation.
\begin{verbatim}
START h2o

TITLE "ROHF-UCCSD/cc-pVTZ H2O"

CHARGE 1

GEOMETRY
O     0.00000000     0.00000000     0.12982363
H     0.75933475     0.00000000    -0.46621158
H    -0.75933475     0.00000000    -0.46621158
END

BASIS
* library cc-pVTZ
END

SCF
ROHF
DOUBLET
THRESH 1.0e-10
TOL2E  1.0e-10
END

TCE
CCSD
END

TASK TCE ENERGY
\end{verbatim}
The same result can be obtained by the following input:
\begin{verbatim}
START h2o

TITLE "ROHF-UCCSD/cc-pVTZ H2O"

CHARGE 1

GEOMETRY
O     0.00000000     0.00000000     0.12982363
H     0.75933475     0.00000000    -0.46621158
H    -0.75933475     0.00000000    -0.46621158
END

BASIS
* library cc-pVTZ
END

SCF
ROHF
DOUBLET
THRESH 1.0e-10
TOL2E  1.0e-10
END

TASK UCCSD ENERGY
\end{verbatim}

EOM-CCSDT calculation for excitation energies, excited-state
dipole, and transition moments.
\begin{verbatim}
START tce_h2o_eomcc
 
GEOMETRY UNITS BOHR
H    1.474611052297904   0.000000000000000   0.863401706825835
O    0.000000000000000   0.000000000000000  -0.215850436155089
H   -1.474611052297904   0.000000000000000   0.863401706825835
END
 
BASIS
* library sto-3g
END
 
SCF
SINGLET
RHF
END
 
TCE
CCSDT
DIPOLE
FREEZE CORE ATOMIC
NROOTS 1
END
 
TASK TCE ENERGY
\end{verbatim}

Active-space CCSDt/EOMCCSDt  calculations (version I) of several excited states of 
the Be$_3$ molecule. Three highest-lying occupied $\alpha$ and $\beta$  orbitals 
({\tt active\_oa} and {\tt active\_ob}) and nine lowest-lying unoccupied 
$\alpha$ and $\beta$  orbitals ({\tt active\_va} and {\tt active\_vb})
define the active space.

\begin{verbatim}

START TCE_ACTIVE_CCSDT

ECHO

GEOMETRY UNITS ANGSTROM
SYMMETRY C2V
BE  0.00  0.00   0.00
BE  0.00  1.137090 -1.96949
end

BASIS spherical
# --- DEFINE YOUR BASIS SET ---
END

SCF
THRESH 1.0e-10
TOL2E 1.0e-10
SINGLET
RHF
END

TCE
FREEZE ATOMIC
CCSDTA
TILESIZE 15
THRESH 1.0d-5
ACTIVE_OA 3
ACTIVE_OB 3
ACTIVE_VA 9
ACTIVE_VB 9
T3A_LVL 1
NROOTS  2
END 

TASK TCE ENERGY


\end{verbatim}

Completely renormalized EOMCCSD(T) (CR-EOMCCSD(T)) calculations for the ozone molecule as described by the POL1 basis 
set. The {\tt CREOMSD(T)} directive automatically initialize three-step procedure: (1) CCSD calculations; (2)
EOMCCSD calculations; (3) non-iterative CR-EOMCCSD(T) corrections.  

\begin{verbatim}
START TCE_CR_EOM_T_OZONE

ECHO

GEOMETRY UNITS BOHR
SYMMETRY C2V
O   0.0000000000        0.0000000000        0.0000000000
O   0.0000000000       -2.0473224350       -1.2595211660
O   0.0000000000        2.0473224350       -1.2595211660
END

BASIS SPHERICAL
O    S
     10662.285000000      0.00079900
      1599.709700000      0.00615300
       364.725260000      0.03115700
       103.651790000      0.11559600
        33.905805000      0.30155200
O    S
        12.287469000      0.44487000
         4.756805000      0.24317200
O    S
         1.004271000      1.00000000
O    S
         0.300686000      1.00000000
O    S
         0.090030000      1.00000000
O    P
        34.856463000      0.01564800
         7.843131000      0.09819700
         2.306249000      0.30776800
         0.723164000      0.49247000
O    P
         0.214882000      1.00000000
O    P
         0.063850000      1.00000000
O    D
         2.306200000      0.20270000
         0.723200000      0.57910000
O    D
         0.214900000      0.78545000
         0.063900000      0.53387000
END

SCF
THRESH 1.0e-10
TOL2E 1.0e-10
SINGLET
RHF
END

TCE
FREEZE ATOMIC
CREOMSD(T)
TILESIZE 20
THRESH 1.0d-6
NROOTS 2
END

TASK TCE ENERGY

\end{verbatim}


The LR-CCSD(T) calculations for the glycine molecule in the aug-cc-pVTZ basis set.
Option {\tt 2EORB} is used in order to minimize memory requirements 
associated with the storage of two-electron integrals.

\begin{verbatim}
START TCE_LR_CCSD_T

ECHO

GEOMETRY UNITS BOHR
O      -2.8770919486        1.5073755650        0.3989960497
C      -0.9993929716        0.2223265108       -0.0939400216
C       1.6330980507        1.1263991128       -0.7236778647
O      -1.3167079358       -2.3304840070       -0.1955378962
N       3.5887721300       -0.1900460352        0.6355723246
H       1.7384347574        3.1922914768       -0.2011420479
H       1.8051078216        0.9725472539       -2.8503867814
H       3.3674278149       -2.0653924379        0.5211399625
H       5.2887327108        0.3011058554       -0.0285088728
H      -3.0501350657       -2.7557071585        0.2342441831
END

BASIS
* library aug-cc-pVTZ
END

SCF
THRESH 1.0e-10
TOL2E 1.0e-10
SINGLET
RHF
END

TCE
FREEZE ATOMIC
2EORB
TILESIZE 15
LR-CCSD(T)
THRESH 1.0d-7
END

TASK TCE ENERGY

\end{verbatim} 


The CCSD calculations for the triplet state of the C$_{20}$ molecule. New 
algorithms for 4-index tranformation are used.

\begin{verbatim}
title "c20_cage"
echo
start c20_cage
memory stack 2320 mb heap 180 mb global 2000 mb noverify

geometry print xyz units bohr
   symmetry c2
   C      -0.761732  -1.112760   3.451966
   C       0.761732   1.112760   3.451966
   C       0.543308  -3.054565   2.168328
   C      -0.543308   3.054565   2.168328
   C       3.190553   0.632819   2.242986
   C      -3.190553  -0.632819   2.242986
   C       2.896910  -1.982251   1.260270
   C      -2.896910   1.982251   1.260270
   C      -0.951060  -3.770169   0.026589
   C       0.951060   3.770169   0.026589
   C       3.113776   2.128908   0.076756
   C      -3.113776  -2.128908   0.076756
   C       3.012003  -2.087494  -1.347695
   C      -3.012003   2.087494  -1.347695
   C       0.535910  -2.990532  -2.103427
   C      -0.535910   2.990532  -2.103427
   C       3.334106   0.574125  -2.322563
   C      -3.334106  -0.574125  -2.322563
   C      -0.764522  -1.081362  -3.453211
   C       0.764522   1.081362  -3.453211
end

basis spherical
 * library cc-pvtz
end

scf
   triplet
   rohf
   thresh 1.e-8
   maxiter 200
end

tce
   ccsd
   maxiter 60
   diis 5
   thresh 1.e-6
   2eorb
   2emet 3
   attilesize 40
   tilesize 30
   freeze atomic
end

task tce energy

\end{verbatim}

\section{TCE Response Properties}

\subsection{Introduction}

Response properties can be calculated within the TCE.  Ground-state dipole polarizabilities can be performed at the CCSD, CCSDT and CCSDTQ levels of theory.  Neither CCSDT-LR nor CCSDTQ-LR are compiled by default.  Like the rest of the TCE, properties can be calculated with RHF, UHF, ROHF and DFT reference wavefunctions.

Specific details for the implementation of CCSD-LR and CCSDT-LR can be found in the following papers:
\begin{itemize}
\item J. R. Hammond, M. Valiev, W. A. deJong and K. Kowalski, \textit{J. Phys. Chem. A}, \textbf{111}, 5492 (2007).
\item J. R. Hammond, K. Kowalski and W. A. de Jong, \textit{J. Chem. Phys.}, \textbf{127}, 144105 (2007).
\item J. R. Hammond, W. A. de Jong and K. Kowalski , \textit{J. Chem. Phys.}, \textbf{128}, 224102 (2008).
\end{itemize}
An appropriate background on coupled-cluster linear response (CC-LR) can be found in the references of those papers.

\subsection{Performance}

The coupled-cluster response codes were generated in the same manner as the rest of the TCE, thus all previous comments on performance apply here as well.  The improved offsets available in the CCSD and EOM-CCSD codes is now also available in the CCSD-$\Lambda$ and CCSD-LR codes.  The bottleneck for CCSD-LR is the same as EOM-CCSD, likewise for CCSDT-LR and EOM-CCSDT.  The CCSD-LR code has been tested on as many as 1024 processors for systems with more than 2000 spin-orbitals, while the CCSDT-LR code has been run on as many as 1024 processors.  The CCSDTQ-LR code is not particularly useful due to the extreme memory requirements of quadruples amplitudes, limited scalability and poor convergence in the CCSDTQ equations in general.

\subsection{Input}

The input commands for TCE response properties exclusively use set directives (see Section 5.7) instead of TCE input block keywords.  There are currently only three commands available:

\begin{verbatim}
set tce:lineresp <logical lineresp default: F>
set tce:afreq <double precision afreq(9) default: 0.0>
set tce:respaxis <logical respaxis(3) default: T T T>
\end{verbatim}

The boolean variable \verb+lineresp+ invokes the linear response equations for the corresponding coupled-cluster method (only CCSD and CCSDT possess this feature) and evaluates the dipole polarizability.  When \verb+lineresp+ is true, the $\Lambda$-equations will also be solved, so the dipole moment is also calculated.  If no other options are set, the complete dipole polarizability tensor will be calculated at zero frequency (static). Up to nine real frequencies can be set; adding more should not crash the code but it will calculate meaningless quanities.  If one desires to calculate more frequencies at one time, merely change the line \verb+double precision afreq(9)+ in \verb+$(NWCHEM_TOP)/src/tce/include/tce.fh+ appropriately and recompile.

The user can choose to calculate response amplitudes only for certain axis, either because of redundancy due to symmetry or because of memory limitations.  The boolean vector of length three \verb+respaxis+ is used to determine whether or not a given set of response amplitudes are allocated, solved for, and used in the polarizability tensor evaluation.  The logical variables represent the X, Y, Z axes, respectively.  If respaxis is set to \verb+T F T+, for example, the responses with respect to the X and Z dipoles will be calculated, and the four (three unique) tensor components will be evaluated.  This feature is also useful for conserving memory.  By calculating only one axis at a time, memory requirements can be reduced by $25\%$ or more, depending on the number of DIIS vectors used.  Reducing the number of DIIS vectors also reduces the memory requirements.

It is strongly advised that when calculating polarizabilities at high-frequencies, that user set the frequencies in increasing order, preferably starting with zero or other small value.  This approach is computationally efficient (the initial guess for subsequent responses is the previously converged value) and mitigates starting from a zero vector for the response amplitudes.

\subsection{Examples}

This example runs in-core on a large workstation.

\begin{verbatim}
geometry units angstrom
 symmetry d2h
 C               0.000    1.390    0.000
 H               0.000    2.470    0.000
 C               1.204    0.695    0.000
 H               2.139    1.235    0.000
 C               0.000   -1.390    0.000
 H               0.000   -2.470    0.000
 C              -1.204   -0.695    0.000
 H              -2.139   -1.235    0.000
 C               1.204   -0.695    0.000
 H               2.139   -1.235    0.000
 C              -1.204    0.695    0.000
 H              -2.139    1.235    0.000
end

basis spherical
  * library aug-cc-pvdz
end

tce
  freeze atomic
  ccsd
  io ga
  2eorb
  tilesize 16
end

set tce:lineresp T
set tce:afreq 0.000 0.072
set tce:respaxis T T T

task tce energy
\end{verbatim}

This is a relatively simple example for CCSDT-LR.

\begin{verbatim}
geometry units au
  symmetry c2v
  H 0       0        0
  F 0       0        1.7328795
end

basis spherical
  * library aug-cc-pvdz
end

tce
  ccsdt
  io ga
  2eorb
end

set tce:lineresp T
set tce:afreq 0.0 0.1 0.2 0.3 0.4
set tce:respaxis T F T

task tce energy
\end{verbatim}

\section{TCE Restart Capability}

\subsection{Overview}

Check-pointing and restart are critical for computational chemistry applications of any scale, but particularly those done on supercomputers, or run for an extended period on workstations and clusters.  The TCE supports parallel check-pointing and restart using the Shared Files (SF) library in the Global Arrays Tools.  The SF library requires that the file system be accessible by every node, so reading and writing restart files can only be performed on a shared file system.  For workstations, this condition is trivially met.  Restart files must be persistent to be useful, so volatile file systems or those which are periodicly erased should not be used for check-pointing.

Restart is possible for all ground-state amplitudes ($T$, $\Lambda$ and $T^{(1)}$) but not for excited-state amplitudes, as in an EOM-CC calculation.  The latter functionality is under development.

Restart capability is useful in the following situations:
\begin{itemize}
 \item The total run time is limited, as is the case for most supercomputing facilities.
 \item The system is volatile and jobs die randomly.
 \item When doing property calculations which require a large number of responses which cannot all be stored in-core simultaneously.
\end{itemize}

At the present time, restarting the amplitudes during a potential energy surface scan or numerical geometry optmization/frequency calculation is not advised due to the phase issue in the molecular orbital coefficients.  If the phase changes, the amplitudes will no longer be a useful guess and may lead to nonsense results.  Expert users may be able to use restart when the geometry varies using careful choices in the SCF input by using the ``rotate'' and ``lock'' options but this use of restart is not supported.

If SF encounters a failure during restart I/O, the job will fail.  The capability to ignore a subset of failures, such as when saving the amplitudes prior to convergence, will be available in the future.  This is useful on some large machines when the filesystem is being taxed by another job and may be appear unavailable at the moment a check-point write is attempted.  

The performance of SF I/O for restart is excellent and the wall time for reading and writing integrals and amplitudes is negligible, even on a supercomputer (such systems have very fast parallel file systems in most cases).  The only platform for which restart may cause I/O problems is BlueGene, due to ratio of compute to I/O nodes (64 on BlueGene/P).

\subsection{Input}

\begin{verbatim}
set tce:read_integrals <logical read_integrals default: F F F F F>
set tce:read_t <logical read_t default: F F F F>
set tce:read_l <logical read_l default: F F F F>
set tce:read_tr <logical read_tr default: F F F F>
set tce:save_integrals <logical save_integrals default: F F F F F>
set tce:save_t <logical save_t default: F F F F>
set tce:save_l <logical save_l default: F F F F>
set tce:save_tr <logical save_tr default: F F F F>
set tce:save_interval <integer save_interval default: 100000>
\end{verbatim}

The boolean variables \verb+read_integrals+ and \verb+save_integrals+ control which integrals are read/saved.  The first location is the 1-e integrals, the second is for the 2-e integrals, and the third is for dipole integrals.  The fourth and fifth positions are reserved for quadrupole and octupole integrals but this functionality is not available.  The \verb+read_t+, \verb+read_l+, \verb+read_tr+, \verb+save_t+, \verb+save_l+ and \verb+save_tr+ variables control the reading/saving of the $T$, $\Lambda$ and $T^{(1)}$ (response) amplitudes.  Restart control on the response amplitudes is implicitly controlled by the value of \verb+respaxis+ (see above).  Requesting amplitudes that are beyond the scope of a given calculation, such as $T_3$ in a CCSD calculation, does not produce an error as these commands will never be processed.

Attempting to restart with a set of amplitudes without the corresponding integrals is ill-advised, due to the phase issue discussed above.  For the same reason, one cannot save a subset of the integrals, so if it is even remotely possible that the dipole moment or polarizabilities will be desired for a given molecule, the dipole integrals should be saved as well.  It is possible to save the dipole integrals without setting \verb+dipole+ in the TCE input block; setting \verb+save_integrals(3)+ true is sufficient for this to occur.

The \verb+save_interval+ variable controls the frequency with which amplitudes are saved.  By default, the amplitudes are saved only when the iterative process has converged, meaning that if the iterations do not converge in less than the maximum, one must start the calculation again from scratch.  The solution is to set \verb+save_interval+ to a smaller value, such as the number of DIIS cycles being used.

The user shall not change the tilesize when reading in saved amplitudes.  The results of this are catastrophic and under no circumstance will this lead to physically meaningful results.

\subsection{Examples}

\begin{verbatim}
geometry units au
  symmetry c2v
  H 0       0        0
  F 0       0        1.7328795
end

basis spherical
  * library aug-cc-pvdz
end

tce
  ccsdt
  io ga
end

set tce:lineresp T
set tce:afreq 0.0 0.1 0.2 0.3 0.4
set tce:respaxis T F T

task tce energy
\end{verbatim}

\section{Maximizing performance}

The following are recommended parameters for getting the best performance and efficiency for common methods on various hardware configurations.  The optimal settings are far from extensible and it is extremely important that users take care in how they apply these recommendations.  Testing a variety of settings on a simple example is recommended when optimal settings are desired.  Nonetheless, a few guiding principles will improve the performance of TCE jobs markedly, including making otherwise impossible jobs possible.

\subsection{Memory considerations and IO schemes}

\subsection{Processor count}

\subsection{SCF options}

\subsection{Convergence criteria}

\subsection{Integral storage, transformation algorithms and tilesizes}

