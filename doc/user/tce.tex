%
% $Id: tce.tex,v 1.5 2004-04-22 04:50:29 edo Exp $
%
\label{sec:tce}

\section{Overview}

The Tensor Contraction Engine (TCE) Module of NWChem implements 
a variety of approximations that converge at the exact solutions
of Schr\"{o}dinger equation.  They include configuration interaction theory
through singles, doubles, triples, and quadruples substitutions,
coupled-cluster theory through connected singles, doubles, triples, and quadruples substitutions,
and many-body perturbation theory through fourth order in its 
tensor formulation.  Not only optimized parallel programs of some
of these high-end correlation theories are new, but also the way in
which they have been developed is unique.  The working equations
of all of these methods have been derived completely automatically by
a symbolic manipulation program called a Tensor Contraction Engine (TCE),
and the optimized parallel programs have also been computer-generated by the same program,
which were interfaced to NWChem.  The development of the TCE
program and this portion of the NWChem program has been financially 
supported by the United States Department of Energy, Office of Science,
Office of Basic Energy Science, through the SciDAC program.

The capabilities of the module include:
\begin{itemize}
\item Restricted Hartree--Fock, unrestricted Hartree--Fock, and restricted open-shell
Hartree--Fock references,
\item Restricted KS DFT and unrestricted KS DFT references,
\item Unrestricted configuration interaction theory (CISD, CISDT, and CISDTQ),
\item Unrestricted coupled-cluster theory (LCCD, CCD, LCCSD, CCSD, QCISD, CCSDT, CCSDTQ),
\item Unrestricted iterative many-body perturbation theory [MBPT(2), MBPT(3), MBPT(4)] in its tensor formulation,
\end{itemize}
and the following optimizations have been used in the module:
\begin{itemize}
\item Spin symmetry (spin integration is performed wherever possible within the
unrestricted framework, making the present unrestricted program 
optimal for an open-shell system.  The spin adaption was not performed, 
although in a restricted calculation for a closed-shell system, certain spin blocks of 
integrals and amplitudes are further omitted by symmetry, and
consequently, the present unrestricted CCSD requires only twice
as many operations as a spin-adapted restricted CCSD for a closed-shell system),
\item Point-group symmetry,
\item Index permutation symmetry,
\item Runtime orbital range tiling for memory management,
\item Dynamic load balancing (local index sort and matrix multiplications) parallelism,
\item Multiple parallel I/O schemes including fully incore algorithm using Global Arrays,
\item Frozen core and virtual approximation.
\item DIIS extrapolation and Jacobi update of excitation amplitudes
\end{itemize}

This extensible module is designed such that an existing or new model of many-electron
theory can be added and further optimization can be incorporated with ease 
by virtue of the TCE.
This module is still being
actively enhanced by the TCE and we hope to include more models and optimizations
in future releases!

\section{Performance of CI, MBPT, and CC methods}

For reviews or tutorials of these highly-accurate correlation methods, the user is 
referred to:
\begin{itemize}
\item A. Szabo and N. S. Ostlund, Modern Quantum Chemistry: Introduction to Advanced Electronic Structure Theory,
\item R. J. Bartlett and J. F. Stanton, Applications of Post-Hartree-Fock Methods: A Tutorial, in Reviews in Computational Chemistry, Volume V,
\item R. J. Bartlett, Coupled-Cluster Theory: An Overview of Recent Developments, in Modern Electronic Structure Theory, Part II,
\item B. O. Roos (editor), Lecture Notes in Quantum Chemistry I and II.
\end{itemize}

\section{Algorithms of CI, MBPT, and CC methods}

\subsection{Spin, spatial, and index permutation symmetry}

The TCE thoroughly analyzes the working equation of many-electron theory models and 
automatically generates a program that takes full advantage of these symmetries at the same time.
To do so, the TCE first recognizes the index permutation symmetries among the working equations,
and perform strength reduction and factorization by carefully monitoring the index permutation
symmetries of intermediate tensors.  Accordingly, every input and output tensor (such as 
integrals, excitation amplitudes, residuals) has just two independent but strictly ordered index strings,
and each intermediate tensor has just four independent but strictly ordered index strings.
The operation cost and storage size of tensor contraction is minimized by using the index range 
restriction arising from these
index permutation symmetries and also spin and spatial symmetry integration.

\subsection{Runtime orbital range tiling}

To maintain the peak local memory usage at a manageable level, in the beginning of the calculation,
the orbitals are rearranged into tiles (blocks) that contains orbitals with the same spin and spatial
symmetries.  So the tensor contractions in these methods are carried out at the tile level; the spin,
spatial, and index permutation symmetry is employed to reduce the operation and storage cost at the tile 
level also.

\subsection{Dynamic load balancing parallelism}

In a parallel execution, dynamic load balancing of tile-level local tensor index sorting and local 
tensor contraction (matrix multiplication) will be invoked.

\subsection{Parallel I/O schemes}

Each process is assigned a local tensor index sorting and tensor contraction dynamically.  It must first
retrieve the tiles of input tensors, and perform these local operations, and accumulate the output
tensors to the storage.  We have developed a uniform interface for these I/O operations to either
(1) a global file on a global file system, (2) a global memory on a global or distributed memory system,
and (3) semi-replicated files on a distributed file systems.  Some of these operations depend on 
the ParSoft library.

\section{Input syntax}

The keyword to invoke the many-electron theories in the module is
\verb+TCE+.  To perform a single-point energy calculation, include
\begin{verbatim}
      TASK TCE ENERGY
\end{verbatim}
in the input file, which may be preceeded by the TCE input block
that details the calculations:
\begin{verbatim}
  TCE
    [(DFT||HF||SCF) default HF=SCF]
    [(LCCD||CCD||CCSD||LCCSD||CCSDT||CCSDTQ|| \
      QCISD||CISD||CISDT||CISDTQ|| \
      MBPT2||MBPT3||MBPT4||MP2||MP3||MP4) default CCSD]
    [THRESH <double thresh default 1e-6>]
    [MAXITER <integer maxiter default 100>]
    [IO (fortran||eaf||ga||sf||replicated) default ga]
    [DIIS <integer diis default 5>]
    [FREEZE [[core] (atomic || <integer nfzc default 0>)] \
             [virtual <integer nfzv default 0>]]
    [PRINT (none||low||medium||high||debug)
      <string list_of_names ...>]
  END
\end{verbatim}
Also supported are energy gradient calculation, geometry optimization,
and vibrational frequency (or hessian) calculation, on the basis of
numerical differentiation.  To perform these calculations, use
\begin{verbatim}
      TASK TCE GRADIENT
\end{verbatim}
or
\begin{verbatim}
      TASK TCE OPTIMIZE
\end{verbatim}
or
\begin{verbatim}
      TASK TCE FREQUENCIES
\end{verbatim}

Alternatively, more descriptive keywords for each individual method can be used.
For instance, to perform a CCSDT energy, gradient, etc. calculation, use
\begin{verbatim}
      TASK UCCSDT ENERGY
\end{verbatim}
or
\begin{verbatim}
      TASK UCCSDT GRADIENT
\end{verbatim}
or
\begin{verbatim}
      TASK UCCSDT OPTIMIZE
\end{verbatim}
or
\begin{verbatim}
      TASK UCCSDT FREQUENCIES
\end{verbatim}
with an (optional) input block enclosed either by \verb+UCCSDT+ and \verb+END+ or
by \verb+UCC+ and \verb+END+.  The keywords for individual methods of TCE module
always start with letter \verb+U+ which stands for ``unrestricted'' to avoid 
confusion with other related methods (such as spin-restricted CCSD and various 
canonical MP2 implementation) already in place in NWChem.
\begin{verbatim}
  (UCCSDT||UCC)
    [(DFT||HF||SCF) default HF=SCF]
    [THRESH <double thresh default 1e-6>]
    [MAXITER <integer maxiter default 100>]
    [IO (fortran||c||ga||sf||replicated) default ga]
    [DIIS <integer diis default 5>]
    [FREEZE [[core] (atomic || <integer nfzc default 0>)] \
             [virtual <integer nfzv default 0>]]
    [PRINT (none||low||medium||high||debug)
      <string list_of_names ...>]
  END
\end{verbatim}
When a method (CCSDT in this example) is specified in the task directive, 
a duplicate method specification
is not necessary (indeed not allowed) in the corresponding (\verb+UCCSDT+ or \verb+UCC+ in this case) 
input block.  The keywords of the other methods for task directive are:
\begin{verbatim}
      TASK (UCCD||ULCCD||UCCSD||ULCCSD||UQCISD||UCCSDT||UCCSDTQ) ENERGY
\end{verbatim}
or
\begin{verbatim}
      TASK (UCISD||UCISDT||UCISDTQ) ENERGY
\end{verbatim}
or
\begin{verbatim}
      TASK (UMP2||UMP3||UMP4||UMBPT2||UMBPT3||UMBPT4) ENERGY
\end{verbatim}
etc.  The input block can be specified by the same name (\verb+UCISDT+ and \verb+END+
block for \verb+TASK UCISDT ENERGY+) or \verb+UCC+ for the CC family, \verb+UCI+ for 
the CI family, and \verb+UMP+ or \verb+UMBPT+ for the MP family of methods.

The user may also specify the parameters of reference wave function calculation
in a separate block for either HF (SCF) or DFT, depending on the first keyword
in the above syntax.

Since each keyword has a default value, a minimal input file will be
\begin{verbatim}
  GEOMETRY
  Be 0.0 0.0 0.0
  END

  BASIS
  Be library cc-pVDZ
  END

  TASK TCE ENERGY
\end{verbatim}
which performs a CCSD/cc-pVDZ calculation of the Be atom in its
singlet ground state with a spin-restricted HF reference.

\section{Keywords of {\tt TCE} input block}

\subsection{{\tt HF}, {\tt SCF}, or {\tt DFT} --- the reference wave function}

This keyword tells the module
which of the HF (SCF) or DFT module is going to be used for the calculation
of a reference wave function.  The keyword \verb+HF+ and \verb+SCF+ are
one and the same keyword internally, and are default.  When these are used,
the details of the HF (SCF) calculation can be specified in the SCF input
block, whereas if \verb+DFT+ is chosen, DFT input block may be provided.

For instance, RHF-RCCSDT calculation (R standing for spin-restricted) 
can be performed with the following input blocks:
\begin{verbatim}
  SCF
  SINGLET
  RHF
  END

  TCE
  SCF
  CCSDT
  END

  TASK TCE ENERGY
\end{verbatim}
or
\begin{verbatim}
  SCF
  SINGLET
  RHF
  END

  UCCSDT
  SCF
  END

  TASK UCCSDT ENERGY
\end{verbatim}
or
\begin{verbatim}
  SCF
  SINGLET
  RHF
  END

  UCC
  SCF
  END

  TASK UCCSDT ENERGY
\end{verbatim}
This calculation (and any correlation calculation in the TCE module using a RHF or RDFT
reference for a closed-shell system) skips the storage and computation of all $\beta$ spin
blocks of integrals and excitation amplitudes.  ROHF-UCCSDT (U standing for spin-unrestricted)
for an open-shell doublet system can be requested by
\begin{verbatim}
  SCF
  DOUBLET
  ROHF
  END

  TCE
  SCF
  CCSDT
  END

  TASK TCE ENERGY
\end{verbatim}
and likewise, UHF-UCCSDT for an open-shell doublet system can be specified with
\begin{verbatim}
  SCF
  DOUBLET
  UHF
  END

  TCE
  SCF
  CCSDT
  END

  TASK TCE ENERGY
\end{verbatim}
The operation and storage costs of the last two calculations are identical.  To use the
KS DFT reference wave function for a UCCSD calculation of an open-shell doublet system,
\begin{verbatim}
  DFT
  ODFT
  MULT 2
  END

  TCE
  DFT
  CCSD
  END

  TASK TCE ENERGY
\end{verbatim}
Note that the default model of the DFT module is LDA.

\subsection{{\tt CCSD},{\tt CCSDT},{\tt CCSDTQ},{\tt CISD},{\tt CISDT},{\tt CISDTQ},
{\tt MBPT2},{\tt MBPT3},{\tt MBPT4}, etc.
 --- the correlation model}

These keywords stand for the following models:
\begin{itemize}
\item LCCD: linearized coupled-cluster doubles,
\item CCD: coupled-cluster doubles,
\item LCCSD: linearized coupled-cluster singles \& doubles,
\item CCSD: coupled-cluster singles \& doubles,
\item CCSDT: coupled-cluster singles, doubles, \& triples,
\item CCSDTQ: coupled-cluster singles, doubles, triples, \& quadruples,
\item QCISD: quadratic configuration interaction singles \& doubles,
\item CISD: configuration interaction singles \& doubles,
\item CISDT: configuration interaction singles, doubles, \& triples,
\item CISDTQ: configuration interaction singles, doubles, triples, \& quadruples,
\item MBPT2=MP2: iterative tensor second-order many-body or M\o ller--Plesset perturbation theory,
\item MBPT3=MP3: iterative tensor third-order many-body or M\o ller--Plesset perturbation theory,
\item MBPT4=MP4: iterative tensor fourth-order many-body or M\o ller--Plesset perturbation theory,
\end{itemize}

All of these models are based on spin-orbital expressions of the amplitude and energy equations, 
and designed primarily for spin-unrestricted reference wave functions.  However, for a restricted 
reference wave function of a closed-shell system, some further reduction of operation and storage
cost will be made.  Within the unrestricted framework, all these methods take full advantage
of spin, spatial, and index permutation symmetries to save operation and storage costs at every
stage of the calculation.  Consequently, these computer-generated programs will perform significantly
faster than, for instance, a hand-written spin-adapted CCSD program in NWChem, although the nominal 
operation cost for a spin-adapted CCSD is just one half of that for spin-unrestricted CCSD (in spin-unrestricted
CCSD there are three independent sets of excitation amplitudes, whereas in spin-adapted CCSD there
is only one set, so the nominal operation cost for the latter is one third of that of the former.  For 
a restricted reference wave function of a closed-shell system, all $\beta$ spin block of the excitation
amplitudes and integrals can be trivially mapped to the all $\alpha$ spin block, reducing the ratio
to one half).

While the MBPT (MP) models implemented in the TCE module give identical correlation energies as 
conventional implementation for a canonical HF reference of a closed-shell system, the former are intrinsically
more general and theoretically robust for other less standard reference wave functions and open-shell systems.
This is because the zeroth order of Hamiltonian is chosen to be the full Fock operatior (not just the diagonal
part), and no further approximation was invoked.  So unlike the conventional implementation where the Fock
matrix is assumed to be diagonal and a correlation energy is evaluated in a single analytical formula that involves
orbital energies (or diagonal Fock matrix elements), the present tensor MBPT requires the iterative solution
of amplitude equations and subsequent energy evaluation and is generally more expensive than the former.
For example, the operation cost of many conventional implementation of MBPT(2) scales as the fourth power 
of the system size, but the cost of the present tensor MBPT(2) scales as the fifth power of the system size,
as the latter permits non-canonical HF reference and the former does not (to reinstate the non-canonical HF 
reference in the former makes it also scale as the fifth power of the system size).
\subsection{{\tt THRESH} --- the convergence threshold of iterative solutions of amplitude equations}

This keyword specifies the convergence threshold of iterative solutions of amplitude equations,
and applies to all of the CI, CC, and MBPT models.
The threshold refers to the norm of residual,
namely, the deviation from the amplitude equations.
The default value is \verb+1e-6+.

\subsection{{\tt MAXITER} --- the maximum number of iterations}

It sets the maximum allowed number iterations for the iterative solutions of amplitude equations.
The default value is \verb+100+.

\subsection{{\tt IO} --- parallel I/O scheme}

There are five parallel I/O schemes implemented for all the models, which need to be
wisely chosen for a particular problem and computer architecture. 
\begin{itemize}
\item \verb+fortran+ : Fortran77 direct access,
\item \verb+eaf+ : Exclusive Access File library,
\item \verb+ga+ : Fully incore, Global Array virtual file,
\item \verb+sf+ : Shared File library,
\item \verb+replicated+ : Semi-replicated file on distributed file system with EAF library.
\end{itemize}
The GA algorithm, which is default, stores all input (integrals and
excitation amplitudes), output (residuals), and intermediate tensors in the shared memory area
across all nodes by virtue of GA library.  This fully incore algorithm replaces disk I/O by
inter-process communications.  This is a recommended algorithm whenever feasible.  Note that 
the memory management through runtime orbital range tiling described above applies to local
(unshared) memory of each node, which may be separately allocated from the shared memory space
for GA.  So when there is not enough shared memory space (either physically or due to software
limitations, in particular, shmmax setting), the GA algorithm can crash due to an out-of-memory error.
The replicated scheme is the currently the only disk-based algorithm for a genuinely distributed
file system.  This means that each node keeps an identical copy of input tensors and
it holds non-identical overlapping segments of intermediate and output tensors in its local disk.
Whenever data coherency is required, a file reconcilation process will take place to make the intermediate
and output data identical throughout the nodes.  This algorithm, while requiring redundant data space on
local disk, performs reasonably efficiently in parallel.  For sequential execution, this reduces 
to the EAF scheme.  For a global file system, the SF scheme is recommended.  This together with
the Fortran77 direct access scheme does not usually exhibit scalability unless shared files on
the global file system also share the same I/O buffer.  For sequential executions, the 
SF, EAF, and replicated schemes are interchangeable, while the Fortran77 scheme is appreciably
slower.

\subsection{{\tt DIIS} --- the convergence acceleration}

It sets the number iterations in which a DIIS extrapolation is performed to accelerate
the convergence of excitation amplitudes.  The default value is 5, which means in every
five iteration, one DIIS extrapolation is performed (and in the rest of the iterations,
Jacobi rotation is used).  When zero or negative value is specified, the DIIS is turned
off.  It is not recommended to perform DIIS every iteration, whereas setting a large 
value for this parameter necessitates a large memory (disk) space to keep the excitation
amplitudes of previous iterations.

\subsection{{\tt FREEZE} --- the frozen core/virtual approximation}

Some of the lowest-lying core orbitals and/or some of the highest-lying
virtual orbitals may be excluded in the calculations
by this keyword (this does not affect the ground state HF or DFT calculation).
No orbitals are frozen by default.  To exclude the atom-like
core regions altogether, one may request
\begin{verbatim}
  FREEZE atomic
\end{verbatim}
To specify the number of lowest-lying occupied orbitals be excluded, one may use
\begin{verbatim}
  FREEZE 10
\end{verbatim}
which causes 10 lowest-lying occupied orbitals excluded.
This is equivalent to writing
\begin{verbatim}
  FREEZE core 10
\end{verbatim}
To freeze the highest virtual orbitals, use the \verb+virtual+
keyword.  For instance, to freeze the top 5 virtuals
\begin{verbatim}
  FREEZE virtual 5
\end{verbatim}

\subsection{{\tt PRINT} --- the verbosity}

This keyword changes the level of output verbosity.  One may also
request some particular items in Table \ref{tbl:tce-printable} printed.

\begin{table}[htbp]
\begin{center}
\caption{Printable items in the TCE modules and their default print levels.}
\label{tbl:tce-printable}
\begin{tabular}{lll}
\hline\hline
Item                     & Print Level   & Description \\
\hline 
``time''                 & vary          & CPU and wall times \\
``tile''                 & vary          & Orbital range tiling information \\
``t1''                   & debug         & $T_1$ excitation amplitude dumping \\
``t2''                   & debug         & $T_2$ excitation amplitude dumping \\
``t3''                   & debug         & $T_3$ excitation amplitude dumping \\
``t4''                   & debug         & $T_4$ excitation amplitude dumping \\
``general information''  & default       & General information \\
``correlation information''       & default       & TCE information \\
``mbpt2''                & debug         & Caonical HF MBPT2 test \\
``get\_block''            & debug         & I/O information \\
``put\_block''            & debug         & I/O information \\
``add\_block''            & debug         & I/O information \\
``files''                & debug         & File information \\
``offset''               & debug         & File offset information \\
``ao1e''                 & debug         & AO one-electron integral evaluation \\
``ao2e''                 & debug         & AO two-electron integral evaluation \\
``mo1e''                 & debug         & One-electron integral transformation \\
``mo2e''                 & debug         & Two-electron integral transformation \\
\hline\hline
\end{tabular}
\end{center}
\end{table}

\section{Sample input}

The following is a sample input for a ROHF-UCCSD energy calculation of a water radical cation.
\begin{verbatim}
START h2o

TITLE "ROHF-UCCSD/cc-pVTZ H2O"

CHARGE 1

GEOMETRY
O     0.00000000     0.00000000     0.12982363
H     0.75933475     0.00000000    -0.46621158
H    -0.75933475     0.00000000    -0.46621158
END

BASIS
* library cc-pVTZ
END

SCF
ROHF
DOUBLET
THRESH 1.0e-10
TOL2E  1.0e-10
END

TCE
CCSD
END

TASK TCE ENERGY
\end{verbatim}
The same result can be obtained by the following input:
\begin{verbatim}
START h2o

TITLE "ROHF-UCCSD/cc-pVTZ H2O"

CHARGE 1

GEOMETRY
O     0.00000000     0.00000000     0.12982363
H     0.75933475     0.00000000    -0.46621158
H    -0.75933475     0.00000000    -0.46621158
END

BASIS
* library cc-pVTZ
END

SCF
ROHF
DOUBLET
THRESH 1.0e-10
TOL2E  1.0e-10
END

TASK UCCSD ENERGY
\end{verbatim}
