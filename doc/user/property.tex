%
% $Id$
%
\label{sec:property}

Properties can be calculated for both the Hartree-Fock and DFT wave 
functions. The properties that are available are:

\begin{itemize}
\item Natural bond analysis
\item Dipole, quadrupole, and octupole moment 
\item Mulliken population analysis and bond order analysis
\item Electrostatic potential (diamagnetic shielding) at nuclei 
\item Electric field and field gradient at nuclei 
\item Electron and spin density at nuclei 
\item NMR shielding (GIAO method)
\item NMR hyperfine coupling (Fermi-Contact and Spin-Dipole expectation values)
\item NMR indirect spin-spin coupling
\end{itemize}

The properties module is started when the task directive
\verb+TASK <theory> property+ is defined in the user input file. The input 
format has the form:

\begin{verbatim}
  PROPERTY
    [property keyword]
    [CENTER ((com || coc || origin || arb <real x y z>) default coc)]
  END
\end{verbatim}

Most of the properties can be computed for Hartree-Fock
(closed-shell RHF, open-shell ROHF, and open-shell UHF), and DFT
(closed-shell and open-shell spin unrestricted) wavefunctions. The NMR
chemical shift is limited to closed-shell wave functions, whereas the NMR 
hyperfine and indirect spin-spin coupling require a UHF or ODFT wave function.

\section{Property keywords}

Each property can be requested by defining one of the following keywords:

\begin{verbatim}
  NBOFILE
  DIPOLE
  QUADRUPOLE
  OCTUPOLE
  MULLIKEN
  ESP
  EFIELD
  EFIELDGRAD
  ELECTRONDENSITY
  HYPERFINE
  SHIELDING [<integer> number_of_atoms <integer> atom_list]
  SPINSPIN [<integer> number_of_pairs <integer> pair_list]
  ALL
\end{verbatim}

The ``{\tt ALL}'' keyword generates all currently available properties. 

Both the NMR shielding and spin-spin coupling have additional optional
parameters that can be defined in the input. For the shielding the user
can define the number of atoms for which the shielding tensor should be 
calculated, followed by the list of specific atom centers. In the case 
of spin-spin coupling the number of atom pairs, followed by the atom
pairs, can be defined (i.e., spinspin 1 1 2 will calculate the coupling 
for one pair, and the coupling will be between atoms 1 and 2).

For both the NMR spin-spin and hyperfine coupling the isotope that has 
the highest abundance and has spin, will be choosen for each atom under
consideration.

The user also has the option to choose the center of expansion for
the dipole, quadrupole, and octupole calculations.

\begin{verbatim}
    [CENTER ((com || coc || origin || arb <real x y z>) default coc)]
\end{verbatim}

\verb+com+ is the center of mass, \verb+coc+ is the center of charge, \verb+origin+ is 
(0.0, 0.0, 0.0) and \verb+arb+ is any arbitrary point which must be accompanied
by the coordinated to be used.  Currently the x, y, and z coordinates
must be given in the same units as \verb+UNITS+ in \verb+GEOMETRY+ (See Section
\ref{sec:geomkeys}).

\subsection{Nbofile}
\label{sec:Nbofile}

The keyword {\tt NBOFILE} does not execute the Natural Bond Analysis
code, but simply creates an input file to be used as input to the
stand-alone NBO code.  All other properties are calculated upon
request.

Following the successful completion of an electronic structure
calculation, a Natural Bond Orbital (NBO) analysis may be carried out
by providing the keyword \verb+NBOFILE+ in the \verb+PROPERTY+ directive.  
NWChem will query the rtdb and construct an ASCII file,
\verb+<file_prefix>.gen+, that may be used as input to the stand alone
version of the NBO program, gennbo.  \verb+<file_prefix>+ is equal to
string following the \verb+START+ directive.  The input deck may be edited
to provide additional options to the NBO calculation, (see the NBO
user's manual for details.)  

