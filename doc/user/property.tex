\label{sec:property}
\begin{verbatim}
  PROPERTY
    ...
  END
\end{verbatim}

The following properties can be extracted from 
Hartree-Fock ( closed-shell RHF, open-shell ROHF,
and open-shell UHF ) wavefunctions, as well as
DFT ( closed-shell and open-shell spin unrestricted )
wavefunctions.

\begin{itemize}
\item natural bond analysis
\item dipole moment
\item quadrupole moment
\item octupole moment
\item Mulliken population analysis and bond order analysis
\item electrostatic potential (diamagnetic shielding) at nuclei 
\item electric field at nuclei 
\item electric field gradient at nuclei 
\item electron density and electron wavefunction at nuclei 
\item Boy's localized orbitals
\item spin density
\end{itemize}

\subsection{subdirectives}

Calculation of properties is accomplished via a call to
TASK PROPERTY following the complition of an energy calculation.
Each property can be requested by means of a subdirective among
the subdirectives provided :

\begin{itemize}
\item nbofile
\item dipole
\item quadrupole
\item octupole
\item mulliken
\item esp
\item efield
\item efieldgrad
\item electrondensity
\item spindensity
\item boyslocalization
\end{itemize}

The request to NBOFILE does not execute the Natural Bond Analysis
code, but simply creates an input file to be used as input to the
stand-alone NBO code. All other properties are calculated upon
request.

An additional subdirective is provided to specify the origin of the
molecular orbitals used in the calculation of the molecular
properties. This is the 'vectors' subdirective, also used in the
SCF and DFT tasks. For a full description of this subdirective
the user is refered to the description found in the SCF description.
By default, the input file used for the calculation of the properties
has the .movecs name extension. The output file name ought to be
used to store the Boys' localized molecular orbitals.

\subsection{nbofile}

Following the successful completion of an electronic structure
calculation, a Natural Bond Orbital (NBO) analysis may be carried out
in the following way.  On restart specify the TASK as PROPERTY and
supply the sub-directive NBOFILE to the PROPERTY directive.  NWchem
will query the rtdb and construct an ASCII file,
\verb+<file_prefix>.gen+, that may be used as input to the stand alone
version of the NBO program, gennbo.  \verb+<file_prefix>+ is equal to
string following the RESTART directive.  The input deck may be edited
to provide additional options to the NBO calculation, (see the NBO
user's manual for details.)

