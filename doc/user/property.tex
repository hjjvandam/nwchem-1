\label{sec:property}
\begin{verbatim}
  PROPERTY
    ...
  END
\end{verbatim}

The following properties can be extracted from the wavefunction.

\begin{itemize}
\item dipole moment
\item quadrupole moment
\item octupole moment
\item Mulliken population analysis and bond order analysis
\item electrostatic potential (diamagnetic shielding) at nuclei and grid
\item electric field at nuclei and grid
\item electric field gradient at nuclei and grid
\item electron density and electron wavefunction at nuclei and grid
\item Boy's localized orbitals
\item spin density
\item Stone's distributed multipole analysis
\item static dipole polarizabilities
\item natural bond analysis
\end{itemize}

\subsection{nbofile}

Following the successful completion of an electronic structure
calculation, a Natural Bond Orbital (NBO) analysis may be carried out
in the following way.  On restart specify the TASK as PROPERTY and
supply the sub-directive NBOFILE to the PROPERTY directive.  NWchem
will query the rtdb and construct an ASCII file,
\verb+<file_prefix>.gen+, that may be used as input to the stand alone
version of the NBO program, gennbo.  \verb+<file_prefix>+ is equal to
string following the RESTART directive.  The input deck may be edited
to provide additional options to the NBO calculation, (see the NBO
user's manual for details.)

