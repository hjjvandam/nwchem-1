%
% $Id: property.tex,v 1.22 2004-04-22 04:50:28 edo Exp $
%
\label{sec:property}
\begin{verbatim}
  PROPERTY
    [property name]
    [CENTER ((com || coc || origin || arb <real x y z>) default coc)]
    [VECTORS ...]
  END
\end{verbatim}

Calculation of properties is accomplished with \verb+TASK PROPERTY+
after the completion of an energy (or MP2 gradient) calculation.  The
following properties can be computed for all wavefunctions that produce
orbitals, including Hartree-Fock (closed-shell RHF, open-shell ROHF, and
open-shell UHF), DFT (closed-shell and open-shell spin unrestricted),
MCSCF (complete active space), and MP2 (closed-shell RHF and open-shell
UHF).

\begin{itemize}
\item natural bond analysis
\item dipole moment
\item quadrupole moment
\item octupole moment
\item Mulliken population analysis and bond order analysis
\item electrostatic potential (diamagnetic shielding) at nuclei 
\item electric field at nuclei 
\item electric field gradient at nuclei 
\item electron and spin density at nuclei 
\item NMR chemical shifts (GIAO method) only for closed-shell RHF
\end{itemize}

The default molecular orbital file \verb+$file_prefix$.movecs+ is used
unless a vectors directive (Section \ref{sec:vectors}) is provided.  It is
therefore only necessary to include a vectors directive if the MO vectors
to be analyzed are not coming from the default file, e.g., if they have
been previously redirected, or if MP2 natural orbitals (file extension
\verb+".mp2nos"+) are being anaylzed.  The MP2 natural orbitals MUST be
used if the user wants MP2 properties.

\section{Subdirectives}

Note that presenting any property input causes all previous property input
to be ``forgotten'', unlike other NWChem modules.

Each property can be requested by means of a subdirective among the
subdirectives provided :

\begin{itemize}
\item {\tt NBOFILE}
\item {\tt DIPOLE}
\item {\tt QUADRUPOLE}
\item {\tt OCTUPOLE}
\item {\tt MULLIKEN}
\item {\tt ESP}
\item {\tt EFIELD}
\item {\tt EFIELDGRAD}
\item {\tt ELECTRONDENSITY}
\item {\tt GIAO}
\item {\tt ALL}
\end{itemize}

The ``{\tt ALL}'' keyword generates all currently available properties. 

The request {\tt NBOFILE} does not execute the Natural Bond Analysis
code, but simply creates an input file to be used as input to the
stand-alone NBO code. To execute the NBO analysis directly, see Section
\ref{sec:nbo}. All other properties are calculated upon
request.

An additional subdirective is provided to specify the origin of the
molecular orbitals used in the calculation of the molecular
properties. This is the `{\tt VECTORS}' subdirective, also used in the
SCF and DFT tasks. For a full description of this subdirective
the user is refered to the description found in the SCF description.
By default, the input file used for the calculation of the properties
has the \verb+.movecs+ name extension. 

The user also has the option to choose the center of expansion for
the dipole, quadrupole, and octupole calculations.

\begin{verbatim}
    [CENTER ((com || coc || origin || arb <real x y z>) default coc)]
\end{verbatim}

\verb+com+ is the center of mass, \verb+coc+ is the center of charge, \verb+origin+ is 
(0.0, 0.0, 0.0) and \verb+arb+ is any arbitrary point which must be accompanied
by the coordinated to be used.  Currently the x, y, and z coordinates
must be given in the same units as \verb+UNITS+ in \verb+GEOMETRY+ (See Section
\ref{sec:geomkeys}).

\subsection{Nbofile}
\label{sec:Nbofile}

Following the successful completion of an electronic structure
calculation, a Natural Bond Orbital (NBO) analysis may be carried out
in the following way.  On restart specify the \verb+TASK+ as \verb+PROPERTY+ and
supply the sub-directive \verb+NBOFILE+ to the \verb+PROPERTY+ directive.  NWChem
will query the rtdb and construct an ASCII file,
\verb+<file_prefix>.gen+, that may be used as input to the stand alone
version of the NBO program, gennbo.  \verb+<file_prefix>+ is equal to
string following the \verb+RESTART+ directive.  The input deck may be edited
to provide additional options to the NBO calculation, (see the NBO
user's manual for details.)  The other option in to directly run the NBO 
analysis (See Section \ref{sec:nbo} for more information).

