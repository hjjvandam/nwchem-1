%
% $Id: fourindex.tex,v 1.6 2004-04-22 04:50:28 edo Exp $
%
\label{sec:fourindex}

{\bf THIS SECTION IS BEING REWRITTEN}

%{\bf ????????????????????}
%\Large
%**What does this module really do?  Is it invoked by any module other than
%DIRECT\_MP2?
%\normalsize
%
%%  The four-index transformation module is not designed for explicit
%% invocation but rather as a utility module for other tasks (e.g. direct
%% MP2 section \ref{sec:mp2}). However, there are input parameters which
%% may be set to modify the default behavior of the four-index
%% transformation module which may enhance performance.
%
%The four-index transformation module is a utility invoked by the fully
%direct MP2 module (see Section \ref{sec:mp2}).  For most applications,
%the default parameters for the four-index transformation will be adequate,
%but in some instances, the user may wish to enhance the performance of
%the module.  This can be done by specifying \verb+SET+ directives for
%selected parameters.  The following sections describe the options
%currently available for this module.
%
%\section{Algorithm for the Four-Index Transformation}
%
%The code currently contains two algorithms for the four-index transformation.
%The algorithm to use for a given calculation can be selected by specifying
%a \verb+SET+ directive of the form,
%
%\begin{verbatim}
% set "fourindex:method" <string method default twofold>
%\end{verbatim}
%
%% There is a choice of two algorithms for effecting the transformation.
%The default entry for the string \verb+method+ is \verb+twofold+, which
%specifies ***what?***.  This option should be used in almost all instances.
%However, in cases of high orders of parallelism (i.e., more than 200 nodes),
% better throughput may be obtained with the alternative algorithm, 
%obtained by setting \verb+method+ to \verb+sixfold+.
%
%\section{AO Integral Blocking and Block Length for the Four-Index Transformation}
%
%The default in the four-index transformation is to have no blocking in
%the AO integral generation.  Blocking can be specified explicitly using a
%\verb+SET+ directive of the form,
%
%\begin{verbatim}
% set "fourindex:aoblock" <logical aoblock default .false.>
%\end{verbatim}
%
%Entering a value of \verb+.true.+ for the logical variable \verb+aoblock+ 
%enables blocking in the AO integral generation. This may result in 
%substantial performance enhancement in cases with many identical 
%atoms and basis sets.
%
%The default is 10 for the vector length in the critical first index 
%transformation of the four-index transformation algorithm.  The user has
%the option of specifying the vector length explicitly by entering a 
%\verb+SET+ directive of the form,
%
%\begin{verbatim}
% set "fourindex:blocklength" <integer blocklength default 10>
%\end{verbatim}
%
%% This parameter determines the vector length in the critical first
%% index transformation. 
%
%For larger basis sets, it may be profitable to increase the vector
%length by specifying a value larger than 10 for the integer variable
%\verb+blocklength+.  However, this value should {\em
%  not} exceed $N_{bf} / 3$, since the cost of the redundant computation 
%in the square matrix multiplication would in that case exceed the 
%gain from a longer vector length.
%
%\Large
%***What is $N_{bf} / 3$?
%\normalsize
%
%\section{Frozen orbitals for the Four-Index Transformation}
%
%In most cases, the frozen orbitals will be specified by the input for the 
%MP2 calculation (or calculation in some other module???) that invokes the
%four-index transformation module.  However, the user can specify frozen
%orbitals separately for the four-index transformation, using \verb+SET+
%directives of the form,
%
%\begin{verbatim}
% set "fourindex:occ_frozen" \
%      <integer frozen_occupied default 0>
% set "fourindex:vir_frozen" \
%      <integer frozen_virtual default 0>
%\end{verbatim}
%
%The value specified for the integer variable \verb+frozen_occupied+ 
%designates the lowest frozen occupied orbital.  The value specified
%for the integer variable \verb+frozen_virtual+ designates the highest
%virtual frozen orbital.  The default is zero for both of these variables,
%which specifies no additional frozen orbitals for the four-index
%transformation module.
%
%% Setting these parameters
%% should not be required in the usual circumstances since these are
%% determined by the calling modules, (see section \ref{sec:mp2}).
%
%
%
