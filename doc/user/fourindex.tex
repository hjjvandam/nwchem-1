\label{sec:fourindex}

{\bf ????????????????????}

The four-index transformation module is not designed for explicit
invocation but rather as a utility module for other tasks (e.g. direct
MP2 section \ref{sec:mp2}). However, there are input parameters which
may be set to modify the default behavior of the four-index
transformation module which may enhance performance.

\subsection{Algorithm}
\begin{verbatim}
 set "fourindex:method" <string method default twofold>
\end{verbatim}
There is a choice of two algorithms for effecting the transformation.
The default, \verb+"twofold"+, should be used in almost all instances.
However, in cases of high orders of parallelism ($>$ 200 nodes) better
throughput maybe obtained with the alternative algorithm, \verb+"sixfold"+.

\subsection{AO Integral Blocking and Block Length}
\begin{verbatim}
 set "fourindex:aoblock" <logical aoblock default .false.>
\end{verbatim}
Setting the \verb+aoblock+ to true enables blocking in the AO
integral generation. This may result to substantial performance
enhancement in cases with many identical atoms and basis sets.

\begin{verbatim}
 set "fourindex:blocklength" <integer blocklength default 10>
\end{verbatim}
This parameter determines the vector length in the critical first
index transformation. For larger basis sets, it may be profitable to
increase this parameter however the \verb+blocklength+ should {\em
  not} exceed $N_{bf} / 3$ since the cost of the redundant computation in the
square matrix multiplication will exceed the gain from a longer vector length.

\subsection{Frozen orbitals}
\begin{verbatim}
 set "fourindex:occ_frozen" \
      <integer frozen_occupied default 0>
 set "fourindex:vir_frozen" \
      <integer frozen_virtual default 0
\end{verbatim}
Designates the lowest \verb+frozen_occupied+ and highest
\verb+frozen_virtual+ orbitals as frozen. Setting these parameters
should not be required in the usual circumstances since these are
determined by the calling modules, (see section \ref{sec:mp2}).
