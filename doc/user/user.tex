% $Id: user.tex,v 1.37 1999-12-15 22:08:20 bert Exp $

\documentstyle[fullpage,12pt,fleqn]{book}
\setlength{\parskip}{6pt}

% Set the version and year of release globally
\newcommand{\nwchemversion}{3.3.1}
\newcommand{\nwchemyear}{1998}

\newcommand{\angstroms}{Angstr{\o}ms}

% copied from dvips-5.515, Apr 5 1993, Rik Littlefield
% Psfig/TeX 
\def\PsfigVersion{1.9}
% dvips version
%
% All psfig/tex software, documentation, and related files
% in this distribution of psfig/tex are 
% Copyright 1987, 1988, 1991 Trevor J. Darrell
%
% Permission is granted for use and non-profit distribution of psfig/tex 
% providing that this notice is clearly maintained. The right to
% distribute any portion of psfig/tex for profit or as part of any commercial
% product is specifically reserved for the author(s) of that portion.
%
% *** Feel free to make local modifications of psfig as you wish,
% *** but DO NOT post any changed or modified versions of ``psfig''
% *** directly to the net. Send them to me and I'll try to incorporate
% *** them into future versions. If you want to take the psfig code 
% *** and make a new program (subject to the copyright above), distribute it, 
% *** (and maintain it) that's fine, just don't call it psfig.
%
% Bugs and improvements to trevor@media.mit.edu.
%
% Thanks to Greg Hager (GDH) and Ned Batchelder for their contributions
% to the original version of this project.
%
% Modified by J. Daniel Smith on 9 October 1990 to accept the
% %%BoundingBox: comment with or without a space after the colon.  Stole
% file reading code from Tom Rokicki's EPSF.TEX file (see below).
%
% More modifications by J. Daniel Smith on 29 March 1991 to allow the
% the included PostScript figure to be rotated.  The amount of
% rotation is specified by the "angle=" parameter of the \psfig command.
%
% Modified by Robert Russell on June 25, 1991 to allow users to specify
% .ps filenames which don't yet exist, provided they explicitly provide
% boundingbox information via the \psfig command. Note: This will only work
% if the "file=" parameter follows all four "bb???=" parameters in the
% command. This is due to the order in which psfig interprets these params.
%
%  3 Jul 1991	JDS	check if file already read in once
%  4 Sep 1991	JDS	fixed incorrect computation of rotated
%			bounding box
% 25 Sep 1991	GVR	expanded synopsis of \psfig
% 14 Oct 1991	JDS	\fbox code from LaTeX so \psdraft works with TeX
%			changed \typeout to \ps@typeout
% 17 Oct 1991	JDS	added \psscalefirst and \psrotatefirst
%

% From: gvr@cs.brown.edu (George V. Reilly)
%
% \psdraft	draws an outline box, but doesn't include the figure
%		in the DVI file.  Useful for previewing.
%
% \psfull	includes the figure in the DVI file (default).
%
% \psscalefirst width= or height= specifies the size of the figure
% 		before rotation.
% \psrotatefirst (default) width= or height= specifies the size of the
% 		 figure after rotation.  Asymetric figures will
% 		 appear to shrink.
%
% \psfigurepath#1	sets the path to search for the figure
%
% \psfig
% usage: \psfig{file=, figure=, height=, width=,
%			bbllx=, bblly=, bburx=, bbury=,
%			rheight=, rwidth=, clip=, angle=, silent=}
%
%	"file" is the filename.  If no path name is specified and the
%		file is not found in the current directory,
%		it will be looked for in directory \psfigurepath.
%	"figure" is a synonym for "file".
%	By default, the width and height of the figure are taken from
%		the BoundingBox of the figure.
%	If "width" is specified, the figure is scaled so that it has
%		the specified width.  Its height changes proportionately.
%	If "height" is specified, the figure is scaled so that it has
%		the specified height.  Its width changes proportionately.
%	If both "width" and "height" are specified, the figure is scaled
%		anamorphically.
%	"bbllx", "bblly", "bburx", and "bbury" control the PostScript
%		BoundingBox.  If these four values are specified
%               *before* the "file" option, the PSFIG will not try to
%               open the PostScript file.
%	"rheight" and "rwidth" are the reserved height and width
%		of the figure, i.e., how big TeX actually thinks
%		the figure is.  They default to "width" and "height".
%	The "clip" option ensures that no portion of the figure will
%		appear outside its BoundingBox.  "clip=" is a switch and
%		takes no value, but the `=' must be present.
%	The "angle" option specifies the angle of rotation (degrees, ccw).
%	The "silent" option makes \psfig work silently.
%

% check to see if macros already loaded in (maybe some other file says
% "\input psfig") ...
\ifx\undefined\psfig\else\endinput\fi

%
% from a suggestion by eijkhout@csrd.uiuc.edu to allow
% loading as a style file. Changed to avoid problems
% with amstex per suggestion by jbence@math.ucla.edu

\let\LaTeXAtSign=\@
\let\@=\relax
\edef\psfigRestoreAt{\catcode`\@=\number\catcode`@\relax}
%\edef\psfigRestoreAt{\catcode`@=\number\catcode`@\relax}
\catcode`\@=11\relax
\newwrite\@unused
\def\ps@typeout#1{{\let\protect\string\immediate\write\@unused{#1}}}
\ps@typeout{psfig/tex \PsfigVersion}

%% Here's how you define your figure path.  Should be set up with null
%% default and a user useable definition.

\def\figurepath{./}
\def\psfigurepath#1{\edef\figurepath{#1}}

%
% @psdo control structure -- similar to Latex @for.
% I redefined these with different names so that psfig can
% be used with TeX as well as LaTeX, and so that it will not 
% be vunerable to future changes in LaTeX's internal
% control structure,
%
\def\@nnil{\@nil}
\def\@empty{}
\def\@psdonoop#1\@@#2#3{}
\def\@psdo#1:=#2\do#3{\edef\@psdotmp{#2}\ifx\@psdotmp\@empty \else
    \expandafter\@psdoloop#2,\@nil,\@nil\@@#1{#3}\fi}
\def\@psdoloop#1,#2,#3\@@#4#5{\def#4{#1}\ifx #4\@nnil \else
       #5\def#4{#2}\ifx #4\@nnil \else#5\@ipsdoloop #3\@@#4{#5}\fi\fi}
\def\@ipsdoloop#1,#2\@@#3#4{\def#3{#1}\ifx #3\@nnil 
       \let\@nextwhile=\@psdonoop \else
      #4\relax\let\@nextwhile=\@ipsdoloop\fi\@nextwhile#2\@@#3{#4}}
\def\@tpsdo#1:=#2\do#3{\xdef\@psdotmp{#2}\ifx\@psdotmp\@empty \else
    \@tpsdoloop#2\@nil\@nil\@@#1{#3}\fi}
\def\@tpsdoloop#1#2\@@#3#4{\def#3{#1}\ifx #3\@nnil 
       \let\@nextwhile=\@psdonoop \else
      #4\relax\let\@nextwhile=\@tpsdoloop\fi\@nextwhile#2\@@#3{#4}}
% 
% \fbox is defined in latex.tex; so if \fbox is undefined, assume that
% we are not in LaTeX.
% Perhaps this could be done better???
\ifx\undefined\fbox
% \fbox code from modified slightly from LaTeX
\newdimen\fboxrule
\newdimen\fboxsep
\newdimen\ps@tempdima
\newbox\ps@tempboxa
\fboxsep = 3pt
\fboxrule = .4pt
\long\def\fbox#1{\leavevmode\setbox\ps@tempboxa\hbox{#1}\ps@tempdima\fboxrule
    \advance\ps@tempdima \fboxsep \advance\ps@tempdima \dp\ps@tempboxa
   \hbox{\lower \ps@tempdima\hbox
  {\vbox{\hrule height \fboxrule
          \hbox{\vrule width \fboxrule \hskip\fboxsep
          \vbox{\vskip\fboxsep \box\ps@tempboxa\vskip\fboxsep}\hskip 
                 \fboxsep\vrule width \fboxrule}
                 \hrule height \fboxrule}}}}
\fi
%
%%%%%%%%%%%%%%%%%%%%%%%%%%%%%%%%%%%%%%%%%%%%%%%%%%%%%%%%%%%%%%%%%%%
% file reading stuff from epsf.tex
%   EPSF.TEX macro file:
%   Written by Tomas Rokicki of Radical Eye Software, 29 Mar 1989.
%   Revised by Don Knuth, 3 Jan 1990.
%   Revised by Tomas Rokicki to accept bounding boxes with no
%      space after the colon, 18 Jul 1990.
%   Portions modified/removed for use in PSFIG package by
%      J. Daniel Smith, 9 October 1990.
%
\newread\ps@stream
\newif\ifnot@eof       % continue looking for the bounding box?
\newif\if@noisy        % report what you're making?
\newif\if@atend        % %%BoundingBox: has (at end) specification
\newif\if@psfile       % does this look like a PostScript file?
%
% PostScript files should start with `%!'
%
{\catcode`\%=12\global\gdef\epsf@start{%!}}
\def\epsf@PS{PS}
%
\def\epsf@getbb#1{%
%
%   The first thing we need to do is to open the
%   PostScript file, if possible.
%
\openin\ps@stream=#1
\ifeof\ps@stream\ps@typeout{Error, File #1 not found}\else
%
%   Okay, we got it. Now we'll scan lines until we find one that doesn't
%   start with %. We're looking for the bounding box comment.
%
   {\not@eoftrue \chardef\other=12
    \def\do##1{\catcode`##1=\other}\dospecials \catcode`\ =10
    \loop
       \if@psfile
	  \read\ps@stream to \epsf@fileline
       \else{
	  \obeyspaces
          \read\ps@stream to \epsf@tmp\global\let\epsf@fileline\epsf@tmp}
       \fi
       \ifeof\ps@stream\not@eoffalse\else
%
%   Check the first line for `%!'.  Issue a warning message if its not
%   there, since the file might not be a PostScript file.
%
       \if@psfile\else
       \expandafter\epsf@test\epsf@fileline:. \\%
       \fi
%
%   We check to see if the first character is a % sign;
%   if so, we look further and stop only if the line begins with
%   `%%BoundingBox:' and the `(atend)' specification was not found.
%   That is, the only way to stop is when the end of file is reached,
%   or a `%%BoundingBox: llx lly urx ury' line is found.
%
          \expandafter\epsf@aux\epsf@fileline:. \\%
       \fi
   \ifnot@eof\repeat
   }\closein\ps@stream\fi}%
%
% This tests if the file we are reading looks like a PostScript file.
%
\long\def\epsf@test#1#2#3:#4\\{\def\epsf@testit{#1#2}
			\ifx\epsf@testit\epsf@start\else
\ps@typeout{Warning! File does not start with `\epsf@start'.  It may not be a PostScript file.}
			\fi
			\@psfiletrue} % don't test after 1st line
%
%   We still need to define the tricky \epsf@aux macro. This requires
%   a couple of magic constants for comparison purposes.
%
{\catcode`\%=12\global\let\epsf@percent=%\global\def\epsf@bblit{%BoundingBox}}
%
%
%   So we're ready to check for `%BoundingBox:' and to grab the
%   values if they are found.  We continue searching if `(at end)'
%   was found after the `%BoundingBox:'.
%
\long\def\epsf@aux#1#2:#3\\{\ifx#1\epsf@percent
   \def\epsf@testit{#2}\ifx\epsf@testit\epsf@bblit
	\@atendfalse
        \epsf@atend #3 . \\%
	\if@atend	
	   \if@verbose{
		\ps@typeout{psfig: found `(atend)'; continuing search}
	   }\fi
        \else
        \epsf@grab #3 . . . \\%
        \not@eoffalse
        \global\no@bbfalse
        \fi
   \fi\fi}%
%
%   Here we grab the values and stuff them in the appropriate definitions.
%
\def\epsf@grab #1 #2 #3 #4 #5\\{%
   \global\def\epsf@llx{#1}\ifx\epsf@llx\empty
      \epsf@grab #2 #3 #4 #5 .\\\else
   \global\def\epsf@lly{#2}%
   \global\def\epsf@urx{#3}\global\def\epsf@ury{#4}\fi}%
%
% Determine if the stuff following the %%BoundingBox is `(atend)'
% J. Daniel Smith.  Copied from \epsf@grab above.
%
\def\epsf@atendlit{(atend)} 
\def\epsf@atend #1 #2 #3\\{%
   \def\epsf@tmp{#1}\ifx\epsf@tmp\empty
      \epsf@atend #2 #3 .\\\else
   \ifx\epsf@tmp\epsf@atendlit\@atendtrue\fi\fi}


% End of file reading stuff from epsf.tex
%%%%%%%%%%%%%%%%%%%%%%%%%%%%%%%%%%%%%%%%%%%%%%%%%%%%%%%%%%%%%%%%%%%

%%%%%%%%%%%%%%%%%%%%%%%%%%%%%%%%%%%%%%%%%%%%%%%%%%%%%%%%%%%%%%%%%%%
% trigonometry stuff from "trig.tex"
\chardef\psletter = 11 % won't conflict with \begin{letter} now...
\chardef\other = 12

\newif \ifdebug %%% turn me on to see TeX hard at work ...
\newif\ifc@mpute %%% don't need to compute some values
\c@mputetrue % but assume that we do

\let\then = \relax
\def\r@dian{pt }
\let\r@dians = \r@dian
\let\dimensionless@nit = \r@dian
\let\dimensionless@nits = \dimensionless@nit
\def\internal@nit{sp }
\let\internal@nits = \internal@nit
\newif\ifstillc@nverging
\def \Mess@ge #1{\ifdebug \then \message {#1} \fi}

{ %%% Things that need abnormal catcodes %%%
	\catcode `\@ = \psletter
	\gdef \nodimen {\expandafter \n@dimen \the \dimen}
	\gdef \term #1 #2 #3%
	       {\edef \t@ {\the #1}%%% freeze parameter 1 (count, by value)
		\edef \t@@ {\expandafter \n@dimen \the #2\r@dian}%
				   %%% freeze parameter 2 (dimen, by value)
		\t@rm {\t@} {\t@@} {#3}%
	       }
	\gdef \t@rm #1 #2 #3%
	       {{%
		\count 0 = 0
		\dimen 0 = 1 \dimensionless@nit
		\dimen 2 = #2\relax
		\Mess@ge {Calculating term #1 of \nodimen 2}%
		\loop
		\ifnum	\count 0 < #1
		\then	\advance \count 0 by 1
			\Mess@ge {Iteration \the \count 0 \space}%
			\Multiply \dimen 0 by {\dimen 2}%
			\Mess@ge {After multiplication, term = \nodimen 0}%
			\Divide \dimen 0 by {\count 0}%
			\Mess@ge {After division, term = \nodimen 0}%
		\repeat
		\Mess@ge {Final value for term #1 of 
				\nodimen 2 \space is \nodimen 0}%
		\xdef \Term {#3 = \nodimen 0 \r@dians}%
		\aftergroup \Term
	       }}
	\catcode `\p = \other
	\catcode `\t = \other
	\gdef \n@dimen #1pt{#1} %%% throw away the ``pt''
}

\def \Divide #1by #2{\divide #1 by #2} %%% just a synonym

\def \Multiply #1by #2%%% allows division of a dimen by a dimen
       {{%%% should really freeze parameter 2 (dimen, passed by value)
	\count 0 = #1\relax
	\count 2 = #2\relax
	\count 4 = 65536
	\Mess@ge {Before scaling, count 0 = \the \count 0 \space and
			count 2 = \the \count 2}%
	\ifnum	\count 0 > 32767 %%% do our best to avoid overflow
	\then	\divide \count 0 by 4
		\divide \count 4 by 4
	\else	\ifnum	\count 0 < -32767
		\then	\divide \count 0 by 4
			\divide \count 4 by 4
		\else
		\fi
	\fi
	\ifnum	\count 2 > 32767 %%% while retaining reasonable accuracy
	\then	\divide \count 2 by 4
		\divide \count 4 by 4
	\else	\ifnum	\count 2 < -32767
		\then	\divide \count 2 by 4
			\divide \count 4 by 4
		\else
		\fi
	\fi
	\multiply \count 0 by \count 2
	\divide \count 0 by \count 4
	\xdef \product {#1 = \the \count 0 \internal@nits}%
	\aftergroup \product
       }}

\def\r@duce{\ifdim\dimen0 > 90\r@dian \then   % sin(x+90) = sin(180-x)
		\multiply\dimen0 by -1
		\advance\dimen0 by 180\r@dian
		\r@duce
	    \else \ifdim\dimen0 < -90\r@dian \then  % sin(-x) = sin(360+x)
		\advance\dimen0 by 360\r@dian
		\r@duce
		\fi
	    \fi}

\def\Sine#1%
       {{%
	\dimen 0 = #1 \r@dian
	\r@duce
	\ifdim\dimen0 = -90\r@dian \then
	   \dimen4 = -1\r@dian
	   \c@mputefalse
	\fi
	\ifdim\dimen0 = 90\r@dian \then
	   \dimen4 = 1\r@dian
	   \c@mputefalse
	\fi
	\ifdim\dimen0 = 0\r@dian \then
	   \dimen4 = 0\r@dian
	   \c@mputefalse
	\fi
%
	\ifc@mpute \then
        	% convert degrees to radians
		\divide\dimen0 by 180
		\dimen0=3.141592654\dimen0
%
		\dimen 2 = 3.1415926535897963\r@dian %%% a well-known constant
		\divide\dimen 2 by 2 %%% we only deal with -pi/2 : pi/2
		\Mess@ge {Sin: calculating Sin of \nodimen 0}%
		\count 0 = 1 %%% see power-series expansion for sine
		\dimen 2 = 1 \r@dian %%% ditto
		\dimen 4 = 0 \r@dian %%% ditto
		\loop
			\ifnum	\dimen 2 = 0 %%% then we've done
			\then	\stillc@nvergingfalse 
			\else	\stillc@nvergingtrue
			\fi
			\ifstillc@nverging %%% then calculate next term
			\then	\term {\count 0} {\dimen 0} {\dimen 2}%
				\advance \count 0 by 2
				\count 2 = \count 0
				\divide \count 2 by 2
				\ifodd	\count 2 %%% signs alternate
				\then	\advance \dimen 4 by \dimen 2
				\else	\advance \dimen 4 by -\dimen 2
				\fi
		\repeat
	\fi		
			\xdef \sine {\nodimen 4}%
       }}

% Now the Cosine can be calculated easily by calling \Sine
\def\Cosine#1{\ifx\sine\UnDefined\edef\Savesine{\relax}\else
		             \edef\Savesine{\sine}\fi
	{\dimen0=#1\r@dian\advance\dimen0 by 90\r@dian
	 \Sine{\nodimen 0}
	 \xdef\cosine{\sine}
	 \xdef\sine{\Savesine}}}	      
% end of trig stuff
%%%%%%%%%%%%%%%%%%%%%%%%%%%%%%%%%%%%%%%%%%%%%%%%%%%%%%%%%%%%%%%%%%%%

\def\psdraft{
	\def\@psdraft{0}
	%\ps@typeout{draft level now is \@psdraft \space . }
}
\def\psfull{
	\def\@psdraft{100}
	%\ps@typeout{draft level now is \@psdraft \space . }
}

\psfull

\newif\if@scalefirst
\def\psscalefirst{\@scalefirsttrue}
\def\psrotatefirst{\@scalefirstfalse}
\psrotatefirst

\newif\if@draftbox
\def\psnodraftbox{
	\@draftboxfalse
}
\def\psdraftbox{
	\@draftboxtrue
}
\@draftboxtrue

\newif\if@prologfile
\newif\if@postlogfile
\def\pssilent{
	\@noisyfalse
}
\def\psnoisy{
	\@noisytrue
}
\psnoisy
%%% These are for the option list.
%%% A specification of the form a = b maps to calling \@p@@sa{b}
\newif\if@bbllx
\newif\if@bblly
\newif\if@bburx
\newif\if@bbury
\newif\if@height
\newif\if@width
\newif\if@rheight
\newif\if@rwidth
\newif\if@angle
\newif\if@clip
\newif\if@verbose
\def\@p@@sclip#1{\@cliptrue}


\newif\if@decmpr

%%% GDH 7/26/87 -- changed so that it first looks in the local directory,
%%% then in a specified global directory for the ps file.
%%% RPR 6/25/91 -- changed so that it defaults to user-supplied name if
%%% boundingbox info is specified, assuming graphic will be created by
%%% print time.
%%% TJD 10/19/91 -- added bbfile vs. file distinction, and @decmpr flag

\def\@p@@sfigure#1{\def\@p@sfile{null}\def\@p@sbbfile{null}
	        \openin1=#1.bb
		\ifeof1\closein1
	        	\openin1=\figurepath#1.bb
			\ifeof1\closein1
			        \openin1=#1
				\ifeof1\closein1%
				       \openin1=\figurepath#1
					\ifeof1
					   \ps@typeout{Error, File #1 not found}
						\if@bbllx\if@bblly
				   		\if@bburx\if@bbury
			      				\def\@p@sfile{#1}%
			      				\def\@p@sbbfile{#1}%
							\@decmprfalse
				  	   	\fi\fi\fi\fi
					\else\closein1
				    		\def\@p@sfile{\figurepath#1}%
				    		\def\@p@sbbfile{\figurepath#1}%
						\@decmprfalse
	                       		\fi%
			 	\else\closein1%
					\def\@p@sfile{#1}
					\def\@p@sbbfile{#1}
					\@decmprfalse
			 	\fi
			\else
				\def\@p@sfile{\figurepath#1}
				\def\@p@sbbfile{\figurepath#1.bb}
				\@decmprtrue
			\fi
		\else
			\def\@p@sfile{#1}
			\def\@p@sbbfile{#1.bb}
			\@decmprtrue
		\fi}

\def\@p@@sfile#1{\@p@@sfigure{#1}}

\def\@p@@sbbllx#1{
		%\ps@typeout{bbllx is #1}
		\@bbllxtrue
		\dimen100=#1
		\edef\@p@sbbllx{\number\dimen100}
}
\def\@p@@sbblly#1{
		%\ps@typeout{bblly is #1}
		\@bbllytrue
		\dimen100=#1
		\edef\@p@sbblly{\number\dimen100}
}
\def\@p@@sbburx#1{
		%\ps@typeout{bburx is #1}
		\@bburxtrue
		\dimen100=#1
		\edef\@p@sbburx{\number\dimen100}
}
\def\@p@@sbbury#1{
		%\ps@typeout{bbury is #1}
		\@bburytrue
		\dimen100=#1
		\edef\@p@sbbury{\number\dimen100}
}
\def\@p@@sheight#1{
		\@heighttrue
		\dimen100=#1
   		\edef\@p@sheight{\number\dimen100}
		%\ps@typeout{Height is \@p@sheight}
}
\def\@p@@swidth#1{
		%\ps@typeout{Width is #1}
		\@widthtrue
		\dimen100=#1
		\edef\@p@swidth{\number\dimen100}
}
\def\@p@@srheight#1{
		%\ps@typeout{Reserved height is #1}
		\@rheighttrue
		\dimen100=#1
		\edef\@p@srheight{\number\dimen100}
}
\def\@p@@srwidth#1{
		%\ps@typeout{Reserved width is #1}
		\@rwidthtrue
		\dimen100=#1
		\edef\@p@srwidth{\number\dimen100}
}
\def\@p@@sangle#1{
		%\ps@typeout{Rotation is #1}
		\@angletrue
%		\dimen100=#1
		\edef\@p@sangle{#1} %\number\dimen100}
}
\def\@p@@ssilent#1{ 
		\@verbosefalse
}
\def\@p@@sprolog#1{\@prologfiletrue\def\@prologfileval{#1}}
\def\@p@@spostlog#1{\@postlogfiletrue\def\@postlogfileval{#1}}
\def\@cs@name#1{\csname #1\endcsname}
\def\@setparms#1=#2,{\@cs@name{@p@@s#1}{#2}}
%
% initialize the defaults (size the size of the figure)
%
\def\ps@init@parms{
		\@bbllxfalse \@bbllyfalse
		\@bburxfalse \@bburyfalse
		\@heightfalse \@widthfalse
		\@rheightfalse \@rwidthfalse
		\def\@p@sbbllx{}\def\@p@sbblly{}
		\def\@p@sbburx{}\def\@p@sbbury{}
		\def\@p@sheight{}\def\@p@swidth{}
		\def\@p@srheight{}\def\@p@srwidth{}
		\def\@p@sangle{0}
		\def\@p@sfile{} \def\@p@sbbfile{}
		\def\@p@scost{10}
		\def\@sc{}
		\@prologfilefalse
		\@postlogfilefalse
		\@clipfalse
		\if@noisy
			\@verbosetrue
		\else
			\@verbosefalse
		\fi
}
%
% Go through the options setting things up.
%
\def\parse@ps@parms#1{
	 	\@psdo\@psfiga:=#1\do
		   {\expandafter\@setparms\@psfiga,}}
%
% Compute bb height and width
%
\newif\ifno@bb
\def\bb@missing{
	\if@verbose{
		\ps@typeout{psfig: searching \@p@sbbfile \space  for bounding box}
	}\fi
	\no@bbtrue
	\epsf@getbb{\@p@sbbfile}
        \ifno@bb \else \bb@cull\epsf@llx\epsf@lly\epsf@urx\epsf@ury\fi
}	
\def\bb@cull#1#2#3#4{
	\dimen100=#1 bp\edef\@p@sbbllx{\number\dimen100}
	\dimen100=#2 bp\edef\@p@sbblly{\number\dimen100}
	\dimen100=#3 bp\edef\@p@sbburx{\number\dimen100}
	\dimen100=#4 bp\edef\@p@sbbury{\number\dimen100}
	\no@bbfalse
}
% rotate point (#1,#2) about (0,0).
% The sine and cosine of the angle are already stored in \sine and
% \cosine.  The result is placed in (\p@intvaluex, \p@intvaluey).
\newdimen\p@intvaluex
\newdimen\p@intvaluey
\def\rotate@#1#2{{\dimen0=#1 sp\dimen1=#2 sp
%            	calculate x' = x \cos\theta - y \sin\theta
		  \global\p@intvaluex=\cosine\dimen0
		  \dimen3=\sine\dimen1
		  \global\advance\p@intvaluex by -\dimen3
% 		calculate y' = x \sin\theta + y \cos\theta
		  \global\p@intvaluey=\sine\dimen0
		  \dimen3=\cosine\dimen1
		  \global\advance\p@intvaluey by \dimen3
		  }}
\def\compute@bb{
		\no@bbfalse
		\if@bbllx \else \no@bbtrue \fi
		\if@bblly \else \no@bbtrue \fi
		\if@bburx \else \no@bbtrue \fi
		\if@bbury \else \no@bbtrue \fi
		\ifno@bb \bb@missing \fi
		\ifno@bb \ps@typeout{FATAL ERROR: no bb supplied or found}
			\no-bb-error
		\fi
		%
%\ps@typeout{BB: \@p@sbbllx, \@p@sbblly, \@p@sbburx, \@p@sbbury} 
%
% store height/width of original (unrotated) bounding box
		\count203=\@p@sbburx
		\count204=\@p@sbbury
		\advance\count203 by -\@p@sbbllx
		\advance\count204 by -\@p@sbblly
		\edef\ps@bbw{\number\count203}
		\edef\ps@bbh{\number\count204}
		%\ps@typeout{ psbbh = \ps@bbh, psbbw = \ps@bbw }
		\if@angle 
			\Sine{\@p@sangle}\Cosine{\@p@sangle}
	        	{\dimen100=\maxdimen\xdef\r@p@sbbllx{\number\dimen100}
					    \xdef\r@p@sbblly{\number\dimen100}
			                    \xdef\r@p@sbburx{-\number\dimen100}
					    \xdef\r@p@sbbury{-\number\dimen100}}
%
% Need to rotate all four points and take the X-Y extremes of the new
% points as the new bounding box.
                        \def\minmaxtest{
			   \ifnum\number\p@intvaluex<\r@p@sbbllx
			      \xdef\r@p@sbbllx{\number\p@intvaluex}\fi
			   \ifnum\number\p@intvaluex>\r@p@sbburx
			      \xdef\r@p@sbburx{\number\p@intvaluex}\fi
			   \ifnum\number\p@intvaluey<\r@p@sbblly
			      \xdef\r@p@sbblly{\number\p@intvaluey}\fi
			   \ifnum\number\p@intvaluey>\r@p@sbbury
			      \xdef\r@p@sbbury{\number\p@intvaluey}\fi
			   }
%			lower left
			\rotate@{\@p@sbbllx}{\@p@sbblly}
			\minmaxtest
%			upper left
			\rotate@{\@p@sbbllx}{\@p@sbbury}
			\minmaxtest
%			lower right
			\rotate@{\@p@sbburx}{\@p@sbblly}
			\minmaxtest
%			upper right
			\rotate@{\@p@sbburx}{\@p@sbbury}
			\minmaxtest
			\edef\@p@sbbllx{\r@p@sbbllx}\edef\@p@sbblly{\r@p@sbblly}
			\edef\@p@sbburx{\r@p@sbburx}\edef\@p@sbbury{\r@p@sbbury}
%\ps@typeout{rotated BB: \r@p@sbbllx, \r@p@sbblly, \r@p@sbburx, \r@p@sbbury}
		\fi
		\count203=\@p@sbburx
		\count204=\@p@sbbury
		\advance\count203 by -\@p@sbbllx
		\advance\count204 by -\@p@sbblly
		\edef\@bbw{\number\count203}
		\edef\@bbh{\number\count204}
		%\ps@typeout{ bbh = \@bbh, bbw = \@bbw }
}
%
% \in@hundreds performs #1 * (#2 / #3) correct to the hundreds,
%	then leaves the result in @result
%
\def\in@hundreds#1#2#3{\count240=#2 \count241=#3
		     \count100=\count240	% 100 is first digit #2/#3
		     \divide\count100 by \count241
		     \count101=\count100
		     \multiply\count101 by \count241
		     \advance\count240 by -\count101
		     \multiply\count240 by 10
		     \count101=\count240	%101 is second digit of #2/#3
		     \divide\count101 by \count241
		     \count102=\count101
		     \multiply\count102 by \count241
		     \advance\count240 by -\count102
		     \multiply\count240 by 10
		     \count102=\count240	% 102 is the third digit
		     \divide\count102 by \count241
		     \count200=#1\count205=0
		     \count201=\count200
			\multiply\count201 by \count100
		 	\advance\count205 by \count201
		     \count201=\count200
			\divide\count201 by 10
			\multiply\count201 by \count101
			\advance\count205 by \count201
			%
		     \count201=\count200
			\divide\count201 by 100
			\multiply\count201 by \count102
			\advance\count205 by \count201
			%
		     \edef\@result{\number\count205}
}
\def\compute@wfromh{
		% computing : width = height * (bbw / bbh)
		\in@hundreds{\@p@sheight}{\@bbw}{\@bbh}
		%\ps@typeout{ \@p@sheight * \@bbw / \@bbh, = \@result }
		\edef\@p@swidth{\@result}
		%\ps@typeout{w from h: width is \@p@swidth}
}
\def\compute@hfromw{
		% computing : height = width * (bbh / bbw)
	        \in@hundreds{\@p@swidth}{\@bbh}{\@bbw}
		%\ps@typeout{ \@p@swidth * \@bbh / \@bbw = \@result }
		\edef\@p@sheight{\@result}
		%\ps@typeout{h from w : height is \@p@sheight}
}
\def\compute@handw{
		\if@height 
			\if@width
			\else
				\compute@wfromh
			\fi
		\else 
			\if@width
				\compute@hfromw
			\else
				\edef\@p@sheight{\@bbh}
				\edef\@p@swidth{\@bbw}
			\fi
		\fi
}
\def\compute@resv{
		\if@rheight \else \edef\@p@srheight{\@p@sheight} \fi
		\if@rwidth \else \edef\@p@srwidth{\@p@swidth} \fi
		%\ps@typeout{rheight = \@p@srheight, rwidth = \@p@srwidth}
}
%		
% Compute any missing values
\def\compute@sizes{
	\compute@bb
	\if@scalefirst\if@angle
% at this point the bounding box has been adjsuted correctly for
% rotation.  PSFIG does all of its scaling using \@bbh and \@bbw.  If
% a width= or height= was specified along with \psscalefirst, then the
% width=/height= value needs to be adjusted to match the new (rotated)
% bounding box size (specifed in \@bbw and \@bbh).
%    \ps@bbw       width=
%    -------  =  ---------- 
%    \@bbw       new width=
% so `new width=' = (width= * \@bbw) / \ps@bbw; where \ps@bbw is the
% width of the original (unrotated) bounding box.
	\if@width
	   \in@hundreds{\@p@swidth}{\@bbw}{\ps@bbw}
	   \edef\@p@swidth{\@result}
	\fi
	\if@height
	   \in@hundreds{\@p@sheight}{\@bbh}{\ps@bbh}
	   \edef\@p@sheight{\@result}
	\fi
	\fi\fi
	\compute@handw
	\compute@resv}

%
% \psfig
% usage : \psfig{file=, height=, width=, bbllx=, bblly=, bburx=, bbury=,
%			rheight=, rwidth=, clip=}
%
% "clip=" is a switch and takes no value, but the `=' must be present.
\def\psfig#1{\vbox {
	% do a zero width hard space so that a single
	% \psfig in a centering enviornment will behave nicely
	%{\setbox0=\hbox{\ }\ \hskip-\wd0}
	%
	\ps@init@parms
	\parse@ps@parms{#1}
	\compute@sizes
	%
	\ifnum\@p@scost<\@psdraft{
		%
		\special{ps::[begin] 	\@p@swidth \space \@p@sheight \space
				\@p@sbbllx \space \@p@sbblly \space
				\@p@sbburx \space \@p@sbbury \space
				startTexFig \space }
		\if@angle
			\special {ps:: \@p@sangle \space rotate \space} 
		\fi
		\if@clip{
			\if@verbose{
				\ps@typeout{(clip)}
			}\fi
			\special{ps:: doclip \space }
		}\fi
		\if@prologfile
		    \special{ps: plotfile \@prologfileval \space } \fi
		\if@decmpr{
			\if@verbose{
				\ps@typeout{psfig: including \@p@sfile.Z \space }
			}\fi
			\special{ps: plotfile "`zcat \@p@sfile.Z" \space }
		}\else{
			\if@verbose{
				\ps@typeout{psfig: including \@p@sfile \space }
			}\fi
			\special{ps: plotfile \@p@sfile \space }
		}\fi
		\if@postlogfile
		    \special{ps: plotfile \@postlogfileval \space } \fi
		\special{ps::[end] endTexFig \space }
		% Create the vbox to reserve the space for the figure.
		\vbox to \@p@srheight sp{
		% 1/92 TJD Changed from "true sp" to "sp" for magnification.
			\hbox to \@p@srwidth sp{
				\hss
			}
		\vss
		}
	}\else{
		% draft figure, just reserve the space and print the
		% path name.
		\if@draftbox{		
			% Verbose draft: print file name in box
			\hbox{\frame{\vbox to \@p@srheight sp{
			\vss
			\hbox to \@p@srwidth sp{ \hss \@p@sfile \hss }
			\vss
			}}}
		}\else{
			% Non-verbose draft
			\vbox to \@p@srheight sp{
			\vss
			\hbox to \@p@srwidth sp{\hss}
			\vss
			}
		}\fi	



	}\fi
}}
\psfigRestoreAt
\let\@=\LaTeXAtSign


\begin{document}

%
% $Id: titlepage.tex,v 1.8 2004-05-24 23:09:33 edo Exp $
%

\begin{titlepage}

\begin{centering}

%{\em\bf DRAFT document --- not for distribution\\[0.5in]}

{\bf\Huge NWChem User Documentation}\\[0.5in] 
{\bf\Huge Release \nwchemversion}\\[1.0in]

{\bf\Large Molecular Sciences Software Group\\
    W.R.\ Wiley Environmental Molecular Sciences Laboratory\\
    Pacific Northwest National Laboratory\\
    P.O. Box 999, Richland, WA 99352\\[0.5in]}

{\bf\Large \nwchemmonth \ \nwchemyear}\\[1.0in]


%\psfig{figure=zsm.major.ps,height=4in,width=4in}


\end{centering}

\end{titlepage}


\chapter*{\center DISCLAIMER}
%
% $Id: disclaimer.tex,v 1.2 2004-04-22 04:50:28 edo Exp $
%

This material was prepared as an account of work sponsored by an agency of the
United States Government.  Neither the United States Government nor the United
States Department of Energy, nor Battelle, nor any of their employees, MAKES
ANY WARRANTY, EXPRESS OR IMPLIED, OR ASSUMES ANY LEGAL LIABILITY OR
RESPONSIBILITY FOR THE ACCURACY, COMPLETENESS, OR USEFULNESS OF ANY
INFORMATION, APPARATUS, PRODUCT, SOFTWARE, OR PROCESS DISCLOSED, OR REPRESENTS
THAT ITS USE WOULD NOT INFRINGE PRIVATELY OWNED RIGHTS.


\begin{center}
{\bf LIMITED USE}
\end{center}

This software (including any documentation) is being made available to
you for your internal use only, solely for use in performance of work
directly for the U.S. Federal Government or work under contracts with
the U.S. Department of Energy or other U.S. Federal Government
agencies.  This software is a version which has not yet been evaluated
and cleared for commercialization.  Adherence to this notice may be
necessary for the author, Battelle Memorial Institute, to successfully
assert copyright in and commercialize this software.  This software is
not intended for duplication or distribution to third parties without
the permission of the Manager of Software Products at Pacific
Northwest National Laboratory, Richland, Washington, 99352.

\begin{center}
{\bf ACKNOWLEDGMENT}
\end{center}

This software and its documentation were produced with Government support under
Contract Number DE-AC06-76RLO-1830 awarded by the United States Department of
Energy.  The Government retains a paid-up non-exclusive, irrevocable worldwide
license to reproduce, prepare derivative works, perform publicly and display
publicly by or for the Government, including the right to distribute to other
Government contractors.


\clearpage

\begin{center}
{\bf AUTHOR DISCLAIMER}
\end{center}

This software contains proprietary information of the authors, Pacific
Northwest National Laboratory (PNNL), and the US Department of Energy (USDOE).
The information herein shall not be disclosed to others, and shall not
be reproduced whole or in part, without written permission from PNNL or
USDOE.  The information contained in this document is provided ``AS
IS'' without guarantee of accuracy.  Use of this software is
prohibited without {\bf written} permission from PNNL or USDOE.  The
authors, PNNL, and USDOE make no representations or warranties
whatsoever with respect to this software, including the implied
warranty of merchant-ability or fitness for a particular purpose.  The
user assumes all risks, including consequential loss or damage, in
respect to the use of the software.  In addition, PNNL and the authors
shall not be obligated to correct or maintain the program, or notify
the user community of modifications or updates that will be made over
the course of time.


\clearpage

\tableofcontents

\clearpage

\chapter{Introduction}
\label{sec:intro}

NWChem is a computational chemistry package designed to run on
high-performance parallel supercomputers and workstation clusters.
Code capabilities include the calculation of molecular electronic
energies and analytic gradients using self-consistent field, Gaussian
density function theory (DFT), and second-order perturbation theory.
Geometry optimization for energy minimization and transition states is
available for all methods.  Classical molecular dynamics capabilities
provide for the simulation of macromolecules and solutions, including
the computation of free energies, using a variety of forcefields.

NWChem is scalable in both its ability to treat large problems
efficiently, and in its utilization of available parallel computing
resources.  The code uses the parallel programming tools TCGMSG and
the Global Array library developed at PNNL for the High Performance
Computing and Communication Initiative (HPCCI) grand-challenge
software program and the Environmental Molecular Sciences Laboratory
(EMSL) Project.  NWChem has been optimized to perform calculations on
large molecules using large parallel computers and it is unique in
this regard.  In contrast, its performance on small calculations
running on small computers is unremarkable.

This document is intended as an aid to chemists attempting to
use the code for their own applications.  Users are not expected to
have a detailed understanding of the code internals, but some
familiarity with the overall structure of the code, how it handles
information, and the nature of the algorithms it contains will
generally be helpful.  The following sections describe the structure
of the input file, and a give a brief overview of the code
archetecture.  All input directives recognized by the code are
described in detail, with options, defaults, and recommended usage,
where applicable.  Additional information on the molecular geometry
and basis function libraries included in the code is presented in the
appendices.

\subsection{Citation}

The EMSL Software Agreement stipulates that the use of NWChem be
acknowledged in any publications which use results obtained with
NWChem.  The acknowledgment should be of the form:
\begin{quote}

  NWChem Version \nwchemversion, as developed and distributed by
  Pacific Northwest National Laboratory, P.~O.~Box 999, Richland,
  Washington 99352 USA, and funded by the U.~S.~Department of Energy,
  was used to obtain some of these results.
\end{quote}

The words ``A modified version of'' should be added at the beginning,
if appropriate.  {\em Note: Your EMSL Software Agreement contains the
complete specification of the required acknowledgment.}

If you wish to cite NWChem in the references section of a publication,
please use the following citation:
\begin{quote}
  High Performance Computational Chemistry Group, {\em NWChem, A
   Computational Chemistry Package for Parallel Computers, Version
    \nwchemversion{}} (\nwchemyear), Pacific Northwest National
  Laboratory, Richland, Washington 99352, USA.
\end{quote}

\subsection{User Feedback}

This software comes without warranty or guarantee of support,
but we do try to meet the needs of our user community.  Please send bug
reports, requests for enhancement, or other comments to

\begin{itemize}
\item {\tt nwchem-support@emsl.pnl.gov}
\end{itemize}

When reporting problems, please provide as much information as possible, 
including;

\begin{itemize}
\item detailed description of problem
\item platform you are running on
\begin{itemize}
\item operating system
\item compiler
\end{itemize}
\item input file
\item output file
\item contact name and telephone number
\end{itemize}

Users can also subscribe to an electronic mailing list of other users
of the code.  This is intended as a general forum through which code
users can contact one another and the developers to share experience
with the code and discuss problems.  Announcements of new releases and
bug fixes will also be made to this list. 

To subscribe to the user list, send a message to 

\begin{itemize}
\item {\tt majordomo@emsl.pnl.gov}
\end{itemize}

The body of the message must contain the line 

\begin{itemize}
\item {\tt subscribe nwchem-users}
\end{itemize}

The automated list manager is capable of recognizing a number of
commands, including 'subscribe', 'unsubscribe', 'get', 'index',
'which', 'who', 'info' and 'lists'.  The command 'end' halts
processing of commands.  It will provide some help if the message
includes the line {\tt help} in the body.  Messages can be posted to
the list by sending mail to {\tt nwchem-users@emsl.pnl.gov}.  Users
are encouraged to use the support address rather than the mailing list
to report problems since the support mailer interfaces to an automated
bug tracking mechanism.

\section{Getting Started}
\label{sec:getstart}

This section provides an overview of NWChem input and program
architecture, and the syntax used to describe the input.  See Sections
\ref{sec:simplesample} and \ref{sec:realsample} for examples of NWChem
input files with detailed explanation.

NWChem consists of independent modules that perform the various
functions of the code.  Examples of modules include the input parser,
SCF energy, SCF analytic gradient, DFT energy, etc..  Data is passed
between modules and saved for restart using a disk-resident database
or dumpfile (see Section \ref{sec:arch}).

The input to NWChem is composed of commands, called directives, which
define data (such as basis sets, geometries, and filenames), and the
actions to be performed.  Directives are processed in the order
presented in the input file, with the exception of certain start-up
directives (see Section \ref{sec:inputstructure}) which provide
critical job control information and are processed before all other
input.  Most directives are specific to a particular module and define
data that is used by that module only.  A few directives (see Section
\ref{sec:toplevel}) potentially affect all modules, for instance by
specifying the total charge on the system.  

There are two types of directives.  Simple directives consist of one
line of input which may contain multiple fields.  Compound directives
group together multiple simple directives that are in some way
related and are terminated with an \verb+END+ directive.  See the
sample inputs (Sections \ref{sec:simplesample}, \ref{sec:realsample})
and the input syntax specification (Section \ref{sec:syntax}).

All input is free format, and directives or blocks of module-specific
directives can appear in any order with the exception of the
\verb+TASK+ directive (see sections \ref{sec:inputstructure} and
\ref{sec:task}) which is used to invoke an NWChem module.  Case is
ignored except for actual data (e.g., names/tags of centers, titles).

To make the input as short and simple as possible, most options have
default values.  The user needs to supply input only for those items that
have no defaults, or for items that must be different from the defaults
for the particular application.  In the discussion of each directive, the
defaults are noted, where applicable.

The input file structure is described in the following subsections, and
illustrated with two examples.  The input format and syntax for directives
is also described in detail.

\subsection{Input File Structure}
\label{sec:inputstructure}

The structure of an input file reflects the internal structure of
NWChem.  At the beginning of a calculation NWChem needs to determine
how much memory to use, the name of the database, whether it is a new,
restarted, or continuing job, where to put scratch/permanent files,
etc..  It is not necessary to put this information at the top of the
input file, however.  NWChem will read through the {\em entire} input
file looking for the start-up directives.  In this pass, all other
directives are ignored.

The start-up directives are
\begin{itemize}
\item START
\item CONTINUE
\item RESTART
\item SCRATCH{\verb+_+}DIR
\item PERMANENT{\verb+_+}DIR
\item MEMORY
\item ECHO
\end{itemize}

After the input file has been scanned for the start-up directives, it
is rewound and read sequentially.  Input is processed either by the
top-level parser (for the directives listed in Section
\ref{sec:toplevel}, such as \verb+TITLE+, \verb+SET+, \ldots) or by
the parsers for specific computational modules (e.g., SCF, DFT,
\ldots).  Any directives that have already been processed (e.g.,
\verb+MEMORY+) are ignored.  Input is read until a \verb+TASK+
directive (see Section \ref{sec:task}) is encountered.  A \verb+TASK+
directive request a calculation to be peformed and specifies the level
of theory and the operation to be performed.  Input processing then
stops and the specified task is executed.  The position of the
\verb+TASK+ directive in effect marks the end of the input for that
task.  Processing of the input resumes upon the successful completion
of the task and the results of that task are available to subsequent
tasks in the same input file.

The name of the input file is usually provided as an argument to the
execute command for NWChem.  That is, the execute command looks
something like the following;

\begin{verbatim}
  nwchem input_file
\end{verbatim}

The default name for the input file is \verb+nwchem.nw+.  If an input
file name \verb+input_file+ is specified without an extension, the code
assumes a default extension of \verb+input_file.nw+.  If the code cannot
locate a file named either \verb+nwchem.nw+ or \verb+input_file.nw+, an 
error is reported and execution terminates.  The following
subsection presents two input files to illustrate the directive syntax and 
input file format for NWChem applications.

\subsection{Simple N2 Input File}
\label{sec:simplesample}

A simple example of an NWChem input file is an SCF geometry optimization of
the nitrogen molecule using a Dunning cc-pvdz basis set.  This input
file contains the bare minimum of information the user must specify in
order to run this type of problem --- fewer than ten lines of input,
as follows;
\begin{verbatim}
  title; Nitrogen cc-pvdz SCF geometry optimization
  geometry 
    n 0 0 0
    n 0 0 2.1
  end
  basis
    n library cc-pvdz
  end
  task scf optimize
\end{verbatim}

Examining the input line by line, it can be seen that it contains
only four directives; \verb+TITLE+, \verb+GEOMETRY+, \verb+BASIS+, and
\verb+TASK+.  The \verb+TITLE+ directive is optional, and is used only
as a means for the user to more easily identify outputs from different
jobs.  An initial geometry is specified in cartesian coordinates and
atomic units by means of the \verb+GEOMETRY+ directive.  The Dunning 
cc-pvdz basis is obtained from the NWChem basis library, as specified
by the \verb+BASIS+ directive input.  The \verb+TASK+ directive requests 
an SCF geometry optimization.

The \verb+GEOMETRY+ directive (Section \ref{sec:geom}) defaults to cartesian
coordinates and atomic units (options include {\AA}ngstr{\o}m units and
Z-matrix format; see Section \ref{sec:Z-matrix}).  The \verb+BASIS+ 
directive input block is structured like the \verb+GEOMETRY+ directive
input block (i.e., name,
keyword, \ldots, end) and {\em must} contain basis set information for
every atom type in the geometry with which it will be used.
Refer to Sections \ref{sec:basis} and \ref{sec:ecp}, and Appendix
\ref{sec:knownbasis} for a description of available basis sets and a
discussion of how to define new ones.

The last line of this sample input file ({\tt task scf optimize}),
tells the program to optimize the molecular geometry by minimization
of the SCF energy.  (For a description of possible tasks and the format
of the {\tt task} directive, refer to Section \ref{sec:task}.)

If the input is stored in the file \verb+n2.nw+, the command to run
the job on a typical UNIX workstation is as follows;

\begin{verbatim}
  nwchem n2
\end{verbatim}

NWChem output is to standard output, and error messages are sent to
both standard output and standard error.

\subsection{Water Molecule Sample Input File}
\label{sec:realsample}

A more complex sample problem is the optimization of a positively
charged water molecule at the MP2 level of theory, followed by a computation of
frequencies at the optimized geometry.  A preliminary SCF geometry
optimization is performed using an inexpensive basis set (STO-3G).
This gives a good starting guess for the geometry, and any Hessian
information generated will be used in the next optimization step.
Then the optimization is finished using second-order M{\o}ller-Plesset
perturbation theory and a basis set with polarization functions.  The
final task is to calculate the vibrational frequencies.  The input
file to accomplish these three tasks is as follows;

\begin{verbatim}
start h2o_freq

title; H2O+ frequencies with MP2 and 6-31g**

geometry units angstrom
  O       0.0  0.0  0.0
  H       0.0  0.0  1.0
  H       0.0  1.0  0.0
end

charge 1

basis "starting basis"
  H library sto-3g
  O library sto-3g
end

basis "property basis"
  H library 6-31g**
  O library 6-31g**
end

scf
  uhf
  doublet
  print low
end

stepper
  trust 0.5
  convgg 0.01
end

set "ao basis" "starting basis"

task scf optimize

set "ao basis" "property basis"

scf
  vectors input atomic
  print none
end

set "mp2_grad:print" low

stepper
  convgg 1d-6
end

task mp2 optimize

set "mp2_grad:print" none

task mp2 freq

eof
\end{verbatim}

The {\tt start} directive (Section \ref{sec:start}) tells NWChem that
this run is to be started from the beginning.  This directive need not
be at the beginning of the input, but it is commonly placed there.
Existing database or vector files are to be ignored or overwritten.
The entry \verb+h2o_freq+ on the \verb+START+ line is the prefix used
for all files created by the calculation.  This convention allows
different jobs to run in the same directory or to share the same
scratch directory (see Section \ref{sec:dirs}), as long as they use
different names in this field.

As in the first sample problem, the geometry is given in cartesian
coordinates.  In this case, however, the units are specified as
{\AA}ngstr{\o}m instead of the default atomic units.  The {\tt CHARGE}
directive defines the total charge of the system.  This calculation is
to be done on an ion with charge +1.  Note that this is a top-level
directive, independent of other input blocks.

The next two directives are the {\tt BASIS} directives.  The names for
the basis sets are arbitrary, and they may be chosen so that they will
serve as a helpful mnemonic for later reference in subsequent input
directives.

The multiple lines of the first {\tt SCF} directive in the {\tt scf
  \ldots end} block specify details about the scf calculation to be
performed.  For open-shell systems, the spin multiplicity has to be
specified (using {\tt doublet} in this case), or it defaults to {\tt
  singlet}.  Unrestricted Hartree-Fock is chosen here (by specifying
the keyword {\tt uhf}), rather than the default high-spin ROHF.  This
is necessary for the subsequent MP2 calculation, because only UMP2 is
available for open-shell systems currently (see Section
\ref{sec:functionality}).  The print level is set to {\tt low} to
reduce the output from the scf module to a minimum during the geometry
optimization.

For this calculation, the gradient convergence threshold for the
geometry optimization module {\tt STEPPER} is set to the
relaxed value of $10^{-2}$, rather than the default $10^{-4}$.  This is
acceptable because the initial optimization only serves to provide a
starting guess.  Increasing the trust radius (using the \verb+TRUST+
directive) beyond the default of $0.1$ makes the optimization more
efficient (though less robust in hard to converge cases).

The final step in setting up the input for the SCF calculation is to
specify that the \verb+"starting basis"+ should be used.  This is
accomplished by means of the \verb+SET+ directive, which specifies that 
the \verb+"ao basis"+ is the \verb+"starting basis"+.  All molecular
orbital (MO) methods look for the \verb+"ao basis"+ in which to expand the
MOs.    In the previous example with the nitrogen molecule, it was not 
necessary to use the \verb+SET+ directive in this manner, since the 
\verb+BASIS+ directive used the default name (which is \verb+"ao basis"+).
An alternative approach that could have been used here which avoids the 
necessity of assigning the basis set name using the
\verb+SET+ directive would be to defer definition of the property basis
until after the first SCF calculation is complete. 

All input up to this point affects only the settings in the run-time
database.  The program takes its information from this data base, so
the sequence of directives up to the first \verb+TASK+ directive is
irrelevant.  An exchange of order of the different blocks or
directives would not affect the result.  The {\tt TASK} directive,
however, must be specified last of all in the list of input directives
for a given problem.  The {\tt TASK} directive invokes the program and
directs the code to perform the specified calculation using the
parameters set in the previous directives. In this case, the first
task is an SCF calculation with geometry optimization, specified with
the input {\tt scf} and {\tt optimize}.  (See Section \ref{sec:task}
for a list of available tasks and operations.)

After the completion of any task, settings in the database are used
in subsequent tasks without change, unless they are overridden by new
input directives.  In this example, there are several important
changes between the first task (\verb+task scf optimize+) and the
second task (\verb+task mp2 optimize+).  The {\tt "ao basis"} is set
to a better basis (i.e., {\tt "property basis"}) using the \verb+SET+
directive.  The SCF output is completely discarded (as a result of the
{\tt print none} input in the {\tt SCF} directive), which means that
no output will be produced unless an error occurs.  The SCF starting
guess vectors are reset to be the atomic guess, since the STO3G
vectors are not appropriate (see Section \ref{sec:vectors} for
details on the different possible starting guesses).

The convergence threshold for the geometry optimization
module is reset to $10^{-4}$.  This is
acceptable because the initial optimization only serves to provide a
starting guess.  Increasing the trust radius (using the \verb+TRUST+
directive) beyond the default of $0.1$ makes the optimization more
efficient (though less robust in hard to converge cases).

The final step in setting up the input for the SCF calculation is to
specify that the \verb+"starting basis"+ should be used.  This is
accomplished by means of the \verb+SET+ directive, which specifies that 
the \verb+"ao basis"+ is the \verb+"starting basis"+.  All molecular
orbital (MO) methods look for the \verb+"ao basis"+ in which to expand the
MOs.    In the previous example with the nitrogen molecule, it was not 
necessary to use the \verb+SET+ directive in this manner, since the 
\verb+BASIS+ directive used the default name (which is \verb+"ao basis"+).
An alternative approach that could have been used here which avoids the 
necessity of assigning the basis set name using the
\verb+set+ directive would be to defer definition of the property basis
until after the first SCF calculation is complete. 

All input up to this point affects only the settings in the run-time
database.  The program takes its information from this data base, so
the sequence of directives up to the first \verb+TASK+ directive is
irrelevant.  An exchange of order of the different blocks or
directives would not affect the result.  The {\tt TASK} directive,
however, must be specified last of all in the list of input directives
for a given problem.  The {\tt TASK} directive invokes the program and
directs the code to perform the specified calculation using the
parameters set in the previous directives. In this case, the first
task is an SCF calculation with geometry optimization, specified with
the input {\tt scf} and {\tt optimize}.  (See Section \ref{sec:task}
for a list of available tasks and operations.)

After the completetion of any task, settings in the database are used
in subsequent tasks without change, unless they are overridden by new
input directives.  In this example, there are several important
changes between the first task (\verb+TASK scf optimize+) and the
second task (\verb+TASK mp2 optimize+).  The {\tt "ao basis"} is set
to a better basis (i.e., {\tt "property basis"}) using the \verb+SET+
directive.  The SCF output is completely discarded (as a result of the
{\tt print none} input in the {\tt scf} directive), which means that
no output will be produced unless an error occurs.  The SCF starting
guess vectors are reset to be the atomic guess, since the STO3G
vectors are not appropriate (see Section \ref{sec:vectors} for
details on the different possible starting guesses).

The convergence threshold for the geometry optimization
module is reset to  $1.0^{-6}$, a value smaller than the default of
$1.0^{-4}$.  The default is appropriate for determination of the
optimized geometry and minimum energy, but a higher accuracy is
required for subsequent computation of the frequencies.  The second
{\tt TASK} directive invokes an MP2 optimization, instead of the scf
optimization.

% There is currently no input block for the MP2 gradients, so parameters
% for this module have to be set in the database via the {\tt set}
% directive. Currently only print options are recognized, and in this
% example, the \verb+SET+ directive is used to reduce the amount of
% output a reasonable minimum.

Once the MP2 optimization is completed, the geometry obtained in the
calculation is used to perform a frequency calculation.  This task is
invoked by the keyword \verb+freq+ on the directive \verb+task mp2
freq+.  The second derivatives of the energy are calculated as
numerical derivatives of analytical gradients. The gradients as such
are not of interest in this case, so the output from the {\tt
  mp2\_grad} program is completely discarded using the directive
\verb+set "mp2_grad:print" none+.

The {\tt EOF} directive marks the end of the input to be read. It is
not necessary to specify the end-of-file as an explicit directive at
the actual end of the file.  However, using the directive allows the
user to have additional lines in the file that will not be processed
as input.  For example, a description of the calculation(s) specified
in the input or notes on the purpose of the task(s), or other
informational comments can be included in the input file in this way.
The {\tt eof} directive can also be used to remove a task (or series
of tasks) from the calculation without actually removing the relevent
input directives from the file.  Any lines in the file that occur
after the {\tt eof} directive are ignored by the program.

\subsection{Input Format and Syntax for Directives}
\label{sec:syntax}

This section describes the syntax used in the rest of this
documentation for describing the format of directives.  
The input format for the directives used for NWChem is similar to that
of UNIX shells, which is also used in other chemistry packages, most
notably GAMESS-UK.  An input line is parsed into whitespace (blanks or
tabs) separating tokens or fields.  Any token that contains whitespace
must be enclosed in double quotes in order to be processed correctly.
For example, the basis set with the descriptive name \verb+modified
Dunning DZ+ must appear in a directive as \verb+"modified Dunning
DZ"+, since the name consists of three separate words.  

A (physical) line in the input file is terminated with a newline
character (also known as a 'return' or 'enter' character).  A
semicolon (\verb+;+) can be also used to indicate the end of an input
line, allowing a given physical line of input to contain multiple
logical lines of input.  For example, five lines of input for the
\verb+GEOMETRY+ directive can be entered as follows;
\begin{verbatim}
  geometry
   O 0  0     0
   H 0  1.430 1.107
   H 0 -1.430 1.107
  end
\end{verbatim}
These same five lines could be entered on a single line, as
\begin{verbatim}
  geometry; O 0 0 0; H 0 1.430 1.107; H 0 -1.430 1.107; end
\end{verbatim}
This is one physical input line comprises five logical
input lines.  Each logical or physical input line must be no longer
than 1023 characters.  Data is read until an end-of-file is detected,
or until an \verb+EOF+ directive is encountered\footnote{The
  free-format input library does not read past the string \verb+EOF+
  (ignoring case) on a line by itself.  This is a convenience that
  allows unused input to be left in the same file.}.

Directives consist of a directive name, keywords, and optional input,
and may contain of one line or many.  Simple directives consist of a
single line of input with one or more fields.  Compound directives can
have multiple input lines, and can also include other optional simple
and compound directives.  A compound directive is terminated with an
END directive.  The directives START (see Section \ref{sec:start}) and
ECHO (see Section \ref{sec:echo}) are examples of simple directives.
The directive GEOMETRY (see Section \ref{sec:geom}) is an example of a
compound directive.

Some limited checking for self-consistency of the input is performed
by the input module, but most defaults are imposed by the application
modules at runtime.  It is therefore usually impossible to determine
beforehand whether or not all selected options are consist with each
other.

The following notation and syntax conventions are used in the generic 
descriptions of the NWChem input in the following sections.

\begin{itemize}
\item a directive name always appears in all-capitals, in computer-type 
face; (e.g.; \verb+GEOMETRY+, \verb+BASIS+, \verb+SCF+).  Note that 
the case of directives is ignored in the actual input.
\item a keyword always appears in lower case, in computer-type face; (e.g.,
{\tt swap}, {\tt print}, {\tt units}, {\tt bqbq})
\item variable names always appear in lower case, in computer-type face, 
and enclosed in angle brackets to distinguish them from keywords (e.g.,
{\tt <input\_filename>}, {\tt <basisname>}, {\tt <tag>})
\item \verb+$variable$+ is used to indicate the substitution of the value of a
      variable
\item a string, token, or field is a sequence of ASCII characters (NOTE: if 
the string includes blanks or tabs (i.e., white space) the entire string must
be enclosed in quotes)
\item \verb+()+ is used to group items (the parentheses and other
      special symbols do not appear in the input)
\item \verb+||+ separate exclusive options/parameters/formats
\item \verb+[ ]+ enclose optional entries with a default value
\item \verb+< >+ enclose a type, a name of a value to be specified,
      and a default value if any.

\item \verb+\+ is used to concatenate lines in a description (NOTE: within 
a string, the \verb+\+ causes the preceeding character (including a blank) 
to be eliminated
\item \verb+;+ (semicolon) is used to mark the end of a logical input 
line within a physical line of input
\item \verb+#+ (the hash or pound symbol) is the comment character.  All
characters following \verb+#+ (up to the physical end of the line) are ignored.

\item \verb+...+ is used to indicate indefinite continuation of a list
\end{itemize}

An input parameter is identified in the description of the directive
by prefacing the item with the name of the type of data expected;
i.e.,

\begin{itemize}
\item \verb+string +  -- an ASCII character string
\item \verb+integer+ --  integer value(s) for a variable or an array
\item \verb+logical+ --  true/false logical variable
\item \verb+real   +  -- real floating point value(s) for a variable or 
\item \verb+double + -- double-precision
an array
\end{itemize}

If an input item is not prefaced by one of these type names,
it is assumed to be of type 'string'.
 
The directive \verb+VECTORS+ (Section \ref{sec:vectors}) is presented here
as an example of an NWChem input directive.  The general form of the
directive is as follows;
\begin{verbatim}
  VECTORS [input (<string input_movecs default atomic>) || \
                   (project <string basisname> <string filename>)] \
          [swap [(alpha|beta)] <integer vec1 vec2> ...] \
          [output <string output_movecs default $file_prefix$.movecs>]
\end{verbatim}

This directive contains three optional keywords, as indicated by the 
three main sets of square brackets enclosing the keywords \verb+input+,
\verb+swap+, and \verb+output+.  The keyword \verb+input+ allows the
user to specify the source of the molecular orbital vectors.  When
this keyword is invoked, there are two mutually exclusive options for
specifying the vectors, as indicated by the \verb+||+ symbol
separating the descriptions of the two input options;

\begin{verbatim}
  (<string input_movecs default atomic>) || \
                  (project <string basisname> <string filename>) \
\end{verbatim}

The first option, \verb+(<string input_movecs default atomic>)+,
allows the user to specify an ASCII character string for the parameter
{\tt input\_movecs}.  If no entry is specified, the code assumes a
standard source of the vectors with a default of \verb+atomic+ (i.e.,
atomic guess).  The second option, {\tt(project <string basisname>
  <string filename>)}, contains the keyword \verb+project+, which
takes two additional string arguments.  When this keyword is used, the
vectors in file \verb+<filename>+ will be projected from the (smaller)
basis \verb+<basisname>+ into the current AO basis.

The second keyword, \verb+swap+, allows the user to re-order the
starting vectors.  The optional keyword \verb+alpha|beta+ allows the
user to swap the alpha and beta spin orbitals, and specify in pairs
the integer numbers of the vectors to be swapped.  As many pairs as
the user wishes to have swapped can be listed for {\tt <integer vec1
vec2 ... >}.

The third keyword, \verb+output+, allows the user to tell the code
where to store the vectors by specifying an ASCII string for the
parameter {\tt output\_movecs}.  If no entry is specified for this
parameter, the default is to write the vectors back into either the user
specified MO vectors input file or, if this is not available, the file
\verb+$file_prefix$.movecs+.

A particular example of the VECTOR directive is shown below.  It specifies
both the \verb+input+ and \verb+output+ keywords, but does not use the 
\verb+swap+ option.

\begin{verbatim}
  vectors input project "small basis" small_basis.movecs \
          output large_basis.movecs
\end{verbatim}

This directive tells the code to generate input vectors by projecting
from vectors in a smaller basis named \verb+"small basis"+, which is
stored in the file \verb+small_basis.movecs+.  The output vectors will be
stored in the file \verb+large_basis.movecs+.

The order of keyed optional entries within a directive should not matter,
unless noted otherwise in the specific instructions for a particular
directive.

\section{NWChem Architecture}
\label{sec:arch}

As noted above, NWChem consists of independent modules that perform
the various functions of the code.  Examples of modules include the
input parser, SCF energy, SCF analytic gradient, and DFT energy.  The
independent NWChem modules can share data only through a disk-resident
database, which is a similiar to the GAMESS dumpfile or the Gaussian
checkpoint file.  This allows the modules to share data, or to share
access to files containing data.

It is not necessary for the user to be intimately familiar with the
contents of the database in order to run NWChem.  However, a nodding
acquaintance with the design of the code will help in clarifying the
logic behind the input requirements, especially when restarting jobs
or performing multiple tasks within one job.  Section
\ref{sec:database} gives a general description of the database.

As described above (Section \ref{sec:inputstructure}), all
start-up directives are processed at the beginning of the job
by the main program, and
then the input module is invoked.  Each input directive usually
results in one or more entries being made in the database.  When a
\verb+TASK+ directive is encountered, control is passed to the
appropriate module, which extracts relevant data from the database and
any associated files.  Upon completion of the task, the module will store
significant results in the database, and may also modify other
database entries in order to modify the behaviour of subsequent
computations.

\subsection{Database Structure}
\label{sec:database}

Data is shared between modules of NWChem by means of the database.  There
are three main types of information stored in the data base; (1) arrays of
data, (2) names of files that contain data, and (3) objects.  
Arrays are stored directly in the database, and contain the following
information;
\begin{enumerate}
\item the name of the array, which is a string of ASCII characters (e.g., 
      \verb+"reference energies"+)
\item the type of the data in the array 
(i.e., real, integer, logical, or character) 
\item the number (N) of data items in the array (Note: A scalar is stored as an array of unit length.)
\item the N items of data of the specified type
\end{enumerate}


From the NWChem input deck it is possible to enter data directly into
the database using the \verb+SET+ directive (see Section
\ref{sec:set}).  For example, to store a (64-bit precision)
three-element real array with the name \verb+"reference energies"+ in
the database, the directive is as follows;
\begin{verbatim}
  set "reference energies" 0.0 1.0 -76.2
\end{verbatim}
NWChem determines the data to be real (based on the type of the first
element, \verb+0.0+), counts the number
of elements in the array, and enters the array into the database.

Much of the data stored in the database is internally managed by
NWChem and should not be modified by the user.  However, other data,
including some NWChem input options, may be freely modified.  

Objects are built in the database by storing the associated data as
multiple entries using an internally consistent naming convention.
This data is exclusively managed by the subroutines (or methods) that
are associated with the object.  Currently, the code has two main
objects; basis sets and geometries.  Sections \ref{sec:geom}, 
\ref{sec:basis}, and
\ref{sec:ecp} present a complete discussion of the input to describe
these objects.  

As an illustration of what comprises a geometry object, the following
table contains a partial listing of the NWChem output of a water molecule
geometry named \verb+"test geom"+.  Each entry contains the field 
\verb+test geom+, which is the unique name of the object.

\begin{verbatim}
 Contents of RTDB h2o.db
 -----------------------

 Entry                                   Type[nelem]
 ---------------------------  ----------------------
 geometry:test geom:efield              double[3]    
 geometry:test geom:coords              double[9]    
 geometry:names                         char[10]   
 geometry:test geom:ncenter                int[1]    
 geometry:ngeom                           int[1]    
 geometry:test geom:charges             double[3]    
 geometry:test geom:tags                  char[6]
 ...
\end{verbatim}

Using this convention, multiple instances of objects may be stored with
different names in the same database.  If a user needed to do calculations 
considering alternative geomteries
for the water molecule, for example, an input file could be constructed with 
all of the geometries of interest by storing them in the 
database under different names.  

% The {\tt
%   GEOMETRY} directive (Section \ref{sec:geom}) permits geometries to
% be named (the default name is \verb+geometry+).  For example, the
% input directive to define a geometry object in the database with the
% name \verb+"test water geometry"+ can be specified as follows;

The run-time database contents for the file \verb+h2o.db+ listed 
above were generated from the user-specified input directive,
\begin{verbatim}
  geometry "test geom"
    O     0.00000000    0.00000000    0.00000000
    H     0.00000000    1.43042809   -1.10715266
    H     0.00000000   -1.43042809   -1.10715266
  end
\end{verbatim}

Obviously, the geometry object contains more information than merely the
numerical input specified in this directive.  The \verb+GEOMETRY+
directive allows the user to specify different values for the x,y,z
coordinates of the atoms (or centers), and identify that object under
a unique name.  In addition, all of the other information in the 
geometry oject also available
to the specifically named objects.  (Refer to Section
\ref{sec:geom} for a complete description of the {\tt GEOMETRY}
directive.)

Unless a specific name is defined for the geometry object in the
database (such as the name \verb+"test water geometry"+ shown in the
example), the object in the database is assigned the default name of
\verb+geometry+.  This is the geometry object name that computational
modules will look for when executing a calculation.  The {\tt SET}
directive can be used in the input to force a module (or modules) to
look for a geometry object with a different name than \verb+geometry+
for a particular task.  For example, to specify use of the 
\verb+"test water geometry"+ example above one would specify

\begin{verbatim}
  set geometry "test water geometry"
\end{verbatim}

NWChem will automatically check for such indirections when loading
geometries.  The basis set object functions in an identical fashion,
using the default name \verb+"ao basis"+.

% , and it is intended that all
% future such objects will do so.  (Note: the naming conventions and
% internal mechanisms for associating data with specific modules or
% tasks is expected to change in the future, but the directive for
% specifying names should remain the same.)

\subsection{Persistence of data and restart}
\label{sec:persist}

The database is persistent, meaning that all input data and results
that are not destroyed in the course of execution are permanently
stored.  These data are therefore available to subsequent tasks or
jobs.  This makes the input for restart jobs very simple since only
new or changed data must be provided.  It also makes the behavior of
successive restart jobs {\em identical} to that of multiple tasks
within one job.  

Sometimes, however, this persistence is undesirable, and it is
necessary to return an NWChem module to its default (input-free)
behavior. In such a case, the \verb+UNSET+ directive (see Section
\ref{sec:unset}) can be used to delete all database entries associated
with a given module (including both inputs and outputs).








\chapter{Getting Started}
\label{sec:getstart}

This section provides an overview of NWChem input and program
architecture, and the syntax used to describe the input.  See Sections
\ref{sec:simplesample} and \ref{sec:realsample} for examples of NWChem
input files with detailed explanation.

NWChem consists of independent modules that perform the various
functions of the code.  Examples of modules include the input parser,
SCF energy, SCF analytic gradient, DFT energy, etc..  Data is passed
between modules and saved for restart using a disk-resident database
or dumpfile (see Section \ref{sec:arch}).

The input to NWChem is composed of commands, called directives, which
define data (such as basis sets, geometries, and filenames) and the
actions to be performed on that data.  Directives are processed in the order
presented in the input file, with the exception of certain start-up
directives (see Section \ref{sec:inputstructure}) which provide
critical job control information, and are processed before all other
input.  Most directives are specific to a particular module and define
data that is used by that module only.  A few directives (see Section
\ref{sec:toplevel}) potentially affect all modules, for instance by
specifying the total electric charge on the system.    

There are two types of directives.  Simple directives consist of one
line of input, which may contain multiple fields.  Compound directives
group together multiple simple directives that are in some way
related and are terminated with an \verb+END+ directive.  See the
sample inputs (Sections \ref{sec:simplesample}, \ref{sec:realsample})
and the input syntax specification (Section \ref{sec:syntax}).

All input is free format and case is ignored except for actual data
(e.g., names/tags of centers, titles). Directives or blocks of
module-specific directives (i.e., compound directives) can appear in
any order, with the exception of the \verb+TASK+ directive (see
sections \ref{sec:inputstructure} and \ref{sec:task}) which is used to
invoke an NWChem module.  All input for a given task must
proceed the \verb+TASK+ directive.  This input specification rule
allows the concatenation of multiple tasks in a single NWChem input
file. 

To make the input as short and simple as possible, most options have
default values.  The user needs to supply input only for those items that
have no defaults, or for items that must be different from the defaults
for the particular application.  In the discussion of each directive, the
defaults are noted, where applicable.

The input file structure is described in the following sections, and
illustrated with two examples.  The input format and syntax for directives
is also described in detail.

\section{Input File Structure}
\label{sec:inputstructure}

The structure of an input file reflects the internal structure of
NWChem.  At the beginning of a calculation, NWChem needs to determine
how much memory to use, the name of the database, whether it is a new,
restarted, or continuing job, where to put scratch/permanent files,
etc..  It is not necessary to put this information at the top of the
input file, however.  NWChem will read through the {\em entire} input
file looking for the start-up directives.  In this first pass, all other
directives are ignored.

The start-up directives are
\begin{itemize}
\item \verb+START+
\item \verb+CONTINUE+
\item \verb+RESTART+
\item \verb+SCRATCH_DIR+
\item \verb+PERMANENT_DIR+
\item \verb=MEMORY=
\item \verb=ECHO=
\end{itemize}

After the input file has been scanned for the start-up directives, it
is rewound and read sequentially.  Input is processed either by the
top-level parser (for the directives listed in Section
\ref{sec:toplevel}, such as \verb+TITLE+, \verb+SET+, \ldots) or by
the parsers for specific computational modules (e.g., SCF, DFT,
\ldots).  Any directives that have already been processed (e.g.,
\verb+MEMORY+) are ignored.  Input is read until a \verb+TASK+
directive (see Section \ref{sec:task}) is encountered.  A \verb+TASK+
directive requests that a calculation be performed and specifies the level
of theory and the operation to be performed.  Input processing then
stops and the specified task is executed.  The position of the
\verb+TASK+ directive in effect marks the end of the input for that
task.  Processing of the input resumes upon the successful completion
of the task, and the results of that task are available to subsequent
tasks in the same input file.

The name of the input file is usually provided as an argument to the
execute command for NWChem.  That is, the execute command looks
something like the following;

\begin{verbatim}
  nwchem input_file
\end{verbatim}

The default name for the input file is \verb+nwchem.nw+.  If an input
file name \verb+input_file+ is specified without an extension, the code
assumes \verb+.nw+ as a default extension, and the input filename becomes \verb+input_file.nw+.  If the code cannot
locate a file named either \verb+nwchem.nw+ or \verb+input_file.nw+, an 
error is reported and execution terminates.  The following
section presents two input files to illustrate the directive syntax and 
input file format for NWChem applications.

\section{Simple Input File --- SCF geometry optimization}
\label{sec:simplesample}

A simple example of an NWChem input file is an SCF geometry optimization of
the nitrogen molecule, using a Dunning cc-pvdz basis set.  This input
file contains the bare minimum of information the user must specify
to run this type of problem --- fewer than ten lines of input,
as follows:
\begin{verbatim}
  title; Nitrogen cc-pvdz SCF geometry optimization
  geometry  
    n 0 0 0
    n 0 0 1.08
  end
  basis
    n library cc-pvdz
  end
  task scf optimize
\end{verbatim}

Examining the input line by line, it can be seen that it contains
only four directives; \verb+TITLE+, \verb+GEOMETRY+, \verb+BASIS+, and
\verb+TASK+.  The \verb+TITLE+ directive is optional, and is provided
as a means for the user to more easily identify outputs from different
jobs.  An initial geometry is specified in Cartesian coordinates and
{\angstroms} by means of the \verb+GEOMETRY+ directive.  The Dunning 
cc-pvdz basis is obtained from the NWChem basis library, as specified
by the \verb+BASIS+ directive input.  The \verb+TASK+ directive requests 
an SCF geometry optimization.

The \verb+GEOMETRY+ directive (Section \ref{sec:geom}) defaults to Cartesian
coordinates and {\angstroms} (options include atomic units and
Z-matrix format; see Section \ref{sec:Z-matrix}).  The input blocks for the  \verb+BASIS+ 
and \verb+GEOMETRY+ directives are structured in similar fashion, 
i.e., name, keyword, \ldots, end (In this simple example, there are no keywords).  The \verb+BASIS+ input block {\em must} contain basis set information for
every atom type in the geometry with which it will be used.
Refer to Sections \ref{sec:basis} and \ref{sec:ecp}, and Appendix
\ref{sec:knownbasis} for a description of available basis sets and a
discussion of how to define new ones.

The last line of this sample input file ({\tt task scf optimize})
tells the program to optimize the molecular geometry by minimizing
the SCF energy.  (For a description of possible tasks and the format
of the \verb+TASK+ directive, refer to Section \ref{sec:task}.)

If the input is stored in the file \verb+n2.nw+, the command to run
this job on a typical UNIX workstation is as follows:

\begin{verbatim}
  nwchem n2
\end{verbatim}

NWChem output is to UNIX standard output, and error messages are sent to
both standard output and standard error.

\section{Water Molecule Sample Input File}
\label{sec:realsample}

A more complex sample problem is the optimization of a positively
charged water molecule at the MP2 level of theory, followed by a
computation of frequencies at the optimized geometry.  A preliminary
SCF geometry optimization is performed using a computationally
inexpensive basis set (STO-3G).  This yields a good starting guess for
the optimal geometry, and any Hessian information generated will be
used in the next optimization step.  Then the optimization is finished
using second-order M{\o}ller-Plesset perturbation theory and a basis
set with polarization functions.  The final task is to calculate the
vibrational frequencies.  The input file to accomplish these three
tasks is as follows:

\begin{verbatim}
start h2o_freq

title; H2O+ frequencies with MP2 and 6-31g**

geometry units angstroms
  O       0.0  0.0  0.0
  H       0.0  0.0  1.0
  H       0.0  1.0  0.0
end

charge 1

basis "starting basis"
  H library sto-3g
  O library sto-3g
end

basis "final basis"
  H library 6-31g**
  O library 6-31g**
end

scf
  uhf
  doublet
  print low
end

set "ao basis" "starting basis"

task scf optimize

set "ao basis" "final basis"

scf
  vectors input atomic
  print none
end

task mp2 optimize

mp2
  print none
end

task mp2 freq

eof
\end{verbatim}

The \verb+START+ directive (Section \ref{sec:start}) tells NWChem that
this run is to be started from the beginning.  This directive need not
be at the beginning of the input file, but it is commonly placed there.
Existing database or vector files are to be ignored or overwritten.
The entry \verb+h2o_freq+ on the \verb+START+ line is the prefix to be used
for all files created by the calculation.  This convention allows
different jobs to run in the same directory or to share the same
scratch directory (see Section \ref{sec:dirs}), as long as they use
different prefix names in this field.

As in the first sample problem, the geometry is given in Cartesian
coordinates.  In this case, the units are specified as {\angstroms}.
(Since this is the default, explicit specification of the units is not
actually necessary, however.)  The {\tt CHARGE} directive defines the
total charge of the system.  This calculation is to be done on an ion
with charge +1.

The next two directives are the {\tt BASIS} directives.  The names for
the basis sets are arbitrary, and they may be chosen so that they will
serve as a helpful mnemonic for later reference in subsequent input
directives.

The multiple lines of the first {\tt SCF} directive in the {\tt scf
  \ldots end} block specify details about the SCF calculation to be
performed.  Unrestricted Hartree-Fock is chosen here (by specifying
the keyword {\tt uhf}), rather than the default, restricted open-shell
high-spin Hartree-Fock (ROHF).  This is necessary for the subsequent
MP2 calculation, because only UMP2 is currently available for
open-shell systems (see Section \ref{sec:functionality}).  For
open-shell systems, the spin multiplicity has to be specified (using
{\tt doublet} in this case), or it defaults to {\tt singlet}.  The
print level is set to {\tt low} to reduce the output from the SCF
module to a minimum during the geometry optimization (to avoid verbose
output for the starting basis calculations).

The final step in setting up the input for the SCF calculation is to
specify that the \verb+"starting basis"+ should be used.  This is
accomplished by means of the \verb+SET+ directive, which specifies
that the \verb+"ao basis"+ is the \verb+"starting basis"+.  All
molecular orbital (MO) methods look for the \verb+"ao basis"+ in which
to expand the MOs.  In the previous example with the nitrogen
molecule, it was not necessary to use the \verb+SET+ directive in this
manner, since the \verb+BASIS+ directive used the default name (which
is \verb+"ao basis"+).  An alternative approach, which avoids the
necessity of assigning the starting basis set name, would be to defer
definition of the final basis until after the first SCF calculation is
complete.

All input up to this point affects only the settings in the runtime
database.  The program takes its information from this database, so
the sequence of directives up to the first \verb+TASK+ directive is
irrelevant.  An exchange of order of the different blocks or
directives would not affect the result.  The {\tt TASK} directive,
however, must be specified last of all in the list of input directives
for a given problem.  The {\tt TASK} directive invokes the program and
directs the code to perform the specified calculation using the
parameters set in the preceding directives. In this case, the first
task is an SCF calculation with geometry optimization, specified with
the input {\tt scf} and {\tt optimize}.  (See Section \ref{sec:task}
for a list of available tasks and operations.)

After the completion of any task, settings in the database are used
in subsequent tasks without change, unless they are overridden by new
input directives.  In this example, there are several important
changes between the first task (\verb+task scf optimize+) and the
second task (\verb+task mp2 optimize+).  The {\tt "ao basis"} is set
to a better basis (i.e., {\tt "final basis"}), using the \verb+SET+
directive.  The SCF output will be completely discarded (as a result of the
{\tt print none} input in the {\tt SCF} directive), which means that
no SCF output will be produced unless an error occurs.  The SCF starting
guess vectors are reset to be the atomic guess,  since the STO-3G
vectors are no longer appropriate (see Section \ref{sec:vectors} for
details on the different possible starting guesses).

The second {\tt TASK} directive invokes an MP2 optimization, instead
of the SCF optimization.

Once the MP2 optimization is completed, the geometry obtained in the
calculation is used to perform a frequency calculation.  This task is
invoked by the keyword \verb+freq+ in the final \verb+TASK+ directive,
\verb+task mp2 freq+.  The second derivatives of the energy are
calculated as numerical derivatives of analytical gradients. The
gradients as such are not of interest in this case, so the amount of
output from the {\tt mp2} module is reduced with the \verb+PRINT+
directive.

The {\tt EOF} directive marks the end of the input to be read. It is
not necessary to specify the end-of-file as an explicit directive at
the actual end of the file.  However, using the directive allows
additional lines in the file that will not be processed as input.  For
example, the user might want to add a description of the
calculation(s) specified in the input or notes on the purpose of the
task(s), or other informational comments.  The {\tt eof} directive can
also be used to remove a task (or series of tasks) from the
calculation without actually removing the relevant input directives
from the file.  Any lines in the file that occur after the {\tt eof}
directive are ignored by the program.

\section{Input Format and Syntax for Directives}
\label{sec:syntax}

This section describes the syntax used in the rest of this
documentation to describe the format of directives.  
The input format for the directives used in NWChem is similar to that
of UNIX shells, which is also used in other chemistry packages, most
notably GAMESS-UK.  An input line is parsed into whitespace (blanks or
tabs) separating tokens or fields.  Any token that contains whitespace
must be enclosed in double quotes in order to be processed correctly.
For example, the basis set with the descriptive name \verb+modified
Dunning DZ+ must appear in a directive as \verb+"modified Dunning
DZ"+, since the name consists of three separate words.  

A (physical) line in the input file is terminated with a newline
character (also known as a 'return' or 'enter' character).  A
semicolon (\verb+;+) can be also used to indicate the end of an input
line, allowing a single physical line of input to contain multiple
logical lines of input.  For example, five lines of input for the
\verb+GEOMETRY+ directive can be entered as follows;
\begin{verbatim}
  geometry
   O 0  0     0
   H 0  1.430 1.107
   H 0 -1.430 1.107
  end
\end{verbatim}
These same five lines could be entered on a single line, as
\begin{verbatim}
  geometry; O 0 0 0; H 0 1.430 1.107; H 0 -1.430 1.107; end
\end{verbatim}
This one physical input line comprises five logical
input lines.  Each logical or physical input line must be no longer
than 1023 characters.  Data is read until an end-of-file is detected,
or until an \verb+EOF+ directive is encountered.

In the input file:
\begin{itemize}
\item a string, token, or field is a sequence of ASCII characters
  (NOTE: if the string includes blanks or tabs (i.e., white space),
  the entire string must be enclosed in double quotes).
\item \verb+\+ (backslash) at the end of a line concatenates it with
  the next line.  Note that a space character is automatically
  inserted at this point so that it is {\em not} possible to split
  tokens across lines.  A backslash is also used to quote special
  characters such as whitespace, semi-colons, and hash symbols so as
  to avoid their special meaning (NOTE: these special symbols must be
  quoted with the backslash even when enclosed within double quotes).
\item \verb+;+ (semicolon) is used to mark the end of a logical input
  line within a physical line of input.
\item \verb+#+ (the hash or pound symbol) is the comment character.
  All characters following \verb+#+ (up to the end of the physical
  line) are ignored.
\end{itemize}

Directives consist of a directive name, keywords, and optional input,
and may contain one line or many.  Simple directives consist of a
single line of input with one or more fields.  Compound directives can
have multiple input lines, and can also include other optional simple
and compound directives.  A compound directive is terminated with an
END directive.  The directives START (see Section \ref{sec:start}) and
ECHO (see Section \ref{sec:echo}) are examples of simple directives.
The directive GEOMETRY (see Section \ref{sec:geom}) is an example of a
compound directive.

Some limited checking of the input for self-consistency is performed
by the input module, but most defaults are imposed by the application
modules at runtime.  It is therefore usually impossible to determine
beforehand whether or not all selected options are consistent with
each other.

\sloppy

In the rest of this document, the following notation and syntax
conventions are used in the generic descriptions of the NWChem input.
\begin{itemize}
\item a directive name always appears in all-capitals, and in computer
  typeface (e.g., \verb+GEOMETRY+, \verb+BASIS+, \verb+SCF+).  Note
  that the case of directives and keywords is ignored in the actual
  input.
\item a keyword always appears in lower case, in computer typeface
  (e.g., {\tt swap}, {\tt print}, {\tt units}, {\tt bqbq}).
\item variable names always appear in lower case, in computer
  typeface, and enclosed in angle brackets to distinguish them from
  keywords (e.g., {\tt <input\_filename>}, {\tt <basisname>}, {\tt
    <tag>}).
\item \verb+$variable$+ is used to indicate the substitution of the
  value of a variable.
\item \verb+()+ is used to group items (the parentheses and other
  special symbols should not appear in the input).
\item \verb+||+ separate exclusive options, parameters, or formats.
\item \verb+[ ]+ enclose optional entries that have a default value.
\item \verb+< >+ enclose a type, a name of a value to be specified, or
  a default value, if any.
\item \verb+\+ is used to concatenate lines in a description.
\item \verb+...+ is used to indicate indefinite continuation of a
  list.
\end{itemize}

\fussy

An input parameter is identified in the description of the directive
by prefacing the name of the item with the type of data expected,
i.e.,
\begin{itemize}
\item \verb+string +  -- an ASCII character string
\item \verb+integer+ --  integer value(s) for a variable or an array
\item \verb+logical+ --  true/false logical variable
\item \verb+real   +  -- real floating point value(s) for a variable or an array
\item \verb+double + -- synonymous with real
\end{itemize}

If an input item is not prefaced by one of these type names,
it is assumed to be of type ``string''.

In addition, integer lists may be specified using Fortran triplet
notation, which interprets \verb+lo:hi:inc+ as \verb+lo+, \verb=lo+inc=,
\verb=lo+2*inc=, \ldots, \verb+hi+.  For example, where a list of
integers is expected in the input, the following two lines are
equivalent
\begin{verbatim}
   7 10 21:27:2 1:3 99
   7 10 21 23 25 27 1 2 3 99
\end{verbatim}
(In Fortran triplet notation,  the increment, if unstated, is 1; e.g., 1:3 = 1:3:1.)
 
The directive \verb+VECTORS+ (Section \ref{sec:vectors}) is presented here
as an example of an NWChem input directive.  The general form of the
directive is as follows:
\begin{verbatim}
  VECTORS [input (<string input_movecs default atomic>) || \
                   (project <string basisname> <string filename>)] \
          [swap [(alpha||beta)] <integer vec1 vec2> ...] \
          [output <string output_movecs default $file_prefix$.movecs>]
\end{verbatim}

This directive contains three optional keywords, as indicated by the 
three main sets of square brackets enclosing the keywords \verb+input+,
\verb+swap+, and \verb+output+.  The keyword \verb+input+ allows the
user to specify the source of the molecular orbital vectors.  
There are two mutually exclusive options for
specifying the vectors, as indicated by the \verb+||+ symbol
separating the option descriptions;
\begin{verbatim}
  (<string input_movecs default atomic>) || \
                  (project <string basisname> <string filename>) \
\end{verbatim}

The first option, \verb+(<string input_movecs default atomic>)+,
allows the user to specify an ASCII character string for the parameter
{\tt input\_movecs}.  If no entry is specified, the code uses the
default \verb+atomic+ (i.e., atomic guess).  The second option,
{\tt(project <string basisname> <string filename>)}, contains the
keyword \verb+project+, which takes two string arguments.  When this
keyword is used, the vectors in file \verb+<filename>+ will be
projected from the (smaller) basis \verb+<basisname>+ into the current
atomic orbital (AO) basis.

The second keyword, \verb+swap+, allows the user to re-order the
starting vectors, specifying the pairs of vectors to be swapped.  As
many pairs as the user wishes to have swapped can be listed for {\tt
  <integer vec1 vec2 ... >}.  The optional keywords \verb+alpha+ and
\verb+beta+ allow the user to swap the alpha or beta spin orbitals.

The third keyword, \verb+output+, allows the user to tell the code
where to store the vectors, by specifying an ASCII string for the
parameter {\tt output\_movecs}.  If no entry is specified for this
parameter, the default is to write the vectors back into either the
user- specified MO vectors input file or, if this is not available,
the file \verb+$file_prefix$.movecs+.

A particular example of the \verb+VECTORS+ directive is shown below.
It specifies both the \verb+input+ and \verb+output+ keywords, but
does not use the \verb+swap+ option.
\begin{verbatim}
  vectors input project "small basis" small_basis.movecs \
          output large_basis.movecs
\end{verbatim}
This directive tells the code to generate input vectors by projecting
from vectors in a smaller basis named \verb+"small basis"+, which is
stored in the file \verb+small_basis.movecs+.  The output vectors will
be stored in the file \verb+large_basis.movecs+.

The order of keyed optional entries within a directive should not
matter, unless noted otherwise in the specific instructions for a
particular directive.



\chapter{NWChem Architecture}
%
% $Id$
%
\label{sec:arch}

As noted above, NWChem consists of independent modules that perform
the various functions of the code.  Examples include the input parser,
self-consistent field (SCF) energy, SCF analytic gradient, and density
functional theory (DFT) energy modules.  The independent NWChem
modules can share data only through a disk-resident database, which is
similar to the GAMESS-UK dumpfile or the Gaussian checkpoint file.
This allows the modules to share data, or to share
access to files containing data.

It is not necessary for the user to be intimately familiar with the
contents of the database in order to run NWChem.  However, a nodding
acquaintance with the design of the code will help in clarifying the
logic behind the input requirements, especially when restarting jobs
or performing multiple tasks within one job.  Section
\ref{sec:database} gives a general description of the database.

As described above (Section \ref{sec:inputstructure}), all start-up
directives are processed at the beginning of the job by the main
program, and then the input module is invoked.  Each input directive
usually results in one or more entries being made in the database.
When a \verb+TASK+ directive is encountered, control is passed to the
appropriate module, which extracts relevant data from the database and
any associated files.  Upon completion of the task, the module will
store significant results in the database, and may also modify other
database entries in order to affect the behavior of subsequent
computations.

\section{Database Structure}
\label{sec:database}

\sloppy

Data is shared between modules of NWChem by means of the database.
Three main types of information are stored in the data base: (1)
arrays of data, (2) names of files that contain data, and (3) objects.
Arrays are stored directly in the database, and contain the following
information:
\begin{enumerate}
\item the name of the array, which is a string of ASCII characters
  (e.g., \verb+"reference energies"+)
\item the type of the data in the array (i.e., real, integer, logical,
  or character)
\item the number (N) of data items in the array (Note: A scalar is
  stored as an array of unit length.)
\item the N items of data of the specified type
\end{enumerate}

\fussy

It is possible to enter data directly into the database using the
\verb+SET+ directive (see Section \ref{sec:set}).  For example, to
store a (64-bit precision) three-element real array with the name
\verb+"reference energies"+ in the database, the directive is as
follows:
\begin{verbatim}
  set "reference energies" 0.0 1.0 -76.2
\end{verbatim}
NWChem determines the data to be real (based on the type of the first
element, \verb+0.0+), counts the number of elements in the array, and
enters the array into the database.

Much of the data stored in the database is internally managed by
NWChem and should not be modified by the user.  However, other data,
including some NWChem input options, can be freely modified.

Objects are built in the database by storing associated data as
multiple entries, using an internally consistent naming convention.
This data is managed exclusively by the subroutines (or methods) that
are associated with the object.  Currently, the code has two main
objects: basis sets and geometries.  Sections \ref{sec:geom} and
\ref{sec:basis} present a complete discussion of the input to describe
these objects.

As an illustration of what comprises a geometry object, the following
table contains a partial listing of the database contents for a water
molecule geometry named \verb+"test geom"+.  Each entry contains the
field \verb+test geom+, which is the unique name of the object.

\begin{verbatim}
 Contents of RTDB h2o.db
 -----------------------

 Entry                                   Type[nelem]
 ---------------------------  ----------------------
 geometry:test geom:efield             double[3]    
 geometry:test geom:coords             double[9]    
 geometry:test geom:ncenter               int[1]    
 geometry:test geom:charges            double[3]    
 geometry:test geom:tags                 char[6]
 ...
\end{verbatim}

Using this convention, multiple instances of objects may be stored
with different names in the same database.  For example, if a user
needed to do calculations considering alternative geometries for the
water molecule, an input file could be constructed containing all the
geometries of interest by storing them in the database under different
names.

The runtime database contents for the file \verb+h2o.db+ listed above
were generated from the user-specified input directive,
\begin{verbatim}
  geometry "test geom"
    O     0.00000000    0.00000000    0.00000000
    H     0.00000000    1.43042809   -1.10715266
    H     0.00000000   -1.43042809   -1.10715266
  end
\end{verbatim}
The \verb+GEOMETRY+ directive allows the user to specify the
coordinates of the atoms (or centers), and identify the geometry with
a unique name.  (Refer to Section \ref{sec:geom} for a complete
description of the {\tt GEOMETRY} directive.)

Unless a specific name is defined for the geometry, (such as the name
\verb+"test geom"+ shown in the example), the default name of
\verb+geometry+ is assigned.  This is the geometry name that
computational modules will look for when executing a calculation.  The
{\tt SET} directive can be used in the input to force NWChem to look
for a geometry with a name other than \verb+geometry+.  For example,
to specify use of the geometry with the name \verb+"test geom"+ in the
example above, the \verb+SET+ directive is as follows:

\begin{verbatim}
  set geometry "test geom"
\end{verbatim}

NWChem will automatically check for such indirections when loading
geometries.  Storage of data associated with basis sets, the other
database resident object, functions in a similar fashion, using the
default name \verb+"ao basis"+.

\section{Persistence of data and restart}
\label{sec:persist}

The database is persistent, meaning that all input data and output
data (calculation results) that are not destroyed in the course of
execution are permanently stored.  These data are therefore available
to subsequent tasks or jobs.  This makes the input for restart jobs
very simple, since only new or changed data must be provided.  It also
makes the behavior of successive restart jobs {\em identical} to that
of multiple tasks within one job.

Sometimes, however, this persistence is undesirable, and it is
necessary to return an NWChem module to its default behavior by
restoring the database to its input-free state.  In such a case, the
\verb+UNSET+ directive (see Section \ref{sec:unset}) can be used to
delete all database entries associated with a given module (including
both inputs and outputs).



\chapter{Functionality}
\label{sec:functionality}

\subsection{Molecular electronic structure}

The following methods are available with energies and analytic first
derivatives w.r.t. atomic coordinates.  Second derivatives are
computed by finite difference of first derivatives.
\begin{itemize}
\item Self Consistent Field or Hartree Fock (RHF, UHF, high-spin
  ROHF).  Analytic second derivatives in alpha testing.  
\item Gaussian Density Functional Theory (DFT) with many local and
  non-local exchange-correlation potentials.
\item MP2 semi-direct with frozen core with RHF and UHF reference.
\item Coupled-cluster single and double excitations (CCSD) with RHF
  reference.
\item Complete active space SCF (CASSCF).
\end{itemize}

The following methods are available with energies only.  First and
second derivatives are computed by finite difference of the energy.
\begin{itemize}
\item MP3, MP4, CCSD(T) (RHF reference).
\item Selected-CI with second-order perturbation correction.
\item MP2 fully-direct (RHF reference).
\item Resolution of the identity integral approximation MP2 (RI-MP2)
  (RHF and UHF reference).
\end{itemize}

For all methods, the following operations may be performed
\begin{itemize}
\item Numerical first and second derivatives automatically computed if
  analytic derivatives are not available.
\item Geometry Optimization (Minimization and Transition State).
\item Generation of the electron density file for the {\em Insight}
      graphical program.
\item Interface to 
\item 
\end{itemize}

In addition, automatic interfaces are provided to
\begin{itemize}
\item The COLUMBUS multi-reference CI package.
\item The natural bond orbital package.
\end{itemize}

\subsection{Periodic system electronic structure}

Currently available are energies with Gaussian Density Functional
Theory (DFT) with many local and non-local exchange-correlation
potentials.

\subsection{Molecular dynamics}

The following functionality is available for classical molecular
simulations
\begin{itemize}
\item Single configuration energy evaluation
\item Energy minimization
\item Molecular dynamics simulation
\item Free energy simulation 
\end{itemize}




\chapter{Top-level directives}
\label{sec:toplevel}

Top-level directives are directives that can affect all modules in the
code.  Some specify molecular properties (e.g., total charge) or other
data that should apply to all subsequent calculations with the current
database.  However, most top-level directives provide the user with
the means to manage the resources for a calculation and to start
computations.  As the first step in the execution of a job, NWChem
scans the entire input file looking for start-up directives, which
NWChem must process before all other input.  The input file is then
rewound and processed sequentially, and each directive is processed in
the order in which it is encountered.  In this second pass, start-up
directives are ignored.

The following sections describe each of the top-level directives in
detail, noting all keywords, options, required input, and defaults.

\section{START and RESTART --- Start-up mode}
\label{sec:start}

A {\tt START} or {\tt RESTART} directive is
optional.  If one of these two directives is not specified
explicitly, the code will infer one, based upon the name of the
input file and the availability of the database.  When allowing NWChem
to infer the start-up directive, the user must be quite certain that
the contents of the database will result in the desired action.  It
is usually more prudent to specify the directive explicitly, using the
following format:

\begin{verbatim}
(RESTART || START)   \
            [<string file_prefix default $input_file_prefix$>] \
            [rtdb <string rtdb_file_name default $file_prefix$.db>]
\end{verbatim}

The \verb+START+ directive indicates that the calculation is one in
which a new database is to be created.  Any relevant information that
already exists in a previous database of the same name is destroyed.
The string variable {\tt <file\_prefix>} will be used as the prefix to
name any files created in the course of the calculation.  

E.g., to start a new calculation on water, one might specify
\begin{verbatim}
  start water
\end{verbatim}
which would result in all files beginning with {\tt "water."}.

If the user does not specify an entry for {\tt <file\_prefix>} on the
\verb+START+ directive (or omits the \verb+START+ directive
altogether), the code uses the base-name of the input file as the file
prefix.  That is, the variable {\tt <file\_prefix>} is assigned the
name of the input file (not its full pathname), but without the last
"dot-suffix".  For example, the input file name
\verb+/home/dave/job.2.nw+ yields \verb+job.2+ as the file prefix, if
a name is not assigned explicitly using the \verb+START+ directive.

The user also has the option of
specifying a unique name for the database, using the keyword {\tt
  rtdb}.  When this keyword is entered, the string entered for {\tt
  rtdb\_file\_name} is used as the database name.  If the keyword {\tt
  rtbd} is omitted, the name of the database defaults to
\verb+$<file_prefix>$.db+ in the directory for permanent files.

If a calculation is to start from a previous calculation and go on
using the existing database, the \verb+RESTART+ directive 
must be used.  In such a case, the previous
database must already exist.  The name specified for {\tt <file\_prefix>} 
usually should
not be changed when restarting a calculation.  If it is changed, NWChem 
will not
be able to find needed files when going on with the
calculation.

In the most common
situation, the previous calculation was completed (with or without an error
condition), and it is desired to perform a new task or restart the
previous one, perhaps with some input changes.  In these instances,
the \verb+RESTART+ directive should be used.  This reuses the previous
database and associated files, and reads the input file for new input
and task information.

The \verb+RESTART+ directive looks immediately for new input and task
information, deleting information about previous incomplete tasks.

To summarize the default options for this start-up directive, if the 
input file does {\em not} contain a \verb+START+ or a
\verb+RESTART+ directive, then
\begin{itemize}
  \item the variable {\tt <file\_prefix>} is assigned the name of the 
input file for the job, without the suffix (which is usually \verb+.nw+)
  \item the variable {\tt <rtdb\_file\_name>} is assigned the default name,
\verb+$file_prefix$.db+
\end{itemize}
If the database with name \verb+$file_prefix$.db+ does {\em not} 
already exist,
the calculation is carried out as if a \verb+START+ directive had
been encountered.  If the database with name \verb+$file_prefix$.db+
{\em does} exist, then the calculation is performed as if a
\verb+RESTART+ directive had been encountered.  

For example, NWChem can be run using an input file with the name 
\verb+water.nw+ 
by typing the UNIX command line,

\begin{verbatim}
   nwchem water.nw
\end{verbatim}

If the NWChem input file \verb+water.nw+ does not contain
a \verb+START+  or \verb+RESTART+ directive, the code
sets the variable {\tt <file\_prefix>} to {\tt water}.  Files created
by the job will have this prefix, and the database will be named
{\tt water.db}.  If the database \verb+water.db+ does {\em not} exist already,
the code behaves as if the input file contains the directive,

\begin{verbatim}
  start water
\end{verbatim}

If the database \verb+water.db+ {\em does} exist,
the code behaves as if the input file contained the directive,

\begin{verbatim}
  restart water
\end{verbatim}


\section{SCRATCH\_DIR and PERMANENT\_DIR --- File directories}
\label{sec:dirs}

These are start-up directives that allow the user to specify the
directory location of scratch and permanent files created by NWChem.
NWChem distinguishes between permanent (or persistent) files and
scratch (or temporary) files, and allows the user the option of
putting them in different locations.  In most installations, however,
permanent and scratch files are all written to the current directory
by default.  What constitutes "local" disk space may also differ from 
machine to machine.

The conventions for file storage are at the discretion of the specific 
installation, and are quite likely to be different on different machines.  
When assigning locations for permanent and
scratch files,
the user must be cognizant of the characteristics of the installation
on a particular platform.
To consider just a few examples, on the IBM SP 
and workstation clusters, machine-specific or process-specific
names must be supplied for both local and shared file
systems, while on the KSR it is useful to specify scratch file directories
with automated striping across processors with round-robin allocation.
On SMP clusters, both of these specifications are required.  

The \verb+SCRATCH_DIR+ and \verb+PERMANENT_DIR+ directives are
identical in format and capability, and enable the user to specify a
single directory for all processes, or different directories for
different processes.  The general form of the directive is as follows:

\begin{verbatim}
   (PERMANENT_DIR || SCRATCH_DIR) [(<string host>||<integer process>):] \
                                       <string directory> \ 
                                  [...]
\end{verbatim}

Directories are extracted from the user input by executing the
following steps, in sequence:
\begin{enumerate}
\item Look for a directory qualified by the process ID number of the
  invoking process.  Processes are numbered from zero.  Else,
\item If there is a list of directories qualified by the name of the
  host machine\footnote{As returned by {\tt util\_hostname()} which
    maps to the output of the command {\tt hostname} on Unix
    workstations.}, then use round-robin allocation from the list for
  processes executing on the given host.  Else, 
\item If there is a list of directories unqualified by any hostname
  or process ID, then use round-robin allocation from this list.
\end{enumerate}
If directory allocation directive(s) are not specified in the input
file, or if no match is found to the directory names specified by
input using these directives, then the  steps above are executed using
the installation-specific defaults.  If the code cannot find a valid
directory name based on the input specified in either the directive(s)
or the system defaults, files are automatically written to the current
working directory (\verb+"."+).

The following is a list of examples of specific allocations of scratch
directory locations:
\begin{itemize}
\item Put scratch files from all processes in the local scratch directory 
(Warning: the definition of ``local scratch directory'' may change from 
machine to machine):
\begin{verbatim}
      scratch_dir /localscratch
\end{verbatim}
\item Put scratch files from Process 0 in \verb+/piofs/rjh+, but put all 
other scratch files in \verb+/scratch+:
\begin{verbatim}
      scratch_dir /scratch 0:/piofs/rjh
\end{verbatim}
\item Put scratch files from Process 0 in directory \verb+scr1+, those from
Process 1 in \verb+scr2+, and so forth, in a round-robin fashion, using the
given list of directories:
\begin{verbatim}
      scratch_dir /scr1 /scr2 /scr3 /scr4 /scr5
\end{verbatim}
\item Allocate files in a round-robin fashion from
  host-specific lists for processes distributed across two
 SGI multi-processor machines (node names {\em coho} and {\em bohr}):
\begin{verbatim}
      scratch_dir coho:/xfs1/rjh coho:/xfs2/rjh coho:/xfs3/rjh \
          bohr:/disk01/rjh bohr:/disk02/rjh bohr:/disk13/rjh
\end{verbatim}
\end{itemize}

\section{MEMORY --- Control of memory limits}

This is a start-up directive that allows the user to specify the
amount of memory that NWChem can use for the job.  If this directive
is not specified, memory is allocated according to
installation-dependent defaults.  {\em The defaults should generally
  suffice for most calculations, since the defaults usually correspond
  to the total amount of memory available on the machine.  It should
usually be unnecessary to provide a memory directive!!!} 

The general form of the directive is as follows:

\begin{verbatim}
  MEMORY [[total] <integer total_size>] \
         [stack <integer stack_size>] \
         [heap <integer heap_size>] \
         [global <integer global_size>] \
         [units <string units default real>] \
         [(verify||noverify)] \
         [(nohardfail||hardfail)] \
\end{verbatim}

NWChem recognizes the following memory units:
\begin{itemize}
\item \verb+real+ and \verb+double+ (synonyms)
\item \verb+integer+
\item \verb+byte+
\item \verb+kb+ (kilobytes)
\item \verb+mb+ (megabytes)
\item \verb+mw+ (megawords, 64-bit word)
\end{itemize}

In most cases, the user need specify only the total memory limit to 
adjust the amount of memory used by NWChem. The following specifications 
all provide for eight megabytes of total
memory (assuming 64-bit floating point numbers), which will be
distributed according to the default partitioning:
\begin{verbatim}
  memory 1048576
  memory 1048576 real
  memory 1 mw
  memory 8 mb
  memory total 8 mb
  memory total 1048576
\end{verbatim}

In NWChem there are three distinct regions of memory: stack, heap, 
and global. Stack and heap are node-private, while the union of the 
global region on all processors is used to provide globally-shared memory.  
The allowed limits on each category are determined from a default 
partitioning (currently 25% heap, 25% stack, and 50% global).
Alternatively, the keywords \verb+stack+, \verb+heap+, and
\verb+global+ can be used to define specific allocations for each of
these categories.  If the user sets only one of the stack, heap, or
global limits by input, the limits for the other two categories are
obtained by partitioning the remainder of the total memory available
in proportion to the weight of those two categories in the default
memory partitioning.  If two of the category limits are given, the
third is obtained by subtracting the two given limits from the total
limit (which may have been specified or may be a default value).  If
all three category limits are specified, they determine the total
memory allocated.  However, if the total memory is also specified, it
must be larger than the sum of all three categories.  The code will
abort if it detects an inconsistent memory specification.

The following memory directives also allocate 8 megabytes, but specify
a complete partitioning as well:

\begin{verbatim}
  memory total 8 stack 2 heap 2 global 4 mb
  memory stack 2 heap 2 global 4 mb
\end{verbatim}

The optional keywords \verb+verify+ and \verb+noverify+ in the
directive give the user the option of enabling or disabling automatic
detection of corruption of allocated memory.  The default is
\verb+verify+, which enables the feature. This incurs a small
overhead, which can be eliminated by specifying \verb+noverify+.

The keywords \verb+hardfail+ and \verb+nohardfail+ give the user the
option of forcing (or not forcing) the local memory management
routines to generate an internal fatal error if any memory operation
fails.  The default is \verb+nohardfail+, which allows the code to
continue past any memory operation failure, and perhaps generate a
more meaningful error message before terminating the calculation.
Forcing a hard-fail can be useful when poorly coded applications do
not check the return status of memory management routines.

When assigning the specific memory allocations using the keywords
\verb+stack+, \verb+heap+, and \verb+global+ in the \verb+MEMORY+
directive, the user should be aware that some of the distinctions
among these categories of memory have been blurred in their actual
implementation in the code.  The memory allocator (MA) allocates both
the heap and the stack from a single memory region of size {\tt
  heap+stack}, without enforcing the partition.  The heap vs. stack
partition is meaningful only to applications developers, and can be
ignored by most users.  Further complicating matters, the global array
(GA) toolkit is allocated from within the MA space on distributed
memory machines, while on shared-memory machines it is
separate\footnote{This is because on true shared-memory machines there
  is no choice but to allocate GAs from within a shared-memory
  segment, which is managed differently by the operating system.}.

On distributed memory platforms, the MA region is actually the total
size of 
\begin{verbatim}
   stack+heap+global
\end{verbatim}
All three types of memory allocation
compete for the same pool of memory, with no limits except on the
total available memory.  This relaxation of the memory category
definitions usually benefits the user, since it can allow allocation
requests to succeed where a stricter memory model would cause the
directive to fail.  These implementation characteristics must be kept
in mind when reading program output that relates to memory usage.

Standard defaults for various platforms are listed in Table
\ref{tbl:default-memory-limits}, though these are commonly 
overriden during installation at many sites.

%RJH: Table reference needs fixing.  Should be Table 5.1, not 5.3.--fmr

\begin{table}

\center

\label{tbl:default-memory-limits}

\begin{tabular}{lr}
\hline\hline
Platform        & Total Memory Limit (MBytes) \\
\hline
CRAY-T3E        & 83 \\
DECOSF          & 200 \\
IBM RS/6000     & 200 \\
IBM SP-X        & 90 \\
Linux           & 90 \\
SGI             & 200 \\
SGI Power Challenge  & 200 \\
Sun             & 200 \\
\hline\hline
\end{tabular}

\caption{Default total memory limits according to hardware platform.}


\end{table}

\section{ECHO --- Print input file}
\label{sec:echo}

This start-up directive is provided as a convenient way to include a
listing of the input file in the output of a calculation.  It causes
the entire input file to be printed to Fortran unit six (standard
output).  It has no keywords, arguments, or options, and consists of
the single line:

\begin{verbatim}
  ECHO
\end{verbatim}

The \verb+ECHO+ directive is processed only
once, by Process 0 when the input file is read.

\section{TITLE --- Specify job title}

This top-level directive allows the user to identify a job or series
of jobs that use a particular database.  It is an optional directive,
and if omitted, the character string containing the input title will
be empty.  Multiple {\tt TITLE} directives may appear in the input
file (e.g., the example file in Section \ref{sec:realsample}) in which
case a task will use the one most recently specified.  The format for
the directive is as follows:

\begin{verbatim}
  TITLE <string title>
\end{verbatim}

The character string \verb+<title>+ is assigned to the contents of the
string following the \verb+TITLE+ directive.  If the string contains
white space, it must be surrounded by double quotes.  For example,

\begin{verbatim}
  title "This is the title of my NWChem job"
\end{verbatim}

The title is stored in the database and will be used in all subsequent
tasks/jobs until redefined in the input.

\section{PRINT and NOPRINT --- Print control}
\label{sec:printcontrol}

The \verb+PRINT+ and \verb+NOPRINT+ directives allow the user to
control how much output NWChem generates.  These two directives are
special in that the compound directives for {\em all} modules are
supposed to recognize them. Each module can control both the overall
print level (general verbosity) and the printing of individual items
which are identified by name (see below).  The standard form of the
\verb+PRINT+ directive is as follows:

\begin{verbatim}
  PRINT [(none || low || medium || high || debug) default medium] \
        [<string list_of_names ... >]

  NOPRINT <string list_of_names ... >
\end{verbatim}
The default print level is medium.

Every output that is printed by NWChem has a print threshold
associated with it. If this threshold is equal to or lower than the
print level requested by the user, then the output is generated.  For
example, the threshold for printing the SCF energy at convergence is
\verb+low+ (Section \ref{sec:scfprint}).  This means that if the
user-specified print level on the \verb+PRINT+ directive is
\verb+low+, \verb+medium+, \verb+high+, or \verb+debug+, then the SCF
energy will be printed at convergence.

The overall print level specified
using the \verb+PRINT+ directive is a convenient tool for controlling 
the verbosity
of NWChem. Setting the print level to \verb+high+ might be helpful in
diagnosing convergence problems.  The print level of \verb+debug+ might
also be of use in evaluating problem cases, but the user should be aware
that this can generate a huge amount of output.  Setting the print level
to \verb+low+ might be the preferable choice for geometry
optimizations that will perform many steps which are in themselves of
little interest to the user.

In addition, it is possible to enable the printing of specific
items by naming them in the \verb+PRINT+ directive in the 
\verb+<list_of_names>+.  Items identified in this way will be printed, 
regardless of the overall print level specified.  Similarly, the 
\verb+NOPRINT+ directive can be used to suppress the printing of specific
items by naming them in its \verb+<list_of_names>+.  These items will
{\em not} be printed, regardless of the overall print level, or the 
specific print level of the individual items.

The list of items that can be printed for each module is documented 
as part of the input instructions for that module.
The items recognized by the top level of the code, and their thresholds, 
are:
\begin{table}[htbp]
\begin{center}
\begin{tabular}{lcc}
  {\bf Name}          & {\bf Print Level} & {\bf Description} \\
 ``total time''        & medium & Print cpu and wall time at job end\\
 ``task time''         & high   & Print cpu and wall time for each task\\
 ``rtdb''              & high    & Print names of RTDB entries\\
 ``rtdbvalues''        & high    & Print name and values of RTDB entries\\
 ``ga summary''        & medium & Summarize GA allocations at job end \\
 ``ga stats''          & high   & Print GA usage statistics at job end \\
 ``ma summary''        & medium & Summarize MA allocations at job end \\
 ``ma stats''          & high   & Print MA usage statistics at job end \\
 ``version''           & debug  & Print version number of all compiled routines \\
  ``tcgmsg''           & never  & Print TCGMSG debug information \\
\end{tabular}
\end{center}
\caption{Top Level Print Control Specifications}
\end{table}


The following example shows how a \verb+PRINT+ directive for the top level
process can be used to limit printout to only essential information.
The directive is

\begin{verbatim}
  print none "ma stats" rtdb
\end{verbatim}

This directive instructs the NWChem main program to print nothing,
except for the memory usage statistics (\verb+ma stats+) and
the names of all items stored in the database at the end of the job.

The print level within a module is inherited from the 
calling layer.  For instance, by specifying the print to be low
within the MP2 module will cause the SCF, CPHF and gradient modules
when invoked from the MP2 to default to low print.  Explicit user
input of print thresholds overrides the inherited value.

\section{SET --- Enter data in the RTDB}
\label{sec:set}

This top-level directive allows the user to enter data directly into the
database (see Section \ref{sec:database} for a description of the database).
The format of the directive is as follows:

\begin{verbatim}
  SET <string name> [<string type default automatic>] <$type$ data>
\end{verbatim}

The entry for variable \verb+<name>+ is the name of 
data to be entered into the database.  This must be specified; there is no default.  The variable \verb+<type>+, which is
optional, allows the user to define a string specifying the type of
data in the array \verb+<name>+.  The data type can be explicitly
specified as \verb+integer+, \verb+real+, \verb+double+,
\verb+logical+, or \verb+string+.  If no entry for \verb+<type>+ is
specified on the directive, its value is inferred from the data type
of the {\em first} datum.  In such a case, floating-point data
entered using this directive must include either an exponent or a
decimal point, to ensure that the correct default type will be
inferred.  The correct default type will be inferred for logical
values if logical-true values are specified as \verb+.true.+,
\verb+true+, or \verb+t+, and logical-false values are specified as
\verb+.false.+, \verb+false+, or \verb+f+.  One exeception to the
automatic detection of the data type is that the data type {\bf must}
be explicitly stated to input integer ranges, unless the first
element in the list is an integer that is not a range (c.f.,
\ref{sec:syntax}).  For example,
\begin{verbatim}
  set atomid 1 3:7 21
\end{verbatim}
will be interpreted as a list of integers.  However, 
\begin{verbatim}
  set atomid 3:7 21
\end{verbatim}
will not work since the first element will be interpreted as a
string and not an integer.  To work around this feature, use instead
\begin{verbatim}
  set atomid integer 3:7 21
\end{verbatim}


The \verb+SET+ directive is useful for providing indirection by
associating the name of a basis set or geometry with the standard
object names (such as \verb+"ao basis"+ or \verb+geometry+) used by
NWChem.  The following input file shows an example using the
\verb+SET+ directive to direct different tasks to different
geometries.  The required input lines are as follows:

\begin{verbatim}
  title "Ar dimer BSSE corrected MP2 interaction energy"
  geometry "Ar+Ar"
    Ar1 0 0 0
    Ar2 0 0 2
  end

  geometry "Ar+ghost"
    Ar1 0 0 0
    Bq2 0 0 2
  end

  basis
    Ar1 library aug-cc-pvdz
    Ar2 library aug-cc-pvdz
    Bq2 library Ar aug-cc-pvdz
  end

  set geometry "Ar+Ar"
  task mp2 

  scf; vectors atomic; end

  set geometry "Ar+ghost"
  task mp2 
\end{verbatim}

This input tells the code to perform MP2 energy calculations 
on an argon dimer in the first task, and then
on the argon atom in the presence of the ``ghost'' basis of the other
atom.

The \verb+SET+ directive can also be used as an indirect means of
supplying input to a part of the code that does not have a separate
input module (e.g., the atomic SCF, Section \ref{sec:atomscf}).
Additional examples of applications of this directive can be found in
the sample input files (see Section \ref{sec:realsample}), and
its usage with basis sets (Section \ref{sec:basis}) and geometries
(Section \ref{sec:geom}).

\section{UNSET --- Delete data in the RTDB}
\label{sec:unset}

This directive gives the user a way to delete simple entries from the
database.  The general form of the directive is as follows:

\begin{verbatim}
  UNSET <string name>[*]
\end{verbatim}

This directive cannot be used with complex objects such as geometries
and basis sets\footnote{Complex objects are stored using a structured
  naming convention that is not matched by a simple wild card.}.  A
wild-card (*) specified at the end of the string \verb+<name>+ will
cause {\em all} entries whose name begins with that string to be
deleted.  This is very useful as a way to reset modules to their
default behavior, since modules typically store information in the
database with names that begin with \verb+module:+.  For example, the
SCF program can be restored to its default behavior by deleting all
database entries beginning with \verb+scf:+, using the directive

\begin{verbatim}
  unset scf:*
\end{verbatim}

The following example makes an entry in the database using the
\verb+SET+ directive, and then immediately deletes it using the
\verb+UNSET+ directive:

\begin{verbatim}
  set mylist 1 2 3 4
  unset mylist
\end{verbatim}


\section{STOP --- Terminate processing}

This top-level directive provides a convenient way of verifying 
an input file without actually running the calculation.  It consists 
of the single line,

\begin{verbatim}
  STOP
\end{verbatim}

As soon as this directive is encountered, all processing ceases and
the calculation terminates with an error condition.

\section{TASK --- Perform a task}
\label{sec:task}

The \verb+TASK+ directive is used to tell the code what to do.  The
input directives are parsed sequentially until a \verb+TASK+ directive
is encountered, as described in Section \ref{sec:inputstructure}.  At
that point, the calculation or operation specified in the \verb+TASK+
directive is performed.  When that task is completed, the code looks
for additional input to process until the next \verb+TASK+ directive
is encountered, which is then executed.  This process continues to the
end of the input file.  NWChem expects the last directive before the
end-of-file to be a \verb+TASK+ directive.  If it is not, a warning
message is printed.  Since the database is persistent, multiple tasks
within one job behave {\em exactly} the same as multiple restart jobs
with the same sequence of input.

There are four main forms of the the \verb+TASK+ directive.  The most
common form is used to tell the code at what level of theory to
perform an electronic structure calculation, and which specific
calculations to perform.  The second form is used to specify tasks
that do not involve electronic structure calculations or tasks that
have not been fully implemented at all theory levels in NWChem, such
as simple property evaluations.  The third form is used to execute
UNIX commands on machines having a Bourne shell.  The fourth form is
specific to combined quantum-mechanics and molecular-mechanics (QM/MM)
calculations.

By default, the program terminates when a task does not complete
successfully.  The keyword \verb+ignore+ can be used to prevent this
termination, and is recognized by all forms of the \verb+TASK+
directive.  When a \verb+TASK+ directive includes the keyword
\verb+ignore+, a warning message is printed if the task fails, and
code execution continues with the next task.

The input options, keywords, and defaults for each of these four forms
for the \verb+TASK+ directive are discussed in the following sections.

\subsection{TASK Directive for Electronic Structure Calculations}
\label{sec:first_task}

This is the most commonly used version of the \verb+TASK+ directive, and
it has the following form:

\begin{verbatim}
  TASK <string theory> [<string operation default energy>] [ignore]
\end{verbatim}

The string \verb+<theory>+ specifies the level of theory to be used in the
calculations for this task.  NWChem currently supports ten different
options.  These are listed below, with the corresponding entry for 
the variable {\tt <theory>}:
\begin{itemize}
 \item \verb+scf+ --- Hartree-Fock
 \item \verb+dft+ --- Density functional theory for molecules
 \item \verb+sodft+ --- Spin-Orbit Density functional theory
 \item \verb+gapss+ --- Density functional theory for periodic systems
 \item \verb+mp2+ --- MP2 using a semi-direct algorithm
 \item \verb+direct_mp2+ --- MP2 using a full-direct algorithm
 \item \verb+rimp2+ --- MP2 using the RI approximation
 \item \verb+ccsd+ --- Coupled-cluster single and double excitations
 \item \verb+mcscf+ --- Multiconfiguration SCF
 \item \verb+selci+ --- Selected configuration interaction with perturbation
   correction 
 \item \verb+md+ --- Classical molecular dynamics simulation using nwARGOS
 \item \verb+pspw+ --- Pseudopotential plane-wave density functional theory for molecules and insulating solids using NWPW
 \item \verb+band+ --- Pseudopotential plane-wave density functional theory for solids using NWPW
%% \item \verb+md_ideaz+ --- Classical molecular dynamics simulation
%%   using IDEAZ
\end{itemize}

The string \verb+<operation>+ specifies the calculation that will
be performed in the task.  The default operation is a single point energy
evaluation.  The following list gives the selection of operations currently
available in NWChem:
\begin{itemize}
\item \verb+energy+ --- Evaluate the single point energy.
\item \verb+gradient+ --- Evaluate the derivative of the energy with respect to\
   nuclear coordinates.
\item \verb+optimize+ --- Minimize the energy by varying the molecular
   structure.  By default, this geometry optimization is presently driven by the Driver
   module (see Section \ref{sec:driver}), but the Stepper module
   (see Section \ref{sec:stepper}) may also be used.
\item \verb+saddle+ --- Conduct a search for a transition state (or saddle point) 
  using either Driver (Section \ref{sec:driver}, the default) or
  Stepper (Section \ref{sec:stepper}).
\item \verb+frequencies+ or \verb+freq+ --- Compute second derivatives 
and print out an analysis of molecular vibrations.
\item \verb+dynamics+ --- Compute molecular dynamics using nwARGOS.
\item \verb+thermodynamics+ --- Perform multi-con\-fig\-ura\-tion
  thermo\-dynamic integ\-ration using nwARGOS.
\end{itemize}


The user should be aware that some of these operations (gradient,
optimize, dynamics, thermodynamics) require computation of
derivatives of the energy with respect to the molecular coordinates.
If analytical derivatives are not available (Section
\ref{sec:functionality}), they must be computed numerically, which can
be very computationally intensive.

Here are some examples of the \verb+TASK+ directive, to illustrate the
input needed to specify particular calculations with the code.  To
perform a single point energy evaluation using any level of theory, the
directive is very simple, since the energy evaluation is the default
for the string \verb+operation+.  For an SCF energy calculation, the
input line is simply
\begin{verbatim}
  task scf
\end{verbatim}
Equivalently, the operation can be specified explicitly, using the
directive
\begin{verbatim}
  task scf energy
\end{verbatim}

Similarly, to perform a geometry optimization using density functional
theory, the \verb+TASK+ directive is
\begin{verbatim}
  task dft optimize
\end{verbatim}

The optional keyword \verb+ignore+ can be used to allow execution to
continue even if the task fails, as discussed above.

\subsection{TASK Directive for Special Operations}

This form of the \verb+TASK+ directive is used in instances where the
task to be performed does not fit the model of the previous version
(such as execution of a Python program, Section \ref{sec:python}), or
if the operation has not yet been implemented in a fashion that
applies to a wide range of theories (e.g., property evaluation).
Instead of requiring \verb+theory+ and \verb+operation+ as input, the
directive needs only a string identifying the task.  The form of the
directive in such cases is as follows:

\begin{verbatim}
  TASK <string task> [ignore]
\end{verbatim}

The supported tasks that can be accessed with this form of the \verb+TASK+
directive are listed
below, with the corresponding entries for string variable \verb+<task>+.

\begin{itemize}
  \item \verb+python+ --- Execute a Python program (Section \ref{sec:python}).
  \item \verb+rtdbprint+ --- Print the contents of the database.
  \item \verb+cphf+ --- Invoke the CPHF module.
  \item \verb+property+ --- Perform miscellaneous property calculations.
  \item \verb+dplot+ --- Execute a DLOT run (Section \ref{sec:dplot})
  \item \verb+nbo+ --- Execute a NBO run (Section \ref{sec:nbo})
\end{itemize}

This directive also recognizes the keyword \verb+ignore+, which allows
execution to continue after a task has failed.

\subsection{TASK Directive for the Bourne Shell}

This form of the \verb+TASK+ directive is supported only on machines
with a fully UNIX-style operating system.  This directive causes
specified processes to be executed using the Bourne shell.  This form
of the task directive is:

\begin{verbatim}
  TASK shell [(<integer-range process = 0>||all)] \
             <string command> 
\end{verbatim}

The keyword \verb+shell+ is required for this directive.  It specifies
that the given command will be executed in the Bourne shell.  The user
can also specify which process(es) will execute this command by
entering values for \verb+process+ on the directive.  The default is
for only process zero to execute the command.  A range of processes
may be specified, using Fortran triplet notation\footnote{The notation
{\tt lo:hi:inc} denotes the integers {\tt lo}, {\tt lo+inc},
{\tt lo+2*inc}, \ldots, {\tt hi}}.  Alternatively, all
processes can be specified simply by entering the keyword \verb+all+.
The input entered for \verb+command+ must form a single string, and
must consist of valid UNIX command(s).  If the string includes white space,
it must be enclosed in double quotes.

For example, the \verb+TASK+ directive to tell process zero to copy the 
molecular orbitals file to a backup location \verb+/piofs/save+ can be input as follows:

\begin{verbatim}
  task shell "cp *.movecs /piofs/save"
\end{verbatim}

The \verb+TASK+ directive to tell all processes to list the contents of 
their \verb+/scratch+ directories is as follows:

\begin{verbatim}
  task shell all "ls -l /scratch"
\end{verbatim}

The \verb+TASK+ directive to tell processes 0 to 10 to remove the 
contents of the current directory is as follows:

\begin{verbatim}
  task shell 0:10:1 "/bin/rm -f *"
\end{verbatim}

Note that NWChem's ability to quote special input characters is {\em
  very} limited when compared with that of the Bourne shell.  To
execute all but the simplest UNIX commands, it is usually much easier
to put the shell script in a file and execute the file from within
NWChem.

\subsection{TASK Directive for QM/MM simulations}

This is very similar to the most commonly used version of the
\verb+TASK+ directive described in Section \ref{sec:first_task}, and
it has the following form;

\begin{verbatim}
  TASK QMMM <string theory> [<string operation default energy>] [ignore]
\end{verbatim}

The string \verb+<theory>+ specifies the QM theory to be used in the
QM/MM simulation\footnote{If theory is ``{\tt md}'' this is not a QM/MM
simulation and will result in an appropriate error}.  The level of
theory may be any QM method that can compute gradients but those
algorithms in NWChem that do not support analytic gradients should be
avoided (c.f., Section \ref{sec:functionality}).  

The string \verb+<operation>+ is used to specify the calculation that will
be performed in the QM/MM task.  The default operation is a single point energy
evaluation.  The following list gives the selection of operations currently
available in the NWChem QM/MM module;
\begin{itemize}
\item \verb+energy+ --- single point energy evaluation
\item \verb+optimize+ --- minimize the energy by variation of the molecular
   structure.  
\item \verb+dynamics+ --- molecular dynamics using nwARGOS
\end{itemize}

Here are some examples of the \verb+TASK+ directive for QM/MM
simulations.  To perform a single point energy of a QM/MM system using
any QM level of theory, the directive is very simple. As with the
general task directive, the QM/MM energy evaluation is the
default. For a DFT energy calculation the task directive input is,
\begin{verbatim}
  task qmmm dft
\end{verbatim}
or completely as
\begin{verbatim}
  task qmmm dft energy
\end{verbatim}

To do a molecular dynamics simulation of a QM/MM system using the SCF
level of theory the task directive input would be
\begin{verbatim}
  task qmmm scf dynamics
\end{verbatim}

The optional keyword \verb+ignore+ can be used to allow execution to
continue even if the task fails, as discussed above.

\section{CHARGE --- Total system charge}
\label{sec:charge}

This is an optional top-level directive that allows the user to specify
the total charge of the system.  The form of the directive is as follows:
\begin{verbatim}
  CHARGE <real charge default 0>
\end{verbatim}

The default charge\footnote{The charge directive, in conjunction with
  the charges of atomic nuclei (which can be changed via the geometry
  input, cf. Section \ref{sec:cart}), determines the total number of
  electrons in the chemical system.  Therefore, a {\tt charge n}
  specification removes "n" electrons from the chemical system.
  Similarly, {\tt charge -n} adds "n" electrons.} is zero
if this directive is omitted.  An example of a case where the
directive would be needed is for a calculation on a doubly charged
cation.  In such a case, the directive is simply,
\begin{verbatim}
  charge 2
\end{verbatim}

If centers with fractional charge have been specified (Section
\ref{sec:geom}) the net charge of the system should be adjusted to
ensure that there are an integral number of electrons.

The charge may be changed between tasks, and is used by all
wavefunction types.  For instance, in order to compute the first two
vertical ionization energies of $LiH$, one might optimize the geometry
of $LiH$ using a UHF SCF wavefunction, and then perform energy
calculations at the optimized geometry on $LiH^+$ and
$LiH^{2+}$ in turn.  This is accomplished with the following input:
\begin{verbatim}
  geometry; Li 0 0 0; H  0 0 1.64; end
  basis; Li library 3-21g; H library 3-21g; end

  scf; uhf; singlet; end
  task scf optimize

  charge 1
  scf; uhf; doublet; end
  task scf

  charge 2
  scf; uhf; singlet; end
  task scf
\end{verbatim}
The \verb+GEOMETRY+, \verb+BASIS+, and \verb+SCF+ directives are
described below (Sections \ref{sec:geom}, \ref{sec:basis} and
\ref{sec:scf} respectively) but their intent should be clear.  The
\verb+TASK+ directive is described above (Section \ref{sec:task}).  




\chapter{Geometries}
\label{sec:geom}

\verb+******************************************************************+
WARNING -- I have substantially re-written this section, and may have
guessed wrong on the more obscure points.  Someone (preferably the original
author), should read this draft and correct the more egregious 
errors. -- J.M. Cuta
\verb+******************************************************************+

The \verb+GEOMETRY+ directive is used to define specific geometry objects
for a calculation or series of calculations.  The input for this directive
allows the user to define the geometry of the molecule(s) to be considered
in the calculation(s).  NWChem already has a fairly good idea of what most
atoms look like and the sort of molecular structures in which they might be
arranged.  This information is stored in the basis sets, which can be specified
for a given task using the \verb+BASIS+ directive (see Section 
\ref{sec:basis}).  The \verb+GEOMETRY+ directive allows the user to specify
particular elements of the geometry objects, such as the coordinates of 
the individual atoms in a molecule, and the appropriate symmetry group.
It is also possible to define multiple geometry objects for the same
molecule under different names, for use in different tasks of the same 
calculation.

The form of the \verb+GEOMETRY+ directive is as follows;

\begin{verbatim}
  GEOMETRY [<string name default geometry>] [zmt]\
           [units <string units default bohr>] \
           [bqbq] \
           [print [xyz] || noprint]
    
    [SYMMETRY GROUP <string group_name> [print]]

    <string tag> <real x> <real y> <real z> \
        [charge <real charge>] [mass <real mass>] [ghost]
    ...

    <string tag> < list_of_Z-matrix_variales> [ghost]
    ...

    <string tag> <list_of_frozen_variables>
    ...

  END
\end{verbatim}

% Cartesian geometry specification, as well as Z-matrix-like format is
% available.  As mentioned above (section
% \ref{sec:arch}), multiple geometries may be stored in the database
% provided each is given an independent name.  The default name of
% \verb+"geometry"+ is used by most application modules to access the
% geometry at which to perform a calculation.  Associating a named
% geometry with the required name of \verb+"geometry"+ is described in
% section \ref{sec:arch}.  Known names for units are \verb+au+,
% \verb+bohr+, or \verb+angstrom+ (the conversion factor used to convert
% from Angstr\"{o}m to Bohr is $1.8897265$). By default, the input
% module prints any geometry it encounters.  Printing can be disabled
% with the \verb+PRINT+ option.  The \verb+XYZ+ qualifier to print
% causes the geometry to also be printed in the \verb+XYZ+ format of
% XMol.  

This is a multiple directive and may consist of many or few lines,
depending on the application.  The input supplied with this directive
consist of three main sections; 

\begin{itemize}
\item basic information to identify the geometry object in the code and
define how it will be processed,
\item input to identify the symmetry group for the molecule,
\item input to specify the geometric locations of the atoms in the molecule
\end{itemize}

The input options and keywords for each of these sections are
discussed in detail in the following subsections.  

\subsection{Geometry Object Identification Input}

The input for the first four lines of the directive allows the user to name 
the object, specify the geometry input format, decide how to treat dummy
centers in the geometry, and tell the code what to do about printing 
the geometry information.

The default in the code is to specify the geometry using
Cartesian coordinates, but the user has the option of specifying
the geometry for some (or all) of the atoms using a Z-matrix-like format
(see Section \ref{sec:zmat}).
This option is flagged by specifying the keyword \verb+zmat+ on the
first line of the directive.  If this keyword is omitted, the geometry
input must be specified using Cartesian coordinates (see section 
\ref{sec:cart}).

The default \verb+name+ for a geometry object is the string \verb+geometry+,
and most modules in the code look for an object with this name.  
The user can redirect the module to a different geometry object by assigning
the string \verb+geomery+ to \verb+name+ using the \verb+SET+ directive 
(see Section \ref{sec:set}).  The default for the geometry input unit
is \verb+bohr+, and this is the working unit for such distances in the code.
However, the geometric coordinates can be supplied in \verb+au+ or 
\verb+angstrom+ by specifying the appropriate keyword on the directive.
(Note: The conversion factor used in the code to convert from Angstr\"{o}m 
to Bohr is $1.8897265$.)

The default in NWChem is to ignore interactions between dummy centers. 
Specifying the keyword \verb+bqbq+ forces the code to include these 
interactions.  The complimentary keywords \verb+print+ and \verb+noprint+
allow the user to specify the print option for the geometry module, 
independent of any specifications in the top-level \verb+PRINT+ directive.
The \verb+print+ keyword tells the code to print everything for this
module, regardless of other instructions.  In addition, the keyword
\verb+xyz+ specifies that the coordinates will be printed in the XYZ
format of XMol.  If the keyword \verb+noprint+ is specified, the printing
of all geometry output for this module is suppressed.

\subsection{Symmetry Group Input}

The input following the keyword \verb+SYMMETRY GROUP+ is used to specify
the symmetry for the molecule modeled with this geometry object.
There is no default for this input in the current version of NWChem, since
the use of molecular symmetry is not yet automated in the code.
Examples of expected input for the string \verb+group_name+ include
such entries as

\begin{itemize}
\item \verb+c2v+ -- C\_{2v}
\item \verb+d2h+ -- D\_{2h}
\item \verb+Td+ -- T\_{d}
\item \verb+d6h+ -- D\_{6h}
\end{itemize}

The user must know the symmetry of the molecule being modeled, and be able
to specify the
coordinates of the symmetry-unique atoms in a suitable orientation
relative to the rotation axes and symmetry planes.  
Appendix \ref{symexamples} lists a number of examples of the \verb+geometry+
directive input for specific molecules having symmetry patterns recognized
by NWChem.

\subsection{Cartesian coordinate input}
\label{sec:cart}

The default in NWChem is to specify the geometry information in Cartesian 
coordinates.  Each atom of the molecule being modeled by the geometry
object must be specified on a line in the directive; that is, on a line of
the form,

\begin{verbatim}

    <string tag> <real x> <real y> <real z> \
        [charge <real charge>] [mass <real mass>] [ghost]

\end{verbatim}

The string \verb+tag+ is the name of the atom or center.  I is limited to
16 characters, and must correspond to the
name of an atom in one of the basis sets defined for the calculation.
Atoms or centers with the same \verb+tag+ will use the same basis set in
the calculation. 
(See Section \ref{sec:basis} for a discussion of the input for the
\verb+BASIS+ directive).  In most cases, the entry for \verb+tag+ is the
chemical symbol for the element, such as \verb+O+ for oxygen, \verb+H+
for hydrogen, \verb+Fe+ for iron, etc.  Dummy centers 
are differentiated from atoms and centers by giving them a \verb+tag+ that 
begins with the letters \verb+bq+.  An atom must have basis functions
associated with it, and so must all centers.  If the keyword \verb+bqbq+
is specified, dummy centers must also have basis functions.

\verb+************************+
Shall we tell the users what a dummy center is, and the difference
between a 'center' and an atom?  Or do we assume they were all born
knowing this?  Also might want to explain 'ghost' atoms here...
\verb+************************+

The xyz coordinates of the atom in the molecule, relative to some origin
(***shall we tell them where it is?***) are specified as real numbers 
following the string \verb+tag+.  The user also has the option of 
specifying the charge of the atom (or center) and its mass.  

The default charge for an
atom is its atomic number, adjusted for the presence of ECPs (see Section
\ref{sec:ecpbasis}).  In order to specify a different value for the
charge on a particular atom, the user must enter
the keyword \verb+charge+, then enter a real number for the unit charge
of the atom in \verb+charge+.  The values specified for the charges of 
all atoms, centers, and
dummy centers are used in conjunction with the 
total charge of the system (as specified with the \verb+CHARGE+ directive; 
see Section 
\ref{sec:charge}) to determine the total number of electrons in the system.

The default mass for an atom is taken to be the mass of its highest naturally
occuring isotope.  If the user wishes to model some other isotope of the
element, its mass must be defined explicitly by specifying the keyword 
\verb+mass+ and entering the appropriate value for \verb+mass+.  (***shall
we tell them the units to use for mass?  Or is this something everybody
already knows?***)


%  Each line in the body of the directive specifies the name or tag, and
% the coordinates of one center or atom.  The charge associated with the
% center is inferred from the atom type.  The charge may be explicitly
% specified using the \verb+CHARGE+ keyword.  Default masses may be
% overriden by specifying the mass.  The default masses are those of the
% highest naturally abundant isotope for the given atom, not the average
% mass of the element.  
% 
% The tag associated with each center is interpreted as follows:
% \begin{itemize}
% \item If it begins with \verb+BQ+ (ignoring case) then it is treated
%       as a dummy center with default zero charge. Dummy centers may 
%       optionally have basis functions or non-zero charge.
% \item If it begins with either the symbol or name of an element then
%       it is thought to be an atom.  Atoms {\em must} have basis
%       functions associated with them and the default charge is the
%       atomic number adjusted for the presence of ECPs (see
%       \ref{sec:ecp}).  The user provided charges (of all centers,
%       atomic and dummy) and the total charge of the system are used to
%       determine the number of electrons
% \item The tag of a center is used in the \verb+BASIS+ directive to
%       associate functions with centers.  All centers with the same tag
%       will have the same basis functions.  Atomic centers may have
%       standard basis sets sited upon them.
% \item When automatic symmetry detection is functional only centers
%       with the same tag will be candidates for testing for symmetry
%       equivalence.
% \end{itemize}
% 
% By default NWCHEM does not include the interaction between dummy
% centers.  The \verb+BQBQ+ qualifier to the \verb+GEOMETRY+ directive
% causes these interactions to be included.
% 
%  The use of molecular symmetry in NWCHEM is not yet automated, thus,
% the user is responsible for detecting symmetry and specifying the
% coordinates of the symmetry unique atoms in a suitable orientation
% relative to the rotation axes and symmetry planes.  Since it seems
% that only the original authors of the symmetry package seem to
% understand the latter, we provide many examples in Appendix
% \ref{symexamples}.

\subsection{Z-matrix input}
\label{sec:Z-matrix}

Specifying the keyword \verb+zmat+ on the \verb+GEOMETRY+ directive allows
the user to specify the structure of the molecule by means of its Z-matrix.
It is also possible to enter the coordinates of some of the atoms using
cartesian coordinates, however.  The code is able to distinguish between
the two types of input by means of the input following the \verb+tag+ string
for the particular atom.  If the Z-matrix input is specified for the atom,
the input consists of pairs numbers that define connectivity indices and 
bond length, bond angle, or torsion angle.  If the
input is in Cartesian coordinates, the input consists of three real numbers
defining the x,y,z coordinates of the atom.  However, the x,y,z coordinates
must be specified in {\AA}ngstr{\o}ms, regardless of the entry specified
for /verb+units+.

The keyword \verb+ghost+ is used to denote a 'ghost' atom in the molecular
geometry.  A 'ghost' atom is like an ordinary atom in that it has its 
associated basis set, but it is assumed to have a nuclear charge of zero.
This is a useful property for such things as calculations to assess basis set
superposition error.  (***How does this differ from setting the \verb+charge+
to zero in the Cartesian coordinates input option?***)

% If the \verb+GEOMETRY+ directive contains the \verb+zmat+ keyword, the
% structure of the molecule is defined by means of its Z-matrix, or
% the cartesian coordinates of the atoms, or a mixture of the two. A
% blank line terminates the list of the atoms.  Bond lengths,  bond
% angles, or torsion angles can be specified by means of numerical
% values assigned to variables. The list of variables and  their
% assigned numerical  values  follow  the  list  of atoms, and is also
% terminated with a blank line.  Then comes the end of the data group
% input with a last line 'end'
% 
% A second set of variables to be qualified as '{\bf frozen}' may be
% specified.  They come  after  the  blank  line  that indicates the end
% of the 'variables'. The second set is also terminated by a blank line.
% {\bf The 'frozen' specification is presently not active, but included
%   to ensure compatibility with other codes. Only cartesian coordinates
%   can be kept constant via the 'active atoms' list (see section
%   \ref{sec:activeatoms})}
% 
% 'Ghost' atoms, often used in the assessment of basis set superposition
% error for  example,  are  atoms  with their associated basis set, but
% for which the nuclear charge is set to zero. Specifying a 'ghost' atom
% may be  accomplished appending the '{\tt ghost}' parameter at the end of the
% appropriate lines below.
%
% When  two  numerical  values,  separated by a comma, are given for some
% variables, they are considered as the initial and final values for the
% definition of a Linearized Synchronous Transit pathway.  The
% geometries generated by linear interpolation between the initial and
% final values.

The input supplied for the atoms describing the molecule by means of its
Z-matrix requires the following sequential approach.

\verb+*****************+
Questions we might want to answer at this point;
\begin{itemize}
\item Is there a particular starting point for the Z-matrix of a molecule,
or can the user pick any old atom?
\item Is there a required order in which the atoms of a molecule must be
entered?
\item What about molecules with multiple bonds?
\end{itemize}
\verb+****************+

\begin{enumerate}

   \item $<$atom$>$ [ghost]

%    Only  the  name  of the first atom is required. 'ghost' is used
%    only when specifying a 'ghost' atom.
    For the first atom in the molecule, the Z-matrix input requires only  
    the name of the atom at it appears in the basis set.  This is usually 
    the chemical symbol for the atom.
    However, if the atom is a 'dummy center', the chemical symbol must
    be preceeded by \verb+x+ or \verb+bq+. (The keyword 'ghost' is required
    only when specifying that the atom is to be defined as a 'ghost' atom.)

   \item $<$atom$>$ $<$i1$>$ $<$blength$>$ [ghost]

%   Only a name and a bond distance is required for atom 2.  For 'ghost',
%   same remark applies as before.
    For the second atom, the Z-matrix input requires the name of the atom
    as it appears in the basis set, 
    plus the connectivity <$i1$> and the bond distance <$blength$> 
    connecting it to another atom. 

    (NOTE: The connectivity for any bond, {\tt i1, i2, i3} ldots, 
    can be specified as an integer (1, 2, 3, ldots) or as a string 
    which matches the name specified for one of the atoms in the molecule.
    If the connectivity (such as {\tt i1}) is defined as a string,
    and the same element appears more than once in the molecule, a number
    must be added to the {\tt atom} string of the Z-matrix input for
    subsequent appearances of that element in the molecule, to ensure 
    a unique \verb+tag+ for each occcurance.)

   \item $<$atom$>$ $<$i1$>$ $<$blength$>$ $<$i2$>$ $<$alpha$>$ [ghost]

%    Only  a  name, distance, and angle are required for atom 3.  For
%    'ghost', same remark applies as before.
    For the third atom of the molecule, the Z-matrix input requires the
    name of the atom as it appears in the basis set and the connectivity 
    <$i1$> and bond distance 
    <$blength$> connecting it to atom 2, plus the connectivity <$i2$>
    and angle <$alpha$> it makes with the plane of the first two atoms. 

   \item $<$atom$>$ $<$i1$>$ $<$blength$>$ $<$i2$>$ $<$alpha$>$ $<$i3$>$ $<$beta$>$ $<$i4$>$ [ghost] %

    For the fourth atom, and all subsequent atoms in the molecule (if it
    has more than four atoms), the Z-matrix input requires the name 
    of the atom as it appears in the basis set,
    plus the connectivities and bond length,
    bond angle, and either the dihedral angle or a second bond angle
    for the atom.
        
    \begin{itemize}
%    \item [$\bullet$]  {\tt atom} is the chemical symbol of this
%        atom;  it  can  be followed  
%        by numbers if desired. The chemical symbol implies the nuclear
%        charge.  
    \item [$\bullet$]  {\tt i1} defines the connectivity of the following bond.  
%    \item [$\bullet$]  {\tt blength} is the bond length 'this atom-atom i1'.
    \item [$\bullet$]    {\tt blength} is the bond length between this atom and the atom denoted by {\tt i1}. 
    \item [$\bullet$]  {\tt i2} defines the connectivity of the following angle.  
%    \item [$\bullet$]  {\tt alpha} is the angle 'this atom-atom i1-atom i2'.
    \item [$\bullet$]  {\tt alpha} is the angle that the bond with this atom makes with the plane of the bond between atom {\tt i1} and atom {\tt i2}.  
    \item [$\bullet$]  {\tt i3} defines the connectivity of the following angle.
%    \item [$\bullet$]  {\tt beta}  is  either  the dihedral angle
%'this atom-atom {\tt i1}-atom {\tt i2}-atom 
%        {\tt i3}', or perhaps a second bond angle, 'this atom-atom i1-atom {\t%t i3}'
    \item [$\bullet$]  {\tt beta}  is  the dihedral angle between
this atom and the plane containing atom {\tt i1} and atom {\tt i2} and atom 
        {\tt i3}'; alternatively, it defines a second bond angle for this atom, between this atom and the plane of the two atoms atom {\tt i1} and atom {\tt i3}.  Which alternative is selected depends on the value specified for {\tt i4}, as explained below. 
%    \item [$\bullet$]  {\tt i4} defines the nature of {\tt beta}.  If
%{\tt beta} is a dihedral
%        angle,  {\tt i4}=0  ( default ).  If {\tt beta} is a second bond angle,
%        then {\tt i4}=+/-1 
%        (sign specifies one of two possible directions).  
%    \item [$\bullet$]  For 'ghost', same remark applies as before.
    \item [$\bullet$]  {\tt i4} defines the nature of {\tt beta}.  If {\tt i4} is zero, 
{\tt beta} is interpreted as a dihedral
        angle.  (This is the default.)  If {\tt i4} is entered as +1 or -1, {\tt beta} is interpreted as a second bond angle.  (The sign of {\tt i4} specifies the direction of the bond angle ***relative to what?***).
    \end{itemize}

%  \item Line no.\ 4  is repeated for each remaining atom. A blank line
%        indicates the end of the atoms in the molecule.

%  \item The use of 'dummy' atoms is possible, by using 'X' or 'BQ' for  the
%chemical symbol.
%
%  \item The connectivity i1, i2, i3, may be given as integers,
% 1, 2, 3, 4, 5, \ldots or as strings which match one of the {\tt atom}s.
% In this case, numbers must be added to the {\tt atom} string to
% ensure uniqueness.
%
%   \item Symbolic strings may be given in place of  numeric  values  for
%{\tt blength}, {\tt alpha},  {\tt beta}.  The  same  string may be repeated.  Any mixture of
%    numeric data and symbols may be given.
% 
%   \item All symbolic definitions follow the blank line that signals  the
%    end  of the Z-matrix input. The list of symbolic definitions ends with a
%    blank line, followed by an '{\tt end}' directive. If there are no symbolic
%    definitions, and  all  the  bond  lengths, angles, and torsions are
%    specified by their numeric values in the Z-matrix data, then the
%    end of the Z-matrix data is detected through two blank lines, one
%    to indicate the end of the Z-matrix proper, the other to indicate
%    the end of the symbolic definitions.
%
%   \item A second set of  symbolic  definitions,  separated  from  the
%    first  set through  a blank line, may be included to specify
%    {\bf frozen} internal 
%    coordinates. A blank line again  defines  the  end  of  the
%    list  of  'frozen' internal coordinates.
%
%   \item Note that atoms in the Z-matrix data may be specified via Cartesian
%    coordinates expressed in units of {\AA}ngstr{\o}ms, Each line has the
%    following form:
%
%     \begin{verbatim}
%       <atom> <x> <y> <z> [ghost]
%     \end{verbatim}
%
%    with  {\tt atom}  being  the atomic name as before, x, y, z being the
%    Cartesian coordinates, and {\tt ghost} used only when specifying a
%    'ghost' atom.
\end{enumerate}

The Z-matrix variables {\tt blength}, {\tt alpha}, and {\tt beta} can be
entered either as numeric values or symbolic strings, or as a mixture of
the two types for a given atom.   The end of the Z-matrix input for the 
molecule is signalled by a blank line in the directive, and all symbolic 
definitions for the variables must be supplied following that blank line.
The list of symbolic definitions is terminated with another blank line.
If there are no symbolic definitions for the Z-matrix input (that is,  
all of the bond lengths, angles, and torsions are specified by their 
numeric values), then the end of the Z-matrix data is signalled by entering
two blank lines in the directive; one to indicate the end of the Z-matrix 
input, and the other to indicate the end of the symbolic definitions.

The \verb+GEOMETRY+ directive also contains a feature that allows the user
to define symbolic definitions for a set of variables that will be 
qualified as '{\bf frozen}'.  A {\bf frozen} atom is one that will not
have any of its geometric parameters changed in the course of an optimization
or other calculation that can change the geometry of a molecule.  
The ability to actually use this feature
for atoms defined by means of the Z-matrix input does not exist in NWChem
as yet, but the input option has been included to ensure compatibilty with
other codes.  Only atoms that have been specified using Cartesian coordinates
can retain their geometry parameters at the specified input values.  This
is accomplished by means of the optional 'active atoms' list (refer to Section
\ref{activeatoms}).

If the user wishes to specify a set of {\bf frozen} variables, however,
% (we won't ask why...)
they must be specified following the blank line that signals the end of 
the symbolic definitions for the Z-matrix input.  The end of the input
for the {\bf frozen} variables is also signaled with a blank line.

The following example illustrates the Z-matrix input for the molecule
$CH_3CF_3$.  This input uses integer numbers for the connectivities {\tt i1},
{\tt i2}, {\tt i3} \ldots, but has symbolic strings as entries for the Z-matrix
variables {\tt blength}, {\tt alpha}, and {\tt beta}.  It therefore must
also include input lines defining the symbolic strings, following the blank
line that terminates the Z-matrix input.  This example also includes a set
of {\bf frozen} variables, which are entered after the blank line terminating
the definitions of the symbolic strings used by the Z-matrix.


% \subsubsection*{example}

% The following example is the z-matrix for $CH_3CF_3$,

The \verb+GEOMETRY+ directive for this example is as follows;

\begin{verbatim}
 geometry zmat
   C 
   C 1 CC 
   H 1 CH1 2 HCH1 
   H 1 CH2 2 HCH2 3 TOR1 +1 
   H 1 CH3 2 HCH3 3 TOR2 -1 
   F 2 CF1 1 CCF1 3 TOR3  0 
   F 2 CF2 1 CCF2 6 FCH1  1 
   F 2 CF3 1 CCF3 6 FCH2 -1

   CC   = 1.4888 
   CH1  = 1.0790 
   CH2  = 1.0789  
   CH3  = 1.0789  
   CF1  = 1.3667 
   CF2  = 1.3669 
   CF3  = 1.3669

   HCH1 = 10428 
   HCH2 = 10474 
   HCH3 = 1047 
   CCF1 = 112.0713 
   CCF2 = 112.0341 
   CCF3 = 112.0340 
   TOR1 = 109.3996 
   TOR2 = 109.3997 
   TOR3 = 180.0000 
   FCH1 = 106.7846 
   FCH2 = 106.7842

 end   
\end{verbatim}

%   The  separation  of the symbolic definitions into two groups is what
%   makes the internal coordinates of the second group 'frozen'.
%   Removal  of  the blank  line between the first and second set of
%   symbolic definitions would remove the 'frozen' character of all the
%   variables defined in  the  second group.

In this example, the symbolic strings listed after the first blank line
(which signals the end of the Z-matrix input for the molecule) are defined
with the values listed ({\tt CC= 1.4888}, {\tt CH1 = 1.0790}, etc.).  The
second blank line forces the symbolic definitions starting with {\tt HCH1
= 10428} to be {\bf frozen}.
\verb+***************************+
What does the code actually do with this input, since the {\bf frozen}
feature is not active?  Does it go ahead and use the specified values
for {\tt HCH1} through {\tt FCH2}, and change them as required by the
calculation?  Or are those strings defined some other way, and the
{\bf frozen} assignments ignored?
\verb+***************************+


\chapter{Basis sets}
%
% $Id: basis.tex,v 1.30 2004-05-17 20:29:46 pollack Exp $
%
\label{sec:basis} 

NWChem currently supports basis sets consisting of generally
contracted\footnote{Generally contracted meaning that the same
  primitive, Gaussian functions are contracted into multiple
  contracted functions using different contraction coefficients.
  Reuse of the radial functions increases the efficiency of integral
  generation.} Cartesian Gaussian functions up to a maximum angular
 momentum of six ($h$ functions), and also $sp$ (or L)
functions\footnote{An $sp$ shell is two-component general contraction.
  However, the first component specifies an $s$ shell and the second a
  $p$ shell.  Again, reuse of the radial functions increases the efficiency
  of integral generation.} .  The {\tt BASIS} directive is used to
define these, and also to specify use of an effective core potential
(ECP) that is associated with a basis set; see Section \ref{sec:ecp}.

The basis functions to be used for a given calculation can be drawn
from a standard set in the EMSL basis set library that is included in
the release of NWChem  (See Appendix \ref{sec:knownbasis} for a list
of the standard basis sets currently supplied with the release of the
code).  Alternatively, the user can specify particular functions
explicitly in the input, to define a particular basis set.

The general form of the \verb+BASIS+ directive is as follows:

\begin{verbatim}
  BASIS [<string name default "ao basis">] \
        [(spherical || cartesian) default cartesian] \
        [(segment || nosegment) default segment] \
        [(print || noprint) default print]
        [rel]

     <string tag> library [<string tag_in_lib>] \
                  <string standard_set> [file <filename>] \
                  [except <string tag list>] [rel]

        ...

     <string tag> <string shell_type> [rel]
        <real exponent> <real list_of_coefficients>
        ...
     
  END
\end{verbatim}    

Examining the keywords on the first line of the \verb+BASIS+ directive:


\begin{itemize}
\item {\tt name}

  By default, the basis set is stored in the database with the name
  \verb+"ao basis"+.  Another name may be specified in the \verb+BASIS+
  directive, thus, multiple basis sets may be stored simultaneously in the
  database.  Also, the DFT (Section \ref{sec:dft}) 
  and RI-MP2 (Section \ref{sec:rimp2}) modules and the
  Dyall-modified-Dirac relativistic method (Section \ref{sec:dyall-mod-dir})
  require multiple basis sets with specific names.

The user can associate the \verb+"ao basis"+ with another named basis
using the \verb+SET+ directive (see Section \ref{sec:set}).  

\item {{\tt SPHERICAL} or {\tt CARTESIAN}}

The keywords \verb+spherical+ and \verb+cartesian+ offer the option of
using either spherical-harmonic (5 d, 7 f, 9 g, \ldots) or Cartesian
(6 d, 10 f, 15 g, \ldots) angular functions.  The default is
Cartesian.  

Note that the correlation-consistent basis sets were designed using
spherical harmonics and to use these, the \verb+spherical+ keyword
should be present in the \verb+BASIS+ directive.  The use of spherical
functions also helps eliminate problems with linear dependence.


\item {{\tt SEGMENT} or {\tt NOSEGMENT}}

By default, NWChem forces all basis sets to be segmented, 
even if they are input with general contractions or $L$ or sp
shells. This is because the current derivative integral program cannot
handle general contractions.  If a calculation is  
computing energies only, a 
performance gain can result from exploiting generally contracted basis
sets, in which case {\tt NOSEGMENT} should be specified.

\item {{\tt PRINT} or {\tt NOPRINT}}

The default is for the input module to print all basis sets encountered.
Specifying the keyword \verb+noprint+ allows the user to suppress this output.

\item {{\tt REL}}

This keyword marks the entire basis as a relativistic basis for the purposes
of the Dyall-modified-Dirac relativistic integral code. The marking of the
basis set is necessary for the code to make the proper association between
the relativistic shells in the ao basis and the shells in the large and/or
small component basis. This is only necessary for basis sets which are to be
used as the ao basis. The user is referred to Section \ref{sec:dyall-mod-dir}  
for more details.

\end{itemize}

Basis sets are associated with centers by using the tag of a center in
a geometry that has either been input by the user (Section
\ref{sec:geom}) or is available elsewhere.  Each atom or center with
the same \verb+tag+ will have the same basis set.  All atoms must have
basis functions assigned to them --- only dummy centers (X or Bq) may have no
basis functions.  To facilitate the specification of the geometry and
the basis set for any chemical system, the matching process of a basis
set tag to a geometry tag first looks for an exact match.  If no match
is found, NWChem will attempt to match, ignoring case, the name or
symbol of the element.  E.g., all hydrogen atoms in a system could be
labeled ``H1'', ``H2'', \ldots, in the geometry but only
one basis set specification for ``H'' or ``hydrogen'' is necessary.
If desired, a special basis may be added to one or more centers (e.g.,
``H1'') by providing a basis for that tag.
If the matching mechanism fails then NWChem stops with an appropriate
error message.

A special set of tags, ``*'' and tags ending with a ``*'' (E.g. ``H*'')
can be used in combination with the keyword \verb+library+ (see section
below). These tags facilitate the definition of a certain type of basis 
set of all atoms, or a group of atoms, in a geometry using only a single 
%<<<<<<< basis.tex
or very few basis set entries. The ``*'' tag will not place basis sets 
on dummy atoms, Bq* can be used for that if necessary.
%=======
%or very few basis set entries. The ``*'' tag will not place basis sets 
%on dummy atoms, Bq* can be used for that if necessary. 
%>>>>>>> 1.25

Examined next is how to reference standard basis sets in the basis set
library, and finally, how to define a basis set using exponents and
coefficients.

\section{Basis set library}

The keyword \verb+library+ associated with each specific \verb+tag+
entry specifies that the calculation will use the standard basis set
in NWChem for that center.  The variable \verb+<standard_set>+ is the
name that identifies the functions in the library.  The names of
standard basis sets are not case sensitive.  See Appendix
\ref{sec:knownbasis} for a complete list of the basis sets in the
NWChem library and their specifications.  

The general form of the input line requesting basis sets from the NWChem
basis set library is:
\begin{verbatim}
     <string tag> library [<string tag_in_lib>] \
                  <string standard set> [file < filename> \
                  [except <string tag list>] [rel]
        ...
\end{verbatim}

For example, the NWChem basis set library contains the Dunning cc-pvdz
basis set.  These may be used as follows
\begin{verbatim}
  basis
    oxygen library cc-pvdz
    hydrogen library cc-pvdz
  end
\end{verbatim}

A default path of the NWChem basis set libraries is provided on installation 
of the code, but a different path can be defined by specifying the keyword 
\verb+file+, and one can explicitly name the file to be accessed
for the basis functions. For example,
\begin{verbatim}
  basis
    o  library 3-21g file /usr/d3g681/nwchem/library
    si library 6-31g file /usr/d3g681/nwchem/libraries/
  end
\end{verbatim}
This directive tells the code to use the basis set \verb+3-21g+ in
the file {\tt /usr/\-d3g681/\-nwchem/\-library} for atom \verb+o+ and
to use the basis set \verb+6-31g+ in the directory 
{\tt /usr/\-d3g681/\-nwchem/\-libraries/} for atom \verb+si+, rather 
than look for them in the default libraries. When a directory is defined 
the code will search for the basis set in a file with the name {\tt 6-31g}.

The ``*'' tag can be used to efficiently define basis set input directives 
for large numbers of atoms. An example is:
\begin{verbatim}
  basis
    *  library 3-21g
  end
\end{verbatim}
This directive tells the code to assign the basis sets \verb+3-21g+ to
all the atom tags defined in the geometry. If one wants to place a
different basis set on one of the atoms defined in the geometry, the 
following directive can be used:
\begin{verbatim}
  basis
    *  library 3-21g except H
  end
\end{verbatim}
This directive tells the code to assign the basis sets \verb+3-21g+ to
all the atoms in the geometry, except the hydrogen atoms. Remember that 
the user will have to explicitly define the hydrogen basis set in this
directive! One may also define tags that end with a ``*'': 
\begin{verbatim}
  basis
    oxy*  library 3-21g 
  end
\end{verbatim}
This directive tells the code to assign the basis sets \verb+3-21g+ to 
all atom tags in the geometry that start with ``oxy''.

If standard basis sets are to be placed upon a dummy center, the
variable \verb+<tag_in_lib>+ must also be entered on this line, to
identify the correct atom type to use from the basis function library
(see the ghost atom example in Section \ref{sec:set} and below).  For
example: To specify the cc-pvdz basis for a calculation on the water
monomer in the dimer basis, where the dummy oxygen and dummy hydrogen
centers have been identified as \verb+bqo+ and \verb+bqh+
respectively, the \verb+BASIS+ directive is as follows:

\begin{verbatim}
  basis
    o   library cc-pvdz
    h   library cc-pvdz
    bqo library o cc-pvdz
    bqh library h cc-pvdz
  end
\end{verbatim}
A special dummy center tag is \verb+bq*+, which will assign the same basis 
set to all bq centers in the geometry. Just as with the ``*'' tag, the 
\verb+except+ list can be used to assign basis sets to unique dummy centers.

The library basis sets can also be marked as relativistic by adding the
\verb+rel+ keyword to the tag line. See Section \ref{sec:dyall-mod-dir} for
more details. The correlation consistent basis sets have been contracted for
relativistic effects and are included in the standard library.

There are also contractions in the standard library for both a point nucleus
and a finite nucleus of Gaussian shape. These are usually distinguished by
the suffixex {\tt \_pt} and {\tt \_fi}. It is the user's responsibility to
ensure that the contraction matches the nuclear type specified in the
geometry object. The specification of a finite nucleus basis set does NOT
automatically set the nuclear type for that atom to be finite.  See 
Section \ref{sec:geom} for information.

\section{Explicit basis set definition}

If the basis sets in the library or available in other external files
are not suitable for a given calculation, 
the basis set may be explicitly defined.
A generally contracted Gaussian basis function is associated with a
center using an input line of the following form:
\begin{verbatim}
     <string tag> <string shell_type> [rel]
        <real exponent> <real list_of_coefficients>
        ...
\end{verbatim}

The variable \verb+<shell_type>+ identifies the angular momentum of the
shell, $s$, $p$, $d$, \ldots.  NWChem is configured to handle up to $h$
shells.  The keyword \verb+rel+ marks the shell as relativistic --- see
Section \ref{sec:dyall-mod-dir} for more details.  Subsequent lines define
the primitive function exponents and contraction coefficients.  General
contractions are specified by including multiple columns of coefficients.

The following example defines basis sets for the water molecule:
\begin{verbatim}
  basis spherical nosegment
    oxygen s
      11720.0000    0.000710  -0.000160
       1759.0000    0.005470  -0.001263
        400.8000    0.027837  -0.006267
        113.7000    0.104800  -0.025716
         37.0300    0.283062  -0.070924
         13.2700    0.448719  -0.165411
          5.0250    0.270952  -0.116955
          1.0130    0.015458   0.557368
          0.3023   -0.002585   0.572759
    oxygen s                
          0.3023    1.000000
    oxygen p                
         17.7000    0.043018
          3.8540    0.228913
          1.0460    0.508728
          0.2753    0.460531
    oxygen p                
          0.2753    1.000000
    oxygen d
          1.1850    1.000000
    hydrogen s
         13.0100    0.019685
          1.9620    0.137977
          0.4446    0.478148
          0.1220    0.501240
    hydrogen s  
          0.1220    1.000000
    hydrogen p  
          0.7270    1.000000
    oxygen s
          0.01      1.0
    hydrogen s
          0.02974   1.0
    hydrogen p
          0.141      1.0
  end
\end{verbatim}


\section{Combinations of library and explicit basis set input}
The user can use a mixture of library basis and explicit basis set
input to define the basis sets used on the various atoms.

For example, the following \verb+BASIS+ directive augments the Dunning
cc-pvdz basis set for the water molecule with a diffuse s-shell on
oxygen and adds the aug-cc-pVDZ diffuse functions onto the hydrogen.
\begin{verbatim}
  basis spherical nosegment
    oxygen library cc-pvdz
    hydrogen library cc-pvdz
    oxygen s
      0.01 1.0
    hydrogen library "aug-cc-pVDZ Diffuse"
  end
\end{verbatim}

The resulting basis set defined is identical to the one defined above 
in the explicit basis set input.



\chapter{Effective Core Potentials}
\label{sec:ecp}

In addition to the standard basis sets that can be specified using the
\verb+BASIS+ directive, the basis functions can be described in NWChem
using a effective core potential (ECP) basis set.  An ECP basis set
consists of contracted gaussian functions that are fit to gaussians by the
function,
\[
r^2V_l(r) = \sum_{k} A_{lk} r^{n_{lk}} e^{B_{lk}r^{2}}
\]
where $A_{lk}$ is the contraction coefficient, $n_{lk}$ is the
exponent of the ``r'' term (r-exponent), and $B_{lk}$ is the gaussian
exponent.  (Note: to be consistent with most of the literature on ECP
functions, the r-exponent is shifted by 2; e.g., an r-exponent of 0 
implies $r^{-2}$).

The EMSL library does not currently support a standard set of ECP basis
functions.  Basis sets using these functions must be specified explicitly
by user input in the \verb+ECP+ directive.  This directive has
essentially the same form as the standard \verb+BASIS+ directive,
except for differences in the type of information to be supplied for
each atom or center using the ECP basis set.  The form of the input for
the \verb+ECP+ directive is as follows;

\begin{verbatim}
  ecp [<string name default "ecp basis">] \
        [spherical || cartesian default cartesian] \
        [segment || nosegment default segment] \
        [print || noprint default print]
\end{verbatim}

%     <string tag> library [<string tag_in_lib>] \
%                  <string standard set> [file <filename>]
%
%         or
%
\begin{verbatim}
     <string tag> [nelec] <integer number_of_electrons_replaced>
 
        ...

     <string tag> <string shell_type>
     <real r-exponent> <real gaussian-exponent> <real list_of_coefficients>
        ...
     
  END
\end{verbatim}    


% This directive describes an effective core potential (ECP) basis set
% of contracted gaussian functions.  These are fit to gaussians by the
% function:
% \[
% r^2V_l(r) = \sum_{k} A_{lk} r^{n_{lk}} e^{B_{lk}r^{2}}
% \]
% Where $A_{lk}$ is the contraction coefficient, $n_{lk}$ is the
% exponent of the ``r'' term (r-exponent), and $B_{lk}$ is the gaussian
% exponent.  The r-exponent is shifted by 2 as per most of the ECP
% literature, e.g., an r-exponent of 0 implies $r^{-2}$.
% 
% By default basis sets are automatically segmented and cartesian even
% if general contractions are input.  Generally contracted ECP basis
% sets are not in wide use but the functionality is available.  ECP
% basis functions are associated with centers in geometries through the
% tags or names of centers which must match exactly (including case) and
% are limited to sixteen (16) characters.  Each center with the same tag
% will have the same ecp basis set.  By default the input module prints
% each ecp basis set encountered; use the \verb+NOPRINT+ option to
% disable printing.  There can be only one active ECP basis set even
% though several may exist in the input deck.  The ECP modules load
% ``ecp basis'' with any ``ao basis'' present.  The ECP functionality
% works for energy and gradients.
% 
% In the same fashion as for geometries or regular basis sets, ecp basis
% sets are named, with the default name being \verb+"ecp basis"+.  It
% should be clear from the above discussion on geometries and database
% entries how indirection is supported.
% 
% Basis functions currently may not be drawn from a standard set in the
% EMSL basis set library; they must be specified explicitly.  All
% directives that are in common with the standard gaussian basis set
% input have the same function and syntax.  

The string \verb+name+ allows the user to identify a specific ECP basis set
in the database for a calculation.  If no name is specified explicitly, the
default name is  \verb+"ecp basis"+  The various modules in the code expect 
to find the
basis set in the database under the name \verb+"ao basis"+.  The user can
assign the string \verb+name+ (or \verb+"ecp basis"+, if the default name 
is used) to \verb+"ao basis"+ using the \verb+SET+
directive (see Section \ref{sec:set}), in the same manner as that in
which standard basis sets of some (non-default) \verb+name+ 
are assigned the default name \verb+"ao basis"+ 
for specific tasks.  This indirection allows the user to assign different
basis sets to the same geometry object, or assign the same basis set to 
different geometry objects, for different calculations in
the course of the same job.  Only one ECP basis set can be active for a
given task, although any number of such basis sets can be defined in the
input under names other than \verb+"ecp basis"+.

The keyword pairs

\begin{verbatim}
spherical || cartesian
segment || nosegment
print || noprint
\end{verbatim}

are interpreted
in the \verb+ECP+ directive in the same manner as for the \verb+BASIS+
directive, but only the defaults are currently available for
the coordinate system and segmentation in the \verb+ECP+ directive.   
NWChem assumes
that all ecp basis sets are segmented and cartesian.  
The keywords are included in the directive, however, because it is
expected that eventually the code will include the options for the
user to specify basis functions in either spherical or 
cartesian coordinates, segmented or unsegmented.  The print keyword is
currently active, and can be used to specify that the descriptions of
the functions will be printed or not printed by the input module, at the user's
discretion.  
% , whether specified by the 
% user or defined in the standard library sets.

As with the input for the standard basis sets using the \verb+BASIS+
directive, the input specified for the \verb+ECP+ directive in lines 
beginning with the string \verb+tag+
allow particular centers or atoms in a calculation to be associated with
particular basis functions.  The values specified for \verb+tag+
must correspond exactly with the names supplied for the \verb+tag+ entries
on the \verb+GEOMETRY+ directive for a particular calculation.  Each atom
or center with the same \verb+tag+ will have the same basis set, which must
also be specified with the same name \verb+tag+.

The keyword \verb+NELEC+ allows the user to specify the number of core 
electrons replaced by
the ecp basis specification for the atom represented by the tag.  Additional
input lines can then be used to define the specific coefficients.
The string \verb+shell_type+ is used to specify the components of the
ECP basis function.  The label \verb+ul+ entered for \verb+shell_type+
for a given atom or center (identified by the string \verb+tag+) denotes
the local part of the ECP basis.  This is equivalent to the highest 
angular momentum
functions specified in the literature for most ecp basis sets.  The
standard entries (\verb+s, p, d+, etc.) for \verb+shell_type+ delineate 
the angular momentum projector onto the local function.  The shell type 
label of \verb+s+ indicates the \verb+ul-s+ projector input, \verb+p+ 
indicates the \verb+ul-p+, etc.

An application of the \verb+ECP+ directive is illustrated in the following 
example using the molecule  H$_2$CO.  This input defines an ECP basis set 
for the  carbon and oxygen atoms in the molecule.
\Large
(*****What about the hydrogen molecule?  What basis set does it use?******)
\normalsize

% The following example illustrate the input of an ECP for H$_2$CO.

% \centerline{{\bf H$_2$CO }}

\begin{verbatim}
ecp print  
C nelec 2     # ecp replaces 2 electrons on C
C ul    # d
        1       80.0000000       -1.60000000
        1       30.0000000       -0.40000000
        2        0.5498205       -0.03990210
C s     # s - d 
        0        0.7374760        0.63810832
        0      135.2354832       11.00916230
        2        8.5605569       20.13797020
C p     # p - d
        2       10.6863587       -3.24684280
        2       23.4979897        0.78505765
O nelec 2     # ecp replaces 2 electrons on O
O ul    # d 
        1       80.0000000       -1.60000000
        1       30.0000000       -0.40000000
        2        1.0953760       -0.06623814
O s     # s - d
        0        0.9212952        0.39552179
        0       28.6481971        2.51654843
        2        9.3033500       17.04478500
O p     # p - s 
        2       52.3427019       27.97790770
        2       30.7220233      -16.49630500
end
\end{verbatim}

Generally contracted ECP basis sets are not in wide use, but the
functionality has been included in NWChem for applications where
they might be useful.  The user should be aware, however, that
the atomic SCF code does not currently handle the guess
generation for ECP centers.  Therefore, it may not be possible to obtain
an initial set of orbitals for either the SCF or DFT when using an ECP
basis set.  It may be necessary to use a smaller basis set and then
project those orbitals to the
basis set that will be used, (refer to Section \ref{sec:vectors}).  Convergence
and starting orbital guesses are being addressed in ongoing development
work with the code.


\chapter{Relativistic Approximations}
%
% $Id: rel.tex,v 1.17 2007-12-10 20:39:27 niri Exp $
%
\label{sec:rel}
All methods which include treatment of relativistic effects are ultimately
based on the Dirac equation, which has a four component wave function. The
solutions to the Dirac equation describe both positrons (the ``negative
energy'' states) and electrons (the ``positive energy'' states), as well as
both spin orientations, hence the four components. The wave function may be
broken down into two-component functions traditionally known as the large
and small components; these may further be broken down into the spin
components. 

The implementation of approximate all-electron relativistic methods in
quantum chemical codes requires the removal of the negative energy states
and the factoring out of the spin-free terms. Both of these may be achieved
using a transformation of the Dirac Hamiltonian known in general as a
Foldy-Wouthuysen transformation. Unfortunately this transformation cannot be
represented in closed form for a general potential, and must be
approximated.  One popular approach is that originally formulated by Douglas
and Kroll\footnote{M.~Douglas and N.~M.~Kroll, Ann. Phys. (N.Y.)  {\bf 82},
89 (1974)} and developed by Hess\footnote{B.A.~Hess, Phys.~Rev.~A~{\bf 32},
756 (1985); {\bf 33}, 3742 (1986)}. This approach decouples the positive and
negative energy parts to second order in the external potential (and also
fourth order in the fine structure constant, $\alpha$). Other approaches include 
the Zeroth Order Regular Approximation (ZORA)\footnote{C.~Chang, M.~Pelissier, 
M.~Durand, Physica Scripta ~{\bf 34}, 294 (1986); E.~van Lenthe, ~{\it The ZORA Equation}, 
doctoral thesis, Vrije Universiteit, Amsterdam (1996); S.~Faas, J.G.~Snijders, 
J.H.~van Lenthe, E.~van Lenthe, and E.J.~Baerends, Chem.~Phys.~ Lett.~{\bf 246}, 632 (1995).}
and modification of the Dirac equation by Dyall\footnote{K.~G.~Dyall,
J.~Chem.~Phys.~{\bf 100}, 2118 (1994)}, and involves an exact FW
transformation on the atomic basis set level\footnote{K.~G.~Dyall,
J.~Chem.~Phys.~{\bf 106}, 9618 (1997); K.~G.~Dyall and T.~Enevoldsen,
J.~Chem.~Phys.~{\bf 111}, 10000 (1999).}.

Since these approximations only modify the integrals, they can in principle
be used at all levels of theory. At present the Douglas-Kroll and ZORA 
implementations can be used at all levels of theory whereas 
Dyall's approach is currently available at the Hartree-Fock level. 
The derivatives have been implemented, allowing both methods to be used in 
geometry optimizations and frequency calculations.

The \verb+RELATIVISTIC+ directive provides input for the implemented relativistic 
approximations and is a compound directive that encloses additional directives 
specific to the approximations:
\begin{verbatim}
  RELATIVISTIC
   [DOUGLAS-KROLL [<string (ON||OFF) default ON> \
                 <string (FPP||DKH||DKFULL||DK3||DK3FULL) default DKH>]  ||
    ZORA [ (ON || OFF) default ON ] || 
    DYALL-MOD-DIRAC [ (ON || OFF) default ON ] 
                  [ (NESC1E || NESC2E) default NESC1E ] ]
   [CLIGHT <real clight default 137.0359895>]
  END
\end{verbatim}

Only one of the methods may be chosen at a time.  If both methods are found
to be on in the input block, NWChem will stop and print an error message.
There is one general option for both methods, the definition of the speed 
of light in atomic units:

\begin{verbatim}
  CLIGHT <real clight default 137.0359895>
\end{verbatim}

The following sections describe the optional sub-directives that
can be specified within the \verb+RELATIVISTIC+ block.

\section{Douglas-Kroll approximation}
\label{sec:douglas-kroll}

The spin-free and spin-orbit one-electron Douglas-Kroll 
approximation have been implemented. The use of relativistic effects 
from this Douglas-Kroll approximation can be invoked by specifying:

\begin{verbatim}
  DOUGLAS-KROLL [<string (ON||OFF) default ON> \
                 <string (FPP||DKH||DKFULL|DK3|DK3FULL) default DKH>]
\end{verbatim}

The \verb+ON|OFF+ string is used to turn on or off the
Douglas-Kroll approximation.  By default, if the \verb+DOUGLAS-KROLL+
keyword is found, the approximation will be used in the calculation.
If the user wishes to calculate a non-relativistic quantity after turning
on Douglas-Kroll, the user will need to define a new \verb+RELATIVISTIC+
block and turn the approximation \verb+OFF+.  The user could also simply
put a blank \verb+RELATIVISTIC+ block in the input file and all options 
will be turned off.

The \verb+FPP+ is the approximation based on free-particle projection 
operators\footnote{B.A.~Hess, Phys.~Rev.~A~{\bf 32}, 756 (1985)} whereas the 
\verb+DKH+ and \verb+DKFULL+ approximations are based on external-field 
projection operators\footnote{B.A.~Hess, Phys.~Rev.~A~{\bf 33}, 3742 (1986)}.
The latter two are considerably better approximations than the former. \verb+DKH+ 
is the Douglas-Kroll-Hess approach and is the approach that is generally 
implemented in quantum chemistry codes. \verb+DKFULL+ includes certain 
cross-product integral terms ignored in the \verb+DKH+ approach (see for example 
H\"{a}berlen and R\"{o}sch\footnote{O.D.~H\"{a}berlen, N.~R\"{o}sch, 
Chem.~Phys.~Lett.~{\bf 199}, 491 (1992)}). The third-order Douglas-Kroll 
approximation has been implemented by T. Nakajima and K. Hirao\footnote{T. Nakajima 
and K. Hirao, Chem.~Phys.~Lett.~{\bf 329}, 5111 (2000); T. Nakajima and K. Hirao, 
J.~Chem.~Phys.~{\bf 113}, 7786 (2000)}. This approximation can be called using
\verb+DK3+ (DK3 without cross-product integral terms) or \verb+DK3FULL+ (DK3 with
cross-product integral terms).

The contracted basis sets used in the calculations should reflect the relativistic
effects, i.e. one should use contracted basis sets which were generated using the 
Douglas-Kroll Hamiltonian. Basis sets that were contracted using the 
non-relativistic (Sch\"{o}dinger) Hamiltonian WILL PRODUCE ERRONEOUS RESULTS for
elements beyond the first row. See appendix \ref{sec:knownbasis} for available
basis sets and their naming convention.

NOTE: we suggest that spherical basis sets are used in the calculation. The use of 
high quality cartesian basis sets can lead to numerical inaccuracies.

In order to compute the integrals needed for the Douglas-Kroll approximation
the implementation makes use of a fitting basis set (see literature given
above for details). The current code will create this fitting basis set
based on the given {\tt "ao basis"} by simply uncontracting that basis. This
again is what is commonly implemented in quantum chemistry codes that
include the Douglas-Kroll method.  Additional flexibility is available to
the user by explicitly specifying a Douglas-Kroll fitting basis
set. This basis set must be named {\tt "D-K basis"} (see Chapter
\ref{sec:basis}).

\section{Zeroth Order regular approximation (ZORA)}
\label{sec:zora}

The spin-free and spin-orbit one-electron zeroth-order regular approximation (ZORA) 
have been implemented. The use of relativistic effects with ZORA 
can be invoked by specifying:

\begin{verbatim}
  ZORA [<string (ON||OFF) default ON>
\end{verbatim}

The \verb+ON|OFF+ string is used to turn on or off ZORA.  
By default, if the \verb+ZORA+ keyword is found, the approximation 
will be used in the calculation. If the user wishes to calculate 
a non-relativistic quantity after turning on ZORA, the user 
will need to define a new \verb+RELATIVISTIC+ block and turn 
the approximation \verb+OFF+.  The user can also simply put 
a blank \verb+RELATIVISTIC+ block in the input file and all options 
will be turned off.

\section{Dyall's Modified Dirac Hamitonian approximation}
\label{sec:dyall-mod-dir}

The approximate methods described in this section are all based on Dyall's
modified Dirac Hamiltonian. This Hamiltonian is entirely equivalent to the
original Dirac Hamiltonian, and its solutions have the same properties.
The modification is achieved by a transformation on the small component,
extracting out \hbox{$\sigma\cdot{\bf p}/2mc$}. This gives the modified small
component the same symmetry as the large component, and in fact it differs
from the large component only at order $\alpha^2$.  The advantage of the
modification is that the operators now resemble the operators of the
Breit-Pauli Hamiltonian, and can be classified in a similar fashion into
spin-free, spin-orbit and spin-spin terms. It is the spin-free terms which
have been implemented in NWChem, with a number of further approximations.

The first is that the negative energy states are removed by a normalized
elimination of the small component (NESC), which is equivalent to an exact
Foldy-Wouthuysen (EFW) transformation. The number of components in the wave
function is thereby effectively reduced from 4 to 2. NESC on its own does
not provide any advantages, and in fact complicates things because the
transformation is energy-dependent. The second approximation therefore
performs the elimination on an atom-by-atom basis, which is equivalent to
neglecting blocks which couple different atoms in the EFW transformation.
The advantage of this approximation is that all the energy dependence can be
included in the contraction coefficients of the basis set.  The tests which
have been done show that this approximation gives results well within
chemical accuracy. The third approximation neglects the commutator of the
EFW transformation with the two-electron Coulomb interaction, so that the
only corrections that need to be made are in the one-electron integrals.
This is the equivalent of the Douglas-Kroll(-Hess) approximation as it is
usually applied.

The use of these approximations can be invoked with the use of the
\verb+DYALL-MOD-DIRAC+ directive in the \verb+RELATIVISTIC+ directive block.
The syntax is as follows.

\begin{verbatim}
  DYALL-MOD-DIRAC [ (ON || OFF) default ON ] 
                  [ (NESC1E || NESC2E) default NESC1E ]
\end{verbatim}

The \verb+ON|OFF+ string is used to turn on or off the
Dyall's modified Dirac approximation. By default, if the \verb+DYALL-MOD-DIRAC+
keyword is found, the approximation will be used in the calculation.
If the user wishes to calculate a non-relativistic quantity after turning
on Dyall's modified Dirac, the user will need to define a new 
\verb+RELATIVISTIC+
block and turn the approximation \verb+OFF+.  The user could also simply
put a blank \verb+RELATIVISTIC+ block in the input file and all options 
will be turned off.

Both one- and two-electron approximations are available
\verb+NESC1E || NESC2E+, and both have
analytic gradients. The one-electron approximation is the default.
The two-electron approximation specified by \verb+NESC2E+ has some sub
options which are placed on the same logical line as the
\verb+DYALL-MOD-DIRAC+ directive, with the following syntax:

\begin{verbatim}
  NESC2E [ (SS1CENT [ (ON || OFF) default ON ] || SSALL) default SSALL ]
         [ (SSSS [ (ON || OFF) default ON ] || NOSSSS) default SSSS ]
\end{verbatim}

The first sub-option gives the capability to limit the two-electron
corrections to those in which the small components in any density must be on
the same center.  This reduces the $(LL|SS)$ contributions to at most
three-center integrals and the $(SS|SS)$ contributions to two centers. For a
case with only one relativistic atom this option is redundant. The second
controls the inclusion of the $(SS|SS)$ integrals which are of order
$\alpha^4$. For light atoms they may safely be neglected, but for heavy
atoms they should be included. 

In addition to the selection of this keyword in the \verb+RELATIVISTIC+
directive block, it is necessary to supply basis sets in addition to the
\verb+ao basis+. For the one-electron approximation, three basis sets are
needed: the atomic FW basis set, the large component basis set and the small
component basis set. The atomic FW basis set should be included in the
\verb+ao basis+.
The large and small components should similarly be incorporated
in basis sets named \verb+large component+ and \verb+small component+,
respectively. For the two-electron approximation, only two basis sets are
needed. These are the large component and the small component. The large component
should be included in the \verb+ao basis+ and the small component
is specified separately as \verb+small component+, as for the one-electron
approximation. This means that the two approximations can {\it not} be run
correctly without changing the \verb+ao basis+, and it is up to the user to
ensure that the basis sets are correctly specified.

There is one further requirement in the specification of the basis sets. In
the \verb+ao basis+, it is necessary to add the \verb+rel+ keyword either to the
\verb+basis+ directive or the library tag line (See below for examples). 
The former marks the basis
functions specified by the tag as relativistic, the latter marks the whole
basis as relativistic. The marking is actually done at the unique shell
level, so that it is possible not only to have relativistic and
nonrelativistic atoms, it is also possible to have relativistic and
nonrelativistic shells on a given atom. This would be useful, for example,
for diffuse functions or for high angular momentum correlating functions,
where the influence of relativity was small. The marking of shells as
relativistic is necessary to set up a mapping between the ao basis and the
large and/or small component basis sets. For the one-electron approximation
the large and small component basis sets MUST be of the same size and
construction, i.e. differing only in the contraction coefficients.

It should also be noted that the relativistic code will NOT work with basis
sets that contain sp shells, nor will it work with ECPs. Both of these are
tested and flagged as an error.

Some examples follow. The first example sets up the data for relativistic
calculations on water with the one-electron approximation and the
two-electron approximation, using the library basis sets.

\begin{verbatim}
  start h2o-dmd

  geometry units bohr
  symmetry c2v
    O       0.000000000    0.000000000   -0.009000000
    H       1.515260000    0.000000000   -1.058900000
    H      -1.515260000    0.000000000   -1.058900000
  end

  basis "fw" rel
    oxygen library cc-pvdz_pt_sf_fw
    hydrogen library cc-pvdz_pt_sf_fw
  end

  basis "large"
    oxygen library cc-pvdz_pt_sf_lc
    hydrogen library cc-pvdz_pt_sf_lc
  end

  basis "large2" rel
    oxygen library cc-pvdz_pt_sf_lc
    hydrogen library cc-pvdz_pt_sf_lc
  end

  basis "small"
    oxygen library cc-pvdz_pt_sf_sc
    hydrogen library cc-pvdz_pt_sf_sc
  end

  set "ao basis" fw
  set "large component" large
  set "small component" small

  relativistic
    dyall-mod-dirac
  end

  task scf

  set "ao basis" large2
  unset "large component"
  set "small component" small

  relativistic
    dyall-mod-dirac nesc2e
  end

  task scf
\end{verbatim}

The second example has oxygen as a relativistic atom and hydrogen nonrelativistic.

\begin{verbatim}
  start h2o-dmd2

  geometry units bohr
  symmetry c2v
    O       0.000000000    0.000000000   -0.009000000
    H       1.515260000    0.000000000   -1.058900000
    H      -1.515260000    0.000000000   -1.058900000
  end

  basis "ao basis"
    oxygen library cc-pvdz_pt_sf_fw rel
    hydrogen library cc-pvdz
  end

  basis "large component"
    oxygen library cc-pvdz_pt_sf_lc
  end

  basis "small component"
    oxygen library cc-pvdz_pt_sf_sc
  end

  relativistic
    dyall-mod-dirac
  end

  task scf
\end{verbatim}


\chapter{Hartree-Fock or Self-consistent Field} 
\label{sec:scf}

The NWChem self-consistent field (SCF) module computes closed-shell
restricted Hartree-Fock (RHF) wavefunctions, restricted high-spin
open-shell Hartree-Fock (ROHF) wavefunctions, and spin-unrestricted
Hartree-Fock (UHF) wavefunctions.

The \verb+SCF+ directive provides input to the SCF module and is a
compound directive that encloses additional directives specific to the
SCF module:
\begin{verbatim}
  SCF
    ...
  END
\end{verbatim}

\section{Wavefunction type}

A spin-restricted, closed shell RHF calculation is performed by
default.  An error results if the number of electrons is inconsistent
with this assumption.  The number of electrons is inferred from the
total charge on the system and the sum of the effective nuclear
charges of all centers (atoms and dummy atoms, Section
\ref{sec:geom}).  The total charge on the system is zero by default,
unless specified at some value by input on the \verb+CHARGE+ directive
(Section \ref{sec:toplevel}).

The options available to define the SCF wavefunction and multiplicity
are as follows:

\begin{verbatim}
  SINGLET 
  DOUBLET 
  TRIPLET 
  QUARTET 
  QUINTET 
  SEXTET
  SEPTET
  OCTET
  NOPEN <integer nopen default 0>
  RHF
  ROHF
  UHF
\end{verbatim}

The optional keywords \verb+SINGLET+, \verb+DOUBLET+, \ldots,
\verb+OCTET+ and \verb+NOPEN+ allow the user to specify the number of
singly occupied orbitals for a particular calculation.  \verb+SINGLET+
is the default, and specifies a closed shell; \verb+DOUBLET+ specifies
one singly occupied orbital; \verb+TRIPLET+ specifies two singly
occupied orbitals; and so forth.  If there are more than seven singly
occupied orbitals, the keyword \verb+NOPEN+ must be used, with the
integer \verb+nopen+ defining the number of singly occupied
orbitals (sometimes referred to as open shells).

If the multiplicity is any value other than \verb+SINGLET+, the
default calculation will be a spin-restricted, high-spin, open-shell
SCF calculation (keyword ROHF).  The open-shell orbitals must be the
highest occupied orbitals.  If necessary, any starting vectors may be
rearranged through the use of the \verb+SWAP+ keyword on the
\verb+VECTORS+ directive (see Section \ref{sec:vectors}) to accomplish
this.

A spin-unrestricted solution can also be performed by specifying the
keyword \verb+UHF+.  In UHF calculations, it is assumed that the
number of singly occupied orbitals corresponds to the difference
between the number of alpha-spin and beta-spin orbitals.  For example,
a UHF calculation with 2 more alpha-spin orbitals than beta-spin
orbitals can be obtained by specifying

\begin{verbatim}
  scf
     triplet ; uhf    # (Note: two logical lines of input)
     ...
  end
\end{verbatim}

The user should be aware that, by default, molecular orbitals are
symmetry adapted in NWChem.  This may not be desirable for fully
unrestricted wavefunctions.  In such cases, the user has the option of
defeating the defaults by specifying the keywords \verb+ADAPT OFF+
(see Section \ref{sec:adapt}) and \verb+SYM OFF+ (see Section
\ref{sec:sym}).

The keywords \verb+RHF+ and \verb+ROHF+ are provided in the code for
completeness. It may be necessary to specify these in order to modify
the behavior of a previous calculation (see Section \ref{sec:persist}
for restart behavior).

\section{{\tt SYM} --- use of symmetry}
\label{sec:sym}

 \begin{verbatim}
   SYM <string (ON||OFF) default ON>
 \end{verbatim}

This directive enables/disables the use of symmetry to speed up Fock matrix
construction (via the petite-list or skeleton algorithm) in the SCF, if
symmetry was used in the specification of the geometry.  Symmetry
adaptation of the molecular orbitals is not affected by this option.
The default is to use symmetry if it is specified in the geometry
directive (Section \ref{sec:geom}). 

For example, to disable use of symmetry in Fock matrix construction:
\begin{verbatim}
  sym off
\end{verbatim}

\section{{\tt ADAPT} -- symmetry adaptation of MOs}
\label{sec:adapt}

\begin{verbatim}
  ADAPT <string (ON||OFF) default ON>
\end{verbatim}

The default in the SCF module calculation is to force symmetry
adaption of the molecular orbitals. This does not affect the speed of
the calculation, but without explicit adaption the resulting orbitals
may be symmetry contaminated for some problems.  This is especially
likely if the calculation is started using orbitals from a distorted
geometry.

The underlying assumption in the use of symmetry in Fock matrix
construction is that the density is totally symmetric.  If the orbitals
are symmetry contaminated, this assumption may not be valid --- which
could result in incorrect energies and poor convergence of the
calculation.  It is thus advisable when specifying \verb+ADAPT OFF+ to
also specify \verb+SYM OFF+ (Section \ref{sec:sym}).

\section{{\tt TOL2E} --- integral screening threshold}
\label{sec:tol2e}

\begin{verbatim}
  TOL2E <real tol2e default min(10e-7 , 0.01*$thresh$)>
\end{verbatim}

The variable \verb+tol2e+ is used in determining the integral
screening threshold for the evaluation of the energy and related
Fock-like matrices.  The Schwarz inequality is used to screen the
product of integrals and density matrices in a manner that results in
an accuracy in the energy and Fock matrices that approximates the
value specified for \verb+tol2e+. 

It is generally not necessary to set this parameter directly.  Specify
instead the required precision in the wavefunction, using the
\verb+THRESH+ directive (Section \ref{sec:thresh}). The default
threshold is the minimum of $10^{-7}$ and 0.01 times the requested
convergence threshold for the SCF calculation (Section
\ref{sec:thresh}).  

The input to specify the threshold explicitly within the \verb+SCF+
directive is, for example:

\begin{verbatim}
  tol2e 1e-9
\end{verbatim}

For very diffuse basis sets, or for high-accuracy calculations it
might be necessary to set this parameter.  A value of $10^{-12}$ is
sufficient for nearly all such purposes.

\section{{\tt VECTORS} --- input/output of MO vectors}
\label{sec:vectors}


\begin{verbatim}
  VECTORS [[input] (<string input_movecs default atomic>) || \
                   (project <string basisname> <string filename>) || \
                   (fragment <string file1> [<string file2> ...])] \
          [swap [alpha||beta] <integer vec1 vec2> ...] \
          [output <string output_filename default input_movecs>] \
          [lock]
\end{verbatim}

The \verb+VECTORS+ directive allows the user to specify the source and
destination of the molecular orbital vectors.  In a startup
calculation (see Section \ref{sec:start}), the default source for
guess vectors is a diagonalized Fock matrix constructed from a
superposition of the atomic density matrices for the particular
problem.  This is usually a very good guess.  For a restarted 
calculation, the default is to use the previous MO vectors.

The optional keyword \verb+INPUT+ allows the user to specify the
source of the input molecular orbital vectors as any of the following:
\begin{itemize}
\item \verb+ATOMIC+ --- eigenvectors of a Fock-like matrix formed from
  a superposition of the atomic densities (the default guess).  See
  Sections \ref{sec:atomscf} and \ref{sec:tolguess}.  
\item \verb+HCORE+ --- eigenvectors of the bare-nucleus Hamiltonian or
  the one-electron Hamiltonian.
\item \verb+filename+ --- the name of a file containing the MO vectors
  from a previous calculation.  Note that unless the path is fully
  qualified, or begins with a dot (``.''), then it is assumed to
  reside in the directory for permanent files (see Section
  \ref{sec:dirs}).
\item \verb+PROJECT basisname filename+ --- projects the existing MO
  vectors in the file \verb+filename+ from the smaller basis with name
  \verb+basisname+ into the current basis.  The definition of the
  basis \verb+basisname+ must be available in the current database,
  and the basis must be smaller than the current basis.  In addition,
  the geometry used for the previous calculations must have the atoms
  in the same order and in the same orientation as the current
  geometry.
\item \verb+FRAGMENT file1 ...+ --- assembles starting MO vectors from
  previously performed calculations on fragments of the system and is
  described in more detail in Section \ref{sec:fragguess}.  Even
  though there are some significant restrictions in the use of the
  initial implementation of this method (see Section
  \ref{sec:fragguess}), this is the most powerful initial guess option
  within the code.  It is particularly indispensible for open shell
  metallic systems.
\end{itemize}
 
The molecular orbitals are saved every iteration if more than 600
seconds have elapsed, and also at the end of the calculation.  At
completion (converged or not), the SCF module always canonically
transforms the molecular orbitals by {\em separately} diagonalizing
the closed--closed, open--open, and virtual--virtual blocks of the
Fock matrix.

The name of the file used to store the MO vectors is determined as
follows:
\begin{itemize}
\item if the \verb+OUTPUT+ keyword was specified on the \verb+VECTORS+
  directive, then the filename that follows this keyword is used, or
\item if the input vectors were read from a file, this file is reused
  for the output vectors (overwriting the input vectors); else,
\item a default file name is generated in the directory for permanent
  files (Section \ref{sec:dirs}) by prepending \verb+".movecs"+ with
  the file prefix, i.e., \verb+"<file_prefix>.movecs"+.
\end{itemize}
The name of this file is stored in the database so that a subsequent
SCF calculation will automatically restart from these MO vectors.

Applications of this directive are illustrated in the following
examples.

Example 1:
\begin{verbatim}
  vectors output h2o.movecs
\end{verbatim}
Assuming a start-up calculation, this directive will result in use of
the default atomic density guess, and will output the vectors to the
file \verb+h2o.movecs+.

Example 2:
\begin{verbatim}
  vectors input initial.movecs output final.movecs
\end{verbatim}
This directive will result in the initial vectors being read from the
file \verb+"initial.movecs"+.  The results will be written to the file
\verb+final.movecs+.  The contents of \verb+"initial.movecs"+ will not
be changed.

Example 3:
\begin{verbatim}
  vectors input project "small basis" small.movecs
\end{verbatim}
This directive will cause the calculation to start from vectors in the
file \verb+"small.movecs"+ which are in a basis named \verb+"small basis"+.
The output vectors will be written to the default file
\verb+"<file_prefix.movecs>"+.
 
Once starting vectors have been obtained using any of the possible
options, they may be reordered through use of the \verb+SWAP+ keyword.
This optional keyword requires a list of orbital pairs that will be
swapped.  For UHF calculations, separate \verb+SWAP+ keywords may be
provided for the alpha and beta orbitals, as necessary.

An example of use of the \verb+SWAP+ directive:
\begin{verbatim}
  vectors input try1.movecs swap 173 175 174 176 output try2.movecs
\end{verbatim}
This directive will cause the initial orbitals to be read from the
file \verb+"try1.movecs"+.  The vectors for the orbitals within the
pairs 173--175 will be swapped with those within 174--176, so the
resulting order is 175, 176, 173, 174.  The final orbitals obtained in
the calculation will be written to the file \verb+"try2.movecs"+.

The swapping of orbitals occurs as a sequential process in the order
(left to right) input by the user.  Thus, regarding each pair as an
elementary transposition it is possible to construct arbitrary
permutations of the orbitals.  For instance, to apply the permutation
$(6 7 8 9)$\footnote{The cyclic permutation $(6 7 8 9)$ maps the
  ordered list {\tt 6 7 8 9} into {\tt 9 6 7 8}.} we note that this
permutation is equal to $(6 7)(7 8)(8 9)$, and thus may be specified
as
\begin{verbatim}
  vectors swap 8 9  7 8  6 7
\end{verbatim}

Another example, now illustrating this feature for a UHF calculation,
is the directive
\begin{verbatim}
  vectors swap beta 4 5 swap alpha 5 6
\end{verbatim}
This input will result in the swapping of the 5--6 alpha orbital pair
and the 4--5 beta orbital pair.  (All other items in the input use the
default values.)

The \verb+LOCK+ keyword allows the user to specify that the ordering
of orbitals will be locked to that of the initial vectors, insofar as
possible. The default is to order by ascending orbital energies within
each orbital space. One application where locking might be desirable
is a calculation where it is necessary to preserve the ordering of a
previous geometry, despite flipping of the orbital energies.  For such
a case, the \verb+LOCK+ directive can be used to prevent the SCF
calculation from changing the ordering, even if the orbital energies
change.

\subsection{Superposition of fragment molecular orbitals}
\label{sec:fragguess}

The fragment initial guess is particularly useful in the following
instances:
\begin{itemize}
\item The system naturally decomposes into molecules that can be
  treated individually, e.g., a cluster.
\item One or more fragments are particularly hard to converge and
  therefore much time can be saved by converging them independently.
\item A fragment (e.g., a metal atom) must be prepared with a specific
  occupation.  This can often be readily accomplished with a
  calculation on the fragment using dummy charges to model a ligand
  field.
\item The molecular occupation predicted by the atomic initial guess
  is often wrong for systems with heavy metals which may have
  partially occupied orbitals with lower energy than some doubly
  occupied orbitals.  The fragment initial guess avoids this problem.
\end{itemize}

\begin{verbatim}
  VECTORS [input] fragment <string file1> [<string file2> ...]
\end{verbatim}
The molecular orbitals are formed by superimposing the previously
generated orbitals of fragments of the molecule being studied.  These
fragment molecular orbitals must be in the same basis as the current
calculation.  The input specifies the files containing the fragment
molecular orbitals.  For instance, in a calculation on the water
dimer, one might specify
\begin{verbatim}
  vectors fragment h2o1.movecs h2o2.movecs
\end{verbatim}
where \verb+h2o1.movecs+ contains the orbitals for the first fragment, and
\verb+h2o2.movecs+ contains the orbitals for the second fragment.

A complete example of the input for a calculation on the water
dimer using the fragment guess is as follows:
\begin{verbatim}
   start dimer

   title "Water dimer SCF using fragment initial guess"

   geometry dimer
     O   -0.595   1.165  -0.048
     H    0.110   1.812  -0.170
     H   -1.452   1.598  -0.154
     O    0.724  -1.284   0.034
     H    0.175  -2.013   0.348
     H    0.177  -0.480   0.010
   end

   geometry h2o1
     O   -0.595   1.165  -0.048
     H    0.110   1.812  -0.170
     H   -1.452   1.598  -0.154
   end

   geometry h2o2
     O    0.724  -1.284   0.034
     H    0.175  -2.013   0.348
     H    0.177  -0.480   0.010
   end

   basis
     o library 3-21g
     h library 3-21g
   end

   set geometry h2o1
   scf; vectors input atomic output h2o1.movecs; end
   task scf

   set geometry h2o2
   scf; vectors input atomic output h2o2.movecs; end
   task scf

   set geometry dimer
   scf
   vectors input fragment h2o1.movecs h2o2.movecs \
           output dimer.movecs
   end
   task scf
\end{verbatim}
First, the geometry of the dimer and the two monomers are specified
and given names.  Then, after the basis specification, calculations
are performed on the fragments by setting the geometry to the
appropriate fragment (Section \ref{sec:set}) and redirecting the
output molecular orbitals to an appropriately named file.  Note also
that use of the atomic initial guess is forced, since the default
initial guess is to use any existing MOs which would not be
appropriate for the second fragment calculation.  Finally, the dimer
calculation is performed by specifying the dimer geometry, indicating
use of the fragment guess, and redirecting the output MOs.

The following points are important in using the fragment initial guess:
\begin{enumerate}
\item The fragment calculations must be in the same basis set as the
  full calculation.
\item The order of atoms in the fragments and the order in which the
  fragment files are specified must be such that when the fragment
  basis sets are concatentated all the basis functions are in the same
  order as in the full system.  This is readily accomplished by first
  generating the full geometry with atoms for each fragment
  contiguous, splitting this into numbered fragments and specifying
  the fragment MO files in the correct order on the \verb+VECTORS+
  directive.
\item The occupation of orbitals is preserved when they are merged
  from the fragments to the full molecule and the resulting occupation
  must match the requested occupation for the full molecule.  E.g., a
  triplet ROHF calculation must be comprised of fragments that have
  a total of exactly two open-shell orbtials. 
\item Because of these restrictions, it is not possible to introduce
  additional atoms (or basis functions) into fragments for the purpose
  of cleanly breaking real bonds.  However, it is possible, and highly
  recommended, to introduce additional point charges to simulate the
  presence of other fragments.
\item MO vectors of partially occupied or strongly polarized systems
  are very sensitive to orientation.  While it is possible to specify
  the same fragment MO vector file multiple times in the
  \verb+VECTORS+ directive, it is usually much better to do a separate
  calculation for each fragment.
\item Linear dependencies which were present in a fragment calculation
  may be magnified in the full calculation.  When this occurs, 
  some of the fragment's highest virtual orbitals will not be copied to the
  full system, and a warning will be printed.
  
\end{enumerate}

A more involved example is now presented.  We wish to model the sextet
state of Fe(III) complexed with water, imidazole and a heme with a net
unit positive charge.  The default atomic guess does not give the
correct $d^5$ occupation for the metal and also gives an incorrect
state for the double anion of the heme.  The following performs
calculations on all of the fragments.  Things to note are:
\begin{enumerate}
\item The use
of a dummy $+2$ charge in the initial guess on the heme which in part
simulates the presence of the metal ion, and also automatically forces
an additional two electrons to be added to the system (the default net
charge being zero).
\item The iron fragment calculation (charge +3, $d^5$, sextet) will
  yield the correct open-shell occupation for the full system.  If,
  instead, the {\it d}-orbitals were partially occupied (e.g., the doublet
  state) it would be useful to introduce dummy charges around the iron
  to model the ligand field and thereby lift the degeneracy to obtain
  the correct occupation.
\item $C_s$ symmetry is used for all of the calculations.  It is not
  necessary that the same symmetry be used in  all of the
  calculations, provided that the order and orientation of the atoms 
  is preserved.
\item The \verb+unset scf:*+ directive is used immediately before
  the calculation on the full system so that the default name for the
  output MO vector file can be used, rather than having to specify it
  explicitly.
\end{enumerate}
\begin{verbatim}
start heme6a1
title  "heme-H2O (6A1) from M.Dupuis"

############################################################
# Define the geometry of the full system and the fragments #
############################################################

geometry full-system
   symmetry cs

   H     0.438   -0.002    4.549
   C     0.443   -0.001    3.457
   C     0.451   -1.251    2.828
   C     0.452    1.250    2.828
   H     0.455    2.652    4.586
   H     0.461   -2.649    4.586
   N1    0.455   -1.461    1.441
   N1    0.458    1.458    1.443
   C     0.460    2.530    3.505
   C     0.462   -2.530    3.506
   C     0.478    2.844    1.249
   C     0.478    3.510    2.534
   C     0.478   -2.848    1.248
   C     0.480   -3.513    2.536
   C     0.484    3.480    0.000
   C     0.485   -3.484    0.000
   H     0.489    4.590    2.664
   H     0.496   -4.592    2.669

   H     0.498    4.573    0.000
   H     0.503   -4.577    0.000
   H    -4.925    1.235    0.000
   H    -4.729   -1.338    0.000
   C    -3.987    0.685    0.000
   N    -3.930   -0.703    0.000
   C    -2.678    1.111    0.000
   C    -2.622   -1.076    0.000
   H    -2.284    2.126    0.000
   H    -2.277   -2.108    0.000
   N    -1.838    0.007    0.000

   Fe    0.307    0.000    0.000

   O     2.673   -0.009    0.000
   H     3.238   -0.804    0.000
   H     3.254    0.777    0.000
end

geometry ring-only
   symmetry cs
   H     0.438   -0.002    4.549
   C     0.443   -0.001    3.457
   C     0.451   -1.251    2.828
   C     0.452    1.250    2.828
   H     0.455    2.652    4.586
   H     0.461   -2.649    4.586
   N1    0.455   -1.461    1.441
   N1    0.458    1.458    1.443
   C     0.460    2.530    3.505
   C     0.462   -2.530    3.506
   C     0.478    2.844    1.249
   C     0.478    3.510    2.534
   C     0.478   -2.848    1.248
   C     0.480   -3.513    2.536
   C     0.484    3.480    0.000
   C     0.485   -3.484    0.000
   H     0.489    4.590    2.664
   H     0.496   -4.592    2.669

   Bq    0.307    0.0      0.0    charge 2  # simulate the iron
end

geometry imid-only
   symmetry cs
   H     0.498    4.573    0.000
   H     0.503   -4.577    0.000
   H    -4.925    1.235    0.000
   H    -4.729   -1.338    0.000
   C    -3.987    0.685    0.000
   N    -3.930   -0.703    0.000
   C    -2.678    1.111    0.000
   C    -2.622   -1.076    0.000
   H    -2.284    2.126    0.000
   H    -2.277   -2.108    0.000
   N    -1.838    0.007    0.000
end

geometry fe-only
   symmetry cs
   Fe    .307    0.000    0.000
end

geometry water-only
   symmetry cs
   O     2.673   -0.009    0.000
   H     3.238   -0.804    0.000
   H     3.254    0.777    0.000
end

############################
# Basis set for everything #
############################

basis nosegment
  O  library 6-31g*
  N  library 6-31g*
  C  library 6-31g*
  H  library 6-31g*
 Fe  library "Ahlrichs pVDZ"
end

##########################################################
# SCF on the fragments for initial guess for full system #
##########################################################

scf; thresh 1e-2; end

set geometry ring-only
scf; vectors atomic swap 80 81 output ring.mo; end
task scf

set geometry water-only
scf; vectors atomic output water.mo; end
task scf

set geometry imid-only
scf; vectors atomic output imid.mo; end
task scf

charge 3
set geometry fe-only
scf; sextet; vectors atomic output fe.mo; end
task scf

##########################
# SCF on the full system #
##########################

unset scf:*     # This restores the defaults

charge 1

set geometry full-system

scf
 sextet
 vectors fragment ring.mo imid.mo fe.mo water.mo
 maxiter 50
end

task scf
\end{verbatim}

\subsection{Atomic guess orbitals with charged atoms}
\label{sec:atomscf}

As noted above, the default guess vectors are based on superimposing
the density matrices of the neutral atoms.  If some atoms are
significantly charged, this default guess may be improved upon by
modifying the atomic densities.  This is done by setting parameters
that add fractional charges to the occupation of the valence atomic
orbitals.  Since the atomic SCF program does not have its own input
block, the \verb+SET+ directive (Section \ref{sec:set}) must be used
to set these parameters.

The input specifies a list of tags (i.e., names of atoms in a
geometry, see Section \ref{sec:geom}) and the charges to be added to
those centers.  Two parameters must be set as follows:
\begin{verbatim}
  set atomscf:tags_z <string list_of_tags>
  set atomscf:z      <real list_of_charges>
\end{verbatim}

\sloppy

The array of strings \verb+atomscf:tags_z+ should be set to the list
of tags, and the array \verb+atomscf:z+ should be set to the list of
charges which must be real numbers (not integers).  All atoms that
have a tag specified in the list of tags will be assigned the
corresponding charge from the list of charges.

\fussy

For example, the following specifies that all oxygen atoms with tag
\verb+O+ be assigned a charge of \verb+-1+ and all iron atoms with tag
\verb+Fe+ be assigned a charge of \verb=+2=
\begin{verbatim}
  set atomscf:z        -1  2.0
  set atomscf:tags_z    O  Fe
\end{verbatim}

There are some limitations to this feature.  It is not possible to add
electrons to closed shell atoms, nor is it possible to remove all
electrons from a given atom.  Attempts to do so will cause the code to
report an error, and it will not report further errors in the input
for modifying the charge even when they are detected.

Finally, recall that the database is persistent (Section
\ref{sec:persist}) and that the modified settings will be used in
subsequent atomic guess calculations unless the data is deleted from
the database with the \verb+UNSET+ directive (Section
\ref{sec:unset}).

\section{Accuracy of initial guess}
\label{sec:tolguess}

For SCF, the initial Fock-matrix construction from the atomic guess is
now (staring from version 3.3) performed to a default precision of
1e-7.  However, other wavefunctions, notably DFT, use a lower
precision.  In charged, or diffuse basis sets, this precision may not
be sufficient and could result in incorrect ordering of the initial
orbitals.  The accuracy may be increased with the following directive
which should be inserted in the top-level of input (i.e., outside of
the SCF input block) and before the {\tt TASK} directive.
\begin{verbatim}
  set tolguess 1e-7
\end{verbatim}

\section{{\tt THRESH} --- convergence threshold}
\label{sec:thresh}

\begin{verbatim}
  THRESH  <real thresh default 1.0e-4>
\end{verbatim}

This directive specifies the convergence threshold for the
calculation.  The convergence threshold is the norm of the orbital
gradient, and has a default value in the code of $10^{-4}$.

The norm of the orbital gradient corresponds roughly to the precision
available in the wavefunction, and the energy should be converged to
approximately the square of this number.  It should be noted, however,
that the precision in the energy will not exceed that of the integral
screening tolerance.  This tolerance (Section \ref{sec:tol2e}) is
automatically set from the convergence threshold, so that sufficient
precision is usually available by default.

The default convergence threshold suffices for most SCF energy and
geometry optimization calculations, providing about 6--8 decimal
places in the energy, and about four significant figures in the
density and energy derivative with respect to nuclear coordinates.
However, greater precision may be required for calculations involving
weakly interacting systems, floppy molecules, finite-difference of
gradients to compute the Hessian, and for post-Hartree-Fock
calculations.  A threshold of $10^{-6}$ is adequate for most such
purposes, and a threshold of $10^{-8}$ might be necessary for very
high accuracy or very weak interactions.  A threshold of $10^{-10}$
should be regarded as the best that can be attained in most
circumstances.

\section{{\tt MAXITER} --- iteration limit}
\label{sec:max}

\begin{verbatim}
  MAXITER <integer maxiter default 8>
\end{verbatim}

\sloppy

The maximum number of iterations for the SCF calculation defaults to
20 for both ROHF/RHF and UHF calculations.  For most molecules, this
number of iterations is more than sufficient for the quadratically
convergent SCF algorithm to obtain a solution converged to the default
threshold (see Section \ref{sec:thresh} above).  If the SCF program
detects that the quad\-ratically con\-ver\-gent algorithm is not
efficient, then it will resort to a lin\-early con\-ver\-gent
algorithm and increase the maximum number of iterations by 10.

\fussy

Convergence may not be reached in the maximum number of iterations for
many reasons, including input error (e.g., an incorrect geometry or a
linearly dependent basis), a very low convergence threshold, a poor
initial guess, or the fact that the system is intrinsically hard to
converge due to the presence of many states with similar energies.

The following sets the maximum number of SCF iterations to 50:
\begin{verbatim}
  maxiter 50
\end{verbatim}

\section{{\tt PROFILE} --- performance profile}

This directive allows the user to obtain timing and parallel
execution information about the SCF module.  It is specified by the
simple keyword

\begin{verbatim}
  PROFILE
\end{verbatim}

This option can be helpful in understanding the computational
performance of an SCF calculation.  However,
it can introduce a significant overhead 
on machines that have expensive timing routines, such as the SUN.

\section{{\tt DIIS} --- DIIS convergence}

This directive allows the user to specify DIIS convergence rather than
second-order convergence for the SCF calculation.  The form of the
directive is as follows:

\begin{verbatim}
  DIIS
\end{verbatim}

The implementation of this option is currently fairly rudimentary.  It
does not have level-shifting and damping, and does not support open
shells or UHF.  It is provided on an ``as is'' basis, and should be
used with caution.

When the \verb+DIIS+ directive is specified in the input, the user has
the additional option of specifying the size of the subspace for the
DIIS extrapolation.  This is accomplished with the \verb+DIISBAS+
directive, which is of the form:
\begin{verbatim}
  DIISBAS <integer diisbas default 5>
\end{verbatim}
The default of 5 should be adequate for most applications, but may be
increased if convergence is poor.  On large systems, it may be necessary
to specify a lower value for \verb+diisbas+, to conserve memory.

\section{{\tt DIRECT} and {\tt SEMIDIRECT} --- recomputation of integrals}
\label{sec:semidirect}

In the context of SCF calculations direct means that all integrals are
recomputed as required and none are stored.  The other extreme are
disk- or memory-resident (sometimes termed conventional) calculations
in which all integrals are computed once and stored.  Semi-direct
calculations are between these two extremes with some integrals being
precomputed and stored, and all other integrals being recomputed as
necessary.

The default behavior of the SCF module is
\begin{itemize}
\item If enough memory is available, the integrals are computed once
  and are cached in memory.
\item If there is not enough memory to store all the integrals at
  once, then 95\% of the available disk space in the scratch directory
  (see Section \ref{sec:dirs}) is assumed to be available for this
  purpose, and as many integrals as possible are cached on disk (with
  no memory being used for caching).  Some attempt is made to store
  the most expensive integrals in the cache.  
 \item If there is not enough room in memory or on disk for all the
   integrals, then the ones that are not cached are recomputed in a
   semidirect fashion.
\end{itemize}

The integral file is deleted at the end of a calculation, so it is not
possible to restart a semidirect calculation when the integrals are
cached in memory or on disk.  Many computer systems (e.g., the EMSL
IBM SP) clear the fast scratch space at the end of each job, adding a
further complication to the problem of restarting a {\em parallel}
semidirect calculation.

%Under some situations, it is possible to
%restart from integrals on disk, but this capability will not be made
%widely available until a later date.

On the IBM SP or any other computer with fast disks local to each
processor, semidirect calculation offers the best behavior.  It can
result in {\em quadratic speedup} as more processors are added.  

A fully direct calculation (with recomputation of the integrals at
each iteration) is forced by specifying the directive

\begin{verbatim}
  DIRECT
\end{verbatim}

Alternatively, the \verb+SEMIDIRECT+ directive can be used to control
the default semidirect calculation by defining the amount of disk
space and the cache memory size.  The form of this directive is as
follows:

\begin{verbatim}
   SEMIDIRECT [filesize <integer filesize default disksize>] 
              [memsize  <integer memsize default available>]
              [filename <string filename default $file_prefix.aoints$>]
\end{verbatim}

The keyword \verb+FILESIZE+ allows the user to specify the amount of
disk space to be used per process for storing the integrals in 64-bit
words.  Similarly, the keyword \verb+MEMSIZE+ allows the user to
specify the number of 64-bit words to be used per process for caching
integrals in memory. (Note: If the amount of storage space specified
by the entry for \verb+memsize+ is not available, the code cuts the
value in half and checks again for available space.  This process is
repeated until the request is satisfied.)

By default, the integral files are placed into the scratch directory
(see Section \ref{sec:dirs}). Specifying the keyword \verb+FILENAME+
overrides this default.  The user-specified name entered in the string
\verb+filename+ has the process number appended to it, so that each
process has a distinct file but with a common base-name and directory.
Therefore, it is not possible to use this keyword to specify different
disks for different processes.  The \verb+SCRATCH_DIR+ directive (see
Section \ref{sec:dirs}) can be used for this purpose.

For example, to force full recomputation of all integrals:
\begin{verbatim}
  direct
\end{verbatim}

Exactly the same result could be obtained by entering the directive:
\begin{verbatim}
  semidirect filesize 0 memsize 0
\end{verbatim}

To disable the use of memory for caching integrals and limit disk
usage by each process to 100 megawords (MW):
\begin{verbatim}
  semidirect memsize 0 filesize 100000000
\end{verbatim}

The integral records are typically 32769 words long and any non-zero
value for \verb+filesize+ or \verb+memsize+ should be enough to hold
at least one record.


\subsection{Integral File Size and Format for the SCF Module}

The file format is rather complex, since it accommodates a variety of
packing and compression options and the distribution of data.  This
section presents some information that may help the user understand
the output, and illustrates how to use the output information to
estimate file sizes.

If integrals are stored with a threshold of greater than $10^{-10}$,
then the integrals are stored in a 32-bit fixed-point format (with
appropriate treatment for large values to retain precision).  If
integrals are stored with a threshold less than $10^{-10}$, however,
the values are stored in 64-bit floating-point format.  If a
replicated-data calculation is being run, then 8 bits are used for
each basis function label, unless there are more than 256 functions,
in which case 16 bits are used.  If distributed data is being used,
then the labels are always packed to 8 bits (the distributed blocks
always being less than 256; labels are relative to the start of the
block).

Thus, the number ($W$) of 64-bit words required to store $N$
integrals, may be computed as
\begin{displaymath}
  W = \left\{ \\
      \begin{array}{c}
        N \mbox{ , 8-bit labels and 32-bit values} \\
        \frac{3}{2}N \mbox{ , 16-bit labels and 32-bit values} \\
        \frac{3}{2}N \mbox{ , 8-bit labels and 64-bit values} \\
        2N \mbox{ , 16-bit labels and 64-bit values} 
      \end{array}
      \right.
\end{displaymath}

The actual number of words required can 
exceed this computed value by up to one percent, due to 
bookkeeping overhead, and because the file itself is
organized into fixed-size records.

With at least the default print level, all semidirect (not direct)
calculations will print out information about the integral file and
the number of integrals computed.  The form of this output is as
follows:

\begin{verbatim}
 Integral file          = ./c6h6.aoints.0
 Record size in doubles =  32769        No. of integs per rec  =  32768
 Max. records in memory =      3        Max. records in file   =      5
 No. of bits per label  =      8        No. of bits per value  =     32

 #quartets = 2.0D+04  #integrals = 7.9D+05  direct = 63.6%  cached = 36.4%
\end{verbatim}

The file information above relates only to process 0.  The line of
information about the number of quartets, integrals, etc., is a sum
over all processes.

When the integral file is closed, additional information of the following
form is printed:

\begin{verbatim}
------------------------------------------------------------
EAF file 0: "./c6h6.aoints.0" size=262152 bytes
------------------------------------------------------------
               write      read    awrite     aread      wait
               -----      ----    ------     -----      ----
     calls:        6        12         0         0         0
   data(b): 1.57e+06  3.15e+06  0.00e+00  0.00e+00
   time(s): 1.09e-01  3.12e-02                      0.00e+00
rate(mb/s): 1.44e+01  1.01e+02
------------------------------------------------------------

 Parallel integral file used       4 records with       0 large values
\end{verbatim}
Again, the detailed file information relates just to process 0, but
the final line indicates the total number of integral records stored
by all processes. 

This information may be used to optimize subsequent calculations, for
instance by assigning more memory or disk space.

\section{SCF Convergence Control Options}
\label{sec:scfconv}

{\em Note to users:} It is desired that the SCF program converge
reliably with the default options for a wide variety of molecules.  In
addition, it should be guaranteed to converge for any system, with
sufficient iterations.  Please report significant convergence problems
to \verb+nwchem+-\verb+support@+\-\verb+emsl.pnl.gov+, and include the
input file.

% An understanding of the output of the SCF program and the options
% controlling convergence requires some knowledge of the convergence
% scheme.

The SCF program uses a preconditioned conjugate gradient (PCG) method
that is unconditionally convergent.  Basically, a search direction is
generated by multiplying the orbital gradient (the derivative of the
energy with respect to the orbital rotations) by an approximation to
the inverse of the level-shifted orbital Hessian.  In the initial
iterations (see Section \ref{sec:nrswitch}), an inexpensive
one-electron approximation to the inverse orbital Hessian is used.
Closer to convergence, the full orbital Hessian is used, which should
provide quadratic convergence.  For both the full or one-electron
orbital Hessians, the inverse-Hessian matrix-vector product is formed
iteratively.  Subsequently, an approximate line search is performed
along the new search direction.  If the exact Hessian is being
employed, then the line search should require a single step (of
unity).  Preconditioning with approximate Hessians may require
additional steps, especially in the initial iterations.  It is the
(approximate) line search that provides the convergence guarantee.
The iterations required to solve the linear equations are referred to
as micro-iterations.  A macro-iteration comprises both the iterative
solution and a line search.

Level-shifting plays the same role in this algorithm as
it does in the conventional iterative solution of the SCF equations.
The approximate Hessian used for preconditioning should be positive
definite.  If this is not the case, then level-shifting by a positive
constant ($\Delta$) serves to make the preconditioning matrix positive
definite, by adding $\Delta$ to all of its eigenvalues.  The
level-shifts employed for the RHF orbital Hessian should be
approximately four times (only twice for UHF) the value that one would
employ in a conventional SCF\footnote{This can be seen by considering
  a one-electron approximation to the closed-shell RHF Hessian in
  canonical orbitals, $A_{ia,jb} = 4 \delta_{ij} \delta_{ab}
  (\epsilon_a - \epsilon_i)$.  Similarly, the level shift
  should be twice as large for UHF.}.  Level-shifting is automatically enabled
in the early iterations, and the default options suffice for most test
cases.

So why do things go wrong and what can be done to fix convergence
problems?  Most problems encountered so far arise either poor initial
guesses or from small or negative eigenvalues of the orbital Hessian.
The atomic orbital guess is usually very good.  However, in
calculations on charged systems, especially with open shells,
incorrect initial occupations may result.  The SCF might then converge
very slowly since very large orbital rotations might be required to
achieve the correct occupation or move charge large distances in the
molecule.  Possible actions are
\begin{itemize}
\item Modify the atomic guess by assigning charges to the atoms
  known to carry substantial charges (Section \ref{sec:atomscf})
\item Examining an analysis of the initial orbitals (Section
  \ref{sec:scfprint}) and then swapping them to attain the desired
  occupation (Section \ref{sec:vectors}).
\item Converging the calculation in a minimial basis set, which is
  usually easier, and then projecting into a larger basis set (Section
  \ref{sec:vectors}).
\item Using the fragment orbital initial guess (Section
  \ref{sec:fragguess}).
\end{itemize}

Small or negative Hessian eigenvalues can occur even though the
calculation seem to be close to convergence (as measured by the
gradient norm, or the off-diagonal Fock matrix elements).  Small
eigenvalues will cause the iterative linear equation solver to
converge slowly, resulting in an excessive number of micro-iterations.
This makes the SCF expensive in terms of computation time, and it is
possible to exceed the maximum number of iterations without achieving
the accuracy required for quadratic convergence --- which causes more
macro-iterations to be performed. 

Two main options are available when a problem will not converge:
Newton-Raphson can be disabled temporarily or permanently (see Section
\ref{sec:nrswitch}), and level-shifting can be applied to the matrix
(see Section \ref{sec:level}).  In some cases, both options may be
necessary to achieve final convergence.

If there is reason to suspect a negative eigenvalue, the first course
is to disable the Newton-Raphson iteration until the solution is
closer to convergence.  It may be necessary to disable it completely.
At some point close to convergence, the Hessian will be positive
definite, so disabling Newton-Raphson should yield a solution with
approximately the same convergence rate as DIIS.

If temporarily disabling Newton-Raphson is not sufficient to achieve
convergence, it may be necessary to disable it entirely and apply a
small level-shift to the approximate Hessian.  This should improve the
convergence rate of the micro-iterations and stabilize the
macro-iterations.  The level-shifting will destroy exact quadratic
convergence, but the optimization process is automatically adjusted to
reflect this by enforcing conjugacy and reducing the accuracy to which
the linear equations are solved.  The net result of this is that the
solution will do more macro-iterations, but each one should take less
time than it would with the unshifted Hessian.

The following sections describe the directives needed to disable the
Newton-Raphson iteration and specify level-shifting.

\section{{\tt NR} --- controlling the Newton-Raphson}
\label{sec:nrswitch}

\begin{verbatim}
    NR <real nr_switch default 0.1>
\end{verbatim}

The exact orbital Hessian is adopted as the preconditioner when the
maximum element of the orbital gradient is below the value specified
for \verb+nr_switch+.  The default value is 0.1, which means that
Newton-Raphson will be disabled until the maximum value of the orbital
gradient (twice the largest off-diagonal Fock matrix element) is less
than 0.1.   To disable the Newton-Raphson entirely, the
value of \verb+nr_switch+ must be set to zero.  The directive to accomplish
this is as follows:
\begin{verbatim}
  nr 0
\end{verbatim}

\section{{\tt LEVEL} --- level-shifting the orbital Hessian}
\label{sec:level}

This directive allows the user to specify level-shifting to obtain a
positive-definite preconditioning matrix for the SCF solution
procedure.  Separate level shifts can be set for the first-order
convergent one-electron approximation to the Hessian used with the
preconditioned conjugate gradient (PCG) method, and for the full
Hessian used with the Newton-Raphson (NR) approach.  It is also
possible to change the level-shift automatically as the solution
attains some specified accuracy.  The form of the directive is as
follows:
\begin{verbatim}
   LEVEL [pcg <real initial default 20.0> \
           [<real tol default 0.5> <real final default 0.0>]] \
         [nr <real initial default 0.0> \
           [<real tol default 0.0> <real final default 0.0>]]
\end{verbatim}

This directive contains only two keywords: one for the PCG method and
the other for the exact Hessian (Newton Raphson, or NR).  Use of PCG
or NR is determined by the input specified for \verb+nr_switch+ on the
\verb+NR+ directive, Section \ref{sec:nrswitch} above.  

Specifying the keyword \verb+pcg+ on the \verb+LEVEL+ directive allows
the user to define the level shifting for the approximate (i.e., PCG)
method.  Specifying the keyword \verb+nr+ allows the user to define
the level shifting for the exact Hessians.  In both options, the
initial level shift is defined by the value specified for the variable
\verb+initial+.  Optionally, \verb+tol+ can be specified independently
with each keyword to define the level of accuracy that must be
attained in the solution before the level shifting is changed to the
value specified by input in the real variable \verb+final+.  Level
shifts and gradient thresholds are specified in atomic units.

For the PCG method (as specified using the keyword \verb+pcg+), the
defaults for this input are 20.0 for \verb+initial+, 0.5 for
\verb+tol+, and 0.0 for \verb+final+.  This means that the
approximate Hessian will be shifted by 20.0 until the maximum element
of the gradient falls below 0.5, at which point the shift will be set
to zero.

For the exact Hessian (as specified using the keyword \verb+nr+), the
defaults are all zero.  The exact Hessian is usually not shifted since
this destroys quadratic convergence.  An example of an input directive
that applies a shift of 0.2 to the exact Hessian is as follows:
\begin{verbatim}
  level nr 0.2
\end{verbatim}

To apply this shift to the exact Hessian only until the maximum
element of the gradient falls below 0.005, the required input
directive is as follows:
\begin{verbatim}
  level nr 0.2 0.005 0
\end{verbatim}

Note that in both of these examples, the parameters for the PCG method
are at the default values.  To obtain values different from the
defaults, the keyword \verb+pcg+ must also be specified.  For example,
to specify the level shifting in the above example for the exact
Hessian {\em and} non-default shifting for the PCG method, the
directive would be something like the following:
\begin{verbatim}
  level pcg 20 0.3 0.0 nr 0.2 0.005 0.0
\end{verbatim}

This input will cause the PCG method to be level-shifted by 20.0 until
the maximum element of the gradient falls below 0.3, then the shift
will be zero.  For the exact Hessian, the level shifting is initially
0.2, until the maximum element falls below 0.005, after which the
shift is zero. 

The default options correspond to
\begin{verbatim}
  level pcg 20 0.5 0 nr 0 0 0
\end{verbatim}

\section{Orbtial Localization}
\label{orbloc}
The SCF module includes an {\em experimental} implementation of
orbital localization, including Foster-Boys and Pipek-Mezey which only
works for closed-shell (RHF) wavefunctions. There is currently no
input in the SCF block to control this so the \verb+SET+ directive
(Section \ref{sec:set}) must be used.

The directive
\begin{verbatim}
  set scf:localize t
\end{verbatim}
will separately localize the core, valence, and virtual orbital spaces
using the Pipek-Mezey algorithm.  If the additional directive
\begin{verbatim}
  set scf:loctype FB
\end{verbatim}
is included, then the Foster-boys algorithm is used.  The partitioning
of core-orbitals is performed using the atomic information described
in Section \ref{mp2:core}.

In the next release, this functionality will be extended to included all
wavefunctions using molecular orbitals.


\newpage
\section{Printing Information from the SCF Module}
\label{sec:scfprint}

All output from the SCF module is controlled using the \verb+PRINT+
directive described in Section \ref{sec:printcontrol}.  The following 
list describes the items from SCF that are currently under direct 
print control, along with the print level for each one.

\begin{table}[htbp]
\begin{center}
\begin{tabular}{lcc}
  {\bf Name}          & {\bf Print Level} & {\bf Description} \\
 ``atomic guess density''     & debug     & guess density matrix \\
 ``atomic scf''               & debug     & details of atomic SCF \\
 ``mo guess''                 & default   & brief info from mo guess \\
 ``information''              & low       & results  \\
 ``initial vectors''          & debug     & \\
 ``intermediate vectors''     & debug     & \\
 ``final vectors''            & debug     & \\
 ``final vectors analysis''   & default   & \\
 ``initial vectors analysis'' & never     & \\
 ``intermediate evals''       & debug     & \\
 ``final evals''              & default   & \\
 ``schwarz''                  & high      & integral screening info \& stats at completion\\
 ``screening statistics''     & debug     & display stats after every Fock build \\
 ``geometry''                 & high      & \\
 ``symmetry''                 & debug     & detailed symmetry info \\
 ``basis''                    & high      & \\
 ``geombas''                  & debug     & detailed basis map info \\
 ``vectors i/o''              & default   & report vectors I/O \\
 ``parameters''               & default   & convergence parameters \\
 ``convergence''              & default   & info each iteration
\end{tabular}
\end{center}
\caption{SCF Print Control Specifications}
\end{table}

\newpage
\section{Hartree-Fock or SCF, MCSCF and MP2 Gradients}

\label{sec:scfgrad}

\begin{verbatim}
  GRADIENTS 
    [chkpt <integer minutes>]
    [restart]
    [print || noprint]
  END
\end{verbatim}

%  This input controls the Hartree-Fock (SCF, UHF and ROHF) gradients.  

The input for this directive allows the user to define 
characteristics of the Hartree-Fock gradients for the SCF, UHF and ROHF
calculations.  The form of the directive is as follows;



The directive contains two keywords, \verb+chkpt+ and \verb+restart+,
that are related to the creation of the gradients.  The keyword \verb+chkpt+
allows the user to specify a time interval at which the current values
for the forces that make up 
the gradient are saved, for access by a later calculation.  The time
interval is specified in the integer variable \verb+minutes+, and defines
the number of minutes of elapsed wall-clock time since the start of the
calculation when the gradient is written to the runtime database.

% \subsection{CHKPT}

%  This keyword is used to specify a time interval after which a
%  checkpoint for later restart is created. After \verb+minutes+
%  minutes of walltime the forces are written to the runtime database.

% \subsection{RESTART}

Specifying the keyword \verb+restart+ allows the user to restart a calculation
using the gradient calculated from a previous calculation that may have
aborted for some reason.  (This implies, of course, that the previous
calculation employed the keyword \verb+chkpt+ with a value specified for
the variable \verb+minutes+ that allowed the calculation to write out the
gradient before failing.)  The keyword \verb+restart+ allows the partially
calculated forces from the previous calculation to be used as the starting
point for the new calculation.  If the gradient was not saved previously,
however, this keyword has no affect.  The gradients area automatically 
recalculated from zero.

%  This keyword tells the program that this is a restart of an aborted
%  gradient calculation. The partially calculated forces are taken from
%  the database of the previous run. If they are not present, the
%  keyword is ignored and a complete calculation of the gradients is started.

  It also works within a geometry optimization. Subsequent gradient 
  calculations are not treated as restarts.
\Large  **Elucidate.**
\normalsize

% \subsection{PRINT, NOPRINT}

The complementary keyword pair \verb+print+ and \verb+noprint+ allow the 
user some additional control on the information that can be obtained from
the SCF calculation.  Currently, only a few items can be explicitly invoked
via print control.  These are as follows;
 
%  Currently only some print control is available.

\begin{tabbing}
  Very\_long\_descriptive\_name \= Print level space \= \kill
  Name                   \> Print Level \> Description \\
                         \>        \> \\
        'information'   \>        low  \> calculation info\\
        'geometry'    \>          high \> \\
        'basis'        \>         high \> \\
        'forces'   \>             low \> \\
        'timing'   \>             default \> 
\end{tabbing}






\chapter{DFT for Molecules (DFT)}
\label{sec:dft}

The NWChem density functional theory (DFT) module uses the
Gaussian basis set approach to compute
closed shell and open shell densities and Kohn-Sham orbitals
in the: 
\begin{itemize}
\item local density approximation (LDA), 
\item non-local density approximation (NLDA), 
\item local spin-density approximation (LSD), 
\item non-local spin-density approximation (NLSD), and
\item any empirical mixture of local and non-local approximations 
(including exact exchange).
\end{itemize}

The formal scaling of the DFT computation can be reduced by choosing
to use auxiliary Gaussian basis sets to fit the charge density (CD) and/or 
fit the exchange-correlation (XC) potential.

DFT input is provided using the compound \verb+DFT+ directive
\begin{verbatim}
  DFT
    ...
  END
\end{verbatim}
The actual DFT calculation will be performed when the input module
encounters the \verb+TASK+ directive (Section \ref{sec:task}).  
\begin{verbatim}
  TASK DFT
\end{verbatim}

Once a user has specified a geometry and a Kohn-Sham orbital basis set
the DFT module can be invoked with no input directives (defaults 
invoked throughout).  There are sub-directives which allow for 
customized application; those currently provided as options for 
the DFT module are:
\begin{verbatim}
  VECTORS [[input] (<string input_movecs default atomic>) || \
                   (project <string basisname> <string filename>)] \
           [swap [alpha||beta] <integer vec1 vec2> ...] \
           [output <string output_filename default input_movecs>] \
           [lock]


  XC [[acm] [b3lyp] [beckehandh] \
      [HFexch <real prefactor default 1.0>] \
      [becke88 [nonlocal] <real prefactor default 1.0>] \
      [lyp <real prefactor default 1.0>] \
      [perdew81 <real prefactor default 1.0>] \
      [perdew86 [nonlocal] <real prefactor default 1.0>] \
      [perdew91 [nonlocal] <real prefactor default 1.0>] \
      [pw91lda <real prefactor default 1.0>] \
      [slater <real prefactor default 1.0>] \
      [vwn_1 <real prefactor default 1.0>] \
      [vwn_2 <real prefactor default 1.0>] \
      [vwn_3 <real prefactor default 1.0>] \
      [vwn_4 <real prefactor default 1.0>] \
      [vwn_5 <real prefactor default 1.0>] \
      [vwn_1_rpa <real prefactor default 1.0>]]


  CONVERGENCE [[energy <real energy default 1e-7>] \
               [density <real density default 1e-5>] \
               [gradient <real gradient default 5e-4>] \
               [dampon <real dampon default 0.0>] \
               [dampoff <real dampoff default 0.0>] \
               [diison <real diison default 0.0>] \
               [diisoff <real diisoff default 0.0>] \
               [levlon <real levlon default 0.0>] \
               [levloff <real levloff default 0.0>] \
               [ncydp <integer ncydp default 2>] \
               [ncyds <integer ncyds default 30>] \
               [ncysh <integer ncysh default 30>] \
               [damp <integer ndamp default 70>] [nodamping] \
               [diis [nfock <integer nfock default 10>]] \
               [nodiis] [lshift <real lshift default 0.5>] \
               [nolevelshifting] \
               [hl_tol <real hl_tol default 0.1>]]


  GRID [(xcoarse||coarse||medium||fine||xfine) default medium] \
       [(gausleg <integer radpts default 50> 
                 <integer nagrid default 10>) ||\ 
        (lebedev <integer radpts default 50> 
                 <integer iangquad default 4>)] \ 
       [delley||becke] \
       [rm <real rm default 2.0>]
        

  TOLERANCES [[tight] [tol_rho <real tol_rho default 1e-10>] \
              [accAOfunc <integer accAOfunc default 20>] \
              [accCDfunc <integer accAOfunc default 20>] \
              [accXCfunc <integer accXCfunc default 20>] \
              [accCoul <integer accCoul default 10>] \
              [accQrad <integer accQrad default 12>] \
              [radius <real radius default 25.0>]]


  DECOMP
  DFT||ODFT
  DIRECT
  INCORE
  ITERATIONS <integer iterations default 30>
  MAX_OVL
  MULLIKEN
  MULT <integer mult default 1>
  NOIO
  PRINT||NOPRINT
\end{verbatim}
%       [store_wght] [nquad_task <integer nquad_task default 1>] \

The following 
sections describe these keywords and
optional sub-directives that can be specified for a \verb+DFT+ calculation
in NWChem.

\section{Specification of Basis Sets for the DFT Module}

The DFT module requires at a minimum the basis set for the Kohn-Sham 
molecular orbitals.  This basis set must be in the default basis set named
{\tt "ao basis"}, or it must be assigned to this default name using the
\verb+SET+ directive (see Section \ref{sec:set}).

In addition to the basis set for the Kohn-Sham orbitals, 
the charge density fitting basis set can also be specified in the 
input directives for the DFT module.  This basis set is used for the 
evaluation of the Coulomb potential in the Dunlap scheme\footnote{B.I.~Dunlap, 
J.W.D.~Connolly and J.R.~Sabin, J.~Chem.~Phys.~{\bf 71},  4993 (1979)}.
The charge density fitting basis set must have the name {\tt "cd basis"}.
This can be the actual name of a basis set, or a basis set can be 
assigned this name using the \verb+SET+ directive, as described in
Section \ref{sec:set}.  If this basis set is not defined by input,
the $O(N^4)$ exact Coulomb contribution is computed.

The user also has the option of specifying a third basis set for the 
evaluation of the exchange-correlation potential.  This basis set must
have the name {\tt "xc basis"}.  If this basis set is not specified
by input, the exchange contribution (XC) is evaluated by numerical
quadrature.  In most applications, this approach is efficient enough,
so the {\tt "xc basis"} basis set is not generally required.

For the DFT module, the input options for defining the basis sets in a given
calculation can be summarized as follows;
\begin{itemize}
\item {\tt "ao basis"} -- Kohn-Sham molecular orbitals; required for all 
calculations
\item {\tt "cd basis"} -- charge density fitting basis set; optional, but
recommended for evaluation of the Coulomb potential
\item {\tt "xc basis"} -- exchange-correlation (XC) fitting basis set; 
optional, and usually not recommended
\end{itemize}


\section{{\tt VECTORS} and {\tt MAX\_OVL} --- KS-MO Vectors}

The \verb+VECTORS+ directive is the same as that in the SCF module
(Section \ref{sec:vectors}).  Currently, the \verb+LOCK+ keyword
is not supported by the DFT module, however the directive
\begin{verbatim}
  MAX_OVL
\end{verbatim}
has the same effect.

\section{{\tt XC} and {\tt DECOMP} --- Exchange-Correlation Potentials}
\begin{verbatim}
  XC [[acm] [b3lyp] [beckehandh] \
      [HFexch <real prefactor default 1.0>] \
      [becke88 [nonlocal] <real prefactor default 1.0>] \
      [lyp <real prefactor default 1.0>] \
      [perdew81 <real prefactor default 1.0>] \
      [perdew86 [nonlocal] <real prefactor default 1.0>] \
      [perdew91 [nonlocal] <real prefactor default 1.0>] \
      [pw91lda <real prefactor default 1.0>] \
      [slater <real prefactor default 1.0>] \
      [vwn_1 <real prefactor default 1.0>] \
      [vwn_2 <real prefactor default 1.0>] \
      [vwn_3 <real prefactor default 1.0>] \
      [vwn_4 <real prefactor default 1.0>] \
      [vwn_5 <real prefactor default 1.0>] \
      [vwn_1_rpa <real prefactor default 1.0>]]
\end{verbatim}

The user has the option of specifying the exchange-correlation
treatment in the DFT Module.  The default exchange-correlation
functional is defined as the local density approximation (LDA) for
closed shell systems and its counterpart the local spin-density (LSD)
approximation for open shell systems.  Within this approximation the
exchange functional is the Slater $\rho^{1/3}$ functional (from
J.C.~Slater, {\sl Quantum Theory of Molecules and Solids, Vol.~4: The
  Self-Consistent Field for Molecules and Solids} (McGraw-Hill, New
York, 1974)), and the correlation functional is the Vosko-Wilk-Nusair
(VWN) functional (functional V) (S.J.~Vosko, L.~Wilk and M.~Nusair,
Can.~J.~Phys.~{\bf 58}, 1200 (1980)).  The parameters used in this
formula are obtained by fitting to the Ceperley and
Alder\footnote{D.M.~Ceperley and B.J.~Alder, Phys. Rev. Lett. {\bf
    45}, 566 (1980).}
Quantum Monte-Carlo solution of the {\em
  homogeneous electron gas}.

These defaults can be invoked explicitly by specifying the following
keywords within the DFT module input directive,
\begin{verbatim}
  XC slater vwn_5
\end{verbatim}

The \verb+DECOMP+ directive causes the components of the energy
corresponding to each functional to be printed, rather than just the
total exchange-correlation energy which is the default.

Many alternative exchange and correlation functionals are available to
the user.  The following sections describe these options.

\subsection{Optional Exchange Functionals}

There are two exchange functionals in addition to the default exchange
functional.  These are the Becke gradient-corrected functional (see A.D.~Becke, 
J.~Chem.~Phys.~88, 3098 (1988)), and the Hartree-Fock exact exchange.

The Becke gradient-corrected functional is invoked by specifying the input
line,
\begin{verbatim}
   XC becke88
\end{verbatim}

The Hartree-Fock exact exchange functional, (which has $O(N^4)$
computation expense), is invoked by specifying the input line,
\begin{verbatim}
   XC HFexch
\end{verbatim}

Note that the user also has the ability to include only the local or
nonlocal contributions of a given functional.  In addition the user
can specify a multiplicative prefactor (the variable
\verb+<prefactor>+ in the input) for the local/nonlocal component or
total.  An example of this might be,
\begin{verbatim}
   XC becke88 nonlocal 0.72
\end{verbatim}
The user should be aware that the Becke88 local component is simply
the Slater exchange and should be input as such.

Any combination of the supported exchange functional options can be
used.  For example the popular Gaussian B3 exchange could be specified
as:
\begin{verbatim}
   XC slater 0.8 becke88 nonlocal 0.72 HFexch 0.2
\end{verbatim}

  
\subsection{Optional Correlation Functionals}

In addition to the default \verb+vwn_5+ correlation functional, the user has
10 alternative correlation functionals to choose from: lyp, perdew81,
perdew86, perdew91, pw91lda, \verb+vwn_1+, \verb+vwn_2+, \verb+vwn_3+,
\verb+vwn_4+, and \verb+vwn_1_rpa+.

As in the exchange functional input, individual local/nonlocal
components as well as multiplicative prefactors can be invoked where
appropriate.  Each of the correlation functionals is listed below along with
appropriate citation. 

\sloppy

\begin{itemize}
\item VWN local density functionals; S.J.~Vosko, L.~Wilk and M.~Nusair, 
  Can.~J.~Phys.~{\bf  58}, 1200 (1980); all five (5) functionals as
  described in this paper (addressed in the paper as I - V) have been
  implemented.  These functionals can be invoked with the keywords:
\begin{verbatim}
   XC vwn_1
   XC vwn_2
   XC vwn_3
   XC vwn_4
   XC vwn_5
\end{verbatim}

  Note that functionals; \verb+vwn_2+, \verb+vwn_3+, and \verb+vwn_4+
  require both sets of parameters (the Monte Carlo parameters of
  Ceperley and Alder and VWN's RPA parameters) used in fitting the
  homogeneous electron gas correlation energy.  Functionals
  \verb+vwn_1+ and \verb+vwn_5+ require only the Monte Carlo fitting
  parameters.  In order to reproduce results in the literature another
  functional was added; the \verb+vwn_1_rpa+.  This is the original
  \verb+vwn_1+ functional with RPA parameters as opposed to the
  prescribed Monte Carlo parameters.  This functional can be invoked
  with the keyword,
\begin{verbatim}
   XC vwn_1_rpa
\end{verbatim}

\item Perdew81 local density functional; J.~P.~Perdew and A.~Zunger,
  Phys.~Rev.~B {\bf23}, 5048 (1981). This functional can be invoked with the
  keyword,
\begin{verbatim}
   XC perdew81
\end{verbatim}

\item Perdew \& Wang 1991 local density functional;  J.P.~Perdew
  and Y.~Wang, Phys. Rev. B {\bf 45}, 13244 (1992).  The parameters
  used in this formula are obtained by fitting to the Ceperley and
  Alder Quantum Monte Carlo solution of the {\em
  homogeneous electron gas}.  This functional can be invoked with the
  keyword,
\begin{verbatim}
   XC pw91lda
\end{verbatim}

\item Perdew86 gradient-corrected functional; J.~P.~Perdew, Phys.~Rev.~B 
  {\bf33}, 8822 (1986).  Note that this is a nonlocal functional and
  in the absence of any local functional specification the local
  component is defaulted to the perdew81 local correlation
  functional. This functional can be invoked with the
  keyword,
\begin{verbatim}
   XC perdew86
\end{verbatim}

\item Perdew91 gradient-corrected functional;  J.P.~Perdew,
  J.A.~Chevary, S.H.~Vosko, K.A.~Jackson, M.R.~Pederson, D.J.~Singh
  and C.~Fiolhais, Phys. Rev. B {\bf 46}, 6671 (1992). Note that this
  is a nonlocal functional and in the absence of any local functional 
  specification the local component is defaulted to the \verb+pw91lda+ local 
  correlation functional.  This functional can be invoked with the keyword,
\begin{verbatim}
   XC perdew91
\end{verbatim}

\item LYP gradient-corrected functional; C.~Lee, W.~Yang and
   R.~G.~Parr, Phys.~Rev.~B {\bf 37}, 785 (1988).  Note that this
  is a local and nonlocal functional but cannot be conveniently split
  into the individual components.  The option to scale the total remains.
  This functional can be invoked with the keyword,
\begin{verbatim}
   XC lyp
\end{verbatim}

\end{itemize}

\fussy

Any combination of the supported correlation functional options can be
used.  For example the correlation component of the popular B3LYP
could be specified as:
\begin{verbatim}
   XC vwn_1_rpa 0.19 lyp 0.81
\end{verbatim}

  
\subsection{Hybrid Exchange and Correlation Functionals}

In addition to the options listed above for the exchange and correlation
functionals, the user has the alternative of specifying a ``canned'' hybrid
functional.  The available hybrid functionals consist of the Becke
``{\sl half and half}'' (see A.D.~Becke, J.~Chem.~Phys.~98, 1372 (1992)), the
adiabatic connection method (see A.D.~Becke, J.~Chem.~Phys.~98, 5648
(1993)), and the b3lyp (popularized by Gaussian9X).

These options can be invoked by specifying any of the following input lines,
\begin{verbatim}
   XC beckehandh
   XC acm
   XC b3lyp
\end{verbatim}

The keyword \verb+beckehandh+ specifies that the exchange-correlation energy will be
computed as 
\begin{eqnarray*}
E_{XC} \ \approx \ \frac{1}{2} E^{\rm HF}_X + \frac{1}{2} E^{\rm
  Slater}_{X} + \frac{1}{2} E^{\rm PW91LDA}_{C}
\end{eqnarray*}
We know this is NOT the correct Becke perscribed implementation which
requires the XC potential in the energy expression.  But this is what
is currently implemented as an approximation to it.



The keyword \verb+acm+ specifies that the exchange-correlation energy
is computed as
\begin{eqnarray*}
E_{XC} \ &=& \ a_0 E^{\rm HF}_X + (1-a_0) E^{\rm Slater}_{X} +
a_X \Delta E^{\rm Becke88}_{X} + E^{\rm VWN}_C + a_C \Delta E^{Perdew91}_C \\
& &{\rm where } \\
a_0 &=& 0.20, \ a_X = 0.72, \ a_C = 0.81
\end{eqnarray*}
and $\Delta$ stands for a non-local component.


The keyword \verb+b3lyp+ specifies that the exchange-correlation energy
is computed as
\begin{eqnarray*}
E_{XC} \ &=& \ a_0 E^{\rm HF}_X + (1-a_0) E^{\rm Slater}_{X} +
a_X \Delta E^{\rm Becke88}_{X} + (1-a_C)E^{\rm \verb+VWN_1_RPA+}_C + a_C E^{LYP}_C \\
& &{\rm where } \\
a_0 &=& 0.20, \ a_X = 0.72, \ a_C = 0.81
\end{eqnarray*}


\section{{\tt ITERATIONS} --- Number of SCF iterations}

\begin{verbatim}
  ITERATIONS <integer iterations default 30>
\end{verbatim}

The default optimization in the DFT module is to iterate on the 
Kohn-Sham (SCF) equations for a specified number of iterations
(default 30).  The keyword that controls this optimization 
is \verb+ITERATIONS+, and has the following general form,

\begin{verbatim}
   iterations <integer iterations default 30>
\end{verbatim}

The optimization procedure will stop when the specified number of
iterations is reached or convergence is met.

\section{{\tt CONVERGENCE} --- SCF Convergence Control}

\begin{verbatim}
  CONVERGENCE [energy <real energy default 1e-6>] \
              [density <real density default 1e-5>] \
              [gradient <real gradient default 5e-4>] \
              [hl_tol <real hl_tol default 0.1>]
              [dampon <real dampon default 0.0>] \
              [dampoff <real dampoff default 0.0>] \
              [ncydp <integer ncydp default 2>] \
              [ncyds <integer ncyds default 30>] \
              [ncysh <integer ncysh default 30>] \
              [damp <integer ndamp default 70>] [nodamping] \
              [diison <real diison default 0.0>] \
              [diisoff <real diisoff default 0.0>] \
              [(diis [nfock <integer nfock default 10>]) || nodiis] \
              [levlon <real levlon default 0.0>] \
              [levloff <real levloff default 0.0>] \
              [(lshift <real lshift default 0.5>) || nolevelshifting] \
\end{verbatim}

Convergence is satisfied by meeting any or all of three criteria;
\begin{itemize}
\item convergence of the total energy; this is defined to be when the
  total DFT energy at iteration N and at iteration N-1 differ by a value less
  than some value (the default is 1e-6).  This value can be modified
  using the key word,
\begin{verbatim}
  CONVERGENCE energy <real energy default 1e-6>
\end{verbatim}

\item convergence of the total density; this is defined to be when the
  total DFT density matrix at iteration N and at iteration N-1 have a
  RMS difference less than some value (the default is 1e-5).  This value can be modified
  using the key word,
\begin{verbatim}
  CONVERGENCE density <real density default 1e-5>
\end{verbatim}

\item convergence of the orbital gradient; this is defined to be when the
  DIIS error vector becomes less than some value (the default is
  5e-4).  This value can be modified using the key word,
\begin{verbatim}
  CONVERGENCE gradient <real gradient default 5e-4>
\end{verbatim}
\end{itemize}

The default optimization strategy is to immediately begin direct
inversion of the iterative subspace\footnote {P.~Pulay, Chem.\ Phys.\ 
  Lett.\ {\bf 73}, 393 (1980) and P.~Pulay, J.~Comp.~Chem.~{\bf 3},
  566 (1982)}.  Damping is also initiated (using 70\% of the previous
density) for the first 2 iteration.  In addition, if the HOMO - LUMO
gap is small and the Fock matrix somewhat diagonally dominant, then
level-shifting is automatically initiated.  There are a variety of ways
to customize this procedure to whatever is desired.

An alternative optimization strategy is to specify, by using the change 
in total energy (from iterations when N and N-1), when to turn
damping, level-shifting, and/or DIIS on/off.  Start and stop keywords for
each of these is available as,
\begin{verbatim}
  CONVERGENCE  [dampon <real dampon default 0.0>] \
               [dampoff <real dampoff default 0.0>] \
               [diison <real diison default 0.0>] \
               [diisoff <real diisoff default 0.0>] \
               [levlon <real levlon default 0.0>] \
               [levloff <real levloff default 0.0>]
\end{verbatim}

So, for example, damping, DIIS, and/or level-shifting can be turned
on/off as desired.

Another strategy can be to simply specify how many iterations (cycles) you wish
each type of procedure to be used.  The necessary keywords to control
the number of damping cycles (ncydp), the number of DIIS cycles
(ncyds), and the number of level-shifting cycles (ncysh) are input as,
\begin{verbatim}
  CONVERGENCE  [ncydp <integer ncydp default 2>] \
               [ncyds <integer ncyds default 30>] \
               [ncysh <integer ncysh default 0>]
\end{verbatim}

The amount of damping, level-shifting, time at which level-shifting is
automatically imposed, and Fock matrices used in the DIIS
extrapolation can be modified by the following keywords
\begin{verbatim}
  CONVERGENCE  [damp <integer ndamp default 70>] \
               [diis [nfock <integer nfock default 10>]] \
               [lshift <real lshift default 0.5>] \
               [hl_tol <real hl_tol default 0.1>]]
\end{verbatim}

Damping is defined to be the percentage of the previous iterations
density mixed with the current iterations density.  So, for example 
\begin{verbatim}
  CONVERGENCE damp 70
\end{verbatim}
would mix 30\% of the current iteration density with 70\% of the
previous iteration density.

Level-Shifting\footnote {M.F.~Guest and 
V.R.~Saunders, Mol.~Phys.~{\bf 28}, 819 (1974)} is defined as the
amount of shift applied to the diagonal elements of the unoccupied
block of the Fock matrix.  The shift is specified by the
keyword \verb+lshift+.  For example the directive,
\begin{verbatim}
  CONVERGENCE lshift 0.5
\end{verbatim}
causes the diagonal elements of the Fock matrix
corresponding to the virtual orbitals to be shifted by 0.5 au.
By default, this level-shifting procedure is switched on whenever the
HOMO-LUMO gap is small.  Small is defined by default to be 0.05 au but
can be modified by the directive \verb+hl_tol+.  An example of
changing the HOMO-LUMO gap tolerance to 0.01 would be,
\begin{verbatim}
  CONVERGENCE hl_tol 0.01
\end{verbatim}

Direct inversion of the iterative subspace with extrapolation of up to
10 Fock matrices is a default optimization procedure.  For large
molecular systems the amount of available memory may preclude the ability to
store this number of N**2 arrays in global memory.  The user may then
specify the number of Fock matrices to be used in the extrapolation
(must be greater than three (3) to be effective).  To set the number of
Fock matrices stored and used in the extrapolation procedure to 3
would take the form,
\begin{verbatim}
  CONVERGENCE diis 3
\end{verbatim}

Finally, the user has the ability to simply turn off any optimization
procedures deemed undesirable with the obvious keywords,
\begin{verbatim}
  CONVERGENCE [nodamping] [nodiis] [nolevelshifting]
\end{verbatim}


\section{{\tt GRID} --- Numerical Integration of the XC Potential}

\begin{verbatim}
  GRID [(xcoarse||coarse||medium||fine||xfine) default medium] \
       [(gausleg <integer radpts default 50> 
                 <integer nagrid default 10>) ||\ 
        (lebedev <integer radpts default 50> 
                 <integer iangquad default 4>)] \ 
       [delley||becke] \
       [rm <real rm default 2.0>]
\end{verbatim}

A numerical integration is necessary for the evaluation of the
exchange-correlation contribution to the density functional.  The
default quadrature used for the numerical integration is an
Euler-MacLaurin scheme for the radial components and a Gauss-Legendre
scheme for the angular components (see C.W.~Murray, N.C.~Handy, and
G.L.Laming, Mol.~Phys.~78, 997-1014, (1993)).  Within this numerical
integration procedure various levels of accuracy have been defined and
are available to the user.  The user can specify the level of accuracy
with the keywords; xcoarse, coarse, medium, fine, and xfine.  The
default is medium.

\begin{verbatim}
  GRID [xcoarse||coarse||medium||fine||xfine]
\end{verbatim}

Our intent is to have a numerical integration scheme which would give
us approximately the accuracy defined below regardless of molecular
composition.  
\begin{center}
  \begin{tabular}[right]{|c|c|} \hline
Keyword & {\tt Total Energy Target Accuracy} \\ \hline
{\tt xcoarse} & $1x10^{-4}$ \\ \hline
{\tt coarse}  & $1x10^{-5}$ \\ \hline
{\tt medium}  & $1x10^{-6}$ \\ \hline
{\tt fine}    & $1x10^{-7}$ \\ \hline
{\tt xfine}   & $1x10^{-8}$ \\ \hline
  \end{tabular} \\
\end{center}

We computed total DFT energies at the LDA level of theory for many
homonuclear atomic, diatomic and triatomic systems in rows 1-4 of the
periodic table.  In each case all bond lengths were set to twice the
Bragg-Slater radius.  The total DFT energy of the system was computed
using the converged SCF density with atoms having radial shells
ranging from 35-235 (at fixed 48/96 angular quadratures) and angular
quadratures of 12/24-48/96 (at fixed 235 radial shells).  The error of
the numerical integration was determined by comparison to a ``best''
or most accurate calculation in which a grid of 235 radial points 48
theta and 96 phi angular points on each atom was used.  This
corresponds to approximately 1 million points per atom.  The following
tables were empirically determined to give the desired target accuracy
for DFT total energies.  These tables show the number of radial and
angular shells which the DFT module will use for for a given atom
depending on the row it is in (in the periodic table) and the desired
accuracy.  Note, differing atom types in a given molecular system will
most likely have differing associated numerical grids.  The intent is
to generate the desired energy accuracy.

\begin{table}[h]
\begin{center}
\caption{Number of radial and angular shells required for Row 1 atoms
  (Li $\rightarrow$ F) to reach the desired accuracies.}

\vspace{.2in}

  \begin{tabular}[right]{|c|c|c|c|} \hline
Keyword & {\tt Radial} & {\tt Theta} & {\tt phi} \\ \hline
{\tt xcoarse} & 30 & 12 & 24  \\ \hline
{\tt coarse}  & 50 & 15 & 30  \\ \hline
{\tt medium}  & 70 & 18 & 36  \\ \hline
{\tt fine}    &100 & 24 & 48  \\ \hline
{\tt xfine}   &140 & 34 & 68  \\ \hline
  \end{tabular} \\
\end{center}
\end{table}

\begin{table}[h]
\begin{center}
\caption{Number of radial and angular shells required for Row 2 atoms
  (Na $\rightarrow$ Cl) to reach the desired accuracies.}

\vspace{.2in}

  \begin{tabular}[right]{|c|c|c|c|} \hline
Keyword & {\tt Radial} & {\tt Theta} & {\tt phi} \\ \hline
{\tt xcoarse} & 45 & 12 & 24  \\ \hline
{\tt coarse}  & 75 & 18 & 36  \\ \hline
{\tt medium}  & 95 & 24 & 48  \\ \hline
{\tt fine}    &125 & 30 & 60  \\ \hline
{\tt xfine}   &175 & 44 & 88  \\ \hline
  \end{tabular} \\
\end{center}
\end{table}

\begin{table}[h]
\begin{center}
\caption{Number of radial and angular shells required for Row 3 atoms
  (K $\rightarrow$ Br) to reach the desired accuracies.}

\vspace{.2in}

  \begin{tabular}[right]{|c|c|c|c|} \hline
Keyword & {\tt Radial} & {\tt Theta} & {\tt phi} \\ \hline
{\tt xcoarse} & 75 & 14 & 28  \\ \hline
{\tt coarse}  & 95 & 22 & 44  \\ \hline
{\tt medium}  &110 & 30 & 60  \\ \hline
{\tt fine}    &160 & 34 & 68  \\ \hline
{\tt xfine}   &210 & 38 & 76  \\ \hline
  \end{tabular} \\
\end{center}
\end{table}

\begin{table}[h]
\begin{center}
\caption{Number of radial and angular shells required for Row 4 atoms
  (Rb $\rightarrow$ I) to reach the desired accuracies.}

\vspace{.2in}

  \begin{tabular}[right]{|c|c|c|c|} \hline
Keyword & {\tt Radial} & {\tt Theta} & {\tt phi} \\ \hline
{\tt xcoarse} &105 & 16 & 32  \\ \hline
{\tt coarse}  &130 & 20 & 40  \\ \hline
{\tt medium}  &155 & 32 & 64  \\ \hline
{\tt fine}    &205 & 44 & 88  \\ \hline
{\tt xfine}   &235 & 48 & 96  \\ \hline
  \end{tabular} \\
\end{center}
\end{table}


In addition to the simple keyword specifying the desired accuracy as
described above, the user has the option of specifying a custom
quadrature of this type in which ALL atoms have the same grid
specification.  This is accomplished by using the \verb+gausleg+ keyword.

\begin{verbatim}
  GRID gausleg <integer nradpts default 50> <integer nagrid default 10> 
\end{verbatim}

In this type of grid, the number of phi points is twice the number of
theta points. So, for example, a specification of,
\begin{verbatim}
  GRID gausleg 80 20
\end{verbatim}
would be interpreted as 80 radial points, 20 theta points, and 40
phi points per center (or 64000 points per center before pruning).

A second quadrature available for the numerical integration is an
Euler-Mac\-Laurin scheme for the radial components and a Lebedev
scheme for the angular components\footnote{The subroutine 
for the Lebedev grid was supplied by M.~Caus\`a of the University of
Torino.}.  Within this numerical integration procedure various levels 
of accuracy have also been defined and are available to the user.  
The input for this type of grid takes the form,
\begin{verbatim}
  GRID lebedev <integer radpts default 50> <integer iangquad default 4> 
\end{verbatim}
In this context the variable iangquad specifies a certain number of
angular points as indicated by the table below.

\begin{center}
  \begin{tabular}[right]{|l|r r r r r r r|} \hline
$IANGQUAD$ & 1 & 2 & 3 & 4 & 5 & 6 & 7 \\ \hline
$N_{angular}$ & 38 & 50 & 110 & 194 & 266 & 302 & 434 \\  \hline
  \end{tabular}
\end{center}

Therefore the user can specify any number of radial points along with
the level of angular quadrature (1-7).

%       [store_wght] [nquad_task <integer nquad_task default 1>] \
%{\bf JEFF: store\_weight and nquad\_task need explaining}
%store\_weight and nquad_task keywords will remain as expert user only
% and not advertised until further tested. -jan
%

The user also has the option of choosing one of two types of spatial weights
implemented in the numerical integration of the XC terms; Delley and Becke.

\begin{verbatim}
  GRID [(becke||delley) default becke]
\end{verbatim}

\section{{\tt TOLERANCES} --- Screening tolerances}

\begin{verbatim}
  TOLERANCES [[tight] [tol_rho <real tol_rho default 1e-10>] \
              [accAOfunc <integer accAOfunc default 20>] \
              [accCDfunc <integer accAOfunc default 20>] \
              [accXCfunc <integer accXCfunc default 20>] \
              [accCoul <integer accCoul default 10>] \
              [accQrad <integer accQrad default 12>] \
              [radius <real radius default 25.0>]]
\end{verbatim}
%
%{\bf JEFF: tight needs explanation}
%
The user has the option of controlling screening for the tolerances in
the integral evaluations for the DFT module.  In most applications,
the default values will be adequate for the calculation, but different
values can be specified in the input for the DFT module using the
keywords described below.

The input to define a screening tolerance for evaluation of the AO 
Gaussian functions is specified with the keyword \verb+accAOfunc+, as
follows,
\begin{verbatim}
  TOLERANCES accAOfunc <integer accAOfunc default 20>
\end{verbatim}
A Gaussian orbital basis (AO) function with exponent $\zeta$
and radial factor $e^{-\zeta\cdot r_i^2}$ is 
evaluated  at a point $r_i$ only if 
$\zeta\cdot r_i^2$ is less than the value specified for ${\tt accAOfunc}$.

The input to define a screening tolerance for evaluation of the exchange-
correlation (XC) Gaussian fitting functions is specified with the
keyword \verb+accXCfunc+, as follows,
\begin{verbatim}
  TOLERANCES accXCfunc <integer accXCfunc default 20>
\end{verbatim}
An exchange-correlation (XC) fitting function with exponent $\zeta$
and radial factor $e^{-\zeta\cdot r_i^2}$ is 
evaluated  at a point $r_i$ only if 
$\zeta\cdot r_i^2$ is less than the value specified for ${\tt accXCfunc}$.

The input to define a screening tolerance for evaluation of the
charge-density (CD) Gaussian fitting functions is specified with the
keyword \verb+accCDfunc+, as follows,
\begin{verbatim}
  TOLERANCES accCDfunc <integer accCDfunc default 20>
\end{verbatim}
A  charge-density (CD) fitting function with exponent $\zeta$
and radial factor $e^{-\zeta\cdot r_i^2}$ is evaluated  at a 
point $r_i$ only if $\zeta\cdot r_i^2$ is less than the value 
specified for ${\tt accCDfunc}$.

The input
parameter {\tt accCoul} is used to define the tolerance in Schwarz 
screening for the Coulomb integrals.  Only integrals with estimated
values greater than $10^{(-{\tt accCoul})}$ are evaluated.

\begin{verbatim}
  TOLERANCES accCoul <integer accCoul default 10>
\end{verbatim}

The user also has the option of specifying the radial quadrature 
grid cut-off for the DFT calculation, using the keyword
\verb+accQrad+.  The input line for this option is as follows,
\begin{verbatim}
  TOLERANCES accQrad <integer accQrad default 12>
\end{verbatim}

The value entered for \verb+accQrad+ is the cutoff distance, in bohr, for grid
points around a given center or atom.  Grid points that lie more than 
\verb+accQrad+ bohr from the center or atom are neglected. 

Screening away needless computation of the XC functional (on the grid)
due to negligible density is also possible with the use of,
\begin{verbatim}
  TOLERANCES tol_rho <real tol_rho default 1e-10>
\end{verbatim}
XC functional computation is bypassed if the corresponding density
elements are less than \verb+tol_rho+.

A screening parameter, \verb+radius+, used in the screening of the
Becke or Delley spatial weights is also available as,
\begin{verbatim}
  TOLERANCES radius <real radius default 25.0>
\end{verbatim}
where radius is the cutoff value in bohr.

The tolerances as discussed previously are insured at convergence.
More sleazy tolerances are invoked early in the iterative process
which can speed things up a bit.  This can also be problematic at
times because it introduces a discontinuity in the convergence
process.  To avoid use of initial sleazy tolerances the user can
invoke the \verb+tight+ option:

\begin{verbatim}
  TOLERANCES tight 
\end{verbatim}

This option sets all tolerances to their
default/user specified values at the very first iteration.


\section{{\tt DIRECT} and {\tt NOIO} --- Hardware Resource Control}
\begin{verbatim}
  DIRECT||INCORE
  NOIO
\end{verbatim}

\sloppy

The inverted charge-density and exchange-correlation matrices
for a DFT calculation are normally written to disk storage.  The user
can prevent this by specifying the keyword \verb+noio+ within the
input for the DFT directive.  The input to exercise this option is
as follows,
\begin{verbatim}
   noio
\end{verbatim}
If this keyword is encountered, then the two matrices (inverted
charge-density and exchange-correlation) are computed ``on-the-fly''
whenever needed.  

The \verb+INCORE+ option is always assumed to be true but can be
overridden with the option \verb+DIRECT+ in which case all integrals
are computed ``on-the-fly''.

\fussy

\section{{\tt DFT}, {\tt ODFT} and {\tt MULT} --- Open shell systems}
\begin{verbatim}
  DFT||ODFT
  MULT <integer mult default 1>
\end{verbatim}

Both {\sl closed-shell} and {\sl open-shell} systems can be studied using
the DFT module.  Specifying the keyword \verb+MULT+ within the \verb+DFT+
directive allows the user to define the spin multiplicity of the system.
The form of the input line is as follows;
\begin{verbatim}
   MULT <integer mult default 1> 
\end{verbatim}
When the keyword \verb+MULT+ is specified, the user can define the integer
variable \verb+mult+, where \verb+mult+ is equal to the number of alpha 
electrons minus beta electrons, plus 1.

The keywords \verb+DFT||ODFT+ were originally intended to specify
closed or open shell and are really unnecessary except in the context
of forcing a closed-shell system to be computed as an open shell
system (i.e., using a spin-unrestricted wavefunction).

\section{{\tt MULLIKEN} --- Mulliken analysis}
\begin{verbatim}
  MULLIKEN
\end{verbatim}

Mulliken analysis of the charge distribution is invoked by the keyword:
\begin{verbatim}
  MULLIKEN
\end{verbatim}
When this keyword is encountered, Mulliken analysis of both the input 
density as well as the output density will occur.

\section{Print Control}
\begin{verbatim}
  PRINT||NOPRINT
\end{verbatim}

The \verb+PRINT||NOPRINT+ options control the level of output in the
DFT.  Known controllable print options are:

\begin{table}[htbp]
\begin{center}
\begin{tabular}{lcc}
  {\bf Name}          & {\bf Print Level} & {\bf Description} \\
 ``semi-direct info''               & default     & semi direct algorithm \\
 ``coulomb fit''                    & high        & fitting electronic charge density \\
 ``io info''                        & debug       & reading from and writing to disk  \\
 ``information''                    & low         & general information  \\
 ``quadrature''                     & high        & numerical quadrature  \\
 ``parameters''                     & default     & input parameters \\
 ``convergence''                    & default     & convergence of SCF procedure \\
 ``intermediate vectors''           & high        & intermediate molecular orbitals \\
 ``intermediate evals''             & high        & intermediate orbital energies \\
 ``intermediate overlap''           & high        & overlaps between the alpha and beta sets \\
 ``intermediate S2''                & high        & values of S2 \\
 ``interm vector symm''             & high        & symmetries of intermediate orbitals \\
 ``dft timings''                    & high        & \\
 ``initial vectors''                & high        & \\
 ``common''                         & debug       & dump of common blocks \\
 ``screening parameters''           & high        & integral accuracies \\
 ``intermediate energy info''       & high        & \\
 ``intermediate fock matrix''       & high        & \\
 ``final vectors''                  & high        & \\
 ``schwarz''                        & high        & integral screening info \& stats at completion\\
 ``all vector symmetries''          & high        & symmetries of all molecular orbitals \\
 ``final vector symmetries''        & default     & symmetries of final molecular orbitals \\
 ``multipole''                      & default     & moments of alpha, beta, and nuclear charge densities \\
\end{tabular}
\end{center}
\caption{DFT Print Control Specifications}
\end{table}






\chapter{DFT for Periodic Systems (GAPSS)}
\input{gapss.tex}

\chapter{MP2}
\label{sec:mp2}
\label{sec:rimp2}
\Large
***need a brief summary here of what the MP2 modules do. Will the input
module be finished in time for the release of the code in March (or is
it April)?  I'm going to naively assume that it will be, and it will follow
the same structure as the other modules.***
\normalsize


When the input module is finished, the MP2 modules will be invoked in NWChem 
by specifying the keyword \verb+MP2+ on the compound directive,

\begin{verbatim}
  MP2
    ...
  END
\end{verbatim}

The keyword \verb+mp2+ tells the code that this is a compound directive,
and additional directives may be specified by the user to define the particular
problem.  The \verb+MP2+ input will be processed until the
\verb+END+ directive is encountered.  The actual MP2 calculation will
be performed when the input encounters a \verb+TASK+ directive of the form,

\begin{verbatim}
  TASK mp2_flag
\end{verbatim}

The present version of NWChem recognizes two options for an MP2 calculation.
One is the
resolution of the identity (RI) intergral approximation, which is invoked
by specifying \verb+mp2_flag+ as \verb+RI_MP2+.  The other is the fully direct
solution using conventional four-index transformation (see also Section 
\ref{sec:fourindex}), and is invoked by
specifying \verb+mp2_flag+ as \verb+DIRECT_MP2+.

\Large
**What about OIMP2 and SEMI-DIR\_MP2?  These are listed under possible
inputs for \verb+theory+ in the \verb+TASK+ directive.***
\normalsize


% The RI-MP2 module can be invoked by specifying

% \begin{verbatim}
%   TASK RI-MP2
% \end{verbatim}
% on a \verb+RESTART+ or \verb+CONTINUE+ job -- the MO vectors and
% database from the SCF calculation must be present.  At present, the
% RI-MP2 does not have its own input module, instead it can be
% controlled by \verb+SET+ing entries in the database.  Currently
% recognized entries are described below. {\em The names of these
%   entries in the database will change in the very near future.}

The RI-MP2 option for this module requires that the molecular orbital
vectors and database from an SCF calculation as the starting point for
the MP2 calculation.  This means that this option can be invoked only
on a job that is being restarted or continued (see Section \ref{sec:start}
for a description of the \verb+RESTART+ and \verb+CONTINUE+ directives).

\Large
**any special requirements of the DIRECT\_MP2 option should be mentioned
here, (as well as those of any other options that might be available in
the released version).***
\normalsize

% Alternatively, the fully-direct MP2 module using conventional
% four-index transformation (see also section \ref{sec:fourindex}) may be
% invoked by the input line
% \begin{verbatim}
%   TASK DIRECT_MP2
% \end{verbatim}
% In future releases, the two MP2 modules will be amalgamated into a
% single task where the choice of method will be an input parameter.  In
% cases where the MP2 is restarted using a \verb+RESTART+ or
% \verb+CONTINUE+ directive -- the MO vectors and database from the SCF
% calculation must be present.  Currently, there is no specific MP2
% input module, instead the behavior can be modified by using the
% \verb+SET+ command on the database entries. Currently recognized
% entries are described below, the first three entries apply to both the
% \verb+RI-MP2+ and \verb+DIRECT_MP2+ tasks while the remainder are
% specific to \verb+RI-MP2+.  {\em The names of these entries in the
%  database are subject to change in the near future.}

Currently, the MP2 module has no input processor.  Input parameters that 
must be different from the defaults for a given application are specified
by means of \verb+SET+ directives.  These \verb+SET+ directives must appear
in the input file {\em before} the \verb+TASK+ directive invoking the
MP2 calculation.  The following subsections describe the input for the
MP2 module that can be modified by the user.

\subsection{Input Parameters Applicable to both RI\_MP2 and DIRECT\_MP2}

\subsubsection{Frozen Orbitals for the MP2 Options}

The user has the option of specifying which orbitals from the SCF reference
calculation are to be frozen (i.e., not correlated) for the MP2 calculation.
This can be done either by freezing atoms, or by freezing orbitals, or
both.  To freeze by atoms, the input parameter must be modified using a 
\verb+SET+ directive of the form,

\begin{verbatim}
  set "mp2:freeze by atoms" <integer occupied default 0> \
                            <integer virtual default 0>
\end{verbatim}

The string \verb+"mp2:freeze by atoms"+ instructs the code to guess how 
many occupied and virtual orbitals to be dropped, based on the atoms
present in the system.  Table \ref{tbl:freeze-by-atoms} details how
the ``guess'' is produced.  The integer variables \verb+occupied+ and
\verb+virtual+ are binary switches, and must have values of 0 or 1.  If
the integer \verb+occupied+ is set to 1, the initial MO vectors are
used to freeze parts of the occupied spaces.  If the integer \verb+virtual+
is set to 1, the initial guess is to freeze the virtual spaces.  Both 
input parameters can be set to 1 for a particular calculation.  The default
for each of these parameters is zero, which means that no orbitals (occupied
or virtual) will be frozen in the MP2 calculation. 

\Large
***NOTE: is this really what the default is?  I'm just guessing...***
\normalsize

Alternatively, the user can define a more general but less automatic
specification of which orbitals are to be frozen.  This option is 
obttained by a \verb+SET+ directive of the form,

\begin{verbatim}
  set "mp2:freeze orbitals" <integer array default 0>
\end{verbatim}

The integer entries specified for \verb+array+ correspond to the virtual
and occupied orbitals of an atom in the system.  By convention, the 
highest-lying virtual orbital is 0, the second-highest is -1, and so on
down to the lowest-energy occupied orbital, -N+1 (for N basis functions).
If values outside the range [-N+1,N] are specified for \verb+array+,
the code will detect an error at the start of the MP2 task and
halt the calculation.

The user can use {\em both} methods for specifying frozen orbitals in a
given calculation.  In such a case, both \verb+SET+ directives are
included in the input, and all of the orbitals specified in each
directive will be frozen in the MP2 calculation.

% Determines which orbitals from the SCF reference are to be frozen (not
% correlated) for the MP2 calculation.  The first entry instructs the
% code to guess how many occupied and virtual orbitals to drop based on
% the atoms present in the system.  The two integer elements expected in
% this entry control whether this guess is used to freeze parts of the
% occupied (first element) and/or virtual spaces (second element).  A
% value of 1 in the element requests the guess be used; any other value
% and it will not be used.  Table \ref{tbl:freeze-by-atoms} details how
% the ``guess'' is produced.

\begin{table}
\caption{Number of orbitals considered ``core'' in the ``freeze by
atoms'' algorithm.}
\label{tbl:freeze-by-atoms}
\begin{tabular}{cclr}
\hline\hline
Period & Elements & Core Orbitals & Number of Core \\
\hline
0 & H -- He  & ---                                          &  0 \\
1 & Li -- Ne & 1$s$                                         &  1 \\
2 & Na -- Ar & 1$s$2$s$2$p$                                 &  5 \\
3 & K -- Kr  & 1$s$2$s$2$p$3$s$3$p$                         &  9 \\
4 & Rb -- Xe & 1$s$2$s$2$p$3$s$3$p$4$s$3$d$4$p$             & 18 \\
5 & Cs -- Rn & 1$s$2$s$2$p$3$s$3$p$4$s$3$d$4$p$5$s$4$d$5$p$ & 27 \\
6 & Fr -- Lr & 1$s$2$s$2$p$3$s$3$p$4$s$3$d$4$p$5$s$4$d$5$p$6$s$4$f$5$d$6$p$
     & 43 \\
\hline\hline
\end{tabular}
\end{table}

% The second entry allows a more general, but less automatic,
% specification of which orbitals are to be frozen.  For convenience,
% non-positive entries are understood to refer to highest-lying virtual
% orbital (0), second highest (-1), and so on, down to the lowest-energy
% occupied orbital ($-N+1$, for $N$ basis functions).  Entries outside
% the range $[-N+1, N]$ constitute an error, which would be detected at
% the start of the MP2 calculation (the number of basis functions is not
% generally known when the input is read).

% Both entries may be present, and orbitals in the union of the two sets
% will be frozen.

{\em Cautions:\/}  The rules for freezing orbitals ``by atoms'' are
rather unsophisticated at the moment and may not do what you want.
From limited experience, it seems that special attention should be
paid to systems including third- and higher- period atoms, and perhaps
to the use of spherical harmonic or Cartesian representation of
higher-angular momentum shells.

\Large
**Are you saying this doesn't really work?  Or just that it's hard to
use correctly?  Some examples might be helpful...***
\normalsize

\subsubsection{Schwarz Integral Screening for MP2 Calculations}

\Large
**Do we have anything to say about this?  Or should this section be deleted?
\normalsize

\subsubsection{Fitting Basis for the MP2 Calculation}

The user must define the basis set to be used in the MP2 module, as
either the ``fitting'' basis for the direct calculation, or the ``resolution
of the identity'' basis for the RI\_MP2 calculation.  This is accomplished
with a \verb+SET+ directive of the following form

\begin{verbatim}
  set "mp2:ri-mp2 basis" <string value>
\end{verbatim}

The basis set named by the string \verb+value+ must already exist in the
database.  Alternatively, it can be defined prior to the MP2 task 
in the current input file, using a \verb+BASIS+ directive 
(see Section \ref{sec:basis}).

\Large
***Is this input really needed for \verb+DIRECT_MP2+?***
\normalsize

% This {\em required} entry specifies the basis to be used as the
% ``fitting'', or ``resolution of the identity'' basis.  It should name
% a basis set defined in a \verb+BASIS+ directive somewhere in the
% current or any previous input file.

\subsection{Input Directives for RI-MP2 Calculations Only}

\subsubsection{Transformed Integral Filename for RI-MP2 Calculations}

The user can specify the file prefix of the filenames for the transformed
three-center integrals using a \verb+SET+ directive of the following form,

\begin{verbatim}
  set "mp2:mo 3-center integral file" \
          <string name default "$file_prefix$.mo3cint">
\end{verbatim}

% Specifies the base for constructing the file names for the transformed
% three-center integrals.  
The full name of a file consists of the string specified for \verb+name+,
followed by a letter indicating the spin case of the integrals and an 
integer corresponding to the node number, padded to four digits.

\subsubsection{Reference Spin Mapping for RI-MP2 Calculations}

The user has the option of specifying that the RI-MP2 calculations 
are to be done with variations of the SCF reference wavefunction.  This
is accomplished with a \verb+SET+ directive of the form,

\begin{verbatim}
  set "mp2:reference spin mapping" <integer array default 0>
\end{verbatim}

% Allows RI-MP2 calculations to be done with variations of the SCF
% reference wavefunction.  

Each element specified for \verb+array+ is the SCF spin case to be used
for the corresponding spin case of the correlated calculation.  The
number of elements set determines the overall type of correlated
calculation to be performed.  The default is to use the unadulterated
SCF reference wavefunction.  

% Some examples of settings and their meanings:
% \begin{enumerate}
% \item[``\verb+1 1+''] 

For example, to perform a spin-unrestricted calculation (two
elements) using the alpha spin orbitals (spin case 1) from the
reference for both of the correlated reference spin cases, the \verb+SET+
directive would be as follows,

\begin{verbatim}
  set "mp2:reference spin mapping" 1 1
\end{verbatim}

The SCF calculation to produce the reference wavefuncion could be either
RHF or UHF in this case.


% \item[``\verb+2 2+''] Similar to the previous instance, 
% but uses the beta-spin
% SCF orbtials for both correlated spin cases.  The SCF reference
% obviously must be UHF in this case.

The \verb+SET+ directive for a similar case, but this time using
the beta-spin SCF orbitals for both correlated spin cases, is as follows,
\begin{verbatim}
  set "mp2:reference spin mapping" 2 2
\end{verbatim}

The SCF reference calculation must be UHF in this case.
 

% \item[``\verb+2+''] Performs a spin-restricted calculation (one element)
% from the beta-spin SCF orbitals.

The \verb+SET+ directive for a spin-restricted calculation (one element)
from the beta-spin SCF orbitals using this option is as follows,
\begin{verbatim}
  set "mp2:reference spin mapping" 2
\end{verbatim}


% \item[``\verb+2 1+''] Performs a spin-unrestricted calculation with the
% spins flipped from the original SCF reference.
% \end{enumerate}

The \verb+SET+ directive for a spin-unrestricted calculation with the
% spins flipped from the original SCF reference wavefunction is as follows,
\begin{verbatim}
  set "mp2:reference spin mapping" 2 1
\end{verbatim}


\subsubsection{Batch Sizes for the RI-MP2 Calculation}

The user can control the size of each batch in the transformation and
energy evaluation in the MP2 calculation, and consequently the memory
requirements and number of passes required.  This is done using two
\verb+SET+ directives of the following form,

\begin{verbatim}
  set "mp2:transformation batch size" <integer size default -1>
  set "mp2:energy batch size" <integer isize jsize default -1 -1>
\end{verbatim}

% These entries control the size of each batch in the transformation and
%  energy evaluation, and consequently, the memory requirements and
% number of passes required.  Values less than 1 for these batch sizes
% tell the code to determine the batch size based on the available
% memory, which is the default.  Should there be problems with the
% program-determined batch sizes, these variables allow the user to
% override them.  The program will always use the smaller of the user's
% value of these entries and the internally computed batch size.

The default is for the code to determine the batch size based on the 
available memory.  The user can specify {\em smaller} batch sizes using
these \verb+SET+ directives, but it is not possible to specify {\em larger}
batch sizes than the code computes internally.

The transformation batch size computed in the code is the number of 
occupied orbitals in the
$({occ}\ {vir} | {fit})$ three-center integrals to
be produced at a time.  If this entry is less than the number of occupied
orbitals in the system, the transformation will require multiple
passes through the two-electron integrals.  The memory requirements of
this stage are {\em two} global arrays of dimension ${<batch
size>}\times {vir} \times {fit}$ with the ``fit''
dimension distributed across all processors (on shell-block
boundaries).  The compromise here is memory space versus multiple
integral evaluations.

The energy evaluation batch sizes are computed in the code from the 
number of occupied orbitals in
the the two sets of three-center integrals to be multiplied together
to produce a matrix of approximate four-center integrals.  Two blocks
of integrals of dimension $({<batch isize>}\times {vir})$ and
$({<batch jsize>}\times {vir})$ by fit are read in from disk and
multiplied together to produce $<batch isize> <batch jsize> {vir}^2$
approximate integrals.  The compromise here is performance of the
distributed matrix multiplication (which requires large matrices)
versus memory space.

\Large
**When would the user know better than the code what size is appropriate?
What are the consequences of specifying batch sizes that are ``too small''?
***
\normalsize

\subsubsection{Energy Memory Allocation Mode: RI-MP2 Calculation}

The user must choose a  strategy for the memory allocation in the energy
evaluation phase of the RI-MP2 calculation, either by minimizing the amount
of I/O, or minimizing the amount of compulation.  This is be accomplished 
using a \verb+SET+ directive of the form,

\begin{verbatim}
  set "mp2:energy mem minimize" <string mem_opt default I>
\end{verbatim}

% Choose the strategy for the memory allocation for the energy
% evaluation phase.  Choices are to minimize the amount of I/O
% (``\verb+I+'') or the amount of computation (``\verb+C+'') done by the
% code.
A value of \verb+I+ entered for the string \verb+mem_opt+ means that
a strategy to minimize I/O will be employed.  A value of \verb+C+ tells
the code to use a strategy that minimizes computation.

When the option to minimize I/O is selected, the block sizes are made
as large as possible so that the total number of passes
through the integral files is as small as possible.  When the option to
minimize computation is selected, the blocks are chosen as close to square as
possible so that permutational symmetry in the energy evaluation can
be used most effectively.

\subsubsection{RI Approximation for the RI-MP2 Calculation}

The type of RI approximation used in the RI-MP2 calculation is controlled
by means of a \verb+SET+ directive of the form,

\begin{verbatim}
  set "mp2:ri approximation" <string approx default V>
\end{verbatim}

% Controls the type of RI approximation used in the calculation.  

The code recognizes two possible values for the string \verb+approx+;
the strings \verb+V+ and \verb+SVS+.  The default is \verb+V+, which 
specifies ***what?***.  The string \verb+SVS+ specifies ***what?***.
Both of these options follow the notation used in O.~Vahtras, J~Alml\"of,
and M.~W.~Feyereisen, {\em Chem. Phys. Lett.} {\bf 213}, 514--518
(1993).  Note that the input for the string \verb+approx+ is case sensitive.

Construction of the RI fit requires the inversion of a matrix of
fitting basis integrals which is carried out via diagonalization.  If
the fitting basis includes near linear dependencies, there will be
small eigenvalues which can ultimately lead to non-physical RI-MP2
correlation energies.  Retention of small eigenvalues can be
controlled by the directive

\begin{verbatim}
  set "mp2:fit min eval" <real mineval default 1.0e-8>
\end{verbatim}

%  The \verb+"S"+ approximation will also be supported eventually.

\subsubsection{Local Memory Usage in Three-Center Transformation}

Local memory usage in the first two steps of the transformation is
controlled in the RI-MP2 ccacluattion using the following \verb+SET+
directives,

\begin{verbatim}
  set "xf3ci:AO 1 batch size" <integer max>
  set "xf3ci:AO 2 batch size" <integer max>
  set "xf3ci:fit batch size" <integer max>
\end{verbatim}

% These entries control local memory usage in the first two steps of the
transformation.  

The size of the local arrays determines the sizes of
the two matrix multiplications.  These entries set limits on the size
of blocks to be used in each index.  The size of \verb+xf3ci:AO 1 batch size+
is tthe most important of the three, in terms of the effect on performance.

%    The listing above indicates the
% order of importance of the parameters to performance, with
% \verb+xf3ci:AO 1 batch size+ being most important.

Note that these entries are only upper bounds and that the program
will size the blocks according to what it determines as the best usage of
the available local memory.  The absolute maximum for a block size is
the number of functions in the AO basis, or the number of fitting basis
functions on a node.  The absolute minimum value for block size is the 
size of the largest shell in the appropriate basis.  Batch size entries 
specified for \verb+max+  that are larger than these limits are 
automatically reset to an appropriate value.

For most applications, the code will be able to size the blocks without
help from the user.  Therefore, it is unlikely that users will have 
any reason to specify values for these entries
except when doing very particular performance measurements.

\subsubsection{Printing Control for the RI-MP2 Calculation}

Printable items which are recognized by the RI-MP2 module are listed
in Table \ref{tbl:mp2-printable}.  At the moment, these items can be
controlled by directly using a \verb+SET+ directive of the form,

\begin{verbatim}
   SET mp2:print [<string type default automatic>] <$type$ data>
\end{verbatim}

Items listed in Table \ref{tbl:mp2-printable} above can be named in 
$type$ to obtain output from the RI-MP2 calculation.

\begin{table}
\caption{Printable items in the RI-MP2 module and their default print levels.}
\label{tbl:mp2-printable}
\begin{tabular}{lll}
\hline\hline
Item                    & Print Level   & Description \\
\hline
``2/3 ints''              & debug         & Partially transformed 3-center integrals \\
``3c ints''               & debug         & MO 3-center integrals in energy evaluation \\
``4c ints b''             & debug         & ``B'' matrix derived from approx. 4c integrals \\
``4c ints''               & debug         & Approximate 4-center integrals \\
``amplitudes''            & debug         & ``B'' matrix with denominators (not really amplitudes, strictly) \\
``basis''                 & high          & \\
``fit xf''                & debug         & Transformation for fitting basis \\
``geombas''               & debug         & Detailed basis map info\\
``geometry''              & high          & \\
``information''           & low           & General information about calc.\\
``integral i/o''          & high          & File size information\\
``mo ints''               & debug         & \\
``pair energies''         & debug         & \\
``partial pair energies'' & debug         & Pair energy matrix each time it is updated \\
``progress reports''      & default       & Report completion of time-consuming steps\\
``reference''             & high          & Details about reference wavefunction\\
``warnings''              & low           & Non-fatal warnings \\
\hline\hline
\end{tabular}
\end{table}



\chapter{Multiconfiguration SCF}
\label{sec:mcscf}

The NWChem multiconfiguration SCF (MCSCF) module can currently perform
complete active space SCF (CASSCF) calculations with at most 20 active
orbitals and about 500 basis functions.  It planned to extend it to 
handle 1000+ basis functions.

\begin{verbatim}
  MCSCF
    STATE <string state>
    ACTIVE <integer nactive>
    ACTELEC <integer nactelec>
    MULTIPLICITY <integer multiplicity>
    [SYMMETRY <integer symmetry default 1>]
    [VECTORS [[input] <string input_file default $file_prefix$.movecs>] 
           [swap <integer vec1 vec2> ...] \
           [output <string output_file default input_file>] \
           [lock]
    [HESSIAN (exact||onel)]
    [MAXITER <integer maxiter default 20>]
    [THRESH  <real thresh default 1.0e-4>]
    [TOL2E <real tol2e default 1.0e-9>]
    [LEVEL <real shift default 0.1d0>]
  END
\end{verbatim}
Note that the \verb+ACTIVE+, \verb+ACTELEC+, and \verb+MULTIPLICITY+
directives are {\em required}.  The symmetry and multiplicity may
alternatively be entered using the \verb+STATE+ directive.

\section{{\tt ACTIVE} --- Number of active orbitals}

The number of orbitals in the CASSCF active space must be specified
using the {\tt ACTIVE} directive.

E.g.,
\begin{verbatim}
  active 10
\end{verbatim}

The input molecular orbitals (see the vectors directive, Sections
\ref{sec:mcscfvectors} and \ref{sec:vectors}) must be arranged in
order
\begin{enumerate}
\item doubly occupied orbitals,
\item active orbitals, and
\item unoccupied orbitals.
\end{enumerate}

\section{{\tt ACTELEC} --- Number of active electrons}

The number of electrons in the CASSCF active space must be specified
using the the {\tt ACTELEC} directive.  An error is reported if the
number of active electrons and the multiplicity are inconsistent.

The number of closed shells is determined by subtracting the number
of active electrons from the total number of electrons (which in turn
is derived from the sum of the nuclear charges minus the total system
charge).

\section{{\tt MULTIPLICITY}}

The spin multiplicity must be specified and is enforced by projection
of the determinant wavefunction.

E.g., to obtain a triplet state
\begin{verbatim}
  multiplicity 3
\end{verbatim}

\section{{\tt SYMMETRY} --- Spatial symmetry of the wavefunction}

This species the irreducible representation of the wavefunction as an
integer in the range 1---8 using the same numbering of representations
as output by the SCF program.  Note that only Abelian point groups are
supported.

E.g., to specify a $B_1$ state when using the $C_{2v}$ group
\begin{verbatim}
  symmetry 3
\end{verbatim}

\section{{\tt STATE} --- Symmetry and multiplicity}

The electronic state (spatial symmetry and multiplicity) may
alternatively be specified using the conventional notation for an
electronic state, such as $^3B_2$ for a triplet state of $B_2$
symmetry.  This would be accomplished with the input
\begin{verbatim}
  state 3b2
\end{verbatim}
which is equivalent to 
\begin{verbatim}
  symmetry 4
  multiplicity 3
\end{verbatim}

\section{{\tt VECTORS} --- Input/output of MO vectors}
\label{sec:mcscfvectors}

Calculations are best started from RHF/ROHF molecular orbitals (see
Section \ref{sec:scf}), and by default vectors are taken from the
previous MCSCF or SCF calculation.  To specify another input file use
the \verb+VECTORS+ directive.  Vectors are by default output to the
input file, and may be redirected using the \verb+output+ keyword.
The \verb+swap+ keyword of the \verb+VECTORS+ directive may be
used to reorder orbitals to obtain the correct active space.
See Section \ref{sec:vectors} for an example.

Output orbitals of a converged MCSCF calculation are canonicalized as
follows:
\begin{itemize}
\item Doubly occupied and unoccupied orbitals diagonalize the
  corresponding blocks of an effective Fock operator.  Note that in
  the case of degenerate orbital energies this does not fully
  determine the orbtials.
\item Active-space orbitals are chosen as natural orbitals by
  diagonalization of the active space 1-particle density matrix.
  Note that in the case of degenerate occupations that this
  does not fully determine the orbitals.
\end{itemize}

\section{{\tt HESSIAN} --- Select preconditioner}
\label{sec:mcscfhessian}

The MCSCF will use a one-electron approximation to the orbital-orbital
Hessian until some degree of convergence is obtained, whereupon it
will attempt to use the exact orbital-orbital Hessian which makes the
micro iterations more expensive but potentially reduces the total
number of macro iterations.  Either choice may be forced throughout
the calculation by specifying the appropriate keyword on the
\verb+HESSIAN+ directive.

E.g., to specify the one-electron approximation throughout
\begin{verbatim}
  hessian onel
\end{verbatim}

\section{{\tt LEVEL} --- Level shift for convergence}

The Hessian (Section \ref{sec:mcscfhessian}) used in the MCSCF
optimization is by default level shifted by 0.1 until the orbital
gradient norm falls below 0.01, at which point the level shift is
reduced to zero.  The initial value of $0.1$ may be changed using
the \verb+LEVEL+ directive.  Increasing the level shift may make
convergence more stable in some instances.

E.g., to set the initial level shift to 0.5
\begin{verbatim}
  level 0.5
\end{verbatim}

\section{{\tt PRINT} and {\tt NOPRINT}}

Specific output items can be selectively enabled or disabled using the
\verb+print+ control mechanism~(\ref{sec:printcontrol}) with the
available print options listed in table(\ref{MCSCF_print_options}).

\begin{table}[htb]

\label{MCSCF_print_options}

\center

\vspace{.2in}
\begin{tabular}{lrl}
\hline\hline
Option                          & Class    &  Synopsis \\
\hline
\verb+ci energy+                & default  &  CI energy eigenvalue \\
\verb+fock energy+              & default  &  Energy derived from Fock matrices \\
\verb+gradient norm+            & default  &  Gradient norm \\
\verb+movecs+                   & default  &  Converged occupied MO vectors \\
\verb+trace energy+             & high     &  Trace Energy \\
\verb+converge info+            & high     &  Convergence data and monitoring \\
\verb+precondition+             & high     &  Orbital preconditioner iterations \\
\verb+microci+                  & high     &  CI iterations in line search \\
\verb+canonical+                & high     &  Canonicalization information \\
\verb+new movecs+               & debug    &  MO vectors at each macro-iteration \\
\verb+ci guess+                 & debug    &  Initial guess CI vector \\
\verb+density matrix+           & debug    &  One- and Two-particle density matrices \\
\hline\hline
\end{tabular}

\caption{MCSCF Print Options}

\end{table}




\chapter{Selected CI}
\label{sec:selci}

The selected CI module is tightly integrated into NWChem but as yet no
input module has been written.  The input thus consists of setting the
appropriate variables in the database.

It is assumed that an initial SCF/MCSCF calculation has completed, and
that MO vectors are available.  These will be used to perform a
four-index transformation, if this has not already been performed.

\subsection{Background}

This is a general spin-adapted, configuration-driven CI program
which can perform arbitrary CI calculations, the only restriction
being that all spin functions are present for each orbital occupation.
CI wavefunctions may be specified using a simple configuration
generation program, but the prime usage is intended to be in
combination with perturbation correction and selection of new
configurations.  The second-order correction (Epstein-Nesbet) to the
CI energy may be computed, and at the same time configurations that
interact greater than a certain threshold with the current CI
wavefunction may be chosen for inclusion in subsequent calculations.
By repeating this process (typically twice is adequate) with the same
threshold until no new configurations are added, the CI expansion may
be made consistent with the selection threshold, enabling tentative 
extrapolation to the full-CI limit.

A typical sequence of calculations is as follows:
\begin{enumerate}
\item Pick as an initial CI reference the previously executed
  SCF/MCSCF.
\item Define an initial selection threshold.
\item Determine the roots of interest in the current reference space.
\item Compute the perturbation correction and select additional
  configurations that interact greater than the current threshold.
\item Repeat steps 3 and 4.
\item Lower the threshold (a factor of 10 is common) and repeat steps
  3, 4, and 5.  The {\em first} pass through step 4 will yield the
  approximately self-consistent CI and CI+PT energies from the {\em
    previous} selection threshold.
\end{enumerate}

To illustrate this, below is some abreviated output from a
calculation on water in an augmented cc-PVDZ basis set with one frozen
core orbital.  The SCF was converged to high precision in $C_{2v}$
symmetry with the following input
\begin{verbatim}
  start h2o
  geometry; symmetry c2v
    O 0 0 0; H 0 1.43042809 -1.10715266
  end
  basis
    H library aug-cc-pvdz; O library aug-cc-pvdz
  end
  task scf
  scf; tol2e 1d-8; thresh 1d-8; end
\end{verbatim}

The following input restarts from the SCF to perform a sequence of
selected CI calculations with the specified tolerances, starting with
the SCF reference.
\begin{verbatim}
  restart h2o
  set fourindex:occ_frozen integer 1
  set "selci:selection thresholds" real \
      0.001 0.001 0.0001 0.0001 0.00001 0.00001 0.000001
  task selci
\end{verbatim}
Table \ref{selcitab} summarizes the output from each of the major
computational steps that were performed.
\begin{table}[h]
  \begin{tabular}{c|l|r|l}
       &               & CI        &    \\
  Step &  Description  & dimension &  Energy        \\ \hline
       &               &           &          \\
    1  &  Four-index, one frozen-core &    &      \\
    2  &  Config. generator, SCF default &  1 & \\
    3+4&  CI diagonalization & 1 &  $E_{CI} = -76.041983$ \\
    5  &  PT selection T=0.001 & 1 & $E_{CI+PT} = -76.304797$ \\
    6+7 &  CI diagonalization & 75 & $E_{CI} = -76.110894$ \\
    8  &  PT selection T=0.001& 75 & $E_{CI+PT} = -76.277912$ \\
    9+10&  CI diagonalization & 75 & $E_{CI}(T=0.001) = -76.110894$ \\
    11 &  PT selection T=0.0001 & 75 & $E_{CI+PT}(T=0.001) = -76.277912$ \\
    12+13 & CI diagonalization & 823 & $E_{CI} = -76.228419$ \\
    14 & PT selection T=0.0001 & 823 & $E_{CI+PT} = -76.273751$ \\
    15+16 & CI diagonalization & 841 & $E_{CI}(T=0.0001) = -76.2300544$ \\
    17 & PT selection T=0.00001& 841 & $E_{CI+PT}(T=0.0001) = -76.274073$ \\
    18+19 & CI diagonalization & 2180 &  $E_{CI} = -76.259285$ \\
    20 & PT selection T=0.00001& 2180 & $E_{CI+PT} = -76.276418$ \\
    21+22 & CI diagonalization & 2235 & $E_{CI}(T=0.00001) = -76.259818$ \\
    23 & PT selection T=0.000001 & 2235 & $E_{CI+PT}(T=0.00001) = -76.276478$\\
    24   & CI diagonalization &  11489 & \\ \hline
\end{tabular}
\caption{\label{selcitab} Summary of steps performed in a selected CI
  calculation on water.}
\end{table}

\subsection{Files}

Currently, no direct control is provided over filenames.  All files
are prefixed with the standard file-prefix, and any files generated by
all nodes are also postfixed with the processor number.  Thus, for
example the molecular integrals file, used only by process zero, might
be called {\tt h2o.moints} whereas the off-diagonal Hamiltonian matrix
element file used by process number eight would be called {\tt
  h2o.hamil.8}.

\begin{description}
\item{\tt ciconf} --- the CI configuration file, which holds
  information about the current CI expansion, indexing vectors, etc.
  This is the most important file and is required for all restarts.
  Note that the CI configuration generator is only run if this file
  does not exist. Referenced only by process zero.
\item{\tt moints} --- the molecular integrals, generated by the four-index
  transformation.  As noted above these must currently be manually
  deleted, or the database entry \verb+selci:moints:force+ set, to
  force regeneration.  Referenced only by process zero.
\item{\tt civecs} --- the CI vectors.    Referenced only by process zero.
\item{\tt wmatrx} --- temporary file used to hold coupling coefficients.
  Deleted at calculation end.  Referenced only by process zero.
\item{\tt rtname, roname} --- restart information for the PT
  selection.  Should be automatically deleted if no restart is
  necessary.  Referenced only by process zero.
\item{\tt hamdg} --- diagonal elements of the Hamiltonian.
  Deleted at calculation end.  Referenced only by process zero.
\item{\tt hamil} --- off-diagonal Hamiltonian matrix elements.  All
  processes generate a file containing a subset of these elements.
  These files can become very large.  Deleted at calculation end.
\end{description}

\subsection{Configuration Generation}

If no configuration is explicitly specified then the previous
SCF/MCSCF wavefunction is used, adjusting for any orbitals frozen in
the four-index transformation.  The four-index transformation must
have completed successfully before this can execute.  Orbital
configurations for use as reference functions may also be explicitly
specified.

Once the default/user-input reference configurations have been
determined additional reference functions may be generated by applying
multiple sets of creation-annihilation operators, permitting for
instance, the ready specification of complete or restricted active
spaces.

Finally, a uniform level of excitation from the current set of
configurations into all orbitals may be applied, enabling, for
instance, the simple creation of single or single+double excitation 
spaces from an MCSCF reference.

\subsubsection{Specifying the reference occupation}

A single orbital configuration or occupation is specified by
\begin{verbatim}
  ns  (socc(i),i=1,ns)  (docc(i),i=1,nd)
\end{verbatim}
where \verb+ns+ specifies the number of singly occupied orbitals,
\verb+socc()+ is the list of singly occupied orbitals, and
\verb+docc()+ is the list of doubly occupied orbitals (the
number of doubly occupied orbitals, \verb+nd+, is inferred from
\verb+ns+ and the total number of electrons).  All occupations may be
strung together and inserted into the database as a single integer
array with name \verb+"selci:conf"+.  For example, the input
\begin{verbatim}
  set "selci:conf" integer \
    0                1  2  3  4 \
    0                1  2  3 27 \
    0                1  3  4 19 \
    2   11 19        1  3  4 \
    2    8 27        1  2  3 \
    0                1  2  4 25 \
    4   3  4 25 27   1  2 \
    4   2  3 19 20   1 4 \
    4   2  4 20 23   1 3  
\end
specifies the following nine orbital configurations
\begin{verbatim}
   1(2)  2(2)  3(2)  4(2)
   1(2)  2(2)  3(2) 27(2)
   1(2)  3(2)  4(2) 19(2)
   1(2)  3(2)  4(2) 11(1) 19(1)
   1(2)  2(2)  3(2)  8(1) 27(1)
   1(2)  2(2)  4(2) 25(2)
   1(2)  2(2)  3(1)  4(1) 25(1) 27(1)
   1(2)  2(1)  3(1)  4(2) 19(1) 20(1)
   1(2)  2(1)  3(2)  4(1) 20(1) 23(1)
\end{verbatim}
The optional formatting of the input is just to make this arcane
notation easier to read.  Relatively few configurations can be
currently specified in this fashion because of the input line limit of
1024 characters.

\subsubsection{Applying creation-annihilation operators}

Up to 10 sets of creation-annihilation operator pairs may be
specified, each set containing up to 255 pairs.  This suffices to
specify complete active spaces with up to ten electrons.

The number of sets is specified as follows,
\begin{verbatim}
  set selci:ngen integer 4
\end{verbatim}
which indicates that there will be four sets.  Each set is then
specified as a separate integer array
\begin{verbatim}
  set "selci:refgen  1" integer 5 4    6 4   5 3   6 3  
  set "selci:refgen  2" integer 5 4    6 4   5 3   6 3  
  set "selci:refgen  3" integer 5 4    6 4   5 3   6 3  
  set "selci:refgen  4" integer 5 4    6 4   5 3   6 3  
\end{verbatim}
In the absence of friendly input note that the names
\verb+"selci:refgen n"\verb+ must be formatted with n in \verb+I2+
format. Each set specifies a list of creation-annihilation operator
pairs (in that order).  So for instance, in the above example each set
is the same and causes the exictations
\begin{verbatim}
  4->5   4->6   3->5   3->6
\end{verbatim}
If orbitals 3 and 4 were initially doubly occupied, and orbitals 5 and
6 initially unoccupied, then the application of this set of operators
four times in succession is sufficient to generate the four electron
in four orbital complete active space.

The precise sequence in which operators are applied is
\begin{enumerate}
\item loop through sets of operators
\item loop through reference configurations
\item loop through opertors in the set
\item apply the operator to the configuration, if the result is new add it
  to the new list
\item end the loop over operators
\item end the loop over reference configurations
\item augment the configuration list with the new list
\item end the loop over sets of operators
\end{enumerate}

\subsubsection{Uniform excitation level}

By default no excitation is applied to the reference configurations.
If, for instance, you wanted to generate a single excitation CI space
from the current configuration list, specify
\begin{verbatim}
set selci:exci integer 1
\end{verbatim}
Any excitation level may be applied, but since the list of
configurations is explicitly generated, as is the CI Hamiltonian
matrix, you will run out of disk space if you attempt to use more than
a few tens of thousands of configurations.

\subsection{Number of roots}

By default, only one root is generated in the CI diagonalization or
perturbation selection.  The following requests that 2 roots be
generated
\begin{verbatim}
  set selci:nroot integer 2
\end{verbatim}
There is no imposed upper limit.  If many roots are required, then, to
minimize root skipping problems, I would suggest doing an initial
approximate diagonalization with several more roots than required,
and then resetting this parameter once you are satisfied you have the
desired states.

\subsection{Accuracy of diagonalization}

By default, the CI wavefunctions are converged to a residual norm of
$10^{-6}$ which provides similiar accuracy in the perturbation
corrections to the energy, and much higher accuracy in the CI
eigenvalues.  This may be adjusted with
\begin{verbatim}
 set "selci:diag tol" real 1d-3
\end{verbatim}
the example setting much lower precision, appropriate for the
approximate diagonalization discussed in the preceeding section.

\subsection{Selection thresholds}

When running in the selected-CI mode the program will loop
through a list of selection thresholds ($T$), performing the CI
diagonalization, computing the perturbation correction, and augmenting
the CI expansion with configurations that make an energy lowering to
any root greater than $T$.  The list of selection thresholds is
specified as follows
\begin{verbatim}
  set "selci:selection thresholds" real \
      0.001 0.001 0.0001 0.0001 0.00001 0.00001 0.000001
\end{verbatim}

There is no default for this parameter.


\subsection{Mode}

By default the program runs in \verb="ci+davids"= mode and just
determines the CI eigenvectors/values in the current configuration
space.  To perform a selected-CI with perturbation correction use the
following
\begin{verbatim}
  set selci:mode select
\end{verbatim}
and remember to define the selection thresholds.

\subsection{Memory requirements}

No global arrays are used inside the selected-CI, though the
four-index transformation can be automatically invoked and it does use
GAs.  The selected CI replicates inside each process
\begin{itemize}
\item all unique two-electron integrals in the MO basis that are
  non-zero by symmetry, and
\item all CI information, including the CI vectors.
\end{itemize}
These large data structures are allocated on the local stack.  A fatal
error will result if insufficient memory is available.

\subsection{Forcing regeneration of the MO integrals}

When scanning a potential energy surface or optimizing a geometry the
MO integrals need to be regenerated each time.  Specify
\begin{verbatim}
  set selci:moints:force logical .true.
\end{verbatim}
to accomplish this.

\subsection{Disabling update of the configuration list}

When computing CI+PT energy the reference configuration list is
normally updated to reflect all configurations that interact more than
the specified threshold.  This is usually desirable.  But when
scanning a potential energy surface or optimizing a geometry the
reference list must be kept fixed to keep the potential energy surface
continuous and well defined.  To do this specify
\begin{verbatim}
  set selci:update logical .false.
\end{verbatim}




\chapter{Coupled Cluster Calculations}
\label{sec:ccsd}

{\bf Some corrections/additions still pending?

\section{Background and Capabilities}

The NWChem coupled cluster module is primarily the work of Alistair
Rendell and Rika Kobayashi, and is derived from the Titan parallel
coupled cluster program.

The coupled cluster code can perform calculations with full iterative
treatment of single and double excitations and non-iterative inclusion
of triple excitation effects.  It is presently limited to closed-shell
(RHF) references.  It is {\em not} presently possible to freeze core or
virtual orbitals.

\section{Control Directives}

The operation of the coupled cluster code is controlled by the input
block
\begin{verbatim}
  CCSD
    ...
  END
\end{verbatim}
Note that the keyword \verb+CCSD+ is used for the input block
regardless of the actual level of theory desired (specified with the
\verb+TASK+ directive).  The following directives are recognized
within the \verb+CCSD+ group.

\subsection{{\tt MAXITER} --- Maximum number of iterations}

The maximum number of iterations is set to 20 by default.  This should
be quite enough for most calculations, although particularly
troublesome cases may require an increase.

\begin{verbatim}
  MAXITER  <integer maxiter default 20>
\end{verbatim}

\subsection{{\tt THRESH} --- Convergence threshold}

Controls the convergence threshold for the iterative part of the
calculation.  Both the RMS error in the amplitudes {\em and} the
change in energy must be less than $10^{-{\tt thresh}}$.

\begin{verbatim}
  THRESH  <integer thresh default 6>
\end{verbatim}

\subsection{{\tt DIISBAS} --- DIIS subspace dimension}

Specifies the maximum size of the subspace used in DIIS convergence
acceleration.  Note that DIIS requires the amplitudes and errors be
stored for each iteration in the subspace.  Obviously this can
significantly increase memory requirements, and could force the user
to reduce \verb+DIISBAS+ for large calculations.

{\em Measures to alleviate this problem, including more compact
storage of the quantities involved, and the possibility of disk
storage are being considered, but have not yet been implemented.}

\begin{verbatim}
  DIISBAS  <integer diisbas default 5>
\end{verbatim}

\subsection{{\tt IPRT} --- Debug printing}

This directive controls the level of output from the code, mostly to
facilitate debugging and the like.  The larger the value, the more
output printed.  From looking at the source code, the interesting
values seem to be \verb+IPRT+ $>$ 5, 10, and 50.

\begin{verbatim}
  IPRT  <integer IPRT default 0>
\end{verbatim}

\subsection{PRINT and NOPRINT}

The coupled cluster module supports the standard NWChem print control
keywords, although very little in the code is actually hooked into
this mechanism yet.

\begin{tabular}{lll}
\hline\hline
Item                    & Print Level   & Description \\
\hline
``reference''             & high          & Wavefunction information\\
``guess pair energies'' & debug & MP2 pair energies\\
``byproduct energies'' & default & Intermediate energies   \\
``term debugging switches'' & debug & Switches for individual terms \\
\hline\hline
\end{tabular}


\section{Methods (Tasks) Recognized}

Currently available methods are
\begin{itemize}
\item \verb+CCSD+ -- Full iterative inclusion of single and double
excitations
\item \verb=CCSD+T(CCSD)= -- The fourth order triples contribution computed with
converged singles and doubles amplitudes
\item \verb=CCSD(T)= -- {\bf ?????????????}
\end{itemize}

The calculation is invoked using the the \verb+TASK+ directive, so to
perform a CCSD+T(CCSD) calculation, for example, the input file should
include the directive
\begin{verbatim}
  TASK CCSD+T(CCSD)
\end{verbatim}

Lower-level results which come as by-products (such as MP3/MP4) of the
requested calculation are generally also printed in the output file
and stored on the run-time database, but the method specified in the
\verb+TASK+ directive is considered the primary result.

\section{Debugging and Development Aids}

The information in this section is intended for use by experts (both
with the methodology and with the code), primarily for debugging and
development work.  Messing with stuff in listed in this section will
probably make your calculation quantitatively {\em\bf wrong}\/!
Consider yourself warned!

\subsection{Switching On and Off Terms}

The \verb+/DEBUG/+ common block contains a number of arrays which
control the calculation of particular terms in the program.  These are
15-element integer arrays (although from the code only a few elements
actually effect anything) which can be set from the input deck.  See
the code for details of how the arrays are interpreted.  

Printing of this data at run-time is controlled by the \verb+''term
debugging switches''+ print option.  The values are checked against
the defaults at run-time and a warning is printed to draw attention to
the fact that the calculation does not correspond precisely to the
requested method.

\begin{verbatim}
  DOA  <integer array default 2 2 2 2 2 2 2 2 2 2 2 2 2 2 2>
  DOB  <integer array default 2 2 2 2 2 2 2 2 2 2 2 2 2 2 2>
  DOG  <integer array default 1 1 1 1 1 1 1 1 1 1 1 1 1 1 1>
  DOH  <integer array default 1 1 1 1 1 1 1 1 1 1 1 1 1 1 1>
  DOJK <integer array default 2 2 2 2 2 2 2 2 2 2 2 2 2 2 2>
  DOS  <integer array default 1 1 1 1 1 1 1 1 1 1 1 1 1 1 1>
  DOD  <integer array default 1 1 1 1 1 1 1 1 1 1 1 1 1 1 1>
\end{verbatim}


%\chapter{Four-Index Transformation}
%\label{sec:fourindex}

{\bf ????????????????????}
\Large
**What does this module really do?  Is it invoked by any module other than
DIRECT\_MP2?
\normalsize

%  The four-index transformation module is not designed for explicit
% invocation but rather as a utility module for other tasks (e.g. direct
% MP2 section \ref{sec:mp2}). However, there are input parameters which
% may be set to modify the default behavior of the four-index
% transformation module which may enhance performance.

The four-index transformation module is a utility invoked by the fully
direct MP2 module (see Section \ref{sec:mp2}).  For most applications,
the default parameters for the four-index transformation will be adequate,
but in some instances, the user may wish to enhance the performance of
the module.  This can be done by specifying \verb+SET+ directives for
selected parameters.  The following subsections describe the options
currently available for this module.

\subsection{Algorithm for the Four-Index Transformation}

The code currently contains two algorithms for the four-index transformation.
The algorithm to use for a given calculation can be selected by specifying
a \verb+SET+ directive of the form,

\begin{verbatim}
 set "fourindex:method" <string method default twofold>
\end{verbatim}

% There is a choice of two algorithms for effecting the transformation.
The default entry for the string \verb+method+ is \verb+twofold+, which
specifies ***what?***.  This option should be used in almost all instances.
However, in cases of high orders of parallelism (i.e., more than 200 nodes),
 better throughput may be obtained with the alternative algorithm, 
obtained by setting \verb+method+ to \verb+sixfold+.

\subsection{AO Integral Blocking and Block Length for the Four-Index Transformation}

The default in the four-index transformation is to have no blocking in
the AO integral generation.  Blocking can be specified explicitly using a
\verb+SET+ directive of the form,

\begin{verbatim}
 set "fourindex:aoblock" <logical aoblock default .false.>
\end{verbatim}

Entering a value of \verb+.true.+ for the logical variable \verb+aoblock+ 
enables blocking in the AO integral generation. This may result in 
substantial performance enhancement in cases with many identical 
atoms and basis sets.

The default is 10 for the vector length in the critical first index 
transformation of the four-index transformation algorithm.  The user has
the option of specifying the vector length explicitly by entering a 
\verb+SET+ directive of the form,

\begin{verbatim}
 set "fourindex:blocklength" <integer blocklength default 10>
\end{verbatim}

% This parameter determines the vector length in the critical first
% index transformation. 

For larger basis sets, it may be profitable to increase the vector
length by specifying a value larger than 10 for the integer variable
\verb+blocklength+.  However, this value should {\em
  not} exceed $N_{bf} / 3$, since the cost of the redundant computation 
in the square matrix multiplication would in that case exceed the 
gain from a longer vector length.

\Large
***What is $N_{bf} / 3$?
\normalsize

\subsection{Frozen orbitals for the Four-Index Transformation}

In most cases, the frozen orbitals will be specified by the input for the 
MP2 calculation (or calculation in some other module???) that invokes the
four-index transformation module.  However, the user can specify frozen
orbitals separately for the four-index transformation, using \verb+SET+
directives of the form,

\begin{verbatim}
 set "fourindex:occ_frozen" \
      <integer frozen_occupied default 0>
 set "fourindex:vir_frozen" \
      <integer frozen_virtual default 0>
\end{verbatim}

The value specified for the integer variable \verb+frozen_occupied+ 
designates the lowest frozen occupied orbital.  The value specified
for the integer variable \verb+frozen_virtual+ designates the highest
virtual frozen orbital.  The default is zero for both of these variables,
which specifies no additional frozen orbitals for the four-index
transformation module.

% Setting these parameters
% should not be required in the usual circumstances since these are
% determined by the calling modules, (see section \ref{sec:mp2}).


%%\chapter{Plane-wave periodic DFT}
%%\input{plnwv.tex}

\chapter{Geometry Optimization with DRIVER}
\label{sec:driver}

The DRIVER module is one of two drivers (see Section \ref{sec:stepper}
for documentation on STEPPER) to perform a geometry optimization
function on the molecule defined by input using the \verb+GEOMETRY+
directive (see Section \ref{sec:geom}).  Geometry optimization is
either an energy minimization or a transition state optimization.
The algorithm programmed in DRIVER is a quasi-newton optimization
with line searches and approximate energy Hessian updates.

DRIVER is selected by default out of the two available modules to
perform geometry optimization.  In order to force use of DRIVER (e.g.,
because a previous optimization used STEPPER) provide a DRIVER input
block (below) --- even an empty block will force use of DRIVER.

Optional input for this module is specified within the compound
directive,
\begin{verbatim}
  DRIVER 
    (LOOSE || DEFAULT || TIGHT)
    GMAX <real value>
    GRMS <real value>
    XMAX <real value>
    XRMS <real value>

    EPREC <real eprec default 1e-7>

    TRUST <real trust default 0.3>
    SADSTP <real sadstp default 0.1>

    CLEAR
    REDOAUTOZ

    INHESS <integer inhess default 0>

    (MODDIR || VARDIR) <integer dir default 0>
    (FIRSTNEG || NOFIRSTNEG)

    MAXITER <integer maxiter default 20>

    BSCALE <real BSCALE default 1.0>
    ASCALE <real ASCALE default 0.25>
    TSCALE <real TSCALE default 0.1>
    HSCALE <real HSCALE default 1.0>
   
    PRINT ...

    XYZ [<string xyz default $file_prefix$>]
    NOXYZ

  END
\end{verbatim}

\sloppy

\section{Convergence criteria}

\begin{verbatim}
    (LOOSE || DEFAULT || TIGHT)
    GMAX <real value>
    GRMS <real value>
    XMAX <real value>
    XRMS <real value>
\end{verbatim}

 In version 3.3 Gaussian-style convergence criteria have been adopted.
The defaults may be used, or the directives \verb+LOOSE+,
\verb+DEFAULT+, or \verb+TIGHT+ specified to use standard sets of
values, or the individual criteria adjusted.  All criteria are in
atomic units.
\verb+GMAX+ and \verb+GRMS+ control the maximum and root mean square
gradient in the coordinates being used (Z-matrix, redundant internals,
or Cartesian).  \verb+XMAX+ and \verb+XRMS+ control the maximum and
root mean square of the Cartesian step.

\begin{verbatim}
                  LOOSE    DEFAULT    TIGHT
         GMAX   0.0045d0   0.00045   0.000015   
         GRMS   0.0030d0   0.00030   0.00001
         XMAX   0.0054d0   0.00180   0.00006
         XRMS   0.0036d0   0.00120   0.00004
\end{verbatim}

 Note that GMAX and GRMS used for convergence of geometry may significantly vary in 
different coordinate systems such as Z-matrix, redundant internals, or Cartesian. 
The coordinate system is defined in the input file (default is Z-matrix). 
Therefore the choice of coordinate system may slightly affect converged energy. 
Although in most cases XMAX and XRMS are last to converge which are always done 
in Cartesian coordinates, which insures convergence to the same geometry in 
different coordinate systems.


The old criterion may be recovered with the input
\begin{verbatim}
   gmax 0.0008; grms 1; xrms 1; xmax 1
\end{verbatim}

\section{Available precision}

\begin{verbatim}
    EPREC <real eprec default 1e-7>
\end{verbatim}

In performing a line search the optimizer must know the
precision of the energy (this has nothing to
do with convergence criteria).  The default value
of 1e-7 should be adjusted if less, or more, precision
is available.  Note that the default EPREC for DFT
calculations is 5e-6 instead of 1e-7.

\section{Controlling the step length}

\begin{verbatim}
    TRUST <real trust default 0.3>
    SADSTP <real sadstp default 0.1>
\end{verbatim}

A fixed trust radius (\verb+trust+) is used to control the step during
minimizations, and is also used for modes being minimized during
saddle-point searches.  It defaults to 0.3 for minimizations and 0.1
for saddle-point searches.  The parameter \verb+sadstp+ is the trust
radius used for the mode being maximized during a saddle-point search
and defaults to 0.1.

\section{Maximum number of steps}

\begin{verbatim}
    MAXITER <integer maxiter default 20>
\end{verbatim}

By default at most 20 geometry optimization steps will be taken,
but this may be modified with this directive.

\section{Discard restart information}
\begin{verbatim}
    CLEAR
\end{verbatim}

By default Driver reuses Hessian information from a previous
optimization, and, to facilitate a restart also stores which mode is
being followed for a saddle-point search.  This option deletes all
restart data.

\section{Regenerate internal coordinates}

\begin{verbatim}
    REDOAUTOZ
\end{verbatim}

Deletes Hessian data and regenerates internal coordinates at the
current geometry.  Useful if there has been a large change in the
geometry that has rendered the current set of coordinates invalid or
non-optimal.

\section{Initial Hessian}
\begin{verbatim}
    INHESS <integer inhess default 0>
\end{verbatim}

\begin{itemize}
\item  0 = Default ... use restart data if available, otherwise use diagonal guess.
\item  1 = Use diagonal initial guess.
\item  2 = Use restart data if available, otherwise transform
Cartesian Hessian from previous frequency calculation.
\end{itemize}


In addition, the diagonal elements of the initial Hessian for
internal coordinates may be scaled using separate factors for
bonds, angles and torsions with the following
\begin{verbatim}
    BSCALE <real bscale default 1.0>
    ASCALE <real ascale default 0.25>
    TSCALE <real tscale default 0.1>
\end{verbatim}
These values typically give a two-fold speedup over unit values, based
on about 100 test cases up to 15 atoms using 3-21g and 6-31g* SCF.
However, if doing many optimizations on physically similar systems it
may be worth fine tuning these parameters.

Finally, the entire Hessian from any source may be scaled
by a factor using the directive
\begin{verbatim}
    HSCALE <real hscale default 1.0>
\end{verbatim}
It might be of utility, for instance, when computing an initial
Hessian using SCF to start a large MP2 optimization.  The SCF
vibrational modes are expected to be stiffer than the MP2, so scaling
the initial Hessian by a number less than one might be beneficial.


\section{Mode or variable to follow to saddle point}

\begin{verbatim}
    (MODDIR || VARDIR) <integer dir default 0>
    (FIRSTNEG || NOFIRSTNEG)
\end{verbatim}

When searching for a transition state the program, by default,
will take an initial step uphill and then do mode following
using a fuzzy maximum overlap (the lowest eigen-mode with an
overlap with the previous search direction of 0.7 times the
maximum overlap is selected).  Once a negative eigen-value
is found, that mode is followed regardless of overlap.

The initial uphill step is appropriate if the gradient points roughly
in the direction of the saddle point, such as might be the case if a
constrained optimization was performed at the starting geometry.
Alternatively, the initial search direction may be chosen to be along
a specific internal variable (using the directive
\verb+VARDIR+) or along a specific eigen-mode (using \verb+MODDIR+).
Following a variable might be valuable if the initial gradient is
either very small or very large.  Note that the eigen-modes in the
optimizer have next-to-nothing to do with the output from a frequency
calculation.  You can examine the eigen-modes used by the optimizer
with

\begin{verbatim}
         driver; print hvecs; end
\end{verbatim}

The selection of the first negative mode is usually a good choice if
the search is started in the vicinity of the transition state and the
initial search direction is satisfactory.  However, sometimes the
first negative mode might not be the one of interest (e.g., transverse
to the reaction direction).  If \verb+NOFIRSTNEG+ is specified, the
code will not take the first negative direction and will continue doing
mode-following until that mode goes negative.

\section{Optimization history as XYZ files}

\begin{verbatim}
    XYZ [<string xyz default $fileprefix>]
    NOXYZ
\end{verbatim}

The \verb+XYZ+ directive causes the geometry at each step (but not
intermediate points of a line search) to be output into separate files
in the permanent directory in XYZ format.  The optional string will
prefix the filename.  The \verb+NOXYZ+ directive turns this off.

For example, the input
\begin{verbatim}
    driver; xyz test; end
\end{verbatim}
will cause files test-000.xyz, test-001.xyz, \ldots\ to be created
in the permanent directory.  

The script \verb+rasmolmovie+ in the NWChem \verb+contrib+ directory
can be used to turn these into an animated GIF movie.

\section{Print options}

The UNIX command \verb+"egrep '^@' < output"+ will extract a pretty
table summarizing the optimization.

If you specify the NWChem input
\begin{verbatim}
      scf; print none; end
      driver; print low; end
      task scf optimize
\end{verbatim}
you'll obtain a pleasantly terse output.

For more control, these options for the standard print directive are
recognized
\begin{itemize}
\item \verb+debug+   - prints a large amount of data.  Don't use in parallel.
\item \verb+high+    - print the search direction in internals
\item \verb+default+ - prints geometry for each major step (not during
                the line search), gradient in internals (before
                and after application of constraints)
\item \verb+low+     - prints convergence and energy information.  At 
                convergence prints final geometry, change in internals
                from initial geometry
\end{itemize}
and these specific print options
\begin{itemize}
\item      {\tt finish} (low)      - print geometry data at end of calculation
\item      {\tt bonds}  (default)  - print bonds at end of calculation
\item      {\tt angles} (default)  - print angles at end of calculation
\item      {\tt hvecs}  (never)    - print eigen-values/vectors of the Hessian
\item      {\tt searchdir} (high)  - print the search direction in internals
\item      `{\tt internal gradient}' (default) - print the gradient in internals
\item      {\tt sadmode} (default) - print the mode being followed to the saddle point
\end{itemize}

\fussy




\chapter{Geometry Optimization with STEPPER}
\label{sec:stepper}

The STEPPER module performs a search for critical points on the
potential energy surface of the molecule defined by input using the
\verb+GEOMETRY+ directive (see Section \ref{sec:geom}).  Since STEPPER
is {\bf not} the primary geometry optimization module in NWChem the
compound directive is required; the DRIVER module is the default (see
Section {\LARGE ref{sec:driver}}).  Input for this module is
specified within the compound directive,

\begin{verbatim}
  STEPPER
    ...
  END
\end{verbatim}

The presence of the STEPPER compound directive automatically turns off
the default geometry optimization tool driver. Input specified for the
STEPPER module must appear in the input file {\em after} the
\verb+GEOMETRY+ directive, since it must know the number of atoms that
are to be used in the geometry optimization.  In the current version
of NWChem, STEPPER can be used only with geometries that are defined
in Cartesian coordinates.  STEPPER removes translational and
rotational components before determining the step direction (5
components for linear systems and 6 for others) using a standard
Eckart algorithm.  The default initial guess nuclear Hessian is the
identity matrix.

The default in STEPPER is to minimize the energy as a function of the
geometry with a maximum of 20 geometry optimization iterations.  When
this is the desired calculation, no input is required other than the
STEPPER compund directive.  However, the user also has the option of
defining different tasks for the STEPPER module, and can vary the
number of iterations and the convergence criteria from the default
values.  The input for these options is described in the following
sections.

\section{Action directives in the STEPPER Module}

The default is for STEPPER to minimize the energy with respect to the
geometry of the system.  STEPPER can also be used to find the
transition state by following the lowest eigenvector of the nuclear
Hessian.  There are several options available to modify the behavior
of the STEPPER module.  The input to define the multiple actions of
STEPPER is defined generically. 

\begin{verbatim}
   <string (action || variable) [variable value]  default MIN>
\end{verbatim}

The value \verb+MIN+ for the string \verb+action+ specifies the default
energy minimization.  Finding the lowest transition state is specified
by entering the value \verb+TS+ for the string \verb+action+.

STEPPER has the ability to ``track'' a specific mode during an
optimation for a transition state search, the user can also have the
module track the eigenvector corresponding to a specific mode.  This
is done by specifying the keyword \verb+TRACK+, using the following
input line,

\begin{verbatim}
  TRACK [nmode <integer nmode default 1>]
\end{verbatim}

The keyword \verb+TRACK+ tells STEPPER to track the eigenvector
corresponding to the integer value of \verb+nmode+ during a transition
state walk.  (Note: this input is invalid for a minimization walk
since following a specific eigenvector will not necessarily give the
desired local minimum .)  The step is constructed to go up in energy
along the \verb+nmode+ eigenvector and down in all other degrees of
freedom.

\section{Control of the STEPPER Calculation}

In most applications, 20 stepper iterations will be sufficient to
obtain the energy minimization.  However, the user has
the option of specifying the maximum number of iterations allowed,
using the input line,

\begin{verbatim}
  MAXITER <integer maxiter default 20>
\end{verbatim}

The value specified for the integer \verb+maxiter+ defines the maximum 
number of geometry optimization steps.  The
geometry optimization will restart automatically.

The size of steps that can be taken in STEPPER is governed by the degree
to which the calculated values of the eigenvectors can be considered 
reasonably good.  This is the 'trust radius', and has a default value of
0.1, which means that the step in the direction determined will be no
longer than the trust radius.  The user has the option of overriding this
default using the keyword \verb+TRUST+, with the following input line,

\begin{verbatim}
  TRUST <real radius default 0.1>
\end{verbatim}

The larger the value specified for the variable \verb+radius+, the larger 
the steps that can be taken by STEPPER. Experience has shown that for
larger systems (i.e., those with 20 or more atoms), a value of 0.5 or
greater should be entered for \verb+radius+.

\section{Convergence Criteria for the STEPPER Calculations}

Two convergence criteria can be specified explicitly for the 
STEPPER calculations.  The keyword \verb+CONVGG+ allows the user to
specify the the convergence tolerence for the gradient norm for
all degrees of freedom.  The input line is of the following form,

\begin{verbatim}
   CONVGG <real convgg default 1.0d-04>
\end{verbatim}

The entry for the real variable \verb+convgg+ should be approximately 
equal to the square root of the energy convergence tolerance.

The energy convergence tolerance is the convergence criterion for the 
energy difference in the geometry optimization in STEPPER.  It can be
specified by input using a line of the following form,

\begin{verbatim}
   CONVGE <real convge default 1.0d-08>
\end{verbatim}


\section{Initial Guess for Nuclear Hessian}

Any initial hessian can be used with the STEPPER module via the ASCII
hessian interface.  The lower triangular [$3N{\times}(3N+1)/2$] matrix
written in any ASCII format (e.g., 1pd20.10) will work but the entries
must be one per line.  This should be stored in a file called
\verb+<default_file_prefix>+.hess in the current working directory of
node zero.  

There are two other options that stepper allows regarding the initial
guess for the nuclear hessian.  By specifying a basis set (smaller
than the desired basis set) with basis set name of ``fd basis'' (c.f.,
Section \ref{sec:basis} users can optimize the geometry using the
smaller basis and then generate a finite difference hessian.
Alternatively users may generate a finite difference hessian at the
current geometry.  

These actions are invoked with the input tokens:

\begin{verbatim}
FDAT
\end{verbatim}

FDAT computes the finite difference nuclear hessian at the current
geometry using the ``fd basis'' and then begins the optimization using
the ``ao basis'' for the particular QM method.  

\begin{verbatim}
FDOPT
\end{verbatim}

FDOPT optimizes the geomety of the system in the ``fd basis'' using
the user specified QM method.  The finite difference nuclear hessian
is then computed at this optimized geometry for the ``fd basis.''
The optimization using the ``ao basis'' for the particular QM method
is then started.  

\section{Backsteping in Stepper}

If a step taken during the optimization is too large (e.g., the step
causes the energy to go up for a minimization or down for a transition
state search), the STEPPER optimizer will automatically ``backstep'' and
correct the step based on information prior to the faulty step.  If
you have an optimization that ``backsteps'' frequently then the inital
trust radius should most likely be decreased.  





\chapter{Vibrational frequencies}
The vibrational frequencies can be computed by finite difference for
SCF and DFT wavefunctions currently.  Drivers for other wavefunctions
will be developed as required.  The vibrational package was integrated
from the Utah Messkit and can use any hessian generated from the
driver routines.

{\it The input module for vib is under development and incomplete.}

\subsection{VIB input: project}

The VIB module can project out the translations and rotations of the
hessian using the standard Eckart projection algorithm.  To set this
option you must enter the following in your input deck.

\begin{verbatim}
  set "vib:project" true
\end{verbatim}

\subsection{VIB input: zero point energy}

The VIB module can also compute the zero point energy for the
molecular system.  This will automatically set the projection flag to
remove the translations and rotations.   To compute the zero point energy
you must enter the following in your input deck.

\begin{verbatim}
  set "vib:zero point energy" logical true
\end{verbatim}

Note: the mass of each atom is set via the geometry input, (c.f.,
\ref{sec:geom}). 


\chapter{DPLOT}
%
% $Id: dplot.tex,v 1.14 2009-03-05 23:46:55 d3p975 Exp $ 
%
\label{sec:dplot}
\begin{verbatim}
  DPLOT
    ...
  END
\end{verbatim}

This directive is used to obtain the plots of various types of electron
densities (or orbitals) of the molecule. The electron density is calculated
on a specified set of grid points using the molecular orbitals
from SCF or DFT calculation. The output file is either in 
\htmladdnormallink{MSI Insight II} 
{http://www.msi.com/materials/insight/Insight2.html} contour format
(default)
 or in the 
\htmladdnormallink{Gaussian Cube}{http://www.gaussian.com/g_ur/u_cubegen.htm}
format.  DPLOT is not executed until the ``\verb+task dplot+'' directive is given.  
Different sub-directives are described below.

\section{{\tt GAUSSIAN} --- Gaussian Cube format}

\begin{verbatim}
  GAUSSIAN
\end{verbatim}

A outputfile is generate in Gaussian Cube format.
You can visualize this file using  \htmladdnormallink{gOpenMol}
{http://www.csc.fi/gopenmol/tutorials/quick_start.phtml\#view}
(after converting the Gaussian Cube file with 
\htmladdnormallink{gcube2plt}
{http://www.csc.fi/lul/chem/scarecrow/gcube2plt.c}),
 \htmladdnormallink{Molden}
{http://www.cmbi.kun.nl/~schaft/molden/molden.html} 
 or 
\htmladdnormallink{Molekel}
{http://www.cscs.ch/molekel/}.

\section{{\tt TITLE} --- Title directive}

\begin{verbatim}
  TITLE <string Title default Unknown Title>
\end{verbatim}

This sub-directive specifies a title line for the generated
input to the {\em Insight} program or for the Gaussian cube
file. Only one line is allowed.

\section{{\tt LIMITXYZ} --- Plot limits}

\begin{verbatim}
  LIMITXYZ [units <string Units default angstroms>]
  <real X_From> <real X_To> <integer No_Of_Spacings_X>
  <real Y_From> <real Y_To> <integer No_Of_Spacings_Y>
  <real Z_From> <real Z_To> <integer No_Of_Spacings_Z>
\end{verbatim}

This sub-directive specifies the limits of the cell to be plotted.
The grid is generated using \verb+No_Of_Spacings+~+~\verb+1+ points
along each direction. The known names for \verb+Units+ are
\verb+angstroms+, \verb+au+ and \verb+bohr+.



\section{{\tt SPIN} --- Density to be plotted}

\begin{verbatim}
  SPIN <string Spin default total>
\end{verbatim}

This sub-directive specifies, what kind of density is to be computed. The 
known names for \verb+Spin+ are \verb+total+, \verb+alpha+, \verb+beta+
and \verb+spindens+, the last being computed as the difference between
$\alpha$ and $\beta$ electron densities.



\section{{\tt OUTPUT} --- Filename}

\begin{verbatim}
  OUTPUT <string File_Name default dplot>
\end{verbatim}

This sub-directive specifies the name of the generated input to the
{\em Insight} program or the generated Gaussian cube file. 
The name \verb+OUTPUT+ is reserved for the
standard NWChem output.



\section{{\tt VECTORS} --- MO vector file name}

\begin{verbatim}
  VECTORS <string File_Name default movecs> [<string File_Name2>]
\end{verbatim}

This sub-directive specifies the name of the molecular orbital file.
If the second file is optionally given the density is computed as the
difference between the corresponding electron densities. The vector 
files have to match.

\section{{\tt DENSMAT} ---  Density matrix file name}

\begin{verbatim}
  DENSMAT <string File_Name> [<string File_Name2>]
\end{verbatim}

This sub-directive specifies the name of the density matrix file.
If the second file is optionally given the density is computed as the
difference between the corresponding electron densities. The density matrix
files have to match. This option is used for plotting the density generated
by correlation methods, such as CCSD and EOM-CCSD.


\section{{\tt WHERE} --- Density evaluation}

\begin{verbatim}
  WHERE <string Where default grid>
\end{verbatim}

This sub-directive specifies where the density is to be computed.
The known names for \verb+Where+ are \verb+grid+ (the calculation of
the density is performed on the set of a grid points specified by the
sub-directive \verb+LimitXYZ+ and the file specified by the sub-directive
\verb+Output+ is generated), \verb+nuclei+ (the density is computed at
the position of the nuclei and written to the NWChem output) and
\verb+g++\verb+n+ (both).


\section{{\tt ORBITAL} --- Orbital sub-space}

\begin{verbatim}
  ORBITALS [<string Option default density>]
  <integer No_Of_Orbitals>
  <integer Orb_No_1 Orb_No_2 ...>
\end{verbatim}

This sub-directive specifies the subset of the orbital space for the
calculation of the electron density. The density is computed using the
occupation numbers from the orbital file modified according to the
\verb+Spin+ directive. If the contours of the orbitals are to be plotted
\verb+Option+ should be set to \verb+view+. Note, that in this case
\verb+No_Of_Orbitals+ should be set to \verb+1+ and sub-directive
\verb+Where+ is automatically set to \verb+grid+. Also specification
of two orbital files conflicts with the \verb+view+ option.
$\alpha$ orbitals are always plotted unless \verb+Spin+ is set to
\verb+beta+.

\section{Examples}

\subsection*{Charge Density}

Example of charge density plot (with Gaussian Cube output):
\begin{verbatim}
start n2
geometry
  n  0 0   0.53879155
  n  0 0  -0.53879155
end
basis;  n library cc-pvdz;end
scf
vectors  output n2.movecs
end
dplot
  TITLE HOMO
  vectors n2.movecs
   LimitXYZ
 -3.0 3.0 10  
-3.0 3.0 10 
-3.0  3.0  10
  spin total
  gaussian
  output lumo.cube
end
task scf     
task dplot
\end{verbatim}

\begin{verbatim}
start n2
geometry
  n  0 0   0.53879155
  n  0 0  -0.53879155
end
basis;  n library cc-pvdz;end
scf
vectors  output n2.movecs
end

tce
ccsd
densmat n2.densmat
end

task tce energy

dplot
  TITLE HOMO
   LimitXYZ
 -3.0 3.0 10
-3.0 3.0 10
-3.0  3.0  10
  spin total
  gaussian
  densmat n2.densmat
  output lumo.cube
end
task dplot
\end{verbatim}

\subsection*{Molecular Orbital}

Example of orbital plot (with Insight II contour output):
\begin{verbatim}
start n2
geometry
  n  0 0   0.53879155
  n  0 0  -0.53879155
end
basis;  n library cc-pvdz;end
scf
vectors  output n2.movecs
end
dplot
  TITLE HOMO
  vectors n2.movecs
   LimitXYZ
 -3.0 3.0 10  
-3.0 3.0 10 
-3.0  3.0  10
  spin total
  orbitals view; 1; 7
  output homo.grd
end
task scf     
task dplot
\end{verbatim}


\chapter{Properties}
\label{sec:property}
\begin{verbatim}
  PROPERTY
    [property name]
    [VECTORS ...]
  END
\end{verbatim}

Calculation of properties is accomplished with \verb+TASK PROPERTY+
after the completion of an energy (or MP2 gradient) calculation.  The
following properties can be computed for all wavefunctions that produce
orbitals, including Hartree-Fock (closed-shell RHF, open-shell ROHF, and
open-shell UHF), DFT (closed-shell and open-shell spin unrestricted),
MCSCF (complete active space), and MP2 (closed-shell RHF and open-shell
UHF).

\begin{itemize}
\item natural bond analysis
\item dipole moment
\item quadrupole moment
\item octupole moment
\item Mulliken population analysis and bond order analysis
\item electrostatic potential (diamagnetic shielding) at nuclei 
\item electric field at nuclei 
\item electric field gradient at nuclei 
\item electron density and electron wavefunction at nuclei 
\item spin density at nuclei
\end{itemize}

The default molecular orbital file \verb+$file_prefix$.movecs+ is used
unless a vectors directive (Section \ref{sec:vectors}) is provided.  It is
therefore only necessary to include a vectors directive if the MO vectors
to be analyzed are not coming from the default file, e.g., if they have
been previously redirected, or if MP2 natural orbitals (file extension
\verb+".mp2nos"+) are being anaylzed.

\section{Subdirectives}

Note that presenting any property input causes all previous property input
to be ``forgotten'', unlike other NWChem modules.

Each property can be requested by means of a subdirective among the
subdirectives provided :

\begin{itemize}
\item nbofile
\item dipole
\item quadrupole
\item octupole
\item mulliken
\item esp
\item efield
\item efieldgrad
\item electrondensity
\item spindensity
\end{itemize}

The request to NBOFILE does not execute the Natural Bond Analysis
code, but simply creates an input file to be used as input to the
stand-alone NBO code. All other properties are calculated upon
request.

An additional subdirective is provided to specify the origin of the
molecular orbitals used in the calculation of the molecular
properties. This is the 'vectors' subdirective, also used in the
SCF and DFT tasks. For a full description of this subdirective
the user is refered to the description found in the SCF description.
By default, the input file used for the calculation of the properties
has the .movecs name extension. 

\subsection{Nbofile}

Following the successful completion of an electronic structure
calculation, a Natural Bond Orbital (NBO) analysis may be carried out
in the following way.  On restart specify the TASK as PROPERTY and
supply the sub-directive NBOFILE to the PROPERTY directive.  NWChem
will query the rtdb and construct an ASCII file,
\verb+<file_prefix>.gen+, that may be used as input to the stand alone
version of the NBO program, gennbo.  \verb+<file_prefix>+ is equal to
string following the RESTART directive.  The input deck may be edited
to provide additional options to the NBO calculation, (see the NBO
user's manual for details.)



\chapter{Prepare}
\label{sec:prepare}
\def\bmu{\mbox{\boldmath $\mu$}}
\def\bE{\mbox{\bf E}}
\def\br{\mbox{\bf r}}
\def\tT{\tilde{T}}
\def\t{\tilde{1}}
\def\ip{i\prime}
\def\jp{j\prime}
\def\ipp{i\prime\prime}
\def\jpp{j\prime\prime}
\def\etal{{\sl et al.}}
\def\nwchem{{\bf NWChem}}
\def\nwargos{{\bf nwargos}}
\def\nwtop{{\bf nwtop}}
\def\nwrst{{\bf nwrst}}
\def\nwsgm{{\bf nwsgm}}
\def\esp{{\bf esp}}
\def\md{{\bf md}}
\def\prepare{{\bf prepare}}
\def\argos{{\bf ARGOS}}
\def\amber{{\bf AMBER}}
\def\charmm{{\bf CHARMM}}

The \prepare\ module is used to set up the necessary files for a molecular
dynamics simulation with \nwchem. User supplied coordinates can be used to
generate topology and restart files. The topology file contains all static
information about a molecular system, such as lists of atoms, bonded
interactions and force field parameters. The restart file contains all
dynamic information about a molecular system, such as coordinates, velocities
and properties.

Without any input, the prepare module checks the existence of a topology
and restart file for the molecular systems. If these files exist, the module
returns to the main task level without action. The module will generate these 
files when they do not exist. Without any input to the module, the generated 
system will be for a non-solvated isolated solute system.

To update existing files, including solvation, the module requires input
directives read from an input deck,

\begin{verbatim}
prepare
 ...
end
\end{verbatim}

The prepare module performs three sub-tasks:
\begin{itemize}
\item[{\bf *}]
{\bf sequence generation}\\
This sub-task analyzes the supplied coordinates from a PDB-formatted file
or from the input geometry, and generates a sequence file, containing the
description of the system in terms of basic building blocks found as
fragment or segment files in the database directories for the force field 
used. If these files do not exist, they are generated based on the supplied
coordinates. This process constists of generating a fragment file with the
list of atoms with their force field dependent atom types, partial atomic
charges calculated from a Hartree Fock calculation for the fragment, followed
by a restrained electrostatic potential fit, and a connectivity list. From
the information on this fragment file the lists of all bonded interactions
are generated, and the complete lists are written to a segment file.
\item[{\bf *}]
{\bf topology generation}\\
Based on the generated or user-supplied sequence file and the force field
specific segment database files, this sub-task compiles the lists of atoms,
bonded interactions, excluded pairs, and substitutes the force field
parameters. Special commands may be given to specify interaction parameters
that will be changing in a free energy evaluation.
\item[{\bf *}]
{\bf restart generation}\\
Using the user supplied coordinates and the topology file for the chemical
system, this sub-task generates a restart file for the system with coordinates,
velocities and other dynamic information. This step may include solvation
of the chemical system and specifying periodic boundary conditions.
\end{itemize}

Files involved in the preparation phase exist in the following hierarchy:
\begin{itemize}
\item[{\bf *}]
{\bf standards}\\
The standard database files contain the original force field information.
These files are to reside in a directory that is specified in the file 
\$HOME/.nwchemrc. There will be such a directory for each supported force 
field. These directories contain fragment files (with extension frg),
segment files (with extension sgm) and a parameter file (with the name
of the force field and with extension par).
\item[{\bf *}]
{\bf extensions}\\
These database files contain generally accepted extensions to the original
force field and are to reside in a separate directory that is specified in 
the file \$HOME/.nwchemrc. There will be such a directory for each supported 
force field. These directories contain fragment files (with extension frg),
segment files (with extension sgm) and a parameter file (with the name
of the force field and with extension par).
\item[{\bf *}]
{\bf user preferences}\\
These database files contain user preferred extensions to the original
force field and are to reside in a separate directory that is specified in 
the file \$HOME/.nwchemrc. Separate directories of this type  should be 
defined for each supported force field. 
These directories may contain fragment files (with extension frg),
segment files (with extension sgm) and a parameter file (with the name
of the force field and with extension par).
\item[{\bf *}]
{\bf temporary files}\\
Temporary database files contain user preferred extensions to the original
force field and are to reside in a separate directory that is specified in 
the file \$HOME/.nwchemrc. There  be such a directory for each supported 
force field. These directories may contain fragment files (with extension frg),
segment files (with extension sgm) and a parameter file (with the name
of the force field and with extension par). If not specified, temporary
files will be taken from the current directory.
\end{itemize}

Data is taken from the database files searched in the above order. If data
is specified more than once, the last found values are used. For example,
if some standard segment is redefined in a temporary file, the latter one
will be used. This allows the user to redefine standards or extensions 
without having to modify those database files, which may reside in a
generally available, non-modifyable directory.
\par
The most common problems with the \prepare\ module are
\begin{itemize}
\item[{\bf ~}]
The format of the pdb file does not conform to the pdb standard. In
particular, atom names need to correspond with definitions in the
fragment and segment database files, and should adhere to IUPAC
recommendations as adopted by the pdb standard. If this problem
occurs, the pdb file will need to be corrected.
\item[{\bf ~}]
Non-standard segments may contain atoms that could not be atom typed
with the existing typing rules in the force field parameter files.
When this happens, additional typing rules can be included in the
parameter file, or the fragment file may be manually typed.
\item[{\bf ~}]
Parameters for atom types or bonded interactions do not exist in
the force field. When this happens, additional parameters may be
defined in the parameter files, or the segment file may be edited
to include explicit parameters.
\end{itemize}

\section{Default database directories}

The file \$HOME/.nwchemrc may contain the following entries that determine
which files are used by the prepare module.

\begin{verbatim}
ffield <string ffname>
\end{verbatim}

This entry specifies the default force field. Database files supplied with
\nwchem\ currently support values for \verb+ffname+ of {\bf amber}, referring 
to AMBER95, and {\bf charmm}, referring to the academic CHARMM22 force field.

\begin{verbatim}
<string ffname>_(s | x | u | t) <string ffdir>
\end{verbatim}

Entries of this type specify the directory \verb+ffdir+ in which force field 
database files can be found. 
The prepare module will only use files in directories
specified here. One exception is that files in the current work directory
will be used if no directory with temporary files is specified.

\begin{verbatim}
<string solvnam> <string solvfil>
\end{verbatim}

This entry may be used to identify a pure solvent restart file \verb+solvfil+
by a name \verb+solvnam+

An example file \$HOME/.nwchemrc is:

\begin{verbatim}
ffield amber
amber_s /msrc/proj/nwchem/share/amber/amber_s/
amber_x /msrc/proj/nwchem/share/amber/amber_x/
amber_u /usr/people/d3j191/data/amber/amber_u/
spce /msrc/proj/nwchem/share/solvents/spce.rst
charmm_s /msrc/proj/nwchem/share/charmm/charmm_s/
charmm_x /msrc/proj/nwchem/share/charmm/charmm_x/
\end{verbatim}

\section{System name and coordinate source}

\begin{verbatim}
system <string sys_calc>
\end{verbatim}

The system name can be explicitly specified for the \prepare\ module.
If not specified, the system name will be taken from a specification
in a previous \md\ input block, or derived from the run time database
name.

\begin{verbatim}
source ( pdb | rtdb )
\end{verbatim}

The source of the coordinates can be explicitly specified to be from
a PDB formatted file \verb+calc+.pdb, or from a geometry object in the run
time database. If not specified, a pdb file will be used when it exists
in the current directory or the rtdb geometry otherwise.

\begin{verbatim}
model <integer modpdb default 0>
\end{verbatim}

If a PDB formatted source file contains different MODELs, the \verb+model+
keyword can be used to specify which MODEL will be used to generate the
topology and restart file. If not specified, the first MODEL found on the
PDB file will be read.

\begin{verbatim}
altloc <character locpdb default ' '>
\end{verbatim}

The \verb+altloc+ keyword may be used to specify the use of alternate
location coordinates on a PDB file.

\begin{verbatim}
chain <character chnpdb default ' '>
\end{verbatim}

The \verb+chain+ keyword may be used to specify the chain identifier
for coordinates on a PDB file.

\begin{verbatim}
sscyx
\end{verbatim}

Keyword \verb+sscyx+ may be used to rename cysteine residues that form
sulphur bridges to CYX.

\begin{verbatim}
hbuild
\end{verbatim}

Keyword \verb+hbuild+ may be used to add hydrogen atoms to the
unknown segemnts of the structure found on the pdb file.


\section{Sequence file generation}

If no existing sequence file is present in the current directory,
or if the \verb+new_seq+ keyword was specified in the \prepare\ 
input deck, a new sequence file is generated from information
from the pdb file, and the following input directives.

\begin{verbatim}
maxscf <integer maxscf default 20>
\end{verbatim}

Variable maxscf specifies the maximum number of atoms in a segment for
which partial atomic charges will be determined from an SCF calculation
followed by RESP charge fitting. For larger segments a crude partial
charge guestimation will be done.

\begin{verbatim}
qscale <real qscale default 1.0>
\end{verbatim}

Variable qscale specifies the factor with which SCF/RESP determined
charges will be multiplied.

\begin{verbatim}
modify sequence { <integer sgmnum>:<string sgmnam> }
\end{verbatim}

This command specifies that segment {\bf sgmnam} should be used
for segment with number {\it sgmnum}. This command can be used
to specify a particular protonation state. For example, the
following command specifies that residue 114 is a hystidine
protonated at the N$_\epsilon$ site and residue 202 is a hystidine
protonated at the N$_\delta$ site:

\begin{verbatim}
modify sequence 114:HIE 202:HID
\end{verbatim}

Links between atoms can be enforced with

\begin{verbatim}
link <string atomname> <string atomname>
\end{verbatim}

For example, to link atom {\rm SG} in segment 20 with atom {\rm FE}
in segment 55, use:

\begin{verbatim}
link 20:_SG 55:FE
\end{verbatim}

\par
The format of the sequence file is given in Table~\ref{tbl:nwaseq}.
In addition to the list of segments this file also includes links
between non-standard segments or other non-standard links. 
These links are generated based on distances found between
atoms on the pdb file. When atoms are involved in such non-standard 
links that have not been identified in the fragment of segment
files as a non-chain link atom, the prepare module will ignore
these links and report them as skipped. If one or more of these links
are required, the user has to include them with explicit link
directives in the sequence file, making them forced links.
Alternatively, these links can be made forced-links by changing 
\verb+link+ into \verb+LINK+ in the sequence file. 

\begin{verbatim}
fraction { <integer imol> }
\end{verbatim}

Directive \verb+fraction+ can be used to separate solute molecules
into fractions for which energies will be separately reported 
during molecular dynamics simulations. The listed molecules will be
the last molecule in a fraction. Up to 10 molecules may be
specified in this directive.

\begin{verbatim}
counter <integer num> <string ion>
\end{verbatim}

Directive \verb+counter+ add \verb+num+ counter ions of type
\verb+ion+ to the sequence file. Up to 10 \verb+counter+
directive may appear in the input block.

\section{Topology file generation}

\begin{verbatim}
new_top [ new_seq ]
\end{verbatim}

Keyword \verb+new_top+ is used to force the generation of a new topology 
file. An existing topology file for the system in the current directory
will be overwritten. If keyword \verb+new_seq+ is also specified, an
existing sequence file will also be overwritten with a newly generated
file.

\begin{verbatim}
amber | charmm
\end{verbatim}

The prepare module generates force field specific fragment, segment and 
topology files. The force field may be explicitly specified in the prepare 
input block by specifying its name.
Currently \amber\ and \charmm\ are the supported force fields.
A default force field may be specified in the file \$HOME/.nwchemrc. 

\begin{verbatim}
standard <string dir_s>
extensions <string dir_x>
user <string dir_u>
temporary <string dir_t>
\end{verbatim}

The user can explicitly specify the directories where force field
specific databases can be found. These include force field standards,
extensions, user preferences and temporary database files.\\
Defaults for the directories where database files reside may be specified 
in the file \$HOME/.nwchemrc for each of the supported force fields. 
Fragment, segment and sequence files generated by the \prepare\ module are 
written in the temporary directory. When not specified, the current 
directory will be used. 
Topology and restart files are always created in the current directory.


The following directives control the modifications of a
topology file. These directives are executed in the order in which 
they appear in the \prepare\ input deck. The topology modifying
commands are not stored on the run-time database and are, therefor,
not persistent.

\begin{verbatim}
modify atom <string atomname> [set <integer mset> | initial | final] \\
	( type <string atomtyp> |  charge <real atomcharge> |  \\
	  polar <real atompolar> | dummy | self | quantum )
\end{verbatim}

These \verb+modify+ commands change the atom type, partial atomic charge,
atomic polarizability, specify a dummy, self-interaction and quantum atom,
respectively. If \verb+mset+ is specified, the modification will only
apply to the specified set, which has to be 1, 2 or 3. If not specified,
the modification will be applied to all three sets. The \verb+atomnam+
should be specified as \verb+<integer isgm>:<string name>+, where
\verb+isgm+ is the segment number, and \verb+name+ is the atom name. A
leading blank in an atom name should be substituted with an underscore.
The modify commands may be combined. For example, the following directive
changes for the specified atom the charge and atom type in set 2 and 
specifies the atom to be a dummy in set 3.

\begin{verbatim}
modify atom 12:_C1 set 2 charge 0.12 type CA set 3 dummy
\end{verbatim}

With the following directives modifications can be made for entire
segments.

\begin{verbatim}
modify segment <integer isgm> [set <integer mset> | initial | final] \\
	( dummy | self | quantum )
\end{verbatim}

Modifications to bonded interaction parameters can be made with the
following modify commands.

\begin{verbatim}
modify ( bond <string atomtyp> <string atomtyp> |  \\
	 angle <string atomtyp> <string atomtyp> <string atomtyp> |        \\ 
 	 torsion <string atomtyp> <string atomtyp> <string atomtyp>        \\
		 <string atomtyp> [ multiplicity <integer multip> ] |      \\
	 plane <string atomtyp> <string atomtyp> <string atomtyp>          \\
		 <string atomtyp> ) [set <integer mset> | initial | final] \\
	 <real value> <real forcon>
\end{verbatim}

where \verb+atomtyp+ and \verb+mset+ are defined as above, \verb+multip+
is the torsion ultiplicity for which the modification is to be applied,
\verb+value+ is the reference bond, angle, torsion angle of out-of-plane
angle value respectively, and \verb+forcon+ is the force constant for
bond, angle, torsion angle of out-of-plane angle. When \verb+multip+
or \verb+mset+ are not defined the modification will be applied to
all multiplicities and sets, respectively, for the identified bonded
interaction.

After modifying atoms to quantum atoms the bonded interactions in which
only quantum atoms are involved are removed from the bonded lists using

\begin{verbatim}
update lists
\end{verbatim}

Error messages resulting from parameters not being defined for bonded
interaction in which only quantum atoms are involved are ignored using

\begin{verbatim}
ignore
\end{verbatim}

\section{Appending to an existing topology file}

\begin{verbatim}
noe <integer isgm1> <string atom1> <integer isgm2> <string atom3> \\
  <real dist1> <real dist2> <real forc1> <real forc2> <real forc3>
\end{verbatim}

This directive specifies a distance restraint.

\begin{verbatim}
select <integer isel> { <string atom1> }
\end{verbatim}

Directive \verb+select+ specifies a group of atoms used in the
definition of potential of mean force potentials.

\begin{verbatim}
pmf ( align | planar ) <integer isel> <real forcon1> <real forcon2>
pmf distance <integer isel> <integer jsel> \\
             <real dist1> <real dist2> <real forcon1> <real forcon2>
pmf angle <integer isel> <integer jsel> <integer ksel> \\
             <real angle1> <real angle2> <real forcon1> <real forcon2>
pmf torsion <integer isel> <integer jsel> <integer ksel> <integer lsel> \\
             <real angle1> <real angle2> <real forcon1> <real forcon2>
\end{verbatim}

Directive \verb+pmf+ specifies a potential of mean force potential
in terms of atom selection.

\section{Generating a restart file}

\begin{verbatim}
new_rst
\end{verbatim}

Keyword \verb+restart+ will cause an existing restart file to be
overwritten with a new file.

The follwing directives control the manipulation of restart
files, and are executed in the order in which they
appear in the \prepare\ input deck.

\begin{verbatim}
solvent name <string*3 slvnam default ``HOH''> \\
        model <string slvmdl default ``spce''>
\end{verbatim}

The solvent keyword can be used to specify the three letter solvent name 
as expected on the PDB formatted file, and the name of the solvent model
for which solvent coordinates will be used.

\begin{verbatim}
solvate ( [ cube <real edge> ] |  \\
          [ box <real xedge> [ <real xedge> [ <real xedge> ]]] | \\
          [ sphere <real radius> ] |
          [ troct <real edge> ])
\end{verbatim}

Solvation can be specified to be in a cubic box with specified edge,
rectangular box with specified edges, or in a sphere with specified
radius. Solvation in a cube or rectangular box will automatically also
set periodic boundary conditions. Solvation in a sphere will only allow
simulations without periodic boundary conditions. The size of the cubic
and rectangular boxes will be expanded by a length specified by the
expand variable. If no shape is specified, solvation will be done for
a cubic box with an edge that leaves 1.0 nm between any solute atom and
the wall after the solute has been centered. An explicit \verb+write+
is not needed to write the restart file. The \verb+solvate+ will
write out a file \verb+sys_calc+.rst.

\begin{verbatim}
touch <real touch default 0.23>
\end{verbatim}

The variable \verb+touch+ specifies the minimum distance between a solvent 
and solute atom for which a solvent molecule will be accepted for solvation.

\begin{verbatim}
expand <real xpndw default 0.1>
\end{verbatim}

The variable \verb+xpndw+ specifies the size in nm with which the simulation
volume will be increased after solvation.

\begin{verbatim}
read [rst | rst_old | pdb] <string filename>
write [rst | [solute [<integer nsolvent>]] (pdb | xyz)] <string filename>
\end{verbatim}

These directives read and write the file \verb+filename+ in the specified
format. The \verb+solute+ option instructs to write out the coordinates
for solute and all, or if specified the first \verb+nsolvent+, crystal solvent 
molecules only.
If no format is specified, it will be derived from the extension of the
filename. Recognized extensions are rst, rst\_old (read only), pdb and xyz
(write only).
Reading and then writing the same restart file will cause the
sub-block size information to be lost. If this information needs to be
retained a shell copy command needs to be used.

\begin{verbatim}
center
\end{verbatim}

This directive centers the solute center of geometry at the origin.

\begin{verbatim}
orient
\end{verbatim}

This directive orients the solute principal axes.

\begin{verbatim}
translate [atom | segment | molecule] \
	 <integer itran> <integer itran> <real xtran(3)>
\end{verbatim}

This directive translates solute atoms in the indicated range by xtran,
without checking for bas contacts in the resulting structure.

\begin{verbatim}
remove solvent [inside | outside] [x <real xmin> <real xmax>] \
[y <real ymin> <real ymax>] [z <real zmin> <real zmax>]
\end{verbatim}

This directive removes solvent molecules inside or outside the
specified coordinate range.

\begin{verbatim}
periodic
\end{verbatim}

This directive enables periodic boundary conditions.

\begin{verbatim}
vacuo
\end{verbatim}

This directive disables periodic boundary conditions.

\begin{verbatim}
grid <integer mgrid default 24> <real rgrid default 0.2>
\end{verbatim}

This directive specifies the grid size of trial counter-ion positions and 
minimum distance between an atom in the system and a counter-ion. 

\begin{verbatim}
fix ( atoms | segments ) ( beyond | within ) <real rfix> <string atmfix>
\end{verbatim}

The \verb+fix+ keyword may be used to specify that the identified
atoms should remain fixed during any operation.

\begin{verbatim}
box <real xsize> <real ysize>  <real zsize>
\end{verbatim}

The \verb+box+ directive resets the box size.

\begin{verbatim}
align <string atomi> <string atomj> <string atomk>
\end{verbatim}

The \verb+align+ directive orients the system such that
\verb+atomi+ and \verb+atomj+ are on the z-axis, and \verb+atomk+
in the x=y plane.

\begin{verbatim}
repeat <integer nx> <integer ny> <integer nz> [<real dist>] [<real zdist>]
\end{verbatim}

The \verb+repeat+ directive causes a subsequent \verb+write pdb+
directive to write out multiple copies of the system, with \verb+nx+
copies in the x, \verb+ny+ copies in the y, and \verb+nz+ copies in 
the z-direction, with a minimum distance of \verb+dist+ between any
pair of atoms from different copies. If \verb+nz+ is -2, an inverted
copy is placed in the z direction, with a separation of \verb+zdist+ nm. 
If \verb+dist+ is negative, the box dimensions will be used. 
For systems with solvent, this directive should be used with a negative 
\verb+dist+.


\chapter{Molecular Dynamics}
\label{sec:NWargos}
\newcommand{\mc}[3]{\multicolumn{#1}{#2}{#3}}
\newcommand{\vb}[1]{\mbox{\verb.#1.}}
\newcommand{\none}{\multicolumn{2}{|c|}{ }}
%%%%%%%\renewcommand{\thetable}{\Roman{table}}
\newcommand{\mcc}[1]{\multicolumn{2}{c}{#1}}
\def\bmu{\mbox{\boldmath $\mu$}}
\def\bE{\mbox{\bf E}}
\def\br{\mbox{\bf r}}
\def\tT{\tilde{T}}
\def\t{\tilde{1}}
\def\ip{i\prime}
\def\jp{j\prime}
\def\ipp{i\prime\prime}
\def\jpp{j\prime\prime}
\def\etal{{\sl et al.}}
\def\nwchem{{\bf NWChem}}
\def\nwargos{{\bf nwARGOS}}
\def\nwtop{{\bf nwTOP}}
\def\nwrst{{\bf nwRST}}
\def\nwsgm{{\bf nwSGM}}
\def\argos{{\bf ARGOS}}

\section{Introduction}

\subsection{Spacial decomposition}
\nwargos\  is the \nwchem\  module for molecular dynamics
simulations of macromolecules and solutions. The module is a parallel
implementation of \argos, a vectorized molecular dynamics
package developed by T.P.Straatsma at the University of Houston.

The distribution of data is based on a spacial decomposition of 
the molecular system, which offers the most efficient parallel 
implementation in terms of both memory requirements and
communication costs, especially for simulations of large molecular 
systems.

Inter-processor communication using the global array tools and the
design of a data structure allowing distribution based on spacial
decomposition are the key elements in taking advantage of
the distribution of memory requirements and computational work with
minimal communication.

In the spacial decomposition approach, the physical simulation
volume is divided into rectangular boxes, each of which is
assigned to a processor. Depending on the conditions of the 
calculation and the number of available processors, each processor 
contains one or more of these spacially grouped boxes.
The most important aspects of this decomposition are the dependence 
of the box sizes and communication cost on the number of processors 
and the shape of the boxes, the frequent reassignment of atoms to 
boxes leading to a fluctuating number of atoms per box, and the 
locality of communication which is the main reason for the efficiency 
of this approach for very large molecular systems.

To improve efficiency, molecular systems are broken up into separately
treated solvent and solute parts.  Solvent molecules are assigned to
the domains according to their center of geometry and are always owned
by a one node. This avoids solvent--solvent bonded interactions
crossing node boundaries.  Solute molecules are broken up into
segments, with each segment assigned to a processor based on its
center of geometry.  This limits the number of solute bonded
interactions that cross node boundaries.  The processor to which a
particular box is assigned is responsible for the calculation of all
interactions between atoms within that box.  For the calculation of
forces and energies in which atoms in boxes assigned to different
processors are involved, data are exchanged between processors. The
number of neighboring boxes is determined by the size and shape of the
boxes and the range of interaction. The data exchange that takes place
every simulation time step represents the main communication
requirements.  Consequently, one of the main efforts is to design
algorithms and data structures to minimize the cost of this
communication. However, for very large molecular systems, memory
requirements also need to be taken into account.

To compromise between these requirements exchange of data is performed
in successive point to point communications rather than using the
shift algorithm which reduces the number of communication calls
for the same amount of communicated data.

For inhomogeneous systems, the computational load of evaluating 
atomic interactions will generally differ between box pairs. 
This will lead to load imbalance between processors. In \nwargos\ 
two algorithms have been implemented that allow for dynamically 
balancing the workload of each processor.
One method is the dynamic resizing of boxes such that boxes gradually
become smaller on the busiest node, thereby reducing the computational
load of that node. Disadvantages of this method are that the 
efficiency depends on the solute distribution in the simulation volume
and the redistribution of work depends on the number of nodes which
could lead to results that depend on the number of nodes used.
The second method is based on the dynamic redistribution of intra-node
box-box interactions. This method represents a more coarse load
balancing scheme, but does not have the disadvantages of the box
resizing algorithm. For most molecular systems the box pair
redistribution is the more efficient and preferred  method.

The description of a molecular system consists of static and dynamic
information. The static information does not change during a
simulation and includes items such as connectivity, excluded and third
neighbor lists, equilibrium values and force constants for all
bonded and non-bonded interactions. The static information is called
the topology of the molecular system, and is kept on a separate
topology file. The dynamic information includes coordinates and
velocities for all atoms in the molecular system, and is kept in a
so-called restart file.

\subsection{Topology}
\label{sec:nwatopology}
The static information about a molecular system that is needed for
a molecular simulation is provided to the simulation module in a
topology file. 
Items in this file include, among many other things, 
a list of atoms, their non-bonded parameters for van der Waals and
electrostatic interactions, and the complete connectivity in terms
of bonds, angles and dihedrals.

In \nwargos\ molecular systems, a distinction is made between 
{\it solvent} and {\it solute}, which are treated separately.
A solvent molecule is defined only once in the topology file,
even though many solvent molecules usually are included in the
actual molecular system. In the current implementation only one 
solvent can be defined. Everything that is not solvent in the 
molecular system is solute. Each solute atom in the system must 
be explicitly defined in the topology. 

Molecules are defined in terms of one or more {\it segment}s. 
Typically, repetitive parts of a molecule are each defined as a single
segment, such as the amino acid residues in a protein.. 
Segments  can be quite complicated to define and are, therefore, 
collected in a set of database files. 
The definition of a molecular system in terms of segments is a
{\it sequence}.

A utility \nwtop\ reads the sequence file, retrieves all needed 
segments from the available segment databases, substitutes parameters
from the force field parameter database files, and generates the
topology file.

A utility \nwsgm\ reads a rudimentary, force-field independent
{\it fragment} defining atom types and connectivity, and 
constructs a template for a force-field dependent segment.
Fragments can also be collected into a set of fragment database
files.

A utility \nwrst\ generates a {\it restart} file, given a topology
and coordinates from a PDB formatted file or another restart file.
This utility can also be used to solvate a molecular system.

\subsection{Files}
\label{sec:nwafilenames}

File names used by \nwargos\ have the form \verb+$project$_$id$.$ext$+, with
exception of the topology file (Section \ref{sec:nwatopology}), which is named 
\verb+$project$.top+.
Anything that refers to the definition of the chemical system can be used
for \verb+<project>+, as long as no periods or underlines are used.
The identifier \verb+<id>+ can be anything that refers to the type of 
calculation to be performed for the system with the topology defined.
This file naming convention allows for the creation of a single
topology file \verb+$project$.top+ that can be used for a number of 
different calculations, each identified with a different \verb+<id>+.
For example, if {\tt crown.top} is the name of the topology file for
a crown ether, {\tt crown\_em}, {\tt crown\_md}, {\tt crown\_mcti} could
be used with appropriate extensions for the filenames for energy
minimization, molecular dynamics simulation and multiconfiguration
thermodynamic integration, respectively. All of these calculations
would use the same topology file {\tt crown.top}.

\label{sec:nwaextensions}

The extensions \verb+<ext>+ identify the kind of information on a file,
and are pre-determined. 
The complete list of extensions is
\begin{tabbing}
xxxxx\=\kill
{\bf coo} \> coordinate trajectory file\\
{\bf day} \> dayfile\\
{\bf frg} \> fragment file\\
{\bf gib} \> free energy data file\\
{\bf out} \> output file\\
{\bf prp} \> property file\\
{\bf qrs} \> quenched restart file, resulting from an energy minimization\\
{\bf rst} \> restart file, used to start, restart or continue a simulation \\
{\bf seq} \> sequence file, describing the system in segments\\
{\bf sco} \> solute coordinate trajectory file\\
{\bf sgm} \> segment file, describing segments\\
{\bf svl} \> solute velocity trajectory file\\
{\bf syn} \> synchronization time file\\
{\bf tim} \> timing analysis file\\
{\bf top} \> topology file, contains the static description of a system\\
{\bf vel} \> velocity trajectory file\\
\end{tabbing}

\subsection{Databases}
Database file names used by the setup programs \nwtop\ (Section \ref{sec:nwanwtop}), 
\nwrst\ (Section \ref{sec:nwanwrst}) and \nwsgm\ (Section \ref{sec:nwanwsgm}) have the form
\verb+$forcefield$_$level$.$ext$+, where \verb+<forcefield>+ is any of the
supported force fields (Section \ref{sec:nwaforcefields}). The source of the data is 
identified by \verb+<level>+, and can be 
\begin{center}
\begin{tabular}{lll}
\hline
level   & Description                 & Availability \\
{\bf s} & original published data     & public       \\
{\bf x} & additional published data   & public       \\
{\bf u} & user preferred data         & private      \\
{\bf t} & user defined temporary data & private    \\
\hline
\end{tabular}
\end{center}

Only the level {\bf s} and {\bf x} databases are publicly available. 
The user is responsible for the private level {\bf u} and {\bf t}
databases. When the utility programs scan the databases, the priority
is {\bf t}$>${\bf u}$>${\bf x}$>${\bf s}$>$.

The extension \verb+<ext>+ defines the type of database file. The
complete list of extensions is
\begin{tabbing}
xxxxx\=\kill
{\bf att} \> atom type translation\\
{\bf frg} \> fragments\\
{\bf par} \> parameters\\
{\bf seq} \> sequences\\
{\bf sgm} \> segments\\
\end{tabbing} 

Filenames including their paths of the different databases should be 
defined in a file 
{\tt .nwargos} in a user's home directory, and provides the user the
option to select which database files are scanned.

\subsection{Force fields}
\label{sec:nwaforcefields}
Force fields supported by \nwargos\ are
\begin{center}
\begin{tabular}{lll}
\hline
Keyword      & Force field   & Current status \\
{\tt amber}  & AMBER4.0      & att, par, sgm  \\
{\tt charmm} & CHARMM        & att, par       \\
{\tt cvff}   & CVFF          & att, par       \\
{\tt gromos} & GROMOS87      & att, par       \\
{\tt oplsa}  & OPLS/AMBER3.0 & att, par       \\
{\tt oplsg}  & OPLS/GROMOS87 & att, par       \\
\hline
\end{tabular}
\end{center}  

\section{Creating fragment files}
Fragment files contain the basic information needed to specify all
interactions that need to be considered in a molecular simulation.
The format of the fragment files is described in Table \ref{tbl:nwafrag}

\begin{table}[htbp]

\label{tbl:nwafrag}

\center

\begin{tabular}{lll}
\hline\hline
Card & Format & Description \\ \hline
I-1-1  & a1     & \$ to identify the start of a fragment \\ % $ for emacs
I-1-2  & a10    & name of the fragment, the tenth character\\
       &        & N: identifies beginning of a chain\\
       &        & C: identifies end of a chain\\
       &        & blank: identifies chain fragment\\
       &        & M: identifies an integral molecule\\
\hline
I-2-1  & i5     & number of atoms in the fragment\\ 
\hline
\multicolumn{3}{l}{For each atom one deck II} \\
\hline
II-1-1  & i5     & atom sequence number \\
II-1-2  & a6     & atom name \\
II-1-3  & a5     & atom type \\
II-1-4  & a1     & dynamics type\\
        &        & \verb+ + : normal\\
        &        & \verb+D+ : dummy atom\\
        &        & \verb+S+ : solute interactions only\\
        &        & \verb+Q+ : quantum atom\\
        &        & other : intramolecular solute interactions only\\
II-1-5  & i5     & link number\\
II-1-6  & i5     & environment type\\
        &        & 0: no special identifier\\
        &        & 1: planar, using improper torsion\\
        &        & 2: R-stereomer, using improper torsion\\
        &        & 3: S-stereomer, using improper torsion\\
        &        & 4: atom in aromatic ring\\
II-1-7  & i5     & charge group\\
II-1-8  & i5     & polarization group\\
II-1-9  & f12.6  & atomic partial charge\\
II-1-10 & f12.6  & atomic polarizability\\
\hline
\multicolumn{3}{l}{Any number of cards in deck III to specify complete 
connectivity} \\
\hline
III-1-1  & 16i5   & connectivity, duplication allowed\\ 
\hline\hline
\end{tabular}

\caption{The format of NWArgos fragment files.}
\end{table}

\section{Creating segment files}
\label{sec:nwanwsgm}
Program \nwsgm\ can be used to generate a template for a segment file 
from a corresponding fragment file. The segment file contains all
information for the calculation of bonded and non-bonded interactions
for a given chemical system using a specific force field. If a
fragment is available in a local file or in a database file, the
segment can be generated using
\begin{verbatim}
  nwsgm <string fragment> <string ffield>
\end{verbatim}
where \verb+<fragment>+ is the name of the fragment. Since fragment
names rely on the last character for the type of fragment, spaces
should be replaced by underlines. \verb+<ffield>+ should be the
name of an available force field.

The program \nwsgm\ only provides a template for a segment. It is
often needed to make additional changes in this file. One important
restriction is that dihedral interactions may only involve atoms on at
most two segments. The segment entries define three sets of parameters
for each interaction. Free energy perturbations can be performed using
set 1 for the generation of the ensemble while using sets 2 and/or 3
as perturbations. Free energy multiconfiguration thermodynamic
integration and multistep thermodynamic perturbation calculations are
performed by gradually changing the interactions in the system from
parameter set 2 to parameter set 3.  The format of a segment is
described in Tables \ref{tbl:nwaseg1}--\ref{tbl:nwaseg6}.

\begin{table}[htbp]
\center

\label{tbl:nwaseg1}

\begin{tabular*}{150mm}{p{12mm}p{12mm}l}
\hline\hline
Deck  & Format & Description \\ \hline
I-1-1 & a1     & \$ to identify the start of a segment \\ %$ for emacs
I-1-2 & a10    & name of the segment, the tenth character\\
      &        & N: identifies beginning of a chain\\
      &        & C: identifies end of a chain\\
      &        & blank: identifies chain fragment\\
      &        & M: identifies an integral molecule\\
I-2-1 & i5     & number of atoms in the segment\\
I-2-2 & i5     & number of bonds in the segment\\
I-2-3 & i5     & number of angles in the segment\\
I-2-4 & i5     & number of proper dihedrals in the segment\\
I-2-5 & i5     & number of improper dihedrals in the segment\\
\hline
\end{tabular*}

\caption{NWArgos segment file format, table 1 of 6.}
\end{table}

\begin{table}[htbp]
\center

\label{tbl:nwaseg2}

\begin{tabular*}{150mm}{p{12mm}p{12mm}l}
\hline\hline
Deck & Format & Description \\ \hline
\multicolumn{3}{l}{For each atom one deck II} \\
II-1-1  & i5     & atom sequence number \\
II-1-2  & a6     & atom name \\
II-1-3  & a5     & atom type, generic set 1 \\
II-1-4  & a1     & dynamics type\\
        &        & \verb+ + : normal\\
        &        & \verb+D+ : dummy atom\\
        &        & \verb+S+ : solute interactions only\\
        &        & \verb+Q+ : quantum atom\\
        &        & other : intramolecular solute interactions only\\
II-1-4  & a5     & atom type, generic set 2 \\
II-1-5  & a1     & dynamics type\\
        &        & \verb+ + : normal\\
        &        & \verb+D+ : dummy atom\\
        &        & \verb+S+ : solute interactions only\\
        &        & \verb+Q+ : quantum atom\\
        &        & other : intramolecular solute interactions only\\
II-1-6  & a5     & atom type, generic set 3 \\
II-1-7  & a1     & dynamics type\\
        &        & \verb+ + : normal\\
        &        & \verb+D+ : dummy atom\\
        &        & \verb+S+ : solute interactions only\\
        &        & \verb+Q+ : quantum atom\\
        &        & other : intramolecular solute interactions only\\
II-1-8  & i5     & charge group\\
II-1-9  & i5     & polarization group\\
II-1-10 & i5     & link number\\
II-1-11 & i5     & environment type\\
        &        & 0: no special identifier\\
        &        & 1: planar, using improper torsion\\
        &        & 2: R-stereomer, using improper torsion\\
        &        & 3: S-stereomer, using improper torsion\\
        &        & 4: atom in aromatic ring\\
II-2-1  & f12.6  & atomic partial charge in e, set 1\\
II-2-2  & f12.6  & atomic polarizability/$4\pi\epsilon_o$ in nm$^3$, set 1\\
II-2-3  & f12.6  & atomic partial charge in e, set 2\\
II-2-4  & f12.6  & atomic polarizability/$4\pi\epsilon_o$ in nm$^3$, set 2\\
II-2-5  & f12.6  & atomic partial charge in e, set 3\\
II-2-6  & f12.6  & atomic polarizability/$4\pi\epsilon_o$ in nm$^3$, set 3\\
\hline
\end{tabular*}

\caption{NWArgos segment file format, table 2 of 6.}
\end{table}

\begin{table}[htbp]
\center

\label{tbl:nwaseg3}

\begin{tabular*}{150mm}{p{12mm}p{12mm}l}
\hline\hline
Deck & Format & Description \\ \hline
\multicolumn{3}{l}{For each bond a deck III} \\
III-1-1 & i5     & bond sequence number \\
III-1-2 & i5     & bond atom i \\
III-1-3 & i5     & bond atom j \\
III-1-4 & i5     & bond type \\
        &        & 0: harmonic\\
        &        & 1: constrained bond\\
III-1-5 & i5     & bond parameter origin\\
        &        & 0: from database, next card ignored \\
        &        & 1: from next card\\
III-2-1 & f12.6  & bond length in nm, set 1\\
III-2-2 & e12.5  & bond force constant in kJ nm$^2$ mol$^{-1}$, set 1 \\
III-2-3 & f12.6  & bond length in nm, set 2\\
III-2-4 & e12.5  & bond force constant in kJ nm$^2$ mol$^{-1}$, set 2 \\
III-2-5 & f12.6  & bond length in nm, set 3\\
III-2-6 & e12.5  & bond force constant in kJ nm$^2$ mol$^{-1}$, set 3 \\
\hline
\end{tabular*}

\caption{NWArgos segment file format, table 3 of 6.}

\end{table}

\begin{table}
\center

\label{tbl:nwaseg4}

\begin{tabular*}{150mm}{p{12mm}p{12mm}l}
\hline\hline
Deck & Format & Description \\ \hline
\multicolumn{3}{l}{For each angle a deck IV} \\
IV-1-1 & i5     & angle sequence number \\
IV-1-2 & i5     & angle atom i \\
IV-1-3 & i5     & angle atom j \\
IV-1-4 & i5     & angle atom k \\
IV-1-5 & i5     & angle type \\
       &        & 0: harmonic\\
IV-1-6 & i5     & angle parameter origin\\
       &        & 0: from database, next card ignored \\
       &        & 1: from next card\\
IV-2-1 & f12.6  & angle in radians, set 1\\
IV-2-2 & e12.5  & angle force constant in kJ mol$^{-1}$, set 1 \\
IV-2-3 & f12.6  & angle in radians, set 2\\
IV-2-4 & e12.5  & angle force constant in kJ mol$^{-1}$, set 2 \\
IV-2-5 & f12.6  & angle in radians, set 3\\
IV-2-6 & e12.5  & angle force constant in kJ mol$^{-1}$, set 3 \\
\hline
\end{tabular*}

\caption{NWArgos segment file format, table 4 of 6.}

\end{table}

\begin{table}[htbp]
\center

\label{tbl:nwaseg5}

\begin{tabular*}{150mm}{p{12mm}p{12mm}l}
\hline\hline
Deck & Format & Description \\ \hline
\multicolumn{3}{l}{For each proper dihedral a deck V} \\
V-1-1 & i5     & proper dihedral sequence number \\
V-1-2 & i5     & proper dihedral atom i \\
V-1-3 & i5     & proper dihedral atom j \\
V-1-4 & i5     & proper dihedral atom k \\
V-1-5 & i5     & proper dihedral atom l \\
V-1-6 & i5     & proper dihedral type \\
      &        & 0: $C\cos(m\phi-\delta)$\\
V-1-7 & i5     & proper dihedral parameter origin\\
      &        & 0: from database, next card ignored \\
      &        & 1: from next card\\
V-2-1 & i5     & multiplicity, set 1\\
V-2-2 & f12.6  & proper dihedral in radians, set 1\\
V-2-3 & e12.5  & proper dihedral force constant in kJ mol$^{-1}$, set 1 \\
V-2-4 & i5     & multiplicity, set 2\\
V-2-5 & f12.6  & proper dihedral in radians, set 2\\
V-2-6 & e12.5  & proper dihedral force constant in kJ mol$^{-1}$, set 2 \\
V-2-7 & i5     & multiplicity, set 3\\
V-2-8 & f12.6  & proper dihedral in radians, set 3\\
V-2-9 & e12.5  & proper dihedral force constant in kJ mol$^{-1}$, set 3 \\
\hline
\end{tabular*}

\caption{NWArgos segment file format, table 5 of 6.}
\end{table}

\begin{table}[htbp]
\center

\label{tbl:nwaseg6}

\begin{tabular*}{150mm}{p{12mm}p{12mm}l}
\hline\hline
Deck & Format & Description \\ \hline
\multicolumn{3}{l}{For each improper dihedral a deck VI} \\
VI-1-1 & i5     & improper dihedral sequence number \\
VI-1-2 & i5     & improper dihedral atom i \\
VI-1-3 & i5     & improper dihedral atom j \\
VI-1-4 & i5     & improper dihedral atom k \\
VI-1-5 & i5     & improper dihedral atom l \\
VI-1-6 & i5     & improper dihedral type \\
       &        & 0: harmonic\\
VI-1-7 & i5     & improper dihedral parameter origin\\
       &        & 0: from database, next card ignored \\
       &        & 1: from next card\\
VI-2-1 & f12.6  & improper dihedral in radians, set 1\\
VI-2-2 & e12.5  & improper dihedral force constant in kJ mol$^{-1}$, set 1 \\
VI-2-3 & f12.6  & improper dihedral in radians, set 2\\
VI-2-4 & e12.5  & improper dihedral force constant in kJ mol$^{-1}$, set 2 \\
VI-2-5 & f12.6  & improper dihedral in radians, set 3\\
VI-2-6 & e12.5  & improper dihedral force constant in kJ mol$^{-1}$, set 3 \\
\hline\hline
\end{tabular*}

\caption{NWArgos segment file format, table 6 of 6.}

\end{table}


\section{Creating sequence files}
A sequence file describes a molecular system in terms of segments. The
file format is given in Table \ref{tbl:nwaseq}

\begin{table}[htbp]
\center

\label{tbl:nwaseq}

\begin{tabular*}{150mm}{p{12mm}p{12mm}l}
\hline\hline
Card & Format & Description \\ \hline
I-1-1  & a1     & \$ to identify the start of a sequence \\ %$ for emacs
I-1-2  & a10    & name of the sequence\\
\multicolumn{3}{l}{Any number of cards in deck II to specify the system} \\
II-1-1 & i5     & segment number, except\\
       &        &  0: identifies solvent, last segment, following \verb+-4+\\
       &        & -1: identifies a break in the chain\\
       &        & -2: identifies end of a molecule\\
       &        & -3: identifies end of a molecule fraction\\
       &        & -4: identifies end of the solute section\\
       &        & -5: identifies end of the sequence file\\
II-1-2 & a10    & segment name, last character will be determined from chain\\
II-1-3 & i5     & link segment 1, if blank previous segment in chain\\
II-1-4 & i3     & link atom, if blank link atom 2\\
II-1-5 & i5     & link segment 2, if blank next segment in chain\\
II-1-6 & i3     & link atom, if blank link atom 1\\
II-1-7 & i5     & link segment 3\\
II-1-8 & i3     & link atom in link segment 3\\
II-1-9 & i5     & link segment 4\\
II-1-0 & i3     & link atom in link segment 4\\
II-1-1 & i5     & link segment 5\\
II-1-2 & i3     & link atom in link segment 5\\
II-1-3 & i5     & link segment 6\\
II-1-4 & i3     & link atom in link segment 6\\
II-1-5 & i5     & link segment 7\\
II-1-6 & i3     & link atom in link segment 7\\
II-1-7 & i5     & link segment 8\\
II-1-8 & i3     & link atom in link segment 8\\
II-1-9 & i5     & link segment 9\\
II-1-0 & i3     & link atom in link segment 9\\
II-1-1 & i5     & link segment 10\\
II-1-2 & i3     & link atom in link segment 10\\
\hline\hline
\end{tabular*}

\caption{The NWArgos sequence file format.}
\end{table}

\section{Creating topology files}
\label{sec:nwanwtop}

The topology (Section \ref{sec:nwatopology}) describes all static information
that describes a molecular system. This includes the connectivity in
terms of bond-stretching, angle-bending and torsional interactions, as well as
the non-bonded van der Waals and Coulombic interactions.
The topology of a molecular system is generated by the topology
generator \nwtop\ from the sequence in terms of segments as read from
the sequence file. For each unique segment specified in this file the 
segment databases are searched for the segment definition. For
segments not found in one of the database files a segment definition
needs to be generated, for example using the utility \nwsgm (Section \ref{sec:nwanwsgm}).

The command line to run \nwtop\ is
\begin{verbatim}
  nwtop <string project> <string forcefield>
\end{verbatim}
where \verb+<project>+ refers to the name of a sequence in one of the
sequence databases or in a local file \verb+$project$.seq+, and
\verb+<forcefield>+ is the name of one of the available force fields 
(Section \ref{sec:nwaforcefields}).
If the sequence is found, the segment definitions are read from the
segment databases or from a local file \verb+$project$.sgm+. If all
segments are found, the parameter substitutions are performed, using
force field parameters taken from the parameter databases or from a 
local file \verb+$project$.par+. After all lists have been generated the
topology is written to a local topology file \verb+$project$.top+.

\section{Creating restart files}
\label{sec:nwanwrst}
Restart files contain all dynamical information about a molecular
system and are created using preparation utility \nwrst. The command
line is
\begin{verbatim}
  nwrst <string project>
\end{verbatim}
in which case directives are taken from \verb+$project$.rin+, or
\begin{verbatim}
  nwrst
\end{verbatim}
in which case directives are taken from standard input.

The directives can be any of
\begin{itemize}

\item
\begin{verbatim}
  title
\end{verbatim}
reading three title cards immediately following the directive

\item
\begin{verbatim}
  read topology <string filtop>
\end{verbatim}
where \verb+<filtop>+ is the topology file name. In general, this
directive needs to be given before any coordinates can be read from a
PDB formatted file or an existing restart file, or solvent coordinates
can be read for solvation.

\item
\begin{verbatim}
  read solvent <string filslv>
\end{verbatim}
where \verb+<filslv>+ is the solvent name. In general, this directive
needs to be given before any coordinates can be read from a PDB
formatted file or an existing restart file.

\item
\begin{verbatim}
  read PDB <string filpdb>
\end{verbatim}
where \verb+<filpdb>+ is the file with coordinates in PDB format. This
directive, in general needs to be preceded by reading the topology
file and reading the solvation coordinates file.

\item
\begin{verbatim}
  write PDB <string filpdc>
\end{verbatim}
where \verb+<filpdc>+ is the file with coordinates in PDB format

\item
\begin{verbatim}
  read rst <string filrsi>
\end{verbatim}
where \verb+<filrsi>+ is a restart file. This directive, in general needs
to be preceded by reading the topology file and reading the solvation
coordinates file.

\item
\begin{verbatim}
  write rst <string filrst>
\end{verbatim}
where \verb+<filrst>+ is a restart file

\item
\begin{verbatim}
  orient
\end{verbatim}
changes the orientation of the solute to fit the smallest rectangular
box with largest dimension along z and smallest dimension along x.

\item
\begin{verbatim}
  solvate box [<real bx>[<real by>[<real bz>]]]
\end{verbatim}
solvates the system in a rectangular box with dimensions 
bx $\times$ by $\times$ bz in nm

\item
\begin{verbatim}
  center
\end{verbatim}
to put the solute center of geometry in the center of the
simulation box

\item
\begin{verbatim}
  expand with <real xx>[<real xy>[<real xz>]]
\end{verbatim}
to expand the simulation box with xx, xy and xz in the
x, y and z direction respectively

\item
\begin{verbatim}
  expand to <real xx>[<real xy>[<real xz>]]
\end{verbatim}
to expand the simulation box to xx, xy and xz in the
x, y and z direction respectively

\item
\begin{verbatim}
  boxsize
\end{verbatim}
to resize the box to fit all atoms present

\item
\begin{verbatim}
  periodic
\end{verbatim}
to specify the use of periodic boundary conditions

\item
\begin{verbatim}
  vacuo
\end{verbatim}
to specify a system in vacuo

\item
\begin{verbatim}
  set touch <real touch>
\end{verbatim}
to specify the minimum distance between any atom in
the system and a solvent atom to be added during solvation

\item
\begin{verbatim}
  velocities (solvent || solute) zero
\end{verbatim}
to initialize velocities

\item
\begin{verbatim}
  reference (solvent || solute) zero
\end{verbatim}
to initialize reference coordinates

\item
\begin{verbatim}
  (end || exit || quit || stop)
\end{verbatim}
to stop the restart file generation
\end{itemize}

\section{Molecular simulations}
The type of molecular dynamics simulation is specified by the
NWChem task directive.
\begin{verbatim}
  task md [ energy || optimize || dynamics || thermodynamics ]
\end{verbatim}
where the theory keyword {\tt md} specifies use of the molecular
dynamics module, and the operation keyword is one of
\begin{itemize}
\item
{\tt energy} for single configuration energy evaluation
\item
{\tt optimize} for energy minimization
\item
{\tt dynamics} for molecular dynamics simulations and single step
thermodynamic perturbation free energy molecular dynamics simulations
\item
{\tt thermodynamics} for combined multiconfiguration thermodynamic
integration and multiple step thermodynamic perturbation free
energy molecular dynamics simulations.
\end{itemize}

\section{System specification}
The chemical system for a calculation is specified in the topology
and restart files. These files should be created using the utilities
\nwtop\ and \nwrst\ before a simulation can be performed.
The names of these files are determined from the NWChem START directive.
There is no default. If the project name is given as {\tt pr\_id},
the topology file used is {\tt pr.top}, while all other files
are named {\tt pr\_id.ext}.

\section{Parameter set}
\begin{itemize}
\item
The parameter set used for the simulation is specified by
\begin{verbatim}
  set <integer iset>
\end{verbatim}
where \verb+<iset>+ is the parameter set found on the
topology file. The default for \verb+<iset>+ is 1.
\item
The perturbation parameter set used in molecular dynamics simulations
to evaluate single step thermodynamic perturbation free energies is 
specified by
\begin{verbatim}
  pset <integer isetp1> [<integer isetp2>]
\end{verbatim}
where \verb+<isetp1>+ specifies the first perturbation parameter set and
\verb+<isetp2>+ specifies the second perturbation parameter set. Legal
values for \verb+<isetp1>+ are 2 and 3. Legal value for \verb+<isetp2>+ is
3, in which case \verb+<isetp1>+ can only be 2. If specified, \verb+<iset>+
is automatically set to 1.
\end{itemize}

\section{Energy minimization algorithms}
The energy minimization of the system as found in the restart file 
is performed with the following directives.
\begin{itemize}
\item
To specify steepest descent steps
\begin{verbatim}
  sd <integer msdit> [init <real dx0sd>] [min <real dxsdmx>] \
                     [max <real dxmsd>] 
\end{verbatim}
where \verb+<msdit>+ is the maximum number of steepest descent steps taken,
for which the default is 100, \verb+<dx0sd>+ is the initial step size in nm
for which the default is 0.001, \verb+<dxsdmx>+ is the threshold for the
step size in nm for which the default is 0.0001, and \verb+<dxmsd>+ is the
maximum allowed step size in nm for which the default is 0.05.
\item
To specify conjugate gradient steps
\begin{verbatim}
  cg <integer mcgit> [init <real dx0cg>] [min <real dxcgmx>] \
                     [cy <integer ncgcy>]
\end{verbatim}
where \verb+<mcgit>+ is the maximum number of conjugate gradient steps taken,
for which the default is 100, \verb+<dx0cg>+ is the initial search
interval size in nm
for which the default is 0.001, \verb+<dxcgmx>+ is the threshold for the
step size in nm for which the default is 0.0001, and \verb+<ncgcy>+
is the number of conjugate gradient steps after which the gradient history
is discarded for which the default is 10.
\end{itemize}
Steepest descent energy minimization precedes conjugate 
gradient minimization if both are specified.

\section{Multiconfiguration thermodynamic integration}
The following keywords control free energy difference simulations.
Multiconfiguration thermodynamic integrations are always combined
with multiple step thermodynamic perturbations.
\begin{itemize}
\item
To specify the direction and number of ensembles
\begin{verbatim}
  (forward || reverse) [[<integer mrun> of] <integer maxlam>]
\end{verbatim}
with {\tt forward} being the default direction, and
where \verb+<mrun>+ is the number of ensembles that will be generated in
this calculation, and \verb+<maxlam>+ is the total number of ensembles
for the thermodynamic integration. The default value for \verb+<maxlam>+
is 21. The default value of \verb+<mrun>+ is the value of \verb+<maxlam>+.
\item
To specify the maximum statistical error per ensemble
\begin{verbatim}
  error <real edacq>
\end{verbatim}
where \verb+<edacq>+ is the maximum error allowed in the ensemble average 
derivative of the Hamiltonian with respect to lambda with a default
of 5.0 kJ~mol$^{-1}$.
\item
To specify the maximum drift in the derivative per ensemble
\begin{verbatim}
  drift <real ddacq>
\end{verbatim}
where \verb+<ddacq>+ is the maximum drift allowed in this
ensemble average with a default of 5.0 kJ~mol$^{-1}$ps$^{-1}$.
\item
To specify the size factor of an ensemble compared to the previous
ensemble
\begin{verbatim}
  factor <real fdacq>
\end{verbatim}
where \verb+<fdacq>+ is the minimum size of an ensemble relative to the
previous ensemble in the calculation with a default value of 0.75.
\item
To specify that a free energy decomposition has to be carried out
\begin{verbatim}
  decomp
\end{verbatim}
Since free energy contributions are path dependent, results from a
decomposition analysis can not be unambiguously interpreted, and
the default is not to perform this decomposition.
\item
To specify separation-shifted scaling
\begin{verbatim}
  sss [delta <real delta>]
\end{verbatim}
where \verb+<delta>+ is the separation-shifted scaling factor with a default
of 0.075 nm$^2$.
\item
To specify the starting point
\begin{verbatim}
  new || renew || extend
\end{verbatim}
where {\tt new} indicates that this is an initial mcti calculation, which
is also the default. {\tt renew} instructs to obtain the initial
conditions for each $\lambda$ from the {\bf mro}-file from a previous 
mcti calculation, which has to be renamed to an {\bf mri}-file. The
keyword {\tt extend} will extend a previous mcti calculation from the
data read from an {\bf mri}-file.
\end{itemize}

\section{Time and integration algorithm directives}
Following directives control the integration of the equations of motion.
\begin{itemize}
\item
To specify the integration algorithm
\begin{verbatim}
  leapfrog || vverlet
\end{verbatim}
where {\tt leapfrog} specifies the default leap frog integration, and
{\tt vverlet} specifies the velocity Verlet integrator.
\item
To specify the number of equilibration steps
\begin{verbatim}
  equil <integer mequi>
\end{verbatim}
where \verb+<mequi>+ is the number of equilibration steps, with a default
of 100.
\item
To specify the number of data gathering steps
\begin{verbatim}
  data <integer mdacq> [over <integer ldacq>]]
\end{verbatim}
where \verb+<mdacq>+ is the number of data gathering steps with a
default of 500. In multiconfiguration thermodynamic integrations
\verb+<mequi>+ and \verb+<mdacq>+ are for each of the ensembles, and
variable \verb+<ldacq>+ specifies the minimum number of data gathering steps 
in each ensemble. In regular molecular dynamics simulations \verb+<ldacq>+
is not used. The default value for \verb+<ldacq>+ is the value of \verb+<mdacq>+.
\item
To specify start time
\begin{verbatim}
  time <real stime>
\end{verbatim}
where \verb+<stime>+ is the start time of a molecular simulation in ps,
with a default of 0.0.
\item
To specify the time step
\begin{verbatim}
  step <real tstep>
\end{verbatim}
where \verb+<tstep>+ is the time step in ps, with 0.001 as the default value.
\end{itemize}

\section{Ensemble selection}
Following directives control the ensemble type.
\begin{itemize}
\item
To specify a constant temperature ensemble using Berendsen's thermostat
\begin{verbatim}
  isotherm [<real tmpext>] [trelax <real tmprlx> [<real tmsrlx>]]
\end{verbatim}
where \verb+<tmpext>+ is the external temperature with a default of 298.15~K,
and \verb+<tmprlx>+ and \verb+<tmsrlx>+ are temperature relaxation times in ps 
with a default of 0.1. If only \verb+<tmprlx>+ is given the complete system
is coupled to the heat bath with relaxation time \verb+<tmprlx>+. If both
relaxation times are supplied, solvent and solute are independently coupled
to the heat bath with relaxation times \verb+<tmprlx>+ and \verb+<tmsrlx>+,
respectively.
\item
To specify a constant pressure ensemble using Berendsen's piston
\begin{verbatim}
  isobar [<real prsext>] [trelax <real prsrlx> ] \
         [compress <real compr>]
\end{verbatim}
where \verb+<prsext>+ is the external pressure with a default of 1.025~10$^5$ Pa,
\verb+<prsrlx>+ is the pressure relaxation time in ps with a default of 0.5, and
\verb+<compr>+ is the system compressibility in m$^2$N$^{-1}$ with a
default of 4.53E-10.
\end{itemize}

\section{Velocity reassignments}
Velocities can be periodically reassigned to reflect a certain temperature.
\begin{itemize}
\item
\begin{verbatim}
  vreass <integer nfgaus> <real tgauss>
\end{verbatim}
specifies that velocities will be reassigned every \verb+<nfgaus>+ molecular
dynamics steps, reflecting a temperature of \verb+<tgauss>+~K. The default
is not to reassign velocities, i.e.\ \verb+<nfgaus>+ is 0.
\end{itemize}

\section{Cutoff radii}
Cutoff radii can be specified for short range and long range interactions.
\begin{itemize}
\item
\begin{verbatim}
  cutoff [short] <real rshort> [long <real rlong>] \
         [qmmm <real rqmmm>]
\end{verbatim}
where \verb+<rshort>+ is the short range cutoff radius in nm, and \verb+<rlong>+
is the long range cutoff radius in nm. If the long range cutoff radius
is larger than the short range cutoff radius the twin range method will
be used, in which short range forces and energies are evaluated every
molecular dynamics step, and long range forces and energies with a
frequency of \verb+<nflong>+ molecular dynamics steps. Keyword
\verb+qmmm+ specifies the radius of the zone around quantum atoms
defining the qmmm bare charges.
The default value for \verb+<rshort>+, \verb+<rlong>+ and \verb+<rqmmm>+ is 0.9~nm.
\end{itemize}

\section{Polarization}
First order and self consistent electronic polarization models have
been implemented.
\begin{itemize}
\item
The use of polarizable potentials is specified by
\begin{verbatim}
  polar (first || scf [[<integer mpolit>] <real ptol>])
\end{verbatim}
where the keyword {\tt first} specifies the first order polarization
model, and {\tt scf} specifies the self consistent polarization field
model, iteratively determined with a maximum of \verb+<mpolit>+
iterations to within a tolerance of \verb+<ptol>+ D in the generated
induced dipoles. The default is not to use polarization models.
\end{itemize}

\section{External electrostatic field}
\begin{itemize}
\item
An external electrostatic field can be specified by
\begin{verbatim}
  field <real xfield> [freq <real xffreq>] [vector <real xfvect(1:3)>]
\end{verbatim}
where \verb+<xfield>+ is the field strength, \verb+<xffreq>+ is the
frequency in MHz and \verb+<xfvect>+ is the external field vector.
\end{itemize}

\section{Constraints}
Constraints involving hydrogens are satisfied using the SHAKE 
coordinate resetting procedure.
\begin{itemize}
\item
\begin{verbatim}
  shake [<integer mshitw> [<integer mshits>]]  \
        [<real tlwsha> [<real tlssha>]]
\end{verbatim}
where \verb+<mshitw>+ is the maximum number of solvent SHAKE iterations,
and \verb+<mshits>+ is the maximum number of solute SHAKE iterations. If
only \verb+<mshitw>+ is specified, the value will also be used for \verb+<mshits>+.
The default maximum number of iterations is 100 for both.
\verb+<tlwsha>+ is the solvent SHAKE tolerance in nm, and \verb+<tlssha>+ is
the solute SHAKE tolerance in nm. If only \verb+<tlwsha>+ is specified, the
value given will also be used for \verb+<tlssha>+. The default tolerance
is 0.001 for both.
\item
\begin{verbatim}
  noshake {solvent || solute}
\end{verbatim}
disables SHAKE and treats the bonded interaction according to the
force field.
\end{itemize}

\section{Long range interaction corrections}
Long range electrostatic interactions are implemented using the
smooth particle mesh Ewald technique.
\begin{itemize}
\item
\begin{verbatim}
  pme [grid <integer ng>] [alpha <real ealpha>] [order <integer morder>]
\end{verbatim}
where $ng$ is the number of grid points per dimension, $ealpha$ is
the Ewald coefficient in nm$^{-1}$, with a default of 4.0, and
$morder$ is order of the Cardinal B-spline interpolation which must
be an even number and at least 4 (default value).
\end{itemize}

\section{Fixing coordinates}
The solvent or solute part of a system may be fixed or unfixed using
the following keywords. Fixing part of the system will not propagate to
simulations using restart files written.
\begin{itemize}
\item
\begin{verbatim}
  fix (all || solvent || solute || non-H)
\end{verbatim}
\item
\begin{verbatim}
  unfix (all || solvent || solute || non-H)
\end{verbatim}
\end{itemize}

\section{Autocorrelation function}
For the evaluation of the statistical error of multiconfiguration
thermodynamic integration free energy results a correlated data 
analysis is carried out, involving the calculation of the
autocorrelation function of the derivative of the Hamiltonian with
respect to the control variable $\lambda$.
\begin{itemize}
\item 
The calculation of the autocorrelation is controlled with the keywords
\begin{verbatim}
  auto <integer lacf> [fit <integer nfit>] [weight <real weight>]
\end{verbatim}
where \verb+<lacf>+ is the length of the autocorrelation function, with
a default of 1000, \verb+<nfit>+ is the number of functions used in the
fit of the autocorrelation function, with a default of 15, and
\verb+<weight>+ is the weight factor for the autocorrelation function,
with a default value of 0.0.
\end{itemize}

\section{Print options}
Keywords that control print to the output file, with extension {\bf out}.
\begin{itemize}
\item
Printing topology information
\begin{verbatim}
  print topol [nonbond] [solvent] [solute]
\end{verbatim}
where {\tt nonbond} refers to the non-bonded interaction parameters,
{\tt solvent} to the solvent bonded parameters, and {\tt solute} to the
solute bonded parameters. If only {\tt topol} is specified, all
topology information will be printed to the output file.
\item
Printing time step information
\begin{verbatim}
  print step <integer nfoutp> [extra] [energy]
\end{verbatim}
where \verb+nfoutp+ is the frequency of printing molecular dynamics step
information to the output file. If the keyword {\tt extra} is specified
additional energetic data are printed for solvent and solute separately.
If the keyword {\tt energy} is specified, information is printed for
all bonded solute interactions.
The default for \verb+nfoutp+ is 0. For molecular dynamics simulations
this frequency is in time steps, and for multiconfiguration thermodynamic
integration in $\lambda$ steps.
\item
Printing statistical information
\begin{verbatim}
  print stat <integer nfstat>
\end{verbatim}
where \verb+<nfstat>+ is the frequency of printing statistical information
of properties that are calculated during the simulation. 
For molecular dynamics simulation
this frequency is in time steps, for multiconfiguration thermodynamic
integration in $\lambda$ steps.
\end{itemize}
Print directives may be combined to a single directive.

\section{Periodic updates}
Following keywords control periodic events during a molecular
dynamics or thermodynamic integration simulation.
\begin{itemize}
\item
Updating pair lists
\begin{verbatim}
  update pairs <integer nfpair>
\end{verbatim}
where \verb+<nfpair>+ is the frequency in molecular dynamics steps of 
updating the pair lists. The default for the frequency is 1.
In addition, pair lists are also updated after each step in which
recording of the restart or trajectory files is performed. Updating
the pair lists includes the redistribution of atoms that changed
domain and load balancing, if specified.
\item
Updating long range forces
\begin{verbatim}
  update long <integer nflong>
\end{verbatim}
where \verb+<nflong>+ is the frequency in molecular dynamics steps 
of updating the long range forces. The default frequency is 1.
The distinction of short range and long range forces is only
made if the long range cutoff radius was specified to be larger
than the short range cutoff radius. Updating the long range forces
is also done in every molecular dynamics step in which the
pair lists are regenerated.
\item
Updating the simulation volume center
\begin{verbatim}
  update center <integer nfcntr> [fraction <integer idscb(1:5)>]
\end{verbatim}
where \verb+<nfcntr>+ is the frequency in molecular dynamics steps in 
which the center of geometry of the solute(s) is translated to the
center of the simulation volume. The solute fractions determining the
solutes that will be centered are specified by the keyword 
{\tt fraction} and the vector \verb+<idscb>+, with a maximum of 5 entries.
This translation is implemented such that it has no effect on any 
aspect of the simulation. The default is not to center, i.e. nfcntr is
0. The default fraction used to center solute is 1.
\item
Periodic removal of center of mass motion is specified by the
following keyword.
\begin{verbatim}
  update motion <integer nfslow>
\end{verbatim}
where \verb+<nfslow>+ is the frequency in molecular dynamics steps in
which the center of mass motion is removed.
\item
Updating the radial distribution functions
\begin{verbatim}
  update rdf <integer nfrdf> [range <real rrdf>] [bins <integer ngl>]
\end{verbatim}
where \verb+<nfrdf>+ is the frequency in molecular dynamics steps in 
which contributions to the radial distribution functions are
evaluated. The default is 0. The range of the radial distribution
functions is given by \verb+<rrdf>+ in nm, with a default of the short
range cutoff radius. Note that radial distribution functions is not
evaluated beyond the short range cutoff radius. The number of
bins in each radial distribution function is given by \verb+<ngl>+, with
a default of 1000.
If radial distribution function are to be
calculated, a {\bf rdi} files needs to be available in which the
contributions are specified as follows.
\begin{center}
\begin{tabular}{lll}
\hline\hline
Card & Format & Description \\ \hline
I-1  & i & Type, 1=solvent-solvent, 2=solvent-solute,
3-solute-solute\\
I-2  & i & Number of the rdf for this contribution\\
I-3  & i & First atom number \\
I-4  & i & Second atom number \\ 
\hline
\end{tabular}
\end{center}
\end{itemize}
Update directives may be combined to a single directive.

\section{Recording}
The following keywords control recording data to file.
\begin{itemize}
\item
The file format of selected recording files is specified with
\begin{verbatim}
  record (binary || ascii [ecce || argos])
\end{verbatim}
with the default of ascii in ecce readable format.
\item
The restart file, file extension {\bf rst}
\begin{verbatim}
  record rest <integer nfrest> [keep]
\end{verbatim}
where \verb+<nfrest>+ is the frequency in molecular dynamics steps
of writing information to this file. For multiconfiguration
thermodynamic integration simulations the frequency is in
steps in $\lambda$. The default is not to record. The restart
file is used to start or restart simulations. The keyword {\tt keep}
causes all restart files written to be kept on disk, rather than
be overwritten, 
\item
The coordinate trajectory file, file extension {\bf coo}
\begin{verbatim}
  record coord <integer nfcoor>
\end{verbatim}
where \verb+<nfcoor>+ is the frequency in molecular dynamics steps
of writing information to this file. For multiconfiguration
thermodynamic integration simulations the frequency is in
steps in $\lambda$. The default is not to record.
\item
The solute coordinate trajectory file, file extension {\bf sco}
\begin{verbatim}
  record scoor <integer nfscoo>
\end{verbatim}
where \verb+<nfscoo>+ is the frequency in molecular dynamics steps
of writing information to this file. For multiconfiguration
thermodynamic integration simulations the frequency is in
steps in $\lambda$. The default is not to record.
\item
The velocity trajectory file, file extension {\bf vel}
\begin{verbatim}
  record veloc <integer nfvelo>
\end{verbatim}
where \verb+<nfvelo>+ is the frequency in molecular dynamics steps
of writing information to this file. For multiconfiguration
thermodynamic integration simulations the frequency is in
steps in $\lambda$. The default is not to record.
\item
The solute velocity trajectory file, file extension {\bf svl}
\begin{verbatim}
  record svelo <integer nfsvel>
\end{verbatim}
where \verb+<nfsvel>+ is the frequency in molecular dynamics steps
of writing information to this file. For multiconfiguration
thermodynamic integration simulations the frequency is in
steps in $\lambda$. The default is not to record.
\item
The property file, file extension {\bf prp}
\begin{verbatim}
  record prop <integer nfprop>
\end{verbatim}
where \verb+<nfprop>+ is the frequency in molecular dynamics steps
of writing information to this file. For multiconfiguration
thermodynamic integration simulations the frequency is in
steps in $\lambda$. The default is not to record.
\item
The energy minimization trajectory, file extension {\bf emt}
\begin{verbatim}
  record mind <integer nfem>
\end{verbatim}
where \verb+<nfem>+ is the frequency in energy minimization steps of
writing the minimization trajectory to file.
\item
The free energy file, file extension {\bf gib}
\begin{verbatim}
  record free <integer nffree>
\end{verbatim}
where \verb+<nffree>+ is the frequency in multiconfiguration
thermodynamic integration simulations to record data to this file.
The default is 1, i.e.\ to record at every $\lambda$.
\item
The free energy convergence file, file extension {\bf cnv}
\begin{verbatim}
  record cnv
\end{verbatim}
\item
The free energy derivative autocorrelation file, file extension {\bf acf}
\begin{verbatim}
  record acf
\end{verbatim}
\item
The free energy vs. time file, file extension {\bf fet}
\begin{verbatim}
  record fet
\end{verbatim}
\item
The synchronization time file, file extension {\bf syn}
\begin{verbatim}
  record sync <integer nfsync>
\end{verbatim}
where \verb+<nfsync>+ is the frequency in molecular dynamics steps
of writing information to this file. The default is not to record.
The information written is the simulation time, the wall clock time
of the previous MD step, the wall clock time of the previous force
evaluation, the total synchronization time, the largest
synchronization time and the node on which the largest synchronization
time was found.
\end{itemize}
Record directive may be combined to a single directive.

\section{Program control options}
\begin{itemize}
\item
Load balancing is determined by
\begin{verbatim}
  load (none || size || pairs)
\end{verbatim}
where the default is {\tt none}. Load balancing option {\tt size}
is resizing boxes on a node, and {\tt pairs} redistributes the
box-box interactions over nodes.
\item
The distribution of the available nodes over the three Cartesian
dimensions is performed automatically such that, $npx*npy*npz=np$
and $npx<=npy<=npz$, where $npx$, $npy$ and $npz$ are the nodes in the
x, y and z dimension respectively, and $np$ is the number of nodes
allocated for the calculation. Where more than one combination
of $npx$, $npy$ and $npz$ are possible, the combination is chosen with
the minimum value of $npx+npy+npz$. To change the default setting
the following optional input option is provided.
\begin{verbatim}
  nodes <integer npx> <integer npy> <integer npz>
\end{verbatim}
\item
The molecular system is decomposed into boxes, that form the smallest
unit for communication of atomic data between nodes. The size of the
boxes is automatically set to the short-range cutoff radius. If
long-range cutoff radii  are used the box size is set to half the
long-range cutoff radius if it is larger than the short-range cutoff.
The number of boxes per dimension can also be set explicitly, using
the following keyword.
\begin{verbatim}
  boxes <integer nbx> <integer nby> <integer nbz>
\end{verbatim}
If the number of boxes in a dimension is less than the number of
processors in that dimension, the number of boxes is set to the number
of processors.
\item
In rare events the amount of memory set aside per node is insufficient
leading to aborts because {\tt mwm} or {\tt msa} is too small. Jobs
may be restarted with additional space allocated by
\begin{verbatim}
  extra <integer madbox>
\end{verbatim}
where \verb+<madbox>+ is the number of additional boxes that are allocated
on each node. The default for \verb+<madbox>+ is 6. In some cases \verb+<madbox>+
can be reduced to 4 if memory usage is a concern. Values of 2 or less
will almost certainly result in memory shortage.
\item
To limit the allocated amount of memory used by the molecular dynamics
module,
\begin{verbatim}
  memory <integer memlim>
\end{verbatim}
where $memlim$ is the amount of memory allocated by the module in
kB. Per default all available memory is allocated.
%\item
%For development purposes debug information can be written to the debug
%file with extension {\bf dbg} with
%\begin{verbatim}
%debug <i idebug>
%\end{verbatim}
%where $idebug$ specifies the type of debug information being written.
\end{itemize}







\chapter{Electrostatic Potentials}
\label{sec:esp}

The NWChem Electrostatic Potential (ESP) module derives partial atomic 
charges that fit the quantum mechanical electrostatic potential on selected
grid points.

The ESP module is specified by the NWChem task directive
\begin{verbatim}
task esp
\end{verbatim}

The input for the module is taken from the ESP input block
\begin{verbatim}
esp
end
\end{verbatim}

\section{Grid specification}
The grid points for which the quantum mechanical electrostatic potential is 
evaluated and used in the fitting procedure of the partial atomic charges
all lie outside the van der Waals radius of the atoms and within a cutoff
distance from the atomic centers. The following input parameters determine
the selection of grid points.
\begin{itemize}
\item
If a grid file is found, the grid will be read from that file. If no grid
file is found, or the keyword
\begin{verbatim}
  recalculate
\end{verbatim}
is given, the grid and the electrostatic potential is recalculated.
\item
The extent of the grid is determined by
\begin{verbatim}
  range <real rcut>
\end{verbatim}
where \verb+rcut+ is the maximum distance in $nm$ between a grid point and
any of the atomic centers. When omitted, a default value for \verb+rcut+ of
0.3 $nm$ is used.
\item
The grid spacing is specified by
\begin{verbatim}
  spacing <real spac>
\end{verbatim}
where \verb+spac+ is the grid spacing in $nm$ for the regularly spaced
grid points. If not specified, a default spacing of 0.1 $nm$ is used.
\item
The van der Waals radius of an element can be specified by
\begin{verbatim}
  radius <integer iatnum> <real atrad>
\end{verbatim}
where \verb+iatnum+ is the atomic number for which a van der Waals radius
of \verb+atrad+ in $nm$ will be used in the grid point determination.
Default values will be used for atoms not specified.
\item
The probe radius in nm determining the envelope around the molecule is
specified by
\begin{verbatim}
  probe <real probe default 0.07> 
\end{verbatim}
\item
The distance between atomic center and probe center can be multiplied
by a constant factor specified by
\begin{verbatim}
  factor <real factor default 1.0> 
\end{verbatim}
All grid points are discarded that lie within a distance 
\verb-factor*(radius(i)+probe)- from any atom $i$.
\item
Schwarz screening is applied using
\begin{verbatim}
  screen [<real scrtol default 1.0D-5>]
\end{verbatim}
\end{itemize}

\section{Constraints}
Additional constraints to the partial atomic charges can be imposed during
the fitting procedure.
\begin{itemize}
\item
The net charge of a subset of atoms can be constrained using
\begin{verbatim}
  constrain <real charge> {<integer iatom>}
\end{verbatim}
where \verb+charge+ is the net charge of the set of atoms \verb+{iatom}+.
A negative atom number \verb+iatom+ can be used to specify that the
partial charge of that atom is substracted in the sum for the set.
\end{itemize}

\section{Restraints}
Restraints can be applied to each partial charge using the RESP charge
fitting procedure.
\begin{itemize}
\item
The directive for charge restraining is
\begin{verbatim}
  restrain [hfree] (harmonic [<real scale>] || \
   hyperbolic [<real scale> [<real tight>]]  \
    [maxiter <integer maxit>]  [tolerance <real toler>])
\end{verbatim}
where \verb+hfree+ can be specified to exclude hydrogen atoms from the
restaining procecure. Variable \verb+scale+ is the strength of the 
restraint potential, with a default of $0.005 au$ for the harmonic
restraint and a default value of $0.001 au$ for the hyperbolic restraint.
For the hyperbolic restraints the tightness \verb+tight+ can be specified
to change the default value of $0.1 e$. The iteration count that needs to
be carried out for the hyperbolic restraint is determined by the
maximum number of allowed iterations \verb+maxiter+, with a default value
of 25, and the tolerance in the convergence of the partial charges
\verb+toler+, with a default of $0.001 e$.
\end{itemize}


\chapter{Combined Quantum and Molecular Mechanics}
% $Id: qmmm.tex,v 1.5 1997-06-24 06:41:35 d3e129 Exp $

\label{sec:qmmm}

Combined or hybrid Quantum Mechanics and Molecular Mechanics (QM/MM)
is a simulation methodology that is about 15 years old but in all the
literature there are cautions that calibration computations must be
done to validate the model for each particular chemical system
studied.  This is not a black box style computation and the NWChem
users are advised that without calibration QM/MM may not give the
appropriate results\footnote{c.f., Singh and Kollman, J. Comp. Chem.
  {\bf 7}, 718 (1986); M.~J.~Field, P.~A.~Bash and M.~Karplus, J.
  Comp. Chem. {\bf 11}, 700, (1990); J. Gao, ``Methods and
  Applications of Combined Quantum Mechanical and Molecular Mechanical
  Potentials.'' In {\it Reviews in Computational Chemistry};
  K.~B.~Lipkowitz, D.~B.~Boyd, Eds.; VCH Publishers: New York, 199X;
  Vol. 7, pp 119-185 (1995); and M. A. Thompson and G. K. Schenter, J.
  Phys. Chem {\bf 99} 6374 (1995) }.

The QM/MM module in NWChem is driven by the molecular dynamics module
(nwARGOS).  This module currently works for any QM method that has
analytic gradients\footnote{The QM/MM method will work with numerical
  gradients available in NWChem, but it is expected that the
  performance will not allow any substantive simulations}.  The input
for this requires the definition of chemical system via the same
interface that is used by the nwARGOS module (c.f. Section
\ref{sec:nwARGOS}).  The extensions to this interface include the
definition of ``Quantum'' atoms and ``Link'' where appropriate.  The
QM information must be present in the traditional NWChem input deck
except for the geometry\footnote{Any geometry information in the
  traditional form will be ignored}.  The geometrical information will
be constructed automatically by nwARGOS.  For dynamics and free energy
simulations the input is again identical to that for nwARGOS with
limitations on the kinds of simulations that can be done.

The QM/MM module is invoked with the task directive where the
``theory'' is QMMM.  The recognized operations on the QM/MM theory
directive are energy, optimize, and dynamics.

\begin{verbatim}
  TASK QMMM (energy || optimize || dynamics)
\end{verbatim}

Tasks \verb+gradient+, \verb+saddle+, \verb+frequencies+ and
\verb+thermodynamics+ are currently not available in the QM/MM mode.  


The QM/MM input consists of the standard NWChem input block:
\begin{verbatim}
  QMMM
    ...
  END
\end{verbatim}

The \verb+QMMM+ has the following the additional sub-directive that the user
may specify for the particular simulation.  These options currently are:

\section{EATOMS}
There is one compound input directive that must exist for the QM/MM
simulation to proceed.  This sets the relative zero of energy for the
QM component of the system.  It is not incorrect to leave this value as
zero but the energetics of the QM system will likely over shadow the
MM component of the system.  Properties based on energy fluctuations
of the system will be overly sensitive to the energy of the QM
component of the system.  The zero of energy for the MM system is by
definition of most parameterized force fields the separated atom
energy.  The zero of energy for QM systems by definition of most QM
methods is the vacuum.  The {\it a priori} determination of the
separated atom energy for a particular QM method is not well defined
and thus leads to a number of assumptions or guess work depending upon
the particular QM method being utilized.  Therefor, the determination
of the QM separated atom energy (``eatoms'') is left to the user.  The
input takes the form:

\begin{verbatim}
  EATOMS <real eatoms>
\end{verbatim}

There is no default for this and the input {\bf must} be present for a
QM/MM simulation.  

All other parameters that control the QM/MM simulation are set via the
input to nwARGOS (see chapter \ref{sec:nwARGOS}).



\chapter{Analysis}
\label{sec:analysis}
\def\bmu{\mbox{\boldmath $\mu$}}
\def\bE{\mbox{\bf E}}
\def\br{\mbox{\bf r}}
\def\tT{\tilde{T}}
\def\t{\tilde{1}}
\def\ip{i\prime}
\def\jp{j\prime}
\def\ipp{i\prime\prime}
\def\jpp{j\prime\prime}
\def\etal{{\sl et al.}}
\def\nwchem{{\bf NWChem}}
\def\nwargos{{\bf nwargos}}
\def\nwtop{{\bf nwtop}}
\def\nwrst{{\bf nwrst}}
\def\nwsgm{{\bf nwsgm}}
\def\esp{{\bf esp}}
\def\md{{\bf md}}
\def\prepare{{\bf prepare}}
\def\analysis{{\bf analysis}}
\def\argos{{\bf ARGOS}}
\def\amber{{\bf AMBER}}
\def\charmm{{\bf CHARMM}}
\def\discover{{\bf DISCOVER}}
\def\povray{{\bf povray}}
\def\gopenmol{{\bf gOpenMol}}
\def\ecce{{\bf ecce}}

The \analysis\ module is used to analyze molecular trajectories generated
by the \nwchem\ molecular dynamics module, or partial charges generated
by the \nwchem\ electrostatic potential fit module. This module should
not de run in parallel mode.

Directives for the \analysis\ module are read from an input deck,

\begin{verbatim}
analysis
 ...
end
\end{verbatim}

The analysis is performed  as post-analysis of trajectory files through 
using the {\rm task} directive

\begin{verbatim}
task analysis
\end{verbatim}
or
\begin{verbatim}
task analyze
\end{verbatim}

\section{Reference coordinates}

Most analyses require a set of reference coordinates. These
coordinates are read from a \nwchem\ restart file by the directive,

\begin{verbatim}
reference <string filename>
\end{verbatim}

where {\rm filename} is the name of an existing restart file. 
This input directive is required.

\section{File specification}

The trajectory file(s) to be analyzed are specified with

\begin{verbatim}
file <string filename> [<integer firstfile> <integer lastfile>] 
\end{verbatim}

where {\rm filename} is an existing {\rm trj} trajectory file.
If {\rm firstfile} and {\rm lastfile} are specified, the specified
{\rm filename} needs to have a {\rm ?} wild card character that will 
be substituted by the 3-character integer number from {\rm firstfile} 
to {\rm lastfile}, and the analysis will be performed on the series 
of files.
For example,

\begin{verbatim}
file tr_md?.trj 3 6
\end{verbatim}

will instruct the analysis to be performed on files {\it tr\_md003.trj},
{\it tr\_md004.trj}, {\it tr\_md005.trj} and {\it tr\_md006.trj}.

\par
From the specified files the subset of frames to be analyzed is 
specified by

\begin{verbatim}
frames [<integer firstframe default 1>] <integer lastframe> \
       [<integer frequency default 1>]
\end{verbatim}

For example, to analyze the first 100 frames from the specified
trajectory files, use

\begin{verbatim}
frames 100
\end{verbatim}

To analyze every 10-th frame between frames 200 and 400 recorded on
the specified trajectory files, use

\begin{verbatim}
frames 200 400 10
\end{verbatim}

To set hydrogen bond criteria:

\begin{verbatim}
hbond [distance [[<real rhbmin default 0.0>] <real rhbmin>]] \
      [donorangle [<real hbdmin> [ <real hbdmax default pi>]]] \
      [acceptorangle [<real hbamin> [ <real hbamax default pi>]]]
\end{verbatim}

\section{Selection}

Analyses can be applied to a selection of solute atoms. The selection
is determined by

\begin{verbatim}
select [ <integer isgm> [ <integer jsgm> ]] [ { <string atom> } ]
\end{verbatim}

where {\rm isgm} is the first segment number, {\rm jsgm} is the last 
segment number in the selection, and {\rm \{atom\}} is the set of atom
names selected from the specified residues. By default all solute
atoms are selected.
\par
For example, all protein backbone atoms are selected by

\begin{verbatim}
select _N _CA _C
\end{verbatim}

To select the backbone atoms in residues 20 to 80 only, use

\begin{verbatim}
select 20 80 _N _CA _C
\end{verbatim}

This selection is reset to apply to all atoms after each file
directive.

Solvent molecules within \verb+range+ nm from any selected solute atom
are selected by

\begin{verbatim}
select solvent <real range>
\end{verbatim}

After solvent selection, the solute atom selection is reset to being all
selected.

\par

Some analysis are performed on groups of atoms. These groups of atoms
are define by

\begin{verbatim}
define <integer igroup> [<real rsel>] [<integer isel> [<integer jsel>]] \\
 [solvent] { <string atom> }
\end{verbatim}

The atom string in this definitions takes the form

\begin{verbatim}
[<integer segment>:]<string atomname>
\end{verbatim}

In the atomname a question mark may be used as a wildcard character.

\section{Coordinate analysis}

To analyze the root mean square deviation from the specified reference
coordinates:

\begin{verbatim}
rmsd
\end{verbatim}

To analyze protein $\phi$-$\psi$ and backbone hydrogen bonding:

\begin{verbatim}
ramachandran
\end{verbatim}

To define a bond:

\begin{verbatim}
bond <integer ibond> <string atomi> <string atomj> 
\end{verbatim}

To define an angle:

\begin{verbatim}
angle <integer iangle> <string atomi> <string atomj> <string atomk> 
\end{verbatim}

To define a torsion:

\begin{verbatim}
torsion <integer itorsion> <string atomi> <string atomj> \
                       <string atomk> <string atoml> 
\end{verbatim}

The atom string in these definitions takes the form

\begin{verbatim}
<integer segment>:<string atomname> | w<integer molecule>:<string atomname>
\end{verbatim}

for solute and solvent atom specification, respectively.

To define charge distribution in z-direction:

\begin{verbatim}
charge_distribution <integer bins>
\end{verbatim}

Analyses on pairs of atoms in predefined groups are specified by

\begin{verbatim}
groups [<integer igroup> [<integer jgroup>]] [periodic [<integer ipbc default 3>]] \ 
       <string function> [<real value1> [<real value2>]] [<string filename>]
\end{verbatim}

where $igroup$ and $jgroup$ are groups of atoms defined with a
\verb+define+ directive. Keyword \verb+periodic+ specifies that
periodic boundary conditions need to be applied in $ipbc$ dimensions.
The type of analysis is define by $function$, $value1$ and $value2$.
If $filename$ is specified, the analysis is applied to the reference
coordinates and written to the specified file. If no filename is
given, the analysis is applied to the specified trajectory and 
performed as part of the \verb+scan+ directive.
Implemented analyses defined by 
\verb+<string function> [<real value1> [<real value2>]]+ include
\verb+distances+ to calculate all atomic distances between atoms
in the specified groups that lie between $value1$ and $value2$.

Coordinate histograms are specified by

\begin{verbatim}
histogram <integer idef> [<integer length>] zcoordinate <string filename>
\end{verbatim}

where $idef$ is the atom group definition number, $length$ is the size
of the histogram, \verb+zcoordinate+ is the currently only histogram option,
and $filename$ is the filname to which the histogram is written.

To perform the coordinate analysis:

\begin{verbatim}
scan [ super ] <string filename>
\end{verbatim}

which will create, depending on the specified analysis options
files filename.rms and filename.ana. After the scan directive
previously defined coordinate analysis options are all reset.
Optional keyword \verb+super+ specifies that frames read from
the trajectory file(s) are superimposed to the reference structure
before the analysis is performed.

\section{Essential dynamics analysis}

Essential dynamics analysis is performed by

\begin{verbatim}
essential
\end{verbatim}

This can be followed by one or more

\begin{verbatim}
project <integer vector> <string filename>
\end{verbatim}

to project the trajectory onto the specified vector. This will
create files filename with extensions frm or trj, val, vec, \_min.pdb
and \_max.pdb, with the projected trajectory, the projection
value, the eigenvector, and the minimum and maximum projection
structure.

For example, an essential dynamics analysis with projection onto
the first vector on files firstvec.\{frm, val, vec, \_min.pdb, \_max.pdb\}
is generated by

\begin{verbatim}
essential
project 1 firstvec.frm
\end{verbatim}

\section{Trajectory format conversion}

To write a single frame in PDB or XYZ format, use

\begin{verbatim}
write  [super] [solute] [<integer number default 1>] <string filename>
\end{verbatim}

To copy the selected frames from the specified trejctory file(s),
onto a new file, use

\begin{verbatim}
copy  [solute] <string filename>
\end{verbatim}

To superimpose the selected atoms for each specified frame to the 
reference coordinates before copying onto a new file, use

\begin{verbatim}
super [solute] <string filename>
\end{verbatim}

The format of the new file is determined from the extension, which
can be one of

\begin{tabular}{rl}
amb & \amber\ formatted trajectory file (obsolete)\\
arc & \discover\ archive file\\
bam & \amber\ unformatted trajectory file\\
crd & \amber\ formatted trajectory file\\
dcd & \charmm\ formatted trajectory file\\
esp & \gopenmol\ formatted electrostatic potential files\\
frm & \ecce\ frames file (obsolete)\\
pov & \povray\ input files\\
trj & \nwchem\ trajectory file\\
\end{tabular}

If no extension is specified, a {\rm trj} formatted file will be written.

A special tag can be added to {\rm frm} and {\rm pov} formatted files  using

\begin{verbatim}
label <integer itag> <string tag>  [ <real rval default 1.0> ] \\
    [ <integer iatag> [ <integer jatag default iatag> ] [ <real rtag default 0.0> ] ]
    [ <string anam> ]
\end{verbatim}

where tag number $itag$ is set to the string $tag$ for all atoms
anam within a distance $rtag$ from segments $iatag$ through $jatag$.
A question mark can be used in anam as a wild card character.
\par

Atom rendering is specified using

\begin{verbatim}
render ( cpk | stick )  [ <real rval default 1.0> ] \\
    [ <integer iatag> [ <integer jatag default iatag> ] [ <real rtag default 0.0> ] ]
    [ <string anam> ]
\end{verbatim}

for all atoms anam within a distance $rtag$ from segments $iatag$ through $jatag$,
and a scaling factor of $rval$. A question mark can be used in anam as a wild card 
character.
\par

Atom color is specified using

\begin{verbatim}
color ( <string color> | atom ) \\
    [ <integer iatag> [ <integer jatag default iatag> ] [ <real rtag default 0.0> ] ]
    [ <string anam> ]
\end{verbatim}

for all atoms anam within a distance $rtag$ from segments $iatag$ through $jatag$.
A question mark can be used in anam as a wild card character.
\par
For example, to display all carbon atoms in segments 34 through 45 
in green and rendered cpk in povray files can be specified with

\begin{verbatim}
render cpk 34 45 _C??
color green 34 45 _C??
\end{verbatim}

Coordinates written to a pov file can be scaled using

\begin{verbatim}
scale <real factor>
\end{verbatim}

A zero or negative scaling factor will scale the coordinates to
lie within [-1,1] in all dimensions.
\par
The cpk rendering in povray files can be scaled by

\begin{verbatim}
cpk <real factor default 1.0>
\end{verbatim}
\par

The stick rendering in povray files can be scaled by

\begin{verbatim}
stick <real factor default 1.0>
\end{verbatim}

The initial sequence number of esp related files is defined by

\begin{verbatim}
index <integer index default 1>
\end{verbatim}

\section{Electrostatic potentials}

A file in plt format of the electrostatic potential resulting
from partial charges generated by the ESP module is generated
by the command

\begin{verbatim}
esp  [ <integer spacing default 10> ] \
     [ <real rcut default 1.0> ] [periodic [<integer iper default 3>]] \
     [ <string xfile> [ <string pltfile> ] ]
\end{verbatim}

The input coordinates are taken from the {\rm xyzq} file that can
be generated from a {\rm rst} by the prepare module. Parameter 
spacing specifies the number of gridpoints per nm, rcut specifies 
extent of the charge grid beyond the molecule.
Periodic boundaries will be used if \verb+periodic+
is specified. If \verb+iper+ is set to 2, periodic boundary
conditions are applied in x and y dimensions only. If \verb+periodic+
is specified, a negative value of \verb+rcut+ will extend the grid
in the periodic dimensions by abs(\verb+rcut+), otherwise this value
will be ignored in the periodic dimensions.
The resulting {\rm plt} formatted file pltfile can be
viewed with the gOpenMol program. The resulting electrostatic 
potential grid is in units of kJ\ mol$^{-1}$e$^{-1}$.
If no files are specified, only the parameters are set.




\chapter{Pseudopotential Plane-Wave DFT (PSPW)}
%
% $Id: pspw.tex,v 1.43 2007-12-21 18:48:11 bylaska Exp $
%
\label{sec:pspw}

\newcounter{algcounter}[chapter]
\def\thealgcounter{\thechapter.\arabic{algcounter}}
\newenvironment{algorithm}[1]
               { \refstepcounter{algcounter}
                \begin{center}
                  {\bf Algorithm} \thealgcounter: #1
                \end{center}
               \begin{center}\begin{enumerate} \begin{em}}
               {\end{em}\end{enumerate}\end{center}}
 

The NWChem plane-wave (NWPW) module uses pseudopotentials and
plane-wave basis sets to perform Density Functional Theory
calculations.  This module complements the capabilities of the more
traditional Gaussian function based approaches by having an accuracy at least as good 
for many applications, yet is still fast enough to treat systems containing hundreds of
atoms.  Another significant advantage is its ability to simulate
dynamics on a ground state potential surface directly at run-time
using the Car-Parrinello algorithm.  This method's efficiency and
accuracy make it a desirable first principles method of simulation in
the study of complex molecular, liquid, and solid state systems.
Applications for this first principles method include the calculation
of free energies, search for global minima, explicit simulation of
solvated molecules, and simulations of complex vibrational modes that
cannot be described within the harmonic  approximation.

The NWPW module is a collection of three modules.
\begin{itemize}
   \item PSPW - (PSeudopotential Plane-Wave) A gamma point code for
     calculating molecules, liquids, crystals, and  surfaces.  
   \item Band - A band structure code for calculating
     crystals and surfaces with small band gaps (e.g. semi-conductors
     and metals).
   \item PAW - a (gamma point) projector augmented plane-wave code
     for calculating molecules, crystals, and surfaces 
\end{itemize}
The PSPW, Band, and PAW modules can be used to compute the energy and  optimize the
geometry.  Both the PSPW and Band modules can also be used to find saddle points, and 
compute numerical second derivatives.  In addition the PSPW module can also be used 
to perform Car-Parrinello molecular  dynamics.

Section \ref{sec:pspw_tasks} describes the tasks contained within the
PSPW module, section \ref{sec:band_tasks} describes the tasks
contained within the Band module, section \ref{sec:paw_tasks} describes
the tasks contained within the PAW module, and section \ref{sec:psp_library}
describes the pseudopotential library included with NWChem.  The
datafiles used by the PSPW module are described in section
\ref{sec:pspw_data}.  Car-Parrinello output data files are described
in section \ref{sec:pspw_cp_data}, and the minimization and
Car-Parrinello algorithms are described in 
% sections \ref{sec:pspw_Minimize}-\ref{sec:pspw_Car-Parrinello}. 
section \ref{sec:pspw_Car-Parrinello}. 
Examples of how
to setup and run a PSPW geometry optimization, a Car-Parrinello
simulation, a band structure minimization, and a PAW geometry
optimization are presented in sections \ref{sec:pspw_sd}, \ref{sec:pspw_cp}, and
\ref{sec:band_tutorial1}, and \ref{sec:paw_tutorial}.  
Finally in section \ref{sec:pspw_limits} the capabilities and limitations of the NWPW module are  discussed.

If you are a first time user of this module it is recommended that you skip the next five sections and proceed directly to the tutorials in sections 
\ref{sec:pspw_sd}-\ref{sec:paw_tutorial}.

\section{PSPW Tasks}
\label{sec:pspw_tasks}

All input to the PSPW Tasks is contained within the compound PSPW block,
\begin{verbatim}
PSPW
   ...
END
\end{verbatim}

To perform an actual calculation a TASK PSPW directive is used
(Section \ref{sec:task}).  
\begin{verbatim}
  TASK PSPW
\end{verbatim} 
In addition to the directives listed in Section \ref{sec:task}, i.e.
\begin{verbatim}
TASK pspw energy          
TASK pspw gradient         
TASK pspw optimize         
TASK pspw saddle           
TASK pspw freqencies       
TASK pspw vib
\end{verbatim}
there are additional directives that are specific to the PSPW module, which are:
\begin{verbatim}
TASK PSPW [Car-Parrinello             ||
           pspw_dplot                 ||
           wannier                    ||
           psp_generator              ||
           steepest_descent           ||
           psp_formatter              ||
           wavefunction_initializer   ||
           v_wavefunction_initializer ||
           wavefunction_expander       ]
\end{verbatim}


Once a user has specified a geometry, the PSPW module can be invoked
with no input  directives (defaults invoked throughout).  However, the
user will probably always specify the  simulation cell used in the
computation, since the default simulation cell is not well suited for
most systems.  There are sub-directives which allow for customized
application; those currently provided as options for the PSPW module are:
\begin{verbatim}
PSPW
  CELL_NAME <string cell_name default 'cell_default'>
  INPUT_WAVEFUNCTION_FILENAME  <string input_wavefunctions  default input_movecs>
  OUTPUT_WAVEFUNCTION_FILENAME <string output_wavefunctions default input_movecs>
  FAKE_MASS <real fake_mass default 400000.0>
  TIME_STEP <real time_step default 5.8>
  LOOP <integer inner_iteration outer_iteration default 10 100>
  TOLERANCES <real tole tolc default 1.0e-7 1.0e-7>
  CUTOFF              <real cutoff>
  ENERGY_CUTOFF       <real ecut default (see input description)>
  WAVEFUNCTION_CUTOFF <real wcut default (see input description)>
  EWALD_NCUT <integer ncut default 1>]
  EWALD_RCUT <real rcut default (see input description)>
  XC (Vosko || LDA || PBE96 || revPBE || HF || PBE0 || revPBE0 || 
     LDA-SIC    || LDA-SIC/2    ||  LDA-0.4SIC    || LDA-SIC/4    || LDA-0.2SIC ||
     PBE96-SIC  || PBE96-SIC/2  ||  PBE96-0.4SIC  || PBE96-SIC/4  || PBE96-0.2SIC ||
     revPBE-SIC || revPBE-SIC/2 ||  revPBE-0.4SIC || revPBE-SIC/4 || revPBE-0.2SIC ||
     default Vosko)
  DFT||ODFT||RESTRICTED||UNRESTRICTED
  MULT <integer mult default 1>
  MULLIKEN
  EFIELD
  ALLOW_TRANSLATION

  SIMULATION_CELL            ... (see input description) END
  DPLOT                      ... (see input description) END
  WANNIER                    ... (see input description) END
  CAR-PARRINELLO             ... (see input description) END
  PSP_GENERATOR              ... (see input description) END
  WAVEFUNCTION_INITIALIZER   ... (see input description) END
  V_WAVEFUNCTION_INITIATIZER ... (see input description) END
  WAVEFUNCTION_EXPANDER      ... (see input description) END
  STEEPEST_DESCENT           ... (see input description) END

  MAPPING <integer mapping default 1>

  ROTATION (ON || OFF)
  TRANSLATION (ON || OFF)

END 
\end{verbatim}

The following list describes the keywords contained in the PSPW input block.
\begin{itemize}
        \item $<$cell\_name$>$ - name of
              the simulation\_cell named $<$cell\_name$>$.  See section \ref{sec:pspw_cell}.
        \item $<$input\_wavefunctions$>$ - name of the
              file containing one-electron orbitals
        \item $<$output\_wavefunctions$>$ - name of the
              file that will contain the one-electron orbitals at the
              end of the run. 
        \item $<$fake\_mass$>$ - value for the electronic
              fake mass ($\mu$). This parameter is not presently used in a 
              conjugate gradient simulation
        \item $<$time\_step$>$ - value for the time step ($\Delta t$).  This
              parameter is not presently used in a conjugate gradient simulation.
        \item $<$inner\_iteration$>$ - number of iterations between the 
              printing out of energies and tolerances
        \item $<$outer\_iteration$>$ - number of outer iterations
        \item $<$tole$>$ - value for the energy tolerance.
        \item $<$tolc$>$ - value for the one-electron orbital tolerance.
        \item $<$cutoff$>$ - value for the cutoff energy used to define the wavefunction.  In addition
                             using the CUTOFF keyword automatically sets the cutoff energy for the density
                             to be twice the wavefunction cutoff.
        \item $<$ecut$>$ - value for the cutoff energy used
                           to define the density. Default is set
                           to be the maximum value that will fit
                           within the simulation\_cell $<$cell\_name$>$.
        \item $<$wcut$>$ - value for the cutoff energy used
                           to define the one-electron orbitals.
                           Default is set to be the maximum value that 
                           will fit within the simulation\_cell $<$cell\_name$>$.
        \item $<$ncut$>$ - value for the number of unit cells
                          to sum over (in each direction) for the real space
                          part of the Ewald summation. Note Ewald summation
                          is only used if the simulation\_cell is periodic.
        \item $<$rcut$>$ - value for the cutoff radius used
                          in the Ewald summation. Note Ewald summation
                          is only used if the simulation\_cell is periodic. \\
                           Default set to be
                          $\frac{MIN(\left| \vec{a_i} \right|)}{\pi}, i=1,2,3$.
        \item (Vosko $||$ PBE96 $||$ revPBE $||$ ...) - Choose between Vosko et al's LDA 
                               parameterization or the orginal and revised Perdew, Burke, 
                               and Ernzerhof GGA functional.  In addition, several hybrid options.
        \item MULT - optional keyword which if specified allows the user to define the spin multiplicity
                     of the system
        \item MULLIKEN - optional keyword which if specified
                         causes a Mulliken analysis to be performed at
                         the end of the simulation.  
        \item ALLOW\_TRANSLATION - By default the the center of mass forces are projected out of the 
                                  computed forces. This optional keyword if specified allows the 
                                  center of mass forces to not be zero.

        \item SIMULATION\_CELL (see section \ref{sec:pspw_cell})
        \item DPLOT (see section \ref{sec:pspw_dplot})
        \item WANNIER (see section \ref{sec:pspw_wannier})
        \item CAR-PARRINELLO(see section \ref{sec:pspw_CP})
        \item PSP\_GENERATOR (see section \ref{sec:pspw_psp_generator})
        \item WAVEFUNCTION\_INITIALIZER (see section \ref{sec:pspw_wavefunction_initializer})
        \item V\_WAVEFUNCTION\_INITIALIZER (see section \ref{sec:pspw_v_wavefunction_initializer})
        \item WAVEFUNCTION\_EXPANDER (see section \ref{sec:pspw_wavefunction_expander}).
        \item STEEPEST\_DESCENT (see section \ref{sec:pspw_steepest_descent})

        \item $<$mapping$>$ - for a value of 1 slab FFT is used, for a value of 2 a 2d-hilbert FFT is used.
\end{itemize}

A prototype limited memory BFGS (LMBFGS) minimizer can be used to minimize the energy.  To
use this new optimizer the following SET directive needs to be specified:
\begin{verbatim}
set nwpw:mimimizer 1  # Default - Grassman conjugate gradient minimizer is used to minimize the energy.
set nwpw:mimimizer 2  # Grassman LMBFGS minimimzer is used to minimize the energy.
set nwpw:minimizer 4  # Stiefel conjugate gradient minimizer is used to minimize the energy. 
set nwpw:minimizer 5  # Band-by-band minimizer is used to minimize the energy.
\end{verbatim}
Limited testing suggests that the Grassman LMBFGS minimizer is about twice as fast as
the conjugate gradient minimizer.  However, there are several known cases
where this optimizer fails, so it is currently not a default option, and
should be used with caution.

In addition the following SET directives can be specified:
\begin{verbatim}
set nwpw:lcao_skip .false. # Default - initial wavefunctions generated using an LCAO guess. 
set nwpw:lcao_skip .true.  # Initial wavefunctions generated using a random plane-wave guess.

set nwpw:lcao_print .false. # Default - Ouput not produced during the generation of the LCAO guess. 
set nwpw:lcao_print .true.  # Output produced during the generation of the LCAO guess.

set nwpw:lcao_iterations 2  #specifies the number of LCAO iterations 
\end{verbatim}


\subsection{Simulation Cell}
\label{sec:pspw_cell}
The  simulation cell parameters
are entered by  defining a simulation\_cell sub-block within the PSPW 
block.  Listed below is the format of a simulation\_cell sub-block.
\begin{verbatim}
PSPW
...
   SIMULATION_CELL
      CELL_NAME <string name default 'cell_default'>
      BOUNDARY_CONDITIONS (periodic || aperiodic default periodic)
      LATTICE_VECTORS
        <real a1.x a1.y a1.z default 20.0  0.0  0.0>
        <real a2.x a2.y a2.z default  0.0 20.0  0.0>
        <real a3.x a3.y a3.z default  0.0  0.0 20.0>
      NGRID <integer na1 na2 na3 default 32 32 32>
   END
...
END
\end{verbatim}
Basically, the user needs to enter the dimensions, gridding and boundary
conditions of the simulation cell.  The following list describes the 
input in detail.
\begin{itemize}
        \item $<$name$>$ - user-supplied name for the simulation block.
        \item periodic - keyword specifying that the simulation cell 
                         has periodic boundary conditions.      
        \item aperiodic - keyword specifying that the simulation cell
                          has free-space boundary conditions. 
        \item $<$a1.x a1.y a1.z$>$ - user-supplied values for the first 
                                   lattice vector 
        \item $<$a2.x a2.y a2.z$>$ - user-supplied values for the second 
                                   lattice vector
        \item $<$a3.x a3.y a3.z$>$ - user-supplied values for the third 
                                   lattice vector
        \item $<$na1 na2 na3$>$ - user-supplied values for discretization 
                                along lattice vector directions.
\end{itemize}

Alternatively, instead of explicitly entering lattice vectors, users can
enter the unit cell using the standard cell parameters, a, b, c, $\alpha$, 
$\beta$, and $\gamma$, by using the LATTICE block.  The format for input
is as follows:
\begin{verbatim}
PSPW
...
   SIMULATION_CELL
      ...
      LATTICE
        [lat_a <real a default 20.0>]
        [lat_b <real b default 20.0>]
        [lat_c <real c default 20.0>]
        [alpha <real alpha default 90.0>]
        [beta  <real beta  default 90.0>]
        [gamma <real gamma default 90.0>]
      END
      ...
   END
...
END
\end{verbatim}


The user can also enter the lattice vectors of standard unit cells using the
keywords SC, FCC, BCC, for simple cubic, face-centered cubic, and body-centered cubic
respectively.  Listed below is an example of the format of this type of input.
\begin{verbatim}
PSPW
...
   SIMULATION_CELL
      SC 20.0
      ....
   END
...
END
\end{verbatim}

Finally, the lattice vectors from the unit cell can also be defined using
the fractional coordinate input in the GEOMETRY input (see section \ref{sec:latticeparam}).   
Listed below is an example of the format of this type of input for an 8 atom silicon carbide unit cell.
\begin{verbatim}
geometry units au 
  system crystal
    lat_a 8.277d0
    lat_b 8.277d0
    lat_c 8.277d0
    alpha 90.0d0
    beta  90.0d0
    gamma 90.0d0
  end
Si    -0.50000d0  -0.50000d0  -0.50000d0
Si     0.00000d0   0.00000d0  -0.50000d0
Si     0.00000d0  -0.50000d0   0.00000d0
Si    -0.50000d0   0.00000d0   0.00000d0
C     -0.25000d0  -0.25000d0  -0.25000d0
C      0.25000d0   0.25000d0  -0.25000d0
C      0.25000d0  -0.25000d0   0.25000d0
C     -0.25000d0   0.25000d0   0.25000d0
end
\end{verbatim}



\subsection{\tt Unit Cell Optimization}
\label{sec:pspw_cell_optimization}

The PSPW module using the DRIVER geometry optimizer can optimize a crystal unit cell.
Currently this type of optimization works only if the geometry is specified in fractional
coordinates.  The following SET directive is used to tell the DRIVER geometry optimizer to
optimize the crystal unit cell in addition to the geometry.
\begin{verbatim}
set includestress .true.
\end{verbatim}

\normalsize
\subsection{\tt DPLOT}
\label{sec:pspw_dplot}
The pspw dplot task is used to generate plots of various types of electron
densities (or orbitals) of a molecule.  The electron density is calculated on the 
specified set of grid points from a PSPW calculation.  The output file
generated is in the Gaussian Cube format.
Input to the DPLOT task is contained
within the DPLOT sub-block.
\begin{verbatim}
PSPW
  ...
  DPLOT
     ...
  END
  ...
END
\end{verbatim}
To run a DPLOT calculation the following directive 
is used:
\begin{verbatim}
TASK PSPW PSPW_DPLOT
\end{verbatim}
Listed below is the format of a DPLOT sub-block.
\begin{verbatim}
PSPW
... 
   DPLOT
     VECTORS <string input_wavefunctions  default input_movecs>
     DENSITY [total||difference||alpha||beta||laplacian||potential default total] <string density_name no default>
     ELF [restricted|alpha|beta] <string elf_name no default>
     ORBITAL <integer orbital_number no default> <string orbital_name no default>


     [LIMITXYZ [units <string Units default angstroms>]
     <real X_From> <real X_To> <integer No_Of_Spacings_X>
     <real Y_From> <real Y_To> <integer No_Of_Spacings_Y>
     <real Z_From> <real Z_To> <integer No_Of_Spacings_Z>]

   END

...
END
\end{verbatim}

The following list describes the input for the DPLOT
sub-block.

\begin{verbatim}
     VECTORS <string input_wavefunctions  default input_movecs>  
\end{verbatim}
 This sub-directive specifies the name of the molecular orbital file. If the second file is optionally given the density is computed as the difference between the corresponding electron densities. The vector files have to match. 

\begin{verbatim}
     DENSITY [total||difference||alpha||beta||laplacian||potential default total] <string density_name no default>
\end{verbatim}
This sub-directive specifies, what kind of density is to be plotted. The known names for total, difference, alpha, beta, laplacian, and potential. 

\begin{verbatim}
     ELF [restricted|alpha|beta] <string elf_name no default>
\end{verbatim}
This sub-directive specifies that an electron localization function (ELF) is to be plotted.

\begin{verbatim}
     ORBITAL <integer orbital_number no default> <string orbital_name no default>
\end{verbatim}
This sub-directive specifies the molecular orbital number that is to be plotted.

\begin{verbatim}
     LIMITXYZ [units <string Units default angstroms>]
     <real X_From> <real X_To> <integer No_Of_Spacings_X>
     <real Y_From> <real Y_To> <integer No_Of_Spacings_Y>
     <real Z_From> <real Z_To> <integer No_Of_Spacings_Z>
\end{verbatim}
By default the grid spacing and the limits of the cell to be plotted are defined by the input wavefunctions.  Alternatively the user can use the LIMITXYZ sub-directive to specify other limits.   The grid is generated using No\_Of\_Spacings + 1 points along each direction. The known names for Units are angstroms, au and bohr. 



\subsection{\tt Wannier}
\label{sec:pspw_wannier}
The pspw wannier task is generate maximally localized (Wannier) molecular orbitals.  The
algorithm proposed by Silvestrelli et al is use to generate the Wannier orbitals.  The
current version of this code works only for cubic cells. 


Input to the Wannier task is contained within the Wannier sub-block.
\begin{verbatim}
PSPW
  ...
  Wannier
     ...
  END
  ...
END
\end{verbatim}
To run a Wannier calculation the following directive 
is used:
\begin{verbatim}
TASK PSPW Wannier
\end{verbatim}
Listed below is the format of a Wannier sub-block.
\begin{verbatim}
PSPW
... 
   Wannier
     OLD_WAVEFUNCTION_FILENAME <string input_wavefunctions  default input_movecs>  
     NEW_WAVEFUNCTION_FILENAME <string output_wavefunctions default input_movecs>
   END
...
END
\end{verbatim}
The following list describes the input for the Wannier
sub-block.
\begin{itemize}
        \item $<$input\_wavefunctions$>$ - name of pspw wavefunction file.
        \item $<$output\_wavefunctions$>$ - name of pspw wavefunction file that
              will contain the Wannier orbitals. 
\end{itemize}



\subsection{\tt Self-Interaction Corrections}
\label{sec:pspw_SIC}

The SET directive is used to specify the molecular orbitals 
contribute to the self-interaction-correction (SIC) term.
\begin{verbatim}
set pspw:SIC_orbitals  <integer list_of_molecular_orbital_numbers>
\end{verbatim}
This defines only the molecular orbitals in the list as SIC active.  All
other molecular orbitals will not contribute to the SIC term.

For example the following directive specifies that the molecular orbitals numbered
1,5,6,7,8, and 15 are SIC active.
\begin{verbatim}
set pspw:SIC_orbitals  1 5:8 15
\end{verbatim}
or equivalently
\begin{verbatim}
set pspw:SIC_orbitals  1 5 6 7 8 15
\end{verbatim}

The following directive turns on self-consistent SIC.
\begin{verbatim}
set pspw:SIC_relax      .false.  # Default - Perturbative SIC calculation
set pspw:SIC_relax      .true.   # Self-consistent SIC calculation
\end{verbatim}

Two types of solvers can be used and they are specified using the following
SET directive
\begin{verbatim}
set pspw:SIC_solver_type 1  # Default - cutoff coulomb kernel
set pspw:SIC_solver_type 2  # Free-space boundary condition kernel
\end{verbatim}
The parameters for the cutoff coulomb kernel are defined by the following
SET directives:
\begin{verbatim}
set pspw:SIC_screening_radius <real rcut>
set pspw:SIC_screening_power  <real rpower>
\end{verbatim}



\subsection{\tt Point Charge Analysis}
\label{sec:pspw_point_charge_analysis}

The MULLIKEN option can be used to generate derived atomic point charges
from a plane-wave density.  This analysis is based on a strategy suggested in the work of
P.E. Blochl, J. Chem. Phys. vol. 103, page 7422 (1995).  In this strategy
the low-frequency components a plane-wave density are fit to a linear
combination of atom centered Gaussian functions.

The following SET directives are used to define the fitting.
\begin{verbatim}
set pspw_APC:Gc <real Gc_cutoff>  # specifies the maximum frequency component of the density to be used in the fitting in units of au.

set pspw_APC:nga <integer number_gauss> # specifies the the number of Gaussian functions per
atom.

set pspw_APC:gamma <real gamma_list> # specifies the decay lengths of each atom centered Gaussian. 
\end{verbatim}

We suggest using the following parameters.
\begin{verbatim}
set pspw_APC:Gc 2.5
set pspw_APC:nga 3
set pspw_APC:gamma 0.6 0.9 1.35 
\end{verbatim}


\subsection{\tt Car-Parrinello}
\label{sec:pspw_CP}
The Car-Parrinello task is used to perform ab initio molecular dynamics
using the scheme developed by Car and Parrinello.  In this unified ab
initio molecular dynamics scheme the motion of the ion cores is coupled to
a fictitious motion for the Kohn-Sham orbitals of density functional
theory.  Constant energy or constant temperature simulations can be
performed.  A detailed description of this method
is described in section \ref{sec:pspw_Car-Parrinello}.

Input to the Car-Parrinello simulation is contained within the
Car-Parrinello sub-block.
\begin{verbatim}
PSPW
  ...
  Car-Parrinello
     ...
  END
  ...
END
\end{verbatim}
To run a Car-Parrinello calculation the following directive is used:
\begin{verbatim}
TASK PSPW Car-Parrinello 
\end{verbatim}
The Car-Parrinello sub-block contains a great deal
of input, including pointers to data, as well as
parameter input.  Listed below is the format of a Car-Parrinello sub-block.
\begin{verbatim}
PSPW
...
   Car-Parrinello
      CELL_NAME <string cell_name default 'cell_default'>
      INPUT_WAVEFUNCTION_FILENAME    <string input_wavefunctions    default input_movecs>
      OUTPUT_WAVEFUNCTION_FILENAME   <string output_wavefunctions   default input_movecs>
      INPUT_V_WAVEFUNCTION_FILENAME  <string input_v_wavefunctions  default input_vmovecs>
      OUTPUT_V_WAVEFUNCTION_FILENAME <string output_v_wavefunctions default input_vmovecs>
      FAKE_MASS <real fake_mass default default 1000.0>
      TIME_STEP <real time_step default 5.0>
      LOOP <integer inner_iteration outer_iteration default 10 1>
      SCALING <real scale_c scale_r default 1.0 1.0>
      ENERGY_CUTOFF       <real ecut default (see input description)>
      WAVEFUNCTION_CUTOFF <real wcut default (see input description)>
      EWALD_NCUT <integer ncut default 1>
      EWALD_RCUT <real rcut    default (see input description)>
      XC (Vosko || LDA || PBE96 || revPBE || HF || PBE0 || revPBE0 || 
          LDA-SIC    || LDA-SIC/2    ||  LDA-0.4SIC    || LDA-SIC/4    || LDA-0.2SIC ||
          PBE96-SIC  || PBE96-SIC/2  ||  PBE96-0.4SIC  || PBE96-SIC/4  || PBE96-0.2SIC ||
          revPBE-SIC || revPBE-SIC/2 ||  revPBE-0.4SIC || revPBE-SIC/4 || revPBE-0.2SIC ||
          default Vosko)
      [Nose-Hoover <real Period_electron Temperature_electrion Period_ion Temperature_ion 
                          default 100.0 298.15 100.0 298.15>]
      [SA_decay <real sa_scale_c sa_scale_r default 1.0 1.0>]
      XYZ_FILENAME <string xyz_filename default XYZ>
      EMOTION_FILENAME <string emotion_filename default EMOTION>
      HMOTION_FILENAME <string hmotion_filename default HMOTION>
      OMOTION_FILENAME <string omotion_filename default OMOTION>
      EIGMOTION_FILENAME <string eigmotion_filename default EIGMOTION>
      ION_MOTION_FILENAME <string ion_motion_filename default MOTION>

   END
...

END
\end{verbatim}
The following list describes the input for the Car-Parrinello
sub-block.
\begin{itemize}
        \item $<$cell\_name$>$ - name of the
              the simulation\_cell named $<$cell\_name$>$.  See section \ref{sec:pspw_cell}.
        \item $<$input\_wavefunctions$>$ - name of the 
              file containing one-electron orbitals
        \item $<$output\_wavefunctions$>$ - name of the
              file that will contain the one-electron orbitals at the
              end of the run. 
        \item $<$input\_v\_wavefunctions$>$ - name of the file
              containing one-electron orbital velocities.
        \item $<$output\_v\_wavefunctions$>$ - name of the
              file that will contain the one-electron orbital velocities
              at the end of the run. 
        \item $<$fake\_mass$>$ - value for the electronic
              fake mass ($\mu$).
        \item $<$time\_step$>$ - value for the Verlet integration 
               time step ($\Delta t$).
        \item $<$inner\_iteration$>$ - number of iterations between the
              printing out of energies.
        \item $<$outer\_iteration$>$ - number of outer iterations
        \item $<$scale\_c$>$ - value for the initial velocity
                              scaling of the one-electron orbital velocities.
        \item $<$scale\_r$>$ - value for the initial velocity
                              scaling of the ion velocities.
        \item $<$ecut$>$ - value for the cutoff energy used
                           to define the density.  Default is set
                           to be the maximum value that will fit
                           within the simulation\_cell $<$cell\_name$>$.
        \item $<$wcut$>$ - value for the cutoff energy used
                           to define the one-electron orbitals.  Default is set
                           to be the maximum value that will fit
                           within the simulation\_cell $<$cell\_name$>$.
        \item $<$ncut$>$ - value for the number of unit cells
                          to sum over (in each direction) for the real space
                          part of the Ewald summation. Note Ewald summation
                          is only used if the simulation\_cell is periodic.
        \item $<$rcut$>$ - value for the cutoff radius used
                          in the Ewald summation.  Note Ewald summation
                          is only used if the simulation\_cell is periodic. \\
                          Default set to be
                          $\frac{MIN(\left| \vec{a_i} \right|)}{\pi}, i=1,2,3$.
        \item (Vosko $||$ PBE96 $||$ revPBE $||$ ...) - Choose between Vosko et al's LDA 
                               parameterization or the orginal and revised Perdew, Burke, 
                               and Ernzerhof GGA functional.  In addition, several hybrid options.
        \item Nose-Hoover - optional subblock which if specified
                         causes the simulation to perform Nose-Hoover dynamics.
                         If this subblock is not specified the 
                         simulation performs constant energy dynamics.
                         See section \ref{sec:pspw_nose} for a description of the parameters.
                         \begin{itemize}
                             \item $<$Period\_electron$>$ $\equiv$ $P_{electron}$ 
                                    - estimated period for fictitious electron thermostat.
                             \item $<$Temperature\_electron$>$ $\equiv$ $T_{electron}$ 
                                    - temperature for fictitious electron motion
                             \item $<$Period\_ion$>$ $\equiv$ $P_{ion}$ 
                                    - estimated period for ionic thermostat
                             \item $<$Temperature\_ion$>$ $\equiv$ $T_{ion}$ 
                                    - temperature for ion motion
                         \end{itemize}
        \item SA\_decay - optional subblock which if specified
                         causes the simulation to run a simulated annealing simulation.
                         For simulated annealing to work the Nose-Hoover subblock needs 
                         to be specified.  The initial temperature are taken from the
                         Nose-Hoover subblock.
                         See section \ref{sec:pspw_nose} for a description of the parameters.
                         \begin{itemize}
                             \item $<$sa\_scale\_c$>$ $\equiv$ $\tau_{electron}$ 
                                    - decay rate in atomic units for electronic temperature.
                             \item $<$sa\_scale\_r$>$ $\equiv$ $\tau_{ionic}$ 
                                    - decay rate in atomic units for the ionic temperature.
                         \end{itemize}

        \item $<$xyz\_filename$>$ - name of the XYZ motion file
                                generated
        \item $<$emotion\_filename$>$ - name of the emotion motion file.
                                See section \ref{sec:pspw_cp_data} for a 
                                description of the datafile.
        \item $<$hmotion\_filename$>$ - name of the hmotion motion file.
                                See section \ref{sec:pspw_cp_data} for a 
                                description of the datafile.
        \item $<$eigmotion\_filename$>$ - name of the eigmotion motion file.
                                See section \ref{sec:pspw_cp_data} for a 
                                description of the datafile.
       \item $<$ion\_motion\_filename$>$ - name of the ion\_motion motion file.
                                See section \ref{sec:pspw_cp_data} for a 
                                description of the datafile.
       \item MULLIKEN - optional keyword which if specified causes an omotion motion file to be created. 
        \item $<$omotion\_filename$>$ - name of the omotion motion file.
                                See section \ref{sec:pspw_cp_data} for a 
                                description of the datafile.
\end{itemize}

When a DPLOT sub-block is specified the following SET directive can be used 
to output dplot data during a Car-Parrinello simulation:
\begin{verbatim}
set pspw_dplot:iteration_list <integer list_of_iteration_numbers>
\end{verbatim}
The Gaussian cube files specified in the DPLOT sub-block are appended
with the specified iteration number. 

For example, the following directive specifies that at the 
3,10,11,12,13,14,15, and 50 iterations Gaussian cube files are to be produced.

\begin{verbatim}
set pspw_dplot:iteration_list 3,10:15,50
\end{verbatim}

\subsection{\tt Adding Geometry Constraints To A Car-Parrinello Simulation}
\label{sec:pspw_CP_constraint}
The Car-Parrinello module allows users to freeze the cartesian coordinates 
in a simulation (Note - the Car-Parrinello code recognizes Cartesian
constraints, but it does not recognize internal coordinate constraints).  
The \verb+SET+ directive (Section \ref{sec:activeatoms}) is used to freeze
atoms, by specifying a directive of the form:
\begin{verbatim}
  set geometry:actlist <integer list_of_center_numbers>
\end{verbatim}
This defines only the centers in the list as active.  All other
centers will have zero force assigned to them, and will remain frozen
at their starting coordinates during a Car-Parrinello simulation.

For example, the following directive specifies that atoms numbered 1,
5, 6, 7, 8, and 15 are active and all other atoms are frozen:
\begin{verbatim}
  set geometry:actlist 1 5:8 15
\end{verbatim}
or equivalently,
\begin{verbatim}
  set geometry:actlist 1 5 6 7 8 15
\end{verbatim}

If this option is not specified by entering a \verb+SET+ directive,
the default behavior in the code is to treat all atoms as active.  To
revert to this default behavior after the option to define frozen
atoms has been invoked, the \verb+UNSET+ directive must be used (since
the database is persistent, see Section \ref{sec:persist}).  The form
of the \verb+UNSET+ directive is as follows:
\begin{verbatim}
  unset geometry:actlist
\end{verbatim}


In addition, the Car-Parrinello module allows users to freeze bond
lengths via a Shake algorithm.  The following \verb+SET+ directive 
shows how to do this.
\begin{verbatim}
  set nwpw:shake_constraint "2 6 L 6.9334"
\end{verbatim}
This input fixes the bond length between atoms 2 and 6 to be
6.9334 bohrs.  Note that this input only recognizes bohrs.  

When using constraints it is usually necessary to turn off
center of mass shifting. This can be done by the following \verb+SET+ directive.
\begin{verbatim}
  set nwpw:com_shift .false.
\end{verbatim}

\subsection{\tt QM/MM}
\label{sec:pspw_qmmm}
A preliminary QM/MM capability that can run Car-Parrinello molecular dynamics  has been integrated 
into the PSPW module.  Currently, the input is  not very robust but it is straightforward.  The first 
step to run a QM/MM simulations is to define the MM atoms in the geometry block.  The MM atoms must be
at the end of the geometry and a carat, " \^\ ", must be appended to the end of the atom name, e.g.
\begin{verbatim}
geometry units angstrom nocenter noautosym noautoz print xyz
C       -0.000283    0.000106    0.000047
Cl      -0.868403    1.549888    0.254229
Cl       0.834043   -0.474413    1.517103
Cl      -1.175480   -1.275747   -0.460606
Cl       1.209940    0.200235   -1.310743

O^       0.3226E+01 -0.4419E+01 -0.5952E+01 
H^       0.3193E+01 -0.4836E+01 -0.5043E+01 
H^       0.4167E+01 -0.4428E+01 -0.6289E+01
O^       0.5318E+01 -0.3334E+01 -0.1220E+01
H^       0.4978E+01 -0.3040E+01 -0.2113E+01
H^       0.5654E+01 -0.2540E+01 -0.7127E+00
end
\end{verbatim}
Next the pseudopotentials have be defined for the every type of MM atom contained in the geometry blocks.  The
following local pseudopotential suggested by Laio, VandeVondele and Rothlisberger can be automatically generated.
\begin{eqnarray}
V(\vec{r}) = -Z_{ion}\frac{{r_c}^{n_{\sigma}} - r^{n_{\sigma}}}{-sign(Z_{ion})*{r_c}^{n_{\sigma}+1}-r^{n_{\sigma}+1}}
\end{eqnarray}
The following input To define this pseudopo the O\^\ MM atom using the following input
\begin{verbatim}
NWPW
   QMMM
     mm_psp O^ -0.8476 4 0.70
   END
END
\end{verbatim}
defines the local pseudopotential for the O\^\  MM atom , where $Z_{ion}=-0.8476$, $n_{\sigma}=4$, and $r_c=0.7$.  
The following input can be used to define the local pseudopotentials for all the MM atoms in the geometry
block defined above
\begin{verbatim}
NWPW
   QMMM
     mm_psp O^ -0.8476 4 0.70
     mm_psp H^  0.4238 4 0.40
   END
END
\end{verbatim}
Next the Lenard-Jones potentials for the QM and MM atoms need to be defined.  This is done as as follows
\begin{verbatim}
NWPW
   QMMM
      lj_ion_parameters C  3.41000000d0 0.10d0
      lj_ion_parameters Cl 3.45000000d0 0.16d0
      lj_ion_parameters O^ 3.16555789d0 0.15539425d0
   END
END
\end{verbatim}
Note that the Lenard-Jones potential is not defined for the MM H atoms in this example.  
The final step is to define the MM fragments in the simulation.  MM fragments are a set of 
atoms in which bonds and angle harmonic potentials are defined, or alternatively shake constraints are defined.  
The following input defines the fragments for the two water molecules in the above geometry,
\begin{verbatim}
NWPW
   QMMM
      fragment spc
         size 3                  #size of fragment
         index_start 6:9:3       #atom index list that defines the start of 
                                 # the fragments (start:final:stride)

         bond_spring 1 2    0.467307856 1.889726878  #bond i j    Kspring r0
         bond_spring 1 3    0.467307856 1.889726878  #bond i j    Kspring r0
         angle_spring 2 1 3 0.07293966  1.910611932  #angle i j k Kspring theta0
      end
   END
END
\end{verbatim}
The fragments can be defined using shake constraints as
\begin{verbatim}
NWPW
   QMMM
      fragment spc
         size 3                  #size of fragment
         index_start 6:9:3       #atom index list that defines the start of 
                                 # the fragments (start:final:stride)

         shake units angstroms 1 2 3 cyclic 1.0 1.632993125 1.0
      end
   END
END
\end{verbatim}
Alternatively, each water could be defined independently as follows.
\begin{verbatim}
NWPW
   QMMM
      fragment spc1
         size 3                  #size of fragment
         index_start 6           #atom index list that defines the start of 
                                 #the fragments 

         bond_spring 1 2    0.467307856 1.889726878  #bond i j    Kspring r0
         bond_spring 1 3    0.467307856 1.889726878  #bond i j    Kspring r0
         angle_spring 2 1 3 0.07293966  1.910611932  #angle i j k Kspring theta0
      end
      fragment spc2
         size 3                 #size of fragment
         index_start 9          #atom index list that defines the start of 
                                #the fragments
         shake units angstroms 1 2 3 cyclic 1.0 1.632993125 1.0
      end
   END
END
\end{verbatim}


\subsection{\tt PSP\_GENERATOR}
\label{sec:pspw_psp_generator}
A one-dimensional pseudopotential code has been integrated into NWChem.
This code allows the user to modify and develop pseudopotentials.  Currently, 
only the Hamann and Troullier-Martins norm-conserving pseudopotentials can be
generated.  In future releases, the pseudopotential library (section \ref{sec:psp_library})
will be more complete, so that the user will not have explicitly generate
pseudopotentials using this module.

Input to the PSP\_GENERATOR task is contained within the
PSP\_GENERATOR  sub-block.
\begin{verbatim}
PSPW
  ...
  PSP_GENERATOR
     ...
  END
  ...
END
\end{verbatim}
To run a PSP\_GENERATOR calculation the following directive 
is used:
\begin{verbatim}
TASK PSPW PSP_GENERATOR
\end{verbatim}
Listed below is the format of a PSP\_GENERATOR sub-block.
\begin{verbatim}
PSPW
... 
   PSP_GENERATOR
      PSEUDOPOTENTIAL_FILENAME: <string psp_name>
      ELEMENT: <string element>
      CHARGE: <real charge>
      MASS_NUMBER: <real mass_number>
      ATOMIC_FILLING: <integer ncore nvalence>
      ( (1||2||...) (s||p||d||f||...) <real filling> \
         ...)
      
      [CUTOFF: <integer lmax> 
         ( (s||p||d||f||g) <real rcut>\
         ...)
      ]
      PSEUDOPOTENTIAL_TYPE: (TROULLIER-MARTINS || HAMANN default HAMANN)
      SOLVER_TYPE: (PAULI || SCRHODINGER default PAULI)
      EXCHANGE_TYPE: (dirac || PBE96 default DIRAC)
      CORRELATION_TYPE: (VOSKO || PBE96 default VOSKO)
      [SEMICORE_RADIUS: <real rcore>]
      
   end
... 
END
\end{verbatim}
The following list describes the input for the PSP\_GENERATOR
sub-block.
\begin{itemize}

        \item $<$psp\_name$>$ - name that points to a.
        \item $<$element$>$ - Atomic symbol.
        \item $<$charge$>$ - charge of the atom
        \item $<$mass$>$ - mass number for the atom
        \item $<$ncore$>$ - number of core states
        \item $<$nvalence$>$ - number of valence states.
        \item ATOMIC\_FILLING:.....(see below)
        \item $<$filling$>$ - occupation of atomic state
        \item CUTOFF:....(see below) 
        \item $<$rcore$>$ - value for the semicore radius (see below) 
\end{itemize}


\subsubsection{\tt ATOMIC\_FILLING Block}
This required block is used to define the reference atom which is used
to define the pseudopotential. After the ATOMIC\_FILLING: $<$ncore$>$
$<$nvalence$>$ line, the core states are listed (one per line), and
then the valence states are listed (one per line). 
Each state contains two integer and a value.  The first integer
specifies the radial quantum number, $n$,
The second integer specifies the angular momentum quantum number, $l$,
and the third value specifies the occupation of the state.

For example to define a pseudopotential
for the Neon atom in the $1s^2 2s^2 2p^6$ state
could have the block
\begin{verbatim}
ATOMIC_FILLING: 1 2
        1  s  2.0   #core state    - 1s^2 
        2  s  2.0   #valence state - 2s^2
        2  p  6.0   #valence state - 2p^6
\end{verbatim}
for a pseudopotential with a $2s$ and $2p$ valence electrons
or the block
\begin{verbatim}
ATOMIC_FILLING: 3 0
        1  s  2.0    #core state
        2  s  2.0    #core state
        2  p  6.0    #core state
\end{verbatim}
could be used for a pseudopotential with no valence electrons.


\subsubsection{{\tt CUTOFF} Block}
This optional block specifies the cutoff distances used
to match the all-electron atom to the pseudopotential atom.  For
Hamann pseudopotentials $r_{cut}(l)$ defines the distance
where the all-electron potential is matched to the pseudopotential, and
for Troullier-Martins pseudopotentials $r_{cut}(l)$ defines the distance
where the all-electron orbital is matched to the pseudowavefunctions. 
Thus the definition of the radii depends on the type of pseudopotential.  
The cutoff radii used in Hamann pseudopotentials will be smaller than
the cutoff radii used in Troullier-Martins pseudopotentials.  

For example to define a softened Hamann pseudopotential for
Carbon would be
\begin{verbatim}
ATOMIC_FILLING: 1 2
  1  s  2.0
  2  s  2.0
  2  p  2.0
CUTOFF: 2
  s  0.8
  p  0.85
  d  0.85
\end{verbatim}
while a similarly softened Troullier-Marting pseudopotential
for Carbon would be
\begin{verbatim}
ATOMIC_FILLING: 1 2
  1  s  2.0
  2  s  2.0
  2  p  2.0
CUTOFF: 2
  s  1.200
  p  1.275
  d  1.275
\end{verbatim}


\subsubsection{{\tt SEMICORE\_RADIUS} Option}
Specifying the SEMICORE\_RADIUS option turns on the semicore correction approximation proposed
by Louie et al (S.G. Louie, S. Froyen, and M.L. Cohen, Phys. Rev. B, \textbf{26}, 1738, (1982)).  
This approximation is known to dramatically improve results for systems containing 
alkali and transition metal atoms.  

The implementation in the PSPW module defines the semi-core density, $\rho_{semicore}$ in terms of 
the core density, $\rho_{core}$, by using the sixth-order polynomial
\begin{eqnarray}
\rho_{semicore}(r) = \left\{ \begin{array}{ll}
                              \rho_{core} & \mbox{if $r \geq r_{semicore}$} \\
                              c_0 + c_3 r^3 + c_4 r^4 + c_5 r^5 + c_6 r^6 &  \mbox{if $r < r_{semicore}$}
                            \end{array}
                     \right.
\end{eqnarray}
This expansion was suggested by Fuchs and Scheffler 
(M. Fuchs, and M. Scheffler, Comp. Phys. Comm.,\textbf{119},67 (1999)), 
and is better behaved for taking derivatives (i.e. calculating ionic forces) than the expansion suggested 
by Louie et al.




\subsection{\tt WAVEFUNCTION\_INITIALIZER}
\label{sec:pspw_wavefunction_initializer}
The functionality of this task is now performed automatically. For backward 
compatibility, we provide a description of the input to this task.

The wavefunction\_initializer task is used to generate an initial wavefunction
datafile.
Input to the WAVEFUNCTION\_INITIALIZER task is contained
within the WAVEFUNCTION\_INITIALIZER sub-block.
\begin{verbatim}
PSPW
  ...
  WAVEFUNCTION_INITIALIZER
     ...
  END
  ...
END
\end{verbatim}
To run a WAVEFUNCTION\_INITIALIZER calculation the following directive 
is used:
\begin{verbatim}
TASK PSPW WAVEFUNCTION_INITIALIZER
\end{verbatim}
Listed below is the format of a WAVEFUNCTION\_INITIALIZER sub-block.
\begin{verbatim}
PSPW
... 
   WAVEFUNCTION_INITIALIZER
     CELL_NAME: <string cell_name>
     WAVEFUNCTION_FILENAME: <string wavefunction_name default input_movecs>
     (RESTRICTED||UNRESTRICTED)
     if (RESTRICTED)   
        RESTRICTED_ELECTRONS: <integer restricted electrons>
     if (UNRESTRICTED) 
        UP_ELECTRONS: <integer up_electrons>
        DOWN_ELECTRONS: <integer down_electrons>
   END
...
END
\end{verbatim}
The following list describes the input for the WAVEFUNCTION\_INITIALIZER
sub-block.
\begin{itemize}
        \item $<$cell\_name$>$ - name of
                the simulation\_cell named $<$cell\_name$>$.  See section \ref{sec:pspw_cell}.
        \item $<$wavefunction\_name$>$ - name that will point
              to a wavefunction file.
        \item RESTRICTED - keyword specifying that the calculation is restricted.
        \item UNRESTRICTED - keyword specifying that the calculation is unrestricted.

        \item $<$restricted\_electrons$>$ - number of restricted electrons.
               Not used if an UNRESTRICTED calculation. 
         \item $<$up\_electrons$>$ - number of spin-up electrons.
               Not used if a RESTRICTED calculation.
        \item $<$down\_electrons$>$ - number of spin-down electrons.
              Not used if a RESTRICTED calculation.
\end{itemize}

\subsubsection{Old Style Input (version 3.3) to {\tt WAVEFUNCTION\_INITIALIZER}}

For backward compatibility, the input to the WAVEFUNCTION\_INITIALIZER 
sub-block can also be of the form
\begin{verbatim}
PSPW
... 
   WAVEFUNCTION_INITIALIZER
     CELL_NAME: <string cell_name>
     WAVEFUNCTION_FILENAME: <string wavefunction_name default input_movecs>
     (RESTRICTED||UNRESTRICTED)
     
     [UP_FILLING: <integer up_filling>
        [0 0 0   0]
        {<integer kx ky kz> (-2||-1||1||2)}]
     [DOWN_FILLING: <integer down_filling>
        [0 0 0   0]
        {<integer kx ky kz> (-2||-1||1||2)}]
   END
...
END
\end{verbatim}
where
\begin{itemize}
        \item $<$cell\_name$>$ - name of the
                simulation\_cell named $<$cell\_name$>$.  See section \ref{sec:pspw_cell}.
        \item $<$wavefunction\_name$>$ - name that will point
              to a wavefunction file.
        \item RESTRICTED - keyword specifying that the calculation is restricted.
        \item UNRESTRICTED - keyword specifying that the calculation is unrestricted.
        \item $<$up\_filling$>$ - number of restricted molecular orbitals if
              RESTRICTED and number of spin-up molecular orbitals if 
              UNRESTRICTED.
        \item $<$down\_filling$>$ - number of spin-down molecular orbitals if
              UNRESTRICTED.  Not used if a RESTRICTED calculation.
        \item $<$kx ky kz$>$ - specifies which planewave is to be filled. 
\end{itemize}

The values for the planewave $(-2||-1||1||2)$ are used to represent whether
the specified planewave is a cosine or a sine function, in addition
random noise can be added to these base functions. That is $+1$ 
represents a cosine function, and $-1$ represents a sine function.
The $+2$ and $-2$ values are used to represent a cosine function with
random components added and a sine function with random components
added respectively.  


\subsection{\tt V\_WAVEFUNCTION\_INITIALIZER}
\label{sec:pspw_v_wavefunction_initializer}
The functionality of this task is now performed automatically. For backward 
compatibility, we provide a description of the input to this task.

The v\_wavefunction\_initializer task is used to generate an initial velocity 
wavefunction datafile.
Input to the V\_WAVEFUNCTION\_INITIALIZER task is contained
within the V\_WAVEFUNCTION\_INITIALIZER sub-block.
\begin{verbatim}
PSPW
  ...
  V_WAVEFUNCTION_INITIALIZER
     ...
  END
  ...
END
\end{verbatim}
To run a V\_WAVEFUNCTION\_INITIALIZER calculation the following directive 
is used:
\begin{verbatim}
TASK PSPW WAVEFUNCTION_INITIALIZER
\end{verbatim}
Listed below is the format of a V\_WAVEFUNCTION\_INITIALIZER sub-block.
\begin{verbatim}
PSPW
... 
   V_WAVEFUNCTION_INITIALIZER
     V_WAVEFUNCTION_FILENAME: <string v_wavefunction_name default input_vmovecs>
     CELL_NAME: <string cell_name>
     (RESTRICTED||UNRESTRICTED)
     UP_FILLING: <integer up_filling>
     DOWN_FILLING: <integer down_filling>
   END
...
END
\end{verbatim}
The following list describes the input for the V\_WAVEFUNCTION\_INITIALIZER
sub-block.
\begin{itemize}
        \item $<$cell\_name$>$ - name of the
                simulation\_cell named $<$cell\_name$>$.  See section \ref{sec:pspw_cell}.
        \item $<$wavefunction\_name$>$ - name that will point
              to a velocity wavefunction file.
        \item RESTRICTED - keyword specifying that the calculation is restricted.
        \item UNRESTRICTED - keyword specifying that the calculation is unrestricted.
        \item $<$up\_filling$>$ - number of restricted velocity molecular 
              orbitals if RESTRICTED and number of spin-up velocity molecular 
              orbitals if UNRESTRICTED.
        \item $<$down\_filling$>$ - number of spin-down velocity molecular 
              orbitals if UNRESTRICTED.  Not used if a RESTRICTED calculation.
\end{itemize}



\subsection{\tt WAVEFUNCTION\_EXPANDER}
\label{sec:pspw_wavefunction_expander}
The functionality of this task is now performed automatically. For backward
compatibility, we provide a description of the input to this task.

The wavefunction\_expander task is used to convert a new wavefunction
file that spans a larger grid space from an old wavefunction file.
Input to the WAVEFUNCTION\_EXPANDER task is contained
within the WAVEFUNCTION\_EXPANDER sub-block.
\begin{verbatim}
PSPW
  ...
  WAVEFUNCTION_EXPANDER
     ...
  END
  ...
END
\end{verbatim}
To run a WAVEFUNCTION\_EXPANDER calculation the following directive 
is used:
\begin{verbatim}
TASK PSPW WAVEFUNCTION_EXPANDER
\end{verbatim}
Listed below is the format of a WAVEFUNCTION\_EXPANDER sub-block.
\begin{verbatim}
PSPW
... 
   WAVEFUNCTION_EXPANDER   
     OLD_WAVEFUNCTION_FILENAME: <string old_wavefunction_name default input_movecs>
     NEW_WAVEFUNCTION_FILENAME: <string new_wavefunction_name default input_movecs>
     NEW_NGRID: <integer na1 na2 na3>
    
   END
...
END
\end{verbatim}
The following list describes the input for the WAVEFUNCTION\_EXPANDER
sub-block.
\begin{itemize}
        \item $<$old\_wavefunction\_name$>$ - name of the
              wavefunction file.
        \item $<$new\_wavefunction\_name$>$ - name that will 
              point to a wavefunction file.
        \item $<$na1 na2 na3$>$ - number of grid points in each dimension
              for the new wavefunction file. 
\end{itemize}

\subsection{\tt STEEPEST\_DESCENT} 
\label{sec:pspw_steepest_descent}
The functionality of this task is now performed automatically by the PSPW minimizer. 
For backward compatibility, we provide a description of the input to this task.

The steepest\_descent task is used to optimize the one-electron orbitals
with respect to the total energy.  In addition it can also be used to optimize
geometries.   This method is meant to be used for coarse optimization of
the one-electron orbitals.
% ref does not exist
%  A detailed description of the this method
%is described in section \ref{sec:pspw_sd2}

Input to the steepest\_descent simulation is contained
within the steepest\_descent sub-block.
\begin{verbatim}
PSPW
  ...
  STEEPEST_DESCENT
     ...
  END
  ...
END
\end{verbatim}
To run a steepest\_descent calculation the following directive is used:
\begin{verbatim}
TASK PSPW steepest_descent 
\end{verbatim}
The steepest\_descent sub-block contains a great deal
of input, including pointers to data, as well as
parameter input.  Listed below is the format of a STEEPEST\_DESCENT sub-block.
\begin{verbatim}
PSPW
...
   STEEPEST_DESCENT
      CELL_NAME <string cell_name>
      [GEOMETRY_OPTIMIZE]
      INPUT_WAVEFUNCTION_FILENAME  <string input_wavefunctions  default input_movecs>
      OUTPUT_WAVEFUNCTION_FILENAME <string output_wavefunctions default input_movecs>
      FAKE_MASS <real fake_mass default 400000.0>
      TIME_STEP <real time_step default 5.8>
      LOOP <integer inner_iteration outer_iteration default 10 1>
      TOLERANCES <real tole tolc tolr default 1.0d-9 1.0d-9 1.0d-4>
      ENERGY_CUTOFF       <real ecut default (see input desciption)>
      WAVEFUNCTION_CUTOFF <real wcut default (see input description)>
      EWALD_NCUT <integer ncut default 1>
      EWALD_RCUT <real rcut default (see input description)>
      XC (Vosko || LDA || PBE96 || revPBE || HF || PBE0 || revPBE0 || 
          LDA-SIC    || LDA-SIC/2    ||  LDA-0.4SIC    || LDA-SIC/4    || LDA-0.2SIC ||
          PBE96-SIC  || PBE96-SIC/2  ||  PBE96-0.4SIC  || PBE96-SIC/4  || PBE96-0.2SIC ||
          revPBE-SIC || revPBE-SIC/2 ||  revPBE-0.4SIC || revPBE-SIC/4 || revPBE-0.2SIC ||
          default Vosko)
      [MULLIKEN]

   END
...

END
\end{verbatim}
The following list describes the input for the STEEPEST\_DESCENT
sub-block.
\begin{itemize}
        \item $<$cell\_name$>$ - name of
              the simulation\_cell named $<$cell\_name$>$. See section \ref{sec:pspw_cell}.
        \item GEOMETRY\_OPTIMIZE - optional keyword which if specified
              turns on geometry optimization.   
        \item $<$input\_wavefunctions$>$ - name of the
              file containing one-electron orbitals
        \item $<$output\_wavefunctions$>$ - name of the
              file tha will contain the one-electron orbitals at the
              end of the run. 
        \item $<$fake\_mass$>$ - value for the electronic
              fake mass ($\mu$).
        \item $<$time\_step$>$ - value for the time step ($\Delta t$).
        \item $<$inner\_iteration$>$ - number of iterations between the 
              printing out of energies and tolerances
        \item $<$outer\_iteration$>$ - number of outer iterations
        \item $<$tole$>$ - value for the energy tolerance.
        \item $<$tolc$>$ - value for the one-electron orbital tolerance.
        \item $<$tolr$>$ - value for the ion position tolerance.
        \item $<$ecut$>$ - value for the cutoff energy used
                           to define the density.  Default is set
                           to be the maximum value that will fit
                           within the simulation\_cell $<$cell\_name$>$.
        \item $<$wcut$>$ - value for the cutoff energy used
                           to define the one-electron orbitals. Default is set
                           to be the maximum value that will fit
                           within the simulation\_cell $<$cell\_name$>$.
        \item $<$ncut$>$ - value for the number of unit cells
                          to sum over (in each direction) for the real space
                          part of the Ewald summation.  Note Ewald summation
                          is only used if the simulation\_cell is periodic.
        \item $<$rcut$>$ - value for the cutoff radius used
                          in the Ewald summation.  Note Ewald summation
                          is only used if the simulation\_cell is periodic. \\
                          Default set to be
                          $\frac{MIN(\left| \vec{a_i} \right|)}{\pi}, i=1,2,3$.
        \item (Vosko $||$ PBE96 $||$ revPBE $||$ ...) - Choose between Vosko et al's LDA 
                               parameterization or the orginal and revised Perdew, Burke, 
                               and Ernzerhof GGA functional.  In addition, several hybrid options.
        \item MULLIKEN - optional keyword which if specified
                         causes a Mulliken analysis to be performed at
                         the end of the simulation.  
\end{itemize}





\section{Band Tasks}
\label{sec:band_tasks}

All input to the Band Tasks is contained within the compound NWPW block,
\begin{verbatim}
NWPW
   ...
END
\end{verbatim}

To perform an actual calculation a TASK Band directive is used (Section \ref{sec:task}).  
\begin{verbatim}
  TASK Band
\end{verbatim} 

Once a user has specified a geometry, the Band module can be invoked with no input directives (defaults invoked throughout).  There are sub-directives which allow for customized application; those currently provided as options for the Band module are:
\begin{verbatim}
NWPW
  CELL_NAME <string cell_name default 'cell_default'>
  ZONE_NAME <string zone_name default 'zone_default'>
  INPUT_WAVEFUNCTION_FILENAME  <string input_wavefunctions  default input_movecs>
  OUTPUT_WAVEFUNCTION_FILENAME <string output_wavefunctions default input_movecs>
  FAKE_MASS <real fake_mass default 400000.0>
  TIME_STEP <real time_step default 5.8>
  LOOP <integer inner_iteration outer_iteration default 10 100>
  TOLERANCES <real tole tolc default 1.0e-7 1.0e-7>
  CUTOFF              <real cutoff>
  ENERGY_CUTOFF       <real ecut default (see input description)>
  WAVEFUNCTION_CUTOFF <real wcut default (see input description)>
  EWALD_NCUT <integer ncut default 1>]
  EWALD_RCUT <real rcut default (see input description)>
  XC (Vosko || PBE96  || revPBE default Vosko)
  DFT||ODFT||RESTRICTED||UNRESTRICTED
  MULT <integer mult default 1>
  
  SIMULATION_CELL ... (see input description) END
  BRILLOUIN_ZONE  ... (see input description) END
  MONKHORST-PACK <real n1 n2 n3 default 1 1 1>

  BAND_DPLOT    ... (see input description) END

  MAPPING <integer mapping default 1>
  SMEAR <sigma default 0.001> [TEMPERATURE <temperature>] [FERMI || GAUSSIAN default FERMI] 
                              [ORBITALS <integer orbitals default 4>]

END 
\end{verbatim}
The following list describes these keywords.
\begin{itemize}
        \item $<$cell\_name$>$ - name of
              the simulation\_cell named $<$cell\_name$>$.  See section \ref{sec:pspw_cell}.
        \item $<$input\_wavefunctions$>$ - name of the
              file containing one-electron orbitals
        \item $<$output\_wavefunctions$>$ - name that will
              point to file containing the one-electron orbitals at the
              end of the run. 
        \item $<$fake\_mass$>$ - value for the electronic
              fake mass ($\mu$). This parameter is not presently used in a 
              conjugate gradient simulation
        \item $<$time\_step$>$ - value for the time step ($\Delta t$).  This
              parameter is not presently used in a conjugate gradient simulation.
        \item $<$inner\_iteration$>$ - number of iterations between the 
              printing out of energies and tolerances
        \item $<$outer\_iteration$>$ - number of outer iterations
        \item $<$tole$>$ - value for the energy tolerance.
        \item $<$tolc$>$ - value for the one-electron orbital tolerance.
        \item $<$cutoff$>$ - value for the cutoff energy used to define the wavefunction.  In addition
                             using the CUTOFF keyword automatically sets the cutoff energy for the density
                             to be twice the wavefunction cutoff.
        \item $<$ecut$>$ - value for the cutoff energy used
                           to define the density. Default is set
                           to be the maximum value that will fit
                            within the simulation\_cell $<$cell\_name$>$.
        \item $<$wcut$>$ - value for the cutoff energy used
                           to define the one-electron orbitals.
                           Default is set to be the maximum value that 
                           will fix within the simulation\_cell $<$cell\_name$>$.
        \item $<$ncut$>$ - value for the number of unit cells
                          to sum over (in each direction) for the real space
                          part of the Ewald summation. Note Ewald summation
                          is only used if the simulation\_cell is periodic.
        \item $<$rcut$>$ - value for the cutoff radius used
                          in the Ewald summation. Note Ewald summation
                          is only used if the simulation\_cell is periodic. \\
                           Default set to be
                          $\frac{MIN(\left| \vec{a_i} \right|)}{\pi}, i=1,2,3$.
        \item (Vosko $||$ PBE96 $||$ revPBE) - Choose between Vosko et al's LDA 
                               parameterization or the orginal and revised Perdew, Burke, 
                               and Ernzerhof GGA functional.
        \item SIMULATION\_CELL (see section \ref{sec:pspw_cell})
        \item BRILLOUIN\_ZONE  (see section \ref{sec:band_brillouin_zone})
        \item MONKHORST-PACK - Alternatively, the MONKHORST-PACK keyword can be used 
                               to enter a MONKHORST-PACK sampling of the Brillouin zone.
        \item $<$smear$>$ - value for smearing broadending
        \item $<$temperature$>$ - same as smear but in units of K.

\end{itemize}


\subsection{Brillouin Zone}
\label{sec:band_brillouin_zone}
To supply the special points of the Brillouin zone,
the user defines a brillouin\_zone sub-block within the NWPW 
block.  Listed below is the format of a brillouin\_zone sub-block.
\begin{verbatim}
NWPW
...
   BRILLOUIN_ZONE
      ZONE_NAME <string name default 'zone_default'>
      (KVECTOR <real k1 k2 k3 no default> <real weight default (see input description)>
       ...)
   END
...
END
\end{verbatim}
The user enters the special points and weights of the
Brillouin zone.  The following list describes the input in detail.
\begin{itemize}
        \item $<$name$>$ - user-supplied name for the simulation block. 
        \item $<$k1 k2 k3$>$ - user-supplied values for a special point in the
                               Brillouin zone.
        \item $<$weight$>$ - user-supplied weight.  Default is to set the weight
                         so that the sum of all the weights for the entered  
                         special points adds up to unity.
\end{itemize}


\normalsize
\subsection{\tt BAND\_DPLOT}
\label{sec:pspw_dplot}
The BAND BAND\_DPLOT task is used to generate plots of various types of electron
densities (or orbitals) of a crystal.  The electron density is calculated on the
specified set of grid points from a Band calculation.  The output file
generated is in the Gaussian Cube format.
Input to the BAND\_DPLOT task is contained
within the BAND\_DPLOT sub-block.
\begin{verbatim}
NWPW
  ...
  BAND_DPLOT
     ...
  END
  ...
END
\end{verbatim}
To run a BAND\_DPLOT calculation the following directive
is used:
\begin{verbatim}
TASK BAND BAND_DPLOT
\end{verbatim}
Listed below is the format of a BAND\_DPLOT sub-block.
\begin{verbatim}
NWPW
...
   BAND_DPLOT
     VECTORS <string input_wavefunctions  default input_movecs>
     DENSITY [total||difference||alpha||beta||laplacian||potential default total] <string density_name no default>
     ELF [restricted|alpha|beta] <string elf_name no default>
     ORBITAL (density || real || complex default density) <integer orbital_number no default> <integer brillion_number default 1> <string orbital_name no default>


     [LIMITXYZ [units <string Units default angstroms>]
     <real X_From> <real X_To> <integer No_Of_Spacings_X>
     <real Y_From> <real Y_To> <integer No_Of_Spacings_Y>
     <real Z_From> <real Z_To> <integer No_Of_Spacings_Z>]

   END

...
END
\end{verbatim}

The following list describes the input for the BAND\_DPLOT
sub-block.

\begin{verbatim}
     VECTORS <string input_wavefunctions  default input_movecs>
\end{verbatim}
 This sub-directive specifies the name of the molecular orbital file. If the second file is optionally given the density is computed as the difference between the corresponding electron densities. The vector files have to match.

\begin{verbatim}
     DENSITY [total||difference||alpha||beta||laplacian||potential default total] <string density_name no default>
\end{verbatim}
This sub-directive specifies, what kind of density is to be plotted. The known names for total, difference, alpha, beta, laplacian, and potential.

\begin{verbatim}
     ELF [restricted|alpha|beta] <string elf_name no default>
\end{verbatim}
This sub-directive specifies that an electron localization function (ELF) is to be plotted.

\begin{verbatim}
     ORBITAL (density || real || complex default density) <integer orbital_number no default> \
             <integer brillion_number default 1> <string orbital_name no default>
\end{verbatim}
This sub-directive specifies the molecular orbital number that is to be plotted.

\begin{verbatim}
     LIMITXYZ [units <string Units default angstroms>]
     <real X_From> <real X_To> <integer No_Of_Spacings_X>
     <real Y_From> <real Y_To> <integer No_Of_Spacings_Y>
     <real Z_From> <real Z_To> <integer No_Of_Spacings_Z>
\end{verbatim}
By default the grid spacing and the limits of the cell to be plotted are defined by the input wavefunctions.  Alternatively the user can use the LIMITXYZ sub-directive to specify other limits.  The grid is generated using No\_Of\_Spacings + 1 points along each direction. The known names for Units are angstroms, au and bohr.

\subsection{SMEAR - Fractional Occupation of the Molecular Orbitals}
\label{sec:band_smear}
The smear keyword to turn on fractional occupation of the molecular orbitals
\begin{verbatim}
  SMEAR <sigma default 0.001> [TEMPERATURE <temperature>] [FERMI || GAUSSIAN default FERMI] 
                              [ORBITALS <integer orbitals default 4>]
\end{verbatim}
Both Fermi-Dirac (FERMI) and Gaussian broadening functions are available.  The ORBITALS keyword is used to change
the number of virtual orbitals to be used in the calculation.  Note to use this option the user must currently use the
SCF minimizer.  The following SCF option is recommended for running fractional occupation
\begin{verbatim}
  SCF Anderson
\end{verbatim}


\section{PAW Tasks}
\label{sec:paw_tasks}


All input to the PAW Tasks is contained within the compound NWPW block,
\begin{verbatim}
NWPW
   ...
END
\end{verbatim}


To perform an actual calculation a TASK PAW directive is used
(Section \ref{sec:task}).  
\begin{verbatim}
  TASK PAW
\end{verbatim} 
In addition to the directives listed in Section \ref{sec:task}, i.e.
\begin{verbatim}
TASK paw energy          
TASK paw gradient         
TASK paw optimize         
TASK paw saddle           
TASK paw freqencies       
TASK paw vib
\end{verbatim}
there are additional directives that are specific to the PSPW module, which are:
\begin{verbatim}
TASK PAW [Car-Parrinello             ||
          steepest_descent            ]
\end{verbatim}


Once a user has specified a geometry, the PAW module can be invoked with no input directives (defaults invoked throughout).  There are sub-directives which allow for customized application; those currently provided as options for the PAW module are:
\begin{verbatim}
NWPW
  CELL_NAME <string cell_name default 'cell_default'>  
  [GEOMETRY_OPTIMIZE]
  INPUT_WAVEFUNCTION_FILENAME  <string input_wavefunctions  default input_movecs>
  OUTPUT_WAVEFUNCTION_FILENAME <string output_wavefunctions default input_movecs>
  FAKE_MASS <real fake_mass default 400000.0>
  TIME_STEP <real time_step default 5.8>
  LOOP <integer inner_iteration outer_iteration default 10 100>
  TOLERANCES <real tole tolc default 1.0e-7 1.0e-7>
  CUTOFF              <real cutoff>
  ENERGY_CUTOFF       <real ecut default (see input description)>
  WAVEFUNCTION_CUTOFF <real wcut default (see input description)>
  EWALD_NCUT <integer ncut default 1>]
  EWALD_RCUT <real rcut default (see input description)>
  XC (Vosko || PBE96 || revPBE  default Vosko)
  DFT||ODFT||RESTRICTED||UNRESTRICTED
  MULT <integer mult default 1>
  INTEGRATE_MULT_L <integer imult default 1>
  
  SIMULATION_CELL ... (see input description) END
  CAR-PARRINELLO  ... (see input description) END

  MAPPING <integer mapping default 1>
END 


END 
\end{verbatim}
The following list describes these keywords.
\begin{itemize}
        \item $<$cell\_name$>$ - name of the
              the simulation\_cell named $<$cell\_name$>$. The
              current version of PAW only accepts periodic unit cells.  
              See section \ref{sec:pspw_cell}.
        \item GEOMETRY\_OPTIMIZE - optional keyword which if specified
              turns on geometry optimization.   
        \item $<$input\_wavefunctions$>$ - name of the
              file containing one-electron orbitals
        \item $<$output\_wavefunctions$>$ - name of the
              file that will contain the one-electron orbitals at the
              end of the run. 
        \item $<$fake\_mass$>$ - value for the electronic
              fake mass ($\mu$). This parameter is not presently used in a 
              conjugate gradient simulation
        \item $<$time\_step$>$ - value for the time step ($\Delta t$).  This
              parameter is not presently used in a conjugate gradient simulation.
        \item $<$inner\_iteration$>$ - number of iterations between the 
              printing out of energies and tolerances
        \item $<$outer\_iteration$>$ - number of outer iterations
        \item $<$tole$>$ - value for the energy tolerance.
        \item $<$tolc$>$ - value for the one-electron orbital tolerance.
        \item $<$cutoff$>$ - value for the cutoff energy used to define the wavefunction.  In addition
                             using the CUTOFF keyword automatically sets the cutoff energy for the density
                             to be twice the wavefunction cutoff.
        \item $<$ecut$>$ - value for the cutoff energy used
                           to define the density. Default is set
                           to be the maximum value that will fit
                            within the simulation\_cell $<$cell\_name$>$.
        \item $<$wcut$>$ - value for the cutoff energy used
                           to define the one-electron orbitals.
                           Default is set to be the maximum value that 
                           will fix within the simulation\_cell $<$cell\_name$>$.
        \item $<$ncut$>$ - value for the number of unit cells
                          to sum over (in each direction) for the real space
                          part of the smooth compensation summation. 
        \item $<$rcut$>$ - value for the cutoff radius used
                          in the smooth compensation summation. \\
                           Default set to be
                          $\frac{MIN(\left| \vec{a_i} \right|)}{\pi}, i=1,2,3$.
        \item (Vosko $||$ PBE96 $||$ revPBE) - Choose between Vosko et al's LDA 
                               parameterization or the orginal and revised Perdew, Burke, 
                               and Ernzerhof GGA functional.
        \item MULT - optional keyword which if specified allows the user to define the spin multiplicity
                     of the system
        \item INTEGRATE\_MULT\_L - optional keyword which if specified allows the user to define the 
                                   angular XC integration of the augmented region
        \item SIMULATION\_CELL (see section \ref{sec:pspw_cell})
        \item CAR-PARRINELLO(see section \ref{sec:pspw_CP})


        \item $<$mapping$>$ - for a value of 1 slab FFT is used, for a value of 2 a 2d-hilbert FFT is used.
\end{itemize}


\section{Pseudopotential and PAW basis Libraries}
\label{sec:psp_library}

A library of pseudopotentials used by PSPW and BAND is currently available  in the
directory \\
\verb+ $NWCHEM_TOP/src/nwpw/libraryps/pspw_default+

The elements listed in the following table are present:

\begin{verbatim}
  H                                                  He
 -------                              ------------------
  Li Be                               B  C  N  O  F  Ne
 -------                             -------------------
  Na Mg                               Al Si P  S  Cl Ar
 -------------------------------------------------------
  K  Ca Sc Ti V  Cr Mn Fe Co Ni Cu Zn Ga Ge As Se Br Kr         
 -------------------------------------------------------
  Rb Sr Y  Zr Nb Mo Tc Ru Rh Pd Ag Cd In Sn Sb Te I  Xe
 -------------------------------------------------------
  Cs Ba La Hf Ta W  Re Os Ir Pt Au Hg Tl Pb Bi Po At Rn
 -------------------------------------------------------
  Fr Ra . 
 -----------------

          ------------------------------------------
           .  .  .  .  .  .  Gd .  .  .  .  .  .  .                      
          ------------------------------------------
           .  .  U  .  Pu .  .  .  .  .  .  .  .  .      
          ------------------------------------------

\end{verbatim}
The pseudopotential libraries are continually being tested
and added.   Also,  the PSPW program can read in pseudopotentials
in CPI and TETER format generated with pseudopotential generation
programs such as the OPIUM package of Rappe et al.
The user can request additional pseudopotentials from 
Eric J. Bylaska at (Eric.Bylaska@pnl.gov).  

Similarly, a library of PAW basis used by PAW is currently available in the
directory \\
\verb+ $NWCHEM_TOP/src/nwpw/libraryps/paw_default+

\begin{verbatim}
  H                                                  He
 -------                              -----------------
  Li Be                               B  C  N  O  F  Ne
 -------                             ------------------
  Na Mg                               Al Si P  S  Cl Ar
 ------------------------------------------------------
  K  Ca Sc Ti V  Cr Mn Fe Co Ni Cu Zn Ga Ge As Se Br Kr         
 ------------------------------------------------------
  .  .  .  .  .  .  .  .  .  .  .  .  .  .  .  .  .  .
 ------------------------------------------------------
  .  .  .  .  .  .  .  .  .  .  .  .  .  .  .  .  .  .
 ------------------------------------------------------
  .  .  . 
 -----------------
                                                      

          ------------------------------------------
           .  .  .  .  .  .  .  .  .  .  .  .  .  .                      
          ------------------------------------------
           .  .  .  .  .  .  .  .  .  .  .  .  .  .      
          ------------------------------------------

\end{verbatim}


Currently there are not very many elements available for PAW.  However,
the user can request additional basis sets from Eric J. Bylaska at (Eric.Bylaska@pnl.gov).

A preliminary implementation of the HGH pseudopotentials (Hartwigsen, Goedecker, and Hutter)
has been implemented into the PSPW module.  To access
the pseudopotentials the pseudopotentials input block is used.  For
example, to redirect the code to use HGH pseudopotentials for carbon
and hydrogen, the following input would be used.
\begin{verbatim}
nwpw
   ...
   pseudopotentials
    C  library  HGH_LDA
    H  library  HGH_LDA
   end
   ...
end
\end{verbatim}
The implementation of HGH pseudopotentials is rather limited in this release.
HGH pseudopotentials cannot be used to optimize unit cells, and they
do not work with the MULLIKEN option.  They also have not yet been implemented 
into the BAND structure code.

To read in pseudopotentials in CPI format the following input would be used.
\begin{verbatim}
nwpw
   ...
   pseudopotentials
    C  CPI  c.cpi
    H  CPI  h.cpi
   end
   ...
end
\end{verbatim}
In order for the program to recognize the CPI format the CPI files, e.g. c.cpi 
have to be prepended with the ``$<$CPI$>$'' keyword.

To read in pseudopotentials in TETER format the following input would be used.
\begin{verbatim}
nwpw
   ...
   pseudopotentials
    C  TETER  c.teter
    H  TETER  h.teter
   end
   ...
end
\end{verbatim}
In order for the program to recognize the TETER format the TETER files, e.g. c.teter 
have to be prepended with the ``$<$TETER$>$'' keyword.


If you wish to redirect the code to a different directory other than
the default one, 
you need to set the environmental variable
{\tt NWCHEM\_NWPW\_LIBRARY}
to the new location of the \verb+libraryps+ directory.



\section{NWPW RTDB Entries and DataFiles}
\label{sec:pspw_data}
Input to the PSPW and Band modules are contained in both the RTDB and datafiles.
The RTDB is used to store input that the user will need to directly specify.
Input of this kind includes ion positions, ion velocities, and simulation cell
parameters.  The datafiles are used to store input, such the one-electron 
orbitals, one-electron orbital velocities, formatted pseudopotentials, 
and one-dimensional pseudopotentials, that the user will in most cases
run a program to generate.

\subsection{Ion Positions}
The positions of the ions are stored in the default geometry structure
in the RTDB and must be specified  using the GEOMETRY directive.

\subsection{Ion Velocities}
The velocities of the ions are stored in the default geometry structure
in the RTDB, and must be specified using the GEOMETRY directive.



\subsection{Wavefunction Datafile}
The one-electron orbitals are stored in a wavefunction datafile.  This
is a binary file and cannot be directly edited.  This datafile is used
by steepest\_descent and Car-Parrinello tasks and can be generated
using the wavefunction\_initializer or wavefunction\_expander tasks.

\subsection{Velocity Wavefunction Datafile}
The one-electron orbital velocities are stored in a velocity wavefunction 
datafile.  This is a binary file and cannot be directly edited.  This datafile 
is used by the Car-Parrinello task and can be generated
using the v\_wavefunction\_initializer task.

\subsection{Formatted Pseudopotential Datafile}
The pseudopotentials in Kleinman-Bylander form expanded on a simulation
cell (3d grid) are stored in a formatted pseudopotential datafile.
This is a binary file and cannot be directly edited.
This datafile 
is used by steepest\_descent and Car-Parrinello tasks and can be generated
using the pseudopotential\_formatter task.

\subsection{One-Dimensional Pseudopotential Datafile}
The one-dimensional pseudopotentials are stored in a one-dimensional 
pseudopotential file.  This is an ASCII file and can be directly edited with
a text editor.  However, the user will usually use the psp\_generator
task to generate this datafile.

The data stored in the one-dimensional pseudopotential file is
\begin{verbatim}
   character*2 element       :: element name
   integer     charge        :: valence charge of ion
   real        mass          :: mass of ion
   integer     lmax          :: maximum angular component
   real        rcut(lmax)    :: cutoff radii used to define pseudopotentials
   integer     nr            :: number of points in the radial grid
   real        dr            :: linear spacing of the radial grid
   real        r(nr)         :: one-dimensional radial grid
   real        Vpsp(nr,lmax) :: one-dimensional pseudopotentials
   real        psi(nr,lmax)  :: one-dimensional pseudowavefunctions
   real        r_semicore        :: semicore radius
   real        rho_semicore(nr)  :: semicore density
\end{verbatim}
and the format of it is:
\begin{verbatim}
[line 1:     ] element  
[line 2:     ] charge mass lmax
[line 3:     ] (rcut(l), l=1,lmax)
[line 4:     ] nr dr
[line 5:     ]    r(1)  (Vpsp(1,l),  l=1,lmax)
[line 6:     ] ....
[line nr+4:  ] r(nr) (Vpsp(nr,l), l=1,lmax)
[line nr+5:  ] r(1)  (psi(1,l), l=1,lmax) 
[line nr+6:  ] ....
[line 2*nr+4:] r(nr) (psi(nr,l), l=1,lmax)
[line 2*nr+5:] r_semicore
if (r_semicore read) then
[line 2*nr+6:] r(1)  rho_semicore(1)
[line 2*nr+7:] ....
[line 3*nr+5:] r(nr) rho_semicore(nr)
end if
\end{verbatim}



\subsection{PSPW Car-Parrinello Output Datafiles}
\label{sec:pspw_cp_data}

\subsubsection{XYZ motion file}
Data file that stores ion positions and velocities as
a function of time in XYZ format.

\begin{verbatim}
[line 1:          ]  n_ion
[line 2:          ]  
do ii=1,n_ion
[line 2+ii:       ] atom_name(ii), x(ii),y(ii),z(ii),vx(ii),vy(ii),vz(ii)
end do
[line n_ion+3     ] n_nion

do ii=1,n_ion
[line n_ion+3+ii: ] atom_name(ii), x(ii),y(ii),z(ii), vx(ii),vy(ii),vz(ii)
end do
[line 2*n_ion+4:  ]  ....
\end{verbatim}


\subsubsection{ION\_MOTION motion file}
Datafile that stores ion positions and velocities
as a function of time

\begin{verbatim}
[line 1:          ]  it_out, n_ion, omega
[line 2:          ]  time
do ii=1,n_ion
[line 2+ii:       ] x(ii),y(ii),z(ii), vx(ii),vy(ii),vz(ii)
end do
[line n_ion+3     ] time
do 
do ii=1,n_ion
[line n_ion+3+ii: ] x(ii),y(ii),z(ii), vx(ii),vy(ii),vz(ii)
end do
[line 2*n_ion+4:  ]  ....
\end{verbatim}

\subsubsection{EMOTION motion file}
Datafile that store energies as a function of time
\begin{verbatim}
[line 1:          ]  time, E1,E2,E3,E4,E5,E6,E7,E8, (E9,E10, if Nose-Hoover)
[line 2:          ] ...
\end{verbatim}


\subsubsection{HMOTION motion file}
Datafile that stores the rotation matrix
as a function of time.

\begin{verbatim}
[line 1:          ] time
[line 2:          ] ms,ne(ms),ne(ms)
do i=1,ne(ms)
[line 2+i:        ] (hml(i,j), j=1,ne(ms)
end do
[line 3+ne(ms):   ] time
[line 4+ne(ms):   ] ....
\end{verbatim}


\subsubsection{EIGMOTION motion file}
Datafile that stores the eigenvalues for the one-electron
orbitals as a function of time.

\begin{verbatim}
[line 1:          ]  time, (eig(i), i=1,number_orbitals)
[line 2:          ] ...
\end{verbatim}


\subsubsection{OMOTION motion file}
Datafile that stores a reduced representation of the
one-electron orbitals.  To be used with a molecular
orbital viewer that will be ported to NWChem
in the near future. 

       

\section{Car-Parrinello Scheme for Ab Initio Molecular Dynamics}
\label{sec:pspw_Car-Parrinello}

Car and Parrinello developed a unified scheme for doing {\it ab initio}
molecular dynamics by combining the motion of the ion cores and a fictitious
motion for the Kohn-Sham orbitals of density-functional theory 
(R. Car and M. Parrinello, Phys. Rev. Lett. \textbf{55}, 2471, (1985)).  
At the heart of this method they introduced a fictitious kinetic energy 
functional for the Kohn-Sham orbitals.

\begin{eqnarray}
\label{appendix:b1}
KE(\{\psi_{i,\sigma}(\vec{r})\}) &=& \sum_{i,\sigma}^{occ} 
                                      \int d\vec{r}\ \mu \left| 
                                      \dot{\psi}_{i,\sigma}(\vec{r}) \right|^2 
\end{eqnarray}

\noindent
Given this kinetic energy the constrained equations of motion are found 
by taking the first variation of the auxiliary Lagrangian.
\begin{eqnarray}
\label{appendix:b2}
L &=& \sum_{i,\sigma}^{occ} \int d\vec{r}\ \mu \left| 
     \dot{\psi}_{i,\sigma}(\vec{r}) \right|^2 
     + \frac 12 \sum_I M_I \left| \dot{\vec{R}}_I \right|^2 
- E\left[ \left\{ \psi_{i,\sigma}(\vec{r})\right\},\left\{\vec{R}_I \right\} \right]  
\nonumber \\
&&+\sum_{ij,\sigma} \Lambda_{ij,\sigma} \left( \int d\vec{r}\ 
\psi_{i,\sigma}^{*}(\vec{r}) \psi_{j,\sigma}(\vec{r}) - \delta_{ij,\sigma} 
\right) 
\end{eqnarray}

\noindent
Which generates a dynamics for the wavefunctions $\psi_{i,\sigma}(\vec{r})$ and 
atoms positions $\vec{R}_I$ through the constrained equations of motion:

\begin{eqnarray}
\mu \ddot{\psi}_{i,\sigma}(\vec{r},t) &=& -\frac{\delta E}{\delta \psi_{i,\sigma }^{*}
\left( \vec{r},t \right) } + \sum\limits_j \Lambda_{ij,\sigma} 
\psi_{j,\sigma} \left( \vec{r},t \right)
\label{eq:b3}
\end{eqnarray}
\begin{eqnarray}
M_I \ddot{\vec{R}}_I &=& -\frac{\partial E}{\partial \vec{R}_I}
\label{eq:b4}
\end{eqnarray}

\noindent
where $\mu$ is the fictitious mass for the electronic degrees of freedom and 
$M_I$ are the ionic masses.  
The adjustable parameter $\mu$ is used to 
describe the relative rate at which the wavefunctions change with time.  
$\Lambda_{ij,\sigma}$ are the 
Lagrangian multipliers for the orthonormalization of the single-particle 
orbitals $\psi_{i,\sigma}(\vec{r})$. 
They are defined by the orthonormalization constraint conditions
and can be rigorously found. 
However, the equations of motion for the Lagrange multipliers
depend on the specific algorithm used to integrate
Eqs.~\ref{eq:b3}-\ref{eq:b4}.

For this method to give ionic motions that are physically meaningful
the kinetic energy of the Kohn-Sham orbitals must be relatively
small when compared to the kinetic energy of the ions.
There are two ways where this criterion can fail.
First, the numerical integrations for the Car-Parrinello equations of motion 
can often lead to large relative values of the kinetic energy of 
the Kohn-Sham orbitals relative to the kinetic energy of the ions.
This kind of failure is easily fixed by requiring a more accurate
numerical integration, i.e. use a smaller time step for the numerical
integration.
Second, during the motion of the system a the ions can be in locations where
there is an Kohn-Sham orbital level crossing, i.e. the density-functional
energy can have two states that are nearly degenerate.  This kind
of failure often occurs in the study of chemical reactions.
This kind of failure is not easily fixed and requires the use
of a more sophisticated density-functional energy that accounts
for low-lying excited electronic states.


\subsection{Verlet Algorithm for Integration}
%\subsection{Verlet Algorithm for Integrating Eqs. \ref{eq:b3} - \ref{eq:b4} }

Eqs.~\ref{eq:b3}-\ref{eq:b4} integrated using the Verlet algorithm
results in

\begin{eqnarray}
\psi_{i,\sigma}^{t+ \Delta t} 
                   &\leftarrow& 
                    2 \psi_{i,\sigma}^{t} - \psi_{i,\sigma}^{t-\Delta t}
                      + \frac{(\Delta t)^2}{\mu}
                        \left[ 
                           \frac{\delta E}{\delta \psi_{i,\sigma}^{*}}
                            + \sum_{j} \psi_{j,\sigma} \Lambda_{ji,\sigma} 
                        \right]_{t}
\label{eq:b6}
\end{eqnarray}
\begin{eqnarray}
\vec{R}_I^{t+\Delta t} &\leftarrow& 
                    2 \vec{R}_I^{t} - \vec{R}_I^{t-\Delta t}
                    + \frac{(\Delta t)^2}{M_I} 
                       \frac{\partial E}{\partial \vec{R}_I}
\label{eq:b7}
\end{eqnarray}
               
In this molecular dynamic procedure we have to know variational derivative
$\frac{\delta E}{\delta \psi_{i,\sigma}^{*}}$ and the matrix 
$\Lambda_{ij,\sigma}$. 
The variational derivative $\frac{\delta E}{\delta \psi_{i,\sigma}^{*}}$ 
can be analytically found and is
\begin{eqnarray}
\frac{\delta E}{\delta \psi_{i,\sigma}^{*}} 
      &=&  -\frac{1}{2} \nabla^2 
            \psi_{i,\sigma}(\vec{r}) \nonumber \\
      &+& \int d\vec{r^{\prime}} 
           W_{ext}(\vec{r},\vec{r^{\prime}}) 
          \psi_{i,\sigma}(\vec{r^{\prime}}) \nonumber \\
      &+& \int d\vec{r^{\prime}} 
                    \frac{n(\vec{r^{\prime}})}{|\vec{r}-\vec{r^{\prime}}|}
          \psi_{i,\sigma}(\vec{r}) \nonumber \\
      &+& \mu_{xc}^{\sigma}(\vec{r}) 
          \psi_{i,\sigma}(\vec{r}) \nonumber \\
& \equiv & \hat{H} \psi_{i,\sigma}
\label{eq:b8}
\end{eqnarray}
                        
\noindent
To find the matrix $\Lambda_{ij,\sigma}$ we impose the orthonormality
constraint on $\psi_{i,\sigma}^{t+\Delta t}$ to obtain a
matrix Riccatti equation, and then Riccatti equation is solved by an iterative
solution 
% ref does not exist
%(see section ~\ref{sec:pspw_sd2}).


\subsection{Constant Temperature Simulations: Nose-Hoover Thermostats}
\label{sec:pspw_nose}

Nose-Hoover Thermostats for the electrons and ions can also be added to the 
Car-Parrinello simulation.  In this type of simulation thermostats variables $x_e$ and $x_R$ 
are added to the simulation by adding the auxiliary energy functionals to the total energy.
\begin{eqnarray}
ION\_THERMOSTAT(x_R)      &=&  \frac{1}{2} Q_R \dot{x_R} + E_{R0}x_R \\ 
ELECTRON\_THERMOSTAT(x_e) &=&  \frac{1}{2} Q_e \dot{x_e} + E_{e0}x_e 
\end{eqnarray}

In these equations, the average kinetic energy for the ions is
\begin{eqnarray}
E_{R0} = \frac{1}{2} f k_B T
\end{eqnarray}
where $f$ is the number of atomic degrees of freedom, $k_B$ is 
Boltzmann's constant, and T is the desired temperature.  Defining
the average fictitious kinetic energy of the electrons is not as straightforward.
Bl\"{o}chl and Parrinello 
(P.E. Bl\"{o}chl and M. Parrinello, Phys. Rev. B, \textbf{45}, 9413, (1992)) 
have suggested the following formula for determining
the average fictitious kinetic energy
\begin{eqnarray}
E_{e0} = 4 k_B T \frac{\mu}{M} \sum_i <\psi_i|-\frac{1}{2} \nabla^2 |\psi_i>
\end{eqnarray}
where $\mu$ is the fictitious electronic mass, $M$ is average mass of one atom,
and $\sum_i <\psi_i|-\frac{1}{2} \nabla^2 |\psi_i>$ is the kinetic energy of the
electrons.

Bl\"{o}chl and Parrinello suggested that the choice of mass parameters, 
$Q_e$, and $Q_R$ should be made such that the period of oscillating thermostats 
should be chosen larger than the typical time scale for the dynamical events of 
interest but shorter than the simulation time.  
\begin{eqnarray}
P_{ion} &=& 2\pi \sqrt{\frac{Q_R}{4E_{R0}}}\\
P_{electron} &=& 2\pi \sqrt{\frac{Q_e}{4E_{e0}}}
\end{eqnarray}
where $P_{ion}$ and $P_{electron}$ are the periods of oscillation for the ionic and fictitious
electronic thermostats.  


In simulated annealing simulations the electronic and ionic Temperatures are scaled 
according to an exponential cooling schedule,
\begin{eqnarray}
T_e(t) = T_e^0 \exp^{-\frac{t}{\tau_e}}\\
T_{ionic}(t) = T_{ionic}^0 \exp^{-\frac{t}{\tau_{ionic}}}
\end{eqnarray}
where $T_e^0$ and $T_{ionic}^0$ are the initial temperatures, and $\tau_e$ and $\tau_{ionic}$
are the cooling rates in atomic units.  



\section{PSPW Tutorial 1: Minimizing the geometry for a C$_2$ molecule}
\label{sec:pspw_sd}

In this section we show how use the PSPW module to optimize the geometry 
for a C$_2$ molecule at the PBE96 levels. 

In the following example we show the input needed to optimize the geometry
for a C$_2$ molecule at the LDA level.  In this example, default pseudopotentials
from the pseudopotential library are used for C, the boundary condition is free-space, 
the exchange correlation functional is PBE96, The boundary condition is free-space, and 
the simulation cell cell is aperiodic and cubic with a side length of 10.0 Angstroms and has
40 grid points in each direction (cutoff energy is 44 Ry).  
\begin{verbatim}
         
start c2_pspw_pbe96
title "C2 restricted singlet dimer optimization - PBE96/44Ry"

geometry  
C    -0.62 0.0 0.0
C     0.62 0.0 0.0
end
       
pspw
   simulation_cell units angstroms
      boundary_conditions aperiodic
      SC 10.0
      ngrid 40 40 40
   end
   xc pbe96
end
set nwpw:minimizer 2
task pspw optimize
\end{verbatim}



\normalsize
\section{PSPW Tutorial 2: Running a Car-Parrinello Simulation}
\label{sec:pspw_cp}
\normalsize

In this section we show how use the PSPW module to perform a Car-Parrinello
molecular dynamic simulation for a C$_2$ molecule at the LDA level.  
Before running a PSPW Car-Parrinello  simulation the system should be
on the Born-Oppenheimer surface, i.e. the one-electron orbitals should be minimized 
with respect to the total energy (i.e. task pspw energy).  The input needed
is basically the same as for optimizing the geometry of a C$_2$ molecule at the LDA level,
except that and additional Car-Parrinello sub-block is added.  

In the following example we show the input needed to run a Car-Parrinello simulation
for a C$_2$ molecule at the LDA level.  In this example, default pseudopotentials
from the pseudopotential library are used for C, the boundary condition is free-space, 
the exchange correlation functional is LDA, The boundary condition is free-space, and 
the simulation cell cell is aperiodic and cubic with a side length of 10.0 Angstroms and has
40 grid points in each direction (cutoff energy is 44 Ry).  The time step and fake mass
for the Car-Parrinello run are specified to be 5.0 au and 600.0 au, respectively.  

\begin{verbatim}
         
start c2_pspw_lda_md
title "C2 restricted singlet dimer, LDA/44Ry - constant energy Car-Parrinello simulation"

geometry  
C    -0.62 0.0 0.0
C     0.62 0.0 0.0
end
       
pspw
   simulation_cell units angstroms
      boundary_conditions aperiodic
      lattice
        lat_a 10.00d0
        lat_b 10.00d0
        lat_c 10.00d0
      end
      ngrid 40 40 40
   end
   Car-Parrinello
     fake_mass 600.0
     time_step 5.0
     loop 10 10
   end
end
set nwpw:minimizer 2
task pspw energy
task pspw Car-Parrinello
\end{verbatim}


\normalsize
\section{PSPW Tutorial 3: optimizing a unit cell and geometry for Silicon-Carbide}
\label{sec:pspw_unitcell_optimization}

The following example demonstrates how to uses the PSPW module to optimize the unit cell
and geometry for a silicon-carbide crystal.

\begin{verbatim}
title "SiC 8 atom cubic cell - geometry and unit cell optimization"

start SiC

#**** Enter the geometry using fractional coordinates ****
geometry units au center noautosym noautoz print
  system crystal
    lat_a 8.277d0
    lat_b 8.277d0
    lat_c 8.277d0
    alpha 90.0d0
    beta  90.0d0
    gamma 90.0d0
  end
Si    -0.50000d0  -0.50000d0  -0.50000d0
Si     0.00000d0   0.00000d0  -0.50000d0
Si     0.00000d0  -0.50000d0   0.00000d0
Si    -0.50000d0   0.00000d0   0.00000d0
C     -0.25000d0  -0.25000d0  -0.25000d0
C      0.25000d0   0.25000d0  -0.25000d0
C      0.25000d0  -0.25000d0   0.25000d0
C     -0.25000d0   0.25000d0   0.25000d0
end

#***** setup the nwpw gamma point code ****
nwpw
   simulation_cell
     ngrid 16 16 16
   end
   ewald_ncut 8
end
set nwpw:minimizer 2
set nwpw:psi_nolattice .true.  # turns of unit cell checking for wavefunctions

driver
  clear
  maxiter 40
end
set includestress .true.    # this option tells driver to optimize the unit cell

task pspw optimize

\end{verbatim}

\normalsize
\section{PSPW Tutorial 4: QM/MM simulation for CCl$_4$ + 64H$_2$O}
\label{sec:pspw_qmmm_simulation}

In this section we show how use the PSPW module to perform a Car-Parrinello
QM/MM simulation for a CCl$_4$ molecule in a box of 64 H$_2$O.
Before running a PSPW Car-Parrinello  simulation the system should be
on the Born-Oppenheimer surface, i.e. the one-electron orbitals should be minimized
with respect to the total energy (i.e. task pspw energy).  

In the following example we show the input needed to run a Car-Parrinello 
QM/MM simulation for a CCl$_4$ molecule in a box of 64 H$_2$O. 
In this example, default pseudopotentials from the pseudopotential library are used 
for C, Cl, O\^\ and H\^\, exchange correlation functional is PBE96, The boundary condition is periodic, and
with a side length of 23.577 Bohrs and has a cutoff energy is 50 Ry).  The time step and fake mass
for the Car-Parrinello run are specified to be 5.0 au and 600.0 au, respectively.
\normalsize
\begin{verbatim}
title "CCl4 + water64 QM/MM simulation- 195 atom cell"

memory 1500 mb
start CCl4-water64

#scratch_dir   ./perm
#permanent_dir ./perm
\end{verbatim}
\tiny
\begin{verbatim}
geometry nocenter noautoz noautosym
C        0.7804E-02 -0.2897E-02  0.1420E-02 -0.2910E-07 -0.1055E-07 -0.2001E-07
Cl      -0.8603E+00  0.1547E+01  0.2556E+00 -0.2910E-07 -0.1055E-07 -0.2001E-07
Cl       0.8421E+00 -0.4774E+00  0.1518E+01 -0.2910E-07 -0.1055E-07 -0.2001E-07
Cl      -0.1167E+01 -0.1279E+01 -0.4592E+00 -0.2910E-07 -0.1055E-07 -0.2001E-07
Cl       0.1218E+01  0.1972E+00 -0.1309E+01 -0.2910E-07 -0.1055E-07 -0.2001E-07

O^       0.1545E+01 -0.3640E+01 -0.2558E+01  0.1675E-03 -0.2134E-03  0.2608E-03
H^       0.6377E+00 -0.4054E+01 -0.2486E+01 -0.8467E-05 -0.6710E-04 -0.1101E-02
H^       0.1860E+01 -0.3690E+01 -0.3506E+01  0.9734E-03  0.1042E-02  0.4566E-03
O^      -0.6138E+01 -0.4627E+01 -0.1181E+01 -0.1477E-03 -0.1616E-03 -0.1670E-03
H^      -0.7068E+01 -0.4458E+01 -0.8549E+00 -0.6948E-03 -0.5435E-03 -0.1521E-02
H^      -0.5628E+01 -0.5133E+01 -0.4858E+00 -0.8855E-03  0.2768E-03  0.6961E-03
O^       0.3808E+01  0.2935E+01  0.2147E+01 -0.6374E-04  0.1081E-03 -0.3184E-04
H^       0.4187E+01  0.2253E+01  0.2772E+01  0.5832E-03  0.5155E-03  0.2134E-04
H^       0.4511E+01  0.3612E+01  0.1926E+01 -0.4943E-03  0.3230E-03 -0.7034E-03
O^      -0.6210E+00 -0.5260E+01 -0.2842E+01  0.1727E-03  0.9623E-04  0.8184E-04
H^      -0.2750E+00 -0.5523E+01 -0.3742E+01  0.1119E-03  0.7072E-03 -0.1203E-03
H^      -0.7042E+00 -0.6073E+01 -0.2266E+01 -0.6331E-03 -0.4150E-03 -0.7442E-03
O^       0.2760E+01  0.2730E+01 -0.4774E+01 -0.1327E-03  0.3354E-03 -0.1366E-03
H^       0.3676E+01  0.2570E+01 -0.5142E+01 -0.5941E-04  0.5145E-03 -0.3057E-04
H^       0.2828E+01  0.2989E+01 -0.3811E+01 -0.2370E-03  0.1001E-02 -0.3029E-03
O^      -0.2387E+01  0.5716E+01  0.3965E+01 -0.6296E-04 -0.1405E-04 -0.2853E-04
H^      -0.1694E+01  0.5121E+01  0.3558E+01  0.5787E-04 -0.5014E-03  0.9024E-03
H^      -0.1939E+01  0.6465E+01  0.4453E+01 -0.2046E-03  0.6948E-04 -0.2150E-04
O^      -0.3456E+01  0.5123E+01 -0.2154E+01 -0.3714E-04 -0.1948E-03  0.4699E-05
H^      -0.3043E+01  0.4342E+01 -0.2622E+01 -0.1119E-05 -0.3220E-03  0.2734E-03
H^      -0.3693E+01  0.5826E+01 -0.2825E+01 -0.2085E-03 -0.4813E-03 -0.2351E-03
O^       0.5940E+00  0.1399E+01  0.3463E+01 -0.1288E-03 -0.9776E-04 -0.1409E-03
H^       0.1245E+01  0.1913E+01  0.2904E+01 -0.3790E-03 -0.4010E-03 -0.6948E-03
H^      -0.3201E+00  0.1790E+01  0.3355E+01 -0.1070E-03  0.6148E-04  0.3431E-03
O^       0.4845E+01  0.4936E+01  0.4088E+01  0.2876E-03 -0.2625E-03 -0.7661E-04
H^       0.4098E+01  0.5168E+01  0.3465E+01 -0.1301E-03 -0.5756E-04  0.4909E-03
H^       0.4740E+01  0.3991E+01  0.4397E+01  0.1202E-02 -0.7660E-03 -0.1345E-02
O^      -0.7209E+00  0.4285E+01  0.1237E+01 -0.1519E-03  0.6569E-04  0.1096E-03
H^       0.1958E-01  0.4499E+01  0.6000E+00 -0.5731E-04 -0.2919E-03  0.9929E-04
H^      -0.1597E+01  0.4541E+01  0.8281E+00 -0.1117E-03 -0.3977E-03 -0.2666E-03
O^       0.3836E+01  0.2390E-01 -0.6670E+00  0.2697E-04 -0.6474E-05 -0.4852E-03
H^       0.4335E+01 -0.5028E+00  0.2166E-01 -0.1127E-02  0.1144E-02  0.1230E-02
H^       0.2962E+01 -0.4225E+00 -0.8598E+00 -0.1721E-03  0.3171E-03 -0.3321E-03
O^      -0.7034E+00 -0.2120E+01 -0.3538E+01  0.1292E-03 -0.1072E-03 -0.1333E-03
H^      -0.7567E+00 -0.3086E+01 -0.3284E+01  0.5037E-03 -0.2660E-03 -0.6455E-03
H^       0.2369E+00 -0.1894E+01 -0.3793E+01  0.2452E-03  0.6329E-03  0.9344E-03
O^       0.1465E+01 -0.4009E+01  0.4737E+01  0.3828E-03 -0.5067E-04  0.2649E-03
H^       0.4923E+00 -0.3944E+01  0.4958E+01  0.3572E-03  0.1205E-02 -0.2187E-03
H^       0.1582E+01 -0.4544E+01  0.3901E+01  0.1254E-03 -0.4486E-03  0.4761E-03
O^      -0.3224E+01 -0.3091E+00  0.1701E+01 -0.2499E-03 -0.2643E-03 -0.1442E-03
H^      -0.2950E+01 -0.9626E+00  0.9949E+00 -0.4237E-03 -0.8809E-04 -0.3749E-03
H^      -0.3005E+01  0.6188E+00  0.1398E+01 -0.6682E-03 -0.1587E-03 -0.1246E-03
O^       0.5321E+01  0.3728E+01 -0.5992E+01  0.1243E-03  0.1407E-03  0.3011E-04
H^       0.5383E+01  0.3541E+01 -0.5012E+01  0.2175E-03 -0.4499E-04  0.4888E-05
H^       0.5850E+01  0.3045E+01 -0.6496E+01 -0.1043E-03  0.1021E-03 -0.1441E-03
O^      -0.3365E+01 -0.2780E+01  0.3200E+01  0.2012E-03 -0.1310E-03 -0.1047E-04
H^      -0.4036E+01 -0.2193E+01  0.3653E+01 -0.5141E-03 -0.4867E-03 -0.6379E-03
H^      -0.3029E+01 -0.2327E+01  0.2375E+01  0.8564E-03  0.1057E-03  0.3940E-03
O^      -0.6115E+01  0.4096E+01 -0.1385E+01 -0.2067E-03 -0.2544E-03  0.9568E-04
H^      -0.6740E+01  0.3315E+01 -0.1402E+01 -0.7231E-04 -0.3492E-03 -0.2122E-03
H^      -0.5445E+01  0.4002E+01 -0.2121E+01 -0.1573E-03  0.1356E-03  0.9095E-04
O^      -0.1742E+01  0.5855E+01 -0.5125E+01 -0.4122E-03 -0.4759E-04 -0.3874E-04
H^      -0.1849E+01  0.6848E+01 -0.5063E+01  0.6007E-03  0.8115E-04 -0.2854E-03
H^      -0.9641E+00  0.5640E+01 -0.5716E+01 -0.3814E-03 -0.9778E-03  0.3468E-03
O^       0.3739E+01  0.4907E+01 -0.2428E+00 -0.1192E-05 -0.2368E-03  0.6724E-04
H^       0.3792E+01  0.3995E+01  0.1643E+00 -0.1132E-02 -0.3235E-03  0.1695E-04
H^       0.4347E+01  0.4954E+01 -0.1035E+01 -0.9395E-03 -0.1354E-02 -0.7298E-03
O^       0.2987E+00 -0.5628E+01 -0.8431E-01 -0.1166E-03  0.1187E-03 -0.7732E-04
H^       0.1276E+01 -0.5804E+01  0.3260E-01 -0.1801E-03  0.3589E-04  0.3022E-03
H^       0.5558E-01 -0.4777E+01  0.3824E+00 -0.2238E-03  0.1517E-03 -0.1903E-03
O^       0.1671E+01 -0.3048E+00 -0.4287E+01 -0.8597E-04  0.3502E-04  0.1369E-03
H^       0.2286E+01 -0.1088E+01 -0.4380E+01  0.2453E-03  0.3302E-03 -0.1838E-03
H^       0.1079E+01 -0.2489E+00 -0.5091E+01  0.4959E-03  0.6558E-03 -0.2504E-03
O^      -0.2941E-01  0.2661E+01 -0.4082E+01  0.1756E-03 -0.5742E-04 -0.1573E-03
H^      -0.8858E+00  0.2586E+01 -0.3571E+01  0.9848E-03 -0.7154E-04  0.1212E-02
H^       0.2989E+00  0.1746E+01 -0.4318E+01  0.2778E-03 -0.9530E-04  0.1150E-03
O^      -0.1659E+01  0.3915E+00 -0.2844E+01 -0.1270E-04 -0.1120E-03 -0.9166E-04
H^      -0.1204E+01  0.8157E+00 -0.2061E+01 -0.1351E-02  0.1320E-02 -0.9371E-04
H^      -0.1101E+01 -0.3654E+00 -0.3184E+01  0.1361E-02  0.3184E-03  0.1229E-02
O^       0.2089E+01  0.5535E+01 -0.3917E+01  0.1204E-04 -0.7803E-04  0.8825E-04
H^       0.2792E+01  0.6204E+01 -0.4157E+01  0.3230E-03 -0.4625E-03 -0.6275E-04
H^       0.2250E+01  0.4687E+01 -0.4423E+01 -0.8063E-03 -0.1769E-04 -0.2737E-03
O^      -0.3593E+01  0.4433E+01  0.9266E+00  0.1739E-03 -0.3290E-04 -0.3843E-04
H^      -0.4481E+01  0.4653E+01  0.1330E+01  0.7439E-04  0.3872E-03 -0.4940E-03
H^      -0.3441E+01  0.5007E+01  0.1217E+00  0.4864E-03 -0.6984E-03 -0.4424E-03
O^       0.5367E+01 -0.2126E+01 -0.1991E+01 -0.2551E-03  0.1323E-04  0.1464E-03
H^       0.4615E+01 -0.2084E+01 -0.1333E+01 -0.9232E-03 -0.8571E-03 -0.5616E-03
H^       0.6028E+01 -0.2817E+01 -0.1698E+01 -0.9047E-03 -0.8682E-03 -0.4681E-03
O^      -0.5302E+01  0.2831E+01  0.3682E+01 -0.8660E-05  0.1412E-03  0.1894E-06
H^      -0.5277E+01  0.3688E+01  0.3167E+01  0.1207E-02  0.1614E-03  0.9149E-04
H^      -0.5660E+01  0.2102E+01  0.3099E+01 -0.2688E-03  0.5247E-03 -0.3316E-03
O^      -0.4788E+01 -0.5922E+01 -0.4919E+01 -0.3929E-03 -0.9853E-05  0.3585E-03
H^      -0.4466E+01 -0.4994E+01 -0.5108E+01  0.2628E-03 -0.1497E-03  0.8332E-03
H^      -0.5586E+01 -0.6120E+01 -0.5489E+01 -0.1623E-03  0.6350E-03 -0.1750E-03
O^       0.2449E+01  0.5722E+01  0.2217E+01  0.1955E-03  0.6679E-06  0.1909E-03
H^       0.1457E+01  0.5804E+01  0.2318E+01  0.1783E-03 -0.2907E-03  0.2435E-03
H^       0.2696E+01  0.4757E+01  0.2130E+01  0.5892E-03 -0.2071E-04  0.1586E-02
O^       0.5651E+01  0.8176E+00 -0.2769E+01 -0.5866E-04 -0.1602E-04 -0.5855E-04
H^       0.4741E+01  0.1046E+01 -0.2421E+01 -0.1450E-04 -0.4401E-03  0.3128E-03
H^       0.5700E+01 -0.1641E+00 -0.2953E+01  0.2489E-03  0.4500E-04 -0.2890E-03
O^       0.2231E+01 -0.2461E+01 -0.6899E-01 -0.1876E-04  0.7416E-04  0.1305E-03
H^       0.2029E+01 -0.2779E+01 -0.9952E+00  0.1240E-02 -0.7506E-03  0.1383E-03
H^       0.1687E+01 -0.2982E+01  0.5886E+00  0.7330E-03 -0.8484E-03  0.8706E-05
O^       0.2294E+01  0.3375E+01  0.4716E+01 -0.1652E-03  0.2400E-03 -0.1586E-04
H^       0.1550E+01  0.2835E+01  0.5111E+01  0.4246E-03 -0.1052E-02 -0.6720E-03
H^       0.3160E+01  0.3107E+01  0.5138E+01  0.1669E-04  0.1270E-02  0.2646E-03
O^       0.7453E+00  0.4511E+01 -0.1428E+01  0.3864E-05  0.1202E-03 -0.1956E-04
H^       0.1116E+01  0.5421E+01 -0.1241E+01  0.2420E-03  0.3195E-06  0.1360E-03
H^       0.4066E+00  0.4476E+01 -0.2368E+01  0.5441E-03  0.1747E-03 -0.2138E-03
O^       0.3229E+01  0.2523E+01 -0.1702E+01  0.8684E-04  0.1770E-03 -0.1302E-03
H^       0.2367E+01  0.3024E+01 -0.1626E+01  0.1473E-03  0.2022E-03  0.3857E-03
H^       0.3259E+01  0.1803E+01 -0.1008E+01  0.2618E-03 -0.1686E-03 -0.4964E-03
O^      -0.4696E+01  0.1761E+01 -0.5781E+01 -0.1399E-03  0.5263E-04 -0.1977E-04
H^      -0.4285E+01  0.1469E+01 -0.4918E+01 -0.2145E-03  0.4890E-03  0.1638E-03
H^      -0.4401E+01  0.2692E+01 -0.5993E+01  0.4905E-03 -0.3279E-03 -0.8519E-03
O^      -0.1324E+01  0.8348E+00  0.5926E+01 -0.4021E-03 -0.2283E-03  0.1866E-03
H^      -0.9165E+00  0.1419E+01  0.5224E+01  0.6640E-03 -0.3537E-03  0.7015E-03
H^      -0.1446E+01 -0.8850E-01  0.5561E+01  0.6728E-04 -0.2690E-03  0.1331E-03
O^       0.3599E+01 -0.4806E+01  0.5923E+01  0.2240E-03  0.1057E-03 -0.1705E-06
H^       0.2749E+01 -0.4798E+01  0.5396E+01  0.7639E-03  0.1003E-02 -0.8568E-03
H^       0.3835E+01 -0.5749E+01  0.6158E+01 -0.8554E-03 -0.1746E-03 -0.3717E-04
O^       0.3944E+01  0.1279E+01  0.4873E+01 -0.1309E-04  0.2875E-03 -0.3979E-03
H^       0.4210E+01  0.1174E+01  0.5831E+01  0.1066E-02  0.1745E-02 -0.5234E-03
H^       0.3429E+01  0.4734E+00  0.4579E+01 -0.2975E-03 -0.1943E-03  0.1395E-02
O^       0.5483E+01 -0.7180E+00 -0.5757E+01  0.1312E-03  0.1083E-03  0.7991E-04
H^       0.6244E+01 -0.1590E+00 -0.6085E+01 -0.2745E-03  0.5021E-03 -0.1907E-03
H^       0.5629E+01 -0.9484E+00 -0.4795E+01  0.4035E-03  0.4934E-03  0.1307E-03
O^      -0.3287E+00  0.3790E+01  0.5524E+01 -0.7176E-05  0.1175E-03  0.2084E-04
H^       0.3281E+00  0.3309E+01  0.4943E+01  0.4936E-04  0.5155E-04  0.1453E-03
H^      -0.5608E+00  0.3217E+01  0.6310E+01  0.1927E-03  0.3956E-03  0.2736E-03
O^      -0.4527E+01 -0.7948E+00 -0.3582E+01 -0.5240E-04  0.1526E-03  0.1337E-03
H^      -0.3613E+01 -0.3987E+00 -0.3669E+01 -0.1226E-03  0.1632E-03 -0.5451E-03
H^      -0.4999E+01 -0.3775E+00 -0.2804E+01  0.5636E-03  0.5893E-04  0.5571E-03
O^      -0.6007E+01 -0.5060E+01  0.4881E+01 -0.1087E-04  0.3392E-03  0.9991E-04
H^      -0.5573E+01 -0.4743E+01  0.4038E+01 -0.1440E-03 -0.6469E-03 -0.3565E-03
H^      -0.6934E+01 -0.4690E+01  0.4939E+01 -0.3791E-03 -0.4391E-03 -0.7065E-03
O^       0.2677E+01 -0.3981E+01  0.2476E+01  0.1089E-03 -0.1054E-03  0.1375E-03
H^       0.3493E+01 -0.3506E+01  0.2147E+01 -0.6883E-04  0.1734E-03  0.8114E-04
H^       0.2947E+01 -0.4819E+01  0.2949E+01  0.4465E-03  0.2834E-03  0.6339E-03
O^      -0.5289E+01  0.1887E+01  0.8394E+00  0.9633E-04 -0.1890E-04 -0.1561E-03
H^      -0.5106E+01  0.1824E+01 -0.1418E+00  0.5279E-03 -0.4110E-03 -0.5213E-04
H^      -0.5189E+01  0.9829E+00  0.1255E+01  0.4782E-03  0.2522E-03  0.3444E-03
O^      -0.2867E+01 -0.3827E+01 -0.4340E+01  0.3624E-03 -0.4131E-03 -0.1321E-03
H^      -0.2075E+01 -0.3938E+01 -0.3739E+01  0.3156E-03  0.3721E-03  0.8843E-04
H^      -0.3635E+01 -0.3461E+01 -0.3814E+01  0.6431E-03  0.1311E-02 -0.9227E-03
O^      -0.3108E+01 -0.5334E+01 -0.4502E+00  0.1643E-03  0.8922E-04 -0.6109E-04
H^      -0.3192E+01 -0.5798E+01 -0.1332E+01 -0.7280E-03  0.6552E-03 -0.2814E-03
H^      -0.3767E+01 -0.4583E+01 -0.4035E+00 -0.1942E-04 -0.1519E-03  0.1222E-02
O^      -0.4554E+01 -0.2685E+01 -0.1490E+01 -0.1628E-03  0.1207E-03 -0.5201E-04
H^      -0.4735E+01 -0.2235E+01 -0.2365E+01 -0.7933E-03  0.5578E-04  0.5343E-04
H^      -0.5002E+01 -0.2175E+01 -0.7554E+00 -0.7371E-03 -0.5704E-03  0.7546E-04
O^       0.3475E+01 -0.4761E+01 -0.1158E+01  0.4877E-04  0.1059E-04 -0.5547E-04
H^       0.3973E+01 -0.5628E+01 -0.1186E+01  0.4343E-03  0.2527E-03 -0.6376E-03
H^       0.2590E+01 -0.4871E+01 -0.1610E+01  0.9492E-04 -0.8052E-04 -0.1260E-03
O^       0.3854E+01 -0.3250E+01 -0.4023E+01  0.1560E-03 -0.1082E-03  0.5321E-04
H^       0.3588E+01 -0.3599E+01 -0.4922E+01  0.9956E-03 -0.1719E-03 -0.1793E-03
H^       0.4732E+01 -0.2777E+01 -0.4097E+01  0.5500E-03 -0.7001E-03  0.9756E-03
O^       0.5345E+01 -0.2101E+01  0.4132E+01 -0.2089E-03 -0.2574E-03  0.1618E-04
H^       0.6127E+01 -0.2332E+01  0.4711E+01 -0.9618E-03 -0.1658E-03  0.1067E-02
H^       0.4498E+01 -0.2226E+01  0.4648E+01 -0.8397E-03  0.5189E-03 -0.8292E-03
O^       0.1412E+00 -0.6149E+01  0.3138E+01  0.2892E-03  0.5036E-03  0.3292E-04
H^      -0.7153E+00 -0.6083E+01  0.2626E+01  0.1289E-03  0.4802E-03  0.3119E-03
H^      -0.2587E-01 -0.6612E+01  0.4008E+01  0.3628E-03  0.1494E-02  0.5832E-03
O^      -0.3946E+00 -0.1042E+01  0.4471E+01 -0.1576E-04 -0.2368E-03  0.1489E-03
H^       0.9542E-03 -0.3942E+00  0.3820E+01 -0.3498E-03  0.1869E-03  0.3694E-03
H^      -0.1312E+01 -0.1300E+01  0.4169E+01  0.4445E-03 -0.1244E-02 -0.3945E-03
O^       0.5287E+01 -0.5793E+01  0.1490E+01 -0.7884E-04 -0.1027E-03 -0.2446E-03
H^       0.5544E+01 -0.5748E+01  0.2455E+01  0.2712E-03 -0.1871E-03 -0.3382E-03
H^       0.4354E+01 -0.5449E+01  0.1377E+01  0.1098E-03  0.5129E-03  0.2576E-04
O^      -0.1759E+01  0.2460E+01  0.3320E+01 -0.7676E-04 -0.2161E-03 -0.1200E-03
H^      -0.2739E+01  0.2657E+01  0.3343E+01 -0.8392E-04 -0.2861E-03 -0.1401E-03
H^      -0.1350E+01  0.2896E+01  0.2518E+01  0.1772E-03  0.9790E-03  0.6584E-03
O^      -0.3269E+01  0.3863E+01 -0.6091E+01 -0.1391E-03  0.5335E-04  0.6947E-04
H^      -0.3641E+01  0.4123E+01 -0.6982E+01 -0.1000E-02  0.3166E-03  0.5059E-03
H^      -0.2735E+01  0.4622E+01 -0.5719E+01 -0.3032E-03  0.1718E-03  0.6328E-04
O^      -0.1995E+01 -0.3980E+01  0.5500E+01 -0.2062E-03  0.5484E-04 -0.1547E-03
H^      -0.1956E+01 -0.3408E+01  0.6320E+01  0.2765E-03 -0.4418E-03  0.1804E-03
H^      -0.2256E+01 -0.3419E+01  0.4714E+01 -0.9111E-03  0.4974E-03  0.3954E-03
O^       0.4884E+01  0.6075E+01 -0.3080E+01  0.7152E-05  0.2327E-03 -0.1652E-05
H^       0.5552E+01  0.5525E+01 -0.2578E+01 -0.1376E-04 -0.1713E-03 -0.4170E-03
H^       0.5056E+01  0.7045E+01 -0.2906E+01  0.7043E-03  0.1266E-03 -0.1535E-03
O^       0.2615E+01 -0.1474E+01  0.4779E+01  0.2480E-03  0.2773E-03  0.6476E-04
H^       0.3294E+01 -0.7726E+00  0.4997E+01  0.6977E-03 -0.5022E-03  0.1176E-02
H^       0.2248E+01 -0.1858E+01  0.5626E+01 -0.5883E-03 -0.2609E-03 -0.5399E-03
O^      -0.3116E+01 -0.1271E+01  0.5862E+01  0.4484E-04 -0.4921E-03 -0.2958E-03
H^      -0.3824E+01 -0.1292E+01  0.6568E+01  0.4160E-03 -0.1014E-02  0.6119E-04
H^      -0.3115E+01 -0.3749E+00  0.5417E+01  0.2101E-03  0.3728E-04  0.7720E-03
O^      -0.3732E+01  0.4587E+00 -0.1024E+01  0.1530E-04  0.1409E-03  0.2144E-03
H^      -0.3291E+01  0.6710E+00 -0.1521E+00  0.4425E-03  0.2528E-04  0.2437E-04
H^      -0.3039E+01  0.3972E+00 -0.1742E+01 -0.2884E-03 -0.8982E-04 -0.6985E-04
O^      -0.6022E+01 -0.2054E+01  0.9880E+00  0.1148E-03  0.2240E-04 -0.6860E-04
H^      -0.5613E+01 -0.2820E+01  0.1483E+01 -0.8247E-03 -0.3674E-03  0.1037E-03
H^      -0.6561E+01 -0.1498E+01  0.1621E+01 -0.9049E-03 -0.3507E-03 -0.6107E-03
O^       0.5388E+01 -0.6596E-01  0.2375E+01  0.1300E-03  0.7636E-04  0.9649E-04
H^       0.4393E+01  0.2216E-01  0.2323E+01  0.8852E-05 -0.9209E-03  0.5667E-03
H^       0.5638E+01 -0.5050E+00  0.3238E+01  0.9431E-03 -0.1611E-03 -0.2704E-03
O^      -0.3777E+00 -0.3378E+01  0.1384E+01 -0.1187E-03  0.6597E-04 -0.9305E-04
H^       0.3137E+00 -0.3344E+01  0.2106E+01  0.6941E-04 -0.1725E-03 -0.2510E-03
H^      -0.2333E+00 -0.2620E+01  0.7483E+00 -0.1763E-03  0.1878E-03  0.5338E-04
O^      -0.5167E+01  0.9137E-01  0.4518E+01 -0.7764E-04 -0.2549E-04  0.4651E-03
H^      -0.4490E+01  0.2494E+00  0.3799E+01 -0.5003E-03 -0.3149E-03  0.3790E-05
H^      -0.5695E+01  0.9276E+00  0.4669E+01 -0.8687E-03 -0.3647E-03 -0.4836E-03
end
\end{verbatim}
\normalsize
\begin{verbatim}
#set up the QM/MM simulation and cell
NWPW
   SIMULATION_CELL
      SC 23.577
   END
   QMMM
      lj_ion_parameters C  3.41000000d0 0.10000000d0
      lj_ion_parameters Cl  3.45000000d0 0.16d0
      lj_ion_parameters O^ 3.16555789d0 0.15539425d0

      # new input format
      fragment spc
         size 3
         index_start 6:195:3
         shake units angstroms 1 2 3 cyclic 1.0 1.632993125 1.0
      end
   END
END

#***** Setup conjugate gradient code ****
nwpw
   cutoff 25.0
   xc pbe96
   lmbfgs
   ewald_ncut 1
end
set nwpw:lcao_skip .true.
task pspw energy


#***** Setup Car-Parrinello code ****
nwpw
   car-parrinello
      Nose-Hoover 1200.0 300.0 1200.0 300.0
      time_step 5.00
      fake_mass 750.0
      loop 10 2000
      xyz_filename         ccl4.00.xyz
      ion_motion_filename  ccl4.00.ion_motion
      emotion_filename     ccl4.00.emotion
    end
end
task pspw car-parrinello

\end{verbatim}




\normalsize
\section{Band Tutorial 1: Minimizing the energy of a silicon-carbide crystal by running a PSPW and Band simulation in tandem}
\label{sec:band_tutorial1}
\normalsize

The following input deck performs a PSPW energy calculation followed
by a Band energy calculation at the $\Gamma$-point  for a cubic (8-atom) 
silicon-carbide crystal.  Since the geometry is entered using fractional coordinates
the unit cell parameters do not have to be re-specified in the simulation\_cell
nwpw sub-block.  In this example, default pseudopotential from the pseudopotential
library are used for C and Si.  The advantage of running these calculations in tandem is that
the Band code uses the wavefunctions generated from the faster PSPW calculation for
its initial guess.  The PSPW energy is -38.353570, and the Band energy is -38.353570.
 
\begin{verbatim}
start SiC_band
title "SiC 8 atom cubic cell"

#**** geometry entered using fractional coordinates ****
geometry units au center noautosym noautoz print 
  system crystal 
    lat_a 8.277d0
    lat_b 8.277d0
    lat_c 8.277d0
    alpha 90.0d0
    beta  90.0d0
    gamma 90.0d0
  end
Si    -0.50000d0  -0.50000d0  -0.50000d0
Si     0.00000d0   0.00000d0  -0.50000d0
Si     0.00000d0  -0.50000d0   0.00000d0
Si    -0.50000d0   0.00000d0   0.00000d0
C     -0.25000d0  -0.25000d0  -0.25000d0
C      0.25000d0   0.25000d0  -0.25000d0
C      0.25000d0  -0.25000d0   0.25000d0
C     -0.25000d0   0.25000d0   0.25000d0
end

#***** setup the nwpw gamma point code ****
nwpw
   simulation_cell
     ngrid 16 16 16
   end
   brillouin_zone
     kvector  0.0 0.0 0.0
   end
   ewald_ncut 8
end
set nwpw:minimizer 2
set nwpw:psi_brillioun_check .false.
task pspw energy
task band energy
\end{verbatim}




\normalsize
\section{BAND Tutorial 2: optimizing a unit cell and geometry for Silicon-Carbide}
\label{sec:band_unitcell_optimization}

The following example demonstrates how to uses the BAND module to optimize the unit cell
and geometry for a silicon-carbide crystal.

\begin{verbatim}
title "SiC 8 atom cubic cell - geometry and unit cell optimization"

start SiC

#**** Enter the geometry using fractional coordinates ****
geometry units au center noautosym noautoz print
  system crystal
    lat_a 8.277d0
    lat_b 8.277d0
    lat_c 8.277d0
    alpha 90.0d0
    beta  90.0d0
    gamma 90.0d0
  end
Si    -0.50000d0  -0.50000d0  -0.50000d0
Si     0.00000d0   0.00000d0  -0.50000d0
Si     0.00000d0  -0.50000d0   0.00000d0
Si    -0.50000d0   0.00000d0   0.00000d0
C     -0.25000d0  -0.25000d0  -0.25000d0
C      0.25000d0   0.25000d0  -0.25000d0
C      0.25000d0  -0.25000d0   0.25000d0
C     -0.25000d0   0.25000d0   0.25000d0
end

#***** setup the nwpw gamma point code ****
nwpw
   simulation_cell
     ngrid 16 16 16
   end
   ewald_ncut 8
   monkhorst-pack 2 2 2 
   lmbfgs
end

driver
  clear
  maxiter 40
end
set includestress .true.          # this option tells driver to optimize the unit cell
#set nwpw:stress_numerical .true.  #option to use numerical stresses 

task band optimize

\end{verbatim}


\normalsize
\section{BAND Tutorial 3: optimizing a unit cell and geometry for Aluminum with fractional occupation}
\label{sec:band_unitcell_optimization}

The following example demonstrates how to uses the BAND module to optimize the unit cell
and geometry for a Aluminum.

\begin{verbatim}
title "Aluminum optimization with fractional occupation"

start aluminumfrac

memory 900 mb

geometry noautoz
   system crystal
     lat_a 3.0
     lat_b 3.0
     lat_c 3.0
     alpha 90.0
     beta  90.0
     gamma 90.0
   end
Al  0.0 0.0 0.0
Al  0.0 0.5 0.5
Al  0.5 0.5 0.0
Al  0.5 0.0 0.5
end
set nwpw:cif_filename aluminum

nwpw
   scf anderson
   mult 1
   smear  temperature 3500.0 fermi
   cutoff 15.0
   monkhorst-pack 3 3 3
   ewald_ncut 8
   mapping 2
end
set nwpw:lcao_skip .true.

set includestress .true.
#set nwpw:stress_numerical .true.

driver
   clear
end
task band optimize ignore

\end{verbatim}



\section{PAW Tutorial}
\label{sec:paw_tutorial}

The following input deck performs for a water molecule a PSPW energy calculation followed
by a PAW energy calculation and a PAW geometry optimization calculation.  
The default unit cell parameters are used (SC=20.0, ngrid 32 32 32).  In this simulation, the
first PAW run optimizes the wavefunction and the second PAW run optimizes the wavefunction
and geometry in tandem.

\begin{verbatim}
title "paw steepest descent test"

start paw_test

charge 0

geometry units au nocenter noautoz noautosym
O      0.00000    0.00000    0.01390
H     -1.49490    0.00000   -1.18710
H      1.49490    0.00000   -1.18710
end

nwpw
   time_step 15.8
   ewald_rcut 1.50
   tolerances 1.0d-8 1.0d-8
end
set nwpw:lcao_iterations 1
set nwpw:minimizer 2
task pspw energy

task paw energy

nwpw
   time_step 5.8
   geometry_optimize
   ewald_rcut 1.50
   tolerances 1.0d-7 1.0d-7 1.0d-4
end
task paw steepest_descent

task paw optimize
\end{verbatim}




\normalsize
\section{PAW Tutorial 2: optimizing a unit cell and geometry for Silicon-Carbide}
\label{sec:pspw_unitcell_optimization}

The following example demonstrates how to uses the PAW module to optimize the unit cell
and geometry for a silicon-carbide crystal.

\begin{verbatim}
title "SiC 8 atom cubic cell - geometry and unit cell optimization"

start SiC

#**** Enter the geometry using fractional coordinates ****
geometry units au center noautosym noautoz print
  system crystal
    lat_a 8.277d0
    lat_b 8.277d0
    lat_c 8.277d0
    alpha 90.0d0
    beta  90.0d0
    gamma 90.0d0
  end
Si    -0.50000d0  -0.50000d0  -0.50000d0
Si     0.00000d0   0.00000d0  -0.50000d0
Si     0.00000d0  -0.50000d0   0.00000d0
Si    -0.50000d0   0.00000d0   0.00000d0
C     -0.25000d0  -0.25000d0  -0.25000d0
C      0.25000d0   0.25000d0  -0.25000d0
C      0.25000d0  -0.25000d0   0.25000d0
C     -0.25000d0   0.25000d0   0.25000d0
end

#***** setup the nwpw gamma point code ****
nwpw
   simulation_cell
     ngrid 16 16 16
   end
   ewald_ncut 8
end
set nwpw:minimizer 2
set nwpw:psi_nolattice .true.  # turns of unit cell checking for wavefunctions

driver
  clear
  maxiter 40
end
set includestress .true.         # this option tells driver to optimize the unit cell
set nwpw:stress_numerical .true. #currently only numerical stresses implemented in paw

task paw optimize

\end{verbatim}



\normalsize
\section{PAW Tutorial 2: Running a Car-Parrinello Simulation}
\label{sec:pspw_cp}
\normalsize

In this section we show how use the PAW module to perform a Car-Parrinello
molecular dynamic simulation for a C$_2$ molecule at the LDA level.  
Before running a PAW Car-Parrinello  simulation the system should be
on the Born-Oppenheimer surface, i.e. the one-electron orbitals should be minimized 
with respect to the total energy (i.e. task pspw energy).  The input needed
is basically the same as for optimizing the geometry of a C$_2$ molecule at the LDA level,
except that and additional Car-Parrinello sub-block is added.  

In the following example we show the input needed to run a Car-Parrinello simulation
for a C$_2$ molecule at the LDA level.  In this example, default pseudopotentials
from the pseudopotential library are used for C, the boundary condition is free-space, 
the exchange correlation functional is LDA, The boundary condition is free-space, and 
the simulation cell cell is aperiodic and cubic with a side length of 10.0 Angstroms and has
40 grid points in each direction (cutoff energy is 44 Ry).  The time step and fake mass
for the Car-Parrinello run are specified to be 5.0 au and 600.0 au, respectively.  

\begin{verbatim}
         
start c2_paw_lda_md
title "C2 restricted singlet dimer, LDA/44Ry - constant energy Car-Parrinello simulation"

geometry  
C    -0.62 0.0 0.0
C     0.62 0.0 0.0
end
       
pspw
   simulation_cell units angstroms
      boundary_conditions aperiodic
      lattice
        lat_a 10.00d0
        lat_b 10.00d0
        lat_c 10.00d0
      end
      ngrid 40 40 40
   end
   Car-Parrinello
     fake_mass 600.0
     time_step 5.0
     loop 10 10
   end
end
set nwpw:minimizer 2
task paw energy
task paw Car-Parrinello
\end{verbatim}



\section{NWPW Capabilities and Limitations}
\label{sec:pspw_limits}
\normalsize

\begin{itemize}
\item For BAND simulations you cannot use more processors than the size of the third dimension 
  (e.g. a 64x64x64 FFT grid can use at most 64 processors).
\item For BAND simulations the second and third dimensions of the FFT grid must be the same 
  (i.e. the parameters na2 and na3 must be the same for each simulation cell).
\item Wannier orbital generation only works with cubic unit cells ($\alpha=\beta=\gamma=90^o$)
\end{itemize}


\section{Questions and Difficulties}
\normalsize

Questions and encountered problems should be reported to 
nwchem-users@emsl.pnl.gov 
or to Eric J. Bylaska, Eric.Bylaska@pnl.gov





% LocalWords:  tolc wcut ncut


\chapter{Controlling NWChem with Python}
%
% $Id: python.tex,v 1.8 2004-04-22 04:50:29 edo Exp $
%
\label{sec:python}

Python (version 1.5.1) programs may be embedded into the NWChem input
and used to control the execution of NWChem.  Python is a very
powerful and widely used scripting language that provides useful
things such as variables, conditional branches and loops, and is also
readily extended.  Example applications include scanning potential
energy surfaces, computing properties in a variety of basis sets,
optimizing the energy w.r.t. parameters in the basis set, computing
polarizabilities with finite field, and simple molecular dynamics.

Look in the NWChem \verb+contrib+ directory for useful scripts and
examples. Visit the Python web-site 
\htmladdnormallink{http://www.python.org}{http://www.python.org}
for a full manual and lots of useful code and resources.  

\section{How to input and run a Python program inside NWChem}

A Python program is input into NWChem inside a Python compound directive.
\begin{verbatim}
  python [print|noprint]
    ...
  end
\end{verbatim}
The \verb+END+ directive must be flush against the left
margin (see the Troubleshooting section for the reason why).

The program is by default printed to standard output when read, but
this may be disabled with the \verb+noprint+ keyword.  Python uses
indentation to indicate scope (and the initial level of indentation
must be zero), whereas NWChem uses optional indentation only to make
the input more readable.  For example, in Python, the contents of a
loop, or conditionally-executed block of code must be indented further
than the surrounding code.  Also, Python attaches special meaning to
several symbols also used by NWChem.  For these reasons, the input
inside a \verb+PYTHON+ compound directive is read verbatim except that
if the first line of the Python program is indented, the same amount
of indentation is removed from all subsequent lines.  This is so that
a program may be indented inside the \verb+PYTHON+ input block for
improved readability of the NWChem input, while satisfying the
constraint that when given to Python the first line has zero
indentation.

E.g., the following two sets of input specify the same Python program.
\begin{verbatim}
  python
    print 'Hello'
    print 'Goodbye'
  end

  python
  print 'Hello'
  print 'Goodbye'
  end
\end{verbatim}
whereas this program is in error since the indentation of the second
line is less than that of the first.
\begin{verbatim}
  python
    print 'Hello'
  print 'Goodbye'
  end
\end{verbatim}

The Python program is not executed until the following directive
is encountered
\begin{verbatim}
  task python
\end{verbatim}
which is to maintain consistency with the behavior of NWChem in general.
{\em The program is executed by all nodes.}  This enables the full functionality and speed of NWChem to be accessible from Python, but there are some gotchas
\begin{itemize}
\item Print statements and other output will be executed by all nodes
so you will get a lot more output than probably desired unless the
output is restricted to just one node (by convention node zero).
\item The calls to NWChem functions are all collective (i.e., all
nodes must execute them).  If these calls are not made collectively
your program may deadlock (i.e., cease to make progress).

\item When writing to the database (\verb+rtdb_put()+) it is the data
from node zero that is written.

\item NWChem overrides certain default signal handlers so care
must be taken when creating processes (see Section \ref{sec:sigchld}).
\end{itemize}

\section{NWChem extensions}

Since we have little experience using Python, the NWChem-Python
interface might change in a non-backwardly compatible fashion as we
discover better ways of providing useful functionality.  We would
appreciate suggestions about useful things that can be added to the
NWChem-Python interface.  In principle, nearly any Fortran or C
routine within NWChem can be extended to Python, but we are also
interested in ideas that will enable users to build completely new
things.  For instance, how about being able to define your own energy
functions that can be used with the existing optimizers or dynamics
package?

Python has been extended with a module named \verb+"nwchem"+ which is
automatically imported and contains the following NWChem-specific
commands.  They all handle NWChem-related errors by raising the
exception \verb+"NWChemError"+, which may be handled in the standard
Python manner (see Section \ref{sec:pyerr}). 
\begin{itemize}
\item \verb+input_parse(string)+ --- invokes the standard NWChem input
parser with the data in \verb+string+ as input.  Note that the usual
behavior of NWChem will apply --- the parser only reads input up to
either end of input or until a \verb+TASK+ directive is encountered
(the task directive is {\em not} executed by the parser).

\item \verb+task_energy(theory)+ --- returns the energy as if computed
with the NWChem directive \verb+TASK ENERGY <THEORY>+.

\item \verb+task_gradient(theory)+ --- returns a tuple
\verb+(energy,gradient)+ as if computed with the NWChem
directive \verb+TASK GRADIENT <THEORY>+.

\item \verb+task_optimize(theory)+ --- returns a tuple
\verb+(energy,gradient)+ as if computed with the NWChem
directive \verb+TASK OPTIMIZE <THEORY>+.  The energy and gradient
will be those at the last point in the optimization and consistent
with the current geometry in the database.

\item \verb+ga_nodeid()+ --- returns the number of the parallel
process.

\item \verb+rtdb_print(print_values)+ --- prints the contents of the
RTDB.  If \verb+print_values+ is 0, only the keys are printed, if it
is 1 then the values are also printed.

\item \verb+rtdb_put(name, values)+ or
\verb+rtdb_put(name, values, type)+ --- puts the values into the
database with the given name.  In the first form, the type is inferred
from the first value, and in the second form the type is specified
using the last argument as one of \verb+INT+, \verb+DBL+,
\verb+LOGICAL+, or \verb+CHAR+.

\item \verb+rtdb_get(name) + --- returns the data from the database
associated with the given name.
\end{itemize}

An example below (Section \ref{sec:pygeom}) explains, in lieu of a
Python wrapper for the geometry object, how to obtain the Cartesian
molecular coordinates directly from the database.

\section{Examples}

Several examples will provide the best explanation of how the extensions
are used, and how Python might prove useful.

\subsection{Hello world}

\begin{verbatim}
  python
    print 'Hello world from process ', ga_nodeid()
  end

  task python
\end{verbatim}

This input prints the traditional greeting from each parallel process.

\subsection{Scanning a basis exponent}
\begin{verbatim}
  geometry units au
    O 0 0 0; H 0 1.430 -1.107; H 0 -1.430 -1.107
  end

  python
    exponent = 0.1
    while (exponent <= 2.01):
       input_parse('''
          basis noprint
             H library 3-21g; O library 3-21g; O d; %f 1.0
          end
       ''' % (exponent))
       print ' exponent = ', exponent, ' energy = ', task_energy('scf')
       exponent = exponent + 0.1
  end

  print none

  task python
\end{verbatim}

This program augments a 3-21g basis for water with a d-function on
oxygen and varies the exponent from 0.1 to 2.0 in steps of 0.1,
printing the exponent and energy at each step.  

The geometry is input as usual, but the basis set input is embedded
inside a call to \verb+input_parse()+ in the Python program.  The
standard Python string substitution is used to put the current value of
the exponent into the basis set (replacing the \verb+%f+) before being
parsed by NWChem.  The energy is returned by \verb+task_energy('scf')+
and printed out.  The \verb+print none+ in the NWChem input switches
off all NWChem output so all you will see is the output from your
Python program.

Note that execution in parallel may produce unwanted output since
all process execute the print statement inside the Python program.

Look in the NWChem \verb+contrib+ directory for a routine that makes
the above task easier.

\subsection{Scanning a basis exponent revisited.}
\label{sec:scan2}

\begin{verbatim}
  geometry units au
    O 0 0 0; H 0 1.430 -1.107; H 0 -1.430 -1.107
  end

  print none

  python
    if (ga_nodeid() == 0): plotdata = open("plotdata",'w')

    def energy_at_exponent(exponent):
       input_parse('''
          basis noprint
             H library 3-21g; O library 3-21g; O d; %f 1.0
          end
       ''' % (exponent))
       return task_energy('scf')

    exponent = 0.1
    while exponent <= 2.01:
       energy = energy_at_exponent(exponent)
       if (ga_nodeid() == 0):
          print ' exponent = ', exponent, ' energy = ', energy
          plotdata.write('%f %f\n' % (exponent , energy))
       exponent = exponent + 0.1

    if (ga_nodeid() == 0): plotdata.close()
  end

  task python
\end{verbatim}

This input performs exactly the same calculation as the previous one,
but uses a slightly more sophisticated Python program, also writes
the data out to a file for easy visualization with a package such as
\verb+gnuplot+, and protects write statements to prevent 
duplicate output in a parallel job.  The only significant differences
are in the Python program.  A file called \verb+"plotdata"+ is opened,
and then a procedure is defined which given an exponent returns the
energy.  Next comes the main loop that scans the exponent through the
desired range and prints the results to standard output and to the
file.  When the loop is finished the additional output file is closed.

\subsection{Scanning a geometric variable}

\begin{verbatim}
  python
    geometry = '''
      geometry noprint; symmetry d2h
         C 0 0 %f; H 0  0.916 1.224
      end
    '''
    x = 0.6
    while (x < 0.721):
      input_parse(geometry % x)
      energy = task_energy('scf')
      print ' x = %5.2f   energy = %10.6f' % (x, energy)
      x = x + 0.01
  end

  basis; C library 6-31g*; H library 6-31g*; end

  print none

  task python
\end{verbatim}

This scans the bond length in ethene from 1.2 to 1.44 in steps 
of 0.2 computing the energy at each geometry.  Since it is using 
$D_{2h}$ symmetry the program actually uses a variable (\verb+x+) that is
half the bond length.

Look in the NWChem \verb+contrib+ directory for a routine that makes
the above task easier.

\subsection{Scan using the BSSE counterpoise corrected energy}

\begin{verbatim}
  basis spherical
    Ne library cc-pvdz; BqNe library Ne cc-pvdz
    He library cc-pvdz; BqHe library He cc-pvdz
  end

  mp2; tight; freeze core atomic; end

  print none

  python noprint
    supermolecule = 'geometry noprint;   Ne 0 0 0;   He 0 0 %f; end\n'
    fragment1     = 'geometry noprint;   Ne 0 0 0; BqHe 0 0 %f; end\n'
    fragment2     = 'geometry noprint; BqNe 0 0 0;   He 0 0 %f; end\n'

    def energy(geometry):
      input_parse(geometry + 'scf; vectors atomic; end\n')
      return task_energy('mp2')

    def bsse_energy(z):
      return energy(supermolecule % z) - \
             energy(fragment1 % z) - \
             energy(fragment2 % z)
    z = 3.3
    while (z < 4.301):
      e = bsse_energy(z)
      if (ga_nodeid() == 0):
        print ' z = %5.2f   energy = %10.7f ' % (z, e)
      z = z + 0.1
  end

  task python
\end{verbatim}

This example scans the He---Ne bond-length from 3.3 to 4.3 and prints out
the BSSE counterpoise corrected MP2 energy.

The basis set is specified as usual, noting that we will need
functions on ghost centers to do the counterpoise correction.  The
Python program commences by defining strings containing the geometry
of the super-molecule and two fragments, each having one variable to be
substituted.  Next, a function is defined to compute the energy given
a geometry, and then a function is defined to compute the counterpoise
corrected energy at a given bond length.  Finally, the bond length is
scanned and the energy printed.  When computing the energy, the atomic
guess has to be forced in the SCF since by default it will attempt to
use orbitals from the previous calculation which is not appropriate
here.

Since the counterpoise corrected energy is a linear combination of
other standard energies, it is possible to compute the analytic
derivatives term by term.  Thus, combining this example and the next
could yield the foundation of a BSSE corrected geometry optimization
package.

\subsection{Scan the geometry and compute the energy and gradient}

\begin{verbatim}
  basis noprint; H library sto-3g; O library sto-3g; end

  python noprint
    print '   y     z     energy                gradient'
    print ' ----- ----- ---------- ------------------------------------'
    y = 1.2
    while y <= 1.61:
       z = 1.0
       while z <= 1.21:
          input_parse('''
             geometry noprint units atomic
                O 0   0   0
                H 0  %f -%f
                H 0 -%f -%f
             end
          ''' % (y, z, y, z))

          (energy,gradient) = task_gradient('scf')

          print ' %5.2f %5.2f %9.6f' % (y, z, energy),
          i = 0
          while (i < len(gradient)):
             print '%5.2f' % gradient[i],
             i = i + 1
          print ''
          z = z + 0.1
       y = y + 0.1
  end

  print none

  task python
\end{verbatim}

This program illustrates evaluating the energy and gradient
by calling \verb+task_gradient()+.  A water molecule is scanned
through several $C_{2v}$ geometries by varying the y and z coordinates
of the two hydrogen atoms.  At each geometry the coordinates, energy 
and gradient are printed.

The basis set (sto-3g) is input as usual.  The two while loops vary
the y and z coordinates.  These are then substituted into a geometry
which is parsed by NWChem using \verb+input_parse()+.  The energy and
gradient are then evaluated by calling \verb+task_gradient()+ which
returns a tuple containing the energy (a scalar) and the gradient (a
vector or list).  These are printed out exploiting the Python
convention that a print statement ending in a comma does not print
end-of-line.

\subsection{Reaction energies varying the basis set}

\begin{verbatim}
  mp2; freeze atomic; end

  print none

  python
    energies = {}
    c2h4 = 'geometry noprint; symmetry d2h; \
            C 0 0 0.672; H 0 0.935 1.238; end\n'
    ch4  = 'geometry noprint; symmetry td; \
            C 0 0 0; H 0.634 0.634 0.634; end\n'
    h2   = 'geometry noprint; H 0 0 0.378; H 0 0 -0.378; end\n'

    def energy(basis, geometry):
      input_parse('''
        basis spherical noprint
          c library %s ; h library %s 
        end
      ''' % (basis, basis))
      input_parse(geometry)
      return task_energy('mp2')

    for basis in ('sto-3g', '6-31g', '6-31g*', 'cc-pvdz', 'cc-pvtz'):
       energies[basis] =   2*energy(basis, ch4) \
                         - 2*energy(basis, h2) - energy(basis, c2h4)
       if (ga_nodeid() == 0): print basis, ' %8.6f' % energies[basis]
  end 

  task python
\end{verbatim}

In this example the reaction energy for 
$2H_2 + C_2H_4 \rightarrow 2CH_4$ is evaluated using MP2 in several
basis sets.  The geometries are fixed, but could be re-optimized in
each basis.  To illustrate the useful associative arrays in Python,
the reaction energies are put into the associative array
\verb+energies+ --- note its declaration at the top of the program.

\subsection{Using the database}

\begin{verbatim}
  python
    rtdb_put("test_int2", 22)
    rtdb_put("test_int", [22, 10, 3],    INT)
    rtdb_put("test_dbl", [22.9, 12.4, 23.908],  DBL)
    rtdb_put("test_str", "hello", CHAR)
    rtdb_put("test_logic", [0,1,0,1,0,1], LOGICAL)
    rtdb_put("test_logic2", 0, LOGICAL)

    rtdb_print(1)

    print "test_str    = ", rtdb_get("test_str")
    print "test_int    = ", rtdb_get("test_int")
    print "test_in2    = ", rtdb_get("test_int2")
    print "test_dbl    = ", rtdb_get("test_dbl")
    print "test_logic  = ", rtdb_get("test_logic")
    print "test_logic2 = ", rtdb_get("test_logic2")
  end

  task python
\end{verbatim}

This example illustrates how to access the database from Python.

\subsection{Handling exceptions from NWChem}
\label{sec:pyerr}

\begin{verbatim}
  geometry; he 0 0 0; he 0 0 2; end
  basis; he library 3-21g; end
  scf; maxiter 1; end

  python
    try:
      task_energy('scf')
    except NWChemError, message:
      print 'Error from NWChem ... ', message
  end

  task python
\end{verbatim}

The above test program shows how to handle exceptions generated by
NWChem by forcing an SCF calculation on $He_2$ to fail due to
insufficient iterations.

If an NWChem command fails it will raise the exception
\verb+"NWChemError"+ (case sensitive) unless the error was fatal.
If the exception is not caught, then it will cause the entire Python
program to terminate with an error.  This Python program catches the
exception, prints out the message, and then continues as if all was
well since the exception has been handled.  

If your Python program detects an error, raise an unhandled
exception.  Do not call \verb+exit(1)+ since this may circumvent
necessary clean-up of the NWChem execution environment.

\subsection{Accessing geometry information --- a temporary hack}
\label{sec:pygeom}

In an ideal world the geometry and basis set objects would have full
Python wrappers, but until then a back-door solution will have to
suffice.  We've already seen how to use \verb+input_parse()+ to put
geometry (and basis) data into NWChem, so it only remains to get the
geometry data back after it has been updated by a geometry optimzation
or some other operation.  

The following Python procedure retrieves the coordinates in the
same units as initially input for a geometry of a given name.
Its full source is included in the NWChem \verb+contrib+ directory.
\begin{verbatim}
  def geom_get_coords(name):
    try:
      actualname = rtdb_get(name)
    except NWChemError:
      actualname = name
    coords = rtdb_get('geometry:' + actualname + ':coords')
    units  = rtdb_get('geometry:' + actualname + ':user units')
    if (units == 'a.u.'):
      factor = 1.0
    elif (units == 'angstroms'):
      factor = rtdb_get('geometry:'+actualname+':angstrom_to_au')
    else:
      raise NWChemError,'unknown units'
    i = 0
    while (i < len(coords)):
      coords[i] = coords[i] / factor
      i = i + 1
    return coords
\end{verbatim}

A geometry (see Section \ref{sec:geom}) with name \verb+NAME+ has its
coordinates (in atomic units) stored in the database entry
\verb+geometry:NAME:coords+.  A minor wrinkle here is that 
indirection is possible (and used by the optimizers) so that we must
first check if \verb+NAME+ actually points to another name.  In the
program this is done in the first \verb+try...except+ sequence.  With
the actual name of the geometry, we can get the coordinates.  Any
exceptions are passed up to the caller.  The rest of the code is just
to convert back into the initial input units --- only atomic units 
or \angstroms\ are handled in this simple example.  Returned 
is a list of the atomic coordinates in the same units as your
initial input.

The routine is used as follows
\begin{verbatim}
    coords = geom_get_coords('geometry')
\end{verbatim}
or, if you want better error handling 
\begin{verbatim}
    try:
      coords = geom_get_coords('geometry')
    except NWChemError,message:
      print 'Coordinates for geometry not found ', message
    else:
      print coords    
\end{verbatim}

This is very dirty and definitely not supported from one release to
another, but, browsing the output of \verb+rtdb_print()+ at the end of
a calculation is a good way to find stuff.  To be on safer ground,
look in the programmers manual since some of the high-level routines
do pass data via the database in a well-defined and supported manner.
{\em Be warned} --- you must be very careful if you try to modify data
in the database. The input parser does many important things that are
not immediately apparent (e.g., ensure the geometry is consistent with
the point group, mark the SCF as not converged if the SCF options are
changed, \ldots).  Where at all possible your Python program should
generate standard NWChem input and pass it to \verb+input_parse()+
rather than setting parameters directly in the database.

\subsection{Scaning a basis exponent yet again --- plotting and 
handling child processes}
\label{sec:sigchld}

\begin{verbatim}
  geometry units au
    O 0 0 0; H 0 1.430 -1.107; H 0 -1.430 -1.107
  end

  print none

  python
    import Gnuplot, time, signal

    def energy_at_exponent(exponent):
       input_parse('''
          basis noprint
             H library 3-21g; O library 3-21g; O d; %f 1.0
          end
       ''' % (exponent))
       return task_energy('scf')

    data = []
    exponent = 0.5
    while exponent <= 0.6:
       energy = energy_at_exponent(exponent)
       print ' exponent = ', exponent, ' energy = ', energy
       data = data + [[exponent,energy]]
       exponent = exponent + 0.02

    if (ga_nodeid() == 0):
       signal.signal(signal.SIGCHLD, signal.SIG_DFL)
       g = Gnuplot.Gnuplot()
       g('set data style linespoints')
       g.plot(data)
       time.sleep(30)  # 30s to look at the plot

  end

  task python
\end{verbatim}

This illustrates how to handle signals from terminating child
processes and how to generate simple plots on UNIX systems.  The
example from Section \ref{sec:scan2} is modified so that instead of
writing the data to a file for subsequent visualization, it is saved
for subsequent visualization with Gnuplot (you'll need both Gnuplot
and the corresponding package for Python in your \verb+PYTHONPATH+.
Look at \htmladdnormallink{http://monsoon.harvard.edu/~mhagger/download}{http://monsoon.harvard.edu/~mhagger/download}).

The issue is that NWChem traps various signals from the O/S that
usually indicate bad news in order to provide better error handling
and reliable clean-up of shared, parallel resources.  One of these
signals is \verb+SIGCHLD+ which is generated whenever a child process
terminates.  If you want to create child processes within Python, then
the NWChem handler for \verb+SIGCHLD+ must be replaced with the
default handler.  There seems to be no easy way to restore the
NWChem handler after the child has completed, but this should have
no serious side effect.

\section{Troubleshooting}

Common problems with Python programs inside NWChem.

\begin{enumerate}
\item You get the message
\begin{verbatim}
      0:python_input: indentation must be >= that of first line: 4
\end{verbatim}
This indicates that NWChem thinks that a line is less indented than
the first line.  If this is not the case then perhaps there is a tab
in your input which NWChem treats as a single space character but
appears to you as more spaces. Try running \verb+untabify+ in Emacs.
The problem could also be the \verb+END+ directive that terminates the
\verb+PYTHON+ compound directive --- since Python also has an
\verb+end+ statement.  To avoid confusion the \verb+END+ directive
for NWChem {\em must} be at the start of the line.

\item Your program hangs or deadlocks --- most likely you have a piece
of code that is restricted to executing on a subset of the processors
(perhaps just node 0) but is calling (perhaps indirectly) a function
that must execute on all nodes.  

\end{enumerate}


\clearpage
\chapter{Acknowledgments}

This work was supported by funds from the Environmental and Molecular
Sciences Laboratory Construction Project at Pacific Northwest National
Laboratory.  Development of some of the parallel programming tools and
algorithms employed by NWChem was performed under the auspices of the
High Performance Computing and Communications Program of the
Mathematical, Information, and Computational Sciences Division, U.S.\
Department of Energy.  Pacific Northwest National Laboratory is
operated by Battelle Memorial Institute for the U.S.\ Department of
Energy under Contract DE-AC06-76RLO 1830.


\clearpage

\appendix

\chapter{Standard Basis Sets}
\label{sec:knownbasis}

Basis sets and effective core potentials were obtained (7/6/2000) from
the Extensible Computational Chemistry Environment (ECCE) Basis Set
Database, as developed and distributed by the Molecular Science
Computing Facility, Environmental and Molecular Sciences Laboratory
which is part of the Pacific Northwest National Laboratory, P.O. Box
999, Richland, Washington 99352, USA, and is funded by the
U.S. Department of Energy.  The Pacific Northwest National Laboratory
is a multi-program laboratory operated by Battelle Memorial Institute
for the U.S. Department of Energy under contract DE-AC06-76RLO.
Contact David Feller (\verb+df_feller@pnl.gov+) or Deborah Gracio
(\verb+gracio@pnl.gov+) for further information.

The names in the NWChem library are consistent with those in the ECCE
database and thus may include spaces.  The standard NWChem input
routines require that strings including spaces are enclosed in
quotation marks (\verb+"..."+) or that blanks are escaped with a
backslash.  As a convenience, basis set names may also have the blanks
replaced with underscores.  Thus, the following all yield the same
basis set for oxygen
\begin{verbatim}
  oxygen library "DZP + Diffuse (Dunning)"
  oxygen library DZP\ +\ Diffuse\ (Dunning)
  oxygen library DZP_+_Diffuse_(Dunning)
\end{verbatim}

Case may be ignored when specifying basis set names, but otherwise
names should be specified exactly as provided below.  A good method is
just to cut/paste from the WWW pages since they were generated
electronically from the library source.

Errors found in the basis set library of NWChem version 3.3 have been corrected in 
the current library of NWChem version 4.0. The changes are listed below:

\begin{enumerate}

\item Basis Set \verb#"Hay-Wadt VDZ (n+1) ECP"# \newline 
 Element: Ag \newline
 Correction of contraction coefficient in first s contraction.

\item Basis Set \verb#"LANL2DZ ECP"# \newline 
 Element: Ag \newline
 Correction of contraction coefficient in first s contraction.

\item Basis Set \verb#"CRENBS ECP"# \newline 
 Element: All elements in the \verb#"CRENBS ECP"# Family \newline
 Complete revision of basis sets. The original sets were completely uncontracted and
 some of the first row transition metals were missing all of their d functions. The
 current sets relflect the minimal contraction of the original papers.

\item ECP \verb#"SBKJC VDZ ECP"# \newline 
 Element: Ce \newline
 Correction of switched integer powers for d-f component.

\end{enumerate}

Relativistic contractions of standard basis sets for use in the Douglas-Kroll 
and Dyall-modified-Dirac method have also been included in the library. These
are identified by tags following the standard basis set name.  \newline

For the Douglas-Kroll method the tag \verb+dk+ should be specified: 

\begin{verbatim}
  oxygen library cc-pVDZ_DK
\end{verbatim}

For the Dyall-modified-Dirac (DMD) method three tags should be specified. The 
first is a tag for the nuclear model, which can be \verb+pt+ or \verb+fi+ for 
a point or a finite Gaussian model (see Section\ref{sec:geom}).  The second 
is a tag for the relativistic Hamiltonian: \verb+sf+ is for the spin-free 
modified Dirac Hamiltonian. The third tags the component type, \verb+fw+ for 
the atomic FW transformed large component, \verb+lc+ for the large component 
and \verb+sc+ for the small component: 
\begin{verbatim}
    oxygen library cc-pvdz_pt_sf_fw
    oxygen library cc-pvdz_pt_sf_lc
\end{verbatim} 
Basis sets which are available with either the DmD or DK contractions are
indicated in the list below.

\sloppy
Here is a list of known all-electron non-relativistic, DK and DmD basis sets, 
effective core potentials with their respective basis sets, and fitting basis sets 
along with the elements included for each. Additional information about each 
basis set in the NWChem library can be obtained from the online EMSL Gaussian 
Basis Set library (http://www.emsl.pnl.gov:2080/forms/basisform.html).

Standard all-electron basis sets:

\begin{enumerate}

\item Basis Set \verb#"STO-2G"# (number of atoms 20)  \newline 
  H He Li Be B C N O F Ne Na Mg Al Si P S Cl Ar K Ca


\item Basis Set \verb#"STO-3G"# (number of atoms 52)  \newline 
  H He Li Be B C N O F Ne Na Mg Al Si P S Cl Ar K Ca Sc Ti V Cr Mn
 Fe Co Ni Cu Zn Ga Ge As Se Br Kr Rb Sr Y Zr Nb Mo Tc Ru Rh Pd Ag Cd In Sn
 Sb Te


\item Basis Set \verb#"STO-6G"# (number of atoms 36)  \newline 
  H He Li Be B C N O F Ne Na Mg Al Si P S Cl Ar K Ca Sc Ti V Cr Mn
 Fe Co Ni Cu Zn Ga Ge As Se Br Kr


\item Basis Set \verb#"STO-3G*"# (number of atoms 18)  \newline 
  H He Li Be B C N O F Ne Na Mg Al Si P S Cl Ar


\item Basis Set \verb#"3-21G"# (number of atoms 55)  \newline 
  H He Li Be B C N O F Ne Na Mg Al Si P S Cl Ar K Ca Sc Ti V Cr Mn
 Fe Co Ni Cu Zn Ga Ge As Se Br Kr Rb Sr Y Zr Nb Mo Tc Ru Rh Pd Ag Cd In Sn
 Sb Te I Xe Cs


\item Basis Set \verb#"3-21++G"# (number of atoms 18)  \newline 
  H He Li Be B C N O F Ne Na Mg Al Si P S Cl Ar


\item Basis Set \verb#"3-21G*"# (number of atoms 17)  \newline 
  H He Li Be B C N O F Ne Na Mg Al Si P S Cl


\item Basis Set \verb#"3-21++G*"# (number of atoms 18)  \newline 
  H He Li Be B C N O F Ne Na Mg Al Si P S Cl Ar


\item Basis Set \verb#"3-21GSP"# (number of atoms 18)  \newline 
  H He Li Be B C N O F Ne Na Mg Al Si P S Cl Ar


\item Basis Set \verb#"4-22GSP"# (number of atoms 18)  \newline 
  H He Li Be B C N O F Ne Na Mg Al Si P S Cl Ar


\item Basis Set \verb#"4-31G"# (number of atoms 13)  \newline 
  H He Li Be B C N O F Ne P S Cl


\item Basis Set \verb#"6-31G"# (number of atoms 30)  \newline 
  H He Li Be B C N O F Ne Na Mg Al Si P S Cl Ar K Ca Sc Ti V Cr Mn
 Fe Co Ni Cu Zn


\item Basis Set \verb#"6-31++G"# (number of atoms 18)  \newline 
  H He Li Be B C N O F Ne Na Mg Al Si P S Cl Ar


\item Basis Set \verb#"6-31G*"# (number of atoms 30)  \newline 
  H He Li Be B C N O F Ne Na Mg Al Si P S Cl Ar K Ca Sc Ti V Cr Mn
 Fe Co Ni Cu Zn


\item Basis Set \verb#"6-31G**"# (number of atoms 30)  \newline 
  H He Li Be B C N O F Ne Na Mg Al Si P S Cl Ar K Ca Sc Ti V Cr Mn
 Fe Co Ni Cu Zn


\item Basis Set \verb#"6-31+G*"# (number of atoms 18)  \newline 
  H He Li Be B C N O F Ne Na Mg Al Si P S Cl Ar


\item Basis Set \verb#"6-31++G*"# (number of atoms 18)  \newline 
  H He Li Be B C N O F Ne Na Mg Al Si P S Cl Ar


\item Basis Set \verb#"6-31++G**"# (number of atoms 18)  \newline 
  H He Li Be B C N O F Ne Na Mg Al Si P S Cl Ar


\item Basis Set \verb#"6-31G(3df,3pd)"# (number of atoms 18)  \newline 
  H He Li Be B C N O F Ne Na Mg Al Si P S Cl Ar


\item Basis Set \verb#"6-31G-Blaudeau"# (number of atoms 2)  \newline 
  K Ca


\item Basis Set \verb#"6-31G*-Blaudeau"# (number of atoms 2)  \newline 
  K Ca


\item Basis Set \verb#"6-311G"# (number of atoms 24)  \newline 
  H He Li Be B C N O F Ne Na Mg Al Si P S Cl Ar Ga Ge As Se Br Kr


\item Basis Set \verb#"6-311G*"# (number of atoms 24)  \newline 
  H He Li Be B C N O F Ne Na Mg Al Si P S Cl Ar Ga Ge As Se Br Kr


\item Basis Set \verb#"6-311G**"# (number of atoms 24)  \newline 
  H He Li Be B C N O F Ne Na Mg Al Si P S Cl Ar Ga Ge As Se Br Kr


\item Basis Set \verb#"6-311++G**"# (number of atoms 10)  \newline 
  H He Li Be B C N O F Ne


\item Basis Set \verb#"6-311++G(2d,2p)"# (number of atoms 10)  \newline 
  H He Li Be B C N O F Ne


\item Basis Set \verb#"6-311G(2df,2pd)"# (number of atoms 10)  \newline 
  H He Li Be B C N O F Ne


\item Basis Set \verb#"6-311+G*"# (number of atoms 10)  \newline 
  H He Li Be B C N O F Ne


\item Basis Set \verb#"6-311++G(3df,3pd)"# (number of atoms 18)  \newline 
  H He Li Be B C N O F Ne Na Mg Al Si P S Cl Ar


\item Basis Set \verb#"MINI (Huzinaga)"# (number of atoms 18)  \newline 
  H He Li Be B C N O F Ne Na Mg Al Si P S Cl Ar


\item Basis Set \verb#"MINI (Scaled)"# (number of atoms 20)  \newline 
  H He Li Be B C N O F Ne Na Mg Al Si P S Cl Ar K Ca


\item Basis Set \verb#"MIDI (Huzinaga)"# (number of atoms 18)  \newline 
  H He Li Be B C N O F Ne Na Al Si P S Cl Ar K


\item Basis Set \verb#"MIDI!"# (number of atoms 11)  \newline 
  H C N O F Si P S Cl Br I


\item Basis Set \verb#"SV (Dunning-Hay)"# (number of atoms 9)  \newline 
  H Li Be B C N O F Ne


\item Basis Set \verb#"SVP (Dunning-Hay)"# (number of atoms 9)  \newline 
  H Li Be B C N O F Ne


\item Basis Set \verb#"SVP + Diffuse (Dunning-Hay)"# (number of atoms 9)  \newline 
  H Li Be B C N O F Ne


\item Basis Set \verb#"TZ (Dunning)"# (number of atoms 8)  \newline 
  H Li B C N O F Ne


\item Basis Set \verb#"Chipman DZP + Diffuse"# (number of atoms 6)  \newline 
  H B C N O F


\item Basis Set \verb#"DZ (Dunning)"# (number of atoms 12)  \newline 
  H B C N O F Ne Al Si P S Cl


\item Basis Set \verb#"DZP (Dunning)"# (number of atoms 12)  \newline 
  H B C N O F Ne Al Si P S Cl


\item Basis Set \verb#"DZP + Diffuse (Dunning)"# (number of atoms 7)  \newline 
  H B C N O F Ne


\item Basis Set \verb#"cc-pVDZ"# (number of atoms 20)  \newline 
  H He B C N O F Ne Al Si P S Cl Ar Ga Ge As Se Br Kr


\item Basis Set \verb#"cc-pVTZ"# (number of atoms 20)  \newline 
  H He B C N O F Ne Al Si P S Cl Ar Ga Ge As Se Br Kr


\item Basis Set \verb#"cc-pVQZ"# (number of atoms 20)  \newline 
  H He B C N O F Ne Al Si P S Cl Ar Ga Ge As Se Br Kr


\item Basis Set \verb#"cc-pV5Z"# (number of atoms 21)  \newline 
  H He Li B C N O F Ne Al Si P S Cl Ar Ga Ge As Se Br Kr


\item Basis Set \verb#"cc-pV6Z"# (number of atoms 14)  \newline 
  H He B C N O F Ne Al Si P S Cl Ar


\item Basis Set \verb#"cc-pCVDZ"# (number of atoms 6)  \newline 
  B C N O F Ne


\item Basis Set \verb#"cc-pCVTZ"# (number of atoms 6)  \newline 
  B C N O F Ne


\item Basis Set \verb#"cc-pCVQZ"# (number of atoms 6)  \newline 
  B C N O F Ne


\item Basis Set \verb#"cc-pCV5Z"# (number of atoms 6)  \newline 
  B C N O F Ne


\item Basis Set \verb#"aug-cc-pVDZ"# (number of atoms 20)  \newline 
  H He B C N O F Ne Al Si P S Cl Ar Ga Ge As Se Br Kr


\item Basis Set \verb#"aug-cc-pVTZ"# (number of atoms 20)  \newline 
  H He B C N O F Ne Al Si P S Cl Ar Ga Ge As Se Br Kr


\item Basis Set \verb#"aug-cc-pVQZ"# (number of atoms 20)  \newline 
  H He B C N O F Ne Al Si P S Cl Ar Ga Ge As Se Br Kr


\item Basis Set \verb#"aug-cc-pV5Z"# (number of atoms 20)  \newline 
  H He B C N O F Ne Al Si P S Cl Ar Ga Ge As Se Br Kr


\item Basis Set \verb#"aug-cc-pV6Z"# (number of atoms 14)  \newline 
  H He B C N O F Ne Al Si P S Cl Ar


\item Basis Set \verb#"aug-cc-pCVDZ"# (number of atoms 5)  \newline 
  B C N O F


\item Basis Set \verb#"aug-cc-pCVTZ"# (number of atoms 6)  \newline 
  B C N O F Ne


\item Basis Set \verb#"aug-cc-pCVQZ"# (number of atoms 6)  \newline 
  B C N O F Ne


\item Basis Set \verb#"aug-cc-pCV5Z"# (number of atoms 5)  \newline 
  B C N O F


\item Basis Set \verb#"d-aug-cc-pVDZ"# (number of atoms 8)  \newline 
  H He B C N O F Ne


\item Basis Set \verb#"d-aug-cc-pVTZ"# (number of atoms 8)  \newline 
  H He B C N O F Ne


\item Basis Set \verb#"d-aug-cc-pVQZ"# (number of atoms 8)  \newline 
  H He B C N O F Ne


\item Basis Set \verb#"d-aug-cc-pV5Z"# (number of atoms 8)  \newline 
  H He B C N O F Ne


\item Basis Set \verb#"d-aug-cc-pV6Z"# (number of atoms 5)  \newline 
  H B C N O


\item Basis Set \verb#"GAMESS VTZ"# (number of atoms 8)  \newline 
  H Be B C N O F Ne


\item Basis Set \verb#"GAMESS PVTZ"# (number of atoms 8)  \newline 
  H Be B C N O F Ne


\item Basis Set \verb#"Partridge Uncontr. 1"# (number of atoms 34)  \newline 
  Li Be B C N O F Ne Na Mg Al Si P S Cl Ar K Ca Sc Ti V Cr Mn Fe Co
 Ni Cu Zn Ga Ge As Se Br Kr


\item Basis Set \verb#"Partridge Uncontr. 2"# (number of atoms 34)  \newline 
  Li Be B C N O F Ne Na Mg Al Si P S Cl Ar K Ca Sc Ti V Cr Mn Fe Co
 Ni Cu Zn Ga Ge As Se Br Kr


\item Basis Set \verb#"Partridge Uncontr. 3"# (number of atoms 28)  \newline 
  Li Be B C N O F Ne Na Mg Al Si P S Cl Ar K Ca Sc Ti V Cr Mn Fe Co
 Ni Cu Zn


\item Basis Set \verb#"Ahlrichs VDZ"# (number of atoms 36)  \newline 
  H He Li Be B C N O F Ne Na Mg Al Si P S Cl Ar K Ca Sc Ti V Cr Mn
 Fe Co Ni Cu Zn Ga Ge As Se Br Kr


\item Basis Set \verb#"Ahlrichs pVDZ"# (number of atoms 36)  \newline 
  H He Li Be B C N O F Ne Na Mg Al Si P S Cl Ar K Ca Sc Ti V Cr Mn
 Fe Co Ni Cu Zn Ga Ge As Se Br Kr


\item Basis Set \verb#"Ahlrichs VTZ"# (number of atoms 36)  \newline 
  H He Li Be B C N O F Ne Na Mg Al Si P S Cl Ar K Ca Sc Ti V Cr Mn
 Fe Co Ni Cu Zn Ga Ge As Se Br Kr


\item Basis Set \verb#"Binning/Curtiss SV"# (number of atoms 6)  \newline 
  Ga Ge As Se Br Kr


\item Basis Set \verb#"Binning/Curtiss VTZ"# (number of atoms 6)  \newline 
  Ga Ge As Se Br Kr


\item Basis Set \verb#"Binning/Curtiss SVP"# (number of atoms 6)  \newline 
  Ga Ge As Se Br Kr


\item Basis Set \verb#"Binning/Curtiss VTZP"# (number of atoms 6)  \newline 
  Ga Ge As Se Br Kr


\item Basis Set \verb#"McLean/Chandler VTZ"# (number of atoms 8)  \newline 
  Na Mg Al Si P S Cl Ar


\item Basis Set \verb#"SV + Rydberg (Dunning-Hay)"# (number of atoms 9)  \newline 
  H Li Be B C N O F Ne


\item Basis Set \verb#"SVP + Rydberg (Dunning-Hay)"# (number of atoms 9)  \newline 
  H Li Be B C N O F Ne


\item Basis Set \verb#"SVP + Diffuse + Rydberg"# (number of atoms 9)  \newline 
  H Li Be B C N O F Ne


\item Basis Set \verb#"DZ + Rydberg (Dunning)"# (number of atoms 12)  \newline 
  H B C N O F Ne Al Si P S Cl


\item Basis Set \verb#"DZP + Rydberg (Dunning)"# (number of atoms 12)  \newline 
  H B C N O F Ne Al Si P S Cl


\item Basis Set \verb#"DZ + Double Rydberg (Dunning-Hay)"# (number of atoms 12)  \newline 
  H B C N O F Ne Al Si P S Cl


\item Basis Set \verb#"SV + Double Rydberg (Dunning-Hay)"# (number of atoms 9)  \newline 
  H Li Be B C N O F Ne


\item Basis Set \verb#"Wachters+f"# (number of atoms 9)  \newline 
  Sc Ti V Cr Mn Fe Co Ni Cu


\item Basis Set \verb#"Bauschlicher ANO"# (number of atoms 9)  \newline 
  Sc Ti V Cr Mn Fe Co Ni Cu


\item Basis Set \verb#"NASA Ames ANO"# (number of atoms 12)  \newline 
  H B C N O F Ne Al P Ti Fe Ni


\item Basis Set \verb#"Sadlej pVTZ"# (number of atoms 18)  \newline 
  H Li Be C N O F Na Mg Si P S Cl K Ca Br Rb Sr


\item Basis Set \verb#"WTBS"# (number of atoms 84)  \newline 
  He Li Be B C N O F Ne Na Mg Al Si P S Cl Ar K Ca Sc Ti V Cr Mn Fe
 Co Ni Cu Zn Ga Ge As Se Br Kr Rb Sr Y Zr Nb Mo Tc Ru Rh Pd Ag Cd In Sn Sb
 Te I Xe Cs Ba La Ce Pr Pm Sm Eu Gd Tb Dy Ho Er Tm Yb Lu Hf Ta W Re Os Ir
 Pt Au Hg Tl Pb Bi Po At Rn


\end{enumerate}

Resolution of Identity (RI) fitting basis sets:

\begin{enumerate}

\item Basis Set \verb#"cc-pVDZ-fit2-1"# (number of atoms 10)  \newline 
  H He Li Be B C N O F Ne


\item Basis Set \verb#"cc-pVTZ-fit2-1"# (number of atoms 10)  \newline 
  H He Li Be B C N O F Ne

\end{enumerate}

Density functional specific basis sets:

\begin{enumerate}

\item Basis Set \verb#"DZVP (DFT Orbital)"# (number of atoms 54)  \newline 
  H He Li Be B C N O F Ne Na Mg Al Si P S Cl Ar K Ca Sc Ti V Cr Mn
 Fe Co Ni Cu Zn Ga Ge As Se Br Kr Rb Sr Y Zr Nb Mo Tc Ru Rh Pd Ag Cd In Sn
 Sb Te I Xe


\item Basis Set \verb#"DZVP2 (DFT Orbital)"# (number of atoms 25)  \newline 
  H He Li Be B C N O F Al Si P S Cl Ar Sc Ti V Cr Mn Fe Co Ni Cu Zn


\item Basis Set \verb#"TZVP (DFT Orbital)"# (number of atoms 11)  \newline 
  H C N O F Al Si P S Cl Ar

\end{enumerate}

Density functional Coulomb and Exchange fitting basis sets:

\begin{enumerate}

\item Basis Set \verb#"DGauss A1 DFT Coulomb Fitting"# (number of atoms 54)  \newline 
  H He Li Be B C N O F Ne Na Mg Al Si P S Cl Ar K Ca Sc Ti V Cr Mn
 Fe Co Ni Cu Zn Ga Ge As Se Br Kr Rb Sr Y Zr Nb Mo Tc Ru Rh Pd Ag Cd In Sn
 Sb Te I Xe


\item Basis Set \verb#"DGauss A1 DFT Exchange Fitting"# (number of atoms 54)  \newline 
  H He Li Be B C N O F Ne Na Mg Al Si P S Cl Ar K Ca Sc Ti V Cr Mn
 Fe Co Ni Cu Zn Ga Ge As Se Br Kr Rb Sr Y Zr Nb Mo Tc Ru Rh Pd Ag Cd In Sn
 Sb Te I Xe


\item Basis Set \verb#"DGauss A2 DFT Coulomb Fitting"# (number of atoms 25)  \newline 
  H He Li Be B C N O F Al Si P S Cl Ar Sc Ti V Cr Mn Fe Co Ni Cu Zn



\item Basis Set \verb#"DGauss A2 DFT Exchange Fitting"# (number of atoms 25)  \newline 
  H He Li Be B C N O F Al Si P S Cl Ar Sc Ti V Cr Mn Fe Co Ni Cu Zn


\item Basis Set \verb#"DeMon Coulomb Fitting"# (number of atoms 54)  \newline 
  H He Li Be B C N O F Ne Na Mg Al Si P S Cl Ar K Ca Sc Ti V Cr Mn
 Fe Co Ni Cu Zn Ga Ge As Se Br Kr Rb Sr Y Zr Nb Mo Tc Ru Rh Pd Ag Cd In Sn
 Sb Te I Xe


\item Basis Set \verb#"Ahlrichs Coulomb Fitting"# (number of atoms 50)  \newline 
  H He Li Be B C N O F Ne Na Mg Al Si P S Cl Ar K Ca Sc Ti V Cr Mn
 Fe Co Ni Cu Zn Ga Ge As Se Br Kr Rb Sr Y Zr Nb Mo Tc Ru Rh Pd Ag Cd In Sn

\end{enumerate}

Effective core potentials and their respective basis sets:

\begin{enumerate}

\item Basis Set \verb#"Hay-Wadt MB (n+1) ECP"# (number of atoms 24)  \newline 
  K Ca Sc Ti V Cr Mn Fe Co Ni Cu Rb Sr Y Zr Nb Mo Tc Ru Rh Pd Ag Cs Ba


\item ECP \verb#"Hay-Wadt MB (n+1) ECP"# (number of atoms 24)  \newline 
  K Ca Sc Ti V Cr Mn Fe Co Ni Cu Rb Sr Y Zr Nb Mo Tc Ru Rh Pd Ag Cs Ba


\item Basis Set \verb#"Hay-Wadt VDZ (n+1) ECP"# (number of atoms 24)  \newline 
  K Ca Sc Ti V Cr Mn Fe Co Ni Cu Rb Sr Y Zr Nb Mo Tc Ru Rh Pd Ag Cs Ba


\item ECP \verb#"Hay-Wadt VDZ (n+1) ECP"# (number of atoms 24)  \newline 
  K Ca Sc Ti V Cr Mn Fe Co Ni Cu Rb Sr Y Zr Nb Mo Tc Ru Rh Pd Ag Cs Ba


\item Basis Set \verb#"LANL2DZ ECP"# (number of atoms 67)  \newline 
  H Li Be B C N O F Ne Na Mg Al Si P S Cl Ar K Ca Sc Ti V Cr Mn Fe
 Co Ni Cu Zn Ga Ge As Se Br Kr Rb Sr Y Zr Nb Mo Tc Ru Rh Pd Ag Cd In Sn Sb
 Te I Xe Cs Ba La Hf Ta W Re Os Ir Pt Au U Np Pu


\item ECP \verb#"LANL2DZ ECP"# (number of atoms 57)  \newline 
  Na Mg Al Si P S Cl Ar K Ca Sc Ti V Cr Mn Fe Co Ni Cu Zn Ga Ge As Se Br
 Kr Rb Sr Y Zr Nb Mo Tc Ru Rh Pd Ag Cd In Sn Sb Te I Xe Cs Ba La Hf Ta W
 Re Os Ir Pt Au U Np


\item Basis Set \verb#"SBKJC VDZ ECP"# (number of atoms 73)  \newline 
  H He Li Be B C N O F Ne Na Mg Al Si P S Cl Ar K Ca Sc Ti V Cr Mn
 Fe Co Ni Cu Zn Ga Ge As Se Br Kr Rb Sr Y Zr Nb Mo Tc Ru Rh Pd Ag Cd In Sn
 Sb Te I Xe Cs Ba La Ce Hf Ta W Re Os Ir Pt Au Hg Tl Pb Bi Po At Rn


\item ECP \verb#"SBKJC VDZ ECP"# (number of atoms 71)  \newline 
  Li Be B C N O F Ne Na Mg Al Si P S Cl Ar K Ca Sc Ti V Cr Mn Fe Co
 Ni Cu Zn Ga Ge As Se Br Kr Rb Sr Y Zr Nb Mo Tc Ru Rh Pd Ag Cd In Sn Sb Te
 I Xe Cs Ba La Ce Hf Ta W Re Os Ir Pt Au Hg Tl Pb Bi Po At Rn


\item Basis Set \verb#"CRENBL ECP"# (number of atoms 116)  \newline 
  H Li Be B C N O F Ne Na Mg Al Si P S Cl Ar K Ca Sc Ti V Cr Mn Fe
 Co Ni Cu Zn Ga Ge As Se Br Kr Rb Sr Y Zr Nb Mo Tc Ru Rh Pd Ag Cd In Sn Sb
 Te I Xe Cs Ba La Ce Pr Nd Pm Sm Eu Gd Tb Dy Ho Er Tm Yb Lu Hf Ta W Re Os
 Ir Pt Au Hg Pb Bi Po At Rn Fr Ra Ac Th Pa U Np Pu Am Cm Bk Cf Es Fm Md No
 Lr Rf Db Sg Bh Hs Mt Un Uu Ub Ut Uq Up Uh Us Uo


\item ECP \verb#"CRENBL ECP"# (number of atoms 115)  \newline 
  Li Be B C N O F Ne Na Mg Al Si P S Cl Ar K Ca Sc Ti V Cr Mn Fe Co
 Ni Cu Zn Ga Ge As Se Br Kr Rb Sr Y Zr Nb Mo Tc Ru Rh Pd Ag Cd In Sn Sb Te
 I Xe Cs Ba La Ce Pr Nd Pm Sm Eu Gd Tb Dy Ho Er Tm Yb Lu Hf Ta W Re Os Ir
 Pt Au Hg Pb Bi Po At Rn Fr Ra Ac Th Pa U Np Pu Am Cm Bk Cf Es Fm Md No Lr
 Rf Db Sg Bh Hs Mt Un Uu Ub Ut Uq Up Uh Us Uo


\item Basis Set \verb#"CRENBS ECP"# (number of atoms 50)  \newline 
  Sc Ti V Cr Mn Fe Co Ni Cu Zn Y Zr Nb Mo Tc Ru Rh Pd Ag Cd La Hf Ta W Re
 Os Ir Pt Au Hg Pb Bi Po At Rn Rf Db Sg Bh Hs Mt Un Uu Ub Ut Uq Up Uh Us Uo



\item ECP \verb#"CRENBS ECP"# (number of atoms 50)  \newline 
  Sc Ti V Cr Mn Fe Co Ni Cu Zn Y Zr Nb Mo Tc Ru Rh Pd Ag Cd La Hf Ta W Re
 Os Ir Pt Au Hg Pb Bi Po At Rn Rf Db Sg Bh Hs Mt Un Uu Ub Ut Uq Up Uh Us Uo



\item Basis Set \verb#"Stuttgart RLC ECP"# (number of atoms 57)  \newline 
  Li Be B C N O F Ne Na Mg Al Si P S Cl Ar K Ca Zn Ga Ge As Se Br Kr
 Rb Sr In Sn Sb Te I Xe Cs Ba Hg Tl Pb Bi Po At Rn Ac Th Pa U Np Pu Am Cm
 Bk Cf Es Fm Md No Lr


\item ECP \verb#"Stuttgart RLC ECP"# (number of atoms 57)  \newline 
  Li Be B C N O F Ne Na Mg Al Si P S Cl Ar K Ca Zn Ga Ge As Se Br Kr
 Rb Sr In Sn Sb Te I Xe Cs Ba Hg Tl Pb Bi Po At Rn Ac Th Pa U Np Pu Am Cm
 Bk Cf Es Fm Md No Lr


\item Basis Set \verb#"Stuttgart RSC ECP"# (number of atoms 64)  \newline 
  K Ca Sc Ti V Cr Mn Fe Co Ni Cu Zn Rb Sr Y Zr Nb Mo Tc Ru Rh Pd Ag Cd Cs
 Ba Ce Pr Nd Pm Sm Eu Gd Tb Dy Ho Er Tm Yb Hf Ta W Re Os Ir Pt Au Hg Ac Th
 Pa U Np Pu Am Cm Bk Cf Es Fm Md No Lr Db


\item ECP \verb#"Stuttgart RSC ECP"# (number of atoms 64)  \newline 
  K Ca Sc Ti V Cr Mn Fe Co Ni Cu Zn Rb Sr Y Zr Nb Mo Tc Ru Rh Pd Ag Cd Cs
 Ba Ce Pr Nd Pm Sm Eu Gd Tb Dy Ho Er Tm Yb Hf Ta W Re Os Ir Pt Au Hg Ac Th
 Pa U Np Pu Am Cm Bk Cf Es Fm Md No Lr Db

\end{enumerate}

Douglas-Kroll (DK) all-electron basis sets:

\begin{enumerate}

\item Basis Set \verb#"cc-pVDZ DK"# (number of atoms 20)  \newline 
  H He B C N O F Ne Al Si P S Cl Ar Ga Ge As Se Br Kr


\item Basis Set \verb#"cc-pVTZ DK"# (number of atoms 20)  \newline 
  H He B C N O F Ne Al Si P S Cl Ar Ga Ge As Se Br Kr


\item Basis Set \verb#"cc-pVQZ DK"# (number of atoms 20)  \newline 
  H He B C N O F Ne Al Si P S Cl Ar Ga Ge As Se Br Kr


\item Basis Set \verb#"cc-pV5Z DK"# (number of atoms 20)  \newline 
  H He B C N O F Ne Al Si P S Cl Ar Ga Ge As Se Br Kr

\end{enumerate}

Dyall's Modified Dirac (DmD) all-electron basis sets:

\begin{enumerate}

\item Basis Set \verb#"cc-pvdz fi sf fw"# (number of atoms 20)  \newline 
  H He B C N O F Ne Al Si P S Cl Ar Ga Ge As Se Br Kr


\item Basis Set \verb#"cc-pvdz fi sf lc"# (number of atoms 20)  \newline 
  H He B C N O F Ne Al Si P S Cl Ar Ga Ge As Se Br Kr


\item Basis Set \verb#"cc-pvdz fi sf sc"# (number of atoms 20)  \newline 
  H He B C N O F Ne Al Si P S Cl Ar Ga Ge As Se Br Kr


\item Basis Set \verb#"cc-pvdz pt sf fw"# (number of atoms 20)  \newline 
  H He B C N O F Ne Al Si P S Cl Ar Ga Ge As Se Br Kr


\item Basis Set \verb#"cc-pvdz pt sf lc"# (number of atoms 20)  \newline 
  H He B C N O F Ne Al Si P S Cl Ar Ga Ge As Se Br Kr


\item Basis Set \verb#"cc-pvdz pt sf sc"# (number of atoms 20)  \newline 
  H He B C N O F Ne Al Si P S Cl Ar Ga Ge As Se Br Kr


\item Basis Set \verb#"cc-pvtz fi sf fw"# (number of atoms 20)  \newline 
  H He B C N O F Ne Al Si P S Cl Ar Ga Ge As Se Br Kr


\item Basis Set \verb#"cc-pvtz fi sf lc"# (number of atoms 20)  \newline 
  H He B C N O F Ne Al Si P S Cl Ar Ga Ge As Se Br Kr


\item Basis Set \verb#"cc-pvtz fi sf sc"# (number of atoms 20)  \newline 
  H He B C N O F Ne Al Si P S Cl Ar Ga Ge As Se Br Kr


\item Basis Set \verb#"cc-pvtz pt sf fw"# (number of atoms 20)  \newline 
  H He B C N O F Ne Al Si P S Cl Ar Ga Ge As Se Br Kr


\item Basis Set \verb#"cc-pvtz pt sf lc"# (number of atoms 20)  \newline 
  H He B C N O F Ne Al Si P S Cl Ar Ga Ge As Se Br Kr


\item Basis Set \verb#"cc-pvtz pt sf sc"# (number of atoms 20)  \newline 
  H He B C N O F Ne Al Si P S Cl Ar Ga Ge As Se Br Kr


\item Basis Set \verb#"cc-pvqz fi sf fw"# (number of atoms 20)  \newline 
  H He B C N O F Ne Al Si P S Cl Ar Ga Ge As Se Br Kr


\item Basis Set \verb#"cc-pvqz fi sf lc"# (number of atoms 20)  \newline 
  H He B C N O F Ne Al Si P S Cl Ar Ga Ge As Se Br Kr


\item Basis Set \verb#"cc-pvqz fi sf sc"# (number of atoms 20)  \newline 
  H He B C N O F Ne Al Si P S Cl Ar Ga Ge As Se Br Kr


\item Basis Set \verb#"cc-pvqz pt sf fw"# (number of atoms 20)  \newline 
  H He B C N O F Ne Al Si P S Cl Ar Ga Ge As Se Br Kr


\item Basis Set \verb#"cc-pvqz pt sf lc"# (number of atoms 20)  \newline 
  H He B C N O F Ne Al Si P S Cl Ar Ga Ge As Se Br Kr


\item Basis Set \verb#"cc-pvqz pt sf sc"# (number of atoms 20)  \newline 
  H He B C N O F Ne Al Si P S Cl Ar Ga Ge As Se Br Kr


\item Basis Set \verb#"cc-pv5z fi sf fw"# (number of atoms 20)  \newline 
  H He B C N O F Ne Al Si P S Cl Ar Ga Ge As Se Br Kr


\item Basis Set \verb#"cc-pv5z fi sf lc"# (number of atoms 20)  \newline 
  H He B C N O F Ne Al Si P S Cl Ar Ga Ge As Se Br Kr


\item Basis Set \verb#"cc-pv5z fi sf sc"# (number of atoms 20)  \newline 
  H He B C N O F Ne Al Si P S Cl Ar Ga Ge As Se Br Kr


\item Basis Set \verb#"cc-pv5z pt sf fw"# (number of atoms 20)  \newline 
  H He B C N O F Ne Al Si P S Cl Ar Ga Ge As Se Br Kr


\item Basis Set \verb#"cc-pv5z pt sf lc"# (number of atoms 20)  \newline 
  H He B C N O F Ne Al Si P S Cl Ar Ga Ge As Se Br Kr


\item Basis Set \verb#"cc-pv5z pt sf sc"# (number of atoms 20)  \newline 
  H He B C N O F Ne Al Si P S Cl Ar Ga Ge As Se Br Kr


\item Basis Set \verb#"aug-cc-pvdz fi sf fw"# (number of atoms 20)  \newline 
  H He B C N O F Ne Al Si P S Cl Ar Ga Ge As Se Br Kr


\item Basis Set \verb#"aug-cc-pvdz fi sf lc"# (number of atoms 20)  \newline 
  H He B C N O F Ne Al Si P S Cl Ar Ga Ge As Se Br Kr


\item Basis Set \verb#"aug-cc-pvdz fi sf sc"# (number of atoms 20)  \newline 
  H He B C N O F Ne Al Si P S Cl Ar Ga Ge As Se Br Kr


\item Basis Set \verb#"aug-cc-pvdz pt sf fw"# (number of atoms 20)  \newline 
  H He B C N O F Ne Al Si P S Cl Ar Ga Ge As Se Br Kr


\item Basis Set \verb#"aug-cc-pvdz pt sf lc"# (number of atoms 20)  \newline 
  H He B C N O F Ne Al Si P S Cl Ar Ga Ge As Se Br Kr


\item Basis Set \verb#"aug-cc-pvdz pt sf sc"# (number of atoms 20)  \newline 
  H He B C N O F Ne Al Si P S Cl Ar Ga Ge As Se Br Kr


\item Basis Set \verb#"aug-cc-pvtz fi sf fw"# (number of atoms 20)  \newline 
  H He B C N O F Ne Al Si P S Cl Ar Ga Ge As Se Br Kr


\item Basis Set \verb#"aug-cc-pvtz fi sf lc"# (number of atoms 20)  \newline 
  H He B C N O F Ne Al Si P S Cl Ar Ga Ge As Se Br Kr


\item Basis Set \verb#"aug-cc-pvtz fi sf sc"# (number of atoms 20)  \newline 
  H He B C N O F Ne Al Si P S Cl Ar Ga Ge As Se Br Kr


\item Basis Set \verb#"aug-cc-pvtz pt sf fw"# (number of atoms 20)  \newline 
  H He B C N O F Ne Al Si P S Cl Ar Ga Ge As Se Br Kr


\item Basis Set \verb#"aug-cc-pvtz pt sf lc"# (number of atoms 20)  \newline 
  H He B C N O F Ne Al Si P S Cl Ar Ga Ge As Se Br Kr


\item Basis Set \verb#"aug-cc-pvtz pt sf sc"# (number of atoms 20)  \newline 
  H He B C N O F Ne Al Si P S Cl Ar Ga Ge As Se Br Kr


\item Basis Set \verb#"aug-cc-pvqz fi sf fw"# (number of atoms 20)  \newline 
  H He B C N O F Ne Al Si P S Cl Ar Ga Ge As Se Br Kr


\item Basis Set \verb#"aug-cc-pvqz fi sf lc"# (number of atoms 20)  \newline 
  H He B C N O F Ne Al Si P S Cl Ar Ga Ge As Se Br Kr


\item Basis Set \verb#"aug-cc-pvqz fi sf sc"# (number of atoms 20)  \newline 
  H He B C N O F Ne Al Si P S Cl Ar Ga Ge As Se Br Kr


\item Basis Set \verb#"aug-cc-pvqz pt sf fw"# (number of atoms 20)  \newline 
  H He B C N O F Ne Al Si P S Cl Ar Ga Ge As Se Br Kr


\item Basis Set \verb#"aug-cc-pvqz pt sf lc"# (number of atoms 20)  \newline 
  H He B C N O F Ne Al Si P S Cl Ar Ga Ge As Se Br Kr


\item Basis Set \verb#"aug-cc-pvqz pt sf sc"# (number of atoms 20)  \newline 
  H He B C N O F Ne Al Si P S Cl Ar Ga Ge As Se Br Kr


\item Basis Set \verb#"aug-cc-pv5z fi sf fw"# (number of atoms 20)  \newline 
  H He B C N O F Ne Al Si P S Cl Ar Ga Ge As Se Br Kr


\item Basis Set \verb#"aug-cc-pv5z fi sf lc"# (number of atoms 20)  \newline 
  H He B C N O F Ne Al Si P S Cl Ar Ga Ge As Se Br Kr


\item Basis Set \verb#"aug-cc-pv5z fi sf sc"# (number of atoms 20)  \newline 
  H He B C N O F Ne Al Si P S Cl Ar Ga Ge As Se Br Kr


\item Basis Set \verb#"aug-cc-pv5z pt sf fw"# (number of atoms 20)  \newline 
  H He B C N O F Ne Al Si P S Cl Ar Ga Ge As Se Br Kr


\item Basis Set \verb#"aug-cc-pv5z pt sf lc"# (number of atoms 20)  \newline 
  H He B C N O F Ne Al Si P S Cl Ar Ga Ge As Se Br Kr


\item Basis Set \verb#"aug-cc-pv5z pt sf sc"# (number of atoms 20)  \newline 
  H He B C N O F Ne Al Si P S Cl Ar Ga Ge As Se Br Kr


\item Basis Set \verb#"dyall relpvdz fi sf fw"# (number of atoms 18)  \newline 
  Ga Ge As Se Br Kr In Sn Sb Te I Xe Tl Pb Bi Po At Rn


\item Basis Set \verb#"dyall relpvdz fi sf lc"# (number of atoms 18)  \newline 
  Ga Ge As Se Br Kr In Sn Sb Te I Xe Tl Pb Bi Po At Rn


\item Basis Set \verb#"dyall relpvdz fi sf sc"# (number of atoms 18)  \newline 
  Ga Ge As Se Br Kr In Sn Sb Te I Xe Tl Pb Bi Po At Rn


\item Basis Set \verb#"dyall relrcpvdz fi sf fw"# (number of atoms 18)  \newline 
  Ga Ge As Se Br Kr In Sn Sb Te I Xe Tl Pb Bi Po At Rn


\item Basis Set \verb#"dyall relrcpvdz fi sf lc"# (number of atoms 18)  \newline 
  Ga Ge As Se Br Kr In Sn Sb Te I Xe Tl Pb Bi Po At Rn


\item Basis Set \verb#"dyall relrcpvdz fi sf sc"# (number of atoms 18)  \newline 
  Ga Ge As Se Br Kr In Sn Sb Te I Xe Tl Pb Bi Po At Rn


\item Basis Set \verb#"dyall relpcpvdz fi sf fw"# (number of atoms 18)  \newline 
  Ga Ge As Se Br Kr In Sn Sb Te I Xe Tl Pb Bi Po At Rn


\item Basis Set \verb#"dyall relpcpvdz fi sf lc"# (number of atoms 18)  \newline 
  Ga Ge As Se Br Kr In Sn Sb Te I Xe Tl Pb Bi Po At Rn


\item Basis Set \verb#"dyall relpcpvdz fi sf sc"# (number of atoms 18)  \newline 
  Ga Ge As Se Br Kr In Sn Sb Te I Xe Tl Pb Bi Po At Rn


\item Basis Set \verb#"dyall relapvdz fi sf fw"# (number of atoms 18)  \newline 
  Ga Ge As Se Br Kr In Sn Sb Te I Xe Tl Pb Bi Po At Rn


\item Basis Set \verb#"dyall relapvdz fi sf lc"# (number of atoms 18)  \newline 
  Ga Ge As Se Br Kr In Sn Sb Te I Xe Tl Pb Bi Po At Rn


\item Basis Set \verb#"dyall relapvdz fi sf sc"# (number of atoms 18)  \newline 
  Ga Ge As Se Br Kr In Sn Sb Te I Xe Tl Pb Bi Po At Rn


\item Basis Set \verb#"dyall relrcapvdz fi sf fw"# (number of atoms 18)  \newline 
  Ga Ge As Se Br Kr In Sn Sb Te I Xe Tl Pb Bi Po At Rn


\item Basis Set \verb#"dyall relrcapvdz fi sf lc"# (number of atoms 18)  \newline 
  Ga Ge As Se Br Kr In Sn Sb Te I Xe Tl Pb Bi Po At Rn


\item Basis Set \verb#"dyall relrcapvdz fi sf sc"# (number of atoms 18)  \newline 
  Ga Ge As Se Br Kr In Sn Sb Te I Xe Tl Pb Bi Po At Rn


\item Basis Set \verb#"dyall relpcapvdz fi sf fw"# (number of atoms 18)  \newline 
  Ga Ge As Se Br Kr In Sn Sb Te I Xe Tl Pb Bi Po At Rn


\item Basis Set \verb#"dyall relpcapvdz fi sf lc"# (number of atoms 18)  \newline 
  Ga Ge As Se Br Kr In Sn Sb Te I Xe Tl Pb Bi Po At Rn


\item Basis Set \verb#"dyall relpcapvdz fi sf sc"# (number of atoms 18)  \newline 
  Ga Ge As Se Br Kr In Sn Sb Te I Xe Tl Pb Bi Po At Rn


\item Basis Set \verb#"dyall reldz aug"# (number of atoms 15)  \newline 
  Ga Ge As Se Br In Sn Sb Te I Tl Pb Bi Po At


\item Basis Set \verb#"dyall nrpvdz fi"# (number of atoms 18)  \newline 
  Ga Ge As Se Br Kr In Sn Sb Te I Xe Tl Pb Bi Po At Rn


\item Basis Set \verb#"dyall nrpcpvdz fi"# (number of atoms 18)  \newline 
  Ga Ge As Se Br Kr In Sn Sb Te I Xe Tl Pb Bi Po At Rn


\item Basis Set \verb#"dyall nrrcpvdz fi"# (number of atoms 18)  \newline 
  Ga Ge As Se Br Kr In Sn Sb Te I Xe Tl Pb Bi Po At Rn


\item Basis Set \verb#"dyall nrapvdz fi"# (number of atoms 15)  \newline 
  Ga Ge As Se Br In Sn Sb Te I Tl Pb Bi Po At


\item Basis Set \verb#"dyall nrrcapvdz fi"# (number of atoms 15)  \newline 
  Ga Ge As Se Br In Sn Sb Te I Tl Pb Bi Po At


\item Basis Set \verb#"dyall nrpcapvdz fi"# (number of atoms 15)  \newline 
  Ga Ge As Se Br In Sn Sb Te I Tl Pb Bi Po At

\end{enumerate}

\fussy




\chapter{Sample input files}
\label{sec:sample}
\subsection{Water SCF calculation and geometry optimization in a 6-31g basis}
\label{sec:sample1}

The input file in section \ref{sec:getstart} performs a geometry optimization
in a single task. A single point SCF energy calculation is performed and then
restarted to perform the optimization (both could of course be
performed in a single task).

\subsubsection{Job 1.  Single point SCF energy}

\begin{verbatim}
  start h2o
  title; Water in 6-31g basis set

  geometry
    O      0.00000000    0.00000000    0.00000000
    H      0.00000000    1.43042809   -1.10715266
    H      0.00000000   -1.43042809   -1.10715266
  end
  basis print
    H library 6-31g
    O library 6-31g
  end
  task scf
\end{verbatim}

The final energy should be -75.9839975707.

\subsubsection{Job 2. Restarting and perform a geometry optimization}

\begin{verbatim}
  restart h2o
  title; Water geometry optimization

  task scf optimize
\end{verbatim}

There is no need to specify anything that has not changed from the
previous input deck, though it will do no harm to repeat it.  The
final energy and geometry should be $-75.9853591759$, O
$(0,0,0.1563305320)$, and H $(0, \pm1.48372809, -0.853122128)$.

\subsection{Compute the polarizability of Ne using finite field}
\label{sec:sample2}

\subsubsection{Job 1. Compute the atomic energy}

\begin{verbatim}
  start ne
  title; Neon
  geometry; ne 0 0 0; end
  basis spherical 
    ne library aug-cc-pvdz
  end
  scf; thresh 1e-10; end
  task scf
\end{verbatim}

The final energy should be -128.49634973.

\subsubsection{Job 2. Compute the energy with applied field}

An external field may be simulated with point charges.  The charges
here apply a field of magnitude 0.01\ atomic units to the atom at the
origin.  Since the basis functions have not been reordered by the
additional centers we can also restart from the previous vectors,
which is the default for a restart job.

\begin{verbatim}
  restart ne
  title; Neon in electric field
  geometry
    bq1 0 0 100 charge 50
    ne  0 0 0
    bq2 0 0 -100 charge -50
  end
  task scf
\end{verbatim}

The final energy should be -128.49644133, which together with the
previous field-free result yields an estimate for the polarizability
of 1.83 atomic units.  Note that by default NWChem does not include
the interaction between the two point charges in the total energy
(section \ref{sec:geom}).

\subsection{Compute the SCF energy of H$_2$CO using ECPs for C and O}
\label{sec:sample3}

The following will compute the SCF energy for formaldehyde with ECPs
on the Carbon and Oxygen centers.

\begin{verbatim}
title; formaldehyde ECP deck

start ecpchho

geometry units au print
C         0.000000  0.000000 -1.025176
O         0.000000  0.000000  1.280289
H         0.000000  1.767475 -2.045628
H         0.000000 -1.767475 -2.045628
end

basis 
C  SP
  0.1675097360D+02 -0.7812840500D-01  0.3088908800D-01
  0.2888377460D+01 -0.3741108860D+00  0.2645728130D+00
  0.6904575040D+00  0.1229059640D+01  0.8225024920D+00
C  SP
  0.1813976910D+00  0.1000000000D+01  0.1000000000D+01
C  D
  0.8000000000D+00  0.1000000000D+01
C  F
  0.1000000000D+01  0.1000000000D+01
O  SP
  0.1842936330D+02 -0.1218775590D+00  0.5975796600D-01
  0.4047420810D+01 -0.1962142380D+00  0.3267825930D+00
  0.1093836980D+01  0.1156987900D+01  0.7484058930D+00
O  SP
  0.2906290230D+00  0.1000000000D+01  0.1000000000D+01
O  D
  0.8000000000D+00  0.1000000000D+01
O  F
  0.1100000000D+01  0.1000000000D+01
H  S
  0.1873113696D+02  0.3349460434D-01
  0.2825394365D+01  0.2347269535D+00
  0.6401216923D+00  0.8137573262D+00
H  S    1 1.00
  0.1612777588D+00  0.1000000000D+01
end

ecp
C nelec 2
C ul
        1       80.0000000       -1.60000000
        1       30.0000000       -0.40000000
        2        0.5498205       -0.03990210
C s
        0        0.7374760        0.63810832
        0      135.2354832       11.00916230
        2        8.5605569       20.13797020
C p
        2       10.6863587       -3.24684280
        2       23.4979897        0.78505765
O nelec 2
O ul
        1       80.0000000       -1.60000000
        1       30.0000000       -0.40000000
        2        1.0953760       -0.06623814
O s
        0        0.9212952        0.39552179
        0       28.6481971        2.51654843
        2        9.3033500       17.04478500
O p
        2       52.3427019       27.97790770
        2       30.7220233      -16.49630500
end

scf
vectors input hcore
maxiter 20
end

task scf
\end{verbatim}

This should produce the following output:

\begin{verbatim}
       Final RHF  results 
       ------------------ 

         Total SCF energy =    -22.507927218024
      One electron energy =    -71.508730162974
      Two electron energy =     31.201960019808
 Nuclear repulsion energy =     17.798842925142
\end{verbatim}


\chapter{Examples of geometries using symmetry}
\label{symexamples}

  \subsection{\protect$C_{2v}$ water}

\begin{verbatim}
  geometry
    O     0.00000000    0.00000000    0.00000000
    H     0.00000000    1.43042809   -1.10715266

    symmetry group c2v
  end
\end{verbatim}

  \subsection{\protect$D_{2h}$ acetylene}

Although acetylene has symmetry $D_{\infty h}$ the subgroup
$D_{2h}$ includes all operations that interchange equivalent atoms
which is what determines how much speedup you gain from using symmetry
in building a Fock matrix.

\begin{verbatim}
  geometry
    symmetry group d2h

    C      0.000000000    0.000000000   -1.115108538
    H      0.000000000    0.000000000   -3.106737425
  end
\end{verbatim}

  \subsection{\protect$T_d$ methane}

\begin{verbatim}
  geometry
    c   0.0000000      0.0000000      0.0000000
    h   1.1828637      1.1828637      1.1828637

    symmetry group Td
  end
\end{verbatim}

  \subsection{\protect$I_h$ buckminsterfullerene}

\begin{verbatim}
  geometry units angstrom # Bonds = 1.4445, 1.3945
    symmetry group Ih

    c   -1.2287651   0.0   3.3143121
  end
\end{verbatim}

  \subsection{\protect$S_4$ porphyrin}

\begin{verbatim}
  geometry units angstroms
     symmetry group s4
  
     fe                0.000  0.000  0.000         
     h                 2.242  6.496 -3.320   
     h                 1.542  4.304 -2.811
     c                 1.947  6.284 -2.433
     c                 1.568  4.987 -2.084
     h                 2.252  8.213 -1.695
     c                 1.993  7.278 -1.458
     h                 5.474 -1.041 -1.143
     c                 1.234  4.676 -0.765
     h                 7.738 -1.714 -0.606
     c                 0.857  3.276 -0.417
     h                 1.380 -4.889 -0.413
     c                 1.875  2.341 -0.234
     h                 3.629  3.659 -0.234
     c                 0.493 -2.964 -0.229
     c                 1.551 -3.933 -0.221
     c                 5.678 -1.273 -0.198
     c                 1.656  6.974 -0.144
     c                 3.261  2.696 -0.100
     n                 1.702  0.990 -0.035
  end
\end{verbatim}

  \subsection{\protect$D_{3h}$ iron penta-carbonyl}

\begin{verbatim}
  geometry
    symmetry group d3h

    fe        0.0         0.0         0.0

    c         0.0         0.0         3.414358
    o         0.0         0.0         5.591323

    c         2.4417087   2.4417087   0.0
    o         3.9810552   3.9810552   0.0
  end
\end{verbatim}

  \subsection{\protect$D_{3d}$ sodium crown ether}

  Note that the oxygen atom is rotated in the x-y plane 15
  degrees away from the y-axis so that it lies in a mirror
  plane.  There is a total of six atoms generated from the
  unique oxygen, in contrast to twelve from each of the carbon
  and hydrogen atoms.

\begin{verbatim}
  geometry print
    symmetry D3d

   NA     .0000000000    .0000000000   .0000000000
   O     1.3384771885   4.9952647969   .1544089284
   H     6.7342048019  -0.6723850379  2.6581562148
   C     6.7599180056  -0.4844977035   .6136583870
   H     8.6497577017   0.0709194071   .0345361934

  end
\end{verbatim}



\chapter{Running NWChem}
A more complete description should be available at 
\begin{verbatim}
   http://emsl.pnl.gov:2080/docs/nwchem/nwchem.html
\end{verbatim}

The command required to invoke NWChem is machine dependent, whereas
most of the NWChem input is machine independent\footnote{Machine
dependence within the input arises from file names, machine
specific resources, and differing services provided by the operating system.} .

\section{Sequential execution}

To run NWChem sequentially on nearly all UNIX-based platforms simply
use the command \verb+nwchem+ and provide the name of the input file
as an argument (section \ref{sec:inputstructure}).

Output is to standard output, standard error and Fortran unit 6
(usually the same as standard output).  Files are created by default
in the current directory, though this may be overridden in the input
(section \ref{sec:dirs}).

\section{Parallel execution on UNIX-based parallel machines
including workstation clusters using TCGMSG}
\label{sec:procgrp}

 These platforms require the use of the TCGMSG\footnote{Where required
TCGMSG is automatically built with NWChem.} \verb+parallel+ command
and thus also require the definition of a process-group (or procgroup)
file.  The process-group file describes how many processes to start,
what program to run, which machines to use, which directories to work
in, and under which userid to run the processes.  By convention the
process-group file has a \verb+.p+ suffix.

The process-group file is read to end-of-file.  The character \verb+#+
(hash or pound sign) is used to indicate a comment which continues to
the next new-line character.  Each line describes a cluster of
processes and consists of the following whitespace separated fields:

\begin{verbatim}
  userid hostname nslave executable workdir
\end{verbatim}

\begin{itemize}
\item \verb+userid+ -- The user-name on the machine that will be executing the
      process. 

\item \verb+hostname+ --  The hostname of the machine to execute this process.
             If it is the same machine on which parallel was invoked
             the name must match the value returned by the command 
             hostname. If a remote machine it must allow remote execution
             from this machine (see man pages for rlogin, rsh).

\item \verb+nslave+ --  The total number of copies of this process to be executing
             on the specified machine. Only ``clusters'' of identical processes
             specified in this fashion can use shared memory to communicate.
             If no shared memory is supported on machine \verb+<hostname>+ then
             only the value one (1) is valid.

\item \verb+executable+ --  Full path name on the host \verb+<hostname>+ of the image to execute.
             If \verb+<hostname>+ is the local machine then a local path will
             suffice.

\item \verb+workdir+ --  Full path name on the host \verb+<hostname>+ of the directory to
             work in. Processes execute a chdir() to this directory before
             returning from pbegin(). If specified as a ``.'' then remote
             processes will use the login directory on that machine and local
             processes (relative to where parallel was invoked) will use
             the current directory of parallel.
\end{itemize}

  For example, if your file \verb+"nwchem.p"+ contained the following
\begin{verbatim}
 d3g681 pc 4 /msrc/apps/bin/nwchem /scr22/rjh
\end{verbatim}
then 4 processes running NWChem would be started on the machine 
\verb+pc+ running as user \verb+d3g681+ in directory \verb+"/scr22/rjh"+.
To actually run this simply type:
\begin{verbatim}
  parallel nwchem big_molecule.nw
\end{verbatim}

{\em N.B.} : The first process specified (process zero) is the only
process that
\begin{itemize}
\item opens and reads the input file, and
\item opens and reads/updates the database.
\end{itemize}
Thus, if your file systems are physically distributed (e.g., most
workstation clusters) you must ensure that process zero can correctly
resolve the paths for the input and database files.

{\em N.B.} : If only one cluster is specified (one line in the
process-group file) then all processes execute NWChem.  If multiple
clusters are specified (multiple lines in the process-group file)
then one process out of each cluster is devoted to sharing
global-arrays between clusters, and therefore one more process than
desired the number of application processes should be specified in
each cluster.

\section{Parallel execution on MPPs}

All of these machines require use of different commands in order to
gain exclusive access to computational resources.

\section{Kendall Square Research}

\begin{verbatim}
  allocate_cells <n> parallel nwchem <input_file>
\end{verbatim}

The KSR command \verb+allocate_cells+ is used to acquire exclusive use
of a set of processors.  It takes the number of processors \verb+n+ and
the command as arguments.  The TCGMSG parallel command is described
above (section \ref{sec:procgrp}).  Note that when running the SCF
code optimal performance is obtained by allocating one more processor
to the processor set than required by your \verb+"nwchem.p"+
file\footnote{This is because dynamic load balanced is supported by
the process executing the command parallel which needs a dedicated
processor to do this efficiently.}.  For instance, if your
process-group file \verb+"nwchem32.p"+ read
\begin{verbatim}
  d3g681 circus 31 /usr/local/bin/nwchem /tmp/rjh
\end{verbatim}
then you might use the following command
\begin{verbatim}
  allocate_cells 32 parallel nwchem32 big_molecule.nw
\end{verbatim}

A useful tool for monitoring usage of the KSR is xringinfo.  See the
manual page for details.

\section{IBM SP}

If using POE (IBM's Parallel Operating Environment) interactively,
simply create the list nodes to use in the file \verb+"host.list"+ in
the current directory and invoke NWChem with
\begin{verbatim}
  nwchem <input_file> -procs <n>
\end{verbatim}
where \verb+n+ is the number of processes to use.  Process 0 will run
on the first node in \verb+"host.list"+ and must have access to the
input and other necessary files.  Very significant performance gains
may be had by setting the following environment variables before
running NWChem (or setting them using POE command line options).
\begin{itemize}
\item \verb+setenv MP_EUILIB us+ --- dedicated user space
  communication over the switch (the default is IP over the switch
  which is much slower).
\item \verb+setenv MP_CSS_INTERRUPT yes+ --- enable interrupts when a 
  message arrives (the default is to poll which significantly slows
  down global array accesses).
\end{itemize}

For batch execution, we recommend use of the \verb+llnw+ command which
is installed in \verb+/usr/local/bin+ on the EMSL/PNNL IBM SP.
Interactive help may be obtained with the command \verb+llnw -help+.
Otherwise, the very simplest job to run NWChem in batch using Load
Leveller is something like this
\begin{verbatim}
#!/bin/csh -x
# @ job_type         =    parallel
# @ class            =    small
# @ requirements     =    (Adapter == "hps_user")
# @ input            =    /dev/null
# @ output           =    <OUTPUT_FILE_NAME>
# @ error            =    <ERROUT_FILE_NAME>
# @ environment      =    COPY_ALL; MP_EUILIB=us ; MP_CSS_INTERRUPT=yes
# @ min_processors   =    7
# @ max_processors   =    7
# @ cpu_limit        =    1:00:00
# @ queue
#

cd /scratch

nwchem <INPUT_FILE_NAME>
\end{verbatim}

Substitute \verb+<OUTPUT_FILE_NAME>+, \verb+<ERROUT_FILE_NAME>+ and
\verb+<INPUT_FILE_NAME>+ with the {\em full} path of the appropriate
files.  These files and the NWChem executable must be in a file system
accessible to all processes.  Put the above into a file (e.g.,
\verb+"test.job"+) and submit it with the command
\begin{verbatim}
  llsubmit test.job
\end{verbatim}
It will run a 7 processor, 1 hour job in the queue \verb+small+.  It
should be apparent how to change these values.

Unfortunately, this simple job becomes very inefficient when running
on many nodes (taking up to 15 minutes to commence execution) because
POE saturates networked file systems when copying the executable to
all of the nodes.  There is a script \verb+llnw+ that may be invoked
either as a one line command or with interactive prompting that
automates job creation and submission, and, by efficient copying of the
executable, reduces startup time to about 1 minute.

Note that on many IBM SPs, including that at EMSL, the local scratch
disks are wiped clean at the beginning of each job and therefore
persistent files should be stored elsewhere.  PIOFS is recommended for
files larger than 1--2 MB.

\section{Intel Paragon}

\begin{verbatim}
  nwchem -sz <n> <input_file>
\end{verbatim}

or if pexec is used (e.g., at ORNL)

\begin{verbatim}
  pexec nwchem <input_file> -sz <n>
\end{verbatim}

where \verb+n+ is the number of processors and \verb+input_file+ is the
name of your input file.


\section{Intel Touchstone Delta}

\begin{verbatim}
  mexec -t"(<rows>,<cols>)" -f "nwchem <input_file>"
\end{verbatim}

where \verb+rows+ and \verb+cols+ specify the dimensions of the
processor mesh and \verb+input_file+ is the name of your input file.
For example, to run using all 512 nodes on the Delta
\begin{verbatim}
  mexec -t"(16,32)" -f "nwchem big_molecule.nw"
\end{verbatim}

\section{Cray T3D}

\begin{verbatim}
  nwchem  <input_file> -npes <n>
\end{verbatim}

where \verb+n+ is the number of processors and \verb+input_file+ is the
name of your input file.

When compiling NWChem on the Cray T3D, you need to setup the
environmental variable {\tt TARGET} for the correct cross-compilation
of C routines by typing
\begin{verbatim}
  setenv TARGET CRAY-T3D
\end{verbatim}

\section{Tested Platforms and O/S versions}

\begin{itemize}
\item KSR-2 
\item Intel Delta 
\item Intel Paragon 
\item IBM SP1 and SP2, AIX 3.2 and 4.1.
\item Cray T3D
\item SGI R8000
\item SGI R4000
\item IBM RS6000, AIX 3.2 and 4.1.
\item SUN workstations, SunOS 4.1.3 and Solaris 5.5
\item x86 computers running Linux 1.2.13 works
\end{itemize}


\end{document}
