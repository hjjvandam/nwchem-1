%
% $Id$
%
\label{sec:rel}
All methods which include treatment of relativistic effects are ultimately
based on the Dirac equation, which has a four component wave function. The
solutions to the Dirac equation describe both positrons (the ``negative
energy'' states) and electrons (the ``positive energy'' states), as well as
both spin orientations, hence the four components. The wave function may be
broken down into two-component functions traditionally known as the large
and small components; these may further be broken down into the spin
components. 

The implementation of approximate all-electron relativistic methods in
quantum chemical codes requires the removal of the negative energy states
and the factoring out of the spin-free terms. Both of these may be achieved
using a transformation of the Dirac Hamiltonian known in general as a
Foldy-Wouthuysen transformation. Unfortunately this transformation cannot be
represented in closed form for a general potential, and must be
approximated.  One popular approach is that originally formulated by Douglas
and Kroll\footnote{M.~Douglas and N.~M.~Kroll, Ann. Phys. (N.Y.)  {\bf 82},
89 (1974)} and developed by Hess\footnote{B.A.~Hess, Phys.~Rev.~A~{\bf 32},
756 (1985); {\bf 33}, 3742 (1986)}. This approach decouples the positive and
negative energy parts to second order in the external potential (and also
fourth order in the fine structure constant, $\alpha$). Other approaches include 
the Zeroth Order Regular Approximation (ZORA)\footnote{C.~Chang, M.~Pelissier, 
M.~Durand, Physica Scripta ~{\bf 34}, 294 (1986); E.~van Lenthe, ~{\it The ZORA Equation}, 
doctoral thesis, Vrije Universiteit, Amsterdam (1996); S.~Faas, J.G.~Snijders, 
J.H.~van Lenthe, E.~van Lenthe, and E.J.~Baerends, Chem.~Phys.~ Lett.~{\bf 246}, 632 (1995).}
and modification of the Dirac equation by Dyall\footnote{K.~G.~Dyall,
J.~Chem.~Phys.~{\bf 100}, 2118 (1994)}, and involves an exact FW
transformation on the atomic basis set level\footnote{K.~G.~Dyall,
J.~Chem.~Phys.~{\bf 106}, 9618 (1997); K.~G.~Dyall and T.~Enevoldsen,
J.~Chem.~Phys.~{\bf 111}, 10000 (1999).}.

Since these approximations only modify the integrals, they can in principle
be used at all levels of theory. At present the Douglas-Kroll and ZORA 
implementations can be used at all levels of theory whereas 
Dyall's approach is currently available at the Hartree-Fock level. 
The derivatives have been implemented, allowing both methods to be used in 
geometry optimizations and frequency calculations.

The \verb+RELATIVISTIC+ directive provides input for the implemented relativistic 
approximations and is a compound directive that encloses additional directives 
specific to the approximations:
\begin{verbatim}
  RELATIVISTIC
   [DOUGLAS-KROLL [<string (ON||OFF) default ON> \
                 <string (FPP||DKH||DKFULL||DK3||DK3FULL) default DKH>]  ||
    ZORA [ (ON || OFF) default ON ] || 
    DYALL-MOD-DIRAC [ (ON || OFF) default ON ] 
                  [ (NESC1E || NESC2E) default NESC1E ] ]
   [CLIGHT <real clight default 137.0359895>]
  END
\end{verbatim}

Only one of the methods may be chosen at a time.  If both methods are found
to be on in the input block, NWChem will stop and print an error message.
There is one general option for both methods, the definition of the speed 
of light in atomic units:

\begin{verbatim}
  CLIGHT <real clight default 137.0359895>
\end{verbatim}

The following sections describe the optional sub-directives that
can be specified within the \verb+RELATIVISTIC+ block.

\section{Douglas-Kroll approximation}
\label{sec:douglas-kroll}

The spin-free and spin-orbit one-electron Douglas-Kroll 
approximation have been implemented. The use of relativistic effects 
from this Douglas-Kroll approximation can be invoked by specifying:

\begin{verbatim}
  DOUGLAS-KROLL [<string (ON||OFF) default ON> \
                 <string (FPP||DKH||DKFULL|DK3|DK3FULL) default DKH>]
\end{verbatim}

The \verb+ON|OFF+ string is used to turn on or off the
Douglas-Kroll approximation.  By default, if the \verb+DOUGLAS-KROLL+
keyword is found, the approximation will be used in the calculation.
If the user wishes to calculate a non-relativistic quantity after turning
on Douglas-Kroll, the user will need to define a new \verb+RELATIVISTIC+
block and turn the approximation \verb+OFF+.  The user could also simply
put a blank \verb+RELATIVISTIC+ block in the input file and all options 
will be turned off.

The \verb+FPP+ is the approximation based on free-particle projection 
operators\footnote{B.A.~Hess, Phys.~Rev.~A~{\bf 32}, 756 (1985)} whereas the 
\verb+DKH+ and \verb+DKFULL+ approximations are based on external-field 
projection operators\footnote{B.A.~Hess, Phys.~Rev.~A~{\bf 33}, 3742 (1986)}.
The latter two are considerably better approximations than the former. \verb+DKH+ 
is the Douglas-Kroll-Hess approach and is the approach that is generally 
implemented in quantum chemistry codes. \verb+DKFULL+ includes certain 
cross-product integral terms ignored in the \verb+DKH+ approach (see for example 
H\"{a}berlen and R\"{o}sch\footnote{O.D.~H\"{a}berlen, N.~R\"{o}sch, 
Chem.~Phys.~Lett.~{\bf 199}, 491 (1992)}). The third-order Douglas-Kroll 
approximation has been implemented by T. Nakajima and K. Hirao\footnote{T. Nakajima 
and K. Hirao, Chem.~Phys.~Lett.~{\bf 329}, 5111 (2000); T. Nakajima and K. Hirao, 
J.~Chem.~Phys.~{\bf 113}, 7786 (2000)}. This approximation can be called using
\verb+DK3+ (DK3 without cross-product integral terms) or \verb+DK3FULL+ (DK3 with
cross-product integral terms).

The contracted basis sets used in the calculations should reflect the relativistic
effects, i.e. one should use contracted basis sets which were generated using the 
Douglas-Kroll Hamiltonian. Basis sets that were contracted using the 
non-relativistic (Sch\"{o}dinger) Hamiltonian WILL PRODUCE ERRONEOUS RESULTS for
elements beyond the first row. See appendix \ref{sec:knownbasis} for available
basis sets and their naming convention.

NOTE: we suggest that spherical basis sets are used in the calculation. The use of 
high quality cartesian basis sets can lead to numerical inaccuracies.

In order to compute the integrals needed for the Douglas-Kroll approximation
the implementation makes use of a fitting basis set (see literature given
above for details). The current code will create this fitting basis set
based on the given {\tt "ao basis"} by simply uncontracting that basis. This
again is what is commonly implemented in quantum chemistry codes that
include the Douglas-Kroll method.  Additional flexibility is available to
the user by explicitly specifying a Douglas-Kroll fitting basis
set. This basis set must be named {\tt "D-K basis"} (see Chapter
\ref{sec:basis}).

\section{Zeroth Order regular approximation (ZORA)}
\label{sec:zora}

The spin-free and spin-orbit one-electron zeroth-order regular approximation (ZORA) 
have been implemented. The use of relativistic effects with ZORA 
can be invoked by specifying:

\begin{verbatim}
  ZORA [<string (ON||OFF) default ON>
\end{verbatim}

The \verb+ON|OFF+ string is used to turn on or off ZORA.  
By default, if the \verb+ZORA+ keyword is found, the approximation 
will be used in the calculation. If the user wishes to calculate 
a non-relativistic quantity after turning on ZORA, the user 
will need to define a new \verb+RELATIVISTIC+ block and turn 
the approximation \verb+OFF+.  The user can also simply put 
a blank \verb+RELATIVISTIC+ block in the input file and all options 
will be turned off.

\section{Dyall's Modified Dirac Hamitonian approximation}
\label{sec:dyall-mod-dir}

The approximate methods described in this section are all based on Dyall's
modified Dirac Hamiltonian. This Hamiltonian is entirely equivalent to the
original Dirac Hamiltonian, and its solutions have the same properties.
The modification is achieved by a transformation on the small component,
extracting out \hbox{$\sigma\cdot{\bf p}/2mc$}. This gives the modified small
component the same symmetry as the large component, and in fact it differs
from the large component only at order $\alpha^2$.  The advantage of the
modification is that the operators now resemble the operators of the
Breit-Pauli Hamiltonian, and can be classified in a similar fashion into
spin-free, spin-orbit and spin-spin terms. It is the spin-free terms which
have been implemented in NWChem, with a number of further approximations.

The first is that the negative energy states are removed by a normalized
elimination of the small component (NESC), which is equivalent to an exact
Foldy-Wouthuysen (EFW) transformation. The number of components in the wave
function is thereby effectively reduced from 4 to 2. NESC on its own does
not provide any advantages, and in fact complicates things because the
transformation is energy-dependent. The second approximation therefore
performs the elimination on an atom-by-atom basis, which is equivalent to
neglecting blocks which couple different atoms in the EFW transformation.
The advantage of this approximation is that all the energy dependence can be
included in the contraction coefficients of the basis set.  The tests which
have been done show that this approximation gives results well within
chemical accuracy. The third approximation neglects the commutator of the
EFW transformation with the two-electron Coulomb interaction, so that the
only corrections that need to be made are in the one-electron integrals.
This is the equivalent of the Douglas-Kroll(-Hess) approximation as it is
usually applied.

The use of these approximations can be invoked with the use of the
\verb+DYALL-MOD-DIRAC+ directive in the \verb+RELATIVISTIC+ directive block.
The syntax is as follows.

\begin{verbatim}
  DYALL-MOD-DIRAC [ (ON || OFF) default ON ] 
                  [ (NESC1E || NESC2E) default NESC1E ]
\end{verbatim}

The \verb+ON|OFF+ string is used to turn on or off the
Dyall's modified Dirac approximation. By default, if the \verb+DYALL-MOD-DIRAC+
keyword is found, the approximation will be used in the calculation.
If the user wishes to calculate a non-relativistic quantity after turning
on Dyall's modified Dirac, the user will need to define a new 
\verb+RELATIVISTIC+
block and turn the approximation \verb+OFF+.  The user could also simply
put a blank \verb+RELATIVISTIC+ block in the input file and all options 
will be turned off.

Both one- and two-electron approximations are available
\verb+NESC1E || NESC2E+, and both have
analytic gradients. The one-electron approximation is the default.
The two-electron approximation specified by \verb+NESC2E+ has some sub
options which are placed on the same logical line as the
\verb+DYALL-MOD-DIRAC+ directive, with the following syntax:

\begin{verbatim}
  NESC2E [ (SS1CENT [ (ON || OFF) default ON ] || SSALL) default SSALL ]
         [ (SSSS [ (ON || OFF) default ON ] || NOSSSS) default SSSS ]
\end{verbatim}

The first sub-option gives the capability to limit the two-electron
corrections to those in which the small components in any density must be on
the same center.  This reduces the $(LL|SS)$ contributions to at most
three-center integrals and the $(SS|SS)$ contributions to two centers. For a
case with only one relativistic atom this option is redundant. The second
controls the inclusion of the $(SS|SS)$ integrals which are of order
$\alpha^4$. For light atoms they may safely be neglected, but for heavy
atoms they should be included. 

In addition to the selection of this keyword in the \verb+RELATIVISTIC+
directive block, it is necessary to supply basis sets in addition to the
\verb+ao basis+. For the one-electron approximation, three basis sets are
needed: the atomic FW basis set, the large component basis set and the small
component basis set. The atomic FW basis set should be included in the
\verb+ao basis+.
The large and small components should similarly be incorporated
in basis sets named \verb+large component+ and \verb+small component+,
respectively. For the two-electron approximation, only two basis sets are
needed. These are the large component and the small component. The large component
should be included in the \verb+ao basis+ and the small component
is specified separately as \verb+small component+, as for the one-electron
approximation. This means that the two approximations can {\it not} be run
correctly without changing the \verb+ao basis+, and it is up to the user to
ensure that the basis sets are correctly specified.

There is one further requirement in the specification of the basis sets. In
the \verb+ao basis+, it is necessary to add the \verb+rel+ keyword either to the
\verb+basis+ directive or the library tag line (See below for examples). 
The former marks the basis
functions specified by the tag as relativistic, the latter marks the whole
basis as relativistic. The marking is actually done at the unique shell
level, so that it is possible not only to have relativistic and
nonrelativistic atoms, it is also possible to have relativistic and
nonrelativistic shells on a given atom. This would be useful, for example,
for diffuse functions or for high angular momentum correlating functions,
where the influence of relativity was small. The marking of shells as
relativistic is necessary to set up a mapping between the ao basis and the
large and/or small component basis sets. For the one-electron approximation
the large and small component basis sets MUST be of the same size and
construction, i.e. differing only in the contraction coefficients.

It should also be noted that the relativistic code will NOT work with basis
sets that contain sp shells, nor will it work with ECPs. Both of these are
tested and flagged as an error.

Some examples follow. The first example sets up the data for relativistic
calculations on water with the one-electron approximation and the
two-electron approximation, using the library basis sets.

\begin{verbatim}
  start h2o-dmd

  geometry units bohr
  symmetry c2v
    O       0.000000000    0.000000000   -0.009000000
    H       1.515260000    0.000000000   -1.058900000
    H      -1.515260000    0.000000000   -1.058900000
  end

  basis "fw" rel
    oxygen library cc-pvdz_pt_sf_fw
    hydrogen library cc-pvdz_pt_sf_fw
  end

  basis "large"
    oxygen library cc-pvdz_pt_sf_lc
    hydrogen library cc-pvdz_pt_sf_lc
  end

  basis "large2" rel
    oxygen library cc-pvdz_pt_sf_lc
    hydrogen library cc-pvdz_pt_sf_lc
  end

  basis "small"
    oxygen library cc-pvdz_pt_sf_sc
    hydrogen library cc-pvdz_pt_sf_sc
  end

  set "ao basis" fw
  set "large component" large
  set "small component" small

  relativistic
    dyall-mod-dirac
  end

  task scf

  set "ao basis" large2
  unset "large component"
  set "small component" small

  relativistic
    dyall-mod-dirac nesc2e
  end

  task scf
\end{verbatim}

The second example has oxygen as a relativistic atom and hydrogen nonrelativistic.

\begin{verbatim}
  start h2o-dmd2

  geometry units bohr
  symmetry c2v
    O       0.000000000    0.000000000   -0.009000000
    H       1.515260000    0.000000000   -1.058900000
    H      -1.515260000    0.000000000   -1.058900000
  end

  basis "ao basis"
    oxygen library cc-pvdz_pt_sf_fw rel
    hydrogen library cc-pvdz
  end

  basis "large component"
    oxygen library cc-pvdz_pt_sf_lc
  end

  basis "small component"
    oxygen library cc-pvdz_pt_sf_sc
  end

  relativistic
    dyall-mod-dirac
  end

  task scf
\end{verbatim}
