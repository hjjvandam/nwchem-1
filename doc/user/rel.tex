\label{sec:rel}



All methods which include treatment of relativistic effects are ultimately
based on the Dirac equation, which has a four component wave function. The
solutions to the Dirac equation describe both positrons (the ``negative
energy'' states) and electrons (the ``positive energy'' states), as well as
both spin orientations, hence the four components. The wave function may be
broken down into two-component functions traditionally known as the large
and small components; these may further be broken down into the spin
components. 

The implementation of approximate all-electron relativistic methods in
quantum chemical codes requires the removal of the negative energy states
and the factoring out of the spin-free terms. Both of these may be achieved
using a transformation of the Dirac Hamiltonian known in general as a
Foldy-Wouthuysen transformation. Unfortunately this transformation cannot be
represented in closed form for a general potential, and must be
approximated.  One popular approach is that originally forulated by Douglas
and Kroll\footnote{M.~Douglas and N.~M.~Kroll, Ann. Phys. (N.Y.)  {\bf 82},
89 (1974)} and developed by Hess\footnote{B.A.~Hess, Phys.~Rev.~A~{\bf 32},
756 (1985); {\bf 33}, 3742 (1986)}. This approach decouples the positive and
negative energy parts to second order in the external potential (and also
fourth order in the fine structure constant, $\alpha$). Another approach is
based on a modification of the Dirac equation by Dyall\footnote{K.~G.~Dyall,
J.~Chem.~Phys.~{\bf 100}, 2118 (1994)}, and involves an exact FW
transformation on the atomic basis set level\footnote{K.~G.~Dyall,
J.~Chem.~Phys.~{\bf 106}, 9618 (1997); K.~G.~Dyall and T.~Enevoldsen,
J.~Chem.~Phys.~{\bf 111}, 10000 (1999).}.

Since these approximations only modify the integrals, they can in principle
be used at all levels of theory. The derivatives have been implemented, allowing
them to be used for geometry optimizations and frequency calculations.

The \verb+RELATIVISTIC+ directive provides input for the implemented relativistic 
approximations and is a compound directive that encloses additional directives 
specific to the approximations:

\begin{verbatim}
  RELATIVISTIC
    ...
  END
\end{verbatim}

There is one general option, the definition of the speed of light in atomic units:

\begin{verbatim}
  CLIGHT <real clight default 137.0359895>
\end{verbatim}

\section{Douglas-Kroll approximation}
\label{sec:douglas-kroll}


The (spin-free) one-electron Douglas-Kroll approximation has been
implemented. The use of relativistic effects from this Douglas-Kroll
approximation can be invoked by specifying:

\begin{verbatim}
  DOUGLAS-KROLL [<string (ON||OFF) default ON> \
                 <string (FPP||DKH||DKFULL) default DKH>]

\end{verbatim}

The \verb+FPP+ is the approximation based on free-particle projection 
operators\footnote{B.A.~Hess, Phys.~Rev.~A~{\bf 32}, 756 (1985)} whereas the 
\verb+DKH+ and \verb+DKFULL+ approximations are based on external-field 
projection operators\footnote{B.A.~Hess, Phys.~Rev.~A~{\bf 33}, 3742 (1986)}.
The latter two are considerably better approximations than the former. \verb+DKH+ 
is the Douglas-Kroll-Hess approach and is the approach that is generally 
implemented in quantum chemistry codes. \verb+DKFULL+ includes certain 
cross-product integral terms ignored in the \verb+DKH+ approach (see for example 
H\"{a}berlen and R\"{o}sch\footnote{O.D.~H\"{a}berlen, N.~R\"{o}sch, 
Chem.~Phys.~Lett.~{\bf 199}, 491 (1992)}).

The contracted basis sets used in the calculations should reflect the relativistic
effects. I.e. one should use contracted basis sets which were generated using the 
Douglas-Kroll Hamiltonian. Basis sets that were contracted using the 
non-relativistic (Sch\"{o}dinger Hamiltonian WILL PRODUCE ERRONEOUS RESULTS for
elements beyond the first row. See chapter \ref{sec:knownbasis} for available
basis sets and their naming convention.

In order to compute the integrals needed for the Douglas-Kroll approximation
the implementation makes use of a fitting basis set (see literature given
above for details). The current code will create this fitting basis set
based on the given {\tt "ao basis"} by simply uncontracting that basis. This
again is what is commonly implemented in quantum chemistry codes that
include the Douglas-Kroll method.  Additional flexibility is available to
the user having by explicitly specifying a Douglas-Kroll fitting basis
set. This basis set must be named {\tt "D-K basis"} (see Chapter
\ref{sec:basis}).

\section{Dyall's Modified Dirac Hamitonian approximation}
\label{sec:dyall-mod-dir}


The approximate methods described in this section are all based on Dyall's
modified Dirac Hamiltonian. This Hamiltonian is entirely equivalent to the
original Dirac Hamiltonian, and its solutions have the same properties.
The modification is achieved by a transformation on the small component,
extracting out $\sigma\cdot{\bf p}/2mc$. This gives the modified small
component the same symmetry as the large component, and in fact it differs
from the large component only at order $\alpha^2$.  The advantage of the
modification is that the operators now resemble the operators of the
Breit-Pauli Hamiltonian, and can be classified in a similar fashion into
spin-free, spin-orbit and spin-spin terms. It is the spin-free terms which
have been implemented in NWChem, with a number of further approximations.

The first is that the negative energy states are removed by a normalized
elimination of the small component (NESC), which is equivalent to an exact
Foldy-Wouthuysen (EFW) transformation. The number of components in the wave
function is thereby effectively reduced from 4 to 2. NESC on its own does
not provide any advantages, and in fact complicates things because the
transformation is energy-dependent. The second approximation therefore
performs the elimination on an atom-by-atom basis, which is equivalent to
neglecting blocks which couple different atoms in the EFW transformation.
The advantage of this approximation is that all the energy dependence can be
included in the contraction coefficients of the basis set.  The tests which
have been done show that this approximation gives results well within
chemical accuracy. The third approximation neglects the commutator of the
EFW transformation with the two-electron Coulomb interaction, so that the
only corrections that need to be made are in the one-electron integrals.
This is the equivalent of the Douglas-Kroll(-Hess) approximation as it is
usually applied.

The use of these approximations can be invoked with the use of the
\verb+dyall-mod-dirac+ directive in the \verb+relativistic+ directive block.
The syntax is as follows.

\begin{verbatim}
  DYALL-MOD-DIRAC [ (ON || OFF) default ON ] 
                  [ (NESC1E || NESC2E) default NESC1E ]

\end{verbatim}

Both one- and two-electron approximations are available, and both have
analytic gradients. The one-electron approximation is the default.
The two-electron approximation specified by \verb+NESC2E+ has some sub
options, with the following syntax:

\begin{verbatim}
  NESC2E [ (SS1CENT [ (ON || OFF) default ON ] || SSALL) default SSALL ]
         [ (SSSS [ (ON || OFF) default ON ] || NOSSSS) default SSSS ]

\end{verbatim}

The first sub-option gives the capability to limit the two-electron
corrections to those in which the small components in any density must be on
the same center.  This reduces the $(LL|SS)$ contributions to at most
three-center integrals and the $(SS|SS)$ contributions to two centers. For a
case with only one relativistic atom this option is redundant. The second
controls the inclusion of the $(SS|SS)$ integrals which are of order
$\alpha^4$. For light atoms they may safely be neglected, but for heavy
atoms they should be included. 

In addition to the selection of this keyword in the \verb+RELATIVISTIC+
directive block, it is necessary to supply basis sets in addition to the
\verb+ao basis+. For the one-electron approximation, three basis sets are
needed: the atomic FW basis set, the large component basis set and the small
component basis set. The atomic FW basis set should be included in the
\verb+ao basis+, either directly or by indirection, using \verb+SET+
directives.  The large and small components should similarly be incorporated
in basis sets named \verb+large component+ and \verb+small component+,
respectively. For the two-electron approximation, it is the large component
which should be included in the verb+ao basis+ and only the small component
need be specified separately.

There is one further requirement in the specification of the basis sets. In
the ao basis, it is necessary to add the \verb+rel+ keyword either to the
tag line, or on the \verb+basis+ directive. The former marks the basis
functions specified by the tag as relativistic, the latter marks the whole
basis as relativistic. The marking is actually done at the unique shell
level, so that it is possible not only to have relativistic and
nonrelativistic atoms, it is also possible to have relativistic and
nonrelativistic shells on a given atom. This would be useful, for example,
for diffuse functions or for high angular momentum correlating functions,
where the influence of relativity was small. The marking of shells as
relativistic is necessary to set up a mapping between the ao basis and the
large and/or small component basis sets. For the one-electron approximation
the large and small component basis sets MUST be of the same size and
construction, i.e. differing only in the contraction coefficients.

It should also be noted that the relativistic code will NOT work with basis
sets that contain sp shells, nor will it work with ECPs. Both of these are
tested and flagged as an error.

Some examples follow. The first example sets up the data for a relativistic
calculation on water, using the library basis sets.

\begin{verbatim}
  basis "ao basis" rel
    oxygen library cc-pvdz_pt_sf_fw
    hydrogen library cc-pvdz_pt_sf_fw
  end

  basis "large component"
    oxygen library cc-pvdz_pt_sf_lc
    hydrogen library cc-pvdz_pt_sf_lc
  end

  basis "small component"
    oxygen library cc-pvdz_pt_sf_sc
    hydrogen library cc-pvdz_pt_sf_sc
  end

  relativistic
    dyall-mod-dirac
  end
\end{verbatim}

The second example has oxygen as a relativistic atom and hydrogen nonrelativistic.

\begin{verbatim}
  basis "ao basis"
    oxygen library cc-pvdz_pt_sf_fw rel
    hydrogen library cc-pvdz
  end

  basis "large component"
    oxygen library cc-pvdz_pt_sf_lc
  end

  basis "small component"
    oxygen library cc-pvdz_pt_sf_sc
  end

  relativistic
    dyall-mod-dirac
  end
\end{verbatim}
