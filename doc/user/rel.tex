\label{sec:rel}

General introduction to all-electron relativistic effects.

The currently implemented all-electron relativistic approximations are Douglas-Kroll and Ken's approx. They can be used at all levels of theory. The derivatives are implemented allowing them to be used for geometry optimizations and frequency calculations.

The \verb+RELATIVISTIC+ directive provides input for the implemented relativistic 
approximations and is a compound directive that encloses additional directives 
specific to the approximations:

\begin{verbatim}
  RELATIVISTIC
    ...
  END
\end{verbatim}

A general option which is currently only used by the Douglas-Kroll 
approximation is the definition of the speed of light:

\begin{verbatim}
  CLIGHT <real clight default 137.0359895>
\end{verbatim}

\section{Douglas-Kroll approximation}
\label{sec:douglas-kroll}

The (spin-free) one-electron Douglas-Kroll approximation\footnote{B.A.~Hess, 
Phys.~Rev.~A~{\bf 32}, 756 (1985) and B.A.~Hess, Phys.~Rev.~A~{\bf 33}, 3742 
(1986)} has been implemented. The use of relativistic effects from this 
Douglas-Kroll approximation can be invoked by 
specifying:

\begin{verbatim}
  DOUGLAS-KROLL [<string (ON||OFF) default ON> \
                 <string (FPP||DKH||DKFULL) default DKH>]

\end{verbatim}

The \verb+FPP+ is the approximation based on free-particle projection 
operators\footnote{B.A.~Hess, Phys.~Rev.~A~{\bf 32}, 756 (1985)} whereas the 
\verb+DKH+ and \verb+DKFULL+ approximations are based on external-field 
projection operators\footnote{B.A.~Hess, Phys.~Rev.~A~{\bf 33}, 3742 (1986)}.
The latter two are considerably better approximations than the former. \verb+DKH+ 
is the Douglas-Kroll-Hess approach and is the approach that is generally 
implemented in quantum chemistry codes. \verb+DKFULL+ includes certain 
cross-product integral terms ignored in the \verb+DKH+ approach (see for example 
H\"{a}berlen and R\"{o}sch\footnote{O.D.~H\"{a}berlen, N.~R\"{o}sch, 
Chem.~Phys.~Lett.~{\bf 199}, 491 (1992)}).

In order to compute the integrals needed for the Douglas-Kroll approximation 
Douglas-Kroll makes use of a fitting basis set (see literature given above 
for details). The current code will create this fitting basis set based on 
the given {\tt "ao basis"} by simply uncontracting that basis. This again what is 
commonly implemented in quantum chemistry codes that include Douglas-Kroll.
Additional flexibility is available to the user having by explicitly specifying
a Douglas-Kroll fitting basis set. This basis set must be named {\tt "D-K basis"}
(see Chapter \ref{sec:basis}).

\section{Ken's approx section}
\label{sec:ken's approx section}

