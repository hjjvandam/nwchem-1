\label{sec:ecp}

This directive describes an effective core potential (ECP) basis set
of contracted gaussian functions.  The ECPs are fit to Gaussians 
with the form 
\[
 r^2V_l(r) = \sum_{k} A_{lk} r^{n_{lk}} e^{B_{lk}r^{2}}
\]
where $A_{lk}$ is the contraction coefficient, $n_{lk}$ is the
exponent of the ``r'' term (r-exponent), and $B_{lk}$ is the gaussian
exponent.  The r-exponent is shifted by 2 as per most of the ECP
literature, e.g., an r-exponent of 0 implies $r^{-2}$.

By default ECP basis sets are automatically segmented and cartesian
even if general contractions are input. 

ECP basis functions are associated with centers in geometries through
the tags or names of centers which must match exactly (including case)
and are limited to sixteen (16) characters.  Each center with the same
tag will have the same ECP basis set.  By default the input module
prints each ECP basis set encountered; use the \verb+NOPRINT+ option
to disable printing.  There can be only one active ECP basis set even
though several may exist in the input deck.  The ECP modules load
``ecp basis'' with any ``ao basis'' present.  The ECP functionality
works for energy and gradients.

In the same fashion as for geometries or regular basis sets, ECP basis
sets are named, with the default name being \verb+"ecp basis"+.  It
should be clear from the above discussion on geometries and database
entries how indirection is supported.

Basis functions currently may not be drawn from a standard set in the
EMSL basis set library; they must be specified explicitly.  All
directives that are in common with the standard gaussian basis set
input have the same function and syntax.

The keyword pairs

\begin{verbatim}
  spherical || cartesian
  segment || nosegment
  print || noprint
\end{verbatim}

are interpreted in the \verb+ECP+ directive in the same manner as for
the \verb+BASIS+ directive, but only the defaults are currently
available for the coordinate system and segmentation in the \verb+ECP+
directive.  NWChem assumes that all ECP basis sets are segmented and
cartesian.  The keywords are included in the directive, however,
because it is expected that eventually the code will include the
options for the user to specify basis functions in either spherical or
cartesian coordinates, segmented or unsegmented.  The print keyword is
currently active, and can be used to specify that the descriptions of
the functions will be printed or not printed by the input module, at
the user's discretion.
% , whether specified by the 
% user or defined in the standard library sets.

As with the input for the standard basis sets using the \verb+BASIS+
directive, the input specified for the \verb+ECP+ directive in lines 
beginning with the string \verb+tag+
allow particular centers or atoms in a calculation to be associated with
particular basis functions.  The values specified for \verb+tag+
must correspond exactly with the names supplied for the \verb+tag+ entries
on the \verb+GEOMETRY+ directive for a particular calculation.  Each atom
or center with the same \verb+tag+ will have the same basis set, which must
also be specified with the same name \verb+tag+.

The keyword \verb+NELEC+ allows the user to specify the number of core 
electrons replaced by
the ECP basis specification for the atom represented by the tag.  Additional
input lines can then be used to define the specific coefficients.
The string \verb+shell_type+ is used to specify the components of the
ECP basis function.  The label \verb+ul+ entered for \verb+shell_type+
for a given atom or center (identified by the string \verb+tag+) denotes
the local part of the ECP basis.  This is equivalent to the highest 
angular momentum
functions specified in the literature for most ECP basis sets.  The
standard entries (\verb+s, p, d+, etc.) for \verb+shell_type+ delineate 
the angular momentum projector onto the local function.  The shell type 
label of \verb+s+ indicates the \verb+ul-s+ projector input, \verb+p+ 
indicates the \verb+ul-p+, etc.

An application of the \verb+ECP+ directive is illustrated in the following 
example for the molecule  H$_2$CO.  This input defines an ECP basis set 
for the  carbon and oxygen atoms in the molecule.

% \centerline{{\bf H$_2$CO }}

\begin{verbatim}
  ecp print  
    C nelec 2     # ecp replaces 2 electrons on C
    C ul    # d
            1       80.0000000       -1.60000000
            1       30.0000000       -0.40000000
            2        0.5498205       -0.03990210
   C s     # s - d 
            0        0.7374760        0.63810832
            0      135.2354832       11.00916230
            2        8.5605569       20.13797020
    C p     # p - d
            2       10.6863587       -3.24684280
            2       23.4979897        0.78505765
    O nelec 2     # ecp replaces 2 electrons on O
    O ul    # d 
            1       80.0000000       -1.60000000
            1       30.0000000       -0.40000000
            2        1.0953760       -0.06623814
    O s     # s - d
            0        0.9212952        0.39552179
            0       28.6481971        2.51654843
            2        9.3033500       17.04478500
    O p     # p - s 
            2       52.3427019       27.97790770
            2       30.7220233      -16.49630500
  end
\end{verbatim}

Generally contracted ECP basis sets are not in wide use, but the
functionality has been included in NWChem for applications where they
might be useful. 

