\subsection{Hartree-Fock or SCF Gradients}

\Large
***This should be part of the SCF section (Section 8, in the current
numbering; it doesn't belong all on its own as a separate Section 9),
 since
it appears to be part of the SCF module input.****
\normalsize

The input for this directive allows the user to define some important
characteristics of the Hartree-Fock gradients for the SCF, UHF and ROHF
calculations.  The form of the directive is as follows;

\begin{verbatim}
  GRADIENTS 
    [chkpt <integer minutes>]
    [restart]
    [print || noprint]
  END
\end{verbatim}

%  This input controls the Hartree-Fock (SCF, UHF and ROHF) gradients.  

The directive contains two keywords, \verb+chkpt+ and \verb+restart+,
that are related to the creation of the gradients.  The keyword \verb+chkpt+
allows the user to specify a time interval at which the current values
for the forces that make up 
the gradient are saved, for access by a later calculation.  The time
interval is specified in the integer variable \verb+minutes+, and defines
the number of minutes of elapsed wall-clock time since the start of the
calculation when the gradient is written to the runtime database.

% \subsection{CHKPT}

%  This keyword is used to specify a time interval after which a
%  checkpoint for later restart is created. After \verb+minutes+
%  minutes of walltime the forces are written to the runtime database.

% \subsection{RESTART}

Specifying the keyword \verb+restart+ allows the user to restart a calculation
using the gradient calculated from a previous calculation that may have
aborted for some reason.  (This implies, of course, that the previous
calculation employed the keyword \verb+chkpt+ with a value specified for
the variable \verb+minutes+ that allowed the calculation to write out the
gradient before failing.)  The keyword \verb+restart+ allows the partially
calculated forces from the previous calculation to be used as the starting
point for the new calculation.  If the gradient was not saved previously,
however, this keyword has no affect.  The gradients area automatically 
recalculated from zero.

%  This keyword tells the program that this is a restart of an aborted
%  gradient calculation. The partially calculated forces are taken from
%  the database of the previous run. If they are not present, the
%  keyword is ignored and a complete calculation of the gradients is started.

  It also works within a geometry optimization. Subsequent gradient 
  calculations are not treated as restarts.
\Large  **Elucidate.**
\normalsize

% \subsection{PRINT, NOPRINT}

The complementary keyword pair \verb+print+ and \verb+noprint+ allow the 
user some additional control on the information that can be obtained from
the SCF calculation.  Currently, only a few items can be explicitly invoked
via print control.  These are as follows;
 
%  Currently only some print control is available.

\begin{tabbing}
  Very\_long\_descriptive\_name \= Print level space \= \kill
  Name                   \> Print Level \> Description \\
                         \>        \> \\
        'information'   \>        low  \> calculation info\\
        'geometry'    \>          high \> \\
        'basis'        \>         high \> \\
        'forces'   \>             low \> \\
        'timing'   \>             default \> 
\end{tabbing}

\subsection{frozen atoms}
\label{sec:activeatoms}

\Large
***This section belongs somewhere near where you explain about atoms
and centers and frozen atoms; somewhere in the section on Geometry,
or maybe Basis sets, I suspect.  It definitely does not belong here,
all on its lonesome.***
\normalsize

Currently the only mechanism for freezing atoms is to enter a list of
active atoms via the \verb+SET+ directive.

\begin{verbatim}
  set geometry:actlist integer <at1> <at2> <at3> ...
\end{verbatim}
defines atoms number \verb+<at1>+ \ldots as 'active', and only forces
on those are calculated. All other atoms remain frozen at their
starting coordinates during a geometry optimization.
% I have no idea how this works with symmetry.
% But in the new release there will be frozen atoms and variables
% anyway.

