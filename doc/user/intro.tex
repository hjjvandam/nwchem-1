\label{sec:intro}

NWChem is a computational chemistry package designed to run on
high-performance parallel supercomputers and workstation clusters.
Code capabilities include the calculation of molecular electronic
energies and analytic gradients using Hartree-Fock self-consistent field, Gaussian
density function theory (DFT), and second-order perturbation theory.
Geometry optimization for energy minima and transition states is
available for all methods.  Classical molecular dynamics capabilities
provide for the simulation of macromolecules and solutions, including
the computation of free energies, using a variety of forcefields.

NWChem is scalable in both its ability to treat large problems
efficiently, and in its utilization of available parallel computing
resources.  The code uses the parallel programming tools TCGMSG and
the Global Array library developed at PNNL for the High Performance
Computing and Communication Initiative (HPCCI) grand-challenge
software program and the Environmental Molecular Sciences Laboratory
(EMSL) Project.  NWChem has been optimized to perform calculations on
large molecules using large parallel computers and it is unique in
this regard.  In contrast, its performance on small calculations
running on small computers is unremarkable.

This document is intended as an aid to chemists attempting to
use the code for their own applications.  Users are not expected to
have a detailed understanding of the code internals, but some
familiarity with the overall structure of the code, how it handles
information, and the nature of the algorithms it contains will
generally be helpful.  The following sections describe the structure
of the input file, and give a brief overview of the code
architecture.  All input directives recognized by the code are
described in detail, with options, defaults, and recommended usage,
where applicable.  Additional information on the molecular geometry
and basis function libraries included in the code is presented in the
appendices.

\section{Citation}

The EMSL Software Agreement stipulates that the use of NWChem will be
acknowledged in any publications which use results obtained with
NWChem.  The acknowledgment should be of the form:
\begin{quote}

  NWChem Version \nwchemversion, as developed and distributed by
  Pacific Northwest National Laboratory, P.~O.~Box 999, Richland,
  Washington 99352 USA, and funded by the U.~S.~Department of Energy,
  was used to obtain some of these results.
\end{quote}

The words ``A modified version of'' should be added at the beginning,
if appropriate.  {\em Note: Your EMSL Software Agreement contains the
complete specification of the required acknowledgment.}

Please use the following citation when publishing results obtained
with NWChem:
\begin{quote}
  High Performance Computational Chemistry Group, {\em NWChem, A
   Computational Chemistry Package for Parallel Computers, Version
    \nwchemversion{}} (\nwchemyear), Pacific Northwest National
  Laboratory, Richland, Washington 99352, USA.
\end{quote}

\section{User Feedback}

This software comes without warranty or guarantee of support,
but we do try to meet the needs of our user community.  Please send bug
reports, requests for enhancement, or other comments to

\begin{itemize}
\item {\tt nwchem-support@emsl.pnl.gov}
\end{itemize}

When reporting problems, please provide as much information as possible, 
including;

\begin{itemize}
\item detailed description of problem
\item site name (e.g., EMSL, NERSC, \ldots)
\item platform you are running on including
\begin{itemize}
\item vendor name
\item computer model
\item operating system
\item compiler
\end{itemize}
\item input file
\item output file
\item contact name and telephone number
\end{itemize}

Users can also subscribe to an electronic mailing list of other users
of the code.  This is intended as a general forum through which code
users can contact one another and the developers to share experience
with the code and discuss problems.  Announcements of new releases and
bug fixes will also be made to this list. 

To subscribe to the user list, send a message to 
\begin{verbatim}
  majordomo@emsl.pnl.gov
\end{verbatim}
The body of the message must contain the line 
\begin{verbatim}
  subscribe nwchem-users
\end{verbatim}

The automated list manager is capable of recognizing a number of
commands, including 'subscribe', 'unsubscribe', 'get', 'index',
'which', 'who', 'info' and 'lists'.  The command 'end' halts
processing of commands.  It will provide some help if the message
includes the line {\tt help} in the body.  Messages can be posted to
the list by sending mail to {\tt nwchem-users@emsl.pnl.gov}.  Users
are encouraged to use the support address rather than the mailing list
to report problems since the support mailer interfaces to an automated
bug tracking mechanism.


