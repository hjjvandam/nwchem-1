

NWChem is a quantum chemistry package
designed to run on high-performance parallel supercomputers and
workstation clusters.  It is scalable in both its ability to
treat large problems efficiently, and in its usage of available
parallel computing resources.  NWChem utilizes parallel programming
tools TCGMSG and the Global Array library developed by PNNL staff
for the High Performance Computing and
Communication Initiative (HPCCI) grand-challenge software program and
the Environmental Molecular Sciences Laboratory (EMSL) Project, which
has supplied the majority of the development costs.

NWChem has been optimized to perform calculations on large molecules
using large parallel computers and it is unique in this regard.  In
contrast, NWChem's performance on small calculations running on
small computers is unremarkable.

\subsection{Citation}

The EMSL Software Agreement stipulates that the use of NWChem be
acknowledged in any publications which use results obtained with
NWChem.  The acknowledgment should be of the form:
\begin{quote}

  NWChem Version \nwchemversion, as developed and distributed by
  Pacific Northwest National Laboratory, P.~O.~Box 999, Richland,
  Washington 99352 USA, and funded by the U.~S.~Department of Energy,
  was used to obtain some of these results.
\end{quote}

The words ``A modified version of'' should be added at the beginning,
if appropriate.  {\em Note: Your EMSL Software Agreement contains the
complete specification of the required acknowledgment.}

If you wish to cite NWChem in the references section of a publication,
please use the following citation:
\begin{quote}
  High Performance Computational Chemistry Group, {\em NWChem, A
   Computational Chemistry Package for Parallel Computers, Version
    \nwchemversion{}} (\nwchemyear), Pacific Northwest National
  Laboratory, Richland, Washington 99352, USA.
\end{quote}

\subsection{User Feedback}

Although this software comes without warranty or guarantee of support,
we do try to meet the needs of our user community.  Please send bug
reports, requests for enhancement, or other comments to

\begin{itemize}
\item {\tt nwchem-support@emsl.pnl.gov}
\end{itemize}

When reporting problems, please provide as much information as possible, 
including;

\begin{itemize}
\item detailed description of problem:
\item platform you are running on:
\begin{itemize}
\item operating system:
\item compiler:
\end{itemize}
\item input file:
\item output file:
\item contact name and telephone number:
\end{itemize}


Users can also subscribe to an electronic mailing list of other users of
 the code.  This is intended as a general forum through which code users 
can contact one another and the developers to share experience with the code
and discuss problems.  Specific bugs and problems should be sent to the
nwchem-support@emsl.pnl.gov address, however.

To subscribe to the user list, send a message to 

\begin{itemize}
\item {\tt majordomo@emsl.pnl.gov}
\end{itemize}

The body of the message must contain the line 

\begin{itemize}
\item {\tt subscribe nwchem-users}
\end{itemize}

The automated list manager is capable of recognizing a number of commands, including
'subscribe', 'unsubscribe', 'get', 'index', 'which', 'who', 'info' and 
'lists'.  The command 'end' halts processing of commands.  It will provide
some help if the message includes the line {\tt help} in the body.  Messages
can be posted to the list by
sending mail to {\tt nwchem-users@emsl.pnl.gov}.  Reports of problems can
be posted to this list, if the user wishes to do so, but they should also
be posted to the {\tt nwchem-support} address.  The NWChem developers
will monitor the user list to help them stay aprised of user traffic and
 current problems, but 
the tracking system for bug reports is based on reports to the
 {\tt nwchem-support} address.  Users are encouraged to use the support 
address rather than the mailing list for
the initial report of any problems with the code.

\section{Getting Started}
\label{sec:getstart}

The input to NWChem is composed of commands, called directives, which define 
functions, actions, and keywords telling the code where to look for needed
input information, what databases to use, and what to write to output files.
The top level directives (see Section 4) are global and known to all modules.
Directives that are specific to a particular module in general define keywords
and other input that is local to that module only.  
The input is free format, and
keywords or blocks of module-specific directives can, with certain
minor restrictions, appear in any order.  Case is ignored except
for actual data (e.g., names/tags of centers, titles).  This section describes
the input format and the syntax used for the directives 
(see subsection \ref{sec:syntax}).
The structure of an NWChem input file is also described (see subsection 
\ref{sec:simplesample}), and  illustrated using a simple example.  In 
addition, a  more realistic and therefore more complicated example input 
file is also presented (see subsection \ref{sec:realsample}).

Sections 4 through 20 present detailed descriptions of the individual
directives, giving the format of each along with options and defaults.  Where
appropriate, examples are presented to illustrate the features and options
available for the directive.  The directives fall into four main categories;

\begin{itemize}
\item top level directives for setup and job control (see Section 4)
\item directives to describe the geometry of the problem (see Section 5)
\item directives to define the basis set(s) for the problem (see Sections 6 and 7)
\item directives to define theory and operations to be used to solve the problem (see Sections 8 through 20)
\end{itemize}

To make the input as short and simple as possible, most of the options and
features in the code have default values.  The user needs to supply input only
for those items that have no defaults, or for items that must be different
from the defaults for the particular application.  In the discussion of each
directive, the defaults are noted, where applicable.

\subsection{Input Format and Syntax for Directives}
\label{sec:syntax}

The input format for the directives used for NWChem is similar to that 
of UNIX shells, 
which is also used in other chemistry packages, most notably GAMESS-UK.  An
input line is parsed into whitespace (blanks or tabs) separating tokens
or fields.  Any token that contains whitespace should be enclosed in
double quotes; for example, the basis set with the descriptive name
\verb+modified Dunning DZ+ must appear in a directive as 
\verb+"modified Dunning DZ"+, since the name consists of three separate words.

  
Each logical or physical input line must be no longer than 1023 characters.
Data is read until an end-of-file is detected, or until an \verb+EOF+ 
directive is encountered\footnote{The
  free-format input library does not read past the string \verb+EOF+
  (ignoring case) on a line by itself.  This is a convenience so that
  unused input can be left in the same file.}.

Directives consist of a keyword and optional input.  The description of
each directive includes the type of input required for each option, which
may be 'string' to specify that the input must be a string variable, 
'integer' to specify that the input must be an integer value, etc.
The default value for any optional input is noted in the description of
the directive by preceeding it with the word 'default'.

A directive can consist of one line or many.  Simple directives 
consist of a single
line of input with one or more fields.  Compound directives can have
multiple input lines, and must be terminated with an END directive.  The
directives START (Section \ref{sec:start}) and ECHO (Section \ref{sec:echo})
are examples of simple directives.  The directive GEOMETRY (Section 5) is 
an example of a compound directive.


Some limited checking for self-consistency of the input is performed
by the input module, but most defaults are imposed by the
application modules at runtime.  It is therefore usually impossible
to determine beforehand whether or not all selected options are
consist with each other.

The following notation is used for definition of the directives:

\begin{itemize}
\item a string, token, or field is a sequence of ASCII characters *NOTE: if 
the string includes blanks or tabs (i.e., white space) the entire string must
be enclosed in quotes*
\item \verb+()+ is used to group items (the parentheses and other
      special symbols do not appear in the input)
\item \verb+||+ separate exclusive options/parameters/formats
\item \verb+[ ]+ enclose optional entries with a default value
\item \verb+< >+ enclose a type, a name of a value to be specified,
      and a default value if any.

\item \verb+\+ is used to concatenate lines in a description *NOTE: within 
a string, the \verb+\+ causes the preceeding character (including a blank) 
to be eliminated* 
\item \verb+$variable$+ is used to indicate the substitution of the value of a
      variable
\item \verb+...+ is used to indicate indefinite continuation of a list
\end{itemize}

An input parameter is identified in the description of the
directive by prefacing the item with the name of the type of data expected;
i.e., 'string' for an ASCII character string, 'integer' for integer
value(s) for a variable or an array, 'logical' for a true/false logical
variable.  If an input item is not prefaced by one of these type names,
it is assumed to be of type 'real'.
 
Using the directive VECTORS  as an example, 
the generic input line is as follows;
\begin{verbatim}
  VECTORS [[input] (<string input_movecs default atomic>) || \
                   (project <string basisname> <string filename>)] \
          [swap [(alpha|beta)] <integer vec1 vec2> ...] \
          [output <string output_movecs default "$file_prefix$.movecs">]
\end{verbatim}

A complete description of the keywords
and options for the VECTORS directive can be found in Section \ref{vectors}.
As an illustration of directive syntax, this example shows that there are 
three optional keywords in the VECTORS directive, as indicated by
the three main sets of square brackets, enclosing the keywords (\verb+input+,
\verb+swap+, and \verb+output+).  The keyword \verb+input+ allows the user
to specify the source of the molecular orbital vectors.  When this keyword
is invoked, there are two mutually exclusive options for specifying the
vectors, as indicated by the \verb+||+ symbol separating the two descriptions;

\begin{verbatim}
(<string input_movecs default atomic>) || \
                   (project <string basisname> <string filename>)] \

\end{verbatim}

The first option, \verb+(<string input_movecs default atomic>)+, allows
the user to specify an ASCII character string for the parameter {\tt 
input_movecs}.
If no entry is specified, the code assumes a
standard source of the vectors with a default of \verb+``atomic''+ 
(atomic guess).  The second option, 
\verb+(project <string basisname> <string filename>)+, contains the
keyword \verb+project+, which takes two additional string arguments.
When this keyword is used, the vectors will be projected from the basis
\verb+basisname+ in the file \verb+filename+.

The second keyword, /verb+swap+, allows the user to re-order the starting
vector.  The optional keyword \verb+[alpha|beta]+ allows the user to swap
the alpha and beta spin orbitals, and specify in pairs the integer numbers
of the vectors to be swapped.  As many pairs as the user wishes to have
swapped can be listed between the angle braces \verb+<integer vec1 vec2 ... >+
enclosing this input.

The third keyword, /verb+output+, allows the user to tell the code where
to put the output values of the vectors by specifying an ASCII string
for the parameteer {\tt output_movecs}.  If no entry is specified for
this parameter, the default is to write them
back into the input file, {\tt $file_prefix$.movecs}.

% The keyword \verb+input+ is the default,
% which means that the directive expects some sort of input to specify the
% vectors.    
% The \verb+input+ keyword is followed by two mutually exclusive
% options.  One is to specify a standard source of the vectors, using the
% default keyword \verb+"atomic"+ for (atomic guess).  The other option is 
% to specify a projected set of vectors.  This option is obtained by 
% specifying 
% the keyword \verb+project+ instead of the keyword \verb+''atomic''+.
% The keyword \verb+project+ takes two additional string arguments.  
% These arguments indicate that vectors should be obtained by projecting 
% from vectors in
% a smaller basis.  The \verb+swap+ keyword may be optionally applied to
% the alpha or beta spin orbitals and takes as an argument a list of
% pairs of integers. The \verb+output+ keyword requires a string
% argument, which takes a de fault value constructed from the value of
% the file prefix appended with \verb+".movecs"+.

A particular example of the VECTOR directive is shown below.  It specifies
both the \verb+input+ and \verb+output+ keywords, but does not use the 
\verb+swap+ option.

\begin{verbatim}
  vectors input project "small basis" "small_basis.movecs" \
          output "large_basis.movecs"
\end{verbatim}

This directive tells the code to generate input vectors by projecting from a 
smaller basis named \verb+''small basis''+, which is stored in file 
\verb+''small_basis.movecs''+.  The output vectors will be stored in the file
 \verb+''large_basis.movecs''+.

The order of keyed optional entries within a directive should not matter,
unless noted otherwise in the specific instructions for a particular directive.


\subsection{Input File Structure}
\label{sec:inputstructure}

The structure of an input file reflects the internal structure of
NWChem.  At the beginning of a calculation NWChem needs to have a number
of start-up directives telling it such things as how much memory to use, 
the name of the database, whether it is a new, restarted, or continuing job, 
where to put scratch/permanent files, etc..  It is not necessary to put
this information at the top of the input file, however.  The first (zeroth)
process in NWChem reads through the {\em entire} the input file looking for 
the start-up directives.  In this pass, all other directives are ignored.  

The start-up directives are

\begin{itemize}
\item START
\item CONTINUE
\item RESTART
\item SCRATCH{\verb+_+}DIR
\item PERMANENT{\verb+_+}DIR
\item MEMORY
\item ECHO

% what about TITLE, PRINT/NOPRINT, SET/UNSET, STOP, TASK, and CHARGE?
% these appear in Section 4 Top Level Directives...

\end{itemize}

The input values for keywords, etc. for these start-up directives are
broadcast to the other nodes\footnote{Once the database has been opened
process zero can insert data into the database which may be
subsequently accessed directly by all processes.}.

After the input file has been scanned for the start-up directives, it
is rewound and read sequentially.  Input is
processed either by the top-level parser (for the directives listed in
Section 4;
\verb+TITLE+, \verb+SET+, \ldots) or by the specific computational modules
invoked for the particular case
(e.g., SCF, DFT, \ldots).  Any directives that have already been processed
(e.g., \verb+MEMORY+) are ignored.  Input is read until a \verb+TASK+
directive (see Section \ref{sec:task}) is encountered, which specifies the 
level of theory to be used to solve the problem and the operations to be 
performed.  Input processing stops and the specified task is executed.  
The TASK directive in effect defines the end of the input for a given job.  
Processing of the input resumes upon the successful completion of the task,
however, and the results of that task can be used as input to a subsequent
problem in the same input file, as defined by a new \verb+TASK+ directive.

The name of the input file is usually provided as an argument to the execute
command for NWChem.  For example;

\begin{itemize}
\item \verb+nwchem input_file+
\end{itemize}

The default name for the input file is \verb+nwchem.nw+.  If an input
file name \verb+input_file+ is specified without an extension, the code
assumes a default extension of \verb+input_file.nw+.  If the code cannot
locate a file named either \verb+nwchem.nw+ or \verb+input_file.nw, an 
error is reported and execution terminates.  The following
subsection presents two input files to illustrate the directive syntax and 
input file format for NWChem applications.

\subsection{Simple N2 Input File}
\label{sec:simplesample}

An example of a very simple problem that can be run with NWChem is the
SCF geometry optimization on the nitrogen molecule using the Dunning 
cc-pvdz basis set.  If this problem is run using the defaults in NWChem, 
it requires only a very short input file.  This input file contains
the bare minimum of information the user must specify in order to run 
this type of problem.  The file consists of fewer than ten lines of input,
as follows;

\begin{verbatim}
  title; Nitrogen cc-pvdz SCF geometry optimization
  geometry 
    n 0 0 0
    n 0 0 2.1
  end
  basis
    n library cc-pvdz
  end
  task scf optimize
\end{verbatim}

Examining the input file line by line, it can be seen that it contains 
only four directives; TITLE, GEOMETRY, BASIS, and TASK.  The TITLE directive
is optional, and is used only as a means for the user to more easily identify
outputs from different jobs.  The TASK directive tells the code what to
do with the case, and the GEOMETRY and BASIS directives are needed
to define the problem.

The default for the input geometry is cartesian coordinates.  (The Z-matrix
can be selected by setting the keyword 'zmat' (see Section \ref{sec:Z-matrix}).
The default for geometric units is the atomic unit (Bohr).  Since no other
option is specified in this directive, the numbers provided on the lines
following the geometry directive are assumed to be in atomic units.
(See Section 5 for a complete discussion of available options for the 
GEOMTERY directive.) 

The basis input block is structured like the geometry block (keyword,
\ldots, end) and contains basis set information for every atom type in
the geometry. In this case, a default basis set from the library is
chosen.  Refer to Section \ref{knownbasis} and section \ref{sec:basis} 
for a description of available basis sets and a discussion of how to define
new ones.

The last line ({\tt task scf optimize})
tells the program what to calculate with the information given so
far. For a description of possible tasks and the format of the input
for the {\tt task} directive, refer to Section \ref{sec:task}.

If this input is in the file \verb+n2.nw+, the command to run the job on
a typical UNIX workstation is as follows;

\begin{verbatim}
  nwchem n2
\end{verbatim}

The result obtained in NWChem for this case would be output to the
default output file, with the default selection for output options.

\subsection{Water Molecule Sample Input File}
\label{sec:realsample}

A more complex sample problem can be obtained using the geometry of
a positively charged water molecule.  This is also a more realistic example
of a NWChem input file.  In this problem, a preliminary optimization
with SCF is performed using a relatively cheap basis set (sto-3g).  Then 
the optimization is finished using second-order M{\o}ller-Plesset 
perturbation theory and a basis set with polarization functions. 
The final task is to calculate the vibrational frequencies.

The input file to accomplish these three tasks is as follows;

\begin{verbatim}
start h2o_freq

title; H2O+ frequencies with MP2 and 6-31g**

geometry units angstrom
  O       0.0  0.0  0.0
  H       0.0  0.0  1.0
  H       0.0  1.0  0.0
end

charge 1

basis "starting basis"
  H library sto-3g
  O library sto-3g
end

basis "property basis" print
  H library 6-31g**
  O library 6-31g**
end

scf
  uhf
  doublet
  maxiter 100
  print low
end

stepper
  convge 1.0d-6
end

set "ao basis" "starting basis"

task scf optimize

set "ao basis" "property basis"

scf
  vectors input atomic
  print none
end

set "mp2_grad:print" low

stepper
  convge 1.0d-10
end

task mp2 optimize

set "mp2_grad:print" none

task mp2 freq

eof
\end{verbatim}

This file beins with a {\tt start} directive, which tells NWChem that this 
run is to be started from the beginning. Existing database or vector files 
are to be ignored or overwritten. The entry \verb+h2o_freq+ on the 
\verb+START+ line gives an internal name for the job and the database.  
The name to be used for each temporary or permanent file associated with the
job will also contain the string \verb+h2o_freq+. 
For this reason,
different jobs running in the same directory or sharing the same
scratch directory (see Section \ref{sec:dirs})
must use different names in this field.

As in the first sample problem, the geometry is given in cartesian 
coordinates.  In this case, however, the units are specified as
{\AA}ngstr{\o}m instead of the default atomic units.

The {\tt charge} directive defines the total charge of the system.  This calculation is to be done on an ion with charge +1. Note that this is a top-level
directive, independent of other input blocks.

The next two directives are the {\tt basis} directives.  These are compound
directives that define the two basis sets to be used in the calculations.  
The basis sets should be given names that will serve as a helpful mnemonic 
for later reference in
subsequent input directives.  The basis set specified on the second 
\verb+BASIS+ directive will appear in the information
written to the output file.

The {\tt scf} directive in the {\tt scf \ldots end} block specifies 
details about the scf
calculation to be performed. Unrestricted Hartree-Fock is chosen
rather than the default high-spin ROHF. For the subsequent MP2
calculation, this is necessary because only UMP2 is available for
open-shell systems currently (refer to Section \ref{sec:function} for
a more complete discussion of code functionality). For open-shell 
systems, the spin
multiplicity has to be specifed, or it defaults to singlet. The maximum
number of iterations is set to a high value to allow convergence in
difficult cases. The print level is set to {\tt low} to reduce the
output from the scf module to a minimum.

The energy convergence threshold for the geometry optimization module
{\tt stepper} is set to a higher value than the default $1.9d-8$
because the initial optimization with scf is not intended to find the
exact minimum.

Finally the {\tt set} directive is used to specify the
starting basis as the ``{\tt ao basis}''.  This basis is the default
basis for all {\em ab initio} calculations.  In this case, the ``{\tt
ao basis}'' is set to the string ``{\tt starting basis}''.

All input up to this point affects only the settings in the run-time database.
The program takes its information from this data base, so the
sequence of directives is irrelevant.  An exchange of order of the 
different blocks or
directives would not affect the result. 

The {\tt task} directive, however, must be specified last of all in the
list of input directives for
a given problem. The {\tt task} directive invokes the program and does a calculation
using the parameters set in the previous directives. In this case, the 
first task is an SCF calculation with geometry optimization.   (See Section 
\ref{sec:task} for a list of available 
tasks and operations.)

After this optimization task is done, settings in the database are used
in subsequent tasks without change, unless they are overridden by new 
input directives.
In this example, there are several important changes between the first
and the second optimization task.  The ``{\tt ao basis}'' is set to a 
better basis (i.e., ``{\tt property basis}'') for the final set of 
calculations.  The scf output is completely 
discarded (using the {\tt print  none} input in the {\tt scf} directive),
which means that no output will be producedd unless an error occurs.
The starting guess for the MP2 optimization task is to be an SCF calculation 
on the atoms. The input must be specified using the {\tt scf} directive,
because the default would be to use existing MO vectors.  Since the basis 
has been changed in this task,
the differences will be too large to make a projection of the old vectors 
a reasonable representation on the new basis.  (See Section \ref{sec:vectors}
for details on the different possible starting guesses.)

As there is currently no input block for the MP2 gradients, parameters
for this module have to be set in the database via the {\tt set}
directive. Currently only print options are recognized, and here the
amount of output is again reduced to the reasonable minimum. The
convergence criterion for the geometry optimization program is set to
a lower value, requesting a more exact minimum search.

The directive {\tt task mp2 optimize} calls for an optimization with MP2, 
starting with the geometry resulting from the previous task.  At the final 
geometry a frequency calculation is performed. The second
derivatives of the energy are calculated as numerical derivatives of
analytical gradients. The gradients as such are not of interest, so
the output from the {\tt mp2\_grad} program is completely discarded.

The {\tt eof} directive marks the end of the input to be read. It is not
necessary to specify the end-of-file as an explicit directive at the
actual end of the file.  However, using the directive allows the user to
have additional lines in the file that will not be processed as input.  For
example, a description of the calculation(s) specified in the input, or
notes on the purpose of the task(s) or other informational comments can be
included in the input file in this way.  The {\tt eof} directive can also
be used to remove a task (or series of tasks) from the calculation without
actually removing the input for them from the file.  Any lines in the file
that occur after the {\tt eof} directive are ignored by
the program.



\section{NWChem Architecture}
\label{sec:arch}

NWChem consists of independent modules that perform the various functions
of the code.  These functions include input processing, creation of basis
sets and geometry sets, and solution operations using different theoretical
models.  Different modules communicate through a database, using an approach
similar in spirit to the GAMESS dumpfile or the Gaussian checkpoint file.  
This allows the modules to share data, or to share access to files containing
data.

It is not necessary for the user to be intimately familiar with the contents
of the database in order to run NWChem.  Nor is it necessary to fully
understand the theory and operation of the solution options in the code.
However, a nodding acquaintance with the structure of the database and 
the functionality of the code would probably help clarify the logic behind
the input requirements of the code, in understanding the usefulness of the
defaults, and in determining which models and operation are most appropriate
for a given application.  Section \ref{sec:database} gives a general 
description of the database, and Section \ref{sec:functionality} describes 
the theorical models and operations currently available in the code.

/subsection{Database Structure}
/label{database}

Data is shared between modules of NWChem by means of the database.  There
are three main types of information stored in the data base; (1) arrays of
data, (2) names of files that contain data, and (3) objects.  

Arrays are stored directly in the database, and contain the following
information;

\begin{enumerate}
\item a string of ASCII characters that serve as the name of the array (e.g., 
      \verb+"reference energies"+),
\item the type of the data in the array (i.e., real, integer, logical, or character), 
\item the number (N) of data items in the array, and
\item the N items of data of the specified type.
\end{enumerate}

Some arrays in the database are defined permanently in the code; some are
defined or created while processing the input, others are created as part
of a calculation.  The user can directly specify some arrays by means of
the input, using the \verb+SET+ directive (see Section \ref{sec:set}).  
For example, to store a (64-bit precision) three-element real array 
with the name "reference energies" in the database, the directive is 
as follows;

\begin{verbatim}
  set "reference energies" real 0 1 -76.2
\end{verbatim}

The number of elements in the array is counted automatically when this input
directive is processed, and this information is entered automatically
into the array in the database.

An item in the database can reference a file containing data, rather
than containing the data itself.  The structure of such an entry is 
similar to an ordinary array, except that it contains only three the following
three items;

\begin{enumerate}
\item a string of ASCII characters that is the name of the file containing
the data,
\item the type of the data in the file (i.e., real, integer, logical, 
or character), and 
\item the number of data items in the file
\end{enumerate}

Objects in the database are in general more complex than simple arrays.
An object consists of a set of related data that different routines in
the code must read, manipulate, or write.  Currently, the code has two
main objects; the basis sets and the geometry set.  Sections 5,6, and 7 
present a complete discussion of the input to describe these objects; for
the purposed of this discussion, these objects can be understood simply
as a related packet of information, such as a description of a
basis set, or the history of a geometry optimization.  A given object is
built up following an internally consistent naming convention and set of 
structural specifications and array ordering for that object. For example, 
a geometry
database with the default name \verb+"geometry"+ contains the following
entries;

\begin{verbatim}
 Contents of RTDB h2o.db
 -----------------------

 Entry                                   Type[nelem]
 ---------------------------  ----------------------
 geometry:geometry:efield              double[3]    
 geometry:geometry:coords              double[9]    
 geometry:names                          char[10]   
 geometry:geometry:ncenter                int[1]    
 geometry:ngeom                           int[1]    
 geometry:geometry:charges             double[3]    
 geometry:geometry:tags                  char[6]
 ...
\end{verbatim}


Using this approach, multiple instances of the geometry object can be 
stored with different names in the same database without confusion.  
For example, if a user needs to do calculations considering alternative
geomteries for the water molecule, an input file can be constructed that
allows more than one geometry object to exist in the database.  The
{\tt geometry} directive can be used to define an alternative geometry
object in the database, under a name different from the geometry object
for water already in the code.  For example, the input directive to define
a geometry object in the database with the name
\verb+"test water geometry"+ can be specified as follows;

\begin{verbatim}
  geometry "test water geometry"
    O     0.00000000    0.00000000    0.00000000
    H     0.00000000    1.43042809   -1.10715266
    H     0.00000000   -1.43042809   -1.10715266
  end
\end{verbatim}

The geometry object contains more information than merely
the numerical input in the directive, however.  The geometry directive
allows the user to specify different values for the x,y,z coordinates of
the atoms (or centers), and identify that object under a unique name, but 
the other information in the geometry oject remains unchanged for all
specifically named objects.  (Refer to Section 5 for a complete description of the {\tt geometry}
directive.)  

Unless a specific name is defined for the geometry object in the database
(such as the name  \verb+"test water geometry"+ shown in the example), 
the object in the database is 
assigned the default name of \verb+"geometry"+.  This is the object name
that computational modules requiring information from the geometry object
will look for.  The {\tt set} directive can be used in the input to force a
module (or modules) to look for a geometry object with 
different name than \verb+"geometry"+ for a particular task.  For example, 
to use the x,y,z coordinates specified for \verb+"my water geometry"+ 
rather than those of the default geometry object \verb+"geometry"+, the
input line is as follows;

\begin{verbatim}
  set geometry "my water geometry"
\end{verbatim}

When the computer code for the geometry object is asked to load a geometry
with name \verb+"geometry"+, during the calculation, it will first check for 
a {\tt set} directive on the geometry object name before actually attempting 
to load the geometry.  The basis set object
functions in an identical fashion and it is intended that all such
objects will do so.  (Note: the naming conventions and mechanisms of 
associating data with specific modules or tasks is expected to change in
the future, but the directive for specifying names should remain the same.)

\fussy

The database is persistent and nearly all input is permanently stored.
This makes the input for restart jobs very simple since very little
data must be provided.  It also makes the behavior of successive
restart jobs {\em identical} to that of multiple tasks within one job.
However, sometimes this persistence is undesirable and in order to
return an NWChem module to its default (input-free) behavior the
\verb+UNSET+ directive (see Section \ref{sec:unset}) 
can be used to delete all
database entries associated with a given module.


\subsection{Functionality}
\label{sec:functionality}

Current functionality includes:
\begin{itemize}
\item Self Consistent Field or Hartree Fock energy and analytic
  gradients (RHF, UHF, high-spin ROHF)
\item Gaussian Density Functional Theory (DFT) energy and analytic
  gradients with many local and
      non-local exchange-correlation potentials
\item MP2 energy using direct or semi-direct algorithms, or using the
  resolution of the identity integral approximation.  Analytic
  gradients are under development.
\item Geometry Optimization (Minimization and Transition State).
\item Second-derivatives by numerical difference of gradients.
\item Generation of the electron density file for the {\em Insight}
      graphical program.
\item Interface to the natural bond orbital package.
\end{itemize}
































