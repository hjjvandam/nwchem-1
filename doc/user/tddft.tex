\label{sec:tddft}

\section{Overview}

NWChem supports a spectrum of single excitation theories for
vertical excitation energy calculations, namely, configuration interaction
singles (CIS),\footnote{J. B. Foreman, M. Head-Gordon, J. A. Pople, and M. J. Frisch, J.\ Phys.\ Chem. {\bf 96,} 135 (1992).} 
time-dependent Hartree--Fock (TDHF or also known as 
random-phase approximation RPA), time-dependent density functional
theory (TDDFT),\footnote{C. Jamorski, M. E. Casida, and D. R. Salahub, J.\ Chem.\ Phys. {\bf 104,} 5134 (1996);
R. Bauernschmitt and R. Ahlrichs, Chem.\ Phys.\ Lett. {\bf 256,} 454 (1996); 
R. Bauernschmitt, M. H\"{a}ser, O. Treutler, and R. Ahlrichs, Chem.\ Phys.\ Lett. {\bf 264,} 573 (1997).} 
and Tamm--Dancoff approximation to TDDFT.\footnote{ S. Hirata and M. Head-Gordon, Chem.\ Phys.\ Lett. {\bf 314,} 291 (1999).}
These methods
are implemented in a single framework that invokes Davidson's trial vector
algorithm (or its modification for a non-Hermitian eigenvalue problem).\footnote{E. R. Davidson, J.\ Comput.\ Phys. {\bf 17,} 87 (1975); J. Olsen, H. J. Aa.\ Jensen, and P. J\o rgensen, J.\ Comput.\ Phys. {\bf 74,} 265 (1988).}
The capabilities of the module are summarized as follows:
\begin{itemize}
\item Vertical excitation energies,
\item Spin-restricted singlet and triplet excited states for closed-shell systems,
\item Spin-unrestricted doublet, etc., excited states for open-shell systems,
\item Tamm--Dancoff and full time-dependent linear response theories,
\item Davidson's trial vector algorithm,
\item Symmetry (irreducible representation) characterization and specification,
\item Spin multiplicity characterization and specification,
\item Transition moments and oscillator strengths,
\item Geometrical first and second derivatives of vertical excitation energies 
by numerical differentiation,
\item Disk-based and fully incore algorithms,
\item Multiple and single trial-vector processing algorithms,
\item Frozen core and virtual approximation.
\end{itemize}

\section{Performance of CIS, TDHF, and TDDFT methods}

The accuracy of CIS and TDHF for excitation energies of closed-shell systems
are comparable to each other, and are normally considered a zeroth-order
description of the excitation process.  These methods are particularly well balanced
in describing Rydberg excited states, in contrast to TDDFT.
However, for open-shell systems,
the errors in the CIS and TDHF excitation energies are often excessive, primarily
due to the multi-determinantal character of the ground and excited state wave functions
of open-shell systems in a HF reference.\footnote{D. Maurice and M. Head-Gordon, J.\ Phys.\ Chem. {\bf 100,} 6131 (1996).} 
The scaling of the computational cost of a CIS
or TDHF calculation per state with respect to the system size is the same as that for 
a HF calculation for the ground state, since the critical step of the both methods are
the Fock build, namely, the contraction of two-electron integrals with density matrices.
It is usually necessary to include two sets of diffuse exponents in the basis set
to properly account for the diffuse Rydberg excited states of neutral species.

The accuracy of TDDFT may vary depending on the exchange-correlation functional.
In general, the exchange-correlation functionals that are widely used today and are implemented
in NWChem work well for low-lying valence excited states.  However, for high-lying diffuse
excited states and Rydberg excited states in particular, TDDFT employing these 
conventional functionals breaks down and the excitation energies are substantially 
underestimated.  This is because of the fact that the exchange-correlation potentials
generated from these functionals decay too rapidly (exponentially) as opposed to the 
slow $-1/r$ asymptotic decay of the true potential.  A rough but useful index is the 
negative of the highest occupied KS orbital energy; when the calculated excitation energies
become close to this threshold, these numbers are most likely underestimated relative
to experimental results.\footnote{M. E. Casida, C. Jamorski, K. C. Casida, and D. R. Salahub, J.\ Chem.\ Phys. {\bf 108,} 4439 (1998).}  It appears that TDDFT provides a better-balanced description
of radical excited states.\footnote{S. Hirata and M. Head-Gordon, Chem.\ Phys.\ Lett. {\bf 302,} 375 (1999).} 
This may be traced to the fact that, in DFT, the ground state
wave function is represented well as a single KS determinant, with less multi-determinantal
character and less spin contamination, and hence the excitation thereof is described well
as a simple one electron transition.  The computational cost per state of TDDFT calculations 
scales as the same as the ground state DFT calculations, although the prefactor of the scaling
may be much greater in the former.

\section{Input syntax}

The module is called TDDFT as TDDFT employing a hybrid HF-DFT functional 
encompasses all of the above-mentioned methods implemented.  To use this
module, one needs to specify \verb+TDDFT+ on the task directive, e.g.,
\begin{verbatim}
      TASK TDDFT ENERGY
\end{verbatim}
for a single-point excitation energy calculation, and
\begin{verbatim}
      TASK TDDFT OPTIMIZE
\end{verbatim}
for an excited-state geometry optimization (and perhaps an adiabatic
excitation energy calculation), and
\begin{verbatim}
      TASK TDDFT FREQUENCIES
\end{verbatim}
for an excited-state vibrational frequency calculation.  The TDDFT module
first invokes DFT module for a ground-state calculation (regardless of 
whether the calculations uses a HF reference as in CIS or TDHF or a DFT
functional), and hence there is no need to perform a separate ground-state
DFT calculation prior to calling a TDDFT task.  When no second argument
of the task directive is given, a single-point excitation energy calculation
will be assumed.  For geometry optimizations, it is usually necessary to
specify the target excited state and its irreducible representation it
belongs to.  See the subsections \verb+TARGET+ and \verb+TARGETSYM+ for
more detail.

Individual parameters and keywords may be supplied in the TDDFT input
block.  The syntax is:
\begin{verbatim}
  TDDFT
    [(CIS||RPA) default RPA]
    [NROOTS <integer nroots default 1>]
    [MAXVECS <integer maxvecs default 1000>]
    [(SINGLET||NOSINGLET) default SINGLET]
    [(TRIPLET||NOTRIPLET) default TRIPLET]
    [THRESH <double thresh default 1e-4>]
    [MAXITER <integer maxiter default 100>]
    [TARGET <integer target default 1>]
    [TARGETSYM <character targetsym default 'none'>]
    [SYMMETRY]
    [ALGORITHM <integer algorithm default 1>]
    [VECTOR <character vector default jobname'.tddft']
    [FREEZE [[core] (atomic || <integer nfzc default 0>)] \
             [virtual <integer nfzv default 0>]]
    [PRINT (none||low||medium||high||debug)
      <string list_of_names ...>]
  END
\end{verbatim}

The user can also specify the reference wave function in the DFT input block
(even when CIS and TDHF calculations are requested).  See the section of Sample
input and output for more details.

Since each keyword has a default value, a minimal input file will be
\begin{verbatim}
  GEOMETRY
  Be 0.0 0.0 0.0
  END

  BASIS
  Be library 6-31G**
  END

  TASK TDDFT ENERGY
\end{verbatim}

\section{Keywords of {\tt TDDFT} input block}

\subsection{{\tt CIS} and {\tt RPA} --- the Tamm--Dancoff approximation}

These keywords toggle the Tamm--Dancoff approximation.  \verb+CIS+ means
that the Tamm--Dancoff approximation is used and the CIS or Tamm--Dancoff TDDFT
calculation is requested.  \verb+RPA+, which is the default, requests 
TDHF (RPA) or TDDFT calculation.

The performance of CIS (Tamm--Dancoff TDDFT) and RPA (TDDFT) are comparable in
accuracy.  However, the computational cost is slightly greater in the latter due to
the fact that the latter involves a non-Hermitian eigenvalue problem and requires
left and right eigenvectors while the former needs just one set of eigenvectors of 
a Hermitian eigenvalue problem.  The latter has much greater chance of
aborting the calculation due to triplet near instability or other instability 
problems.

\subsection{{\tt NROOTS} --- the number of excited states}

One can specify the number of excited state roots to be determined.  The default 
value is \verb+1+.  It is advised that the users request several more roots than actually
needed, since owing to the nature of the trial vector algorithm, some low-lying
roots can be missed when they do not have sufficient overlap with the initial guess
vectors.

\subsection{{\tt MAXVECS} --- the subspace size}

This keyword limits the subspace size of Davidson's algorithm; in other words, it
is the maximum number of trial vectors that the calculation is allowed to hold.
Typically, 10 to 20 trial vectors are needed for each excited state root to be
converged.  However, it need not exceed the product of the number of occupied 
orbitals and the number of virtual orbitals.  The default value is \verb+1000+.

\subsection{{\tt SINGLET} and {\tt NOSINGLET} --- singlet excited states}

\verb+SINGLET+ (\verb+NOSINGLET+) requests (suppresses) the calculation of singlet 
excited states when the reference wave function is closed shell.  The default 
is \verb+SINGLET+.

\subsection{{\tt TRIPLET} and {\tt NOTRIPLET} --- triplet excited states}

\verb+TRIPLET+ (\verb+NOTRIPLET+) requests (suppresses) the calculation of triplet 
excited states when the reference wave function is closed shell.  The default 
is \verb+TRIPLET+.

\subsection{{\tt THRESH} --- the convergence threshold of Davidson iteration}

This keyword specifies the convergence threshold of Davidson's iterative algorithm
to solve a matrix eigenvalue problem.  The threshold refers to the norm of residual,
namely, the difference between the left-hand side and right-hand side of the matrix
eigenvalue equation with the current solution vector.  With the default value of 
\verb+1e-4+, the excitation energies are usually converged to \verb+1e-5+ hartree.

\subsection{{\tt MAXITER} --- the maximum number of Davidson iteration}

It typically takes 10--30 iterations for the Davidson algorithm to get converged results.
The default value is \verb+100+.

\subsection{{\tt TARGET} and {\tt TARGETSYM}--- the target root and its symmetry}

At the moment, the first and second geometrical derivatives of excitation 
energies that are needed in force, geometry, and frequency calculations are
obtained by numerical differentiation.  These keywords may be used to specify
which excited state root is being used for the geometrical derivative calculation.
For instance, when \verb+TARGET 3+ and \verb+TARGETSYM a1g+ are included in the
input block, the total energy (ground state energy plus excitation energy) 
of the third lowest excited state root (excluding the ground state) transforming as
the irreducible representation \verb+a1g+ will be passed to the module which performs
the derivative calculations.  The default values of these keywords are \verb+1+ and \verb+none+,
respectively.

The keyword \verb+TARGETSYM+ is essential in excited state geometry 
optimization, since it is very common that the order of excited states changes due to 
the geometry changes in the course of optimization.  Without specifying the \verb+TARGETSYM+,
the optimizer could (and would likely) be optimizing the geometry of an excited state that
is different from the one the user had intended to optimize at the starting geometry.
On the other hand, in the frequency calculations, \verb+TARGETSYM+ must be \verb+none+,
since the finite displacements given in the course of frequency calculations will lift
the spatial symmetry of the equilibrium geometry.  When these finite displacements can
alter the order of excited states including the target state, the frequency calculation
is not be feasible.

\subsection{{\tt SYMMETRY} --- restricting the excited state symmetry}

By adding this keyword to the input block, the user can request the module to
generate the initial guess vectors transforming as the 
same irreducible representation as \verb+TARGETSYM+.  This causes the final
excited state roots be (exclusively) dominated by those with the specified 
irreducible representation.  This may be useful, when the user is interested in
just the optically allowed transitions, or in the geometry optimization of
an excited state root with a particular irreducible representation.  By default,
this option is not set.  \verb+TARGETSYM+ must be specified when \verb+SYMMETRY+ 
is invoked.

\subsection{{\tt ALGORITHM} --- algorithms for tensor contractions}

There are four distinct algorithms to choose from, and the default value
of \verb+0+ (optimal) means that the program makes an optimal choice from the four
algorithms on the basis of available memory.  In the order of decreasing memory requirement,
the four algorithms are:
\begin{itemize}
\item \verb+ALGORITHM 1+ : Incore, multiple tensor contraction,
\item \verb+ALGORITHM 2+ : Incore, single tensor contraction,
\item \verb+ALGORITHM 3+ : Disk-based, multiple tensor contraction,
\item \verb+ALGORITHM 4+ : Disk-based, single tensor contraction.
\end{itemize}
The incore algorithm stores all the trial and product vectors in memory across
different nodes with the GA,
and often decreases the \verb+MAXITER+ value to accommodate them.  The disk-based
algorithm stores the vectors on disks across different nodes with the DRA, and
retrieves each vector one at a time when it is needed.  The multiple and single
tensor contraction refers to whether just one or more than one trial vectors
are contracted with integrals.  The multiple tensor contraction algorithm is 
particularly effective (in terms of speed) for CIS and TDHF, since the number of 
the direct evaluations of two-electron integrals is diminished substantially.

\subsection{{\tt VECTOR} --- initial guess vectors}

The user may request the module to read the initial guess vectors from a file,
by specifying the file name following the keyword \verb+VECTOR+.  When no file name
is supplied for this keyword, the \verb+<file_prefix>+ appended by \verb+.tddft+ is assumed as 
the file name.  For the file to be compatible, it must be created from the calculation
with the same wave function type (spin-restricted or unrestricted) and the same trial
vector lengths, but the exchange-correlation functionals, 
whether to use the Tamm--Dancoff approximation,
may be different.

\subsection{{\tt FREEZE} --- the frozen core/virtual approximation}

Some of the lowest-lying core orbitals and/or some of the highest-lying
virtual orbitals may be excluded in the CIS, TDHF, and TDDFT calculations
by this keyword (this does not affect the ground state HF or DFT calculation).
No orbitals are frozen by default.  To exclude the atom-like
core regions altogether, one may request
\begin{verbatim}
  FREEZE atomic
\end{verbatim}
To specify the number of lowest-lying occupied orbitals be excluded, one may use
\begin{verbatim}
  FREEZE 10
\end{verbatim}
which causes 10 lowest-lying occupied orbitals excluded.
This is equivalent to writing
\begin{verbatim}
  FREEZE core 10
\end{verbatim}
To freeze the highest virtual orbitals, use the \verb+virtual+
keyword.  For instance, to freeze the top 5 virtuals
\begin{verbatim}
  FREEZE virtual 5
\end{verbatim}

\subsection{{\tt PRINT} --- the verbosity}

This keyword changes the level of output verbosity.  One may also
request some particular items in Table \ref{tbl:tddft-printable} printed.

\begin{table}[htbp]
\center
\caption{Printable items in the TDDFT modules and their default print levels.}
\label{tbl:tddft-printable}
\begin{tabular}{lll}
\hline\hline
Item                     & Print Level   & Description \\
\hline 
``timings''              & high          & CPU and wall times spent in each step \\
``trial vectors''        & high          & Trial CI vectors \\
``initial guess''        & debug         & Initial guess CI vectors \\
``general information''  & default       & General information \\
``xc information''       & default       & HF/DFT information \\
``memory information''   & default       & Memory information \\
``convergence''          & debug         & Convergence \\
``subspace''             & debug         & Subspace representation of CI matrices \\
``transform''            & debug         & MO to AO and AO to MO transformation of CI vectors \\
``diagonalization''      & debug         & Diagonalization of CI matrices \\
``iteration''            & default       & Davidson iteration update \\
``contract''             & debug         & Integral transition density contraction \\
``ground state''         & default       & Final result for ground state \\
``excited state''        & low           & Final result for target excited state \\
\hline\hline
\end{tabular}
\end{center}
\end{table}

\section{Sample input}

The following is a sample input for a spin-restricted TDDFT calculation of 
singlet excitation energies for the water molecule at the B3LYP/6-31G*.
\begin{verbatim}
START h2o

TITLE "B3LYP/6-31G* H2O"

GEOMETRY
O     0.00000000     0.00000000     0.12982363
H     0.75933475     0.00000000    -0.46621158
H    -0.75933475     0.00000000    -0.46621158
END

BASIS
* library 6-31G*
END

DFT
XC B3LYP
END

TDDFT
RPA
NROOTS 20
END

TASK TDDFT ENERGY
\end{verbatim}

To perform a spin-unrestricted TDHF/aug-cc-pVDZ calculation for the CO+ radical,
\begin{verbatim}
START co

TITLE "TDHF/aug-cc-pVDZ CO+"

CHARGE 1

GEOMETRY
C  0.0  0.0  0.0
O  1.5  0.0  0.0
END

BASIS
* library aug-cc-pVDZ
END

DFT
XC HFexch
MULT 2
END

TDDFT
RPA
NROOTS 5
END

TASK TDDFT ENERGY
\end{verbatim}

A geometry optimization followed by a frequency calculation for an excited state
is carried out for BF at the CIS/6-31G* level in the following sample input.
\begin{verbatim}
START bf

TITLE "CIS/6-31G* BF optimization frequencies"

GEOMETRY
B 0.0 0.0 0.0
F 0.0 0.0 1.2
END

BASIS
* library 6-31G*
END

DFT
XC HFexch
END

TDDFT
CIS
NROOTS 3
NOTRIPLET
TARGET 1
END

TASK TDDFT OPTIMIZE

TASK TDDFT FREQUENCIES
\end{verbatim}
