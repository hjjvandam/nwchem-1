\label{sec:sodft}


The spin-orbit DFT module (SODFT) in the NWChem code allows for the variational treatment
of the one-electron spin-orbit operator within the DFT framework. The implementation 
requires the definition of an effective core potential (ECP) and a matching spin-orbit
potential (SO). The current implementation does NOT use symmetry. 

The actual SODFT calculation will be performed when the input module
encounters the \verb+TASK+ directive (Section \ref{sec:task}).  

\begin{verbatim}
  TASK SODFT
\end{verbatim}

Input parameters are the same as for the DFT, see section \ref{sec:dft} for specifications. 
Some of the DFT options are not available in the SODFT. These are \verb+max_ovl+ and 
\verb+sic+.

Besides using the standard ECP and basis sets, see Section \ref{sec:ecp} for details, one 
also has to specify a spin-orbit (SO) potential. The input specification for the SO potential
can be found in section \ref{sec:spinorb_ecp}. At this time we have not included any spin-orbit
potentials in the basis set library.

Note: One should use a combination of ECP and SO potentials that were designed for the same 
size core, i.e. don't use a small core ECP potential with a large core SO potential (it will
produce erroneous results).
