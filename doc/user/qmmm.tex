% $Id: qmmm.tex,v 1.15 2005-12-20 01:50:48 marat Exp $

\label{sec:qmmm}

Combined or hybrid Quantum Mechanics and Molecular Mechanics (QM/MM)
is a simulation methodology that is about 15 years old but in all the
literature there are cautions that calibration computations must be
done to validate the model for each particular chemical system
studied.  This is not a black box style computation and the NWChem
users are advised that without calibration QM/MM may not give the
appropriate results\footnote{c.f., Singh and Kollman, J. Comp. Chem.
  {\bf 7}, 718 (1986); M.~J.~Field, P.~A.~Bash and M.~Karplus, J.
  Comp. Chem. {\bf 11}, 700, (1990); J. Gao, ``Methods and
  Applications of Combined Quantum Mechanical and Molecular Mechanical
  Potentials.'' In {\it Reviews in Computational Chemistry};
  K.~B.~Lipkowitz, D.~B.~Boyd, Eds.; VCH Publishers: New York;
  Vol. 7, pp 119-185 (1995); and M. A. Thompson and G. K. Schenter, J.
  Phys. Chem {\bf 99} 6374 (1995) }. Since both quantum-mechanical and classical
molecular mechanics are involved in the calculation good working knowledge of the two methods
is required to ensure meaningful results. 

The QM/MM module is invoked with the following task directive.

\begin{verbatim}
  task qmmm <string qmtheory> <string operation> [numerical] [ignore]
\end{verbatim}
where {\it qmtheory} specifies quantum method for the calculation of the quantum region. It is expected that 
most of QM/MM simulations will be performed with 
with HF or DFT theories, but any other QM theory supported by NWChem should also work.
Currently the supported operations for QM/MM runs
are energy, optimize, saddle, dynamics, numerical hessian, and numerical
frequencies. 


Unlike pure 
quantum mechanical calculations the information about the chemical system
for QM/MM simulations is contained  not in the geometry block but in the externally prepared topology and restart files.
These files have to be present prior to any QM/MM simulation. 
The input file for QM/MM simulation can be divided into three major parts -- specification of the molecular 
mechanics parameters for the classical region, specification of the quantum mechanical method for the quantum region, 
and the parameters of the interaction between quantum and classical methods. 
All this discussed in detail in the sections below.

\section{Preparation of the restart and topology files}

Generated by the prepare
module (see section \ref{sec:prepare}) restart and topology files contain
information about the classical force field as well as the
coordinates of quantum (QM) and molecular
mechanics (MM) regions. 
In a typical setting this "preparation stage" 
would be run separately from main QM/MM simulation. This will require a 
properly formatted
PDB file for the system. In more complex cases (e.g.non-standard residues or nucleotides) additional fragment and parameter
files might have to be provided by the user. The definition of the quantum region
in the input for the prepare module is specified by either {\it modify atom} directive (see Section \ref{sec:prepare}):

\begin{verbatim}
modify atom <integer isgm>:<string atomname> quantum
\end{verbatim}

or {\it modify segment} directive

\begin{verbatim}
modify segment <integer isgm> quantum
\end{verbatim}

Here {\it isgm} and {\it atomname} refer to the residue number and atom name record 
as given in the PDB file.
It is important to note that  
that the leading blanks 
in atom name record should be indicated with underscores.
Per PDB format quidelines the atom name record starts at column 13. If, for example, 
the atom name record "OW" starts 
in the 14th column in PDB file, it will appear 
as "\_OW" in the modify atom directive in the prepare block. In the current implementation
only solute atoms can be declared as quantum. If part of the solvent has to be treated quantum mechanically
then it has to redeclared to be solute.
In addition to modify commands the prepare input block should 
also contain {\it update lists} and {\it ignore} directives. There are other options
that can be used in the input block for the prepare module ( e.g. solvating the structure, etc ),
 those discussed 
in more details in Section \ref{sec:prepare}.
The successful run of the prepare module will result in generation of
topology and restart files. Similar to classical MD, both files are required for QM/MM simulations 
and have to be placed in the same directory
as the input file. Here is an example input file that will generate QM/MM restart and topology files for the ethanol molecule

\begin{verbatim}
title "Prepare QM/MM calculation of ethanol"
start etl

prepare
#--name of the pdb file
   source etl0.pdb                    
#--generate new topology and sequence file
   new_top new_seq                    
#--generate new restart file
   new_rst                            
#--define quantum region (note the use of underscore)
   modify atom 1:_C1  quantum         
   modify atom 1:2H1  quantum         
   modify atom 1:3H1  quantum         
   modify atom 1:4H1  quantum         
#
 update lists
 ignore
end
task prepare
\end{verbatim}

These are contents of etl0.pdb file used in the above input file.
\begin{verbatim}
ATOM      1  O   etl     1       1.201  -0.271  -0.000  1.00  0.00           O
ATOM      2  H   etl     1       1.995   0.329  -0.000  1.00  0.00           H
ATOM      3  C1  etl     1      -1.180  -0.393   0.000  1.00  0.00           C
ATOM      4 2H1  etl     1      -2.128   0.155  -0.000  1.00  0.00           H
ATOM      5 3H1  etl     1      -1.130  -1.030   0.887  1.00  0.00           H
ATOM      6 4H1  etl     1      -1.130  -1.030  -0.887  1.00  0.00           H
ATOM      7  C2  etl     1       0.006   0.573   0.000  1.00  0.00           C
ATOM      8 2H2  etl     1      -0.042   1.220   0.890  1.00  0.00           H
ATOM      9 3H2  etl     1      -0.042   1.220  -0.890  1.00  0.00           H
END
\end{verbatim}

Running the input shown above will produce (among other things) the topology file (etl.top) and the restart file
(etl{\_}md.rst). The naming of the topology file follows after the rtdb name specified in the start directive in the input (i.e. "start etl"),
while the "{\_}md" suffix in the restart file name is specific to the way prepare module works in this particular case. If necessary, this
particular naming scheme can be altered using {\it system} keyword 
in the prepare input block (for more details see Section \ref{sec:prepare}).

Here is more complicated example where the entire ethanol molecule
is declared quantum and solvated in a box of spce waters:
\begin{verbatim}
title "Prepare QM/MM calculation of solvated ethanol"
start etl
prepare
source etl0.pdb
new_top new_seq
new_rst
#center and orient prior to solvation
center
orient
#solvation in 1 nm by 2 nm by 3 nm box
solvate box 1.0 2.0 3.0
#the whole ethanol is declared quantum now
modify segment 1 quantum

update lists
ignore
end
task prepare

\end{verbatim}

Fixing atoms outside a certain distance
from the QM region can also be accomplished using prepare module. These 
constraints will then be permanently 
embedded in the resulting restart file, which may be advantageous for ceratain
types of QM/MM simulations. The actual format for the constraint directive to fix whole residues
is 

\begin{verbatim}
fix segments beyond  <real radius>  <integer residue number>:<string atom name>
\end{verbatim}
or to fix on atom basis
\begin{verbatim}
fix atoms beyond  <real radius>  <integer residue number>:<string atom name>
\end{verbatim}

The following input file illustrated the use of fix segments directive

\begin{verbatim}
start etl

prepare
source etl0.pdb
new_top new_seq
new_rst
center
orient
#solvation in 40 A cubic box
solvate cube 4.0
modify segment 1 quantum
#fix residues more than 20 A away from ethanol oxygen atom
fix segments beyond 2.0 1:_O
update lists
ignore
end
task prepare

\end{verbatim}

\section{Molecular Mechanics Parameters}
The molecular mechanics parameters are given in the form of standard MD input block as
used by the MD module (c.f. Section \ref{sec:nwmd}). This input block
is required for QM/MM simulations. It specifies the 
restart and topology file that will be used in the calculation.
It also contains information relevant to the calculation 
of the  classical region 
(e.g. cutoff distances, constraints, optimization and dynamics parameters, etc)
in the system. In this input block one can also set fixed atom constraints on both classical and quantum atoms. Continuing with our
example for ethanol molecule here is a simple input block that may be used for this system.

\begin{verbatim}
 md
# this specifies that etl_md.rst will be used as a restart file
#  and etl.top will be a topology file
  system etl_md
# if we ever wanted to fix C1 atom 
  fix solute 1 _C1
 end
\end{verbatim}


\section{Quantum Mechanical Parameters}
 
The parameters defining calculation of the QM region (including basis sets) 
must be present in the traditional NWChem input format
except for the geometry block. The geometrical information will
be constructed automatically using information contained in the restart file

\section{QM/MM interface parameters}

The QM/MM interface parameters define the interaction between classical and quantum regions.
The  input follows standard NWChem format:
\begin{verbatim}
  qmmm
  [ eref <double precision default 0.0d0>]
  [ bqzone <double precision default 9.0d0>]
  [ mm_charges [exclude <(none||all||linkbond||linkbond_H) default none>]
           [ expand  <none||all||solute||solvent> default none]
           [ update  <integer default 0>] 
        
  [ link_atoms <(hydrogen||halogen) default hydrogen>]
  [ link_ecp  <(auto||user) default auto>]
  [ region   < [region1]  [region2]  [region3] > ]
  [ method   [method1]  [method2]  [method3]  ]
  [ maxiter  [maxiter1] [maxiter2] [maxiter3] ]
  [ ncycles  < [number] default 1 > ]
  [ density  [espfit] [static] [dynamical] default dynamical ]
  end
\end{verbatim}

Detailed explanation of the subdirectives in the QM/MM input block is given below:

\begin{itemize}

\item
\begin{verbatim}
 eref <double precision default 0.0d0>
\end{verbatim}

This directive sets the relative zero of energy for the
QM component of the system. The need for this directive
arizes from different definitions of zero energy for QM and MM methods.
Most QM methods define the zero of energy for the system as
vacuum. The zero of energy for the MM system is by
definition of most parameterized force fields the separated atom
energy. Therefore in many cases the energetics of the QM system 
will likely overshadow the
MM component of the system. This imbalance can be corrected by
suitably chosen value of {\it eref}

\item
\begin{verbatim}
 cutoff <double precision default 9.0d0>
\end{verbatim}

This directive defines the radius of the zone (in angstroms) around the quantum region
where classical residues/segments
will be allowed to interact with quantum region both electrostatically and through
Van der Waals interactions. It should be noted
that classical atoms interacting with quantum region via bonded interactions are always
included (this is true even if bqzone is set to $0.0$). In addition, even if one atom
of a given charged group is in the bqzone (residues are typically treated as one charged group) then the whole
group will be included.


\item
\begin{verbatim}
mm_charges [exclude <(none||all||linkbond||linkbond_H) default none>]
           [expand  <none||all||solute||solvent> default none]
           [update  <integer default 0>] 
\end{verbatim}

This directive controls treatment of classical point (MM) charges that are interacting
with QM region. For most QM/MM applications the use of directive will be not be necessary. Its
absence would be simply mean that all MM charges within the cuttof distance ( as specified by {\it cutoff} ) as
well those belonging to the charges groups directly bonded to QM region will be allowed to interact with QM region.

Keyword {\it exclude} specifies the subset MM charges that will be specifically excluded from interacting with QM region.
\begin{itemize}
\item
{\it none} default value reverts to the original set of MM charges as described above.
\item
{\it all} excludes all MM charges from interacting 
with QM region ("gas phase" calculation). 
\item
{\it linkbond} excludes MM charges that are connected to a quantum region
by at most two bonds,
\item
{\it linkbond{\_}H} similar to {\it linkbond} but excludes only hydrogen atoms.
\end{itemize}

Keyword {\it expand} expands the set MM charges interacting with QM region beyond the limits imposed by {\it cutoff} value.
\begin{itemize}
\item
{\it none} default value reverts to the original set of MM charges
\item
{\it solute} expands electrostatic interaction to all solute MM charges
\item
{\it solvent} expands electrostatic interaction to all solvent MM charges
\item
{\it all} expands electrostatic interaction to all MM charges
\end{itemize}

Keyword {\it update} specifies how often list of MM charges will be updated in the course of the calculation. Default behavior
is not to update.

\item
\begin{verbatim}
link_atoms <(hydrogen||halogen) default halogen>
\end{verbatim}

This directive controls the treatment of bonds crossing the boundary between quantum and classical regions.
The use of {\it hydrogen } keyword will trigger truncation of such bonds with hydrogen link atoms. The position of the hydrogen
atom will be calculated from the coordinates of the quantum and classical atom of the truncated bond using the
following expression
\begin{displaymath}
\mathbf{R}_{hlink} = (1-g)\mathbf{R}_{quant} + g*\mathbf{R}_{class}
\end{displaymath}
where $g$ is the scale factor set at $0.709$

Setting link{\_}atoms to {\it halogen } will result in the modification of the {\it \underline{quantum}} 
atom of the truncated bond to 
to the fluoride atom. This fluoride atom will typically carry an effective core potential (ECP) basis set as specified
in {\it link{\_}ecp} directive.

\item
\begin{verbatim}
 link_ecp  <(auto||user) default auto> 
\end{verbatim}
This directive specifies ECP basis set on fluoride link atoms. If set to {\it auto } 
the ECP basis set given by Zhang, Lee, Yang for 6-31G* basis.\footnote{Y. Zhang, T. Lee, and W. Yang, J. Chem. Phys. 110, 46 (1999)} 
will be used. Strictly speaking, this implies the use of 6-31G* spherical basis as the main basis set. 
If other choices are desired then keyword {\it user } should be used and ECP basis set should be entered separatelly 
following the format given in
section \ref{sec:ecp}. 
The name tag for fluoride link atoms is F{\_}L. 

\item
\begin{verbatim}
region  < [region1]  [region2]  [region3] >
\end{verbatim}

This directive specifies active region(s) for optimization, dynamics, and frequency calculations.
Up to three regions can be specified, those are limited to 
\begin{itemize}
\item
"qm" - all quantum atoms
\item
"qmlink" - quantum and link atoms,
\item
"mm{\_}solute" - all classical atoms excluding link atoms
\item
"solvent" all solvent atoms
\item
"solute" - all solute atoms quantum or not,
\item
"mm" all classical solute and solvent atoms, excluding link atoms
\item
"all" all atoms 
\end{itemize}
Only the first region will be used in dynamics and frequency calculations. In the geometry optimizations, all three regions will
be optimized in succession possibly with different optimization algorithms (see directives {\it method},  {\it maxiter},
{\it ncycles} below).

\item
\begin{verbatim}
method   [method1]  [method2]  [method3]
\end{verbatim}

This directive controls which optimization algorithm will be used for the regions as defined
by {\it regions} directive. The allowed values are "bfgs" aka driver, "lbfgs" limited memory version of quasi-newton,
and "sd" simple steepest descent algorithm. The "bfgs" is the most expensive algorithm and is not 
recomended for more than ~100 particles (this essentially restricts its usage only to "qm" or "qmlink" regions).
The "lbfgs" algorithm is recomended for both small and large regions and should be used whenever is possible (in many
cases it outperforms "bfgs"). Finally "sd" the most ineficient and slow way to optimize regions, yet it is the only option
available for the optimization of solvent regions. The default is to assign "sd" to optimization involving solvent region (if any),
and "lbfgs" to all others.



\item
\begin{verbatim}
maxiter  [maxiter1] [maxiter2] [maxiter3]
\end{verbatim}

This directive controls maximum number of iterations for the optimizations of regions as defined by
by {\it regions} directive. 



\item
\begin{verbatim}
density  [espfit] [static] [dynamical] default dynamical
\end{verbatim}

This directive controls the way electron density of the quantum region is treated when calculating
electrostatic interactions with classical charges. This directive affects QM/MM calculations where both
quantum and link atoms are inactive ( i.e. active region is "mm{\_}solute","solvent", or "mm" ).
If set to "espfit" electron density of the quantum region will be approximated to by point charges
fitted by ESP procedure. This can dramatically speed up both dynamical and optimization tasks that involve purely classical regions.
Note that "espfit" option does require movecs file which for example
can be obtained by running qmmm energy calculation prior to optimization. 
The "static" option will freeze the electron density when quantum and link atoms are inactive. It is more accurate than
"espfit" option but also more expensive. Finally, the default "dynamical" option means that the electron density
is treated the normal way throught the solution of Schrodinger equation. All calculations that involve any geometry
changes in quantum or links atoms will automatically use this option.

\item
\begin{verbatim}
ncycles  < [number] default 1 >
\end{verbatim}

This directive controls the number of optimization cycles where the defined regions will be optimized in succession.
See also {\it etol}

\item
\begin{verbatim}
load  < esp > [<filename>]
\end{verbatim}

This directive instructs to load external file (located in permanent directory) containing esp charges for QM region.
If filename is not provided it will be constructed from the name of the restart file by replacing ".rst" 
suffix with ".esp".
Note that file containing esp charges is always generated whenever esp charge calculation is performed,

\item
\begin{verbatim}
convergence  < double precision etol default 1.0d-4> 
\end{verbatim}

This directive controls convergence of geometry optimization. The optimization is deemed converged if absolute difference in 
total energies between consequitive optimization cycles becomes less than 
{\it etol} 



\end{itemize}



