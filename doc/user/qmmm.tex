% $Id: qmmm.tex,v 1.7 2000-05-03 19:37:59 d3j191 Exp $

\label{sec:qmmm}

Combined or hybrid Quantum Mechanics and Molecular Mechanics (QM/MM)
is a simulation methodology that is about 15 years old but in all the
literature there are cautions that calibration computations must be
done to validate the model for each particular chemical system
studied.  This is not a black box style computation and the NWChem
users are advised that without calibration QM/MM may not give the
appropriate results\footnote{c.f., Singh and Kollman, J. Comp. Chem.
  {\bf 7}, 718 (1986); M.~J.~Field, P.~A.~Bash and M.~Karplus, J.
  Comp. Chem. {\bf 11}, 700, (1990); J. Gao, ``Methods and
  Applications of Combined Quantum Mechanical and Molecular Mechanical
  Potentials.'' In {\it Reviews in Computational Chemistry};
  K.~B.~Lipkowitz, D.~B.~Boyd, Eds.; VCH Publishers: New York;
  Vol. 7, pp 119-185 (1995); and M. A. Thompson and G. K. Schenter, J.
  Phys. Chem {\bf 99} 6374 (1995) }.

The QM/MM module in NWChem is driven by the molecular dynamics module.
This module currently works for any QM method that has
analytic gradients\footnote{The QM/MM method will work with numerical
  gradients available in NWChem, but it is expected that the
  performance will not allow any substantive simulations}.  The input
for this requires the definition of chemical system via the same
interface that is used by the MD module (c.f. Section
\ref{sec:nwmd}).  The extensions to this interface include the
definition of ``Quantum'' atoms and ``Link'' where appropriate.  The
QM information must be present in the traditional NWChem input deck
except for the geometry\footnote{Any geometry information in the
  traditional form will be ignored}.  The geometrical information will
be constructed automatically by nwmd.  For dynamics and free energy
simulations the input is again identical to that for nwmd with
limitations on the kinds of simulations that can be done.

The QM/MM module is invoked with the task directive where the
``theory'' is QMMM.  The recognized operations on the QM/MM theory
directive are energy, optimize, and dynamics.

\begin{verbatim}
  TASK QMMM (energy | optimize | dynamics)
\end{verbatim}

Tasks \verb+gradient+, \verb+saddle+, \verb+frequencies+ and
\verb+thermodynamics+ are currently not available in the QM/MM mode.  


The QM/MM input consists of the standard NWChem input block:
\begin{verbatim}
  QMMM
    ...
  END
\end{verbatim}

The \verb+QMMM+ has the following the additional sub-directive that the user
may specify for the particular simulation.  These options currently are:

\section{EATOMS}
There is one compound input directive that must exist for the QM/MM
simulation to proceed.  This sets the relative zero of energy for the
QM component of the system.  It is not incorrect to leave this value as
zero but the energetics of the QM system will likely over shadow the
MM component of the system.  Properties based on energy fluctuations
of the system will be overly sensitive to the energy of the QM
component of the system.  The zero of energy for the MM system is by
definition of most parameterized force fields the separated atom
energy.  The zero of energy for QM systems by definition of most QM
methods is the vacuum.  The {\it a priori} determination of the
separated atom energy for a particular QM method is not well defined
and thus leads to a number of assumptions or guess work depending upon
the particular QM method being utilized.  Therefor, the determination
of the QM separated atom energy (``eatoms'') is left to the user.  The
input takes the form:

\begin{verbatim}
  EATOMS <real eatoms>
\end{verbatim}

There is no default for this and the input {\bf must} be present for a
QM/MM simulation.  

All other parameters that control the QM/MM simulation are set via the
input to nwmd (see chapter \ref{sec:nwmd}).

