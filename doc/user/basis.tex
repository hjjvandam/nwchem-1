\label{sec:basis} 

NWChem currently supports basis sets consisting of generally
contracted\footnote{Generally contracted meaning that the same
  primitive, Gaussian functions are contracted into multiple
  contracted functions using different contraction coefficients.
  Reuse of the radial functions increases the efficiency of integral
  generation.} Cartesian Gaussian functions up to a maximum angular
momentum of seven ($i$ functions), and also $sp$ (or L)
functions\footnote{An $sp$ shell is two-component general contraction.
  However, the first component specifies an $s$ shell and the second a
  $p$ shell.  Again, reuse of the radial functions increases the efficiency
  of integral generation.} .  The {\tt BASIS} directive is used to
define these, and also to specify use of an effective core potential
(ECP) that is associated with a basis set; see Section \ref{sec:ecp}.)

The basis functions to be used for a given calculation can be drawn
from a standard set in the EMSL basis set library that is included in
the release of NWChem  (See Appendix \ref{sec:knownbasis} for a list
of the standard basis sets currently supplied with the release of the
code).  Alternatively, the user can specify particular functions
explicitly in the input, to define a particular basis set.

The general form of the \verb+BASIS+ directive is as follows:

\begin{verbatim}
  BASIS [<string name default "ao basis">] \
        [(spherical || cartesian) default cartesian] \
        [(segment || nosegment) default segment] \
        [(print || noprint) default print]
        [rel]

     <string tag> library [<string tag_in_lib>] \
                  <string standard_set> [file <filename>] \
                  [except <string tag list>] [rel]

        ...

     <string tag> <string shell_type> [rel]
        <real exponent> <real list_of_coefficients>
        ...
     
  END
\end{verbatim}    

Examining the keywords on the first line of the \verb+BASIS+ directive:


\begin{itemize}
\item {\tt name}

  By default, the basis set is stored in the database with the name
  \verb+"ao basis"+.  Another name may be specified in the \verb+BASIS+
  directive, thus, multiple basis sets may be stored simultaneously in the
  database.  Also, the DFT (Section \ref{sec:dft}), RI-SCF (Section
  \ref{sec:riscf}) and RIMP2 (Section \ref{sec:rimp2}) modules and the
  Dyall-modified-Dirac relativistic method (Section \ref{sec:dyall-mod-dir})
  require multiple basis sets with specific names.

The user can associate the \verb+"ao basis"+ with another named basis
using the \verb+SET+ directive (see Section \ref{sec:set}).  

\item {{\tt SPHERICAL} or {\tt CARTESIAN}}

The keywords \verb+spherical+ and \verb+cartesian+ offer the option of
using either spherical-harmonic (5 d, 7 f, 9 g, \ldots) or Cartesian
(6 d, 10 f, 15 g, \ldots) angular functions.  The default is
Cartesian.  

Note that the correlation-consistent basis sets were designed using
spherical harmonics and to use these, the \verb+spherical+ keyword
should be present in the \verb+BASIS+ directive.  The use of spherical
functions also helps eliminate problems with linear dependence.


\item {{\tt SEGMENT} or {\tt NOSEGMENT}}

By default, NWChem forces all basis sets to be segmented, 
even if they are input with general contractions or $L$ or sp
shells. This is because the current derivative integral program cannot
handle general contractions.  If a calculation is  
computing energies only, a 
performance gain can result from exploiting generally contracted basis
sets, in which case {\tt NOSEGMENT} should be specified.

\item {{\tt PRINT} or {\tt NOPRINT}}

The default is for the input module to print all basis sets encountered.
Specifying the keyword \verb+noprint+ allows the user to suppress this output.

\item {{\tt REL}}

This keyword marks the entire basis as a relativistic basis for the purposes
of the Dyall-modified-Dirac relativistic integral code. The marking of the
basis set is necessary for the code to make the proper association between
the relativistic shells in the ao basis and the shells in the large and/or
small component basis. This is only necessary for basis sets which are to be
used as the ao basis. The user is referred to Section \ref{sec:dyall-mod-dir}  
for more details.

\end{itemize}

Basis sets are associated with centers by using the tag of a center in
a geometry that has either been input by the user (Section
\ref{sec:geom}) or is available elsewhere.  Each atom or center with
the same \verb+tag+ will have the same basis set.  All atoms must have
basis functions assigned to them --- only dummy centers may have no
basis functions.  To facilitate the specification of the geometry and
the basis set for any chemical system, the matching process of a basis
set tag to a geometry tag first looks for an exact match.  If no match
is found, NWChem will attempt to match, ignoring case, the name or
symbol of the element.  E.g., all hydrogen atoms in a system could be
labeled ``H1'', ``H2'', \ldots, in the geometry but only
one basis set specification for ``H'' or ``hydrogen'' is necessary.
If desired, a special basis may be added to one or more centers (e.g.,
``H1'') by providing a basis for that tag.
If the matching mechanism fails then NWChem stops with an appropriate
error message.

A special set of tags, ``*'' and tags ending with a ``*'' (E.g. ``H*'')
can be used in combination with the keyword \verb+library+ (see section
below). These tags facilitate the definition of a certain type of basis 
set of all atoms, or a group of atoms, in a geometry using only a single 
or very few basis set entries.

Examined next is how to reference standard basis sets in the basis set
library, and finally, how to define a basis set using exponents and
coefficients.

\section{Basis set library}

The keyword \verb+library+ associated with each specific \verb+tag+
entry specifies that the calculation will use the standard basis set
in NWChem for that center.  The variable \verb+<standard_set>+ is the
name that identifies the functions in the library.  The names of
standard basis sets are not case sensitive.  See Appendix
\ref{sec:knownbasis} for a complete list of the basis sets in the
NWChem library and their specifications.  

The general form of the input line requesting basis sets from the NWChem
basis set library is:
\begin{verbatim}
     <string tag> library [<string tag_in_lib>] \
                  <string standard set> [file < filename> \
                  [except <string tag list>] [rel]
        ...
\end{verbatim}

For example, the NWChem basis set library contains the Dunning cc-pvdz
basis set.  These may be used as follows
\begin{verbatim}
  basis
    oxygen library cc-pvdz
    hydrogen library cc-pvdz
  end
\end{verbatim}

A default location/path of the NWChem basis set libraries is provided on 
installation of the code, but a different path can be defined by specifying 
the keyword \verb+file+, and one can explicitly name the file to be accessed
for the basis functions. For example,
\begin{verbatim}
  basis
    o  library 3-21g file /usr/d3g681/nwchem/library
    si library 3-21g file /usr/d3g681/nwchem/library
  end
\end{verbatim}
This directive tells the code to use the basis sets \verb+3-21g+ in
the file {\tt /usr/\-d3g681/\-nwchem/\-library} for atoms \verb+o+ and
\verb+si+, rather than look for them in the default librares.

The ``*'' tag can be used to efficiently define basis set input directives 
for large numbers of atoms. For example,
\begin{verbatim}
  basis
    *  library 3-21g
  end
\end{verbatim}
This directive tells the code to assign the basis sets \verb+3-21g+ to
all the atom tags defined in the geometry. If one wants to place a
different basis set on one of the atoms defined in the geometry, the 
following directive can be used:
\begin{verbatim}
  basis
    *  library 3-21g except H
  end
\end{verbatim}
This directive tells the code to assign the basis sets \verb+3-21g+ to
all the atoms in the geometry, except the hydrogen atoms. Remember that 
the user will have to explicitly define the hydrogen basis set in this
directive! One may also define tags that end with a ``*'': 
\begin{verbatim}
  basis
    oxy*  library 3-21g 
  end
\end{verbatim}
This directive tells the code to assign the basis sets \verb+3-21g+ to 
all atom tags in the geometry that start with ``oxy''.

If standard basis sets are to be placed upon a dummy center, the
variable \verb+<tag_in_lib>+ must also be entered on this line, to
identify the correct atom type to use from the basis function library
(see the ghost atom example in Section \ref{sec:set} and below).  For
example: To specify the cc-pvdz basis for a calculation on the water
monomer in the dimer basis, where the dummy oxygen and dummy hydrogen
centers have been identified as \verb+bqo+ and \verb+bqh+
respectively, the \verb+BASIS+ directive is as follows:

\begin{verbatim}
  basis
    o   library cc-pvdz
    h   library cc-pvdz
    bqo library o cc-pvdz
    bqh library h cc-pvdz
  end
\end{verbatim}
A special dummy center tag is \verb+bq*+, which will assign the same basis 
set to all bq centers in the geometry. Just as with the ``*'' tag, the 
\verb+except+ list can be used to assign basis sets to unique dummy centers.

The library basis sets can also be marked as relativistic by adding the
\verb+rel+ keyword to the tag line. See Section \ref{sec:dyall-mod-dir} for
more details. The correlation consistent basis sets have been contracted for
relativistic effects and are included in the standard library.

There are also contractions in the standard library for both a point nucleus
and a finite nucleus of Gaussian shape. These are usually distinguished by
the suffixex {\tt \_pt} and {\tt \_fi}. It is the user's responsibility to
ensure that the contraction matches the nuclear type specified in the
geometry object. The specification of a finite nucleus basis set does NOT
automagically set the nuclear type for that atom to be finite.  See 
Section \ref{sec:geom} for information.

\section{Explicit basis set definition}

If the basis sets in the library or available in other external files
are not suitable for a given calculation, 
the basis set may be explicitly defined.
A generally contracted Gaussian basis function is associated with a
center using an input line of the following form:
\begin{verbatim}
     <string tag> <string shell_type> [rel]
        <real exponent> <real list_of_coefficients>
        ...
\end{verbatim}

The variable \verb+<shell_type>+ identifies the angular momentum of the
shell, $s$, $p$, $d$, \ldots.  NWChem is configured to handle up to $i$
shells.  The keyword \verb+rel+ marks the shell as relativistic --- see
Section \ref{sec:dyall-mod-dir} for more details.  Subsequent lines define
the primitive function exponents and contraction coefficients.  General
contractions are specified by including multiple columns of coefficients.

For example, the following \verb+BASIS+ directive augments the Dunning
cc-pvdz basis set for the water molecule with a diffuse s-shell on
oxygen:
\begin{verbatim}
  basis spherical nosegment
    oxygen library cc-pvdz
    hydrogen library cc-pvdz
    oxygen s
      0.01 1.0
  end
\end{verbatim}

This is equivalent to the following explicit specification:
\begin{verbatim}
  basis spherical nosegment
    oxygen s
      11720.0000    0.000710  -0.000160
       1759.0000    0.005470  -0.001263
        400.8000    0.027837  -0.006267
        113.7000    0.104800  -0.025716
         37.0300    0.283062  -0.070924
         13.2700    0.448719  -0.165411
          5.0250    0.270952  -0.116955
          1.0130    0.015458   0.557368
          0.3023   -0.002585   0.572759
    oxygen s                
          0.3023    1.000000
    oxygen p                
         17.7000    0.043018
          3.8540    0.228913
          1.0460    0.508728
          0.2753    0.460531
    oxygen p                
          0.2753    1.000000
    oxygen d
          1.1850    1.000000
    hydrogen s
         13.0100    0.019685
          1.9620    0.137977
          0.4446    0.478148
          0.1220    0.501240
    hydrogen s  
          0.1220    1.000000
    hydrogen p  
          0.7270    1.000000
    oxygen s
          0.01      1.0
  end
\end{verbatim}




