\label{sec:basis} 

NWChem currently supports basis sets consisting of generally
contracted Cartesian Gaussian functions.  The {\tt BASIS} directive is
used to define these, and also to specify use of an effective core
potential (ECP) that is associated with a basis set; see section
\ref{sec:ecp}.)

The basis functions to be used for a given calculation can be drawn
from a standard set in the EMSL basis set library that is included in
the release of NWChem.  (See Appendix \ref{sec:knownbasis} for a list
of the standard basis sets currently supplied with the release of the
code.)  Alternatively, the user can specify particular functions
explicitly in the input, defining a particular basis set.

The general form of the \verb+BASIS+ directive is as follows;

\begin{verbatim}
  BASIS [<string name default "ao basis">] \
        [(spherical || cartesian) default cartesian] \
        [(segment || nosegment) default segment] \
        [(print || noprint) default print]

     <string tag> library [<string tag_in_lib>] \
                  <string standard set> [file <filename>]

        ...

     <string tag> <string shell_type>
        <real exponent> <real list_of_coefficients>
        ...
     
  END
\end{verbatim}    

Examining the keywords on the first line of the \verb+BASIS+ directive;


\begin{itemize}
\item {\tt name}

By default the basis set is stored in the database with the name
\verb+"ao basis"+.  By specifying another name in the \verb+BASIS+
directive, multiple basis sets may be stored simultaneously in the database.
Also, the DFT (Section \ref{sec:dft}), RISCF (Section \ref{sec:riscf})
and RIMP2 (Section \ref{sec:rimp2}) modules require multiple basis
sets with specific names.

The user can associate the \verb+"ao basis"+ with another named basis
using the \verb+SET+ directive (see Section \ref{sec:set}).  

\item {{\tt SPHERICAL} and {\tt CARTESIAN}}

The keywords \verb+spherical+ and \verb+cartesian+ offer the option of
using either spherical-harmonic (5 d, 7 f, 9 g, \ldots) or Cartesian
(6 d, 10 f, 15 g, \ldots) angular functions.  The default is
Cartesian because the current integral derivative program cannot
handle spherical functions.  If only energy calculations are being
performed, then {\tt SPHERICAL} can be specified.  

Note that the correlation-consistent basis sets were designed using
spherical harmonics and to use these the \verb+SPHERICAL+ keyword
should be present on the \verb+BASIS+ directive (in energy
calculations).  Use of spherical functions also helps eliminate
problems with linear dependence.


\item {{\tt SEGMENT} or {\tt NOSEGMENT}}

By default NWChem forces all basis sets to be segmented, 
even if they are input with either general contractions, or $L$ or sp
shells. This is because the current derivative integral package cannot
handle general contractions.  If a calculation is only 
computing energies, a substantial
performance gain can result from exploiting generally contracted basis
sets, in which case {\tt NOSEGMENT} should be specified.

\item {{\tt PRINT} and {\tt NOPRINT}}

The default is for the input module to print all basis sets encountered.
Specifying the keyword \verb+noprint+ allows the user to supress this output.

\end{itemize}

Basis sets are associated with centers by using the tag of a center in
a geometry that has either been input by the user (Section
\ref{sec:geom}) or is available elsewhere.  Each atom or center with
the same \verb+tag+ will have the same basis set.  All atoms must have
basis functions assigned to them --- only dummy centers may have no
basis functions.

Examined next is how to reference standard basis sets in the basis set
library, and, finally, how to define a basis using exponents and
coefficients.

\section{Basis set library}

The keyword \verb+library+ associated with each specific \verb+tag+
entry specifies that the calculation will use the
standard basis set in NWChem for that center.  The string \verb+standard_set+ is the
name that identifies the functions in the library.  The names of
standard basis sets are not case sensitive.  See Appendix
\ref{sec:knownbasis} for a complete list of the basis functions in the
NWChem library and their specification.  If basis functions are to be
placed upon a dummy center, the string \verb+tag_in_lib+ must also be
entered on this line, to identify the correct atom type to use in the
basis function library (see the ghost atom example in Section
\ref{sec:set} and below).

A default path to the basis set library is provided on installation of
the code, but a different path can be defined by specifying the
keyword \verb+file+ and then explicitly naming the file to be accessed
for the basis functions.

For example, NWChem contains the Dunning cc-pvdz basis sets for
oxygen and hydrogen in the standard basis set library.  The \verb+BASIS+
directive to use these basis sets in a calculation of the water molecule
could be formulated quite simply as follows;

\begin{verbatim}
  basis
    oxygen library cc-pvdz
    hydrogen library cc-pvdz
  end
\end{verbatim}

In this example, the cc-pvdz basis functions will be obtained from the
NWChem library, as defined by the default path on installation.  To obtain
functions from some other library, such as the user's own experimental
library of functions, the path name for the library must be specified
explicitly.  An example illustrating this option is as follows,

\begin{verbatim}
  basis
    o  library 3-21g file /usr/d3g681/nwchem/library
    si library 3-21g file /usr/d3g681/nwchem/library
  end
\end{verbatim}

This directive tells the code to use the basis sets \verb+3-21g+ in
the file {\tt /usr/\-d3g681/\-nwchem/\-library} for atoms \verb+o+ and
\verb+si+, rather than looking for them in the default library.

Another example;  to specify the cc-pvdz basis for a calculation on
the water monomer in the dimer basis, where the dummy oxygen and dummy
hydrogen centers have been identified as \verb+bqo+ and \verb+bqh+
respectively, the \verb+BASIS+ directive is as follows,

\begin{verbatim}
  basis
    o   library cc-pvdz
    h   library cc-pvdz
    bqo library o cc-pvdz
    bqh library h cc-pvdz
  end
\end{verbatim}

\section{Explicit basis set definition}

If the basis sets in the library or available in other external files
are not suitable for a given calculation, the user has the option
of defining the basis set explicitly.
A generally contracted Gaussian basis function is associated with a
center using an input line of the following form,
\begin{verbatim}
     <string tag> <string shell_type>
        <real exponent> <real list_of_coefficients>
        ...
\end{verbatim}

The \verb+shell_type+ identifies the angular momentum of the shell,
$s$, $p$, $d$, \ldots.  By default, NWChem is configured to handle up
to $i$ functions.  Subsequent lines define the primitive function
exponents and contraction coefficients.  General contractions are
specified by including multiple coefficients.

For example, the following \verb+BASIS+ directive augments the Dunning
cc-pvdz basis set for the water molecule with a diffuse s-shell on
oxygen
\begin{verbatim}
  basis spherical nosegment
    oxygen library cc-pvdz
    hydrogen library cc-pvdz
    oxygen s
      0.01 1.0
  end
\end{verbatim}

This is equivalent to the following explicit specification
\begin{verbatim}
  basis spherical nosegment
    oxygen s
      11720.0000    0.000710  -0.000160
       1759.0000    0.005470  -0.001263
        400.8000    0.027837  -0.006267
        113.7000    0.104800  -0.025716
         37.0300    0.283062  -0.070924
         13.2700    0.448719  -0.165411
          5.0250    0.270952  -0.116955
          1.0130    0.015458   0.557368
          0.3023   -0.002585   0.572759
    oxygen s                
          0.3023    1.000000
    oxygen p                
         17.7000    0.043018
          3.8540    0.228913
          1.0460    0.508728
          0.2753    0.460531
    oxygen p                
          0.2753    1.000000
    oxygen d
          1.1850    1.000000
    hydrogen s
         13.0100    0.019685
          1.9620    0.137977
          0.4446    0.478148
          0.1220    0.501240
    hydrogen s  
          0.1220    1.000000
    hydrogen p  
          0.7270    1.000000
    oxygen s
          0.01      1.0
  end
\end{verbatim}




