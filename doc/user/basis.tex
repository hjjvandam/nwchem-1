\label{sec:basis}
In NWChem, basis sets consist of sets of generally contracted Cartesian
Gaussian functions.  (An optional functionality also exists that allows the
user to specify an effective core potential (ECP) basis set; see
section \ref{sec:ecpbasis}.)  The basis functions to be used for a given
calculation can be drawn from a standard set in the EMSL basis set library
that is included in the release of NWChem.  (See Appendix \ref{knownbasis} 
for a list of the standard basis sets currently supplied with the release 
of the code.)    Alternatively, the user can specify particular functions
explicitly in the input defining a particular basis set.  

The form of the \verb+BASIS+ directive is as follows;

\begin{verbatim}
  BASIS [<string name default "ao basis">] \
        [spherical || cartesian default cartesian] \
        [segment || nosegment default segment] \
        [print || noprint default print]

     <string tag> library [<string tag_in_lib>] \
                  <string standard set> [file <filename>]

        ...

     <string tag> <string shell_type>
        <real exponent> <real list_of_coefficients>
        ...
     
  END
\end{verbatim}    

The string \verb+name+ allows the user to identify a specific basis set
in the database for a calculation.  The various modules in the code expect 
to find the
basis set in the database under the name \verb+"ao basis"+.  The user can
assign the string \verb+name+ to \verb+"ao basis"+ using the verb+SET+
directive (see Section \ref{sec:set}), in the same manner as that in
which different 
\verb+geometry+ objects are assigned the default name \verb+geometry+ 
for specific tasks.  This indirection allows the user to assign different
basis sets to the same geometry object, or the same basis set to different
geometry objects, for different calculations in
the course of the same job.

The keywords \verb+spherical+ and \verb+cartesian+ offer the option of
specifying the basis functions in either spherical or cartesian coordinates.
The keywords \verb+segment+ and \verb+nosegment+ allow the user to specify
whether or not the functions will be segmented.  Currently, NWChem assumes
that all basis sets are segmented  and cartesian, whether specified by the 
user or by means of the standard library sets.
The options for spherical coordinates and unsegmented functions will become
usable in all calculations only when the derivative integral code is 
extended to handle 
general contractions and sphericals.  Currently, only the energy codes are
able to handle these, which means that spherical basis functions can be used
only for (************???what sort of calculations???**********).

The keywords \verb+print+ and \verb+noprint+ allow the user to override
instructions from the top level \verb+PRINT+ directive for the basis sets.
The default is for the input module to print all basis sets encountered.
Specifying the keyword \verb+noprint+ allows the user to supress this output.

% This directive describes a basis set of generally contracted Cartesian
% Gaussian functions.  By default basis sets are automatically segmented
% and cartesian\footnote{This will change when the derivative integral
%  code is extended to handle general contractions and sphericals. Only
%  the energy codes can presently handle these.} even if general
% contractions are input.  Basis functions are associated with centers
% in geometries through the tags or names of centers which must match
% exactly (including case) and are limited to sixteen (16) characters.
% Each center with the same tag will have the same basis set.  By
% default the input module prints each basis set encountered; use the
% \verb+NOPRINT+ option to disable printing.
%
% In the same fashion as for geometries, basis sets are named, with the
% default name being \verb+"ao basis"+.  It should be clear from the
% above discussion on geometries and database entries how indirection is
% supported.
% 
% Basis functions may be either drawn from a standard set in the EMSL
% basis set library or specified explicitly.  See Appendix
% \ref{knownbasis} for the current list of standard basis sets. An
% installation default is provided for the path to the basis set
% library, which may be overridden in the input.  The case of names of
% standard basis sets is ignored.  By default the standard basis
% appropriate to the atom on which it is centered is used, however this
% may be overriden, for instance to place ghost functions on dummy
% centers in counterpoise calculations.  When explicitly specifying
% generally contracted basis sets, simply specify on each line the
% exponent followed by the contraction coefficients for each contraction
% of that exponent.  The following examples will make the use of this
% directive clear.

The input specified in lines beginning with the string \verb+tag+
allows a particular center or atom in a calculation to be associated with
a particular set of basis functions.  The values specified for \verb+tag+
must correspond exactly with the names supplied for the \verb+tag+ entries
on the \verb+GEOMETRY+ directive for a particular calculation.  Each atom
or center with the same \verb+tag+ will have the same basis set, which must
also be specified with the same name \verb+tag+.

The keyword \verb+library+ specifies that the calculation will use the 
standard basis sets in NWChem.  The string \verb+standard_set+ is the
name that identifies the functions in the library.  (See Appendix 
\ref{sec:knownbasis} for a complete list of the basis functions in the
NWChem library.)  If the string \verb+tag+ corresponds to
a dummy center, the string \verb+tag_in_lib+ must also be entered on
this line, to identify
the correct name for the atom in the basis function library.

A default path to the basis set library
is provided on installation of the code, and the user can specify a 
particular standard set by supplying the name in the library
in the string \verb+tag_in_lib+.  However, a different path can be defined
for the particular basis set by specifying the keyword \verb+file+ after
the keyword \verb+library+, and then explicitly naming the file to be
accessed for the basis functions.

For example, NWChem contains the Thom Dunning cc-pvdz basis sets for
oxygen and hydrogen in the standard basis set library.  The \verb+BASIS+
directive to use these basis sets in a calculation of the water molecule
could be formulated quite simply as follows;

\begin{verbatim}
  basis
    oxygen library cc-pvdz
    hydrogen library cc-pvdz
  end
\end{verbatim}

\verb+******************************+
QUESTION: In Appendix A, this basis set is listed as ``cc-pVDZ''.  Is
this input case-sensitive, or not?
\verb+******************************+

In this example, the cc-pvdz basis functions will be obtained from the
NWChem library, as defined by the default path on installation.  To obtain
functions from some other library, such as the user's own experimental
library of functions, the path name for the library must be specified
explicitly.  An example illustrating this option is as follows,

\begin{verbatim}
  basis
    o  library 3-21g file /usr/d3g681/nwchem/library
    si library 3-21g file /usr/d3g681/nwchem/library
  end
\end{verbatim}

This directive tells the code to use the basis sets \verb+3-21g+ in the
file /usr/d3g681/nwchem/library for atoms  or centers with \verb+tag+ 
strings \verb+o+ and \verb+si+ , rather than looking for them in the
default library.

The basis functions must be defined for dummy centers as well for
as atoms.  Since the dummy centers are given \verb+tag+ names that begin
with \verb+bq+ to distinguish them from atoms (see the description of the
\verb+GEOMETRY+ directive in Section \ref{sec:geometry}), the \verb+BASIS+
directive must identify their corresponding functions explicitly.  For
example, to specify the cc-pvdz basis for a calculation on the water 
monomer in the dimer basis, where the dummy oxygen and dummy hydrogen
centers have been identified as \verb+bqo+ and \verb+bqh+ respectively,
the \verb+BASIS+ directive is as follows,

\begin{verbatim}
  basis
    o   library cc-pvdz
    h   library cc-pvdz
    bqo library o cc-pvdz
    bqh library h cc-pvdz
  end
\end{verbatim}


The user also has the option of explicitly specifying the generally
contracted basis functions for particular elements.  This can be useful
as a means of specifying alternative basis functions for 'ghost' atoms
or dummy centers in counterpoise calculations, for instance.

For those particular elements named in some string \verb+tag+ (and identified
appropriately on a \verb+GEOMETRY+ directive), the string \verb+tag+ input 
line on the \verb+BASIS+ directive is of the form,

     <string tag> <string shell_type>
        <real exponent> <real list_of_coefficients>

The string \verb+shell_type+ is used to specify the outer electron shell of
the atom, which may be \verb+s+ for the s-shell, \verb+p+ for the p-shell,
(***or whatever***).  The user must then specify the real exponent for
each electron(?), followed by the contraction coefficients for each 
contraction of that exponent.  The input for the \verb+list_of_coefficients+
is free-format, and assumed to be type 'real'.

For example, the \verb+BASIS+ directive to use Thom Dunning's cc-pvdz basis 
functions for the water molecule, as illustrated above, 
can be modified to add a specific basis function that defines a diffuse 
s-shell (specified by string {\em s} for \verb+shell_type+) for oxygen.
The modified directive is of the form,

\begin{verbatim}
  basis
    oxygen library cc-pvdz
    hydrogen library cc-pvdz
    oxygen s
      0.01 1.0
  end
\end{verbatim}

In all of these examples, the basis set is given the default name
\verb+"ao basis"+.  The functions are assumed segmented and cartesian,
and the input module will print a description of the basis set.  The functions
of this basis set consist of the library functions for hydrogen, plus
the library functions for oxygen as modifed based on the specified contraction
coefficients.

The basis functions for a molecule can be explicitly specified for all
atoms, with none of the functions drawn from the standard library.  In such
a case, the \verb+shell_type+, \verb+exponent+, and \verb+list_of_coefficients+
must be supplied for each atom or center in the molecule.  (Note that 
the name of
each atom or center must also correspond to an appropriate entry for 
a \verb+tag+ string
identifying each atom or center of the molecule on the corresponding
\verb+GEOMETRY+ directive.)

% The following directive specifies exactly the same basis set as the
% previous directive except that all basis functions are explicitly
% described (all input is free format --- the formatting here is just
% for readability).

The following example illustrates this option for the water molecule.  The
user must enter explicitly all the information that could be obtained
simply by specifying the NWChem library functions using the lines
 
\begin{verbatim}
    oxygen library cc-pvdz
    hydrogen library cc-pvdz
\end{verbatim}

The input for the \verb+BASIS+ directive in this example when the option
for specifying explicit functions is selected becomes,

\begin{verbatim}
  basis
    oxygen s
      11720.0000    0.000710  -0.000160
       1759.0000    0.005470  -0.001263
        400.8000    0.027837  -0.006267
        113.7000    0.104800  -0.025716
         37.0300    0.283062  -0.070924
         13.2700    0.448719  -0.165411
          5.0250    0.270952  -0.116955
          1.0130    0.015458   0.557368
          0.3023   -0.002585   0.572759
    oxygen s                
          0.3023    1.000000
    oxygen p                
         17.7000    0.043018
          3.8540    0.228913
          1.0460    0.508728
          0.2753    0.460531
    oxygen p                
          0.2753    1.000000
    oxygen d
          1.1850    1.000000
    hydrogen s
         13.0100    0.019685
          1.9620    0.137977
          0.4446    0.478148
          0.1220    0.501240
    hydrogen s  
          0.1220    1.000000
    hydrogen p  
          0.7270    1.000000
    oxygen s
          0.01      1.0
  end
\end{verbatim}

%  Note that the correlation-consistent basis sets were designed using
% spherical harmonics and to use these the \verb+SPHERICAL+ keyword must
% be present on the \verb+BASIS+ directive.
%    *******NOTE: the above paragraph is not consistent with what has
%                 already been said about the keywords spherical and
%                 cartesian.  And it's not consistent with the directive.
%                 What is right here?
%    *******************************************************************
%    
% This example uses a 3-21g basis set for centers \verb+o+ and \verb+si+
% and explicitly specifies the path to the basis set library (perhaps
% because you cannot find the copy installed with NWChem).
% \begin{verbatim}
%   basis
%    o  library 3-21g file /usr/d3g681/nwchem/library
%    si library 3-21g file /usr/d3g681/nwchem/library
%   end
% \end{verbatim}
%
% In order to perform counterpoise corrections in standard basis sets it
% is necessary to specify the atom type.  The following input specifies a
% cc-pvdz basis for a calculation on the water monomer in the dimer basis.
% \begin{verbatim}
%   basis
%     o   library cc-pvdz
%     h   library cc-pvdz
%     bqo library o cc-pvdz
%     bqh library h cc-pvdz
%    end
% \end{verbatim}



