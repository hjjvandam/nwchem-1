\label{sec:functionality}

NWChem provides many methods to compute properties of molecular and
periodic systems using standard quantum mechanical descriptions of the
electronic wavefunction or density.  In addition, NWChem has the
capability to perform classical molecular dynamics and free energy
simulations.  These approaches may be combined to perform mixed
quantum-mechanics and molecular-mechanics simulations. 

\section{Molecular electronic structure}

The following quantum mechanical methods are available to calculate
energies, and analytic first derivatives with respect to atomic
coordinates.  Second derivatives are computed by finite difference of
the first derivatives.

\begin{itemize}
\item Self Consistent Field (SCF) or Hartree Fock (RHF, UHF, high-spin
  ROHF).  
\item Gaussian Density Functional Theory (DFT), using many local and
  non-local exchange-correlation potentials (RHF or UHF).
\item Spin-orbit DFT (SODFT), using many local and non-local
  exchange-correlation potentials (UHF).
\item MP2 semi-direct using frozen core and RHF and UHF reference.
\item Complete active space SCF (CASSCF).

\end{itemize}

The following methods are available to compute energies only.  First
and second derivatives are computed by finite difference of the
energies.
\begin{itemize}
\item CCSD, CCSD(T), CCSD+T(CCSD), with RHF reference.
\item Selected-CI with second-order perturbation correction.
\item MP2 fully-direct with RHF reference.
\item Resolution of the identity integral approximation MP2 (RI-MP2), with
  RHF or UHF reference.
\end{itemize}

For all methods, the following operations may be performed:
\begin{itemize}
\item Single point energy
\item Geometry optimization (minimization and transition state)
\item Molecular dynamics on the fully {\em ab initio} potential energy
  surface
\item Numerical first and second derivatives automatically computed if
  analytic derivatives are not available
\item Normal mode vibrational analysis.
\item ONIOM hybrid method of Morokuma and co-workers.
\item Generation of the electron density file for the {\em Insight}
      graphical program
%Need to elaborate on or reference the "Insight" program?--fmr
\item Evaluation of static, one-electron properties.
\end{itemize}

In addition, automatic interfaces are provided to
\begin{itemize}
\item The natural bond orbital (NBO) package
\end{itemize}

\section{Relativistic effects}

The following methods for including relativity in quantum chemistry 
calculations are available:
\begin{itemize}
\item The spin-free one-electron Douglas-Kroll approximation is available for all 
 quantum mechanical methods and their gradients.
\item Dyall's spin-free Modified Dirac Hamiltonian approximation is available 
 for the Hartree-Fock method and its gradients.
\item One-electron spin-orbit effects can be included via spin-orbit potentials.
 This option is available for DFT and its gradients, but has to be run without 
 symmetry.
\end{itemize}

\section{Periodic system electronic structure}

A module is available to compute energies by Gaussian Density
Functional Theory (DFT) with many local and non-local
exchange-correlation potentials.

\section{Molecular dynamics}

The following functionality is available for classical molecular
simulations:
\begin{itemize}
\item Single configuration energy evaluation
\item Energy minimization
\item Molecular dynamics simulation
\item Free energy simulation 
\end{itemize}

NWChem also has the capability to combine classical and quantum
descriptions in order to perform:
\begin{itemize}
\item Mixed quantum-mechanics and molecular-mechanics (QM/MM)
  minimizations, and
\item Molecular dynamics using any of the quantum
  mechanical wavefunctions.
\end{itemize}

\section{Python}

The Python programming language has been embedded within NWChem and
many of the high level capabilities of NWChem can be easily combined
and controlled by the user to perform complex operations.
