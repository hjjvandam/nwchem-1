\label{SCF} 

The NWChem self-consistent field (SCF) module computes closed-shell
restricted Hartree-Fock (RHF), restricted high-spin open-shell
Hartree-Fock (ROHF) and spin-unrestricted Hatree-FOck (UHF)
wavefunctions.

\begin{verbatim}
  SCF
    ...
  END
\end{verbatim}

The keyword \verb+scf+ tells the code that this is a compound directive,
and additional directives may be specified by the user to define the particular
problem.  The \verb+scf+ input will be processed until the
\verb+END+ directive is encountered.  The actual SCF calculation will
be performed when the input encounters a \verb+TASK+ directive of the form,

\begin{verbatim}
  TASK SCF
\end{verbatim}

The default task in the SCF module is a spin-restricted, 
closed shell SCF calculation.  This task is identified by keyword RHF if specified 
explicitly in the input.  (Refer to the \verb+TASK+ directive description in
Section \ref{sec:task} for a complete list of operations that can be
specified in the SCF module.)  The options that are currently working
in the code are closed shell RHF, ROHF, and UHF. The following 
subsections describe the keywords and
optional subdirectives that can be specified for a \verb+SCF+ calculation
in NWChem.

% This compound directive controls the SCF program.  Closed shell RHF,
% ROHF, and UHF are currently working.  The various optional
% sub-directives are described below.

\subsection{Type of SCF wavefunction and specification of multiplicity
and open shells}

Unless some other type of calculation is specified explicitly in
the directive input, NWChem assumes that the SCF module is to be used
to perform a spin-restricted,
closed-shell SCF calculation (keyword RHF).  In such a case, an error results
if the number of electrons is inconsistent with this assumption.
The number of electrons is inferred from the
total charge on the system and the nuclear charges of the
specific atoms (which are by default their atomic numbers
adjusted for the presence of ECPs, unless specified
explicitly by input on the \verb+GEOMETRY+ directive -- see Section 
\ref{sec:geom}).  The total charge on the system is zero by default, 
unless specified at
some value by input on the \verb+CHARGE+ directive -- see Section 
\ref{sec:toplevel}).

The options available to define the SCF wavefunction and multiplicity are
as follows;


\begin{verbatim}
  SINGLET 
  DOUBLET 
  TRIPLET 
  QUARTET 
  QUINTET 
  NOPEN <integer nopen default 0>
  RHF
  ROHF
  UHF
\end{verbatim}

The optional keywords \verb+SINGLET, DOUBLET, TRIPLET, QUARTET, QUINTET+ 
and \verb+NOPEN+ allow the user to specify the number of open shells for a 
particular calculation.  \verb+SINGLET+ is the default, and specifies a 
closed shell; \verb+DOUBLET+ specifies one open shell; \verb+TRIPLET+
specifies two open shells; and so forth.  If there are more than four
open shells, the keword \verb+NOPEN+ must be used, with the integer input
for \verb+nopen+ defining the exact number of open shells.

If the multiplicity is any value other than \verb+SINGLET+, the default
calculation will be a spin-restricted,
high-spin, open-shell SCF calculation (keyword ROHF).  The open-shell
orbitals must be the highest occupied orbitals.  If necessary, the vectors
may be rearranged through the use of the \verb+SWAP+ keyword on the
\verb+VECTORS+ directive (see Section \ref{sec:vectors}) to accomplish this. 

% If there are
% more than four open shells then \verb+NOPEN+ may be used to specify
% directly the number of open shells.  If in addition it is desired to
% run a spin-unrestricted calculation (UHF) then the keyword \verb+UHF+
% must be present.  

The keywords \verb+RHF+ and \verb+ROHF+ are provided in
the code for completeness, but they do not necessarily need to be 
specified explicitly in the input.  One or the other type of calculation
will be performed, depending on the optional
input specified for the multiplicity, as described above.

An additional calculation of
a spin-unrestricted solution can also be performed as part of the task
by specifying the keyword \verb+UHF+.  In UHF calculations, it is
assumed that the number of open shells corresponds to the difference
between the number of alpha-spin and beta-spin orbitals.  For example, 
a UHF calculation
with 2 more alpha-spin orbitals than beta-spin orbitals can be obtained
by specifying

\begin{verbatim}
scf
     triplet uhf

     ...

end
\end{verbatim}


% The option \verb+RHF+ is the default
% and can be obtained by supplying no other input for the wavefunction.
% The default \verb+ROHF+ is obtained by supplying a
% multiplicity keyword for the problem, and nothing else.  



% Examples:
% \begin{itemize}
% \item No input runs a spin-restricted closed-shell singlet.
%  \item \verb+doublet+ runs a ROHF calculation with one open shell.
% \item \verb+triplet; uhf+ runs a UHF calculation with 2 more
%   alpha-spin orbitals than beta-spin.
% \end{itemize}

The user should be aware that by default molecular orbitals are symmetry 
adapted in NWChem.  This may not be desirable for fully unrestricted 
wavefunctions.  In such cases, the user has the option of defeating the
defaults by specifying the keywords \verb+ADAPT OFF+ (see Section 
\ref{sec:adapt}) and \verb+SYM OFF+ (see Section \ref{sec:sym}).

\subsection{Symmetry Specification for the SCF Module}
\label{sec:sym}

The default in NWChem is to assume that the molecular orbitals are 
symmetry adapted, so the default keyword for symmetry is \verb+SYM ON+.
This option enables speedup of the Fock matrix construction via the
petite or skeleton algorithm.  The keyword \verb+SYM OFF+ can be used to
%
% \begin{verbatim}
%   SYM (ON|OFF)
% \end{verbatim}
%This directive enables/disables use of symmetry to speedup Fock matrix
% construction (via the petite or skeleton algorithm) in the SCF even if
% symmetry was used in the specification of the geometry.  Symmetry
% adaption of the molecular orbitals is not affected by this option.
disable the use of symmetry in the Fock matrix construction.  This
option can be used in the calculation in the SCF module even when
symmetry was used in the specification of the geometry.

The directive to specify the symmetry option is as follows,

 \begin{verbatim}
   SYM (ON|OFF) default SYM ON
 \end{verbatim}


\subsection{Symmetry Adaptation for the SCF Module}
\label{sec:adapt}

The default in the SCF module calculation is to assume symmetry adaption
of the molecular orbitals.  The complementary keywords \verb+ADAPT ON+
and \verb+ADAPT OFF+ allow the user to enable or disable this assumption,
depending on the particular application.  The default assumption of
adaptive symmetry does not affect the speed of the calculation, but the
resulting orbitals may be symmetry contaminated for some problems.  This
is especially likely if the calculation is started using orbitals
from a distorted geometry.

The underlying assumption in the use of symmetry in Fock matrix contruction
is that the density is totally symmetric.  If the orbitals are symmetry 
contaminated, this assumption may not be valid.  This could result in
incorrect energies and poor convergence of the calculation.  Therefore,
the option \verb+ADAPT OFF+ is recommended for any calculation for which
the user has elected to specify \verb+SYM OFF+.


The directive to specify the adaptive option explicitly within the
\verb+SCF+ directive is as follows,

\begin{verbatim}
 ADAPT (ON|OFF) default ADAPT ON
\end{verbatim}

% This directive enables/disables symmetry adaption of the molecular
% orbitals.  This does not impact the speed of the calculation but the
% resulting orbitals may be symmetry contaminated, especially if
% starting a calculation from orbitals from a distorted geometry.
% Underlying the use of symmetry in Fock-matrix construction is the
% assumption that the density is totally symmetric --- if the orbitals
% are symmetry contaminated this assumption may be invalid which will
% potentially result in incorrect energies and poor convergence.  It is
% thus advisable when specifying \verb+ADAPT OFF+ to also specify
% \verb+SYM OFF+.

\subsection{Integral Screening Threshold for the SCF Module}

The keyword \verb+TOL2E+ can be specified as part of the \verb+SCF+ 
directive for a particular task to define a sort of convergence
criterion for the calculation.  
% \begin{verbatim}
%   TOL2E <real tol2e default 1.0e-7>
% \end{verbatim}
The variable \verb+tol2e+ is used in determining the integral screening 
threshold for the
evaluation of the energy and related Fock-like matrices.  The Schwarz
inequality is used to screen upon the product of integrals and density
matrices in a manner that results in approximately an accuracy of 
the value specified for
\verb+tol2e+ in the energy and Fock-matrices.  This differs from many
codes where the error in the energy is typically much greater than
this threshold.  The default threshold of $10^{-7}$ is suitable for most 
purposes.
A less stringent value of $10^{-5}$ can speed up some low-accuracy 
exploratory calculations.  Conversely, a tighter threshold of 
$10^{-9}$ might be
necessary to reliably explore very weak
interactions, such as micro-Hartree or below.

The input to specify the threshold explicitly within the
\verb+SCF+ directive is as follows,

\begin{verbatim}
  TOL2E <real tol2e default 1.0e-7>
\end{verbatim}


\subsection{Orbital Lagrangian for the SCF Module}

% \begin{verbatim}
%   LAGRANGIAN
% \end{verbatim}


The keyword \verb+LAGRANGIAN+ can be included in the input for a
\verb+SCF+ directive to ensure that the calculation will generate
the orbital Lagrangian and write the data to disk. The
lagrangian is required to compute ROHF gradients.  Under most
circumstances this option is automatically enabled and need not be set
by the user.

The input to specify this option explicitly is symply
\begin{verbatim}
  LAGRANGIAN
\end{verbatim}

There are no arguments with this keyword.

\subsection{Granularity Definition for the SCF Module}

% \begin{verbatim}
%   CHUNK <integer chunk default -1>
% \end{verbatim}

The keyword \verb+CHUNK+ and its associated input allows the user to
define the  `chunk-size' or granularity of the
parallel work decomposition for a particular calculation.  The input for this
option is
\begin{verbatim}
  CHUNK <integer chunk default -1>
\end{verbatim}

The default of \verb+-1+ means that the chunk size will be defined
automatically within the code.  
\Large 
(****What does this default mean?***)
\normalsize  
It might be necessary to use a smaller
or larger value to obtain more efficient execution on a large
parallel computer, or a computer with very high communication costs.
Users unfamiliar with the internal function of the program should seek
advice before changing this parameter.

\subsection{Molecular Orbital Vectors for the SCF Module}
\label{sec:vectors}

The \verb+VECTORS+ directive allows the user to specify the source and 
destination of the molecular orbital vectors.  In a startup calculation 
(see Section \ref{sec:start}),the default
source for guess vectors is to diagonalize a Fock matrix constructed
from a superposition of the atomic density matrices for the particular
problem.  This is usually
a very good guess.   For a restarted or continued calculation, the default
is to use whatever is in the current database.

When a particular calculation requires some other starting point than 
the default initial vectors, the \verb+VECTORS+ directive can be specified
within the compound 
\verb+SCF+ directive.  The form of the \verb+VECTORS+ directive is
as follows;  
% Alternative sources for orbitals include the result of
% diagonalization of the bare-nucleus or one-electron Hamiltonian
% (\verb+HCORE+), projection from molecular orbitals in a smaller basis
% (\verb+PROJECT+), or from a named file that contains the results of a
% previous calculation.


\begin{verbatim}
  VECTORS [input (<string input_movecs default atomic>) || \
                   (project <string basisname> <string filename>)] \
          [swap [alpha|beta] <integer vec1 vec2> ...] \
          [output <string output_movecs default input_movecs>] \
          [lock]
\end{verbatim}

The keyword \verb+INPUT+ allows the user to specify the source of the 
initial molecular orbital vectors as the file \verb+input_movecs.movecs+.  
The default name for this string is \verb+ATOMIC.movecs+, for the diagonalized
Fock matrix.  Another option in the code is to specify \verb+HCORE+ for
the input \verb+input_movecs+.  This will produce initial vectors from
diagonalization of the bare-nucleus or one-electron Hamiltonian.  If 
neither of these options is suitable for the given problem, the input
for \verb+input_movecs+ can be the name of a file that contains the
results of a previous calculation.

The keyword \verb+PROJECT+ allows the user to specify the starting point
for the vectors using a projection of the molecular orbitals in a smaller
basis set.  When this keyword is invoked, the user must supply the name
of the basis set in the string \verb+basisname+, and the name of the file 
containing the vectors in the string \verb+filename+.  The definition of 
the basis \verb+basisname+ 
must be available in the current database, and the basis must be
smaller than the current one.  In addition, the geometry used for the
previous calculations must have the atoms in the same order and in the
same orientation as the current geometry.
 
By default, the output orbitals are written over the initial vectors that
were used to start the calculation.  When the initial guesses specified
by means of the keywords \verb+ATOMIC+, \verb+HCORE+, or \verb+PROJECT+ 
are used, the default destination for
vectors is a file named

\verb+"<file_prefix>.movecs"+

For a calculation 
that completes normally (i.e., is not interrupted or does not terminate
with an error), the database is automatically modified in that the output
vectors are written over the starting vectors for the calculation.  Then
if the job is restarted,
 the default source of vectors will be the results 
of the previous calculation.  Specifying the keyword \verb+OUTPUT+ allows
the user the option of sending the results of the calculation to a different
destination, thereby preserving the initial vectors for other calculations.

Applications of this directive are illustrated in the following examples,
using the different options available through the keyword and input variables
specifiction.

\begin{verbatim}
Example 1:

vectors output h2o.movecs
\end{verbatim}

This directive will result in the initial vectors being read from the 
default location,

\verb+atomic.movecs+

The final vectors 
obtained in the calculation
will be saved in 

\verb+h2o.movecs+

The contents of \verb+atomic.movecs+
will not be changed.  Note that SCF orbitals are {\em always}
canonicalized on output.



\begin{verbatim}
Example 2:

vectors input initial.movecs output final.movecs
\end{verbatim}

This directive will result in the initial vectors being read from the file
\verb+initial.movecs+.  The results will be written to the file
\verb+final.movecs+.  The contents of \verb+initial.movecs+ will not be
changed.

\begin{verbatim}
Example 3:

vectors input project "small basis" small.movecs
\end{verbatim}

This directive will cause the calculation to start from vectors in the
file \verb+"small.movecs"+ which are in a basis named \verb+"small basis"+.
The final vectors will be written over the initial values in the file
\verb+"small.movecs"+ at the completion of the calculation.
 

Once starting vectors have been obtained using any of the possible
options, they may be reordered through use of the \verb+SWAP+ keyword.
When this optional keyword is specified, it requires an
argument list consisting of integer numbers corresponding to the pairs 
that will be swapped.
For UHF calculations, separate \verb+SWAP+ keywords must be provided
for the alpha and beta orbitals, as necessary.

%
%  The \verb+LOCK+ directive locks the ordering of orbitals to that of the
% initial vectors, as far as possible. The default is to order by
% ascending orbital energies within each orbital space. The \verb+LOCK+
% directive is useful, for example, to force the SCF to preserve the
% ordering of a previous geometry despite flipping of orbital energies.
%
% To re-cap, the default options suffice to provide a good starting
% guess for most instances as well as to route vectors to a file
% (\verb+"movecs"+) from which they are automatically used for restart.
% Note that the SCF orbitals are {\em always} canonicalized on output.
% Examples follow.
% 
% To use the default sources for vectors and route result vectors
%  to a different file
% \begin{verbatim}
%   vectors output h2o.movecs
% \end{verbatim}
% 
% To load the vectors from one file and route result vectors to a
% second file
% \begin{verbatim}
%   vectors input initial.movecs output final.movecs
% \end{verbatim}
% 
%  \sloppy
%
% To restart from vectors in the file \verb+"small.movecs"+ which are in
% a basis named \verb+"small basis"+ (NB: the definition of this basis
%  must be available in the current database and the basis must be
% smaller than the current one.  Also, the geometry used for the
% previous calculations must have the atoms in the same order and in the
% same orientation as the current geometry).  The keyword \verb+INPUT+
% is not necessary but is provided for clarity.  Output vectors are
% routed to the file \verb+"big.movecs"+.
% \begin{verbatim}
%   vectors input project "small basis" small.movecs
% \end{verbatim}
% 
% \fussy

An example illustrating the use of the \verb+swap+ keyword is as follows;
\begin{verbatim}
  vectors input try1.movecs swap 173 175 174 176 output try2.movecs
\end{verbatim}

This directive will cause the initial orbitals to be read 
from the file \verb+"try1.movecs"+.  The vectors for the orbitals
within the pairs 173--175 will be swapped with those within 174--176.
The final orbitals obtained in the calculation will be written to
the file \verb+"try2.movecs"+.

An example illustrating this feature for a UHF calculation is the directive

\begin{verbatim}
  vectors swap beta 4 5 swap alpha 5 6
\end{verbatim}

This input will result in the swapping of the 5--6 alpha orbital pair and 
the 4--5 beta orbital pair.  (All other items in the input use the
default values.)

The \verb+LOCK+ directive allows the user to specify that the ordering of 
orbitals will be locked to that of the
initial vectors, as far as possible. The default is to order by
ascending orbital energies within each orbital space. One application
where this might be desirable is a calculation where it is necessary 
to preserve the ordering of a previous geometry despite flipping of 
orbital energies.  For such a case, the \verb+LOCK+ directive can be used
to prevent the SCF calculation from changing the ordering, even if the orbital
energies change.



\subsubsection{Atomic guess orbitals with charged atoms}

An additional option available to the user when specifying the initial
molecular orbital vectors allows modifications to be made to the charge
of the atomic guess.  This input is not part of the \verb+VECTORS+ directive,
however.  It must be specified separately, using the \verb+SET+
directive (see Section \ref{sec:set}).  This option allows the user to 
modify the charges on specific atoms or
centers by using the \verb+SET+ directive to define the contents
of the three arrays \verb+atomscf:num_z+, \verb+atomscf:z+, and
\verb+atomscf:tags_z+.  The form of the \verb+SET+ directives 
for this option is as follows;

% It is possible to alter the charge of the atomic guess with \verb+SET+
% directives (c.g., Section \ref{sec:set}.  The input is as follows:

\begin{verbatim}
set atomscf:num_z integer <integer num_z >
set atomscf:z     real    <real list_of_charges >
set atomscf:tags_z ``t(1)'', ``t(2)'', ... ``t(num_z)''
\end{verbatim}

The first \verb+SET+ directive fills \verb+atomscf:num_z+ with an
integer value \verb+num_z+, which is the number of atoms or centers 
with modified charge.
The second \verb+SET+ directive fills \verb+atomscf:z+ with \verb+num_z+
values of charge to be applied to these centers.  The third 
\verb+SET+ directive fills \verb+atomscf:tags_z+ with \verb+num_z+ values
indentifying the
\verb+tag+ names (as entered on the \verb+GEOMETRY+ directive for the
calculation) of the atoms or centers with modified charge.

For example, to specify an oxygen ion O+2 with a charge of 6.0, the input
for these \verb+SET+ directives would be as follows;

\begin{verbatim}
set atomscf:num_z integer 1
set atomscf:z     real    6.0
set atomscf:tags_z   O
\end{verbatim}

This input will assign a charge of 6.0 to the oxygen atoms or centers
in the problem.

\Large
*****************
Question: is this a good example?
Can we make up a better one?
Does this modified charge apply to all occurances of O in the molecule?
*****************
\normalsize



% Where \verb+num_z+ is the number of centers with modified charge,
% \verb+z+ is a list of the actual charges to be applied to the
% ``\verb+num_z+'' centers (e.g., for O+2 user a charge of 6.0), and
% \verb+tags_z+ is the list of element tags to be modified (\verb+num_z+
% of them).  

There are some limitations to this feature.  It is not possible to add
electrons to closed shell atoms, nor is it possible to remove all electrons
from a given atom.  Attempts to do so will cause the code to report an error,
and it will not report further errors in the input for modifying the charge 
even when they are detected.

\subsection{Convergence Threshold for the SCF Module}
\label{sec:thresh}

% \begin{verbatim}
%   THRESH  <real thresh default 1.0e-5>
% \end{verbatim}

This directive allows the user to specify the convergence threshold for the
calculation.  The convergence threshold is the norm of the orbital gradient,
and has a default value in the code of $10^{-5}$.  The form of the directive
is a follows;

\begin{verbatim}
  THRESH  <real thresh default 1.0e-5>
\end{verbatim}

The norm of the orbital gradient corresponds
roughly to the precision available in the wavefunction, and the energy
should be converged to approximately the square of this number.  It should
be noted, however, that
the precision in the energy is also limited by such factors as 
the integral selection thresholds.  The value specified for \verb+thresh+
does not provide an absolute measure of convergence for a calculation. 

%   This corresponds to
% roughly the precision available in the wavefunction.  The energy
%  should be converged to roughly the square of this number.  Of course,
%  the precision in the energy is also limited by the integral selection
% thresholds, etc. 

\subsection{Maximum Number of Iterations for the SCF Calculation}
\label{sec:max}

% \begin{verbatim}
%    MAXITER <integer maxiter default 8>
% \end{verbatim}

The maximum number of iterations for the SCF calculation is set to eight, by
default.  For most molecules, this number of iterations is sufficient for
the quadratically convergent SCF algorithm to obtain a
solution that converges to the default threshold (see Section  
\ref{sec:thresh} above).  However, the user has the option of specifying
a different limit, using the directive

\begin{verbatim}
   MAXITER <integer maxiter default 8>
\end{verbatim}

\Large
**************
What sort of problem might need this option?  What are some
good values for maxiter, if 8 isn't enough?
How can a user tell that more iterations might help a given problem?
**************
\normalsize

\subsection{Resolution of the Identity Integral Approximation for the SCF Module}

The \verb+RI+ directive allows the user to specify an SCF calculation
using the RI approximation for the Fock matrix.  The RI-SCF method is
recommended for small molecules with expensive basis sets.
\Large
***** Why? ****
\normalsize
When this directive is specified, the input must also include an expansion
basis with the basis name ''riscf basis''.
\Large
**Will the code give an error message if this basis is missing?*******
\normalsize

The form of the directive is as follows;

\begin{verbatim}
   RI <hessian || preconverge || full> <disk || memory || auto>
\end{verbatim}

% This directive requests the RI approximation being used for the Fock matrix.
%  The RI-SCF method is especially suggested for small molecules with
% expensive basis sets.

The extent of the approximation is specified with one of the three
alternative keywords \verb+FULL+, \verb+HESSIAN+, or \verb+PRECONVERGED+.
The default is \verb+FULL+, which specifies that the RI approximation will
be used for all Fock builds in the calculation.  This is the fastest of
the three options, but will result in only an approximate value for the
energy.  The keyword \verb+HESSIAN+ specifies that only the hessian will
be calculated with an approximated Fock matrix.  The keyword 
\verb+PRECONVERGED+ specifies a two-part calculation.  First, the SCF
calculation will be performed with the RI approximation for the Fock
matrix.  Then an ``exact'' direct SCF calculation using the hessian obtained
with the RI approximated Fock matrix will be performed.

\Large
*********Any comments/advice on what the trade-offs are between these
three options?  Suggestions on when one or the other might be the preferred
approach?**************
\normalsize

% If called with the \verb+full+ keyword (default)  it is used for all
% Fock builds, which is 
% fastest but results in an approximate energy.. The \verb+Hessian+ keyword
% means only the hessian is calculated with an approximated Fock matrix, and
% \verb+preconverge+ requests an ''exact'' direct SCF (with RI hessian, which
% does not change the result) after convergence of the
% RI approximated calculation.

If the ''Disk Resident Array'' library is implemented in the user's
installation of NWChem, the keyword \verb+DISK+ can be used 
to specify that the 3-center integrals will be stored on disk.  Alternatively,
the keyword \verb+MEMORY+ specifies in-core storage of the integrals.
The default for this option is the keyword \verb+AUTO+, which allows the
code to make its own decision on where to store the integrals.  If there
is enough memory available, it will use the in-core option; otherwise,
the disk based algorithm will be used, if available.

\Large
********what does it do if the disk is not available, and there's not
enough in-core memory?  Fail? Print an error message?*********
\normalsize

% \verb+Disk+ indicates storage of 3-center integrals on disk. This only
% works where the ''Disk Resident Array'' library is implemented. \verb+Memory+
% requests in-core storage of the integrals. This
% significantly reduces the maximum problem size that can be handled.
% Default is \verb+auto+, which lets the program determine if there is
% enough memory for the in-core version, otherwise the disk based
% algorithm is used (if available).

% An expansion basis named ''riscf basis'' has to be present in the input deck.

\subsection{Performance Profile for the SCF Module}

This directive allows the user to obtain a detailed analysis of the 
performance of the SCF module.  It is specified by the simple keyword

\begin{verbatim}
  PROFILE
\end{verbatim}

This option provides information that can be very helpful
in evaluating the results of an SCF calculation.  However, on machines
that have expensive timing routines, such as the SUN, it can introduce
a significant overhead.

\Large
****Will most users already know what \verb+PROFILE+ will get them in the
way of useful information?**********
\normalsize

% If present, this directive enables a detailed analysis of the
% performance of the SCF code.  This can introduce significant
% overhead on machines which have expensive timing routines (e.g., SUN).

\subsection{Optional DIIS Convergence for the SCF Module}

This directive allows the user to specify DIIS convergence rather than
second-order convergence for the SCF calculation.  The form of the
directive is as follows;

\begin{verbatim}
  DIIS
\end{verbatim}

When this option is specified in the input, the \verb+MAXITER+ directive
(see Section \ref{sec:max}) must also be specified, with the macro-iteration
count \verb+maxiter+ set to a value around 20.
The implementation of this option is currently fairly rudimentary.  It
does not have level-shifting and damping, and does not support open-shells.
It is provided on an ``as is'' basis, and should be used with caution.

% This directive toggles on the DIIS convergence as opposed to the
% second-order convergence. This is a fairly rudimentary implementation
% without level-shifting and damping and, currently, does not support
% open-shells. This is provided on an ``as is'' basis. Use the
% \verb+MAXITER+ directive to increase the macroiteration count to
% approximately 20.

When the \verb+DIIS+ directive is specified in the input, the user has
the additional option of specifying the size of the subspace for the DIIS
extrapolation.  This is accomplished with the \verb+DIISBAS+ directive,
which is of the form,

\begin{verbatim}
  DIISBAS <integer diisbas default 5>
\end{verbatim}

% controls the size of the subspace for DIIS extrapolation.  The default
% should be adequate in most instances but it may be reduced to conserve
% memory for large systems or increased if convergence is poor.

The default of 5 should be adequate for most applications, but may be
increased if convergence is poor.  On large systems, it may be necessary
to specify a lower value for \verb+diisbas+, to conserve memory.

\subsection{Control of Memory Allocation and Utilization in the SCF Module}
\label{sec:semidirect}

Where information is stored for the SCF calculation can significantly 
affect code performance, and depends on the installation configuration,
the platform, and the type of problem being executed.  If left to its own
devices,
the default behaviour of the SCF module is
\begin{itemize}
% \item If all integrals can be stored in memory then integrals are
%   computed once and cached in memory. Otherwise,
\item if there is enough memory available, the integrals are computed
once and cached in memory.
%\item 95\% of the available disk space in the scratch directory
\item If there is not enough memory to store all of the integrals at once,
then 95\% of the available disk space in the scratch directory
  (see Section \ref{sec:dirs}) is assumed available for this purpose,
  and as many integrals as possible are cached on disk (with no
  memory being used for caching).  Some attempt is made to 
  store the most expensive integrals in the cache.  (Note that no allowance 
is made for processes sharing disks when computing available space.)
% \item If not all integrals can be cached in memory or on disk,
\item If there is not enough room in memory or on disk for all of the
integrals, then the ones that are not cached are
%  then the additional integrals are 
recomputed in a semi-direct fashion.
\end{itemize}

The integral file is deleted at the end of a calculation, so it is not
possible to restart a semi-direct calculation when the integrals are
cached in memory.  Many computer systems (e.g., the EMSL IBM SP) clear
the fast scratch space at the end of each job, adding a further complication
to the problem of restarting a {\em parallel} semi-direct calculation.
Under some situations, it is possible to restart from integrals on disk,
but this capability will not be made widely available until a later date.

% Since many computer systems (e.g., the EMSL IBM SP) clear the
% fast scratch space at the end of each job, and given the complexity
% of restarting {\em parallel} semi-direct calculations, the integral
% file is deleted at the end of the calculation.  Under some
% situations it is possible to restart from integrals on disk, but
%  this capability will not be made widely available until a later date.

On the IBM SP or any other computer with fast disks local to each
processor, semi-direct gives the best behaviour.  It can result in {\em
  quadratic speedup} as more processors are added.  On other machines,
it may be necessary to resort to additional strategies to achieve faster
performance, such as limiting the use of disk space, forcing the use of
more memory for caching, changing the default file names, or  using
fully-direct SCF.

The user has the option of forcing a fully-direct calculation (with
recomputation of the integrals, as required).  This is accomplished
by specifying the directive

\begin{verbatim}
  DIRECT
\end{verbatim}
% forces a fully-direct calculation with recomputation of the integrals
% as required.  More detailed control is provided with

Alternatively, the \verb+SEMIDIRECT+ directive can be used to
specify a semi-direct calculation and define
the amount of disk space and the cache memory size.  The form of this
directive is as follows;

\begin{verbatim}
   SEMIDIRECT [filesize <integer filesize = disksize>] 
              [memsize  <integer memsize = available>]
              [filename <string filename>]
\end{verbatim}
Entering the keyword \verb+FILESIZE+ allows the user to define
an integer value for \verb+filesize+ to specify the amount of disk space 
to be used per process
for storing integrals in 64-bit words.  Similarly, the keyword
\verb+MEMSIZE+ allows the user to specify \verb+memsize+,
the number of 64-bit words to be used per process for
caching integrals in memory. (Note: If the amount of storage space
specified by the entry for \verb+memsize+ is not available, 
the code cuts the value in half and checks again for available space.
If the value is still too large, it is halved again and checked against
available space.  This process is repeated until the request is satisfied.)
By default, the integral file is placed into the scratch directory
(see Section \ref{sec:dirs}). Specifying the keyword \verb+FILENAME+ 
overrides this default.  The user-specified name entered in the string
\verb+filename+ has the process number appended to it, so that
each process has a distinct file, but otherwise the same name is used by all
processes.  Therefore, it is not possible to use this keyword to specify
different disks for different processes.  The \verb+SCRATCH_DIR+ directive
(see Section \ref{sec:dirs}) can be used for this purpose.  

%Some examples follow.

For example, to specify full recomputation of all integrals, the user
needs to enter only

\begin{verbatim}
  direct
\end{verbatim}

Exactly the same result could be obtained by entering the directive as

\begin{verbatim}
  semidirect filesize 0 memsize 0
\end{verbatim}

To disable the use of memory for caching integrals and
limit disk usage by each process to 100 MW, the input for the directive is
specified as follows;

\begin{verbatim}
  semidirect memsize 0 filesize 200000000
\end{verbatim}

% These two directives accomplish the same effect (full recomputation of
% integrals)
% \begin{verbatim}
%   direct
%   semidirect filesize 0 memsize 0
% \end{verbatim}


% To disable use of memory for caching integrals and to limit disk
% usage by each process to 100 MW



\subsubsection{Integral File Size and Format for the SCF Module}

The file format is rather complex since it accomodates a variety of
packing and compression options and distribution of data.  This section 
presents some information that may help the user understand the output 
and illustrates how to use the output information to estimate file sizes.

If integrals are stored with a threshold of greater than $10^{-10}$,
then the integrals
themselves are stored in a 32-bit fixed-point format. (This threshold
is sufficient for nearly all purposes, but large integrals will
require special treatment, as discussed ***WHERE?***.)  
If integrals are stored with a threshold that is less than $10^{-10}$,
howwever, the values are stored in 64-bit floating-point format.  
If a replicated-data calculation is being run,
then 8 bits are used for each basis function label, unless there are
more than 256 functions, in which case 16 bits are used.  If
distributed-data is being used, then the labels are always packed to
8-bits (the distributed blocks always being less than 256).

Thus, the number ($W$) of 64-bit words required to store $N$ integrals,
may be computed as
\begin{displaymath}
  W = \left\{ \\
      \begin{array}{c}
        N \mbox{ , 8-bit labels and 32-bit values} \\
        \frac{3}{2}N \mbox{ , 16-bit labels and 32-bit values} \\
        \frac{3}{2}N \mbox{ , 8-bit labels and 64-bit values} \\
        2N \mbox{ , 16-bit labels and 64-bit values} 
      \end{array}
      \right.
\end{displaymath}

The actual number of words required can be up to about one percent larger 
than the value of $W$ computed by the above relationship.  This is due 
to bookkeeping overhead, and because the file itself is organized into
fixed-size records.

With at least the default print level, all semidirect (not direct)
calculations will print out information about
the integral file and the number of integrals computed.  The form of this
output is as follows;

\begin{verbatim}
 Integral file          = ./c6h6.aoints.0
 Record size in doubles =  32769        No. of integs per rec  =  32768
 Max. records in memory =      3        Max. records in file   =      5
 No. of bits per label  =      8        No. of bits per value  =     32

 #quartets = 2.0D+04  #integrals = 7.9D+05  direct = 63.6%  cached = 36.4%
\end{verbatim}

The above file information relates only to process 0.  Apart from the
name, however, the same information will be written out for all nodes.  
The information about the number
of integrals, etc., is a sum over all processes.

When the integral file is closed, additional information of the following
form is printed.

\begin{verbatim}
------------------------------------------------------------
EAF file 0: "./c6h6.aoints.0" size=262152 bytes
------------------------------------------------------------
               write      read    awrite     aread      wait
               -----      ----    ------     -----      ----
     calls:        6        12         0         0         0
   data(b): 1.57e+06  3.15e+06  0.00e+00  0.00e+00
   time(s): 1.09e-01  3.12e-02                      0.00e+00
rate(mb/s): 1.44e+01  1.01e+02
------------------------------------------------------------

 Parallel integral file used       4 records with       0 large values
\end{verbatim}
The detailed file information relates just to process 0, but the 
line below that indicates the total number of integral records stored
by all processes.  

This information may be used to optimize subsequent calculations.


\subsection{SCF Module Convergence Control Options}
\label{sec:scfconv}

{\em Note to users: The SCF program will converge for all molecules
  currently included in the NWChem basis set library using only the default
  options, if given a sufficient number of iterations.  Please send
  Robert Harrison (at e-mail address \verb+nwchem-support@emsl.pnl.gov+)
  input file (and any relevent supporting information) for all applications
  where the SCF calculation fails to converge, or for which it is necessary
  to modify the default values of the convergence parameters in order to
  obtain a solution.  If we
  do not know about problems we cannot fix them.  Presently, there are
  no molecules that do not converge, given sufficient iterations, with
  the default options.}

% An understanding of the output of the SCF program and the options
% controlling convergence requires some knowledge of the convergence
% scheme.

The SCF program uses a preconditioned conjugate gradient (PCG) method 
that is unconditionally convergent if the problem is properly specified.  
Basically, the orbital gradient (which is the derivative of the energy 
with respect to the orbital rotations) is multiplied by a
level-shifted approximation to the inverse of the orbital Hessian to
form a search direction.  In the initial iterations (see Section
\ref{sec:nrswitch}) an inexpensive one-electron approximation to the
inverse orbital Hessian is used.  Closer to convergence, the full
orbital Hessian is used, which should provide quadratic convergence.
For both the full or one-electron orbital Hessians, the
inverse-Hessian matrix-vector product is formed iteratively.
Subsequently, an approximate line search is performed along the new
search direction.  If the exact Hessian is being employed then the
line search should require a single step (of unity).  Preconditioning
with approximate Hessians may require additional steps, especially in
the initial iterations.  It is the line search that provides the
convergence guarantee.  The iterations required to solve the linear
equations are referred to as micro-iterations.  A macro-iteration
comprises both the iterative solution and a line search.

  Level-shifting plays {\em exactly} the same role in this algorithm
as it does in the conventional iterative solution of the SCF
equations.  The approximate Hessian used for preconditioning should be
positive definite.  If this is not the case, then level-shifting by a
positive constant ($\Delta$) serves to make the preconditioning matrix
positive definite by adding $\Delta$ onto all of its eigenvalues.  The
level-shifts employed for the orbital Hessian should be approximately
four times the value that one would employ in a conventional
SCF\footnote{This can be seen by considering a one-electron
approximation to the closed-shell RHF Hessian in canonical orbitals,
$A_{ia,jb} = 4 \delta_{ij} \delta_{ab} (\epsilon_a - \epsilon_i)$.}.
Level-shifting is automatically enabled in the early iterations and
the default options suffice for most test cases.

  So why do things go wrong and what can be done to fix convergence
problems?  Most problems encountered so far arise from small or
negative eigenvalues of the orbital Hessian which can occur even
though you seem to be close to convergence (as measured by the
gradient norm, or off diagonal Fock matrix elements).  Small
eigenvalues will cause the iterative linear equation solver to
converge slowly, causing an excessive number of micro-iterations.
This makes the SCF expensive in terms of computation time, and
it is possible to exceed the maximum
number of iterations without achieving the accuracy required for
quadratic convergence which causes more macro-iterations to be
performed.  A negative eigenvalue in the Hessian will usually also
cause slow convergence of the micro-iterations (since negative
eigenvalues are usually small) and also cause components of the
line-search direction to point uphill, which again slows convergence
of the macro-iterations and causes more steps to be taken in the line
search.

  There are two main options available when a problem will not converge;
Newton-Raphson can be disabled temporarily or permanently, and level-shifting 
can be applied to the matrix.  In some cases, both options may be 
necessary to acheive final convergence.

If there is reason to suspect a negative eigenvalue,
the first course is to disable the Newton-Raphson iteration until the
solution is closer to convergence.  It may be necessary to disable it
completely.  At some point close to convergence, the Hessian will be
positive definite, so disabling Newton-Raphson should yeild a solution with
approximately the same convergence rate as DIIS.

If temporarily disabling Newton-Raphson is not sufficient to acheive 
convergence, it may be necessary to disable it entirely and apply a 
small level-shift to the approximate Hessian.  
This should improve the convergence rate of the
micro-iterations and stabilize the macro-iterations.  The
level-shifting will destroy exact quadratic convergence, but the
optimization process is automatically adjusted to reflect this by
enforcing conjugacy and reducing the accuracy to which the linear
equations are solved.  The net result of this is that the solution will
do more macro-iterations, but each one should take less time than it would
with the unshifted Hessian.

The following subsections describe the directives needed to disable the
Newton-Raphson iteration and specify level-shifting.


\subsection{Newton-Raphson Switch for the SCF Module}
\label{sec:nrswitch}

This directive allows the user to specify the point at which the full orbital 
Hessian will be applied as a preconditioner in the SCF solution.  The form
of the directive is as follows;

\begin{verbatim}
    NR <real nr_switch default 0.1>
\end{verbatim}

The exact orbital Hessian is adopted as the preconditioner
when the maximum element is below the value specified for \verb+nr_switch+.
The default value is 0.1, which means that Newton-Raphson will be disabled
until the solution is within 0.1 of convergence.  To disable Newton-Raphson
entirely, the value of \verb+nr_switch+ must be set to zero.  The directive
for this option is as follows;

\begin{verbatim}
  nr 0
\end{verbatim}

This input directive will entirely disable the use of the exact Hessian.

\subsection{Level-shifting of the Matrix in the SCF Module}
\label{sec:level}

This directive allows the user to specify level-shifting to obtain a
positive-definite preconditioned matrix for the SCF solution procedure.  
Separate level shifts can be set for the preconditioned conjugate
gradient (PCG) method (which yeilds an approximate solution), and for the
exact Hessians.  It is also possible to change the level-shift automatically
as the solution attains some specified accuracy.  The form of the 
directive is as follows;

\begin{verbatim}
   LEVEL [pcg <real initial default 20.0> \
           [<real tol default 0.5> <real final default 0.0>]] \
         [nr <real initial default 0.0> \
           [<rel tol default 0.0> <real final default 0.0>]]
\end{verbatim}

% Section \ref{sec:scfconv} discussed the use of level shifts to control
% convergence.  Level shifts may be set independently for both the
% approximate (denoted here by PCG) and exact Hessians (denoted by
% NR).  You can also have the level shift automatically changed when a
% certain accuracy is attained.  

This directive contains only two keywords; one for the PCG method and the
other for the exact Hessians.  Specifying the keyword \verb+pcg+ 
allows the user to define the level shifting for the approximate 
(i.e., PCG) method.  Specifying the keyword \verb+nr+ allows the user
to define the level shifting for the exact Hessians.  In both options,
the initial level shift is defined by the value specified for the local
real variable \verb+initial+.  The variable \verb+tol+ can be used 
independently with each keyword to define the level of accuracy that must
be attained in the solution before the level shifting is changed to the
value specified by input in the real variable \verb+final+.

For the PCG method (as specified using the keyword \verb+pcg+), the
defaults for this input are 20.0 for \verb+initial+, 0.5 for \verb+tol+,
and 0.0 for \verb+final+.  This meeans that the approximate Hessian will
be shifted by 20.0 until the maximum element of
the gradient falls below 0.5, at which point the shift will be set to zero.

For the exact Hessian (as specified using the keyword \verb+nr+), the
defaults are all zero.  The exact Hessian is usually not shifted.  An
example of an input directive that applies a shift of 0.2 to 
the exact Hessian is as follows;

\begin{verbatim}
  level nr 0.2
\end{verbatim}

To apply this shift to the exact Hessian only until the
maximum element of the gradient falls below 0.005, the required input
directive is as follows;

\begin{verbatim}
  level nr 0.2 0.005 0
\end{verbatim}

Note that for both of these examples the parameters for the PCG method 
are at the default values.  To obtain values different from the defaults,
the keyword \verb+pcg+ must be specified also.  For example, to specify
the level shifting in the above example for the exact Hessian {\em and} 
non-default shifting for the PCG method, the directive would be something
like the following;

\begin{verbatim}
  level pcg 20 0.3 0.0 nr 0.2 0.005 0.0
\end{verbatim}

This input will cause the PCG method to be level-shifted by 20.0 until the
maximum element of the gradient falls below 0.3, then the shift will be
zero.  For the exact Hessian, the level shifting is initially 0.2, until
the maximum element falls below 0.005, after which the shift is zero.
(Which solution will actually be used for a given calculation depends on
the input specified for \verb+nr_switch+ on the \verb+NR+ directive; see
Section \ref{sec:nrswitch} above.)

% The default options correspond to
% \begin{verbatim}
%   level pcg 20 0.5 0 nr 0 0 0
% \end{verbatim}
% The following applies a shift of 0.2 to the exact Hessian
% for the duration of the calculation:
% \begin{verbatim}
%   level nr 0.2
% \end{verbatim}
% To reduce this shift to zero when the maximum element of the gradient
% falls below 0.005:
% \begin{verbatim}
%   level nr 0.2 0.005 0
% \end{verbatim}


\subsection{Printing Information from the SCF Module}
\label{sec:scfprint}

All output from the SCF module is controlled using the \verb+PRINT+
directive described in Section \ref{sec:printcontrol}.  The following 
list describes the items from SCF that are currently under direct 
print control, along with the print level or each one.

\begin{tabbing}
  Very\_long\_descriptive\_name \= Print level space \= \kill
  Name                    \> Print Level \> Description \\
                          \>        \> \\
 ``atomic guess density'' \> debug  \> guess density matrix \\
 ``atomic scf''           \> debug  \> details of atomic SCF \\
 ``mo guess''             \> default\> brief info from mo guess \\
 ``information''          \> low    \> results  \\
 ``initial vectors''      \> debug  \> \\
 ``intermediate vectors'' \> debug  \> \\
 ``final vectors''        \> debug  \> \\
 ``intermediate evals''   \> debug  \> \\
 ``final evals''          \> default\> \\
 ``schwarz''              \> high   \> integral screening info \&
  stats at completion\\
 ``screening statistics'' \> debug  \> display stats after every Fock build \\
 ``geometry''             \> high   \> \\
 ``symmetry''             \> debug  \> detailed symmetry info \\
 ``basis''                \> high   \> \\
 ``geombas''              \> debug  \> detailed basis map info \\
 ``vectors i/o''          \> default\> report vectors I/O \\
 ``parameters''           \> default\> convergence parameters \\
 ``convergence''          \> default\> info each iteration
\end{tabbing}

If no \verb+PRINT+ directives are defined explicitly in the input for the
SCF module, the printed output from an SCF calculation will consist only
of the above items with a ``default'' print level (i.e., \verb+''mo guess''+,
\verb+''final evals''+, \verb+''vectors i/o''+, \verb+''parameters''+,
and \verb+''convergence''+).  If the keyword \verb+debug+ is explicitly
invoked on a \verb+PRINT+ directive, the output will include all items
in the above list with a print level of \verb+debug+, in addition to the
items with a print level of \verb+default+.  If the keyword \verb+high+ is
also invoked, all items in the above list will be printed.  If only the
keyword \verb+low+ is invoked, the only item that will be printed is
\verb+''information''+.



