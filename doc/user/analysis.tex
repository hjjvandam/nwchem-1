\label{sec:analysis}
\def\bmu{\mbox{\boldmath $\mu$}}
\def\bE{\mbox{\bf E}}
\def\br{\mbox{\bf r}}
\def\tT{\tilde{T}}
\def\t{\tilde{1}}
\def\ip{i\prime}
\def\jp{j\prime}
\def\ipp{i\prime\prime}
\def\jpp{j\prime\prime}
\def\etal{{\sl et al.}}
\def\nwchem{{\bf NWChem}}
\def\nwargos{{\bf nwargos}}
\def\nwtop{{\bf nwtop}}
\def\nwrst{{\bf nwrst}}
\def\nwsgm{{\bf nwsgm}}
\def\esp{{\bf esp}}
\def\md{{\bf md}}
\def\prepare{{\bf prepare}}
\def\analysis{{\bf analysis}}
\def\argos{{\bf ARGOS}}
\def\amber{{\bf AMBER}}
\def\charmm{{\bf CHARMM}}
\def\discover{{\bf DISCOVER}}
\def\povray{{\bf povray}}
\def\gopenmol{{\bf gOpenMol}}
\def\ecce{{\bf ecce}}

The \analysis\ module is used to analyze molecular trajectories generated
by the \nwchem\ molecular dynamics module, or partial charges generated
by the \nwchem\ electrostatic potential fit module. This module should
not de run in parallel mode.

Directives for the \analysis\ module are read from an input deck,

\begin{verbatim}
analysis
 ...
end
\end{verbatim}

The analysis is performed  as post-analysis of trajectory files through 
using the {\rm task} directive

\begin{verbatim}
task analysis
\end{verbatim}
or
\begin{verbatim}
task analyze
\end{verbatim}

\section{Reference coordinates}

Most analyses require a set of reference coordinates. These
coordinates are read from a \nwchem\ restart file by the directive,

\begin{verbatim}
reference <string filename>
\end{verbatim}

where {\rm filename} is the name of an existing restart file. 
This input directive is required.

\section{File specification}

The trajectory file(s) to be analyzed are specified with

\begin{verbatim}
file <string filename> [<integer firstfile> <integer lastfile>] 
\end{verbatim}

where {\rm filename} is an existing {\rm trj} trajectory file.
If {\rm firstfile} and {\rm lastfile} are specified, the specified
{\rm filename} needs to have a {\rm ?} wild card character that will 
be substituted by the 3-character integer number from {\rm firstfile} 
to {\rm lastfile}, and the analysis will be performed on the series 
of files.
For example,

\begin{verbatim}
file tr_md?.trj 3 6
\end{verbatim}

will instruct the analysis to be performed on files {\it tr\_md003.trj},
{\it tr\_md004.trj}, {\it tr\_md005.trj} and {\it tr\_md006.trj}.

\par
From the specified files the subset of frames to be analyzed is 
specified by

\begin{verbatim}
frames [<integer firstframe default 1>] <integer lastframe> \
       [<integer frequency default 1>]
\end{verbatim}

For example, to analyze the first 100 frames from the specified
trajectory files, use

\begin{verbatim}
frames 100
\end{verbatim}

To analyze every 10-th frame between frames 200 and 400 recorded on
the specified trajectory files, use

\begin{verbatim}
frames 200 400 10
\end{verbatim}

Solute coordinates of the reference set and ech subsequent frame
read from a trajectory file are translated to have the center of
geometry of the specified solute molecule at the center of the
simulation box. After this translation all molecules are folded
back into the box according to the periodic boundary conditions.
The directive for this operation is

\begin{verbatim}
center <integer imol> [<integer jmol default imol>]
\end{verbatim}

Coordinates of each frame read from a trajectory file can be 
rotated using

\begin{verbatim}
rotate ( off | x | y | z ) <real angle units degrees>
\end{verbatim}

If \verb+center+ was defined, rotation takes place after
the system has been centered. The \verb+rotate+ directives
only apply to frames read from the trajectory files, and not
to the reference coordinates. Upto 100 \verb+rotate+ directives
can be specified, which will be carried out in the order in which
they appear in the input deck. \verb+rotate off+ cancels all
previously defined \verb+rotate+ directives.

To set hydrogen bond criteria:

\begin{verbatim}
hbond [distance [[<real rhbmin default 0.0>] <real rhbmin>]] \
      [donorangle [<real hbdmin> [ <real hbdmax default pi>]]] \
      [acceptorangle [<real hbamin> [ <real hbamax default pi>]]]
\end{verbatim}

\section{Selection}

Analyses can be applied to a selection of solute atoms and solvent molecules. 
The selection is determined by

\begin{verbatim}
select ( [ <integer isgm> [ <integer jsgm> ]] [ { <string atom> } ] |
	solvent <real range> | save <string filename> | read <string filename> )
\end{verbatim}

where {\rm isgm} is the first segment number, {\rm jsgm} is the last 
segment number in the selection, and {\rm \{atom\}} is the set of atom
names selected from the specified residues. By default all solute
atoms are selected.
\par
For example, all protein backbone atoms are selected by

\begin{verbatim}
select _N _CA _C
\end{verbatim}

To select the backbone atoms in residues 20 to 80 only, use

\begin{verbatim}
select 20 80 _N _CA _C
\end{verbatim}

This selection is reset to apply to all atoms after each file
directive.

Solvent molecules within \verb+range+ nm from any selected solute atom
are selected by

\begin{verbatim}
select solvent <real range>
\end{verbatim}

After solvent selection, the solute atom selection is reset to being all
selected.

The current selection can be saved to, or read from a file using the 
\verb+save+ and \verb+read+ keywords, respectively.

\par

Some analysis are performed on groups of atoms. These groups of atoms
are define by

\begin{verbatim}
define <integer igroup> [<real rsel>] [<integer isel> [<integer jsel>]] \\
 [solvent] { <string atom> }
\end{verbatim}

The atom string in this definitions takes the form

\begin{verbatim}
[<integer segment>:]<string atomname>
\end{verbatim}

In the atomname a question mark may be used as a wildcard character.

\section{Coordinate analysis}

To analyze the root mean square deviation from the specified reference
coordinates:

\begin{verbatim}
rmsd
\end{verbatim}

To analyze protein $\phi$-$\psi$ and backbone hydrogen bonding:

\begin{verbatim}
ramachandran
\end{verbatim}

To define a distance:

\begin{verbatim}
distance <integer ibond> <string atomi> <string atomj> 
\end{verbatim}

To define an angle:

\begin{verbatim}
angle <integer iangle> <string atomi> <string atomj> <string atomk> 
\end{verbatim}

To define a torsion:

\begin{verbatim}
torsion <integer itorsion> <string atomi> <string atomj> \
                       <string atomk> <string atoml> 
\end{verbatim}

The atom string in these definitions takes the form

\begin{verbatim}
<integer segment>:<string atomname> | w<integer molecule>:<string atomname>
\end{verbatim}

for solute and solvent atom specification, respectively.

To define charge distribution in z-direction:

\begin{verbatim}
charge_distribution <integer bins>
\end{verbatim}

Analyses on pairs of atoms in predefined groups are specified by

\begin{verbatim}
groups [<integer igroup> [<integer jgroup>]] [periodic [<integer ipbc default 3>]] \ 
       <string function> [<real value1> [<real value2>]] [<string filename>]
\end{verbatim}

where $igroup$ and $jgroup$ are groups of atoms defined with a
\verb+define+ directive. Keyword \verb+periodic+ specifies that
periodic boundary conditions need to be applied in $ipbc$ dimensions.
The type of analysis is define by $function$, $value1$ and $value2$.
If $filename$ is specified, the analysis is applied to the reference
coordinates and written to the specified file. If no filename is
given, the analysis is applied to the specified trajectory and 
performed as part of the \verb+scan+ directive.
Implemented analyses defined by 
\verb+<string function> [<real value1> [<real value2>]]+ include
\verb+distances+ to calculate all atomic distances between atoms
in the specified groups that lie between $value1$ and $value2$.

Coordinate histograms are specified by

\begin{verbatim}
histogram <integer idef> [<integer length>] zcoordinate <string filename>
\end{verbatim}

where $idef$ is the atom group definition number, $length$ is the size
of the histogram, \verb+zcoordinate+ is the currently only histogram option,
and $filename$ is the filname to which the histogram is written.

To perform the coordinate analysis:

\begin{verbatim}
scan [ super ] <string filename>
\end{verbatim}

which will create, depending on the specified analysis options
files filename.rms and filename.ana. After the scan directive
previously defined coordinate analysis options are all reset.
Optional keyword \verb+super+ specifies that frames read from
the trajectory file(s) are superimposed to the reference structure
before the analysis is performed.

\section{Essential dynamics analysis}

Essential dynamics analysis is performed by

\begin{verbatim}
essential
\end{verbatim}

This can be followed by one or more

\begin{verbatim}
project <integer vector> <string filename>
\end{verbatim}

to project the trajectory onto the specified vector. This will
create files filename with extensions frm or trj, val, vec, \_min.pdb
and \_max.pdb, with the projected trajectory, the projection
value, the eigenvector, and the minimum and maximum projection
structure.

For example, an essential dynamics analysis with projection onto
the first vector generating files firstvec.\{trj, val, vec, \_min.pdb, \_max.pdb\}
is generated by

\begin{verbatim}
essential
project 1 firstvec
\end{verbatim}

\section{Trajectory format conversion}

To write a single frame in PDB or XYZ format, use

\begin{verbatim}
write [<integer number default 1>] [super] [solute] <string filename>
\end{verbatim}

To copy the selected frames from the specified trejctory file(s),
onto a new file, use

\begin{verbatim}
copy  [solute] [rotate <real tangle>] <string filename>
\end{verbatim}

To superimpose the selected atoms for each specified frame to the 
reference coordinates before copying onto a new file, use

\begin{verbatim}
super [solute] [rotate <real tangle>] <string filename>
\end{verbatim}

The \verb+rotate+ directive specifies that the structure will make
a full ratation every tangle ps. This directive only has effect when
writing povray files.

The format of the new file is determined from the extension, which
can be one of

\begin{tabular}{rl}
amb & \amber\ formatted trajectory file (obsolete)\\
arc & \discover\ archive file\\
bam & \amber\ unformatted trajectory file\\
crd & \amber\ formatted trajectory file\\
dcd & \charmm\ formatted trajectory file\\
esp & \gopenmol\ formatted electrostatic potential files\\
frm & \ecce\ frames file (obsolete)\\
pov & \povray\ input files\\
trj & \nwchem\ trajectory file\\
\end{tabular}

If no extension is specified, a {\rm trj} formatted file will be written.

A special tag can be added to {\rm frm} and {\rm pov} formatted files  using

\begin{verbatim}
label <integer itag> <string tag>  [ <real rval default 1.0> ] \\
    [ <integer iatag> [ <integer jatag default iatag> ] [ <real rtag default 0.0> ] ]
    [ <string anam> ]
\end{verbatim}

where tag number $itag$ is set to the string $tag$ for all atoms
anam within a distance $rtag$ from segments $iatag$ through $jatag$.
A question mark can be used in anam as a wild card character.
\par

Atom rendering is specified using

\begin{verbatim}
render ( cpk | stick )  [ <real rval default 1.0> ] \\
    [ <integer iatag> [ <integer jatag default iatag> ] [ <real rtag default 0.0> ] ]
    [ <string anam> ]
\end{verbatim}

for all atoms anam within a distance $rtag$ from segments $iatag$ through $jatag$,
and a scaling factor of $rval$. A question mark can be used in anam as a wild card 
character.
\par

Atom color is specified using

\begin{verbatim}
color ( <string color> | atom ) \\
    [ <integer iatag> [ <integer jatag default iatag> ] [ <real rtag default 0.0> ] ]
    [ <string anam> ]
\end{verbatim}

for all atoms anam within a distance $rtag$ from segments $iatag$ through $jatag$.
A question mark can be used in anam as a wild card character.
\par
For example, to display all carbon atoms in segments 34 through 45 
in green and rendered cpk in povray files can be specified with

\begin{verbatim}
render cpk 34 45 _C??
color green 34 45 _C??
\end{verbatim}

Coordinates written to a pov file can be scaled using

\begin{verbatim}
scale <real factor>
\end{verbatim}

A zero or negative scaling factor will scale the coordinates to
lie within [-1,1] in all dimensions.
\par
The cpk rendering in povray files can be scaled by

\begin{verbatim}
cpk <real factor default 1.0>
\end{verbatim}
\par

The stick rendering in povray files can be scaled by

\begin{verbatim}
stick <real factor default 1.0>
\end{verbatim}

The initial sequence number of esp related files is defined by

\begin{verbatim}
index <integer index default 1>
\end{verbatim}

\section{Electrostatic potentials}

A file in plt format of the electrostatic potential resulting
from partial charges generated by the ESP module is generated
by the command

\begin{verbatim}
esp  [ <integer spacing default 10> ] \
     [ <real rcut default 1.0> ] [periodic [<integer iper default 3>]] \
     [ <string xfile> [ <string pltfile> ] ]
\end{verbatim}

The input coordinates are taken from the {\rm xyzq} file that can
be generated from a {\rm rst} by the prepare module. Parameter 
spacing specifies the number of gridpoints per nm, rcut specifies 
extent of the charge grid beyond the molecule.
Periodic boundaries will be used if \verb+periodic+
is specified. If \verb+iper+ is set to 2, periodic boundary
conditions are applied in x and y dimensions only. If \verb+periodic+
is specified, a negative value of \verb+rcut+ will extend the grid
in the periodic dimensions by abs(\verb+rcut+), otherwise this value
will be ignored in the periodic dimensions.
The resulting {\rm plt} formatted file pltfile can be
viewed with the gOpenMol program. The resulting electrostatic 
potential grid is in units of kJ\ mol$^{-1}$e$^{-1}$.
If no files are specified, only the parameters are set. This
analysis applies to solute(s) only.

