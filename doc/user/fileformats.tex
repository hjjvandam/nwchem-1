%
% $Id: fileformats.tex,v 1.8 2007-12-28 18:55:48 bylaska Exp $
%
\label{sec:formats}
\section {Format fragment file}

\begin{table}[h]
\begin{center}
\begin{tabular}{p{15mm}p{12mm}l}
\hline\hline
Card & Format & Description \\ \hline
I-1-1  & a1     & \$ to identify the start of a fragment \\ % $ for emacs
I-1-2  & a10    & name of the fragment, the tenth character\\
       &        & N: identifies beginning of a chain\\
       &        & C: identifies end of a chain\\
       &        & blank: identifies chain fragment\\
       &        & M: identifies an integral molecule\\
\hline
I-2-1  & i5     & number of atoms in the fragment\\ 
I-2-2  & i5     & number of parameter sets\\
I-2-3  & i5     & default parameter set\\
I-2-4  & i5     & number of z-matrix definition\\
\hline
\mc{3}{l}{For each parameter set one card I-3} \\
\hline
I-3-1  & a      & residue name for parameter set\\
\hline
\mc{3}{l}{For each atom one deck II} \\
\hline
II-1-1  & i5     & atom sequence number \\
II-1-2  & a6     & atom name \\
II-1-3  & a5     & atom type \\
II-1-4  & a1     & dynamics type\\
        &        & \verb+ + : normal\\
        &        & \verb+D+ : dummy atom\\
        &        & \verb+S+ : solute interactions only\\
        &        & \verb+Q+ : quantum atom\\
        &        & other : intramolecular solute interactions only\\
II-1-5  & i5     & link number\\
        &        & 0: no link\\
        &        & 1: first atom in chain\\
        &        & 2: second atom in chain\\
        &        & 3 and up: other links\\
II-1-6  & i5     & environment type\\
        &        & 0: no special identifier\\
        &        & 1: planar, using improper torsion\\
        &        & 2: tetrahedral, using improper torsion\\
        &        & 3: tetrahedral, using improper torsion\\
        &        & 4: atom in aromatic ring\\
II-1-7  & i5     &  \\
II-1-8  & i5     & charge group\\
II-1-9  & i5     & polarization group\\
II-1-10 & f12.6  & atomic partial charge\\
II-1-11 & f12.6  & atomic polarizability\\
\hline
For each additional parameter set on card II-2\\
\hline
II-2-1  & 11x,a6     & atom type\\
II-2-2  & a1         & dynamics type\\
        &            & \verb+ + : normal\\
        &            & \verb+D+ : dummy atom\\
        &            & \verb+S+ : solute interactions only\\
        &            & \verb+Q+ : quantum atom\\
        &            & other : intramolecular solute interactions only\\
II-2-7  & 25x,f12.6  & atomic partial charge\\
II-2-8  & f12.6      & atomic polarizability\\
\hline
\mc{3}{l}{Any number of cards in deck III to specify complete 
connectivity} \\
\hline
III-1-1  & 16i5   & connectivity, duplication allowed\\
\hline
One blanc card to signal the end of the connectivity list\\
\hline
For each z-matrix definition one card IV\\
\hline
IV-1-1 & i5    & atom i \\
IV-1-2 & i5    & atom j \\
IV-1-3 & i5    & atom k \\
IV-1-4 & i5    & atom l \\
IV-1-5 & f12.6 & bond length i-j\\
IV-1-6 & f12.6 & angle i-j-k\\
IV-1-7 & f12.6 & torson i-j-k-l\\
\hline\hline
\end{tabular}
\caption{The format of fragment files.\label{tbl:nwmdfrg}}
\end{center}
\end{table}

\section {Format segment file}

\begin{table}[h]
\begin{center}
\begin{tabular*}{150mm}{p{15mm}p{12mm}l}
\hline\hline
Deck  & Format & Description \\ \hline
I-1-1 & a1     & \$ to identify the start of a segment \\ %$ for emacs
I-1-2 & a10    & name of the segment, the tenth character\\
      &        & N: identifies beginning of a chain\\
      &        & C: identifies end of a chain\\
      &        & blank: identifies chain fragment\\
      &        & M: identifies an integral molecule\\
\hline
I-2-1 & f12.6  & version number \\
\hline
I-3-1 & i5     & number of atoms in the segment\\
I-3-2 & i5     & number of bonds in the segment\\
I-3-3 & i5     & number of angles in the segment\\
I-3-4 & i5     & number of proper dihedrals in the segment\\
I-3-5 & i5     & number of improper dihedrals in the segment\\
I-3-6 & i5     & number of z-matrix definitions\\
I-3-7 & i5     & number of parameter sets \\
1-3-8 & i5     & default parameter set\\
\hline
For each prarmeter set one card I-4\\
\hline
I-4-1 & f12.6  & dipole correction energy \\
\hline
\end{tabular*}
\caption{Segment file format, table 1 of 7.\label{tbl:nwmdseg1}}
\end{center}
\end{table}

\begin{table}[h]
\begin{center}
\begin{tabular*}{150mm}{p{15mm}p{12mm}l}
\hline\hline
Deck & Format & Description \\ \hline
\mc{3}{l}{For each atom one deck II} \\
\hline
II-1-1  & i5     & atom sequence number \\
II-1-2  & a6     & atom name \\
II-1-3  & i5     & link number\\
II-1-4  & i5     & environment type\\
        &        & 0: no special identifier\\
        &        & 1: planar, using improper torsion\\
        &        & 2: tetrahedral, using improper torsion\\
        &        & 3: tetrahedral, using improper torsion\\
        &        & 4: atom in aromatic ring\\
II-1-5  & i5     & \\
II-1-6  & i5     & charge group\\
II-1-7  & i5     & polarization group\\
\hline
For each parameter set one card II-2\\
\hline
II-2-1  & 5x,a5  & atom type \\
II-2-2  & a1     & dynamics type\\
        &        & \verb+ + : normal\\
        &        & \verb+D+ : dummy atom\\
        &        & \verb+S+ : solute interactions only\\
        &        & \verb+Q+ : quantum atom\\
        &        & other : intramolecular solute interactions only\\
II-2-3  & f12.6  & atomic partial charge in e\\
II-2-4  & f12.6  & atomic polarizability/$4\pi\epsilon_o$ in nm$^3$\\
\hline
\end{tabular*}
\caption{Segment file format, table 2 of 7.\label{tbl:nwmdseg2}}
\end{center}
\end{table}

\begin{table}[h]
\begin{center}
\begin{tabular*}{150mm}{p{15mm}p{12mm}l}
\hline\hline
Deck & Format & Description \\ \hline
\mc{3}{l}{For each bond a deck III} \\
III-1-1 & i5     & bond sequence number \\
III-1-2 & i5     & bond atom i \\
III-1-3 & i5     & bond atom j \\
III-1-4 & i5     & bond type \\
        &        & 0: harmonic\\
        &        & 1: constrained bond\\
III-1-5 & i5     & bond parameter origin\\
        &        & 0: from database, next card ignored \\
        &        & 1: from next card\\
\hline
For each parameter set one card III-2\\
\hline
III-2-1 & f12.6  & bond length in nm\\
III-2-2 & e12.5  & bond force constant in kJ nm$^2$ mol$^{-1}$\\
\hline
\end{tabular*}
\caption{Segment file format, table 3 of 7.\label{tbl:nwmdseg3}}
\end{center}
\end{table}

\begin{table}
\begin{center}
\begin{tabular*}{150mm}{p{15mm}p{12mm}l}
\hline\hline
Deck & Format & Description \\ \hline
\mc{3}{l}{For each angle a deck IV} \\
IV-1-1 & i5     & angle sequence number \\
IV-1-2 & i5     & angle atom i \\
IV-1-3 & i5     & angle atom j \\
IV-1-4 & i5     & angle atom k \\
IV-1-5 & i5     & angle type \\
       &        & 0: harmonic\\
IV-1-6 & i5     & angle parameter origin\\
       &        & 0: from database, next card ignored \\
       &        & 1: from next card\\
\hline
For each parameter set one card IV-2\\
\hline
IV-2-1 & f10.6  & angle in radians\\
IV-2-2 & e12.5  & angle force constant in kJ mol$^{-1}$\\
\hline
\end{tabular*}
\caption{Segment file format, table 4 of 7.\label{tbl:nwmdseg4}}
\end{center}
\end{table}

\begin{table}[h]
\begin{center}
\begin{tabular*}{150mm}{p{15mm}p{12mm}l}
\hline\hline
Deck & Format & Description \\ \hline
\mc{3}{l}{For each proper dihedral a deck V} \\
V-1-1 & i5     & proper dihedral sequence number \\
V-1-2 & i5     & proper dihedral atom i \\
V-1-3 & i5     & proper dihedral atom j \\
V-1-4 & i5     & proper dihedral atom k \\
V-1-5 & i5     & proper dihedral atom l \\
V-1-6 & i5     & proper dihedral type \\
      &        & 0: $C\cos(m\phi-\delta)$\\
V-1-7 & i5     & proper dihedral parameter origin\\
      &        & 0: from database, next card ignored \\
      &        & 1: from next card\\
\hline
For each parameter set one card V-2\\
\hline
V-2-1 & i3     & multiplicity\\
V-2-2 & f10.6  & proper dihedral in radians\\
V-2-3 & e12.5  & proper dihedral force constant in kJ mol$^{-1}$\\
\hline
\end{tabular*}
\caption{Segment file format, table 5 of 7.\label{tbl:nwmdseg5}}
\end{center}
\end{table}

\begin{table}[h]
\begin{center}
\begin{tabular*}{150mm}{p{15mm}p{12mm}l}
\hline\hline
Deck & Format & Description \\ \hline
\mc{3}{l}{For each improper dihedral a deck VI} \\
VI-1-1 & i5     & improper dihedral sequence number \\
VI-1-2 & i5     & improper dihedral atom i \\
VI-1-3 & i5     & improper dihedral atom j \\
VI-1-4 & i5     & improper dihedral atom k \\
VI-1-5 & i5     & improper dihedral atom l \\
VI-1-6 & i5     & improper dihedral type \\
       &        & 0: harmonic\\
VI-1-7 & i5     & improper dihedral parameter origin\\
       &        & 0: from database, next card ignored \\
       &        & 1: from next card\\
\hline
For each parameter set one card VI-2\\
\hline
VI-2-1 & 3x,f10.6  & improper dihedral in radians\\
VI-2-2 & e12.5  & improper dihedral force constant in kJ mol$^{-1}$\\
\hline\hline
\end{tabular*}
\caption{Segment file format, table 6 of 7.\label{tbl:nwmdseg6}}
\end{center}
\end{table}

\begin{table}[h]
\begin{center}
\begin{tabular*}{150mm}{p{15mm}p{12mm}l}
\hline\hline
For each z-matrix definition one card VII\\
\hline
VII-1-1 & i5    & atom i \\
VII-1-2 & i5    & atom j \\
VII-1-3 & i5    & atom k \\
VII-1-4 & i5    & atom l \\
VII-1-5 & f12.6 & bond length i-j\\
VII-1-6 & f12.6 & angle i-j-k\\
VII-1-7 & f12.6 & torson i-j-k-l\\
\hline\hline
\end{tabular*}
\caption{Segment file format, table 6 of 7.\label{tbl:nwmdseg7}}
\end{center}
\end{table}

\section {Format sequence file}

\begin{table}[h]
\begin{center}
\begin{tabular*}{150mm}{p{15mm}p{12mm}l}
\hline\hline
Card & Format & Description \\ \hline
I-1-1  & a1     & \$ to identify the start of a sequence \\ %$ for emacs
I-1-2  & a10    & name of the sequence\\
\hline
\mc{3}{l}{Any number of cards 1 and 2 in deck II to specify the system} \\
\hline
II-1-1 & i5     & segment number\\
II-1-2 & a10    & segment name, last character will be determined from chain\\
\hline
II-2-1 & a      & \verb+break+ to identify a break in the molecule chain\\
\hline
II-2-1 & a      & \verb+molecule+ to identify the end of a solute molecule\\
\hline
II-2-1 & a      & \verb+fraction+ to identify the end of a solute fraction\\
\hline
II-2-1 & a5     & \verb+link + to specify a link\\
II-2-2 & i5     & segment number of first link atom\\
II-2-3 & a4     & name of first link atom \\
II-2-4 & i5     & segment number of second link atom\\
II-2-5 & a4     & name of second link atom \\
\hline
II-2-1 & a      & \verb+solvent+ to identify solvent definition on next card\\
\hline
II-2-1 & a      & \verb+stop+ to identify the end of the sequence\\
\hline
II-2-1 & a6     & \verb+repeat+ to repeat next $ncard$ cards $ncount$
times\\
II-2-2 & i5     & number of cards to repeat ($ncards$)\\
II-2-3 & i5     & number of times to repeat cards ($ncount$)\\
\mc{3}{l}{Any number of cards in deck II to specify the system} \\
\hline\hline
\end{tabular*}
\caption{Sequence file format.\label{tbl:nwmdseq}}
\end{center}
\end{table}

\section {Format trajectory file}

\begin{table}[h]
\begin{center}
\begin{tabular*}{150mm}{p{15mm}p{12mm}l}
\hline\hline
Card & Format & Description \\ \hline
I-1-1  & a6     & keyword \verb+header+ \\
\hline
I-2-1  & i10    & number of atoms per solvent molecule \\
I-2-2  & i10    & number of solute atoms \\
I-2-3  & i10    & number of solute bonds \\
I-2-4  & i10    & number of solvent bonds \\
I-2-5  & i10    & number of solvent molecules\\
I-2-6  & i10    & precision of the coordinates. 0=standard, 1=high\\
\hline
\mc{3}{l}{For each atom per solvent molecule one card I-3} \\
\hline
I-3-1  & a5     & solvent name \\
I-3-2  & a5     & atom name \\
\hline
\mc{3}{l}{For each solute atom one card I-4} \\
\hline
I-4-1  & a5     & segment name \\
I-4-2  & a5     & atom name \\
I-4-3  & i6     & segment number \\
I-4-4  & i10    & solute atom counter \\
I-4-5  & i5     & integer 1\\
\hline
\mc{3}{l}{For each solvent bond one card I-5} \\
\hline
I-5-1  & i8     & atom index i for bond between i and j \\
I-5-2  & i8     & atom index j for bond between i and j \\
\hline
\mc{3}{l}{For each solute bond one card I-6} \\
\hline
I-6-1  & i8     & atom index i for bond between i and j \\
I-6-2  & i8     & atom index j for bond between i and j \\
\hline
\mc{3}{l}{For each frame one deck II} \\
\hline
II-1-1  & a5     & keyword \verb+frame+ \\
\hline
II-2-1 & f12.6  & time of frame in ps \\
II-2-2 & f12.6  & temperature of frame in K \\
II-2-3 & e12.5  & pressure of frame in Pa \\
II-2-4 & a10    & date \\
II-2-5 & a10    & time \\
\hline
II-3-1 & f12.6  & box dimension x \\
\hline
II-3-2 & 12x,f12.6  & box dimension y \\
\hline
II-3-3 & 24x,f12.6  & box dimension z \\
\hline
II-4-1 & l1     & logical lxw for solvent coordinates \\
II-4-2 & l1     & logical lvw for solvent velocities \\
II-4-3 & l1     & logical lfw for solvent forces \\
II-4-4 & l1     & logical lpw for solvent induced dipoles \\
II-4-5 & l1     & logical lxs for solute coordinates \\
II-4-6 & l1     & logical lvs for solute velocities \\
II-4-7 & l1     & logical lfs for solute forces \\
II-4-8 & l1     & logical lps for solute induced dipoles \\
II-4-5 & i10    & number of solvent molecules \\
II-4-6 & i10    & number of solvent atoms \\
II-4-7 & i10    & number of solute atoms \\
\hline\hline
\end{tabular*}
\caption{Trajectory file format.\label{tbl:nwmdtrj}}
\end{center}
\end{table}


\begin{table}[h]
\begin{center}
\begin{tabular*}{150mm}{p{15mm}p{12mm}l}
\hline\hline
Card & Format & Description \\ \hline
\mc{3}{l}{For each solvent molecule one card II-5 for each atom, if stardard precision} \\
\hline
II-5-1 & 3f8.3  & solvent atom coordinates, if lxw or lvw \\
II-5-4 & 3f8.3  & solvent atom velocities, if lvw \\
II-5-7 & 3f8.1  & solvent atom forces, if lfw\\
\hline
\mc{3}{l}{For each solute atom one card II-6 for each atom, if stardard precision} \\
\hline
II-6-1 & 3f8.3  & solute atom coordinates, if lxs or lvs \\
II-6-4 & 3f8.3  & solute atom velocities, if lvs \\
II-6-7 & 3f8.1  & solute atom forces, if lfs \\
\hline
\hline
\mc{3}{l}{For each solvent molecule one card II-5 for each atom, if high precision} \\
\hline
II-5-1 & 3e12.6  & solvent atom coordinates, if lxw or lvw \\
II-5-4 & 3e12.6  & solvent atom velocities, if lvw \\
II-5-7 & 3e12.6  & solvent atom forces, if lfw (on new card if both lxw and lvw)\\
\hline
\mc{3}{l}{For each solute atom one card II-6 for each atom, if high precision} \\
\hline
II-6-1 & 3e12.6  & solute atom coordinates, if lxs or lvs \\
II-6-4 & 3e12.6  & solute atom velocities, if lvs \\
II-6-7 & 3e12.6  & solute atom forces, if lfs (on new card if both lxs and lvs)\\
\hline\hline
\end{tabular*}
\caption{continued trajectory file format.\label{tbl:nwmdtrj2}}
\end{center}
\end{table}

\section {Format free energy file}

\begin{table}[h]
\begin{center}
\begin{tabular*}{150mm}{p{15mm}p{12mm}l}
\hline\hline
Card & Format & Description \\ \hline
\mc{3}{l}{For each step in $\lambda$ one deck I} \\
I-1-1  & i7     & number $nderiv$ of data summed in derivative decomposition array $deriv$ \\
I-1-2  & i7     & length $ndata$ of total derivative array $drf$ \\
I-1-3  & f12.6  & current value of $\lambda$ \\
I-1-4  & f12.6  & step size of $\lambda$ \\
\hline
I-2-1  & 4e12.12 & derivative decomposition array $deriv(1:24)$ \\
\hline
I-3-1  & 4e12.12 & total derivative array $dfr(1:nda)$ \\
\hline
I-4-1  & i10    & size of ensemble at current $\lambda$ \\
I-4-2  & e20.12 & average temperature at current $\lambda$ \\
I-4-3  & e20.12 & average exponent reverse perturbation energy at current $\lambda$ \\
I-4-4  & e20.12 & average exponent forward perturbation energy at current $\lambda$ \\
\hline\hline
\end{tabular*}
\caption{Free energy file format.\label{tbl:nwmdgib}}
\end{center}
\end{table}

\section {Format root mean square deviation file}

\begin{table}[h]
\begin{center}
\begin{tabular*}{150mm}{p{15mm}p{12mm}l}
\hline\hline
Card & Format & Description \\ \hline
\mc{3}{l}{For each analyzed time step one card I-1} \\
I-1-1  & f12.6  & time in ps \\
I-1-2  & f12.6  & total rms deviation of the selected atoms before superimposition \\
I-1-3  & f12.6  & total rms deviation of the selected atoms after superimposition \\
\hline
II-1-1 & a8     & keyword \verb+analysis+ \\
\hline
\mc{3}{l}{For each solute atom one card II-2} \\
\hline
II-2-1  & a5     & segment name \\
II-2-2  & a5     & atom name \\
II-2-3  & i6     & segment number \\
II-2-4  & i10    & atom number \\
II-2-5  & i5     & selected if 1 \\
II-2-6  & f12.6  & average atom rms deviation after superimposition \\
\hline
III-1-1 & a8     & keyword \verb+analysis+ \\
\hline
\mc{3}{l}{For each solute segment one card III-2} \\
III-2-1 & a5     & segment name \\
III-2-2 & i6     & segment number \\
III-2-3 & f12.6  & average segment rms deviation after superimposition \\
\hline\hline
\end{tabular*}
\caption{Root mean square deviation file format.\label{tbl:nwmdrms}}
\end{center}
\end{table}

\section {Format property file}

\begin{table}[h]
\begin{center}
\begin{tabular*}{150mm}{p{15mm}p{12mm}l}
\hline\hline
Card & Format & Description \\ \hline
I-1-1  & i7     & number $nprop$ of recorded properties \\
I-1-2  & 1x,2a10 & date and time \\
\hline
\mc{3}{l}{For each of the $nprop$ properties one card I-2} \\
\hline
I-2-1  & a50    & description of recorded property \\
\hline
\mc{3}{l}{For each recorded step one deck II} \\
\hline
II-1-1 & 4(1pe12.5) & value of property \\
\hline\hline
\end{tabular*}
\caption{Property file format.\label{tbl:nwmdprp}}
\end{center}
\end{table}

