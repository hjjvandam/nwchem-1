% $Id: vib.tex,v 1.12 2002-02-06 19:38:53 sohirata Exp $
\label{sec:vib}

The nuclear hessian which is used to compute the vibrational
frequencies can be computed by finite difference for any ab initio
wave-function that has analytic gradients.  The appropriate
nuclear hessian generation algorithm is chosen based on the user input
when \verb+TASK <theory> frequencies+ is the task directive.

The vibrational package was integrated from the Utah Messkit and can
use any nuclear hessian generated from the driver routines, finite
difference routines or any analytic hessian modules.  There is no required
input for the ``VIB'' package.  VIB computes the Infra Red frequencies
and intensities\footnote{Intensities are only computed if the dipole
derivatives are available; these are computed by default for most
methods that use the finite difference driver routines} for the
computed nuclear hessian and the ``projected'' nuclear hessian.  The
VIB module projects out the translations and rotations of the nuclear
hessian using the standard Eckart projection algorithm.  
It also computes the zero point energy for the molecular system
based on the frequencies obtained from the projected hessian.

The default mass of each atom is used unless an alternative mass is
provided via the geometry input, (c.f., \ref{sec:geom}) or redefined
using the vibrational module input.  The default mass is the mass of
the most abundant isotope of each element.\footnote{c.f., "The
Elements" by John Emsley, Oxford University Press, (C) 1989, ISBN
0-19-855237-8.} If the abundance was roughly equal, the mass of the
isotope with the longest half life was used.

\section{Vibrational Module Input}

All input for the Vibrational Module is optional since the default
definitions will compute the frequencies and IR
intensities\footnote{The geometry specification at the point where the
hessian is computed must be the default ``geometry'' on the current
run-time-data-base for the projection to work properly.}.  The generic
module input can begin with \verb+vib+, \verb+freq+, \verb+frequency+
and has the form:
\begin{verbatim}
  {freq || vib || frequency}
    reuse [<string> hessian_filename]
    mass <integer> lexical_index <real> new_mass
    mass <string> tag_identifier <real> new_mass
    animate [<real> step_size_for_animation]     
  end
\end{verbatim}

\subsection{Hessian File Reuse}
By default the \verb+task <theory> frequencies+ directive will
recompute the hessian.  To reuse the previously computed hessian you
need only specify \verb+reuse+ in the module input block.  If you
have stored the hessian in an alternate place you may redirect the 
reuse directive to that file by specifying the path to that file.
\begin{verbatim}
  reuse /path_to_hessian_file
\end{verbatim}
This will reuse your saved Hessian data but one caveat is that the
geometry specification at the point where the hessian is computed must
be the default ``geometry'' on the current run-time-data-base for the
projection to work properly.

\subsection{Redefining Masses of Elements}
You may also modify the mass of a specific center or a group of
centers via the input.  

To modify the mass of a specific center you can simply use:
\begin{verbatim}
  mass 3 4.00260324
\end{verbatim}
which will set the mass of center 3 to 4.00260324 AMUs.  The lexical
index of centers is determined by the geometry object. 

To modify all Hydrogen atoms in a molecule you may use the tag based
mechanism:
\begin{verbatim}
  mass hydrogen 2.014101779
\end{verbatim}

The mass redefinitions always start with the default masses and
change the masses in the order given in the input.  Care must be taken to change
the masses properly.  For example, if you want all hydrogens to have
the mass of Deuterium and the third hydrogen (which is the 6th atomic
center) to have the mass of Tritium you must set the Deuterium masses
first with the tag based mechanism and then set the 6th center's mass
to that of Tritium using the lexical center index mechanism.  

The mass redefinitions are not fully persistent on the
run-time-data-base.  Each input block that redefines masses will
invalidate the mass definitions of the previous input block.
For example, 
\begin{verbatim}
freq
  reuse
  mass hydrogen 2.014101779
end
task scf frequencies
freq
  reuse
  mass oxygen 17.9991603
end
task scf frequencies
\end{verbatim}
will use the new mass for all hydrogens in the first frequency
analysis.  The mass of the oxygen atoms will be redefined in the second
frequency analysis but the hydrogen atoms will use the default mass.
To get a modified oxygen and hydrogen analysis you would have to use:
\begin{verbatim}
freq
  reuse
  mass hydrogen 2.014101779
end
task scf frequencies
freq
  reuse
  mass hydrogen 2.014101779
  mass oxygen 17.9991603
end
task scf frequencies
\end{verbatim}

\subsection{Animation} 
The ``VIB'' module also can generate mode animation input files in the
standard xyz file format for graphics packages like
RasMol or XMol {There are scripts to automate this for RasMol in
{\verb+$NWCHEM_TOP/contrib/rasmolmovie+}.  Each mode will have 20 xyz
files generated that cycle from the equilibrium geometry to 5 steps in
the positive direction of the mode vector, back to 5 steps in the
negative direction of the mode vector, and finally back to the
equilibrium geometry.  By default these files are {\bf not} generated.
To activate this mechanism simply use the following input directive 
\begin{verbatim}
  animate
\end{verbatim}
anywhere in the frequency/vib input block.
\subsubsection{Controlling the Step Size Along the Mode Vector}
By default, the step size used is 0.15 a.u. which will give reliable
animations for most systems.  This can be changed via the input directive
\begin{verbatim}
  animate real <step_size>
\end{verbatim}
where \verb+<step_size>+ is the real number that is the magnitude of
each step along the eigenvector of each nuclear hessian mode in atomic
units.

\newpage

\subsection{An Example Input Deck}
This example input deck will optimize the geometry for the given basis
set, compute the frequencies for H$_2$O, D$_2$O, HDO, and TDO.
\begin{verbatim}
start  h2o
title Water 
geometry units au autosym
  O      0.00000000    0.00000000    0.00000000
  H      0.00000000    1.93042809   -1.10715266
  H      0.00000000   -1.93042809   -1.10715266
end
basis noprint
  H library sto-3g 
  O library sto-3g
end
scf; thresh 1e-6; end
driver; tight; end
task scf optimize

scf; thresh 1e-8; print none; end
task scf freq 

freq 
 reuse; mass H 2.014101779
end
task scf freq

freq
 reuse; mass 2 2.014101779
end
task scf freq

freq
 reuse; mass 2 2.014101779 ; mass 3 3.01604927
end
task scf freq
\end{verbatim}
