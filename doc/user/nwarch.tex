\label{sec:arch}

As noted above, NWChem consists of independent modules that perform
the various functions of the code.  Examples of modules include the
input parser, SCF energy, SCF analytic gradient, and DFT energy.  The
independent NWChem modules can share data only through a disk-resident
database, which is a similiar to the GAMESS dumpfile or the Gaussian
checkpoint file.  This allows the modules to share data, or to share
access to files containing data.

It is not necessary for the user to be intimately familiar with the
contents of the database in order to run NWChem.  However, a nodding
acquaintance with the design of the code will help in clarifying the
logic behind the input requirements, especially when restarting jobs
or performing multiple tasks within one job.  Section
\ref{sec:database} gives a general description of the database.

As described above (Section \ref{sec:inputstructure}), all
start-up directives are processed at the beginning of the job
by the main program, and
then the input module is invoked.  Each input directive usually
results in one or more entries being made in the database.  When a
\verb+TASK+ directive is encountered, control is passed to the
appropriate module, which extracts relevant data from the database and
any associated files.  Upon completion of the task, the module will store
significant results in the database, and may also modify other
database entries in order to modify the behaviour of subsequent
computations.

\section{Database Structure}
\label{sec:database}

\sloppy

Data is shared between modules of NWChem by means of the database.  There
are three main types of information stored in the data base; (1) arrays of
data, (2) names of files that contain data, and (3) objects.  
Arrays are stored directly in the database, and contain the following
information;
\begin{enumerate}
\item the name of the array, which is a string of ASCII characters (e.g., 
      \verb+"reference energies"+)
\item the type of the data in the array 
(i.e., real, integer, logical, or character) 
\item the number (N) of data items in the array (Note: A scalar is stored as an array of unit length.)
\item the N items of data of the specified type
\end{enumerate}

\fussy

From the NWChem input deck it is possible to enter data directly into
the database using the \verb+SET+ directive (see Section
\ref{sec:set}).  For example, to store a (64-bit precision)
three-element real array with the name \verb+"reference energies"+ in
the database, the directive is as follows;
\begin{verbatim}
  set "reference energies" 0.0 1.0 -76.2
\end{verbatim}
NWChem determines the data to be real (based on the type of the first
element, \verb+0.0+), counts the number
of elements in the array, and enters the array into the database.

Much of the data stored in the database is internally managed by
NWChem and should not be modified by the user.  However, other data,
including some NWChem input options, can be freely modified.  

Objects are built in the database by storing associated data as
multiple entries using an internally consistent naming convention.
This data is exclusively managed by the subroutines (or methods) that
are associated with the object.  Currently, the code has two main
objects; basis sets and geometries.  Sections \ref{sec:geom}, 
\ref{sec:basis}, and
\ref{sec:ecp} present a complete discussion of the input to describe
these objects.  

As an illustration of what comprises a geometry object, the following
table contains a partial listing of the NWChem output of a water molecule
geometry named \verb+"test geom"+.  Each entry contains the field 
\verb+test geom+, which is the unique name of the object.

\begin{verbatim}
 Contents of RTDB h2o.db
 -----------------------

 Entry                                   Type[nelem]
 ---------------------------  ----------------------
 geometry:test geom:efield             double[3]    
 geometry:test geom:coords             double[9]    
 geometry:names                          char[10]   
 geometry:test geom:ncenter               int[1]    
 geometry:ngeom                           int[1]    
 geometry:test geom:charges            double[3]    
 geometry:test geom:tags                 char[6]
 ...
\end{verbatim}

Using this convention, multiple instances of objects may be stored with
different names in the same database.  If a user needed to do calculations 
considering alternative geometries
for the water molecule, for example, an input file could be constructed with 
all of the geometries of interest by storing them in the 
database under different names.  

% The {\tt
%   GEOMETRY} directive (Section \ref{sec:geom}) permits geometries to
% be named (the default name is \verb+geometry+).  For example, the
% input directive to define a geometry object in the database with the
% name \verb+"test water geometry"+ can be specified as follows;

The run-time database contents for the file \verb+h2o.db+ listed 
above were generated from the user-specified input directive,
\begin{verbatim}
  geometry "test geom"
    O     0.00000000    0.00000000    0.00000000
    H     0.00000000    1.43042809   -1.10715266
    H     0.00000000   -1.43042809   -1.10715266
  end
\end{verbatim}

Obviously, the geometry object contains more information than merely the
numerical input specified in this directive.  The \verb+GEOMETRY+
directive allows the user to specify different values for the x,y,z
coordinates of the atoms (or centers), and identify that object under
a unique name.  In addition, all of the other information in the 
geometry object is also available
to the specifically named objects.  (Refer to Section
\ref{sec:geom} for a complete description of the {\tt GEOMETRY}
directive.)

Unless a specific name is defined for the geometry object in the
database (such as the name \verb+"test geom"+ shown in the
example), the object in the database is assigned the default name of
\verb+geometry+.  This is the geometry object name that computational
modules will look for when executing a calculation.  The {\tt SET}
directive can be used in the input to force a module (or modules) to
look for a geometry object with a name other than \verb+geometry+
for a particular task.  For example, to specify use of the 
\verb+"test geom"+ object in the example above, the \verb+SET+
directive is as follows;

\begin{verbatim}
  set geometry "test geom"
\end{verbatim}

NWChem will automatically check for such indirections when loading
geometries.  The basis set object functions in an identical fashion,
using the default name \verb+"ao basis"+.

% , and it is intended that all
% future such objects will do so.  (Note: the naming conventions and
% internal mechanisms for associating data with specific modules or
% tasks is expected to change in the future, but the directive for
% specifying names should remain the same.)

\section{Persistence of data and restart}
\label{sec:persist}

The database is persistent, meaning that all input data and results
that are not destroyed in the course of execution are permanently
stored.  These data are therefore available to subsequent tasks or
jobs.  This makes the input for restart jobs very simple, since only
new or changed data must be provided.  It also makes the behavior of
successive restart jobs {\em identical} to that of multiple tasks
within one job.  

Sometimes, however, this persistence is undesirable, and it is
necessary to return an NWChem module to its default (input-free)
behavior. In such a case, the \verb+UNSET+ directive (see Section
\ref{sec:unset}) can be used to delete all database entries associated
with a given module (including both inputs and outputs).
