\label{sec:stepper}

\begin{verbatim}
  STEPPER
    ...
  END
\end{verbatim}
  No input is required for STEPPER.  If no input is present the
default actions are to minimize the energy as a function of the
geometry with a maximum of 20 stepper iterations.  The STEPPER input
must follow the geometry specification (and later the symmetry
specification) in the input deck because it requires the number of
atoms that are used in the geometry optimization.  Currently only
Cartesian coordinates are used.  STEPPER removes translational and
rotational components before determining the step direction (5
components for linear systems and 6 for others).  The initial guess
nuclear Hessian is the identity matrix and there is an ASCII interface
file to input an Hessian from another code.  The automatic generation
of finite difference Hessians will be available soon.  During the
optimization if the step taken is too large e.g., the step causes the
energy to go up, enough information is stored to allow the optimizer
to ``backstep'' and correct the step based on information prior to the
faulty step.

 The following optional sub-directives control the optimization
performed.

\subsection{MIN}

\begin{verbatim}
  MIN
\end{verbatim}

\verb+MIN+ instructs STEPPER to minimize the energy as a function of the
geometry of the system. This is the default action for STEPPER.

\subsection{MAXITER}

\begin{verbatim}
  MAXITER <integer maxiter default 20>
\end{verbatim}

\verb+MAXITER+ is the maximum number of geometry optimization steps.  The
geometry optimization will restart automatically.

\subsection{TS}

\begin{verbatim}
  TS
\end{verbatim}

\verb+TS+ instructs stepper to find the transition state by following the
lowest eigenvalue of the nuclear Hessian.

\subsection{TRACK}

\begin{verbatim}
  TRACK [nmode <integer nmode default 1>]
\end{verbatim}

\verb+TRACK+ allows stepper to track the eigenvector corresponding to
\verb+nmode+ during a transition state walk.  This input is invalid
for a minimization walk.  The step is constructed to go up in energy
along the \verb+nmode+ eigenvector and down in all other degrees of
freedom.  

\subsection{TRUST}

\begin{verbatim}
  TRUST <real radius default 0.1>
\end{verbatim}

  This directive overrides the default trust-radius of 0.1.  A larger
value enables larger steps to be taken. Experience shows that for
larger systems 20+ atoms it should be set to 0.5 or greater.

\subsection{CONVGG}
\begin{verbatim}
 CONVGG <real convgg default 1.0d-04>
\end{verbatim}

\verb+CONVGG+ sets the convergence tolerence for the gradient norm for
all degrees of freedom.  This should be approximately the square root
of the energy convergence tolerance.  

\subsection{CONVGE}
\begin{verbatim}
 CONVGE <real convge default 1.0d-08>
\end{verbatim}

\verb+CONVGE+ sets the convergence tolerence for the energy difference
in the geometry optimization.  


\subsection{Initial Guess for Nuclear Hessian}

{\bf BROKEN BUT EASILY FIXED?}

