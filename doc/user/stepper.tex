\label{sec:stepper}

The STEPPER module performs an energy minimization function on the molecule
defined by input using the \verb+GEOMETRY+ directive (see Section 
\ref{sec:geom}).  Optional input for this module is specified within
the compound directive,

\begin{verbatim}
  STEPPER
    ...
  END
\end{verbatim}

Input specified for the STEPPER module must appear in the input file 
{\em after} the \verb+GEOMETRY+ directive, since it must know the
number of atoms that are to be used in the geometry optimization.
In the current version of NWChem, STEPPER can be used  only with geometries
that are defined in 
Cartesian coordinates.  STEPPER removes translational and
rotational components before determining the step direction (5
components for linear systems and 6 for others).  The initial guess
nuclear Hessian is the identity matrix and there is an ASCII interface
file to input a Hessian from another code.  The automatic generation
of finite difference Hessians will be available soon.

Execution of the STEPPER module calculation is invoked immediately after the 
\verb+STEPPER+ directive has been processed.  This is a significantly
different from the way the other modules are executed.  Modules such as
the SCF, DFT, and RI\_MP2 are invoked with a \verb+TASK+ directive (see
Section \ref{sec:task}).

The default in STEPPER is to minimize the energy as a function of the 
geometry with a maximum of 20 stepper iterations.  When this is the
desired calculation, no input is required for the STEPPER module, except
the above directive to invoke it.  However, the user also has the option
of defining different tasks for the STEPPER module, and can vary the
number of iterations and the convergence criteria from the default values.  
The input for these options
is described in the following sections.

%  No input is required for STEPPER.  If no input is present the
% default actions are to minimize the energy as a function of the
% geometry with a maximum of 20 stepper iterations.  The STEPPER input
% must follow the geometry specification (and later the symmetry
% specification) in the input deck because it requires the number of
% atoms that are used in the geometry optimization.  Currently only
% Cartesian coordinates are used.  STEPPER removes translational and
% rotational components before determining the step direction (5
% components for linear systems and 6 for others).  The initial guess
% nuclear Hessian is the identity matrix and there is an ASCII interface
% file to input a Hessian from another code.  The automatic generation
% of finite difference Hessians will be available soon.  During the
%  optimization if the step taken is too large e.g., the step causes the
% energy to go up, enough information is stored to allow the optimizer
% to ``backstep'' and correct the step based on information prior to the
% faulty step.


% \section{MIN}

% \begin{verbatim}
%   MIN
% \end{verbatim}

% \verb+MIN+ instructs STEPPER to minimize the energy as a function of the
% geometry of the system. This is the default action for STEPPER.


\section{Calculations Performed by the STEPPER Module}

The STEPPER module can be used to perform one of two(?) different calculations
for a given system.  The default is for STEPPER to minimize the
energy of the geometry of the system.  STEPPER can also be used to
find the transition state by following the lowest eigenvector of the nuclear
Hessian.  The input to define the action of STEPPER is as follows,

\begin{verbatim}
   <string action default min>
\end{verbatim}

The value \verb+min+ for the string \verb+action+ specifies the default
energy minimization.  Finding the lowest transition state is specified
by entering the value \verb+ts+ for the string \verb+action+.

\Large
**NOTE: I just made the above input line up.  If STEPPER doesn't work this
way, it really should be fixed.  But this is consistent with the input
conventions described for NWChem.***
\normalsize

When the action of STEPPER is to find the transition state, the user
can also have the code track the eigenvector corresponding to a specific
node during a transition state walk.  This is done by specifying the
keyword \verb+TRACK+, using the following input line,

\begin{verbatim}
  TRACK [nmode <integer nmode default 1>]
\end{verbatim}

The keyword \verb+TRACK+ tells STEPPER to track the eigenvector 
corresponding to the integer entered for \verb+nmode+ during a 
transition state walk.  (Note: this input is invalid
for a minimization walk.)  The step is constructed to go up in energy
along the \verb+nmode+ eigenvector and down in all other degrees of
freedom.

\section{Control of the STEPPER Calculation}

% \section{MAXITER}

In most applications, 20 stepper iterations will be sufficient to obtain
the energy minimization.  If a step taken during the
optimization is too large (e.g., the step causes the energy to go up), 
the optimizer will automatically ``backstep'' and correct the step 
based on information prior to the faulty step.  However, the user has 
the option of specifying 
the maximum number of iterations allowed, using the input line,

\begin{verbatim}
  MAXITER <integer maxiter default 20>
\end{verbatim}

The value specified for the integer \verb+maxiter+ defines the maximum 
number of geometry optimization steps.  The
geometry optimization will restart automatically.

%  \section{TS}

% \begin{verbatim}
%   TS
% \end{verbatim}

% \verb+TS+ instructs stepper to find the transition state by following the
% lowest eigenvalue of the nuclear Hessian.

% \section{TRACK}

% \begin{verbatim}
%   TRACK [nmode <integer nmode default 1>]
% \end{verbatim}

% \verb+TRACK+ allows stepper to track the eigenvector corresponding to
% \verb+nmode+ during a transition state walk.  This input is invalid
% for a minimization walk.  The step is constructed to go up in energy
% along the \verb+nmode+ eigenvector and down in all other degrees of
% freedom.  

% \section{TRUST}

The size of steps that can be taken in STEPPER is governed by the degree
to which the calculated values of the eigenvectors can be considered 
reasonably good.  This is the 'trust radius', and has a default value of
0.1, which means ****what?***.  The user has the option of overriding this
default using the keyword \verb+TRUST+, with the following input line,

\begin{verbatim}
  TRUST <real radius default 0.1>
\end{verbatim}

%  This directive overrides the default trust-radius of 0.1.  A larger

The larger the value specified for the variable \verb+radius+, the larger 
the steps that can be taken by STEPPER. Experience has shown that for
larger systems (i.e., those with 20 or more atoms), a value of 0.5 or
greater should be entered for \verb+radius+.

\section{Convergence Criteria for the STEPPER Calculations}

Two convergence criteria can be specified explicitly for the 
STEPPER calculations.  The keyword \verb+CONVGG+ allows the user to
specify the the convergence tolerence for the gradient norm for
all degrees of freedom.  The input line is of the following form,

% \section{CONVGG}
\begin{verbatim}
   CONVGG <real convgg default 1.0d-04>
\end{verbatim}

The entry for the real variable \verb+convgg+ should be approximately 
equal to the square root of the energy convergence tolerance.

The energy convergence tolerance is the convergence criterion for the 
energy difference in the geometry optimization in STEPPER.  It can be
specified by input using a line of the following form,

% \section{CONVGE}
\begin{verbatim}
   CONVGE <real convge default 1.0d-08>
\end{verbatim}


\section{Initial Guess for Nuclear Hessian}

{\bf BROKEN BUT EASILY FIXED?}

\Large
**will it be fixed for the release?  If not, this section shuold
be deleted.  What is this, anyway?  Another 'action' STEPPER can do?
Or an option for any calculation?
\normalsize

