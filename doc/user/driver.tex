\label{sec:driver}

The DRIVER module is one of two drivers to perform a geometry
optimization function on the molecule defined by input using the
\verb+GEOMETRY+ directive (see Section \ref{sec:geom}).  Geometry
optimization is either an energy minimization or a transition state
optimization. DRIVER is selected by default out of the two available
modules to perform geometry optimization.  Selection of the other
optimization driver STEPPER may be made through the use of a SET
directive : ( SET OPT:DRIVER F ) .  Optional input for this module is
specified within the compound directive,

\begin{verbatim}
  DRIVER 
    ...
  END
\end{verbatim}

The algorithm programmed in DRIVER is a quasi-newton optimization
with line searches and approximate energy hessian updates.

Input specified for the DRIVER module may appear anywhere in the input
file.  In the current version of NWChem, DRIVER uses  geometries that
are defined via Cartesian coordinates or internal coordinates. The
latter may be user-defined with the ZMT input or may be automatically
defined as a result of the AUTOZ option.  The initial guess nuclear
Hessian is the identity matrix and there is an ASCII interface file to
input a Hessian from another code.  The automatic generation of finite
difference Hessians will be available soon. When internal coordinates
are selected an appropriate initial hessian matrix is automatically
created to match the internal coordinate definition.

Execution of the DRIVER module calculation is invoked with a
\verb+TASK+ directive (see Section \ref{sec:task}).

No input is required for DRIVER.  If no input is present the default
actions are to minimize the energy as a function of the geometry with a
maximum of 20 stepper iterations.

\begin{verbatim}
  NTPOPT  <integer nptopt  default 20>
\end{verbatim}

The value specified for the integer \verb+ntpopt+ defines the maximum 
number of geometry optimization steps.  

\begin{verbatim}
  CVGOPT  <real cvgopt  default 0.0008>
\end{verbatim}
 
The value specified for the real \verb+cvgopt+ defines the convergence
threshold of the optimization algorithm. The convergence criterion is
the largest component of the energy gradient for the coordinates used
in the optimization ( cartesian or internal coordinates ).

\begin{verbatim}
  LINOPT  <integer linopt  default 10>
\end{verbatim}

The value specified for the integer \verb+linopt+ defines the maximum 
number of energy points during any linear search. 

\begin{verbatim}
  INHESS  <integer inhess  default 0>
\end{verbatim}

\begin{verbatim}
  MODUPD  <integer modupt  default 1>
\end{verbatim}

The value specified for the integer \verb+modupd+ defines the hessian 
update algorithm, Fletcher-Powell update ( modupd 0 ) or
Broyden-Fletcher-Goldfar-Shanno update ( modupd 1 )

\begin{verbatim}
  MODSAD  <integer modsad  default 0>
\end{verbatim}

The value specified for the integer \verb+modsad+ defines the type   
of optimization to be performed, an energy minimization ( modsad 0 )
or a transition state optimization ( modsad 1 ).


\begin{verbatim}
  MODDIR  <integer moddir  default 1>
\end{verbatim}

The value specified for the integer \verb+moddir+ defines    
which normal mode of displacement to follow initially in the
transition state search.
