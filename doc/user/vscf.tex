%
% $Id: vscf.tex,v 1.2 2007-04-06 16:21:09 windus Exp $
%
\label{sec:vscf}

The VSCF module can be used to calculate the anharmonic contributions to the 
vibrational modes of the molecule of interest. Energies are calculated on a 
one-dimensional grid along each normal mode, on a two-dimensional grid along
each pair of normal modes, and optionally on a three-dimensional grid along
each triplet of normal modes. These energies are then used to calculate the
vibrational nuclear wavefunction at an SCF- (VSCF) and MP2-like (cc-VSCF) level
of theory. 

VSCF can be used at all levels of theory, SCF and correlated methods, and DFT.  
For correlated methods, only the SCF level dipole is evaluated and used to 
calculate the IR intensity values. 

The VSCF module is started when the task directive
\verb+TASK <theory> vscf+ is defined in the user input file. The input 
format has the form:

\begin{verbatim}
  VSCF
    [coupling <string couplelevel default "pair">]
    [ngrid    <integer default 16 >]
    [iexcite  <integer default 1  >]
    [vcfct    <real    default 1.0>]
  END
\end{verbatim}

The order of coupling of the harmonic normal modes included in the calculation
is controlled by the specifying:

\begin{verbatim}
    coupling <string couplelevel default "pair">
\end{verbatim}

For \verb+coupling=diagonal+ a one-dimensional grid along each normal mode is computed.
For \verb+coupling=pair+ a two-dimensional grid along each pair of normal modes is computed.
For \verb+coupling=triplet+ a three-dimensional grid along each triplet of normal modes is computed.

The number of grid points along each normal mode, or pair of modes can be defined 
by specifying:

\begin{verbatim}
    ngrid <integer default 16>
\end{verbatim}

This VSCF module by default calculates the ground state (v=0), but can also calculate excited
states (such as v=1). The number of excited states calculated is defined by specifying:

\begin{verbatim}
    iexcite <integer default 1>
\end{verbatim}

With \verb+iexcite=1+ the fundamental frequencies are calculated.
With \verb+iexcite=2+ the first overtones are calculated.
With \verb+iexcite=3+ the second overtones are calculated.

In certain cases the pair coupling potentials can become larger than those for a single 
normal mode. In this case the pair potentials need to be scaled down. The scaling factor 
used can be defined by specifying:

\begin{verbatim}
    vcfct <real default 1.0>
\end{verbatim}


