\label{sec:nwARGOS}
%\newcommand{\mc}[3]{\multicolumn{#1}{#2}{#3}}
\newcommand{\mc}[3]{\mbox{\bf #3}}
\newcommand{\vb}[1]{\mbox{\verb.#1.}}
\newcommand{\none}{\multicolumn{2}{|c|}{ }}
%%%%%%%\renewcommand{\thetable}{\Roman{table}}
\newcommand{\mcc}[1]{\multicolumn{2}{c}{#1}}
\def\bmu{\mbox{\boldmath $\mu$}}
\def\bE{\mbox{\bf E}}
\def\br{\mbox{\bf r}}
\def\tT{\tilde{T}}
\def\t{\tilde{1}}
\def\ip{i\prime}
\def\jp{j\prime}
\def\ipp{i\prime\prime}
\def\jpp{j\prime\prime}
\def\etal{{\sl et al.}}
\def\nwchem{{\bf NWChem}}
\def\nwargos{{\bf nwargos}}
\def\prepare{{\bf prepare}}
\def\nwtop{{\bf nwtop}}
\def\nwrst{{\bf nwrst}}
\def\nwsgm{{\bf nwsgm}}
\def\argos{{\bf ARGOS}}

\section{Introduction}

\subsection{Spacial decomposition}
The molecular dynamics module of \nwchem\ uses a distribution of data
based on a spacial decomposition of the molecular system, offering
an efficient parallel implementation in terms of both memory 
requirements and communication costs, especially for simulations of 
large molecular systems.

Inter-processor communication using the global array tools and the
design of a data structure allowing distribution based on spacial
decomposition are the key elements in taking advantage of
the distribution of memory requirements and computational work with
minimal communication.

In the spacial decomposition approach, the physical simulation
volume is divided into rectangular boxes, each of which is
assigned to a processor. Depending on the conditions of the 
calculation and the number of available processors, each processor 
contains one or more of these spacially grouped boxes.
The most important aspects of this decomposition are the dependence 
of the box sizes and communication cost on the number of processors 
and the shape of the boxes, the frequent reassignment of atoms to 
boxes leading to a fluctuating number of atoms per box, and the 
locality of communication which is the main reason for the efficiency 
of this approach for very large molecular systems.

To improve efficiency, molecular systems are broken up into separately
treated solvent and solute parts. Solvent molecules are assigned to
the domains according to their center of geometry and are always owned
by a one node. This avoids solvent--solvent bonded interactions
crossing node boundaries.  Solute molecules are broken up into
segments, with each segment assigned to a processor based on its
center of geometry.  This limits the number of solute bonded
interactions that cross node boundaries.  The processor to which a
particular box is assigned is responsible for the calculation of all
interactions between atoms within that box.  For the calculation of
forces and energies in which atoms in boxes assigned to different
processors are involved, data are exchanged between processors. The
number of neighboring boxes is determined by the size and shape of the
boxes and the range of interaction. The data exchange that takes place
every simulation time step represents the main communication
requirements.  Consequently, one of the main efforts is to design
algorithms and data structures to minimize the cost of this
communication. However, for very large molecular systems, memory
requirements also need to be taken into account.

To compromise between these requirements exchange of data is performed
in successive point to point communications rather than using the
shift algorithm which reduces the number of communication calls
for the same amount of communicated data.

For inhomogeneous systems, the computational load of evaluating 
atomic interactions will generally differ between box pairs. 
This will lead to load imbalance between processors.
Two algorithms have been implemented that allow for dynamically 
balancing the workload of each processor.
One method is the dynamic resizing of boxes such that boxes gradually
become smaller on the busiest node, thereby reducing the computational
load of that node. Disadvantages of this method are that the 
efficiency depends on the solute distribution in the simulation volume
and the redistribution of work depends on the number of nodes which
could lead to results that depend on the number of nodes used.
The second method is based on the dynamic redistribution of intra-node
box-box interactions. This method represents a more coarse load
balancing scheme, but does not have the disadvantages of the box
resizing algorithm. For most molecular systems the box pair
redistribution is the more efficient and preferred  method.

The description of a molecular system consists of static and dynamic
information. The static information does not change during a
simulation and includes items such as connectivity, excluded and third
neighbor lists, equilibrium values and force constants for all
bonded and non-bonded interactions. The static information is called
the topology of the molecular system, and is kept on a separate
topology file. The dynamic information includes coordinates and
velocities for all atoms in the molecular system, and is kept in a
so-called restart file.

The \nwchem\ molecular dynamics module is the parallel implementation
of the vectorized code \argos\ developed at the University of Houston.

\subsection{Topology}
\label{sec:nwatopology}
The static information about a molecular system that is needed for
a molecular simulation is provided to the simulation module in a
topology file. 
Items in this file include, among many other things, 
a list of atoms, their non-bonded parameters for van der Waals and
electrostatic interactions, and the complete connectivity in terms
of bonds, angles and dihedrals.

In molecular systems, a distinction is made between 
{\it solvent} and {\it solute}, which are treated separately.
A solvent molecule is defined only once in the topology file,
even though many solvent molecules usually are included in the
actual molecular system. In the current implementation only one 
solvent can be defined. Everything that is not solvent in the 
molecular system is solute. Each solute atom in the system must 
be explicitly defined in the topology. 

Molecules are defined in terms of one or more {\it segment}s. 
Typically, repetitive parts of a molecule are each defined as a single
segment, such as the amino acid residues in a protein. 
Segments  can be quite complicated to define and are, therefore, 
collected in a set of database files. 
The definition of a molecular system in terms of segments is a
{\it sequence}.

Topology files are created using the \prepare\ module.

\subsection{Files}
\label{sec:nwafilenames}

File names used have the form \verb+$system$_$calc$.$ext$+, with
exception of the topology file (Section \ref{sec:nwatopology}), which is named 
\verb+$system$.top+.
Anything that refers to the definition of the chemical system can be used
for \verb+$system$+, as long as no periods or underlines are used.
The identifier \verb+$calc$+ can be anything that refers to the type of 
calculation to be performed for the system with the topology defined.
This file naming convention allows for the creation of a single
topology file \verb+$system$.top+ that can be used for a number of 
different calculations, each identified with a different \verb+$calc$+.
For example, if {\tt crown.top} is the name of the topology file for
a crown ether, {\tt crown\_em}, {\tt crown\_md}, {\tt crown\_ti} could
be used with appropriate extensions for the filenames for energy
minimization, molecular dynamics simulation and multi-configuration
thermodynamic integration, respectively. All of these calculations
would use the same topology file {\tt crown.top}.

\label{sec:nwaextensions}

The extensions \verb+<ext>+ identify the kind of information on a file,
and are pre-determined. 
The complete list of extensions is
\begin{tabbing}
xxxxxxx\=\kill
{\bf acf} \> free energy correlation data file\\
{\bf cnv} \> free energy convergence file\\
{\bf coo} \> coordinate trajectory file\\
{\bf day} \> dayfile\\
{\bf dbg} \> debug file\\
{\bf dre} \> distance restraints file\\
{\bf emt} \> minimization trajectory file\\
{\bf fet} \> free energy step contribution file\\
{\bf frg} \> fragment file\\
{\bf gib} \> free energy data file\\
{\bf mri} \> free energy multiple run input file\\
{\bf mro} \> free energy multiple run output file\\
{\bf nw}  \> \nwchem\ input file\\
{\bf nwout}  \> \nwchem\ output file\\
{\bf out} \> molecular dynamics output file\\
{\bf pdb} \> PDB formatted coordinate file\\
{\bf prp} \> property file\\
{\bf qrs} \> quenched restart file, resulting from an energy minimization\\
{\bf rdf} \> radial distribution function output file\\
{\bf rdi} \> radial distribution function input file\\
{\bf rst} \> restart file, used to start, restart or continue a simulation \\
{\bf seq} \> sequence file, describing the system in segments\\
{\bf sco} \> solute coordinate trajectory file\\
{\bf sgm} \> segment file, describing segments\\
{\bf slv} \> solvent coordinate file\\
{\bf svl} \> solute velocity trajectory file\\
{\bf syn} \> synchronization time file\\
{\bf tst} \> test file\\
{\bf tim} \> timing analysis file\\
{\bf top} \> topology file, contains the static description of a system\\
{\bf vel} \> velocity trajectory file\\
\end{tabbing}

\subsection{Databases}
Database file used by the \prepare\ module are found in directories with name
\verb+$ffield$_$level$, where \verb+$ffield$+ is any of the
supported force fields (Section \ref{sec:nwaforcefields}). 
The source of the data is identified by \verb+$level$+, and can be 
\begin{center}
\begin{tabular}{lll}
\hline
level   & Description                 & Availability \\
{\bf s} & original published data     & public       \\
{\bf x} & additional published data   & public       \\
{\bf u} & user preferred data         & private      \\
{\bf t} & user defined temporary data & private    \\
\hline
\end{tabular}
\end{center}

Typically, only the level {\bf s} and {\bf x} databases are publicly 
available. 
The user is responsible for the private level {\bf u} and {\bf t}
database files. When the \prepare\ module scans the databases, the priority
is {\bf t}$>${\bf u}$>${\bf x}$>${\bf s}$>$.

The extension \verb+<ext>+ defines the type of database file within each
database directory. The complete list of extensions is
\begin{tabbing}
xxxxx\=\kill
{\bf frg} \> fragments\\
{\bf par} \> parameters\\
{\bf seq} \> sequences\\
{\bf sgm} \> segments\\
\end{tabbing} 

The paths of the different database directoriess should be defined in a file 
{\tt .nwchemrc} in a user's home directory, and provides the user the
option to select which database files are scanned.

\subsection{Force fields}
\label{sec:nwaforcefields}
Force fields recognized are
\begin{center}
\begin{tabular}{ll}
\hline
Keyword      & Force field   \\
{\tt amber}  & AMBER95       \\
{\tt charmm} & CHARMM        \\
%{\tt cvff}   & CVFF          \\
%{\tt gromos} & GROMOS87      \\
%{\tt oplsa}  & OPLS/AMBER3.0 \\
%{\tt oplsg}  & OPLS/GROMOS87 \\
\hline
\end{tabular}
\end{center}  

\section{Format of fragment files}
Fragment files contain the basic information needed to specify all
interactions that need to be considered in a molecular simulation.
The format of the fragment files is described in Table \ref{tbl:nwafrag}.
Normally these files are created by the \prepare\ module. Manual
editing is needed when, for example, the \prepare\ module could not
complete atom typing, or when modified charges are required. 

\begin{table}[htbp]
\begin{center}
\begin{tabular}{p{15mm}p{12mm}l}
\hline\hline
Card & Format & Description \\ \hline
I-1-1  & a1     & \$ to identify the start of a fragment \\ % $ for emacs
I-1-2  & a10    & name of the fragment, the tenth character\\
       &        & N: identifies beginning of a chain\\
       &        & C: identifies end of a chain\\
       &        & blank: identifies chain fragment\\
       &        & M: identifies an integral molecule\\
\hline
I-2-1  & i5     & number of atoms in the fragment\\ 
\hline
\mc{3}{l}{For each atom one deck II} \\
\hline
II-1-1  & i5     & atom sequence number \\
II-1-2  & a6     & atom name \\
II-1-3  & a5     & atom type \\
II-1-4  & a1     & dynamics type\\
        &        & \verb+ + : normal\\
        &        & \verb+D+ : dummy atom\\
        &        & \verb+S+ : solute interactions only\\
        &        & \verb+Q+ : quantum atom\\
        &        & other : intramolecular solute interactions only\\
II-1-5  & i5     & link number\\
        &        & 0: no link\\
        &        & 1: first atom in chain\\
        &        & 2: second atom in chain\\
        &        & 3 and up: other links\\
II-1-6  & i5     & environment type\\
        &        & 0: no special identifier\\
        &        & 1: planar, using improper torsion\\
        &        & 2: tetrahedral, using improper torsion\\
        &        & 3: tetrahedral, using improper torsion\\
        &        & 4: atom in aromatic ring\\
II-1-7  & i5     & charge group\\
II-1-8  & i5     & polarization group\\
II-1-9  & f12.6  & atomic partial charge\\
II-1-10 & f12.6  & atomic polarizability\\
\hline
\mc{3}{l}{Any number of cards in deck III to specify complete 
connectivity} \\
\hline
III-1-1  & 16i5   & connectivity, duplication allowed\\ 
\hline\hline
\end{tabular}
\caption{The format of fragment files.\label{tbl:nwafrag}}
\end{center}
\end{table}

\section{Creating segment files}
\label{sec:nwanwsgm}
The \prepare\ module is used to generate segment files 
from corresponding fragment files. A segment file contains all
information for the calculation of bonded and non-bonded interactions
for a given chemical system using a specific force field.

Which atoms form a fragment is specified in the coordinate file,
currently only in PDB format.One important
restriction is that dihedral interactions may only involve atoms on at
most two segments. The segment entries define three sets of parameters
for each interaction. 

Free energy perturbations can be performed using set 1 for the 
generation of the ensemble while using sets 2 and/or 3
as perturbations. Free energy multiconfiguration thermodynamic
integration and multistep thermodynamic perturbation calculations are
performed by gradually changing the interactions in the system from
parameter set 2 to parameter set 3. These modifications can be 
edited into the segment files manually, or introduced directly into
the topology file using the \verb+modify+ commands in the input for
the \prepare\ module.

The format of a segment is
described in Tables \ref{tbl:nwaseg1}--\ref{tbl:nwaseg6}.

\begin{table}[htbp]
\begin{center}
\begin{tabular*}{150mm}{p{15mm}p{12mm}l}
\hline\hline
Deck  & Format & Description \\ \hline
I-1-1 & a1     & \$ to identify the start of a segment \\ %$ for emacs
I-1-2 & a10    & name of the segment, the tenth character\\
      &        & N: identifies beginning of a chain\\
      &        & C: identifies end of a chain\\
      &        & blank: identifies chain fragment\\
      &        & M: identifies an integral molecule\\
I-2-1 & i5     & number of atoms in the segment\\
I-2-2 & i5     & number of bonds in the segment\\
I-2-3 & i5     & number of angles in the segment\\
I-2-4 & i5     & number of proper dihedrals in the segment\\
I-2-5 & i5     & number of improper dihedrals in the segment\\
\hline
\end{tabular*}
\caption{Segment file format, table 1 of 6.\label{tbl:nwaseg1}}
\end{center}
\end{table}

\begin{table}[htbp]
\begin{center}
\begin{tabular*}{150mm}{p{15mm}p{12mm}l}
\hline\hline
Deck & Format & Description \\ \hline
\mc{3}{l}{For each atom one deck II} \\
II-1-1  & i5     & atom sequence number \\
II-1-2  & a6     & atom name \\
II-1-3  & a5     & atom type, generic set 1 \\
II-1-4  & a1     & dynamics type\\
        &        & \verb+ + : normal\\
        &        & \verb+D+ : dummy atom\\
        &        & \verb+S+ : solute interactions only\\
        &        & \verb+Q+ : quantum atom\\
        &        & other : intramolecular solute interactions only\\
II-1-4  & a5     & atom type, generic set 2 \\
II-1-5  & a1     & dynamics type\\
        &        & \verb+ + : normal\\
        &        & \verb+D+ : dummy atom\\
        &        & \verb+S+ : solute interactions only\\
        &        & \verb+Q+ : quantum atom\\
        &        & other : intramolecular solute interactions only\\
II-1-6  & a5     & atom type, generic set 3 \\
II-1-7  & a1     & dynamics type\\
        &        & \verb+ + : normal\\
        &        & \verb+D+ : dummy atom\\
        &        & \verb+S+ : solute interactions only\\
        &        & \verb+Q+ : quantum atom\\
        &        & other : intramolecular solute interactions only\\
II-1-8  & i5     & charge group\\
II-1-9  & i5     & polarization group\\
II-1-10 & i5     & link number\\
II-1-11 & i5     & environment type\\
        &        & 0: no special identifier\\
        &        & 1: planar, using improper torsion\\
        &        & 2: tetrahedral, using improper torsion\\
        &        & 3: tetrahedral, using improper torsion\\
        &        & 4: atom in aromatic ring\\
II-2-1  & f12.6  & atomic partial charge in e, set 1\\
II-2-2  & f12.6  & atomic polarizability/$4\pi\epsilon_o$ in nm$^3$, set 1\\
II-2-3  & f12.6  & atomic partial charge in e, set 2\\
II-2-4  & f12.6  & atomic polarizability/$4\pi\epsilon_o$ in nm$^3$, set 2\\
II-2-5  & f12.6  & atomic partial charge in e, set 3\\
II-2-6  & f12.6  & atomic polarizability/$4\pi\epsilon_o$ in nm$^3$, set 3\\
\hline
\end{tabular*}
\caption{Segment file format, table 2 of 6.\label{tbl:nwaseg2}}
\end{center}
\end{table}

\begin{table}[htbp]
\begin{center}
\begin{tabular*}{150mm}{p{15mm}p{12mm}l}
\hline\hline
Deck & Format & Description \\ \hline
\mc{3}{l}{For each bond a deck III} \\
III-1-1 & i5     & bond sequence number \\
III-1-2 & i5     & bond atom i \\
III-1-3 & i5     & bond atom j \\
III-1-4 & i5     & bond type \\
        &        & 0: harmonic\\
        &        & 1: constrained bond\\
III-1-5 & i5     & bond parameter origin\\
        &        & 0: from database, next card ignored \\
        &        & 1: from next card\\
III-2-1 & f12.6  & bond length in nm, set 1\\
III-2-2 & e12.5  & bond force constant in kJ nm$^2$ mol$^{-1}$, set 1 \\
III-2-3 & f12.6  & bond length in nm, set 2\\
III-2-4 & e12.5  & bond force constant in kJ nm$^2$ mol$^{-1}$, set 2 \\
III-2-5 & f12.6  & bond length in nm, set 3\\
III-2-6 & e12.5  & bond force constant in kJ nm$^2$ mol$^{-1}$, set 3 \\
\hline
\end{tabular*}
\caption{Segment file format, table 3 of 6.\label{tbl:nwaseg3}}
\end{center}
\end{table}

\begin{table}
\begin{center}
\begin{tabular*}{150mm}{p{15mm}p{12mm}l}
\hline\hline
Deck & Format & Description \\ \hline
\mc{3}{l}{For each angle a deck IV} \\
IV-1-1 & i5     & angle sequence number \\
IV-1-2 & i5     & angle atom i \\
IV-1-3 & i5     & angle atom j \\
IV-1-4 & i5     & angle atom k \\
IV-1-5 & i5     & angle type \\
       &        & 0: harmonic\\
IV-1-6 & i5     & angle parameter origin\\
       &        & 0: from database, next card ignored \\
       &        & 1: from next card\\
IV-2-1 & f12.6  & angle in radians, set 1\\
IV-2-2 & e12.5  & angle force constant in kJ mol$^{-1}$, set 1 \\
IV-2-3 & f12.6  & angle in radians, set 2\\
IV-2-4 & e12.5  & angle force constant in kJ mol$^{-1}$, set 2 \\
IV-2-5 & f12.6  & angle in radians, set 3\\
IV-2-6 & e12.5  & angle force constant in kJ mol$^{-1}$, set 3 \\
\hline
\end{tabular*}
\caption{Segment file format, table 4 of 6.\label{tbl:nwaseg4}}
\end{center}
\end{table}

\begin{table}[htbp]
\begin{center}
\begin{tabular*}{150mm}{p{15mm}p{12mm}l}
\hline\hline
Deck & Format & Description \\ \hline
\mc{3}{l}{For each proper dihedral a deck V} \\
V-1-1 & i5     & proper dihedral sequence number \\
V-1-2 & i5     & proper dihedral atom i \\
V-1-3 & i5     & proper dihedral atom j \\
V-1-4 & i5     & proper dihedral atom k \\
V-1-5 & i5     & proper dihedral atom l \\
V-1-6 & i5     & proper dihedral type \\
      &        & 0: $C\cos(m\phi-\delta)$\\
V-1-7 & i5     & proper dihedral parameter origin\\
      &        & 0: from database, next card ignored \\
      &        & 1: from next card\\
V-2-1 & i5     & multiplicity, set 1\\
V-2-2 & f12.6  & proper dihedral in radians, set 1\\
V-2-3 & e12.5  & proper dihedral force constant in kJ mol$^{-1}$, set 1 \\
V-2-4 & i5     & multiplicity, set 2\\
V-2-5 & f12.6  & proper dihedral in radians, set 2\\
V-2-6 & e12.5  & proper dihedral force constant in kJ mol$^{-1}$, set 2 \\
V-2-7 & i5     & multiplicity, set 3\\
V-2-8 & f12.6  & proper dihedral in radians, set 3\\
V-2-9 & e12.5  & proper dihedral force constant in kJ mol$^{-1}$, set 3 \\
\hline
\end{tabular*}
\caption{Segment file format, table 5 of 6.\label{tbl:nwaseg5}}
\end{center}
\end{table}

\begin{table}[htbp]
\begin{center}
\begin{tabular*}{150mm}{p{15mm}p{12mm}l}
\hline\hline
Deck & Format & Description \\ \hline
\mc{3}{l}{For each improper dihedral a deck VI} \\
VI-1-1 & i5     & improper dihedral sequence number \\
VI-1-2 & i5     & improper dihedral atom i \\
VI-1-3 & i5     & improper dihedral atom j \\
VI-1-4 & i5     & improper dihedral atom k \\
VI-1-5 & i5     & improper dihedral atom l \\
VI-1-6 & i5     & improper dihedral type \\
       &        & 0: harmonic\\
VI-1-7 & i5     & improper dihedral parameter origin\\
       &        & 0: from database, next card ignored \\
       &        & 1: from next card\\
VI-2-1 & f12.6  & improper dihedral in radians, set 1\\
VI-2-2 & e12.5  & improper dihedral force constant in kJ mol$^{-1}$, set 1 \\
VI-2-3 & f12.6  & improper dihedral in radians, set 2\\
VI-2-4 & e12.5  & improper dihedral force constant in kJ mol$^{-1}$, set 2 \\
VI-2-5 & f12.6  & improper dihedral in radians, set 3\\
VI-2-6 & e12.5  & improper dihedral force constant in kJ mol$^{-1}$, set 3 \\
\hline\hline
\end{tabular*}
\caption{Segment file format, table 6 of 6.\label{tbl:nwaseg6}}
\end{center}
\end{table}


\section{Creating sequence files}
A sequence file describes a molecular system in terms of segments. This
file is generated by the \prepare\ module for the molecular system
provided on a PDB-formatted coordinate file.
The file format is given in Table \ref{tbl:nwaseq}

\begin{table}[htbp]
\begin{center}
\begin{tabular*}{150mm}{p{15mm}p{12mm}l}
\hline\hline
Card & Format & Description \\ \hline
I-1-1  & a1     & \$ to identify the start of a sequence \\ %$ for emacs
I-1-2  & a10    & name of the sequence\\
\mc{3}{l}{Any number of cards 1 and 2 in deck II to specify the system} \\
II-1-1 & i5     & segment number\\
II-1-2 & a10    & segment name, last character will be determined from chain\\
II-1-3 & i5     & link segment 1, if blank previous segment in chain\\
II-1-4 & i3     & link atom, if blank link atom 2\\
II-1-5 & i5     & link segment 2, if blank next segment in chain\\
II-1-6 & i3     & link atom, if blank link atom 1\\
II-1-7 & i5     & link segment 3\\
II-1-8 & i3     & link atom in link segment 3\\
II-1-9 & i5     & link segment 4\\
II-1-0 & i3     & link atom in link segment 4\\
II-1-1 & i5     & link segment 5\\
II-1-2 & i3     & link atom in link segment 5\\
II-1-3 & i5     & link segment 6\\
II-1-4 & i3     & link atom in link segment 6\\
II-1-5 & i5     & link segment 7\\
II-1-6 & i3     & link atom in link segment 7\\
II-1-7 & i5     & link segment 8\\
II-1-8 & i3     & link atom in link segment 8\\
II-1-9 & i5     & link segment 9\\
II-1-0 & i3     & link atom in link segment 9\\
II-1-1 & i5     & link segment 10\\
II-1-2 & i3     & link atom in link segment 10\\
II-2-1 & a      & \verb+break+ to identify a break in the molecule chain\\
II-2-1 & a      & \verb+molecule+ to identify the end of a solute molecule\\
II-2-1 & a      & \verb+fraction+ to identify the end of a solute fraction\\
II-2-1 & a5     & \verb+link + to specify a link\\
II-2-2 & i5     & segment number of first link atom\\
II-2-3 & a4     & name of first link atom \\
II-2-4 & i5     & segment number of second link atom\\
II-2-5 & a4     & name of second link atom \\
II-2-1 & a      & \verb+solvent+ to identify solvent definition on next card\\
II-2-1 & a      & \verb+stop+ to identify the end of the sequence\\
II-2-1 & a6     & \verb+repeat+ to repeat next $ncard$ cards $ncount$
times\\
II-2-2 & i5     & number of cards to repeat ($ncards$)\\
II-2-3 & i5     & number of times to repeat cards ($ncount$)\\
\mc{3}{l}{Any number of cards in deck II to specify the system} \\
\hline\hline
\end{tabular*}
\caption{Sequence file format.\label{tbl:nwaseq}}
\end{center}
\end{table}

\section{Creating topology files}
\label{sec:nwanwtop}

The topology (Section \ref{sec:nwatopology}) describes all static information
that describes a molecular system. This includes the connectivity in
terms of bond-stretching, angle-bending and torsional interactions, as well as
the non-bonded van der Waals and Coulombic interactions.

The topology of a molecular system is generated by the \prepare\ module
from the sequence in terms of segments as specified on the PDB file.
For each unique segment specified in this file the 
segment database directories are searched for the segment definition. 
For segments not found in one of the database directories a segment definition
is generated in the temporary directory if a fragment file was found.
If a fragment file could not be found, it is generated by the \prepare\ module
base on what is found on the PDB file.

When all segments are found or created, the parameter substitutions are 
performed, using force field parameters taken from the parameter 
databases. After all lists have been generated the
topology is written to a local topology file \verb+$system$.top+.

\section{Creating restart files}
\label{sec:nwanwrst}

Restart files contain all dynamical information about a molecular
system and are created by the \prepare\ module if a topology file
is available. The \prepare\ module will automatically generate
coordinates for hydrogen atoms and monatomic counter ions
not found on the PDB formatted coordinate file, if no fragment or
segment files were generated using that PDB file.

The \prepare\ module has a number of other optional input command,
including solvation.

\section{Molecular simulations}
The type of molecular dynamics simulation is specified by the
\nwchem\ task directive.
\begin{verbatim}
task md [ energy || optimize || dynamics || thermodynamics ]
\end{verbatim}
where the theory keyword {\tt md} specifies use of the molecular
dynamics module, and the operation keyword is one of
\begin{description}
\item
{\tt energy} for single configuration energy evaluation
\item
{\tt optimize} for energy minimization
\item
{\tt dynamics} for molecular dynamics simulations and single step
thermodynamic perturbation free energy molecular dynamics simulations
\item
{\tt thermodynamics} for combined multi-configuration thermodynamic
integration and multiple step thermodynamic perturbation free
energy molecular dynamics simulations.
\end{description}

\section{System specification}
The chemical system for a calculation is specified in the topology
and restart files. These files should be created using the utilities
\nwtop\ and \nwrst\ before a simulation can be performed.
The names of these files are determined from the \nwchem\ \verb+start+ 
directive.
There is no default. If the \verb+rtdb+ name is given as {\tt system\_calc},
the topology file used is {\tt system.top}, while all other files
are named {\tt system\_calc.ext}.

\section{Parameter set}
\begin{description}

\item
\begin{verbatim}
set <integer iset>
\end{verbatim}
specifies the use of parameter set \verb+<iset>+ for the 
molecular dynamics simulation. 
The topology file contains three separate parameters sets that can 
be used. The default for \verb+<iset>+ is 1.

\item
\begin{verbatim}
pset <integer isetp1> [<integer isetp2>]
\end{verbatim}
specifies the parameter sets to be used as perturbation potentials
in single step thermodynamic perturbation free energy evaluations,
where \verb+<isetp1>+ specifies the first perturbation parameter set and
\verb+<isetp2>+ specifies the second perturbation parameter set. Legal
values for \verb+<isetp1>+ are 2 and 3. Legal value for \verb+<isetp2>+ is
3, in which case \verb+<isetp1>+ can only be 2. If specified, \verb+<iset>+
is automatically set to 1.
\end{description}

\section{Energy minimization algorithms}
The energy minimization of the system as found in the restart file 
is performed with the following directives. If both are specified,
steepest descent energy minimization precedes conjugate gradient 
minimization.

%\begin{description}
\begin{description}
\item
\begin{verbatim}
sd <integer msdit> [init <real dx0sd>] [min <real dxsdmx>] \
                   [max <real dxmsd>] 
\end{verbatim}
specifies the variables for steepest descent energy minimizations,
where \verb+<msdit>+ is the maximum number of steepest descent steps taken,
for which the default is 100, \verb+<dx0sd>+ is the initial step size in nm
for which the default is 0.001, \verb+<dxsdmx>+ is the threshold for the
step size in nm for which the default is 0.0001, and \verb+<dxmsd>+ is the
maximum allowed step size in nm for which the default is 0.05.
\item
\begin{verbatim}
cg <integer mcgit> [init <real dx0cg>] [min <real dxcgmx>] \
                   [cy <integer ncgcy>]
\end{verbatim}
specifies the variables for conjugate gradient energy minimizations,
where \verb+<mcgit>+ is the maximum number of conjugate gradient steps 
taken, for which the default is 100, \verb+<dx0cg>+ is the initial search
interval size in nm for which the default is 0.001, \verb+<dxcgmx>+ is the 
threshold for the step size in nm for which the default is 0.0001, and 
\verb+<ncgcy>+ is the number of conjugate gradient steps after which the 
gradient history is discarded for which the default is 10. If conjugate
gradient energy minimization is preceded by steepest descent energy 
minimization, the search interval is set to twice the final step of the
steepest descent energy minimization.
\end{description}
%\end{description}

\section{Multi-configuration thermodynamic integration}
The following keywords control free energy difference simulations.
Multi-configuration thermodynamic integrations are always combined
with multiple step thermodynamic perturbations.
\begin{description}

\item
\begin{verbatim}
(forward || reverse) [[<integer mrun> of] <integer maxlam>]
\end{verbatim}
specifies the direction and number of integration steps in free
energy evaluations, with {\tt forward} being the default direction.
\verb+<mrun>+ is the number of ensembles that will be generated in
this calculation, and \verb+<maxlam>+ is the total number of ensembles
to complete the thermodynamic integration. The default value for 
\verb+<maxlam>+ is 21. The default value of \verb+<mrun>+ is the 
value of \verb+<maxlam>+.

\item
\begin{verbatim}
error <real edacq>
\end{verbatim}
specifies the maximum allowed statistical error in each generated
ensemble, where \verb+<edacq>+ is the maximum error allowed in the 
ensemble average derivative of the Hamiltonian with respect to 
$\lambda$ with a default of 5.0 kJ~mol$^{-1}$.

\item
\begin{verbatim}
drift <real ddacq>
\end{verbatim}
specifies the maximum allowed drift in the free energy result,
where \verb+<ddacq>+ is the maximum drift allowed in the
ensemble average derivative of the Hamiltonian with respect to 
$\lambda$with a default of 5.0 kJ~mol$^{-1}$ps$^{-1}$.

\item
\begin{verbatim}
factor <real fdacq>
\end{verbatim}
specifies the maximum allowed change in ensemble size
where \verb+<fdacq>+ is the minimum size of an ensemble relative to the
previous ensemble in the calculation with a default value of 0.75.

\item
\begin{verbatim}
decomp
\end{verbatim}
specifies that a free energy decomposition is to be carried out.
Since free energy contributions are path dependent, results from a
decomposition analysis can not be unambiguously interpreted, and
the default is not to perform this decomposition.

\item
\begin{verbatim}
sss [delta <real delta>]
\end{verbatim}
specifies that atomic non-bonded interactions describe a dummy atom
in either the initial or final state of the thermodynamic calculation
will be calculated using separation-shifted scaling, where \verb+<delta>+ 
is the separation-shifted scaling factor with a default of 0.075 nm$^2$.
This scaling method prevents problems associated with singularities in
the interaction potentials.

\item
\begin{verbatim}
new || renew || extend
\end{verbatim}
specifies the initial conditions for thermodynamic calculations.
{\tt new} indicates that this is an initial mcti calculation, which
is the default. {\tt renew} instructs to obtain the initial
conditions for each $\lambda$ from the {\bf mro}-file from a previous 
mcti calculation, which has to be renamed to an {\bf mri}-file. The
keyword {\tt extend} will extend a previous mcti calculation from the
data read from an {\bf mri}-file.
\end{description}

\section{Time and integration algorithm directives}
Following directives control the integration of the equations of motion.
\begin{description}

\item
\begin{verbatim}
leapfrog || vverlet
\end{verbatim}
specifies the integration algorithm,
where {\tt leapfrog} specifies the default leap frog integration, and
{\tt vverlet} specifies the velocity Verlet integrator.

\item
\begin{verbatim}
equil <integer mequi>
\end{verbatim}
specifies the number of equilibration steps \verb+<mequi>+, with a default
of 100.

\item
\begin{verbatim}
data <integer mdacq> [over <integer ldacq>]]
\end{verbatim}
specifies the number of data gathering steps \verb+<mdacq>+ with a
default of 500. In multi-configuration thermodynamic integrations
\verb+<mequi>+ and \verb+<mdacq>+ are for each of the ensembles, and
variable \verb+<ldacq>+ specifies the minimum number of data gathering steps 
in each ensemble. In regular molecular dynamics simulations \verb+<ldacq>+
is not used. The default value for \verb+<ldacq>+ is the value of \verb+<mdacq>+.

\item
\begin{verbatim}
time <real stime>
\end{verbatim}
specifies the initial time \verb+<stime>+ of a molecular simulation in ps,
with a default of 0.0.

\item
\begin{verbatim}
step <real tstep>
\end{verbatim}
specifies the time step \verb+<tstep>+ in ps, with 0.001 as the default value.
\end{description}

\section{Ensemble selection}
Following directives control the ensemble type.

\begin{description}

\item
\begin{verbatim}
isotherm [<real tmpext>] [trelax <real tmprlx> [<real tmsrlx>]]
\end{verbatim}
specifies a constant temperature ensemble using Berendsen's thermostat,
where \verb+<tmpext>+ is the external temperature with a default of 298.15~K,
and \verb+<tmprlx>+ and \verb+<tmsrlx>+ are temperature relaxation times in ps 
with a default of 0.1. If only \verb+<tmprlx>+ is given the complete system
is coupled to the heat bath with relaxation time \verb+<tmprlx>+. If both
relaxation times are supplied, solvent and solute are independently coupled
to the heat bath with relaxation times \verb+<tmprlx>+ and \verb+<tmsrlx>+,
respectively.

\item
\begin{verbatim}
isobar [<real prsext>] [trelax <real prsrlx> ] \
       [compress <real compr>]
\end{verbatim}
specifies a constant pressure ensemble using Berendsen's piston,
where \verb+<prsext>+ is the external pressure with a default of 1.025~10$^5$ Pa,
\verb+<prsrlx>+ is the pressure relaxation time in ps with a default of 0.5, and
\verb+<compr>+ is the system compressibility in m$^2$N$^{-1}$ with a
default of 4.53E-10.
\end{description}

\section{Velocity reassignments}
Velocities can be periodically reassigned to reflect a certain temperature.
\begin{description}
\item
\begin{verbatim}
vreass <integer nfgaus> <real tgauss>
\end{verbatim}
specifies that velocities will be reassigned every \verb+<nfgaus>+ molecular
dynamics steps, reflecting a temperature of \verb+<tgauss>+~K. The default
is not to reassign velocities, i.e.\ \verb+<nfgaus>+ is 0.
\end{description}

\section{Cutoff radii}
Cutoff radii can be specified for short range and long range interactions.
\begin{description}
\item
\begin{verbatim}
cutoff [short] <real rshort> [long <real rlong>] \
       [qmmm <real rqmmm>]
\end{verbatim}
specifies the short range cutoff radius \verb+<rshort>+, and the long range
cutoff radius \verb+<rlong>+ in nm. If the long range cutoff radius
is larger than the short range cutoff radius the twin range method will
be used, in which short range forces and energies are evaluated every
molecular dynamics step, and long range forces and energies with a
frequency of \verb+<nflong>+ molecular dynamics steps. Keyword
\verb+qmmm+ specifies the radius of the zone around quantum atoms
defining the QM/MM bare charges.
The default value for \verb+<rshort>+, \verb+<rlong>+ and \verb+<rqmmm>+ 
is 0.9~nm.
\end{description}

\section{Polarization}
First order and self consistent electronic polarization models have
been implemented.
\begin{description}
\item
\begin{verbatim}
polar (first || scf [[<integer mpolit>] <real ptol>])
\end{verbatim}
specifies the use of polarization potentials,
where the keyword {\tt first} specifies the first order polarization
model, and {\tt scf} specifies the self consistent polarization field
model, iteratively determined with a maximum of \verb+<mpolit>+
iterations to within a tolerance of \verb+<ptol>+ D in the generated
induced dipoles. The default is not to use polarization models.
\end{description}

\section{External electrostatic field}
\begin{description}

\item
\begin{verbatim}
field <real xfield> [freq <real xffreq>] [vector <real xfvect(1:3)>]
\end{verbatim}
specifies an external electrostatic field,
where \verb+<xfield>+ is the field strength, \verb+<xffreq>+ is the
frequency in MHz and \verb+<xfvect>+ is the external field vector.
\end{description}

\section{Constraints}
Constraints are satisfied using the SHAKE 
coordinate resetting procedure.
\begin{description}

\item
\begin{verbatim}
shake [<integer mshitw> [<integer mshits>]]  \
      [<real tlwsha> [<real tlssha>]]
\end{verbatim}
specifies the use of SHAKE constraints,
where \verb+<mshitw>+ is the maximum number of solvent SHAKE iterations,
and \verb+<mshits>+ is the maximum number of solute SHAKE iterations. If
only \verb+<mshitw>+ is specified, the value will also be used for \verb+<mshits>+.
The default maximum number of iterations is 100 for both.
\verb+<tlwsha>+ is the solvent SHAKE tolerance in nm, and \verb+<tlssha>+ is
the solute SHAKE tolerance in nm. If only \verb+<tlwsha>+ is specified, the
value given will also be used for \verb+<tlssha>+. The default tolerance
is 0.001~nm for both.

\item
\begin{verbatim}
noshake (solvent || solute)
\end{verbatim}
disables SHAKE and treats the bonded interaction according to the force 
field.

\end{description}

\section{Long range interaction corrections}
Long range electrostatic interactions are implemented using the
smooth particle mesh Ewald technique, for neutral periodic cubic systems in
the constant volume ensemble, using pair interaction potentials. Particle-mesh
Ewald long range interactions can only be used in molecular dynamics simulations
using effective pair potentials, and not in free energy simulations, QMD or
QM/MM simulations.

\begin{description}
\item
\begin{verbatim}
pme [grid <integer ng>] [alpha <real ealpha>] \
    [order <integer morder>] [fft <integer imfft>] 
\end{verbatim}
specifies the use of smooth particle-mesh Ewald long range
interaction treatment,
where \verb+ng+ is the number of grid points per dimension, 
\verb+ealpha+ is the Ewald coefficient in nm$^{-1}$, with a default
that leads to a tolerance of $10^{-4}$ at the short range cutoff radius, 
and \verb+morder+ is order of the Cardinal B-spline
interpolation which must be an even number and at least 4 (default
value). A platform specific 3D fast Fourier transform is used, if
available, when \verb+imfft+ is set to 2.
\end{description}

\section{Fixing coordinates}
The solvent or solute part of a system may be fixed or unfixed using
the following keywords. Fixing part of the system will not propagate to
simulations using restart files written. In the commands \verb+fix+
and \verb+unfix+, the keywords \verb+all+, \verb+solvent+,
\verb+solute+ and \verb+non-H+ specify the entire molecular system,
the solvent, the solute and the solute atoms other than hydrogen,
respectively.
\begin{description}
\item
\begin{verbatim}
fix (all || solvent || solute || non-H)
\end{verbatim}
fixes all atoms, solvent molecule, solute atom or solute non-hydrogen atoms,
respectively.
\item
\begin{verbatim}
unfix (all || solvent || solute || non-H)
\end{verbatim}
makes all atoms, solvent molecule, solute atom or solute non-hydrogen atoms,
respectively, dynamic.
\end{description}

\section{Autocorrelation function}
For the evaluation of the statistical error of multi-configuration
thermodynamic integration free energy results a correlated data 
analysis is carried out, involving the calculation of the
autocorrelation function of the derivative of the Hamiltonian with
respect to the control variable $\lambda$.
\begin{description}

\item 
\begin{verbatim}
auto <integer lacf> [fit <integer nfit>] [weight <real weight>]
\end{verbatim}
controls the calculation of the autocorrelation,
where \verb+<lacf>+ is the length of the autocorrelation function, with
a default of 1000, \verb+<nfit>+ is the number of functions used in the
fit of the autocorrelation function, with a default of 15, and
\verb+<weight>+ is the weight factor for the autocorrelation function,
with a default value of 0.0.
\end{description}

\section{Print options}
Keywords that control print to the output file, with extension {\bf out}.
Print directives may be combined to a single directive.

\begin{description}

\item
\begin{verbatim}
print topol [nonbond] [solvent] [solute]
\end{verbatim}
specifies printing the topology information,
where {\tt nonbond} refers to the non-bonded interaction parameters,
{\tt solvent} to the solvent bonded parameters, and {\tt solute} to the
solute bonded parameters. If only {\tt topol} is specified, all
topology information will be printed to the output file.

\item
\begin{verbatim}
print step <integer nfoutp> [extra] [energy]
\end{verbatim}
specifies the frequency \verb+nfoutp+ of printing molecular dynamics step
information to the output file. If the keyword {\tt extra} is specified
additional energetic data are printed for solvent and solute separately.
If the keyword {\tt energy} is specified, information is printed for
all bonded solute interactions.
The default for \verb+nfoutp+ is 0. For molecular dynamics simulations
this frequency is in time steps, and for multi-configuration thermodynamic
integration in $\lambda$-steps.

\item
\begin{verbatim}
print stat <integer nfstat>
\end{verbatim}
specifies the frequency \verb+<nfstat>+ of printing statistical information
of properties that are calculated during the simulation. 
For molecular dynamics simulation
this frequency is in time steps, for multi-configuration thermodynamic
integration in $\lambda$-steps.
\end{description}

\section{Periodic updates}
Following keywords control periodic events during a molecular
dynamics or thermodynamic integration simulation.
Update directives may be combined to a single directive.

\begin{description}
\item
\begin{verbatim}
update pairs <integer nfpair>
\end{verbatim}
specifies the frequency \verb+<nfpair>+ in molecular dynamics steps of 
updating the pair lists. The default for the frequency is 1.
In addition, pair lists are also updated after each step in which
recording of the restart or trajectory files is performed. Updating
the pair lists includes the redistribution of atoms that changed
domain and load balancing, if specified.

\item
\begin{verbatim}
update long <integer nflong>
\end{verbatim}
specifies the frequency \verb+<nflong>+ in molecular dynamics steps 
of updating the long range forces. The default frequency is 1.
The distinction of short range and long range forces is only
made if the long range cutoff radius was specified to be larger
than the short range cutoff radius. Updating the long range forces
is also done in every molecular dynamics step in which the
pair lists are regenerated.

\item
\begin{verbatim}
update center <integer nfcntr> [fraction <integer idscb(1:5)>]
\end{verbatim}
specifies the frequency \verb+<nfcntr>+ in molecular dynamics steps in 
which the center of geometry of the solute(s) is translated to the
center of the simulation volume. The solute fractions determining the
solutes that will be centered are specified by the keyword 
{\tt fraction} and the vector \verb+<idscb>+, with a maximum of 5 entries.
This translation is implemented such that it has no effect on any 
aspect of the simulation. The default is not to center, i.e. nfcntr is
0. The default fraction used to center solute is 1.

\item
\begin{verbatim}
update motion <integer nfslow>
\end{verbatim}
specifies the frequency \verb+<nfslow>+ in molecular dynamics steps of
removing the center of mass motion.

\item
\begin{verbatim}
update rdf <integer nfrdf> [range <real rrdf>] [bins <integer ngl>]
\end{verbatim}
specifies the frequency \verb+<nfrdf>+ in molecular dynamics steps of
calculating contributions to the radial distribution functions.
The default is 0. The range of the radial distribution
functions is given by \verb+<rrdf>+ in nm, with a default of the short
range cutoff radius. Note that radial distribution functions are not
evaluated beyond the short range cutoff radius. The number of
bins in each radial distribution function is given by \verb+<ngl>+, with
a default of 1000.
If radial distribution function are to be
calculated, a {\bf rdi} files needs to be available in which the
contributions are specified as follows.
\begin{center}
\begin{tabular}{lll}
\hline\hline
Card & Format & Description \\ \hline
I-1  & i & Type, 1=solvent-solvent, 2=solvent-solute,
3-solute-solute\\
I-2  & i & Number of the rdf for this contribution\\
I-3  & i & First atom number \\
I-4  & i & Second atom number \\ 
\hline
\end{tabular}
\end{center}
\end{description}

\section{Recording}
The following keywords control recording data to file.
Record directives may be combined to a single directive.

\begin{description}

%\item
%The file format of selected recording files is specified with
%\begin{verbatim}
%record (binary || ascii [ecce || argos])
%\end{verbatim}
%with the default of ascii in ecce readable format.

\item
\begin{verbatim}
record rest <integer nfrest> [keep]
\end{verbatim}
specifies the frequency \verb+<nfrest>+ in molecular dynamics steps
of rewriting the restart file, with extension \verb+rst+. 
For multi-configuration
thermodynamic integration simulations the frequency is in
steps in $\lambda$. The default is not to record. The restart
file is used to start or restart simulations. The keyword {\tt keep}
causes all restart files written to be kept on disk, rather than
to be overwritten.

\item
\begin{verbatim}
record coord <integer nfcoor>
\end{verbatim}
specifies the frequency \verb+<nfcoor>+ in molecular dynamics steps
of writing information to the coordinate file, with extension \verb+coo+. 
For multi-configuration
thermodynamic integration simulations the frequency is in
steps in $\lambda$. The default is not to record.

\item
\begin{verbatim}
record scoor <integer nfscoo>
\end{verbatim}
specifies the frequency \verb+<nfscoo>+ in molecular dynamics steps
of writing information to the solute coordinate file, with extension
\verb+sco+. For multi-configuration
thermodynamic integration simulations the frequency is in
steps in $\lambda$. The default is not to record.

\item
\begin{verbatim}
record veloc <integer nfvelo>
\end{verbatim}
specifies the frequency \verb+<nfvelo>+ in molecular dynamics steps
of writing information to the velocity file, with extension \verb+vel+.
For multi-configuration
thermodynamic integration simulations the frequency is in
steps in $\lambda$. The default is not to record.

\item
\begin{verbatim}
record svelo <integer nfsvel>
\end{verbatim}
specifies the frequency \verb+<nfsvel>+ in molecular dynamics steps
of writing information to the solute velocity file, with extension
\verb+svl+. For multi-configuration
thermodynamic integration simulations the frequency is in
steps in $\lambda$. The default is not to record.

\item
\begin{verbatim}
record prop <integer nfprop>
\end{verbatim}
specifies the frequency \verb+<nfprop>+ in molecular dynamics steps
of writing information to the property file, with extension
\verb+prp+. For multi-configuration
thermodynamic integration simulations the frequency is in
steps in $\lambda$. The default is not to record.

\item
\begin{verbatim}
record mind <integer nfem>
\end{verbatim}
specifies the frequency \verb+<nfem>+ in energy minimization steps of
writing the minimization trajectory to file, with extension \verb+emt+.

\item
\begin{verbatim}
record free <integer nffree>
\end{verbatim}
specifies the frequency \verb+<nffree>+ in multi-configuration
thermodynamic integration steps to record data to the
free energy data file, with extension \verb+gib+.
The default is 1, i.e.\ to record at every $\lambda$.

\item
\begin{verbatim}
record cnv
\end{verbatim}
specifies that free energy convergence data will be written to the
free energy convergence file, with extension \verb+cnv+.

\item
\begin{verbatim}
record acf
\end{verbatim}
specifies that free energy derivative autocorrelation data will be
written to the free energy autocorrelation file, with extension
\verb+acf+.

\item
\begin{verbatim}
record fet
\end{verbatim}
that free energy vs.\ time data will be recorded to the free energy
data file, with extension \verb+fet+.

\item
\begin{verbatim}
record sync <integer nfsync>
\end{verbatim}
specifies the frequency  \verb+<nfsync>+ in molecular dynamics steps
of writing information to the synchronization file, with extension
\verb+syn+. 
The default is not to record.
The information written is the simulation time, the wall clock time
of the previous MD step, the wall clock time of the previous force
evaluation, the total synchronization time, the largest
synchronization time and the node on which the largest synchronization
time was found. The recording of synchronization times is part of the
load balancing algorithm. Since load balancing is only performed when
pair-lists are updated, the frequency \verb+<nfsync>+ is correlated
with the frequency of pair-list updates \verb+<nfpair>+.
\end{description}

\section{Program control options}
\begin{description}
\item
\begin{verbatim}
load [reset] ( none || size [<real factld>] || pairs ||
     (pairs [<integer ldpair>] size [<real factld>]) )
\end{verbatim}
determines the type of dynamic load balancing performed,
where the default is {\tt none}. Load balancing option {\tt size}
is resizing boxes on a node, and {\tt pairs} redistributes the
box-box interactions over nodes. Keyword \verb+reset+ will reset the
load balancing read from the restart file. The level of box resizing
can be influenced with $factld$. The boxes on the busiest node are
resized with a factor
\begin{equation}
\left( 1 - factld * { {T_{sync} \over n_p} - t^{min}_{sync} \over t_{wall}}
\right)^{1\over 3}
\end{equation}
where $T_{sync}$ is the accumulated synchronization time of all nodes,
$n_p$ is the total number of nodes, $t^{min}_{sync}$ is the synchronization
time of the busiest node, and $t_{wall}$ is the wall clock time of the
molecular dynamics step.\\
For the combined load balancing, \verb+ldpair+ is the number of successive pair 
redistribution load balancing steps in which the accumulated synchronization
time increases, before a resizing load balancing step will be attempted.\\
Load balancing is only performed in molecular dynamics steps in which the
pair-list is updated.

\item
\begin{verbatim}
nodes <integer npx> <integer npy> <integer npz>
\end{verbatim}
specifies the distribution of the available nodes over the three 
Cartesian dimensions. The default distribution is chosen such that, 
\verb+<npx>+$*$\verb+<npy>+$*$\verb+<npz>+=\verb+<np>+
and \verb+<npx>+ $<=$ \verb+<npy>+ $<=$ \verb+<npz>+, 
where \verb+<npx>+, \verb+<npy>+ and \verb+<npz>+ are the nodes in the
x, y and z dimension respectively, and \verb+<np>+ is the number of nodes
allocated for the calculation. Where more than one combination
of \verb+<npx>+, \verb+<npy>+ and \verb+<npz>+ are possible, the 
combination is chosen with the minimum value of 
\verb+<npx>+$+$\verb+<npy>+$+$\verb+<npz>+. To change the default setting
the following optional input option is provided.

\item
\begin{verbatim}
boxes <integer nbx> <integer nby> <integer nbz>
\end{verbatim}
specifies the distribution of boxes,
where \verb+<nbx>+, \verb+<nby>+ and \verb+<nbz>+ are the number of
boxes in x, y and z direction, respectively.
The molecular system is decomposed into boxes that form the smallest
unit for communication of atomic data between nodes. The size of the
boxes is per default set to the short-range cutoff radius. If
long-range cutoff radii  are used the box size is set to half the
long-range cutoff radius if it is larger than the short-range cutoff.
If the number of boxes in a dimension is less than the number of
processors in that dimension, the number of boxes is set to the number
of processors.

\item
\begin{verbatim}
extra <integer madbox>
\end{verbatim}
sets the number of additional boxes for which memory is allocated.
In rare events the amount of memory set aside per node is insufficient
to hold all atomic coordinates assigned to that node. This leads to 
execution which aborts with the message that {\tt mwm} or {\tt msa} is too
small. Jobs may be restarted with additional space allocated by
where \verb+<madbox>+ is the number of additional boxes that are allocated
on each node. The default for \verb+<madbox>+ is 6. 
In some cases \verb+<madbox>+ can be reduced to 4 if memory usage is a
concern. Values of 2 or less will almost certainly result in memory 
shortage.

\item
\begin{verbatim}
mwm <integer mwmreq>
\end{verbatim}
sets the maximum number of solvent molecules \verb+<mwmreq>+ per node,
allowing increased memory to be allocated for solvent molecules. This
option can be used if execution aborted because \verb+mwm+ was too
small.

\item
\begin{verbatim}
msa <integer msareq>
\end{verbatim}
sets the maximum number of solute atoms \verb+<msareq>+ per node,
allowing increased memory to be allocated for solute atoms. This
option can be used if execution aborted because \verb+msa+ was too
small.

\item
\begin{verbatim}
memory <integer memlim>
\end{verbatim}
sets a limit \verb+<memlim>+ in kB on the allocated amount of memory used by 
the molecular dynamics module.
Per default all available memory is allocated. Use of this command
is required for QM/MM simulations only.
%\item
%For development purposes debug information can be written to the debug
%file with extension {\bf dbg} with
%\begin{verbatim}
%debug <i idebug>
%\end{verbatim}
%where $idebug$ specifies the type of debug information being written.
\end{description}
