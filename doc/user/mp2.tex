\label{sec:mp2}
The RI-MP2 module can be invoked by specifying

\begin{verbatim}
  TASK RI-MP2
\end{verbatim}
on a \verb+RESTART+ or \verb+CONTINUE+ job -- the MO vectors and
database from the SCF calculation must be present.  At present, the
RI-MP2 does not have its own input module, instead it can be
controlled by \verb+SET+ing entries in the database.  Currently
recognized entries are described below. {\em The names of these
  entries in the database will change in the very near future.}

Alternatively, the fully-direct MP2 module using conventional
four-index transformation (see also section \ref{sec:fourindex}) may be
invoked by the input line
\begin{verbatim}
  TASK DIRECT_MP2
\end{verbatim}
In future releases, the two MP2 modules will be amalgamated into a
single task where the choice of method will be an input parameter.  In
cases where the MP2 is restarted using a \verb+RESTART+ or
\verb+CONTINUE+ directive -- the MO vectors and database from the SCF
calculation must be present.  Currently, there is no specific MP2
input module, instead the behavior can be modified by using the
\verb+SET+ command on the database entries. Currently recognized
entries are described below, the first three entries apply to both the
\verb+RI-MP2+ and \verb+DIRECT_MP2+ tasks while the remainder are
specific to \verb+RI-MP2+.  {\em The names of these entries in the
  database are subject to change in the near future.}

\subsection{Frozen Orbitals}

\begin{verbatim}
  set "mp2:freeze by atoms" integer <occupied> <virtual>
  set "mp2:freeze orbitals" <integer array>
\end{verbatim}

Determines which orbitals from the SCF reference are to be frozen (not
correlated) for the MP2 calculation.  The first entry instructs the
code to guess how many occupied and virtual orbitals to drop based on
the atoms present in the system.  The two integer elements expected in
this entry control whether this guess is used to freeze parts of the
occupied (first element) and/or virtual spaces (second element).  A
value of 1 in the element requests the guess be used; any other value
and it will not be used.  Table \ref{tbl:freeze-by-atoms} details how
the ``guess'' is produced.

\begin{table}
\caption{Number of orbitals considered ``core'' in the ``freeze by
atoms'' algorithm.}
\label{tbl:freeze-by-atoms}
\begin{tabular}{cclr}
\hline\hline
Period & Elements & Core Orbitals & Number of Core \\
\hline
0 & H -- He  & ---                                          &  0 \\
1 & Li -- Ne & 1$s$                                         &  1 \\
2 & Na -- Ar & 1$s$2$s$2$p$                                 &  5 \\
3 & K -- Kr  & 1$s$2$s$2$p$3$s$3$p$                         &  9 \\
4 & Rb -- Xe & 1$s$2$s$2$p$3$s$3$p$4$s$3$d$4$p$             & 18 \\
5 & Cs -- Rn & 1$s$2$s$2$p$3$s$3$p$4$s$3$d$4$p$5$s$4$d$5$p$ & 27 \\
6 & Fr -- Lr & 1$s$2$s$2$p$3$s$3$p$4$s$3$d$4$p$5$s$4$d$5$p$6$s$4$f$5$d$6$p$
     & 43 \\
\hline\hline
\end{tabular}
\end{table}

The second entry allows a more general, but less automatic,
specification of which orbitals are to be frozen.  For convenience,
non-positive entries are understood to refer to highest-lying virtual
orbital (0), second highest (-1), and so on, down to the lowest-energy
occupied orbital ($-N+1$, for $N$ basis functions).  Entries outside
the range $[-N+1, N]$ constitute an error, which would be detected at
the start of the MP2 calculation (the number of basis functions is not
generally known when the input is read).

Both entries may be present, and orbitals in the union of the two sets
will be frozen.

{\em Cautions:\/}  The rules for freezing orbitals ``by atoms'' are
rather unsophisticated at the moment and may not do what you want.
From limited experience, it seems that special attention should be
paid to systems including third- and higher- period atoms, and perhaps
to the use of spherical harmonic or Cartesian representation of
higher-angular momentum shells.

\subsection{Schwarz Integral Screening}

\subsection{Fitting Basis}

\begin{verbatim}
  set "mp2:ri-mp2 basis" <string value>
\end{verbatim}

This {\em required} entry specifies the basis to be used as the
``fitting'', or ``resolution of the identity'' basis.  It should name
a basis set defined in a \verb+BASIS+ directive somewhere in the
current or any previous input file.


\subsection{Transformed Integral Filename}

\begin{verbatim}
  set "mp2:mo 3-center integral file" <string name default "$file_prefix$.mo3cint">
\end{verbatim}

Specifies the base for constructing the file names for the transformed
three-center integrals.  The full name consists of this value followed
by a letter indicating the spin case of the integrals and the node
number padded to four digits.

\subsection{Reference Spin Mapping}

\begin{verbatim}
  set "mp2:reference spin mapping" <integer array>
\end{verbatim}

Allows RI-MP2 calculations to be done with variations of the SCF
reference wavefunction.  Each element is the SCF spin case to be used
for the corresponding spin case of the correlated calculation.  The
number of elements set determines the overall type of correlated
calculation to be performed.  The default is to use the unadulterated
SCF reference.  Some examples of settings and their meanings:
\begin{enumerate}
\item[``\verb+1 1+''] Performs a spin-unrestricted calculation (two
elements) using the alpha spin orbitals (spin case 1) from the
reference for both of the correlated reference spin cases. SCF
reference may be RHF or UHF in this case.
\item[``\verb+2 2+''] Similar to the previous instance, but uses the beta-spin
SCF orbtials for both correlated spin cases.  The SCF reference
obviously must be UHF in this case.
\item[``\verb+2+''] Performs a spin-restricted calculation (one element)
from the beta-spin SCF orbitals.
\item[``\verb+2 1+''] Performs a spin-unrestricted calculation with the
spins flipped from the original SCF reference.
\end{enumerate}

\subsection{Batch Sizes}

\begin{verbatim}
  set "mp2:transformation batch size" <integer size default -1>
  set "mp2:energy batch size" <integer isize jsize default -1 -1>
\end{verbatim}

These entries control the size of each batch in the transformation and
energy evaluation, and consequently, the memory requirements and
number of passes required.  Values less than 1 for these batch sizes
tell the code to determine the batch size based on the available
memory, which is the default.  Should there be problems with the
program-determined batch sizes, these variables allow the user to
override them.  The program will always use the smaller of the user's
value of these entries and the internally computed batch size.

For the transformation, this is the number of occupied orbitals in the
$({occ}\ {vir} | {fit})$ three-center integrals to
be produced at a time.  If this entry is less than the number of occupied
orbitals in the system, the transformation will require multiple
passes through the two-electron integrals.  The memory requirements of
this stage are {\em two} global arrays of dimension ${<batch
size>}\times {vir} \times {fit}$ with the ``fit''
dimension distributed across all processors (on shell-block
boundaries).  The compromise here is memory space versus multiple
integral evaluations.

For the energy evaluation, this is the number of occupied orbitals in
the the two sets of three-center integrals to be multiplied together
to produce a matrix of approximate four-center integrals.  Two blocks
of integrals of dimension $({<batch isize>}\times {vir})$ and
$({<batch jsize>}\times {vir})$ by fit are read in from disk and
multiplied together to produce $<batch isize> <batch jsize> {vir}^2$
approximate integrals.  The compromise here is performance of the
distributed matrix multiplication (which requires large matrices)
versus memory space.

\subsection{Energy Memory Allocation Mode}

\begin{verbatim}
  set "mp2:energy mem minimize" <string mem_opt default "I">
\end{verbatim}

Choose the strategy for the memory allocation for the energy
evaluation phase.  Choices are to minimize the amount of I/O
(``\verb+I+'') or the amount of computation (``\verb+C+'') done by the
code.  In order to minimize I/O, the strategy is to make the
\verb+ISIZE+ as large as possible so that the total number of passes
through the integral files is as small as possible.  In order to
minimize computation, the blocks are chosen as close to square as
possible so that permutational symmetry in the energy evaluation can
be used most effectively.

\subsection{RI Approximation}

\begin{verbatim}
  set "mp2:ri approximation" <string approx default "V">
\end{verbatim}

Controls the type of RI approximation used in the calculation.  The
strings \verb+"V"+ and \verb+"SVS"+ are recognized (case
sensitive!), following the notation used in O.~Vahtras, J~Alml\"of,
and M.~W.~Feyereisen, {\em Chem. Phys. Lett.} {\bf 213}, 514--518
(1993).  The \verb+"S"+ approximation will also be supported eventually.

\subsection{Local Memory Usage in Three-Center Transformation}

\begin{verbatim}
  set "xf3ci:AO 1 batch size" <integer max>
  set "xf3ci:AO 2 batch size" <integer max>
  set "xf3ci:fit batch size" <integer max>
\end{verbatim}

These entries control local memory usage in the first two steps of the
transformation.  The size of the local arrays determines the sizes of
the two matrix multiplications.  These entries set limits on the size
of blocks to be used in each index.  The listing above indicates the
order of importance of the parameters to performance, with
\verb+xf3ci:AO 1 batch size+ being most important.

Note that these entries are only upper bounds and that the program
will size the blocks according to what it determines the best usage of
the available local memory is.  The absolute maximum block sizes are
the number of functions in the AO basis or the number of fitting basis
functions on a node.  The absolute minimum block sizes are the largest
shell in the appropriate basis.  Batch size entries outside of this
range are silently reset to an appropriate value.

It is unlikely that users will have any reason to use these entries
unless they are doing very particular performance measurements.

\subsection{Printing Control}

Printable items which are recognized by the RI-MP2 module are listed
in Table \ref{tbl:mp2-printable}.  At the moment, these items can be
controlled by directly setting the
\verb+mp2:print+ entry in the RTDB.

\begin{table}
\caption{Printable items in the RI-MP2 module and their default print levels.}
\label{tbl:mp2-printable}
\begin{tabular}{lll}
\hline\hline
Item                    & Print Level   & Description \\
\hline
'2/3 ints'              & debug         & Partially transformed 3-center integrals \\
'3c ints'               & debug         & MO 3-center integrals in energy evaluation \\
'4c ints b'             & debug         & ``B'' matrix derived from approx. 4c integrals \\
'4c ints'               & debug         & Approximate 4-center integrals \\
'amplitudes'            & debug         & ``B'' matrix with denominators (not really amplitudes, strictly) \\
'basis'                 & high          & \\
'fit xf'                & debug         & Transformation for fitting basis \\
'geombas'               & debug         & Detailed basis map info\\
'geometry'              & high          & \\
'information'           & low           & General information about calc.\\
'integral i/o'          & high          & File size information\\
'mo ints'               & debug         & \\
'pair energies'         & debug         & \\
'partial pair energies' & debug         & Pair energy matrix each time it is updated \\
'progress reports'      & default       & Report completion of time-consuming steps\\
'reference'             & high          & Details about reference wavefunction\\
'warnings'              & low           & Non-fatal warnings \\
\hline\hline
\end{tabular}
\end{table}

