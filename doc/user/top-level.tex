
\subsection{START and RESTART}
\label{sec:start}

This is a startup directive (\ref{sec:inputstructure}).

A \verb+START+ or \verb+RESTART+ directive is optional but care must
be taken if no directive is presented since the behaviour of the code
will vary depending upon the existence of the database. 
\begin{verbatim}
  (RESTART || START) [<string file_prefix>] [rtdb <string rtdb_file_name>]
\end{verbatim}
\verb+START+ indicates a startup calculation in which the database is
created and any existing database is destroyed.  \verb+RESTART+
indicates that an existing database should be used and an error will
result if it does not exist.  All files created by NWChem are by
default prefixed with the string specified on the directive, which
defaults to the basename of the input file (i.e., the input file name
with the directory and any trailing \verb+.*+ stripped off).  Note
that if you change the file prefix in a restart job, NWChem will be
unable to locate most files automatically.  The \verb+RTDB+ keyword
permits specification of the name of the database which otherwise
defaults to \verb+<file_prefix>.db+ in the directory for permanent
files.

If no \verb+START+ or \verb+RESTART+ directive is presented then
\begin{itemize}
  \item the default file prefix is used,
  \item the default database name is used,
  \item if the default database exists, then the job is assumed to be
    a restart, otherwise it is assumed to be a startup.
\end{itemize}

E.g., if running NWChem with the command \verb+nwchem water.nw+ and no
start/restart directive is presented, then the file prefix will be
\verb+water+ and the database name \verb+water.db+.  The same is
accomplished by either of the following directives,
\begin{verbatim}
  start water

  start water rtdb water.db
\end{verbatim}

\subsection{SCRATCH\_DIR and PERMANENT\_DIR}
\label{sec:dirs}

This is a startup directive (\ref{sec:inputstructure}).

NWChem distinguishes between permanent (or persistent) and scratch (or
temporary) files and can optionally put them in different locations.
The default location of the scratch and permanent directories is
installation specific, but most commonly both permanent and scratch
files default to the current directory.

On the IBM SP and workstation clusters, there is need to specify
machine/process specific names for both local and shared file systems.
On the KSR it is useful to specify scratch file directories with
automated striping across processors with round-robin allocation.  SMP
clusters have both requirements.  The anticipated usage of the EMSL
IBM SP has NWChem executing in /verb+/scratch+ with all files
being created there by default.  Files to be saved for restart {\em
  must} be either explicitly redirected or copied.  This has the
advantages of minimizing very expensive network traffic on shared-file
systems and also reducing the amount of data stored in the limited
shared-file space.  The manual copying fails if the job crashes or
runs out of time so it is preferable to automatically directly store
{\em small} restart files to a persistent file space.

The directives \verb+SCRATCH_DIR+ and \verb+PERMANENT_DIR+, which have
identical format and capability, enable user input of one or more
directories.

\begin{verbatim}
   SCRATCH_DIR  [(<string host>|<integer process>):]<string directory> \ 
                [...]
\end{verbatim}

Directories are extracted from the user input by excuting in order the
following steps
\begin{enumerate}
\item Look for a directory qualified by the process number of the
  invoking process.
\item If there are a list of directories qualified by the name of the
  host machine (as returned by \verb+util_hostname()+) then use
  round-robin allocation from the list for processes executing on the
  given host.
\item If there are a list of directories unqualified by any hostname
  or process ID then use round-robin allocation from this list.
\end{enumerate}
If there is no user input, or no match is found from the user input,
then the same steps are repeated using the installation specific
defaults.  Finally, there is a built in default of the current working
directory (\verb+"."+).

Some examples:
\begin{itemize}
\item  All processes put scratch files in the local scratch directory
\begin{verbatim}
      scratch_dir /localscratch
\end{verbatim}
\item Process 0 puts scratch files in \verb+/piofs/rjh+, all others put them
  in \verb+/scratch+
\begin{verbatim}
      scratch_dir /scratch 0:/piofs/rjh
\end{verbatim}
\item All processes allocate files in a round-robin fashion from the
given list of directories
\begin{verbatim}
      scratch_dir /scr1 /scr2 /scr3 /scr4 /scr5
\end{verbatim}
\item Processes distributed across two SGI multi-processor machines
  (coho and bohr) allocate files in a round-robin fahion from
  host-specific lists
\begin{verbatim}
      scratch_dir coho:/xfs1/rjh coho:/xfs2/rjh coho:/xfs3/rjh \
                  bohr:/disk01/rjh bohr/rjh:/disk02/rjh bohr:/disk13/rjh
\end{verbatim}
\end{itemize}

\subsection{MEMORY}

\begin{verbatim}
  MEMORY [[total] <integer total_size>] \
         [stack <integer stack_size>] \
         [heap <integer heap_size>] \
         [global <integer global_size>] \
         [units <string units default "real">] \
         [(verify||noverify)] \
         [(nohardfail||hardfail)] \
\end{verbatim}

This is a startup directive (\ref{sec:inputstructure}).

This directive controls the memory available to the program.  Known
memory units are \verb+real+, \verb+double+ (synonyms),
\verb+integer+, \verb+byte+, \verb+kb+ (kilobytes), \verb+mb+
(megabytes), \verb+mw+ (megawords, 64 bit word).  Installation
dependent defaults should suffice for most calculations (for instance,
the default on the Intel Delta is the maximum available).  The memory
model used by the input is that there are three distinct regions of
memory: stack, heap, and global.  Stack and heap are node-private,
while the global region on all processors is used to provide
globally-shared memory.  In the actual implementation, some of the
distinctions between these categories of memory are blurred.  This is
explained in more detail below.

In most cases the user need only specify the total memory limit and
the limits on each category will be determined from a default
partitioning (currently 10\% for heap, 10\% for stack, and 80\% for
global).  If the user sets only one of the stack, heap, or global
limits specifically, the difference from the default partitioning of
the total memory limit is distributed to or borrowed from the other
categories in proportion to their weight in the default memory
partitioning.  If two of the category limits are given, the third is
obtained by subtracting the two given limits from the total limit,
which may have been specified or may be a default value.  If all three
category limits are specified, they determine the total memory (if the
total memory is also specified, it must be larger than the sum of the
three categories, but is otherwise not used).  In all cases, the code
attempts to check for inconsistent memory specifications.

The following specifications all provide eight megabytes (assuming 64
bit floating point numbers) of total memory to the application, which
will be distributed according to the default partitioning.
\begin{verbatim}
  memory 1000000
  memory 1000000 real
  memory 1 mw
  memory 8 mb
\end{verbatim}
These memory directive also allocate 8 MB, but specify a complete
partitioning as well.
\begin{verbatim}
  memory total 8 stack 2 heap 2 global 4 mb
  memory stack 2 heap 2 global 4 mb
\end{verbatim}
The options \verb+VERIFY+ and \verb+NOVERIFY+ enable/disable automatic
detection of corruption of allocated memory which is enabled by
default. The small overhead incurred by this can be eliminated by
specifying \verb+NOVERIFY+.  The \verb+HARDFAIL+ option causes the
local memory management routines to generate an internal fatal error
if any memory operation fails.  This is useful if applications are
poorly coded and do not always check the return status of memory
management routines.  The default is \verb+NOHARDFAIL+ which should
result in a more meaningful error message from the application if a
memory allocation fails.

As mentioned above, the actual implementation of the code differs
somewhat from the input model described above -- the software relaxes
some of the distinctions between categories.  The memory allocator
(MA) allocates both the heap and the stack from a single memory region
of size {\em heap+stack} without enforcing the partition.  The heap
vs. stack partition is only really meaningful to application
developers and can be ignored by users.  Further complicating matters,
the global array (GA) toolkit is allocated from within the MA space on
distributed memory machines, while on shared-memory machines it is
separate\footnote{This is because on true shared-memory machines there
  is no choice but to allocate GAs from within a shared-memory segment
  which is mangaged differently by the operating system}.  On
distributed memory platforms, the MA region is actually the size of
(stack+heap+global), and all three types of memory allocation compete
for the same pool of memory with no limits except on the total
available memory.  This relaxation usually benefits the user since it
may allow allocation requests to succeed where a stricter memory model
would fail.  These implementation details must be kept in mind when
reading output from the program relating to memory usage, or when jobs
fail due to memory limitations.

Current standard defaults for various platforms are listed in Table
\ref{tbl:default-memory-limits}.  On machines for which individual
processors are commonly used in single user mode (e.g., IBM SP-X,
Linux laptops, CRAY-T3D, Intel Paragon, KSR) then the defaults reflect
the maximum memory available to applications with common hardware
configurations.
\begin{table}
\caption{Default total memory limits according to hardware platform.}
\label{tbl:default-memory-limits}
\begin{tabular}{lr}
\hline\hline
Platform        & Total Memory Limit (MBytes) \\
\hline
CRAY-T3D        & 40 \\
DECOSF          & 48 \\
IBM RS/6000     & 56 \\
IBM SP-X        & 90 \\
Intel Paragon   & 16 \\
KSR             & 20 \\
Linux           & 16 \\
SGI             & 48 \\
Sun             & 48 \\
\hline\hline
\end{tabular}
\end{table}

\subsection{ECHO}

\begin{verbatim}
  ECHO
\end{verbatim}

This is a startup directive (\ref{sec:inputstructure}).

This directive causes the entire input file to be printed to Fortran
unit six.

\subsection{TITLE}

\begin{verbatim}
  TITLE 
  <string title>
\end{verbatim}

  The title for the job is set to the entire contents of the {\em
next} input line.  E.g.,
\begin{verbatim}
  title
  My first NWChem job

  title; This is the title
\end{verbatim}
The title is stored on the database and will be used in all subsequent
tasks/jobs unless respecified in the input.  The default title is an
empty character string.

\subsection{PRINT and NOPRINT}
\label{sec:printcontrol}

\begin{verbatim}
  PRINT [none | low | medium | high | debug] \
        [<string list_of_names>]

  NOPRINT [<string list_of_names>]
\end{verbatim}

The \verb+PRINT+ and \verb+NOPRINT+ directives are special in that
{\em all} modules (will) recognize them and they  have a
standard form.  Items that are under standard print control are
assigned a print threshold which is one of none, low, medium, high and
debug.  Each module can control both the print level and the printing
of individual items.

An example will help.  If the print level is medium (which is the
default) all items with a print threshold less than or equal to medium
(i.e., low or medium) will be printed.  In addition, the printing of
items can be explicitly enabled or disabled by name.  Thus, to keep
the NWChem driver routine as quiet as possible, but to force printing
of the memory usage statistics and the entire contents of the database
at the end of the job, use this directive
\begin{verbatim}
  print none 'ma stats' rtdb
\end{verbatim}

The top level of the code recognizes these names. 
\begin{tabbing}
  Very\_long\_descriptive\_name \= Print level space \= \kill
  Name                   \> Print Level \> Description \\
                         \>        \> \\
 ``total time''        \> medium \> Prints cpu and wall time at job end\\
 ``rtdb''              \> high    \> Print names of RTDB entries\\
 ``rtdbvalues''        \> high    \> Print name and values of RTDB entries\\
 ``ga summary''        \> medium \> Summarizes GA allocations at job end \\
 ``ma summary''        \> medium \> Summarizes MA allocations at job end \\
 ``ma stats''          \> high   \> Prints MA usage statistics at job end \\
 ``version''           \> debug  \> Prints version of all compiled routines \\
\end{tabbing}

Look in the documentation of each module to learn about names specific
to those modules.  Currently, the print level is local to each module
and defaults to medium within each module.  It is also possible to
change the implementation to make the default print level inherited
from the calling layer --- email nwchem-support@emsl with your
preferences.

\subsection{SET}
\label{sec:set}

\begin{verbatim}
  SET <name> [<string type default "string">] <$type$ list_of_values>
\end{verbatim}

This directive makes an entry in the database.  There are only two
common circumstances where this directive is necessary.  The first is
to provide indirection.  One example of this was encountered above
(section \ref{sec:arch}).  Another example of indirection would be if
you entered a couple of basis sets and wanted to change the one being
used for the SCF calculation.  This input specifies two basis sets
\begin{verbatim}
  basis "trial basis"
    o library sto-3g
    h library sto-3g
  end

  basis "production basis"
    o library cc-pvqz
    h library cc-pvqz
  end
\end{verbatim}
and the following specifies that the basis set used to represent the
molecular orbitals (which must have name \verb+"ao basis"+) is the
\verb+"trial basis"+
\begin{verbatim}
  set "ao basis" "trial basis"
\end{verbatim}

The other common circumstance in which \verb+SET+ is used is for
modules under development.  \verb+SET+ can be used to provide a
rudimentary input mechanism for options that do not yet have more
friendly input routines, or for options that are very infrequently
exercised.  For instance, some of the parameters that tune the
integral evaluation routines may be adjusted in this fashion
\begin{verbatim}
  set int:cando_sp logical false
\end{verbatim}

Logical values may be entered as \verb+.true.+, \verb+true+, or
\verb+t+, and \verb+.false.+, \verb+false+, or \verb+f+.

\subsection{UNSET}
\label{sec:unset}

\begin{verbatim}
  UNSET <string name>[*]
\end{verbatim}

This deletes simple entries from the database.  Note that complex
objects such as geometries and basis sets cannot be deleted in this
fashion.  For example, 
\begin{verbatim}
  unset int:cando_sp
\end{verbatim}

If a wildcard (*) is specified at the end of the name then {\em all}
entries beginning with the name are deleted.  This is very useful for
resetting modules to their default behaviour.  Modules typically store
their information in the database with names that commence with
\verb+module:+.

\subsection{STOP}

\begin{verbatim}
  STOP
\end{verbatim}

As soon as this directive is encountered all processing ceases, and
the calculation terminates with an error condition.  It is useful for
verifying input.

\subsection{TASK}
\label{sec:task}

As noted above (\ref{sec:inputstructure}), input is parsed
sequentially until a \verb+TASK+ directive is encountered at which
point the requesed task is performed.  Subsequent to successful
completion of the task, additional input is read up to the next
\verb+TASK+ directive which is then executed.  This repeats until all
input is processed.  If the last line of the input is not a
\verb+TASK+ directive, a warning message is printed.  Since the
database is persistent multiple tasks within one job behave {\em
  exactly} the same as multiple restart jobs with the same sequence of
input.

The \verb+TASK+ directive takes three forms.  The most common form,
\begin{verbatim}
  TASK <string theory> [<string operation = energy>]
\end{verbatim}
is used to perform most electronic structure calculations.  The
\verb+theory+ is used to specify the level of theory to be used (SCF,
DFT, MP2, \ldots) and the operation specifies the desired action,
defaulting to energy evaluation.  Known values of theory and operation
are:
\begin{itemize}
\item Theory \newline
\begin{itemize}
 \item SCF --- Hatree-Fock
 \item DFT --- Density functional theory for molecules
 \item RIMP2 --- MP2 using the RI approximation
 \item OIMP2 --- Orbital invariant MP2
 \item DIRECT\_MP2 --- MP2 direct from the AO integrals
 \item SEMI\_DIR\_MP2 --- MP2 semi-direct from the AO integrals
 \item CCSD --- Coupled-cluster single and double excitations
 \item MCSCF --- multiconfiguration SCF
 \item SELCI --- selected configuration interaction with perturbation
   correction 
\end{itemize}
\item Operation \newline
\begin{itemize}
 \item ENERGY --- single point energy evaluation
 \item GRADIENT --- evaluate the derivative of the energy w.r.t.\
   nuclear coordinates
 \item OPTIMIZE --- optimize the energy by variation of the molecular
   structure.  This is presently driven by the Stepper module
   (\ref{sec:stepper}). 
 \item FREQUENCIES or FREQ --- compute second-derivatives and print
   out an analysis of molecular vibrations.
\end{itemize} 
\end{itemize}
Note that if an operation requires computation of derivatives of the
energy and analytic derivatives are not available, then derivatives
are computed numerically, which can be very expensive.

For example, to perform a single point SCF calculation
\begin{verbatim}
  task scf
\end{verbatim}
or
\begin{verbatim}
  task scf energy
\end{verbatim}
To perform a DFT geometry optimization
\begin{verbatim}
  task dft optimize
\end{verbatim}

The second form of the \verb+TASK+ directive
\begin{verbatim}
  TASK <string task>
\end{verbatim}
is used in instances where either the task to be performed does not
fit the model of the previous version (e.g., print the contents of the
database) or if the operation has not yet been implemented in a
fashion that applies to a wide range of theories (e.g., property
evaluation).  The list of tasks supported in this fashion are
\begin{itemize}
  \item RTDBPRINT --- print the contents of the database
  \item MD --- molecular dynamics using Ideaz
  \item CPHF --- invoke the CPHF module
  \item NWARGOS --- molecular dynamics using ARGOS
  \item PROPERTY --- miscellaneous property calculations
\end{itemize}

The final form of the task directive
\begin{verbatim}
  TASK shell [(<integer-range processor = 0>|all)] <string command>
\end{verbatim}
is only supported on those machines that have a full UNIX-like
operating system.  This directive causes selected processes to execute
the given specified command using the Bourne shell.  The command must
form a single string, and thus must be enclosed in double quotes
\verb+"+ if it contains spaces.  Note that NWChem's quoting of
special input characters is {\em very} limited when compared with that
of the shell, so when trying to execute all but the simplest commands,
put the shell script in a file and execute the file from within
NWChem.  By default the command is executed only by process zero.  A
range of processes may be specified using Fortran triplet notation,
and all processes using the keyword all.

For example, for process zero to copy the molecular orbitals file to a
backup location
\begin{verbatim}
  task shell "cp *.movecs /pnnlfs/save"
\end{verbatim}
For all processes to list the contents of their \verb+/scratch+
directory
\begin{verbatim}
  task shell all "ls -l /scratch"
\end{verbatim}
For processes 0 to 10 to remove the contents of the current directory
\begin{verbatim}
  task shell 0:10:1 "/bin/rm -f *"
\end{verbatim}

\subsection{CHARGE}
\label{sec:charge}

\begin{verbatim}
  CHARGE <real charge default 0>
\end{verbatim}

Specifies the total charge of the system which defaults to zero.
E.g., to run a doubly charged cation
\begin{verbatim}
  charge 2
\end{verbatim}
