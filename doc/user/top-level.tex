\label{sec:toplevel}

Top-level directives are directives that potentially affect all
modules in the code.  Some specify molecular properties (e.g., total
charge) or other data that should apply to all subsequent
calculations.  However, most top-level directives provide the user
with the means to manage the resources for a calculation and to start
computations.  As the first step in the execution of a job, NWChem
scans the entire input file looking for start-up directives which
NWChem must process before all other input.  The input file is then
rewound and processed sequentially, and each directive is processed in
the order it is encountered.  In this second pass start-up directives
are ignored.

The following subsections describe each of these directives in detail,
noting all keywords, options, required input, and defaults.

\subsection{START, CONTINUE and RESTART}
\label{sec:start}

A {\tt START}, or {\tt CONTINUE}, or {\tt RESTART} directive is
optional.  If one of these three directives is not specified
explicitly, the code will infer one based on upon the name of the
input file and the availability of the database.  When allowing NWChem
to infer the start-up directive, the user must be quite certain that
the contents of the database will result in the desired inference.  It
is usually more prudent to specify the directive explicitly, using the
following format.

\begin{verbatim}
(RESTART || CONTINUE || START) [<string file_prefix>] \
                               [rtdb <string rtdb_file_name>]
\end{verbatim}

The \verb+START+ directive indicates that the calculation is one in
which the database is to be created.  Any relevent information that
already exists in a previous database of the same name is destroyed.
The string {\tt file\_prefix} will be used as the prefix when naming
any files created in the course of the calculation.  The user can also
specify a unique name for the database itself, using the keyword {\tt
  rtdb}.  When this keyword is entered, the string {\tt
  rtdb\_file\_name} is used as the database name.  If the keyword {\tt
  rtbd} is omitted, the name of the database defaults to
\verb+<file_prefix>.db+ in the directory for permanent files.

If a calculation is to start from a previous calcultion and go on
using the existing database, the \verb+RESTART+ directive or the
\verb+CONTINUE+ directive must be used.  In such a case, the previous
database must already exist.  The {\tt file\_prefix} usually should
not be changed when restarting a calculation otherwise NWChem will not
be able to find necessary files automatically when going on with the
calculation.

There are two ways to restart a previous calculation.  Most commonly
the previous calculation completed (with or without an error
condition) and it is desired to perform a new task or restart the
previous one, perhaps with some input changes.  In these instances,
the \verb+RESTART+ directive should be used.  This reuses the previous
database and associated files, and reads the input file for new input
and task information.

Less commonly, if the last task of the previous job terminated in such
a way that enables it to be continued (e.g., from an automatic
checkpoint in the molecular dynamics program) with no additional
input, then the \verb+CONTINUE+ directive should be used.  This reuses
the previous database and associated files, continues the previous
task, and then reads the input file for additional input and task
information.

In summary, the \verb+CONTINUE+ directive tells NWChem to finish the
previous task, and then perform any new tasks that are specified by
the input. The \verb+RESTART+ directive immediately looks for new
input and task information.

If the input file does not contain a \verb+START+ or a \verb+CONTINUE+ or a
\verb+RESTART+ directive, then
\begin{itemize}
  \item the file prefix is the name of the input file with any
    trailing \verb+.nw+ removed, and
  \item the name of the database for the calculation is {\tt
      file\_prefix.db}
\end{itemize}
If the database with name {\tt file\_prefix.db} does {\it not} exist,
then the calculation is carried out as if a \verb+START+ directive had
been encountered.  If the database with name {\tt file\_prefix.db}
{\it does} exist, then the calculation is performed as if a
\verb+RESTART+ directive had been encountered.  

For example, if the NWChem input file \verb+water.nw+ does not contain
a \verb+START+, \verb+RESTART+, or \verb+CONTINUE+ directive, the default
{\tt file\_prefix} is set to {\tt water}, and the default database name is
{\tt water.db}.  If the database \verb+water.db+ does {\it not} exist already,
the code behaves as if the input file contained the directive,

\begin{verbatim}
  start water
\end{verbatim}

If the database \verb+water.db+ {\it does} exist,
the code behaves as if the input file contained the directive,

\begin{verbatim}
  restart water
\end{verbatim}


\subsection{SCRATCH\_DIR and PERMANENT\_DIR}
\label{sec:dirs}

% This is a startup directive (section \ref{sec:inputstructure}).

These are startup directives that allow the user to specify the
directory location of permanent and scratch files created by NWChem.
NWChem distinguishes between permanent (or persistent) files and
scratch (or temporary) files, and allows the user the option of
putting them in different locations.  In most installations, however,
permanent and scratch files are all written to the current directory
by default.  This is at the discretion of the specific installation
and may be different on different machines.  What constitutes 'local'
disk space may also differ from machine to machine.

For example, on the IBM SP and workstation clusters, machine/process
specific names must be specified for both local and shared file
systems.  On the KSR it is useful to specify scratch file directories
with automated striping across processors with round-robin allocation.
SMP clusters have both requirements.  The anticipated usage of the
EMSL IBM SP has NWChem executing in \verb+scratch+ with all files
being created there by default.  Files to be saved for restart {\em
  must} be either explicitly redirected or copied.  This has the
advantages of minimizing very expensive network traffic on shared-file
systems and also reduces the amount of data stored in the limited
shared-file space.  However, this manual copying fails if the job
crashes or runs out of time, so it may be preferable to automatically
store {\em small} restart files directly to a persistent file space.

The \verb+SCRATCH_DIR+ and \verb+PERMANENT_DIR+ directives have
identical format and capability, and enable the user to specify a
single directory for all processes, or different directories for
different processes.  The general form of the directive is as follows;

\begin{verbatim}
   (PERMANENT_DIR || SCRATCH_DIR) [(<string host>|<integer process>):] \
                                       <string directory> \ 
                                  [...]
\end{verbatim}

Directories are extracted from the user input by executing the
following steps in order
\begin{enumerate}
\item Look for a directory qualified by the process number of the
  invoking process.
\item If there is a list of directories qualified by the name of the
  host machine (as returned by \verb+util_hostname()+), then use
  round-robin allocation from the list for processes executing on the
  given host. 
\item If there is a list of directories unqualified by any hostname
  or process ID, then use round-robin allocation from this list.
\end{enumerate}
If directory allocation directive(s) are not specified in the input
file, or if no match is found to the directory names specified by
input using these directives, then the above steps are executed using
the installation specific defaults.  If the code cannot find a valid
directory name based on the input specified in either the directive(s)
or the system defaults, files are automatically written to the current
working directory (\verb+"."+).

The following is a list of examples of specific allocations of scratch
directory locations:
\begin{itemize}
\item put scratch files from all processes in the local scratch directory; 
(Warning: the definition of 'local scratch directory' may change from 
machine to machine);
\begin{verbatim}
      scratch_dir /localscratch
\end{verbatim}
\item put scratch files from Process 0 in \verb+/piofs/rjh+, but put all 
other scratch files in \verb+/scratch+;
\begin{verbatim}
      scratch_dir /scratch 0:/piofs/rjh
\end{verbatim}
\item put scratch files from Process 1 in directory \verb+scr1+, those from
Process 2 in \verb+scr2+, and so forth, in a round-robin fashion, using the
given list of directories;
\begin{verbatim}
      scratch_dir /scr1 /scr2 /scr3 /scr4 /scr5
\end{verbatim}
\item allocate files in a round-robin fahion from
  host-specific lists for processes distributed across two
 SGI multi-processor machines (node names {\it coho} and {\it bohr});
\begin{verbatim}
      scratch_dir coho:/xfs1/rjh coho:/xfs2/rjh coho:/xfs3/rjh \
          bohr:/disk01/rjh bohr/rjh:/disk02/rjh bohr:/disk13/rjh
\end{verbatim}
\end{itemize}

\subsection{MEMORY}

This is a start-up directive that allows the user to specify the
amount of memory that NWChem can use for the job.  If this directive
is not specified, memory is allocated according to installation
dependent defaults.  These generally should suffice for most
calculations, since the defaults usually correspond the total amount
of memory available on the machine.  The general form of the directive
is as follows;

\begin{verbatim}
  MEMORY [[total] <integer total_size>] \
         [stack <integer stack_size>] \
         [heap <integer heap_size>] \
         [global <integer global_size>] \
         [units <string units default "real">] \
         [(verify||noverify)] \
         [(nohardfail||hardfail)] \
\end{verbatim}

There are three distinct regions of memory: stack, heap, and global.
Stack and heap are node-private, while the union of the global region
on all processors is used to provide globally-shared memory.  NWChem
recognizes the following memory units;
\begin{itemize}
\item \verb+real+ and \verb+double+ (synonyms)
\item \verb+integer+
\item \verb+byte+
\item \verb+kb+ (kilobytes)
\item \verb+mb+ (megabytes)
\item \verb+mw+ (megawords, 64 bit word)
\end{itemize}

In most cases the user needs to specify only the total memory limit,
and allow the limits on each category to be determined from a default
partitioning (currently 25\% heap, 25\% stack, and 50\% global).
Alternatively, the keywords \verb+stack+, \verb+heap+, and
\verb+global+ can be used to define specific allocations for each of
these categories.  If the user sets only one of the stack, heap, or
global limits by input, the limits for the other two categories are
obtained by partitioning the remainder of the total memory available
in proportion to the weight of those two categories in the default
memory partitioning.  If two of the category limits are given, the
third is obtained by subtracting the two given limits from the total
limit (which may have been specified or may be a default value).  If
all three category limits are specified, they determine the total
memory allocated.  However, if the total memory is also specified, it
must be larger than the sum of all three categories.  The code
attempts to check for inconsistent memory specifications.

The following specifications all provide eight megabytes (assuming 64
bit floating point numbers) of total memory to the application, which
will be distributed according to the default partitioning.
\begin{verbatim}
  memory 1000000
  memory 1000000 real
  memory 1 mw
  memory 8 mb
\end{verbatim}

The following memory directives also allocate 8 MB, but specify a complete
partitioning as well;

\begin{verbatim}
  memory total 8 stack 2 heap 2 global 4 mb
  memory stack 2 heap 2 global 4 mb
\end{verbatim}

The optional keywords \verb+VERIFY+ and \verb+NOVERIFY+ in the
directive give the user the option of enabling or disabling automatic
detection of corruption of allocated memory.  The default is
\verb+VERIFY+, which enables the feature. This incurs a small
overhead, which can be eliminated by specifying \verb+NOVERIFY+.

The keywords \verb+HARDFAIL+ and \verb+NOHARDFAIL+ give the user the
option of forcing an internal fatal error to be generated by the local
memory management routines if any memory operation fails.  The default
in the code is \verb+NOHARDFAIL+, which allows the code to continue
past the memory operation failure, and perhaps generate a more
meaningful error message before terminating the calculation.  This can
be useful when poorly coded applications do not check the return
status of memory management routines.

When specifying the specific memory allocations using the keywords
\verb+stack+, \verb+heap+, and \verb+global+ in the \verb+MEMORY+
directive, the user should be aware that some of the distinctions
between these categories of memory have been blurred in their actual
implementation in the code.  The memory allocator (MA) allocates both
the heap and the stack from a single memory region of size {\em
  heap+stack} without enforcing the partition.  The heap vs. stack
partition is meaningful only to applications developers, and can be
ignored by most users.  Further complicating matters, the global array
(GA) toolkit is allocated from within the MA space on distributed
memory machines, while on shared-memory machines it is
separate\footnote{This is because on true shared-memory machines there
  is no choice but to allocate GAs from within a shared-memory
  segment, which is mangaged differently by the operating system.}.
On distributed memory platforms, the MA region is actually the total
size of {\tt stack+heap+global}.  All three types of memory allocation
compete for the same pool of memory, with no limits except on the
total available memory.  This relaxation of the memory category
definitions usually benefits the user since it can allow allocation
requests to succeed where a stricter memory model would cause the
directive to fail.  These implementation characteristics must be kept
in mind when reading output from the program relating to memory usage.

Current standard defaults for various platforms are listed in Table
\ref{tbl:default-memory-limits}.  On machines for which individual
processors are commonly used in single user mode (e.g., IBM SP-X,
Linux laptops, CRAY-T3D, Intel Paragon, KSR), the defaults reflect
the maximum memory available to applications with common hardware
configurations.

\begin{table}
\caption{Default total memory limits according to hardware platform.}
\label{tbl:default-memory-limits}
\begin{tabular}{lr}
\hline\hline
Platform        & Total Memory Limit (MBytes) \\
\hline
CRAY-T3D        & 40 \\
DECOSF          & 48 \\
IBM RS/6000     & 56 \\
IBM SP-X        & 90 \\
Intel Paragon   & 16 \\
KSR             & 20 \\
Linux           & 16 \\
SGI             & 48 \\
Sun             & 48 \\
\hline\hline
\end{tabular}
\end{table}

\subsection{ECHO}
\label{sec:echo}

This start-up directive is provided as a convenient means to include a
listing of the input file in the output of a calculation.  It causes
the entire input file to be printed to Fortran unit six.  It has no
keywords, arguments, or options, and consists of the single line,

\begin{verbatim}
  ECHO
\end{verbatim}

The \verb+ECHO+ directive is a start-up directive.  It is processed only
once, by Process 0 when the input file is read.

\subsection{TITLE}

This directive allows the user to identify a job or series of jobs that use a
particular database.  It is an optional directive, and if omitted, the 
character string containing the input title will be empty.  The format for 
the directive is as follows;

\begin{verbatim}
  TITLE 
  <string title>
\end{verbatim}

The character string \verb+title+ is assigned to the contents of the
line following the \verb+TITLE+ directive.  For example,

\begin{verbatim}
  title
  This is the title of my first NWChem job
\end{verbatim}

Alternatively, since semicolon may be used to end an input line
(Section \ref{sec:syntax}), we may write

\begin{verbatim}
  title;  This is the title of my first NWChem job
\end{verbatim}

The title is stored in the database and will be used in all subsequent
tasks/jobs until redefined in the input.

\subsection{PRINT and NOPRINT}
\label{sec:printcontrol}

The \verb+PRINT+ and \verb+NOPRINT+ directives allow the user to
control how much output is generated.  These two directives are
special in that {\em all} modules are supposed to recognize
them.\footnote{If one does not, this is probably an error, and should
  be reported to the developers.  Send email to
  \verb+nwchem-support@emsl+.} Each module can control both the print
level and the printing of individual items.  The standard form of the
\verb+PRINT+ directive is as follows;

\begin{verbatim}
  PRINT [none | low | medium | high | debug] \
        [<string list_of_names ... >]

  NOPRINT <string list_of_names ... >
\end{verbatim}
The default print level is medium.

Everything that is printed by NWChem has a print threshold associated
with it. If this threshold is lower than the print level requested by
the user, then the output is generated.  For instance, the threshold
for printing the SCF energy at convergence is \verb+low+ (Section
\ref{sec:scfprint}), meaning that if the user print level is
\verb+low+ or higher, then it will be printed.  The print level is
intended to be a convenient way contolling how verbose NWChem is.
Setting print to \verb+high+ might be helpful to diagnose convergence
problems, whereas \verb+low+ might be better for geometry
optimizations which will perform many steps in which you are not
interested.  A print level of \verb+debug+ can generate huge amounts
of output and is best avoided.

In addition, it is possible to enable/disable the printing of items by
name regardless of the print level.  The list of items that each
module can print is documented with the rest of that modules input.

The top level of the code recognizes these items:
\begin{tabbing}
  Very\_long\_descriptive\_name \= Print level space \= \kill
  Name                   \> Print Level \> Description \\
                         \>        \> \\
 ``total time''        \> medium \> Prints cpu and wall time at job end\\
 ``rtdb''              \> high    \> Print names of RTDB entries\\
 ``rtdbvalues''        \> high    \> Print name and values of RTDB entries\\
 ``ga summary''        \> medium \> Summarizes GA allocations at job end \\
 ``ma summary''        \> medium \> Summarizes MA allocations at job end \\
 ``ma stats''          \> high   \> Prints MA usage statistics at job end \\
 ``version''           \> debug  \> Prints version of all compiled routines \\
\end{tabbing}


The following example shows how a print directive for the top level
process can be used to limit printout to only essential information.
The directive is,

\begin{verbatim}
  print none "ma stats" rtdb
\end{verbatim}

This directive instructs the NWChem main program to print nothing,
except for the memory usage statistics (\verb+ma stats+) and
the names of all items stored in the database at the end of the job.

Currently, the print level is local to each module and defaults to
medium at the beginning of each module.  It is also possible to change
the implementation so that a module inherits the default print level
of its calling layer.  (Users should send their preferences for these
defaults via email to \verb+nwchem-support@emsl+, for consideration in
future releases of the code.)

\subsection{SET}
\label{sec:set}

The \verb+SET+ directive allows the user to enter data directly into the
database (see Section \ref{sec:database} for details of the database).

\begin{verbatim}
  SET <name> [<string type default automatic>] <$type$ data>
\end{verbatim}

The \verb+<name>+ is the name of some array or object to be entered
into the database.  This must be specified explicitly; there is no
default.

The second, optional, item allows the user to define a string
specifying the type of data in the array \verb+<name>+.  The data type
can be explicitly specified as \verb+integer+, \verb+real+,
\verb+double+, \verb+logical+, or \verb+string+.  If this item is
omitted from the directive, the value for \verb+type+ is inferred from
the data type of the {\em first} data element.  In such a case,
floating-point data entered using this directive must include either
an exponent or decimal point to ensure that the correct default type
will be assumed.  The correct default type will be inferred for
logical values if logical-true values are specified as \verb+.true.+,
\verb+true+, or \verb+t+, and logical-false values are entered as
\verb+.false.+, \verb+false+, or \verb+f+.

The \verb+SET+ directive is useful for providing indirection by 
associating the name of a basis set or geometry with the standard
names (\verb+ao basis+ and \verb+geometry+ respectively) that are used
by NWChem.  See one of the sample inputs (Section \ref{sec:realsample}) for
an example of this with basis sets.  Another example using multiple geometries
 follows.

\begin{verbatim}
  title; Ar dimer BSSE corrected MP2 interaction energy
  geometry "Ar+Ar"
    Ar1 0 0 0
    Ar2 0 0 2
  end

  geometry "Ar+ghost"
    Ar1 0 0 0
    Bq2 0 0 2
  end

  basis
    Ar1 library aug-cc-pvdz
    Ar2 library aug-cc-pvdz
    Bq2 library Ar aug-cc-pvdz
  end

  set geometry "Ar+Ar"
  task mp2 

  scf; vectors atomic; end

  set geometry "Ar+ghost"
  task mp2 
\end{verbatim}

The above input performs MP2 energy calculations on an Argon dimer and
on the Argon atom in the presence of the ``ghost'' basis of the other
atom at the same geometry.

The \verb+SET+ directive can also be used as an indirect means of
supplying input to modules under development, that do not yet have
user friendly input.

\subsection{UNSET}
\label{sec:unset}

This directive gives the user a way to delete simple entries from the
database. 

\begin{verbatim}
  UNSET <string name>[*]
\end{verbatim}

This directive cannot be used with complex objects such as geometries
and basis sets.  The input string \verb+name+ must be an array, not an
object name.  A wildcard (*) specified at the end of the string
\verb+name+ will cause {\em all} entries beginning with that string to
be deleted.  This is very useful as a way to reset modules to their
default behaviour, since modules typically store information in the
database with names that begin with \verb+module:+.

The following example makes an entry in the database using \verb+SET+
and then immediately deletes it using \verb+UNSET+.

\begin{verbatim}
  set mylist 1 2 3 4
  unset mylist
\end{verbatim}

This example restores the SCF program to its default behaviour by
deleting all database entries beginning with \verb+scf:+

\begin{verbatim}
  unset scf:*
\end{verbatim}

\subsection{STOP}

\begin{verbatim}
  STOP
\end{verbatim}

As soon as this directive is encountered, all processing ceases and
the calculation terminates with an error condition.  It is useful for
verifying input.

\subsection{TASK}
\label{sec:task}

The \verb+TASK+ directive is used to tell the code what to do with the
input.  The input directives are parsed sequentially until a
\verb+TASK+ directive is encountered, as described in Section
\ref{sec:inputstructure}.  At that point, the calculation or operation
specified in the \verb+TASK+ directive is performed.  When that task
is completed, the code looks for additional input to process until the
next \verb+TASK+ directive is encountered, which is then executed.
This process continues to the end of the input file.  NWChem expects
the last directive before the end-of-file to be a \verb+TASK+
directive.  If it is not, a warning message is printed.  Since the
database is persistent, multiple tasks within one job behave {\em
  exactly} the same as multiple restart jobs with the same sequence of
input.

There are three main forms of the the \verb+TASK+ directive.  One is
used to perform most elctronic structure and molecular dynamics
calculations.  The second form is used to perform tasks such as
printing the contents of the database, or performing simple property
evaluations.  The third is used to execute UNIX commands on machines
having a Bourne shell.  Additional forms will be added in the near
future to accomodate mixed molecular-mechanics and quantum-mechanics
(MM/QM) calculations.

By default, the program terminates when a task does not complete
successfully.  The keyword \verb+IGNORE+ can be used to prevent this
and is recognized by all forms of the \verb+TASK+ directive.  When a
\verb+TASK+ directive includes the keyword \verb+IGNORE+, task failure
results in only a warning message being printed, and code execution
continues with the next task.

The input options, keywords, and defaults for each of
these three forms for the \verb+TASK+ directive are discussed in the
following subsections.

\subsubsection{TASK Directive for Electronic Structure Calculations}

This is the most commonly used version of the \verb+TASK+ directive, and
it has the following form;

\begin{verbatim}
  TASK <string theory> [<string operation default energy>] [ignore]
\end{verbatim}

The string \verb+theory+ specifies the level of theory to be used in the
calculations for this task.  NWChem currently supports eleven different
options:
\begin{itemize}
 \item \verb+SCF+ --- Hartree-Fock
 \item \verb+DFT+ --- Density functional theory for molecules
 \item \verb+RIMP2+ --- MP2 using the RI approximation
 \item \verb+OIMP2+ --- Orbital invariant MP2
 \item \verb+DIRECT_MP2+ --- MP2 direct from the AO integrals
 \item \verb+SEMI_DIR_MP2+ --- MP2 semi-direct from the AO integrals
 \item \verb+CCSD+ --- Coupled-cluster single and double excitations
 \item \verb+MCSCF+ --- Multiconfiguration SCF
 \item \verb+SELCI+ --- Selected configuration interaction with perturbation
   correction 
 \item \verb+MD+ --- Classical molecular dynamics simulation using NWARGOS
 \item \verb+MD_IDEAZ+ --- Classical molecular dynamics simulation
   using IDEAZ
\end{itemize}

The string \verb+operation+ is used to specify the calculation that will
be performed in the task.  The default operation is a single point energy
evaluation.  The following list gives the selection of operations currently
available in NWChem, and the string entry for \verb+operation+ for each one.
\begin{itemize}
 \item \verb+ENERGY+ --- single point energy evaluation
 \item \verb+GRADIENT+ --- evaluate the derivative of the energy with respect to\
   nuclear coordinates
 \item \verb+OPTIMIZE+ --- optimize the energy by variation of the molecular
   structure.  This is presently driven by the Stepper module
   (see Section \ref{sec:stepper}). 
 \item \verb+FREQUENCIES+ or \verb+FREQ+ --- compute second-derivatives and print
   out an analysis of molecular vibrations.
 \item \verb+DYNAMICS+ --- molecular dynamics using NWArgos.
 \item \verb+THERMODYNAMICS+ --- multiconfiguration thermodynamic integration
    using NWArgos.
\end{itemize}

The user should be aware that some of these operations (gradient,
optimization, dynamics, thermodynamics) require computation of
derivatives of the energy with respect to the molecular coordinates.
If analytical derivatives are not available (Section
\ref{sec:functionality}), they must be computed numerically, which can
be very expensive.

Here are some examples of the \verb+TASK+ directive, to illustrate the
input needed to specify particular calculations with the code.  To
perform a single point energy evaluation using any level of theory the
directive is very simple, since the energy evaluation is the default
for the string \verb+operation+.  For an SCF energy calculation the
input line is simply,
\begin{verbatim}
  task scf
\end{verbatim}
or equivalently,
\begin{verbatim}
  task scf energy
\end{verbatim}

Similarly, to perform a geometry optimization using density functional
theory, the \verb+TASK+ directive is
\begin{verbatim}
  task dft optimize
\end{verbatim}

The keyword \verb+IGNORE+ allows execution to continue even if the
task fails, as discussed above.

\subsubsection{TASK Directive for Special Operations}

This form of the \verb+TASK+ directive is used in instances where the
task to be performed does not fit the model of the previous version or
if the operation has not yet been implemented in a fashion that
applies to a wide range of theories (e.g., property evaluation).
Instead of requiring \verb+theory+ and \verb+operation+ as input, the
directive needs only a string identifying the task.  The form of the
directive in such cases is as follows;

\begin{verbatim}
  TASK <string task> [ignore]
\end{verbatim}

The supported tasks that can be accessed with this directive are listed
below, with the corresponding string entry for \verb+task+.

\begin{itemize}
  \item \verb+RTDBPRINT+ --- print the contents of the database
  \item \verb+CPHF+ --- invoke the CPHF module
  \item \verb+PROPERTY+ --- miscellaneous property calculations
\end{itemize}

This directive also recognizes the keyword \verb+IGNORE+, which allows
execution to continue after a task has failed.

\subsubsection{TASK Directive for the Bourne Shell}

This form of the \verb+TASK+ directive is supported only on machines
with a fully UNIX-style operating system.  This directive causes
specified processes to be executed using the Bourne shell.  This form
of the task directive is as follows;

\begin{verbatim}
  TASK shell [(<integer-range processor = 0>|all)] <string command> [ignore]
\end{verbatim}

The keyword \verb+SHELL+ is required for this directive.  It specifies
that the given command will be executed in the Bourne shell.  The user
can also specify which process(es) will be execute this command by
entering values for \verb+processor+ on the directive.  The default is
for only process zero to execute the command.  A range of processors
may be specified, using Fortran triplet notation.  Alternatively, all
processes can be specified simply by entering the keyword \verb+all+.
The input entered for \verb+command+ must form a single string, and
comprise valid UNIX command(s).  If the string includes white space,
it must be enclosed in double quotes.

For example, the \verb+TASK+ directive to tell process zero to copy the 
molecular orbitals file to a backup location can be input as follows;

\begin{verbatim}
  task shell "cp *.movecs /piofs/save"
\end{verbatim}

The \verb+TASK+ directive to tell all processes to list the contents of 
their \verb+/scratch+ directory is as follows;

\begin{verbatim}
  task shell all "ls -l /scratch"
\end{verbatim}

The \verb+TASK+ directive to tell processes 0 to 10 to remove the 
contents of the current directory is as follows;

\begin{verbatim}
  task shell 0:10:1 "/bin/rm -f *"
\end{verbatim}

Note that NWChem's ability to quote special input characters is {\em
  very} limited when compared with that of the Bourne shell.  When
executing all but the simplest commands, it is usually much easier to
put the shell script in a file and execute the file from within
NWChem.

\subsection{CHARGE}
\label{sec:charge}

\begin{verbatim}
  CHARGE <real charge default 0>
\end{verbatim}

Specifies the total charge of the system which defaults to zero.
For example, to run a doubly charged cation
\begin{verbatim}
  charge 2
\end{verbatim}

The default charge is zero. 
