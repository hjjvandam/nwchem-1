
Top-level directives are directives that affect all modules in the code.  In
general, they provide the user with the means to start a calculation, 
allocate resources for the calculation, tell the code what to do with
the problem, when and how to write output describing the problem, and how
to stop a calculation.  As the first step in the execution of a given case,
NWChem scans the entire input file looking for start-up directives.  The
input file is then rewound and processed sequentially, and each directive
is processed in the order it is encountered.  Top-level directives are
parsed by the top-level parser.  With the exception of the {\tt task} 
directive, the top-level directives are processed only once, the first time
they are encountered in the input stream.  Subsequent occurances of the 
same directive will be ignored.

The following subsections describe each of these
directives in detail, noting all keywords, options, required input, and
defaults.

\subsection{START, CONTINUE and RESTART}
\label{sec:start}

Every calculation with NWChem requires either a {\tt start}, or a
{\tt continue}, or a {\tt restart} directive.  If one of these three
directives is not specified explicitly, the code will infer one based on
the input for the other directives and the contents of the database.
When allowing NWChem to infer the start-up directive, the user must be
quite certain that the contents of the database will result in the 
desired inference.  It is usually more prudent to specify the directive
explicitly, using the following format;

% A \verb+START+, \verb+CONTINUE+ or \verb+RESTART+ directive is
% optional but care must be taken if no directive is presented since the
% behaviour of the code will vary depending upon the existence of the
% database.

\begin{verbatim}
(RESTART || CONTINUE || START) [<string file_prefix>] \
                               [rtdb <string rtdb_file_name>]
\end{verbatim}

The \verb+START+ directive indicates that the calculation is one 
in which the database is to be created.  Any relevent information that
already exists in the database is destroyed.  The string {\tt file\_prefix}
will be used as the prefix when naming any files created in the 
course of the calculation.
The user can also specify a unique name for the database itself, using
the keyword {\tt rtdb}.  When this keyword is entered, the
string {\tt rtdb\_file\_name} is used as the database name.  If the keyword
{\tt rtbd} is omitted, the name of the database defaults to 
\verb+<file_prefix>.db+ in the directory for permanent files.

If a calculation is to start from a previous calcultion and go on
using the existing database, the \verb+RESTART+ directive or the
\verb+CONTINUE+ directive must be used.  In such a case, the database
\verb+file_prefix+ must already exist.  If a database with this name 
does not exist, use of either of these directives will generate an error.
This also means that the name {\tt file\_prefix} usually should not be 
changed when restarting a calculation.  If the name {\tt file\_prefix} is 
changed on the \verb+RESTART+ or \verb+CONTINUE+ directive, 
NWChem will not be able to
find needed files automatically when going on with the calculation.
The user will have to explicitly name the file(s) to be accessed by each
directive that needs to look at something in the old database.

The \verb+RESTART+ and \verb+CONTINUE+ directives are very nearly identical
in their effect, except for one very important difference.  When the
/verb+CONTINUE+ directive is used, no new input data is parsed before
carrying out the task stored in the database file.  When the \verb+RESTART+
directive is used, however, the input file is read and acted upon up to
the next \verb+TASK+ directive.  This can result in the overwriting of
existing information in the database, and may result in the old task in
the database being discarded.
% indicate that an existing database should be used and
% an error will result if it does not exist.  All files created by
% NWChem are by default prefixed with the string specified on the
% directive, which defaults to the basename of the input file (i.e., the
% input file name with the directory and any trailing \verb+.*+ stripped
% off).  Note that if you change the file prefix in a restart job,
% NWChem will be unable to locate most files automatically.  The
% \verb+RTDB+ keyword permits specification of the name of the database
% which otherwise defaults
% 
% The difference between \verb+CONTINUE+ and \verb+RESTART+ is subtle
% but useful. \verb+CONTINUE+ will not parse any new input data before
% it does the task stored in the database file.  \verb+RESTART+ will
% parse new input data and overwrite any existing data if appropriate.
The \verb+CONTINUE+ directive can be viewed as telling the code to
go ahead and finish the old task, then do any new tasks that are
specified by the input. The \verb+RESTART+ directive can 
be viewed as telling the code to take instruction before continuing.  The
new instructions may or may not require it to finish the old task before 
it goes on and does the new task(s) specified.

%  restart with information available, new input is
% also parsed, and do the tasks outlined in the input so the old task
% may be discarded.  \verb+CONTINUE+ implies restart the old task and
% \verb+RESTART+ implies restart with possibly a new task.

% For example, in a SCF energy calculation where you underestimated the
% time necessary and the job was killed by the scheduling software.  
% You could use 
% either the \verb+CONTINUE+ or the \verb+RESTART+ directive with the
% appropriate file prefix and database specifiction to restart the job.
% If,  On the other hand if you wanted to finish the task and do an
% additional SCF energy computation with a new geometry you could use
% \verb+CONTINUE+ directive, input the new geometry and use a single
% task directive for the second energy point.  Again, the difference is
% subtle but useful.

If the input file does not contain a \verb+START+ or a \verb+CONTINUE+ or a
\verb+RESTART+ directive, NWChem assumes the following defaults;

\begin{itemize}
  \item the name of the database for the calculation is {\tt file\_prefix.db}
  \item the file prefix is {\tt file\_prefix}
\end{itemize}

If the database 
with name {\tt file\_prefix.db} does {\it not} exist, then the calculation 
is carried out as if a \verb+START+ directive had been encountered.  
If the database with name {\tt file\_prefix.db} {\it does} exist, then 
the calculation
is performed as if a \verb+RESTART+ directive had been encountered.  That is,
the input file is parsed up to the first \verb+TASK+ directive, and that
task is executed using the contents of the database.  NWChem cannot infer
a \verb+CONTINUE+ directive, but if the input makes no changes to the 
database before the first \verb+TASK+ directive, the effect is the same as 
if a \verb+CONTINUE+ directive had been specified.

% It is not possible to infer a /verb+CONTINUE+ directive.  
%   \item if the default database exists, then the job is assumed to be
%     a restart, otherwise it is assumed to be a startup.
%   \item continue is never assumed.
% \end{itemize}

For example, if the NWChem input file \verb+water.nw+ does not contain
a \verb+START+, \verb+RESTART+, or \verb+CONTINUE+ directive, the default
{\tt file\_prefix} is set to {\tt water}, and the default database name is
{\tt water.db}.  If the database \verb+water.db+ does {\it not} exist already,
the code behaves as if the input file contained the directive,

\begin{verbatim}
  start water rtdb water.db
\end{verbatim}

If the database \verb+water.db+ {\it does} exist already,
the code behaves as if the input file contained the directive,

\begin{verbatim}
  restart water rtdb water.db
\end{verbatim}


\subsection{SCRATCH\_DIR and PERMANENT\_DIR}
\label{sec:dirs}

% This is a startup directive (section \ref{sec:inputstructure}).

This is a startup directive that allows the user to specify the directory
location of permanent and scratch files created by NWChem.  The programming
model and the local disk I/O access model used by the code is installation
dependent, and the user should refer to the Programmer's Manual for a
complete discussion of these topics.  
% NWChem distinguishes between permanent (or persistent) files and scratch (or
% temporary) files, and allows the user the option of putting them in 
% different locations.
In most installations, however, permanent and scratch files are all 
written to the current directory by default.  This is at the discretion
of the specific installation, however, and may be different on different
machines.  What constitutes 'local' disk space may also differ from
machine to machine.

% The default location of the scratch and permanent directories is
% installation specific, but most commonly both permanent and scratch
% files default to the current directory.

For example, on the IBM SP and workstation clusters, machine/process 
specific names 
must be specified for both local and shared file systems.
On the KSR it is useful to specify scratch file directories with
automated striping across processors with round-robin allocation.  SMP
clusters have both requirements.  The anticipated usage of the EMSL
IBM SP has NWChem executing in /verb+/scratch+ with all files
being created there by default.  Files to be saved for restart 
{\em must} be either explicitly redirected or copied.  This has the
advantages of minimizing very expensive network traffic on shared-file
systems and also reduces the amount of data stored in the limited
shared-file space.  However, this manual copying fails if the job crashes or
runs out of time, so it may be preferable to automatically store
{\em small} restart files directly to a persistent file space.

The \verb+SCRATCH_DIR+ and \verb+PERMANENT_DIR+ directives have
identical format and capability, and enable the user to specify a single
directory for all processes, or different directories for different
processes.  The general form of the directive is as follows;

\begin{verbatim}
   (PERMANENT_DIR || SCRATCH_DIR) [(<string host>|<integer process>):] \
                                       <string directory> \ 
                                  [...]
\end{verbatim}

Directories are extracted from the user input by executing the
following steps
\begin{enumerate}
\item Look for a directory qualified by the process number of the
  invoking process.
\item If there is a list of directories qualified by the name of the
  host machine (as returned by \verb+util_hostname()+), then use
  round-robin allocation from the list for processes executing on the
  given host.  (NOTE: the user must know the logical I/O access model
for the current installation to determine the process numbering procedure,
and should refer to the NWChem Programmer's Manual for a complete
discussion of the hostname utility.)
\item If there is a list of directories unqualified by any hostname
  or process ID, then use round-robin allocation from this list.
\end{enumerate}
If directory allocation directive(s) are not specified in the
input file, or if no match is found to the directory names specified by 
input using these directives,
then the above steps are executed using the installation specific
defaults.  If the code cannot find a valid directory name based on the
input specified in either the directive(s) or the system defaults, files are
automatically written to the current working directory (\verb+"."+).

The following is a list of examples of specific allocations of scratch
directory locations:
\begin{itemize}
\item put scratch files from all processes in the local scratch directory; 
(NOTE: the definition of 'local scratch directory' may change from 
machine to machine.)
\begin{verbatim}
      scratch_dir /localscratch
\end{verbatim}
\item put scratch files from Process 0 in \verb+/piofs/rjh+, but put all 
other scratch files in \verb+/scratch+;
\begin{verbatim}
      scratch_dir /scratch 0:/piofs/rjh
\end{verbatim}
\item put scratch files from Process 1 in directory \verb+scr1+, those from
Process 2 in \verb+scr2+, and so forth, in a round-robin fashion, using the
given list of directories
\begin{verbatim}
      scratch_dir /scr1 /scr2 /scr3 /scr4 /scr5
\end{verbatim}
\item allocate files in a round-robin fahion from
  host-specific lists for processes distributed across two
 SGI multi-processor machines (node names {\it coho} and {\it bohr})
\begin{verbatim}
      scratch_dir coho:/xfs1/rjh coho:/xfs2/rjh coho:/xfs3/rjh \
                  bohr:/disk01/rjh bohr/rjh:/disk02/rjh bohr:/disk13/rjh
\end{verbatim}
\end{itemize}

\subsection{MEMORY}

% This is a startup directive (section \ref{sec:inputstructure}).

This is a start-up directive that allows the user to allocate the amount
of available memory that NWChem can have for a particular calculation.
If this directive is not specified explicitly in the input stream, memory
is allocated according to installation
dependent defaults.  These generally should suffice for most calculations,
since in many cases they represent the total amount of memory available on
the machine.  (For example, the default for the total memory allocation
on the Intel Delta is the maximum available).  The general form of the 
directive is as follows;


\begin{verbatim}
  MEMORY [[total] <integer total_size>] \
         [stack <integer stack_size>] \
         [heap <integer heap_size>] \
         [global <integer global_size>] \
         [units <string units default "real">] \
         [(verify||noverify)] \
         [(nohardfail||hardfail)] \
\end{verbatim}

The input for
this directive assumes that there are three distinct regions of
memory: stack, heap, and global.  Stack and heap are node-private,
while the global region on all processors is used to provide
globally-shared memory.  NWChem
recognizes the following memory units;

\begin{itemize}
\item \verb+real+
\item \verb+double+ (synonyms)
\item \verb+integer+
\item \verb+byte+
\item \verb+kb+ (kilobytes)
\item \verb+mb+ (megabytes)
\item \verb+mw+ (megawords, 64 bit word)
\end{itemize}

In most cases the user needs to specify only the total memory limit, and
allow the limits on each category to be determined from a default
partitioning (currently 10\% for heap, 10\% for stack, and 80\% for
global).  Alternatively, the keywords \verb+stack+, \verb+heap+, and
\verb+global+ can be used to define specific allocations for each of
these categories.  If the user sets only one of the stack, heap, or global
limits by input, the limits for the other two categories are obtained
by partitioning the remainder of the total memory available in proportion 
to the weight of those two categories in the default memory
partitioning.  If two of the category limits are given, the third is
obtained simply by subtracting the two given limits from the total limit
(which may have been specified or may be a default value).  If all three
category limits are specified, they determine the total memory allocated.
However,  if the
total memory is also specified, it must be larger than the sum of the
three categories.  In all cases, the code
attempts to check for inconsistent memory specifications.

The following specifications all provide eight megabytes (assuming 64
bit floating point numbers) of total memory to the application, which
will be distributed according to the default partitioning.
\begin{verbatim}
  memory 1000000
  memory 1000000 real
  memory 1 mw
  memory 8 mb
\end{verbatim}
These memory directive also allocate 8 MB, but specify a complete
partitioning as well.
\begin{verbatim}
  memory total 8 stack 2 heap 2 global 4 mb
  memory stack 2 heap 2 global 4 mb
\end{verbatim}

The keywords \verb+VERIFY+ and \verb+NOVERIFY+ give the user the option
of enabling or disabling automatic
detection of corruption of allocated memory.  The default is \verb+VERIFY+,
which enables the feature. This incurs a small overhead, which can be 
eliminated by
specifying \verb+NOVERIFY+.  

The keywords \verb+HARDFAIL+ and \verb+NOHARDFAIL+  give the user the
option of forcing an internal fatal error to be generated by the local
memory management routines if any memory operation fails.  The default
in the code is \verb+NOHARDFAIL+, which allows the code to continue past
the memory operation failure, and perhaps generate a more meaningful error
message before terminating the calculation.   This can be useful when
poorly coded applications do not check the return status of memory
management routines.


When specifying the specific memory allocations using the keywords
\verb+stack+, \verb+heap+, and \verb+global+ in the \verb+MEMORY+
directive, the user should be aware that some of the distinctions between 
these categories of memory have been blurred in their actual implementation 
in the code.  The memory allocator
(MA) allocates both the heap and the stack from a single memory region
of size {\em heap+stack} without enforcing the partition.  The heap
vs. stack partition is meaningful only to applications
developers, and can be ignored by most users.  Further complicating matters,
the global array (GA) toolkit is allocated from within the MA space on
distributed memory machines, while on shared-memory machines it is
separate\footnote{This is because on true shared-memory machines there
is no choice but to allocate GAs from within a shared-memory segment,
which is mangaged differently by the operating system.}.  On
distributed memory platforms, the MA region is actually the total size of
{\tt stack+heap+global}.  All three types of memory allocation compete
for the same pool of memory, with no limits except on the total
available memory.  This relaxation of the memory category definitions 
usually benefits the user since it can allow allocation requests to 
succeed where a stricter memory model would cause the directive to fail.  
These implementation characteristics must be kept in mind when
reading output from the program relating to memory usage, however,
particularly when interpreting output of jobs that have
failed due to memory limitations.

Current standard defaults for various platforms are listed in Table
\ref{tbl:default-memory-limits}.  On machines for which individual
processors are commonly used in single user mode (e.g., IBM SP-X,
Linux laptops, CRAY-T3D, Intel Paragon, KSR), the defaults reflect
the maximum memory available to applications with common hardware
configurations.

\begin{table}
\caption{Default total memory limits according to hardware platform.}
\label{tbl:default-memory-limits}
\begin{tabular}{lr}
\hline\hline
Platform        & Total Memory Limit (MBytes) \\
\hline
CRAY-T3D        & 40 \\
DECOSF          & 48 \\
IBM RS/6000     & 56 \\
IBM SP-X        & 90 \\
Intel Paragon   & 16 \\
KSR             & 20 \\
Linux           & 16 \\
SGI             & 48 \\
Sun             & 48 \\
\hline\hline
\end{tabular}
\end{table}

\subsection{ECHO}

% This is a startup directive (section \ref{sec:inputstructure}).

This directive is provided as a convenient means to include a listing of
the input file in the output of a calculation.  It causes the entire input 
file to be printed to Fortran unit six.  It has no keywords, arguments, or
options, and consists of the single line,

\begin{verbatim}
  ECHO
\end{verbatim}

The \verb+ECHO+ directive is a start-up directive.  It is processed only
once, by Process 0 when the input file is read.

\subsection{TITLE}

This directive allows the user to identify a job or series of jobs that use a
particular database.  It is an optional directive, and if omitted, the 
character string containing the input title will be empty.  The format for 
the directive is as follows;

\begin{verbatim}
  TITLE 
  <string title>
\end{verbatim}

The character string \verb+title+ is assigned to the contents of the
line following the \verb+TITLE+ directive.  For example,

\begin{verbatim}
  title
  This is the title of my first NWChem job
\end{verbatim}

Alternatively, the character string can be specified on the same line
as the directive keyword.  This option is obtained by placing a semicolon
after the directive keyword.  In such case, the previous example would
appear as;

\begin{verbatim}
  title; This is the title of my first NWChem job
\end{verbatim}

The title is stored in the database and will be used in all subsequent
tasks/jobs unless respecified in the input.

\subsection{PRINT and NOPRINT}
\label{sec:printcontrol}

The \verb+PRINT+ and \verb+NOPRINT+ directives are top level directives 
that allow the user to define how much information to write to the output
files.  These two directives are special in that
{\em all} modules are supposed to recognize them{\footnote: If one does
not, this is probably an error, and should be reported to the developers.
Send email to \verb+nwchem-support@emsl+.}
and they  have a standard form.  Each module can control both the print 
level and the printing of individual items.  The standard form of the
\verb+PRINT+ directive is as follows;

\begin{verbatim}
  PRINT [none | low | medium | high | debug] \
        [<string list_of_names ... >]
\end{verbatim}

The form for the \verb+NOPRINT+ directive is essentially the same, except that
there is no need for the optional keywords controlling the amount of 
information to be printed.  The \verb+NOPRINT+ directive has the form,

\begin{verbatim}
  NOPRINT [<string list_of_names ... >]
\end{verbatim}

The items that are to be printed (or not printed, in the case of the
\verb+NOPRINT+ directive) can be listed explicitly in the input, or the
users can rely on the defaults.  There
is a specific set of items that each module can control the output for
by means of the \verb+PRINT/NOPRINT+ directives.
% could we tell the user what these are?  Or do we want to make them guess?
The top level of the code recognizes 7 specific items that will be printed
(or not printed) according to the input specified for these two directives.
These items are as follows;
\begin{tabbing}
  Very\_long\_descriptive\_name \= Print level space \= \kill
  Name                   \> Print Level \> Description \\
                         \>        \> \\
 ``total time''        \> medium \> Prints cpu and wall time at job end\\
 ``rtdb''              \> high    \> Print names of RTDB entries\\
 ``rtdbvalues''        \> high    \> Print name and values of RTDB entries\\
 ``ga summary''        \> medium \> Summarizes GA allocations at job end \\
 ``ma summary''        \> medium \> Summarizes MA allocations at job end \\
 ``ma stats''          \> high   \> Prints MA usage statistics at job end \\
 ``version''           \> debug  \> Prints version of all compiled routines \\
\end{tabbing}

The \verb+PRINT+ directive can be used to assign a print threshold to
each of the items that the given module controls.  The print threshold is
specified with the optional keywords \verb+none+, \verb+low+, \verb+medium+,
\verb+high+, and \verb+debug+.  The keyword applies to all items listed
in the \verb+list_of_names+ following that keyword.  If no items are listed
explicitly, the keyword applies to all print items in the module.

The keywords are interpreted as follows;

\begin{itemize}
\item \verb+none+ -- none of the items will be printed except for those
specifically listed in the directive
\item \verb+low+ -- all named items will be printed with a 'low' print level
\item \verb+medium+ -- all named items will be printed with a 'medium' print
level, as well as all items that have a 'low' print level
\item \verb+high+ -- all named items will be printed with a 'high' print
level, as well as all items that have 'low' or 'medium' print level
\item \verb+debug+ -- all named items will be printed with a special debug print level
\end{itemize}

If the \verb+PRINT+ directive appears without a list of names, then the
selected print level applies to all items that the module can print.
Multiple \verb+PRINT+ directives can be used within a given module.  Naming
specific strings with a particular print level in a directive will force the
items to be printed, regardless of the print level specified in any other 
\verb+PRINT+ directive(s) that apply to all items in the module.

   
% Items that are under standard print control are
% assigned a print threshold which is one of none, low, medium, high and
% debug.  
% An example will help.  If the print level is medium (which is the
% default) all items with a print threshold less than or equal to medium
% (i.e., low or medium) will be printed.  In addition, the printing of
% items can be explicitly enabled or disabled by name.  

The following example shows how a print directive for the top level process 
can be used to limit printout to only essential information.
The directive is,

\begin{verbatim}
  print none 'ma stats' rtdb
\end{verbatim}

This directive instructs the NWChem driver routine to print nothing, except
for the memory usage statistics stored in \verb+'ma stats'+ and the entire 
contents of the database at the end of the job. 

%Look in the documentation of each module to learn about
The documentation in the Programmer's Manual of the specific modules 
in the code contains lists of the printout item names specific
to those modules.  Currently, the print level is local to each module
and defaults to 'medium' within each module.  It is also possible to
change the implementation so that a module inherits the default print level 
of its calling layer.  (Users should send their preferences for these
defaults via email to \verb+nwchem-support@emsl+, for consideration in
future releases of the code.)

\subsection{SET}
\label{sec:set}

The \verb+SET+ directive allows the user to enter data directly into the
database.  It also permits the user to substitute new objects for named
objects in the database.  The format of the directive is
as follows;

\begin{verbatim}
  SET <name> [<string type default automatic>] <$type$ data>
\end{verbatim}

The \verb+<name>+ is the name of some array or object that exists in
the database.  This must be specified explicitly; there is no default.
The second item allows the user to define a string specifying the type
of data in the array \verb+<name>+.  The data type can be explicitly 
specified as \verb+integer+,
\verb+real+, \verb+double+, \verb+logical+, or \verb+string+.  If
this item is omitted from the directive, the value for \verb+type+ is
inferred from the data type of the {\em first} element of
\verb+<$type$ data>+.  In such a case, floating-point data entered in 
\verb+<$type$ data>+ using this directive must include either an 
exponent or decimal point to ensure that the correct default type will 
be assumed.  The correct default type will be inferred for logical values 
if logical-true values are specified as \verb+.true.+, \verb+true+, or
\verb+t+, and logical-false values are entered as \verb+.false.+, 
\verb+false+, or \verb+f+.

The main purpose of the \verb+SET+ directive is to allow the user to
define indirect links between different objects or arrays in the database.  
Indirection is a very useful 
feature in that it allows the user to have multiple objects, such as
basis sets and geometry objects, in the same database under unique 
names.  These objects can then be assigned to the default basis set or 
object names for different tasks, at the
user's discretion.  For example, if the user wants the \verb+GEOMETRY+
directive to load a different geometry object for two different tasks in
the course of a calculation, it is necessary only to assign the appropriate
object name, using the \verb+SET+ directive, for each task.  To use two 
different geometry objects for a calculation with the water molecule, one
task might have the directive,

\begin{verbatim}
SET geometry ``my water geometry''
\end{verbatim}

A subsequent task might contain the directive,

\begin{verbatim}
SET geometry ``my other water geometry''
\end{verbatim}

Similar indirection can be used when different basis set objects 
are specified in the data base.  For example, two different basis
sets representing the molecular orbitals could be specified in the input 
using the following directives;

\begin{verbatim}
  basis "trial basis"
    o library sto-3g
    h library sto-3g
  end

  basis "production basis"
    o library cc-pvqz
    h library cc-pvqz
  end
\end{verbatim}

The code expects the basis set to have the name "ao basis".
The \verb+SET+ directive can be used to assign the basis set named
"trial basis" to the "ao basis", using the input line,

\begin{verbatim}
  set "ao basis" "trial basis"
\end{verbatim}

In a subsequent task in the same input stream, the basis can be redefined
using the directive

\begin{verbatim}
  set "ao basis" "production basis"
\end{verbatim}

The \+SET+ directive can also be used as an indirect means of 
supplying input to modules under development.  It can also allow the more 
intrepid users of the code to access modules that do not yet have
input routines.  For example, some of the parameters used to tune the
integral evaluation routines may be adjusted in this fashion.  The directive 
to set the variable \verb+int:cando_sp+ to a 
logical-false value is as follows,
% The other common circumstance in which \verb+SET+ is used is for
% modules under development.  \verb+SET+ can be used to provide a
% rudimentary input mechanism for options that do not yet have more
% friendly input routines, or for options that are very infrequently
% exercised.  For instance, some of the parameters
 
\begin{verbatim}
  set int:cando_sp logical false
\end{verbatim}

\subsection{UNSET}
\label{sec:unset}

This directive gives the user a way to delete simple entries from the
database.  It is especially useful as a  means to undo the assignment 
of an array that has been specified using the \verb+SET+ directive.  
The form of the \verb+UNSET+ directive is as follows;

\begin{verbatim}
  UNSET <string name>[*]
\end{verbatim}

This directive cannot be used with complex objects such as geometries and
basis sets, however.  The input string \verb+name+ must be an array, not
an object name. A wildcard (*) specified at the end of the 
string \verb+name+ will cause {\em all}
entries beginning with that name to be deleted.  This is very useful as a
way to reset modules to their default behaviour, since modules typically store
information in the database with names that begin with
\verb+module:+.

An example illustrating how to reverse the above example of the
\verb+SET+ directive setting the logical variable \verb+int:cando_sp+ 
to the value 'false' is as follows;

\begin{verbatim}
  unset int:cando_sp
\end{verbatim}



\subsection{STOP}

\begin{verbatim}
  STOP
\end{verbatim}

As soon as this directive is encountered all processing ceases, and
the calculation terminates with an error condition.  It is useful for
verifying input.

\subsection{TASK}
\label{sec:task}

The \verb+TASK+ directive is used to tell the code what to do with the input.
The input directives are parsed
sequentially until a \verb+TASK+ directive is encountered, as described
in Section \ref{sec:inputstructure}  At that
point, the calculation or operation specified in the \verb+TASK+ directive
is performed.  When that task is completed,
the code looks for additional input to process until the next
\verb+TASK+ directive is encountered, which is then executed.  This 
process continues until the end of the input file, or an \verb+EOF+
direccive is encountered.  The code expects the last directive before the
end-of-file to be a \verb+TASK+ directive.  If it not, a warning message 
is printed.  Since the
database is persistent, multiple tasks within one job behave {\em
  exactly} the same as multiple restart jobs with the same sequence of
input.

In NWChem, a 'task' is some operation or calculattion performed on the 
database using some particular theory or module.  There are three forms
of the the \verb+TASK+ directive, corresponding to the three main uses
of the code.  One is used to perform most elctronic structure calculations.
Another is used to perform tasks such as printing the contents of the
database, or performing propety evaluations.  The third is used to
execute UNIX commands on machines having a Bourne shell.  The input options,
keywords, and defaults for each of these three forms for the \verb+TASK+
directive are discussed in the following subsections.

\subsubsection{Task Directive for Electronic Structure Calculations}

This is the most commonly used version of the \verb+TASK+ directive, and
it has the following form;

\begin{verbatim}
  TASK <string theory> [<string operation default energy>] [ignore]
\end{verbatim}

The string \verb+theory+ specifies the level of theory to be used in the
calculations for this task.  NWChem currently supports eleven different
options for this function.  the following list briefly describes the
choices and gives the string input to identify each selection.

\begin{itemize}
\item \verb+theory+ \newline
\begin{itemize}
 \item \verb+SCF+ --- Hartree-Fock
 \item \verb+DFT+ --- Density functional theory for molecules
 \item \verb+RIMP2+ --- MP2 using the RI approximation
 \item \verb+OIMP2+ --- Orbital invariant MP2
 \item \verb+DIRECT_MP2+ --- MP2 direct from the AO integrals
 \item \verb+SEMI_DIR_MP2+ --- MP2 semi-direct from the AO integrals
 \item \verb+CCSD+ --- Coupled-cluster single and double excitations
 \item \verb+MCSCF+ --- Multiconfiguration SCF
 \item \verb+SELCI+ --- Selected configuration interaction with perturbation
   correction 
 \item \verb+MD+ --- Molecular dynamics simulation using nwARGOS
 \item \verb+MD_IDEAZ+ --- Molecular dynamics simulation using IDEAZ
\end{itemize}

The string \verb+operation+ is used to specify the calculation that will
be performed in the task.  The default operation is a single point energy
evaluation.  The following list gives the selection of operations currently
available in NWChem, and the string entry for \verb+operation+ for each one.

\item Operation \newline
\begin{itemize}
 \item \verb+ENERGY+ --- single point energy evaluation
 \item \verb+GRADIENT+ --- evaluate the derivative of the energy w.r.t.\
   nuclear coordinates
 \item \verb+OPTIMIZE+ --- optimize the energy by variation of the molecular
   structure.  This is presently driven by the Stepper module
   (section \ref{sec:stepper}). 
 \item \verb+FREQUENCIES+ or \verb+FREQ+ --- compute second-derivatives and print
   out an analysis of molecular vibrations.
 \item \verb+DYNAMICS+ --- molecular dynamics using NWArgos.
 \item \verb+THERMODYNAMICS+ --- multiconfiguration thermodynamic integration
    using NWArgos.
\end{itemize} 
\end{itemize}

The user should be aware that some of these operations require
computation of derivatives of the energy.
%Shall we tell them which ones, or was everyone born knowing this?
If analytical derivatives are not available, then they must be
computed numerically, which can be very expensive.

Here are some examples of the \verb+TASK+ directive, to illustrate
the input needed to specify particular calculations with the code.  
To perform a single point energy evaluation using Hartree-Fock theory,
the directive is very simple, since the energy evaluation is the default
for the string \verb+operation+.  The input line is simply,

\begin{verbatim}
  task scf
\end{verbatim}

Alternatively, in case the string \verb+operation+ could have
been reset to some other value by an earlier \verb+TASK+ directive, the
current \verb+TASK+ directive can be specified as

\begin{verbatim}
  task scf energy
\end{verbatim}

Similarly, to perform a geometry optimization using density functional
theory, the \verb+TASK+ directive is

\begin{verbatim}
  task dft optimize
\end{verbatim}


By default, the program terminates when a task does not complete 
successfully.  The keyword \verb+IGNORE+ can be used to prevent this.
When a \verb+TASK+ directive includes the keyword \verb+IGNORE+, 
task failure results in only a warning message being printed, and 
code execution continues with the next task.

\subsubsection{Task Directive for Special Operations}

This form of the \verb+TASK+ directive is used in instances where 
the task to be performed does not fit the model of the previous 
version (e.g., print the contents of the
database) or if the operation has not yet been implemented in a
fashion that applies to a wide range of theories (e.g., property
evaluation).  Instead of requiring \verb+theory+ and \verb+operation+
as input, it needs only a string identifying the task.  The form of the
directive in such cases is as follows;

\begin{verbatim}
  TASK <string task> [ignore]
\end{verbatim}

The supported tasks that can be accessed with this directive are listed
here, with the corresponding string entry for \verb+task+.

\begin{itemize}
  \item \verb+RTDBPRINT+ --- print the contents of the database
  \item \verb+MD+ --- molecular dynamics using Ideaz
  \item \verb+CPHF+ --- invoke the CPHF module
  \item \verb+PROPERTY+ --- miscellaneous property calculations
\end{itemize}

This directive also recognizes the keyword \verb+IGNORE+, which allows
execution to continue after a task has failed.

\subsubsection{Task Directive for the Bourne Shell}

This form of the \verb+TASK+ directive is supported only on machines 
with a fully UNIX-style operating system.  This directive causes specified
processes to be executed using the Bourne shell.  The final form of this
version of the task directive is as follows;

\begin{verbatim}
  TASK shell [(<integer-range processor = 0>|all)] <string command> [ignore]
\end{verbatim}

The keyword \verb+SHELL+ is required for this directive.  It specifies
that the task defined by the string \verb+command+ will be executed in
the Bourne shell.  The user can also specify which processor(s) will be
executed in the Bourne shell, by entering values for \verb+processor+ on
the directive.  The default is to execute only processor zero in the 
Bourne shell.  A range of processors may be specified, using Fortran
triple notattion.  Alternatively, all processors can be specified simply
by entering the keyword
\verb+all+.

The input entered for \verb+command+ must form a single string, and comprise
valid UNIX command(s).  If the string includes white space, it must be 
enclosed in double quotes.

For example, the \verb+TASK+ directive to tell process zero to copy the 
molecular orbitals file to a backup location can be input as follows;

\begin{verbatim}
  task shell "cp *.movecs /piofs/save"
\end{verbatim}

The \verb+TASK+ directive to tell all processes to list the contents of 
their \verb+/scratch+ directory is

\begin{verbatim}
  task shell all "ls -l /scratch"
\end{verbatim}

The \verb+TASK+ directive to tell processes 0 to 10 to remove the 
contents of the current directory is

\begin{verbatim}
  task shell 0:10:1 "/bin/rm -f *"
\end{verbatim}

Note that NWChem's ability to  quote
special input characters is {\em very} limited when compared with that
of the shell.  So when executing all 
but the simplest commands, it is usually much easier to put the shell 
script in a file and execute the file from within NWChem.

\subsection{CHARGE}
\label{sec:charge}

\begin{verbatim}
  CHARGE <real charge default 0>
\end{verbatim}

Specifies the total charge of the system which defaults to zero.
E.g., to run a doubly charged cation
\begin{verbatim}
  charge 2
\end{verbatim}
