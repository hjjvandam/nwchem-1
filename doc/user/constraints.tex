\label{sec:constraints}

The constraints directive  allows the user to specify which constraints
should be imposed on the system during the geometry optimization. Currently
such constraints are limited to fixed atom positions and 
harmonic restraints (springs) on the distance between the two atoms. The
general form of constraints directive is presented below: 
\begin{verbatim}
  CONSTRAINTS [string name ] \
           [clear] \
           [enable||disable] \
           [disable] \
           [fix atom <integer list>] \
           [spring bond <integer atom1> <integer atom2> <real k> <real r0> ]
  END
          
\end{verbatim}
The keywords are described below 
\begin{itemize}
\item name -- optional keyword that associates a name with a given set of
constraints. Any unnamed set of constraints will be given a name ''default''
and will be automatically loaded prior to a calculation. Any constraints
with the name other than ''default''  will have to be loaded manually using
SET\ directive. For example,
\end{itemize}

\begin{verbatim}
  CONSTRAINTS one
    spring bond 1 3 5.0 1.3
    fix atom 1
  END
   
  #the above constraints can be loaded using set directive
  set constraints one
  ....
  task ....
\end{verbatim}

\begin{itemize}
\item clear -- destroys any prior constraint information.  This may be
useful when the same constraints have to be redefined or completely removed from
the runtime database.

\item enable||disable -- enables or
disables particular set of constraints without actually removing the
information from the runtime database.

\item fix atom -- fixes atom positions during geometry optimization. This
directive requires an integer list that specifies which atoms are to be
fixed. This directive can be repeated with a given constraints block. Let us
consider a case where atoms 1, 3, 4, 5, 6 need to be fixed. There are several
ways to enter this particular constraint. 
There is a straightforward way which requires the most typing

\begin{verbatim}
  constraints
    fix atom 1 3 4 5 6
  end
\end{verbatim}

Second method uses list input
\begin{verbatim}
  constraints
    fix atom 1 3:6
  end
\end{verbatim}

Third approach uses multiple fix atom directives
\begin{verbatim}
  constraints
    fix atom 1
    fix atom 3:6
  end
\end{verbatim}

\item spring bond <$i j  k r_0$> -- places a spring with a spring constant $k$ and equilibrium length $r_0$
between atoms $i$ and $j$. Note that this type of constraint adds an additional term to 
the total energy expression
\[
E=E_{total}+\frac{1}{2}k(r_{ij}-r_0)^2
\]
This additional term forces the distance between atoms $i$ and $j$ to be in the vicinity of $r_0$ but never exactly that. In general
the spring energy term will always have some nonzero residual value, and this has to be accounted for when comparing total
energies. The spring bond directive can be repeated within a given constraints block. If the spring between the same pair of atoms
is defined more than once, it will be replaced by the latest specification in the order it appears in the input block.
\end{itemize}

