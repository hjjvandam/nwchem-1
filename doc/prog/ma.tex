\label{sec:ma}

MA is a library of routines that comprises a dynamic memory allocator
for use by C, FORTRAN, or mixed-language applications.  FORTRAN
applications require such a library because the language does not
support dynamic memory allocation.  C applications can benefit from
using MA instead of the ordinary {\tt malloc()} and {\tt free()}
routines because of the extra features MA provides: both heap and
stack memory management disciplines, debugging and verification
support, usage statistics, and quantitative memory availability
information.  MA is designed to be portable across a large variety of
platforms.

Detailed information on specific routines is available in the MA man pages.
This can be accessed by means of the command, {\tt man ma}.  (Note: this 
will work only if the local environmental variable {\tt MANPATH} includes
the path {\tt \$(NWCHEM\_TOP)/src/man/ma/man}.  See Section \ref{sec:envar}
for information on system and environmental requirements for running NWChem.)
The following subsections present a summary list of the MA routines, and 
a brief discussion of the implementation of this feature.

\subsubsection{MA Data Types}

All MA memory must be explicitly assigned a specific type by defining each
data item in units of integer, logical, 
double precision, or character words.  The type of data is specified
in arguments using predefined Fortran parameters (or macros in C).
These parameters are available in the include files \verb+mafdecls.fh+ in 
Fortran and in \verb+macdecls.h+ in C.  The parameters are typed as follows:

\begin{description} 
\item{\verb+MT_INT+} --- integer
\item{\verb+MT_DBL+} --- double precision
\item{\verb+MT_LOG+} --- logical
\item{\verb+MT_CHAR+} --- character\verb+*+1
\end{description}

\subsubsection{Implementation}

To access required MA definitions, C applications should include
{\tt macdecls.h} and FORTRAN applications should include
{\tt mafdecls.fh}.  These are public header files for a dynamic memory 
allocator, and are included in the .../src/ma subdirectory of the
NWChem directory tree.  The files contain the type declarations
and parameter declarations for the datatype constants, and define
needed functions and variable types.

The memory allocator uses the following memory layout definitions:
\begin{itemize}
\item segment = heap\_region stack\_region
\item region = block block block \ldots
\item block = AD gap1 guard1 client\_space guard2 gap2
\end{itemize}

A segment of memory is obtained from the OS upon initialization.  The
low end of the segment is managed as a heap.  The heap region grows
from low addresses to high addresses.  The high end of the segment is
managed as a stack. The stack region grows from high addresses to low
addresses.

Each region consists of a series of contiguous blocks, one per
allocation request, and possibly some unused space.  Blocks in the
heap region are either in use by the client (allocated and not yet
deallocated) or not in use by the client (allocated and already
deallocated).  A block on the rightmost end of the heap region becomes
part of the unused space upon deallocation.  Blocks in the stack
region are always in use by the client, because when a stack block is
deallocated, it becomes part of the unused space.

A block consists of the client space, i.e., the range of memory
available for use by the application.  Guard words adjacent to each end
of the client space to help detect improper memory access by the
client.  Bookkeeping information is stored(?) in an "allocation descriptor" AD
Two gaps, each zero or more bytes long, are defined to satisfy alignment constraints
(specifically, to ensure that AD and client\_space are aligned
properly).  



\subsubsection{List of MA routines}

All MA routines are shown below, grouped by category and listed
alphabetically within each category.  The FORTRAN interface is given
here.  Information on the the C interface are available in the \verb+man+ pages.
(The \verb+man+ pages also contain more
detailed information on the
arguments for these routines.)

Initialization: 
\begin{itemize}
\item {\tt MA\_init(datatype, nominal\_stack, nominal\_heap)}
\begin{itemize}
\item      integer datatype
\item      integer nominal\_stack
\item      integer nominal\_heap
\end{itemize}

\item {\tt MA\_sizeof(datatype1, nelem1, datatype2)}
\begin{itemize}
\item      integer datatype1
\item      integer nelem1
\item      integer datatype2
\end{itemize}

\item {\tt MA\_sizeof\_overhead(datatype)}
\begin{itemize}
\item      integer datatype
\end{itemize}

\item {\tt MA\_initialized()}

\end{itemize}

Allocation:
\begin{itemize}
\item {\tt MA\_alloc\_get(datatype, nelem, name, memhandle, index)}
\begin{itemize}
\item      integer datatype
\item      integer nelem
\item      character*(*) name
\item      integer memhandle
\item      integer index
\end{itemize}

\item {\tt MA\_allocate\_heap(datatype, nelem, name, memhandle)}
\begin{itemize}
\item      integer datatype
\item      integer nelem
\item      character*(*) name
\item      integer memhandle
\end{itemize}

\item {\tt MA\_get\_index(memhandle, index)}
\begin{itemize}
\item      integer memhandle
\item      integer index
\end{itemize}

\item {\tt MA\_get\_pointer()} --- C only
\item {\tt MA\_inquire\_avail(datatype)}
\begin{itemize}
\item      integer datatype
\end{itemize}

\item {\tt MA\_inquire\_heap(datatype)}
\begin{itemize}
\item      integer datatype
\end{itemize}

\item {\tt MA\_inquire\_stack(datatype)}
\begin{itemize}
\item      integer datatype
\end{itemize}

\item {\tt MA\_push\_get(datatype, nelem, name, memhandle, index)}
\begin{itemize}
\item      integer datatype
\item      integer nelem
\item      character*(*) name
\item      integer memhandle
\item      integer index
\end{itemize}

\item {\tt MA\_push\_stack(datatype, nelem, name, memhandle)}
\begin{itemize}
\item      integer datatype
\item      integer nelem
\item      character*(*) name
\item      integer memhandle
\end{itemize}

\end{itemize}

Deallocation:
\begin{itemize}
\item {\tt MA\_chop\_stack(memhandle)}
\begin{itemize}
\item      integer memhandle
\end{itemize}

\item {\tt MA\_free\_heap(memhandle)}
\begin{itemize}
\item      integer memhandle
\end{itemize}

\item {\tt MA\_pop\_stack(memhandle)}
\begin{itemize}
\item      integer memhandle
\end{itemize}

\end{itemize}

Debugging:
\begin{itemize}
\item {\tt MA\_set\_auto\_verify(value)}
\begin{itemize}
\item      logical value
\item      integer ivalue
\end{itemize}

\item {\tt MA\_set\_error\_print(value)}
\begin{itemize}
\item      logical value
\item      integer ivalue
\end{itemize}

\item {\tt MA\_set\_hard\_fail(value)}
\begin{itemize}
\item      logical value
\item      integer ivalue
\end{itemize}

\item {\tt MA\_summarize\_allocated\_blocks}
\item {\tt MA\_verify\_allocator\_stuff()}
\end{itemize}

Iteration Over Allocated Blocks:
\begin{itemize}
\item {\tt MA\_get\_next\_memhandle(ithandle, memhandle)}
\begin{itemize}
\item      integer ithandle
\item      integer memhandle
\end{itemize}

\item {\tt MA\_init\_memhandle\_iterator(ithandle)}
\begin{itemize}
\item      integer ithandle
\end{itemize}

\end{itemize}

Statistics:
\begin{itemize}
\item {\tt MA\_print\_stats(oprintroutines)}
\begin{itemize}
\item      logical printroutines
\end{itemize}

\end{itemize}


\subsubsection{MA Errors}

Errors considered fatal by MA result in program termination.  Errors
considered nonfatal by MA cause the MA routine to return an error
value to the caller.  For most boolean functions, false is returned
upon failure and true is returned upon success.  (The boolean
functions for which the return value means something other than
success or failure are {\tt MA\_set\_auto\_verify()}, {\tt
  MA\_set\_error\_print()}, and {\tt MA\_set\_hard\_fail()}.)  Integer
functions return zero upon failure; depending on the function, zero
may or may not be distinguishable as an exceptional value.

An application can force MA to treat all errors as fatal via
{\tt MA\_set\_hard\_fail()}.

If a fatal error occurs, an error message is printed on the standard
error (stderr).  By default, error messages are also printed for
nonfatal errors.  An application can force MA to print or not print
error messages for nonfatal errors via {\tt MA\_set\_error\_print()}.


