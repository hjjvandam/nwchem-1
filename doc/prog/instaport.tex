\label{sec:install}

This chapter contains guidance on how to obtain a copy of NWChem and
install it on your system.  If you are very lucky you will be working
at EMSL, and will have the assistance of friendly colleagues who have
done this before, will know how to fix the process when it breaks,
and will know what has changed since the last time the documentation
was updated.
If you are not so lucky and are located off-site, you will at least have
access to a sympathetic EMSL-based collaborator and help from NWChem 
support via e-mail at {\tt nwchem-support@emsl.pnl.gov}.

The following subsections discuss some of the important considerations
when installing NWChem, and provide information on important environmental
variables, libraries, and makefiles needed to run the code.

\section{How to Obtain NWChem}

The NWChem source code tree current release is version 3.2.  All code versions, alpha and
beta, are available only to formal EMSL collaborators\footnote{For information on
how to become an EMSL collaborator, visit the EMSL Homepage at 
\verb+http://www.emspl.pnl.gov:2080/+.} with specific
needs for the software.  (This generally means an interest in developing
parallel computational chemistry software within or around NWChem, and
access to appropriate parallel computing resources.)

The NWChem source code tree (including source code or binaries only, at the
users discretion) can be downloaded from the NWChem homepage 

\begin{verbatim}
     htpp:\\www.emsl.pnl.gov:2080/docs/nwchem
\end{verbatim}

by clicking on the hot link "Download the NWChem Software" and following the
instructions as they appear.  In order to do this, you must have a signed user agreement
with EMSL, and have been issued a user name and password.  If you have any problems
using the WWW pages or forms, or getting access to the code, send e-mail to
{\tt nwchem-support@emsl.pnl.gov}.

\section{Supported Platforms}
\label{sec:platforms}
%%%%%%%%%%%%%%%%%%%%%%%%%%%%%%%%%%%%%%%%%%%%%%%%%%%%%%%%%%%%%%%%%%%%%
% NOTE: this section is adapted from the current (as of 9/29/98) version of
% the script INSTALL for NWChem, in the in the CVS repository.  If INSTALL
% has been updated since, this section should be updated, too.
%%%%%%%%%%%%%%%%%%%%%%%%%%%%%%%%%%%%%%%%%%%%%%%%%%%%%%%%%%%%%%%%%%%%%
NWChem is readily portable to essentially any sequential or parallel computer.  
The source code currently contains options for versions that will run
on the following platforms.

\begin{verbatim}

     Platform               OS/Version    Precision
    ------------------------------------------------
       Sun                    SunOS        double
       Sun                    Solaris 2.X  double
       IBM RS/6000            AIX          double
       DEC AXP                OSF/1        double
       SGI PowerChallenge     IRIX         double
       Cray T3D               UNICOS       single
       Cray T3E               UNICOS       single
       Intel Paragon          OSF/1        double
       Intel Delta            NX           double
       IBM SP1,SP2            AIX          double
       KSR2                   KSR OS       single
    -------------------------------------------------

\end{verbatim}

\subsection{Porting Notes}
\label{sec:PortingNotes}
%%%%%%%%%%%%%%%%%%%%%%%%%%%%%%%%%%%%%%%%%%%%%%%%%%%%%%%%%%%%%%%%%%%%%
% NOTE: this section is adapted from the current (as of 9/29/98) version of
% the file Porting.notes, from ~/doc/ in the CVS repository.  If Porting.notes
% has been updated since, this section should be updated, too.
%%%%%%%%%%%%%%%%%%%%%%%%%%%%%%%%%%%%%%%%%%%%%%%%%%%%%%%%%%%%%%%%%%%%%

While it is true that NWChem will run on {\em almost} any computer, there are always
a few jokers in the deck.  Here are a few that have not only been found, but were
considered sufficiently amusing to be documented.

\begin{itemize}
\item from the Intel Paragon OSF/1 R1.2.1 (discovered 16 July 1994 by DE Bernholdt);
PGI's compilation system is braindamaged in some fascinating ways:
\begin{enumerate}
\item cpp860 by default defines {\tt \_\_PARAGON\_\_} and other things, as stated in
   the {\tt man} page, but when invoked by {\tt if77}, these things are {\em not} defined.
\item ld's -L prepends directories to the search path instead of
   appending, as is done in almost every other unix compiler package
\end{enumerate}

\item from the HP-UX 9000/735, also some others (reported 08 Feb 1996 by Jarek Nieplocha):

\begin{enumerate}
\item Avoid the "free" HP C compiler - use gcc instead:
HP cc does not generate any symbols or code for several routines in one of
the GA files. To make the user's life more entertaining, there are no 
warning or error messages either -- compiler creates a junk object file
quietly and pretends that everything went well.
(Karl Anderson says: "(HP) cc is worth every penny you paid for it.")

\item {\tt fort77} instead of {\tt f77} should be used to link fortran programs since 
{\tt f77} doesn't support the {\tt -L} flag. Fortran code should be compiled with the 
{\tt +ppu} flag that adds underscores to the subroutine names.
\end{enumerate}

\end{itemize}

\section{Environmental Variables}
\label{sec:envar}

This section is a place-holder for the list of environmental variables.
