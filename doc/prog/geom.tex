
  `Geometry', unfortunately, is a bit of a misnomer since the
`geometry object' serves several purposes
\begin{itemize}
\item a definition of the coordinate system and positioning in space
  (also lattice vectors for periodic systems)
\item an association of names/tags with coordinates in space
\item a specification of the external potential (nuclear multipole
  moments, external fields, effective core potentials, \ldots) that
  defines the Hamiltonian for all electronic structure methods
\item The geometry is home for most Hamiltonian related information (and
  not wavefunction related information).
\end{itemize}

The tag associated with a geometric center is overloaded with many
different meanings
\begin{itemize}
\item an element (to provide default specification of nuclear charge,
  mass, number of electrons, \ldots)
\item as a name of an `atomic' basis set
\item as a test for symmetry equivalence (lower symmetry can be forced
     by specifying different tags for otherwise symmetry equivalent
     centers)
\item etc.
\end{itemize}

The data in (or derived from) the geometry object includes, or will
eventually include
\begin{enumerate}
\item A description of the coordinates of all types of centers (e.g.,
      atom, basis function)
\item Charges (eventually ECPs, \ldots) associated with those centers
\item Tags (names) of centers
\item Masses associated with centers
\item Variables for optimization (e.g., via constrained cartesians
      or zmatrix variables)
\item Symmetry information
\item Any other simple scalar/vector attributed associated
      specifically with a center
\end{enumerate}

Geometries are referenced through an integer handle.  In this fashion,
multiple geometries may be accessible at any instant, though since
geometries can consume a lot of memory the number of simultaneously
`open' geometries should be kept to a minimum.

All logical functions return true on sucess, false on failure.  Only
other actions are discussed below.

\subsection{Creating, destroying, loading and storing geometries}

\subsubsection{{\tt geom\_create}}
\begin{verbatim}
  logical function geom_create(geom, name)
  integer geom          [output]
  character*(*) name    [input]
\end{verbatim}
The only place from which to get a valid geometry handle.  {\tt Name}
is used only for identification in printout and subsequent creates.
If the geometry is already opened, a handle to the existing copy is
returned.

\subsubsection{{\tt geom\_destroy}}
\begin{verbatim}
  logical function geom_destroy(geom)
  integer geom          [input]
\end{verbatim}
Delete the incore data structures associated with the geometry and
make the geometry invalid for further use.

\subsubsection{{\tt geom\_check\_handle}}
\begin{verbatim}
  logical function geom_check_handle(geom, msg)
  integer geom          [input]
  character*(*) msg     [input]
\end{verbatim}
If {\tt geom} is not a valid geometry handle then print out {\tt msg}
and return \FALSE.

\subsubsection{{\tt geom\_rtdb\_load}}
\begin{verbatim}
  logical function geom_rtdb_load(rtdb, geom, name)
  integer rtdb          [input]
  integer geom          [input]
  character*(*) name    [input]
\end{verbatim}
Load named geometry from the data base.  One level of translation is
attempted upon the name --- an entry with name {\tt name} is searched
for in the database and if located the value of that entry is used as
the name of the geometry, rather than {\tt name} itself.  {\tt Geom}
must be a valid handle created by \verb+geom_create+.  The same
geometry in the databse may be loaded into distinct in-memory geometry
objects.

\subsubsection{{\tt geom\_rtdb\_store}}
\begin{verbatim}
  logical function geom_rtdb_store(rtdb, geom, name)
  integer rtdb          [input]
  integer geom          [input]
  character*(*) name    [input]
\end{verbatim}  
Store named geometry into the database.  One level of translation is
attempted upon the name.

\subsubsection{{\tt geom\_rtdb\_delete}}
\begin{verbatim}
  logical function geom_rtdb_delete(rtdb, name)
  integer rtdb          [input]
  character*(*) name    [input]
\end{verbatim}
Delete the named geometry from the data base.  One level of
translation is attempted.  Nothing happens to in-core copies of any
geometries.

\subsection{Information about the geometry}

\subsubsection{{\tt geom\_ncent}}
\begin{verbatim}
  logical function geom_ncent(geom, ncent)
  integer geom          [input]
  integer ncent         [output]
\end{verbatim}
Returns in {\tt ncent} the number of centers.

\subsubsection{{\tt geom\_nuc\_charge}}
\begin{verbatim}
  logical function geom_nuc_charge(geom, total_charge)
  integer geom                   [input]
  double precision total_charge  [output]
\end{verbatim}
Return the sum of the nuclear charges.

\subsubsection{{\tt geom\_nuc\_rep\_energy}}
\begin{verbatim}
  logical function geom_nuc_rep_energy(geom, energy)
  integer geom              [input]
  double precision energy   [output]
\end{verbatim}
Return the effective nuclear repulsion energy.  See also
\verb+geom_incude_bqbq()+ (\ref{sec:incbqbq}) and
\verb+geom_set_bqbq()+ (\ref{sec:setbqbq}).

\subsubsection{{\tt geom\_include\_bqbq}}
\label{sec:incbqbq}
\begin{verbatim}
  logical function geom_include_bqbq(geom)
  integer geom          [input]
\end{verbatim}
By default the nuclear repulsion energy returned by
\verb+geom_nuc_rep_energy+ does not include the interactions between
point-charges (i.e., centers which tag begins with \verb+bq+).  This
is so that it is easy for QM-MM programs to generate effective
Hamiltonians based on point charges and avoid double counting of
contributions.  This routine returns \TRUE or \FALSE
if the BQ-BQ contributions are or are not being computed.  The default
(don't include BQ-BQ interactions) thus corresponds to a return value
of \FALSE.

\subsubsection{{\tt geom\_set\_bqbq}}
\label{sec:setbqbq}
\begin{verbatim}
  logical function geom_set_bqbq(geom, value)
  integer geom          [input]
  logical value         [input]
\end{verbatim}
Set the logical variable that determines if BQ-BQ interactions are
included to {\tt value}.

\subsection{Information about centers and coordinates}

\subsubsection{{\tt geom\_cart\_set}}
\begin{verbatim}
  logical function geom_cart_set(geom, ncent, t, c, q)
  integer geom                [input]
  integer ncent               [input]
  character*16 t(ncent)       [input]
  double precision c(3,ncent) [input]
  double precision q(ncent)   [input]
\end{verbatim}
Simple interface for setting tags ({\tt t}), cartesian coords ({\tt
  c}) and charges ({\tt q}) for the geometry.  Atomic units are
currently assumed but might soon be able to specify what units the
interface will use.

\subsubsection{{\tt geom\_cart\_get}}
\begin{verbatim}
  logical function geom_cart_get(geom, ncent, t, c, q)
  integer geom                [input]
  integer ncent               [output]
  character*16 t(ncent)       [output]
  double precision c(3,ncent) [output]
  double precision q(ncent)   [output]
\end{verbatim}
Extracts info from the geometry (opposite of set).  The user is
responsible for determining that the arrays are of sufficient
dimension to hold the output.

\subsubsection{{\tt geom\_cent\_get}}
\begin{verbatim}  
  logical function geom_cent_get(geom, icent, t, c, q)
  integer geom          [input]
  integer icent         [input]
  character*16 t        [output]
  double precision c(3) [output]
  double precision q    [output]
\end{verbatim}
Returns tag/coords/charge about the center {\tt icent}.

\subsubsection{{\tt geom\_cent\_set}}
\begin{verbatim}
  logical function geom_cent_set(geom, icent, t, c, q)
  integer geom          [input]
  integer icent         [input]
  character*16 t        [input]
  double precision c(3) [input]
  double precision q    [input]
\end{verbatim}
Sets values for center {\tt icent} inside the geometry --- opposite of
\verb+geom_cent_get+.

\subsubsection{{\tt geom\_cent\_tag}}
\begin{verbatim}
  logical function geom_cent_tag(geom, icent, tag)
  integer geom          [input]
  integer icent         [input]
  character*16 tag      [output]
\end{verbatim}
Returns just the tag of the center {\tt icent}.

\subsubsection{{\tt geom\_check\_cent}}
\begin{verbatim}
  logical function geom_check_cent(geom, msg, icent)
  integer geom          [input]
  character*(*) msg     [input]
  integer icent         [input]
\end{verbatim}
Return \TRUE if center \verb+icent+ is a valid center,
otherwise return \FALSE and print out the message and other
information. 

\subsection{Support for periodic systems}

\subsubsection{{\tt geom\_systype\_get}}
\begin{verbatim}
  logical function geom_systype_get(geom, itype)
  integer geom          [input]
  integer itype         [input]
\end{verbatim}
Return in {\tt systype} the system type
\begin{itemize}
\item 0 = Molecule
\item 1 = Polymer
\item 2 = Slab
\item 3 = Crystal
\end{itemize}

\subsubsection{{\tt geom\_latvec\_get}}
\begin{verbatim}
  logical function geom_latvec_get(geom, vectors)
  integer geom                [input]
  double precision vectors(3) [output]
\end{verbatim}
For periodic systems, return the lattice constants ({\bf units?}).

\subsubsection{{\tt geom\_latang\_get}}
\begin{verbatim}
  logical function geom_latang_get(geom, angles)
  integer geom                [input]
  double precision angles(3)  [output]
\end{verbatim}
For periodic systems, return the angles defining the lattice.

\subsubsection{{\tt geom\_recipvec\_get}}
\begin{verbatim}
  logical function geom_recipvec_get(geom,rvectors)
  integer geom                [input]
  double precision rvectors(3)[output]
\end{verbatim}
For periodic systems, return the constants of the reciprocal lattice.

\subsubsection{{\tt geom\_recipang\_get}}
\begin{verbatim}
  logical function geom_recipang_get(geom, rangles)
  integer geom                [input]
  double precision rangles(3) [output]
\end{verbatim}
For periodic systems, return the angles defining the reciprocal
lattice ({\bf units?}).

\subsubsection{{\tt geom\_volume\_get}}
\begin{verbatim}
  logical function geom_volume_get(geom,volume)
  integer geom            [input]
  double precision volume [output]
\end{verbatim}
For periodic systems, return the volume of the unit cell ({\bf units?}).

\subsubsection{{\tt geom\_amatrix\_get} and {\tt geom\_amatinv\_get}}
\begin{verbatim}
  logical function geom_amatrix_get(geom,amat)
  integer geom                  [input]
  double precision amat(3,3)    [output]

  logical function geom_amatinv_get(geom,amatinv)
  integer geom                  [input]
  double precision amatinv(3,3) [output]
\end{verbatim}
For periodic systems, return the `A-matrix' or its inverse.  This is
the matrix that transforms fractional coordinates to a Cartesian
system in atomic units (???).  This matrix is the unit matrix for
molecular systems.

\subsection{Printing and miscellaneous routines}

\subsubsection{{\tt geom\_print} and {\tt geom\_print\_xyz}}
\begin{verbatim}
  logical function geom_print(geom)
  integer geom          [input]

  logical function geom_print_xyz(geom, unit)
  integer geom          [input]
  integer unit          [input]
\end{verbatim}
Print out the geometry to standard output.  The {\tt XYZ} form prints
the geometry out to the specified Fortran unit in the XYZ format of 
the molecular viewer {\em Xmol}.

\subsubsection{{\tt geom\_set\_user\_units}}
\begin{verbatim}
  logical function geom_set_user_units(geom, units)
  integer geom          [input]
  character*(*) units   [input]
\end{verbatim}
Set the coordinates that the user expects for input/output.  It
currently understands either `a.u.' or `angstrom'.  Note that
geometries are always internally stored as cartesians in atomic units.

\subsubsection{{\tt geom\_tag\_to\_element}}
\begin{verbatim}
  logical function geom_tag_to_element(tag, symbol, element, atn)
  character*16 tag      [input]
  character*(*) symbol  [output]
  character*(*) element [output]
  integer atn           [output]
\end{verbatim}
Attempt to interpret a tag as the name of an element.  If successful,
return the symbol, full name and atomic number.

\subsubsection{{\tt geom\_charge\_center}}
\begin{verbatim}
  logical function geom_charge_center(geom)
  integer geom          [input]
\end{verbatim}
Adjust the cartesian coordinates so that the nuclear dipole moment is
zero --- i.e., the origin of the coordinate system is at the center of
charge.

\subsubsection{{\tt geom\_num\_core}}
\begin{verbatim}
  logical function geom_num_core(geom, ncore)
  integer geom          [input]
  integer ncore         [output]
\end{verbatim}
Determines the number of core orbitals in a system based on the
constituent atoms and the standard general chemistry ideas of core and
valance.

\subsection{Bugs}

\begin{itemize}
\item It is currently only possible to create a geometry with symmetry
  from the input
\item No internal coordinates
\end{itemize}


