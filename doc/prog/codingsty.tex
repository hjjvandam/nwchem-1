\label{sec:coding-style}
%%%%%%%%%%%%%%%%%%%%%%%%%%%%%%%%%%%%%%%%%%%%%%%%%%%%%%%%%%%%%%%%%%%%%%%%%%%%%%%

In a project this large, it is necessary to impose some standards on
the coding style employed by developers.  The primary goal of these
standards is not to constrain developers, but to enhance both the
quality of the final product and its functionality. 

Code quality is somewhat subjective, but clearly embraces the ideas of
\begin{itemize}
\item correctness, 
\item maintainability, 
\item efficiency, 
\item readability, 
\item re-usability, 
\item modularity, 
\item ease of integration with other packages,
\item speed of development, 
\item density of bugs, 
\item ease of debugging, 
\item detection of errors at run time,
\item exposure of available functionality,
\item ease-of-use of the API,
\item etc..
\end{itemize}
Compromise is clearly necessary. We are interested in
high-performance, so some key kernels may sacrifice readability (but
perhaps not modularity) for efficiency, but most code (i.e., 99.9\%)
is not an inner loop in need of such optimization, as long as the
overall structure is correct.

The single most important thing you can do to achieve quality code has
little to do with programming style --- {\em it is design} --- putting
in the necessary thought and effort before even a single line of code
is written.  Do not assume that if you comply with all of the
suggestions below that you are necessarily writing quality code or
that any routine cannot be further improved.

\section{No globally defined common blocks}

Use of global variables (e.g., common blocks) is generally a bad idea.
Such variables break modularity, form hidden dependencies and make
code hard to reuse and maintain.  {\em Do not use common blocks to
  pass data between routines}.

However, common blocks are very useful in supporting a modular
programming style which encourages code reuse and improves
maintainability.  To this end common blocks can be used to hide data
behind a subroutine interface so that access to the common is limited
to a few tightly integrated routines.  The benefits of using common
blocks (smaller argument lists, static data allocation, contiguous
memory layout) can thus, with care, be realized without any problems.
Examples of this include the basis, geometry, RTDB, integral,
symmetry, global array, message passing, SCF, optimizer, input, and
MP2 libraries.

\section{Naming of routines and common blocks}

To avoid name clashes and for easy identification, prefix all
subroutine, function and common block names with the name of the
module they are associated with.  For instance,
\begin{itemize}
\item {\tt rtdb\_\ldots} --- run-time database
\item {\tt ma\_\ldots} --- memory allocator
\item {\tt ga\_\ldots} --- global array
\item {\tt scf\_\ldots} --- SCF
\item {\tt stpr\_\ldots} --- Stepper (geometry optimization)
\end{itemize}

We have already had name clashes which have wasted several days of
man-time to resolve.

\section{Inclusion of common block definitions}

All common block definitions (including typing of variables in the
common) are to be made once only in a single file (a {\tt.fh} file)
which is to be included in other source using the C preprocessor.  The
include file should document the meaning of all variables.  

This is so that variables in a common block are consistently named and
so that dependencies of routines on common blocks are easily generated
and maintained.


\section{No implicitly typed variables}

Insert {\tt implicit none} at the top of every routine.  No other
implicit statements are permitted and all variables must be explicitly
declared.  {\em This rule should be strongly enforced in new code.} It
\begin{itemize}
\item helps the compiler help you find typos and other errors,
\item makes the code more readable and more maintainable,
\item provides a natural point to document arguments and local
  variables, and
\item makes use of silly variable names like {\tt iii, ii1} both
  obvious and even more embarrassing when others catch you doing it.
\end{itemize}

When integrating existing code this rule may make more work than it is
worth, however several bugs in existing code have been found in this
fashion (ask Martyn).

\section{Use {\tt double precision} rather than {\tt real*8}}

{\tt REAL*8} is not standard Fortran.  {\tt DOUBLE PRECISION} is the
standard, it is usually what you want, it is more portable and
standardization of declarations enables us to perform more readily 
necessary code transformations.

\section{Naming of variables holding handles/pointers obtained from
  MA/GA}

So that these critical variables are immediately recognizable the
following conventions are recommended, though commenting of
variables at the point of declaration suffices if you don't want to
follow these:
\begin{itemize}
\item Handles obtained from MA should be prefaced with {\tt l\_}.
\item Pointers (into {\tt dbl\_mb()}, etc.) obtained from MA should be
  prefaced with {\tt k\_}.
\item Handles obtained from GA should be prefaced with {\tt g\_}.
\end{itemize}

\section{C macro definitions should be in upper case}

NWChem uses the ANSI C preprocessor to handle machine dependencies and
other conditional compilation requirements.  By forcing all C macros
to be upper case the code is made more readable and we also avoid
potential accidental munging of Fortran source.  This practice is
consistent with conventional use of the preprocessor in C programs.

\section{Fortran source should be in lower or mixed case}

This convention is complementary to the above C macro convention.
If there are no fully upper-case Fortran tokens then there can
be no accidental conflict with the C preprocessor.

\section{Syntax for including files using the C preprocessor}

The two different forms
\begin{itemize}
\item \verb+#include "filename"+, and
\item \verb+#include <filename>+,
\end{itemize}
mean different things.  Quoting from Kernighan and Ritchie:
\begin{quotation}
 If the {\em} filename is quoted, searching for the file typically
 begins where the source program was found; if it is not found there,
 or if the name is enclosed in \verb+<+ and \verb+>+, searching follows
 an implementation-defined rule to find the file.
\end{quotation}
For this reason, and by common convention, only system defined include
files are included using angle brackets and include files defined
within an application are included using quotes.  The
automatic generation of dependencies of source files upon include
files with NWChem {\em relies} upon this convention.

\section{Convention for naming include files}

\begin{itemize}
\item Use \verb+.fh+ for files that can be included only by Fortran
\item Use \verb+.h+ for files that can be included by C, or
  for files that are included by both C and Fortran.
\end{itemize}

\section{Parameterize message IDs}

Why use tags/IDs/types on messages?  If all messages with the program
have distinct types and the message-passing software forces the types
of messages to match between sender and receiver, then either a
runtime error will be detected or you have a proof that the messages
are being sent and received correctly.  This is especially important
to NWChem since we use many third party linear algebra libraries that
do a lot of message passing.

Modules which do any messaging should reserve a section of the message
ID space for their use (e.g., GA or PEIGS).  Most modules, however, do
only a small amount of messaging.  For these, the include file {\tt
  msgids.fh} should be used to reserve individual message IDs.  This
file defines Fortran parameters for message IDs used in most NWChem
modules referenced to a single base value.  No NWChem routine should
contain a hardwired message ID.

\section{Parameterize Fortran unit numbers}

All references to Fortran I/O units should be done with parameters or
variables instead of hardwired constants.  For the ``standard I/O''
units, corresponding to the C stdin, stdout, and stderr, you should
include the file {\em stdio.fh} and use the variables \verb+luin+,
\verb+luout+, and \verb+luerr+ instead of 5, 6, and 0.

We use very few other files, so there is nothing organized for
non-stdio units at the moment.  Parameterization helps insure that
these can be changed simply if needed, and facilitates moving to a
more general mechanism later.

\section{Comments}

The more the merrier.  At least
\begin{itemize}
\item include terse comments at the top of each subroutine to describe
  (accurately!) its function,
\item document dependencies/effects on state that are not passed
  directly through its argument list (e.g., files, the database, common
  blocks)
\item describe arguments including the flow of information (i.e.,
  label arguments as in/out/in-out)
\item document local variables whose function is not apparent
  from their names, or whose algorithmic role is opaque or obscure.
\end{itemize}

In some circumstances, comments at the top of a routine can be quite
lengthy since this is a {\tt very} good place to store details of the
algorithm.  However, too many comments in the body of the source can
impair readability.

If an interface is finalized and is to be exported for use by others,
it should be documented here in {\tt prog.tex} --- nearly all of the
documentation in this file was generated by pasting in existing
comments in the source.  {\em Automatic generation of documentation
  from code comments is being designed.}

Not much more than the following is required.  E.g., 
\begin{verbatim}
  logical function bas_numbf(basis,nbf)
  implicit none
  integer basis   ! [input] basis set handle         
  integer nbf     ! [output] number of basis functions
*
*  nbf returns the total number of functions.
*  Returns true on success, false if the handle is invalid
*  
\end{verbatim}

Or, e.g.,
\begin{verbatim}
      subroutine sym_symmetrize(geom, basis, odensity, g_a)
C$Id: codingsty.tex,v 1.1 1997-05-28 08:23:02 d3e129 Exp $
      implicit none
      integer geom, basis  ! [input] Handles
      integer g_a          ! [input] Handle to input/output GA
      logical odensity     ! [input] True if matrix is a density
c
c     Symmetrize a skeleton matrix (in a global array) in the
c     given basis set.
c
c     A <- (1/2h) * sum(R) [RT * (A + AT) * R]
c
c     where h = the order of the group and R = operators of the
c     group (including the identity)
c
c     Note that density matrices transform according to slightly
c     different rules to Hamiltonian matrices if components
c     of a shell (e.g., Cartesian d's) are not orthonormal.
c     (see Dupuis and King, IJQC 11, 613-625, 1977)
\end{verbatim}


\section{Version information}

Each source file should include a comment line that contains the CVS
revision and date information.  This is accomplished by including a
comment line containing the string \verb+$+\verb+Id+\verb+$+.  CVS
substitutes the correct version information each time the file is
checked out or updated.  These lines are processed from the source and can be
output at runtime to aid in bug-tracking.

\section{Standard print control}

All modules should understand the \verb+PRINT+ directive and
accept at least the following keywords for this
\begin{itemize}
\item \verb+none+ --- no output whatsoever except for error messages
\item \verb+low+ --- minimal output, e.g., title, critical parameters
and a final energy
\item \verb+medium+ = \verb+default+ --- usual output
\item \verb+high+ --- extra verbose output
\item \verb+debug+ --- anything useful for diagnosing problems
\end{itemize}

Ideally all applications should control most printing via the print
control routines described below (section \ref{sec:print}).  A uniform
look and feel is important.


\section{Standard interface for top-level modules}

In order to allow for automatic configuration of various modules in
a compilation of NWChem (to control the size of the executable
in memory-critical situations), all top-level modules must have a
standard interface.  Currently it looks like
\begin{verbatim}
  logical function MODULE(rtdb)
\end{verbatim}
where \verb+rtdb+ is the handle for the run-time database.  The
function should return \TRUE\ or \FALSE\ on success or failure
respectively.

The only sources of information for a module is the database, or files
whose name can be inferred from data in the database or from defaults.
Futhermore the naming of database entries is standardized:
\begin{itemize}
\item The string with which database entries are prefixed must be
  lowercase and match the module name used in the input.  E.g., input
  for the SCF module appears in the \verb+scf;...;end+ block and the
  prefix used in the databse is \verb+scf+.  This is so that the user
  can delete all state information using the \verb+UNSET+ directive.
\item Common quantities (such as energy, gradient, \ldots) should be
  stored using that name.  E.g., \verb+scf:energy+.
\end{itemize}


\section{Error handling}

All fatal errors should result in a call to \verb+errquit()+ (section
\ref{errquit}), which prints out the string and status value to both
standard error and standard output and attempts to kill all parallel
processes and to tidy any allocated system resources (e.g., system V
shared memory).

\section{Bit operations --- {\tt bitops.fh}}

{\em Dave is looking into additions to the F77 standard to see what 
is stated there --- we will adopt whatever is the standard and change
existing source accordingly.}

We have standardized upon the following bitwise operations (see Table
\ref{tabbit} for definitions) since they are readily generated using
in-line functions from most other definitions
\begin{itemize}
\item \verb+ior(i,j)+ --- inclusive OR
\item \verb+ieor(i,j)+ --- exlusive OR
\item \verb+iand(i,j)+ --- AND
\item \verb+not(i)+ --- NOT or one's complement
\item \verb+rshift(i,nbits)+ --- right shift with zero fill
\item \verb+lshift(i,nbits)+ --- left shift with zero fill
\end{itemize}
\begin{table}[h]
\begin{center}

\begin{tabular}{c|c|c|c|c|c}
 ior   &   ieor   &    iand  &   not  & lshift   & rshift   \\ \hline
       &          &          &        &          &          \\
 110   &   110    &    110   &   10   & 10111011 & 10111011 \\
 100   &   100    &    100   &        & 2 bits   & 2 bits   \\ \hline
 110   &   010    &    100   &   01   & 11101100 & 00101110 
\end{tabular}
\vspace{0.2in}
\caption{\label{tabbit} Effect of the bit operations.  The shift examples
  use an eight bit word written with the most significant bit on the
  left.}


\end{center}
\end{table}

All operations operate on full integer words (32 or 64 bit as
necessary) and produce integer results.  The declarations and any
necessary statement functions are in \verb+bitops.fh+.  The presence
of data statements makes it impossible to have a single include file
make declarations and define statement functions.  To circumvent this
the declarations are in \verb+bitops_decls.fh+ and the statement
functions are in \verb+bitops_funcs.fh+.

\section{Blockdata statements and linking}

At least one machine (the CRAY-T3D) discards all symbols that are not
explicitly referenced, even if other symbols from the same \verb+.o+
file are used.  Thus, \verb+BLOCK DATA+ subprograms are not linked in.
One fix to this was to declare each \verb+BLOCK DATA+ subprogram as an
undefined external on the link command, but this makes the link
command depend on the list of modules being built.  An alternative
mechanism that works on the T3D is to reference each \verb+BLOCK DATA+
subprogram in an \verb+EXTERNAL+ statement within a \verb+SUBROUTINE+ or
\verb+FUNCTION+ that is guaranteed to be linked if any reference is to
be made to the \verb+COMMON+ block being initialized.


