%
% $Id: integral_api.tex,v 1.2 1997-05-28 08:23:06 d3e129 Exp $
%
%-----------------------------------------------------------------%  
%                                                                 % 
%                                                                 % 
%  This latex source file should not be edited.  It is generated  % 
%  automatically from the integral API source code using "seetex" % 
%  modifications required sould be made in the source code which  % 
%  is in the source directory ".../nwchem/src/NWints/api" from    % 
%  the standard repository. If you have questions or problems     % 
%  contact Ricky Kendall at ra_kendall@pnl.gov or (509)375-2602   % 
%                                                                 % 
%                                                                 % 
%-----------------------------------------------------------------%  
\chapter{Integral Application Programmer's Interface} 
The integral (INT) Application Programmer's Interface (API) is the 
interface to the base integral technology available in the NWChem 
application software.  The INT-API interfaces currently three integral 
codes, the sp rotated axis code (from GAMESS-UK), the 
McMurchie-Davidson code (PNNL, Stave, Fr\"uchtl, and Kendall), and the 
Texas 93/95 Integral code (Wolinski and Pulay).  The API is currently 
limited to the requisite functionality of NWChem.  Further 
functionality will be added over time as requirements are determined, 
prioritized and implemented.   
 
The integral code operates as a single threaded suite and all 
parallelization is achieved at the level of the routines that call the 
API or above.  The API requires a collective initialization phase to 
determine operating parameters for the particular run based on both 
user input and the basis set specification.  The API will select the 
appropriate base integral code for the requested integrals at the time 
of each request.  Once all integral computations have completed for 
the module the termination routines should be called.   
 
Coupled initialization and termination can be executed as many times 
as required. It is imperative that the basis set object and geometry 
object are constant between initialization and termination, e.g., 
normalization must occur prior to initialization.  If this data must 
be modified then a termination and re-initialization of the integral 
API is {\bf{\it required}}. 
 
The INT-API has the following kinds of routines: 
\begin{itemize} 
\item initialization, integral accuracy and termination, 
\item memory requirements, 
\item integral routines (both shell based and blocked), 
\item derivative integral routines, 
\item property integral routines, 
\item periodic integral routines, 
\item Internal API Routines 
\end{itemize} 
 
%\subsection{API Initialization and Termination Routines} 
%int\_init, intd\_init, int\_terminate, int\_acc,  
%  
%\subsection{API Internal Routines} 
%int\_chk\_init, int\_chk\_sh, exact\_mem, exactd\_mem, cando\_sp, int\_canon, 
%int\_hf1sp 
%  
%\subsection{API include files} 
%apiP.fh, apispP.fh, int\_tim.fh, numb\_qP.fh 
%  
%\subsection{API Standard Integral Routines} 
%int\_mem int\_1cg, int\_1e3ov, int\_1eall, int\_1eh1, int\_1eke, int\_1eov, int\_1epe, 
%int\_2e2c, int\_2e3c, int\_2e4c, int\_l1e3ov, int\_l1eall, int\_l1eh1, 
%int\_l1eke, int\_l1eov, int\_l1epe, int\_l2e2c, int\_l2e3c, int\_l2e4c, 
%int\_lgen1e, int\_mpole, int\_nint, int\_pgen1e, int\_projpole 
%  
%\subsection{API blocking routines} 
%intb\_2e4c, intb\_init4c 
%  
%\subsection{API Integral Derivative Routines} 
%intd\_1eh1, intd\_1eov, intd\_2e2c, intd\_2e3c, intd\_2e4c  
%  
%\subsection{API Periodic Integral Routines} 
%intp\_* routines include only the specific translations of centers 
%required currently by the periodic DFT code. 
%  
%intp\_txyz  
%  
