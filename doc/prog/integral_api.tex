%
% $Id: integral_api.tex,v 1.8 2002-04-11 15:05:56 windus Exp $
%
% 
%-----------------------------------------------------------------%  
%                                                                 % 
%                                                                 % 
%  This latex source file should NOT be edited.  It is generated  % 
%  automatically from the integral API source code using "seetex" % 
%  modifications required sould be made in the source code which  % 
%  is in the source directory ".../nwchem/src/NWints/api" from    % 
%  the standard repository. If you have questions or problems     % 
%  contact Ricky Kendall at ra_kendall@pnl.gov or (509)375-2602   % 
%                                                                 % 
%  To make the ``current version'' of these files:                % 
%  1) cd ${NWCHEM_TOP}/src/NWints/api                            $% 
%  2) make doc                                                    % 
%                                                                 % 
%-----------------------------------------------------------------%  
% 
\chapter{Integral Application Programmer's Interface} 
The integral (INT) Application Programmer's Interface (API) is the 
interface to the base integral technology available in the NWChem 
application software.  The INT-API interfaces currently three integral 
codes, the sp rotated axis code (from GAMESS-UK), the 
McMurchie-Davidson code (PNNL, Stave, Fr\"uchtl, and Kendall), and the 
Texas 93/95 Integral code (Wolinski and Pulay).  The API is currently 
limited to the requisite functionality of NWChem.  Further 
functionality will be added over time as requirements are determined, 
prioritized and implemented.   
 
\section{Overview} 
The integral code operates as a single threaded suite and all 
parallelization is achieved at the level of the routines that call the 
API or above.  The API requires a collective initialization phase to 
determine operating parameters for the particular run based on both 
user input and the basis set specification.  The API will select the 
appropriate base integral code for the requested integrals at the time 
of each request.  Once all integral computations have completed for 
the module the termination routines should be called (in a collective 
fashion). 
 
Coupled initialization and termination can be executed as many times 
as required. It is imperative that the basis set object, ECP object, 
and the geometry object are constant between initialization and 
termination, e.g., normalization must occur prior to initialization. 
If this data must be modified then a termination and re-initialization 
of the integral API is {\it required}. 
 
The INT-API has the following kinds of routines: 
\begin{itemize} 
\item initialization, integral accuracy and termination, 
\item memory requirements, 
\item integral routines (both shell based and blocked), 
\item derivative integral routines, 
\item property integral routines, 
\item periodic integral routines, 
\item Internal API Routines 
\end{itemize} 
 
Details of the API spcification are in appendix \ref{appendix_intapi}. 
 
\section{Adding a new base integral code to the NWChem INT-API} 
 
This is a straightforward but non-trivial task.  Requirements include 
a set of APIs for the base integral code to marry it to the NWChem 
style.  The computation of integral batches (e.g., in shell quartets 
or groups of shell quartets, i.e., blocks) must be autonomous and use 
a scratch buffer passed at the time integral batch request for unified 
memory management.  Any precomputation must be done in the 
initialization phase and stored for later use.  The initialization 
routines must be based on using the NWChem basis set and geomtry data. 
This may be translated and stored for later use in the base integral 
code format but it must not require significant amounts of memory.  A 
memory estimate routine that tells the application code the amount of 
scratch memory and buffer memory that is required.  This should be 
dynamic in nature and not be a fixed dimension.  In other words, the 
memory utilization should scale with the size of the problem. 
Termination routines should completely cleanup all temporary memory 
storage that is done in the Memory Allocator.   
 
 
%\subsection{API Initialization and Termination Routines} 
%int\_init, intd\_init, int\_terminate, int\_acc,  
%  
%\subsection{API Internal Routines} 
%int\_chk\_init, int\_chk\_sh, exact\_mem, exactd\_mem, cando\_sp, int\_canon, 
%int\_hf1sp 
%  
%\subsection{API include files} 
%apiP.fh, apispP.fh, int\_tim.fh, numb\_qP.fh 
%  
%\subsection{API Standard Integral Routines} 
%int\_mem int\_1cg, int\_1e3ov, int\_1eall, int\_1eh1, int\_1eke, int\_1eov, int\_1epe, 
%int\_2e2c, int\_2e3c, int\_2e4c, int\_l1e3ov, int\_l1eall, int\_l1eh1, 
%int\_l1eke, int\_l1eov, int\_l1epe, int\_l2e2c, int\_l2e3c, int\_l2e4c, 
%int\_lgen1e, int\_mpole, int\_nint, int\_pgen1e, int\_projpole 
%  
%\subsection{API blocking routines} 
%intb\_2e4c, intb\_init4c 
%  
%\subsection{API Integral Derivative Routines} 
%intd\_1eh1, intd\_1eov, intd\_2e2c, intd\_2e3c, intd\_2e4c  
%  
%\subsection{API Periodic Integral Routines} 
%intp\_* routines include only the specific translations of centers 
%required currently by the periodic DFT code. 
%  
%intp\_txyz  
%  
