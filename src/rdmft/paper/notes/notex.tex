\documentclass{amsart}

\begin{document}

\section{H2 in 2 orbitals}

\begin{itemize}
\item Consider H2 aligned along the z-axis in a minimal basis. The atoms are labeled A and B.
\item The natural orbitals for this system are fixed by symmetry. The basis functions $\chi_A^S$ and 
        $\chi_B^S$ are normalized. I.e. $\langle \chi_A^S|\chi_A^S\rangle = 1$. However, the basis 
        functions  different atoms are not orthogonal. So $\langle \chi_A^S|\chi_B^S\rangle = S_{AB}$.
\item From these basis functions we can construct 2 natural orbitals:
         \begin{itemize}
         \item $\phi_1 = N_1 \left(\chi_A^S + \chi_B^S\right)$ where $N_1 = 1/\sqrt{2+2S_{AB}}$
         \item $\phi_2 = N_2 \left(\chi_A^S - \chi_B^S\right)$ where $N_2 = 1/\sqrt{2-2S_{AB}}$
         \end{itemize}
\item For the singlet ground state only 2 determinants contribute (others are symmetry forbidden). Hence  
        the wave function is 
        $\Psi = c_{11}|\phi_1^\alpha\phi_1^\beta\rangle + c_{22}|\phi_2^\alpha\phi_2^\beta\rangle $
\item The 1-electron density matrix represented in the natural orbitals is given by
         \begin{eqnarray}
         D_{1} &=&
         \left(\begin{matrix}
         c_{11}^2 & 0 \\
         0 & c_{22}^2
         \end{matrix}\right) \\
         &=&
         \left(\begin{matrix}
         d_{1} & 0 \\
         0 & d_{2}
         \end{matrix}\right)
         \end{eqnarray}
\item The $\alpha$-$\beta$ block of the 2-electron density matrix can be represented in
        \begin{eqnarray}
         D_{2} &=&
         \left(\begin{matrix}
         c_{11}^2 & 0 & 0 & c_{11}c_{22} \\
         0 & 0 & 0 & 0 \\
         0 & 0 & 0 & 0 \\
         c_{22}c_{11} & 0 &0 & c_{22}^2
         \end{matrix}\right)
        \end{eqnarray}
        In terms of the 1-electron density matrix occupation numbers this can also be written as
        \begin{eqnarray}
        \label{Eq:D2-2el-2orb}
         D_{2} &=&
         \left(\begin{matrix}
         \sqrt{d_1^\alpha d_1^\beta} & 0 & 0 & -\sqrt[4]{d_1^\alpha d_1^\beta d_2^\alpha d_2^\beta} \\
         0 & 0 & 0 & 0 \\
         0 & 0 & 0 & 0 \\
         -\sqrt[4]{d_1^\alpha d_1^\beta d_2^\alpha d_2^\beta} & 0 &0 & \sqrt{d_2^\alpha d_2^\beta} 
         \end{matrix}\right)
        \end{eqnarray}
\item Writing the 2-electron density as an Cartesian product of 1-electron density matrices for an 
         independent electron model gives
         \begin{eqnarray}
         D_{2} &=&
         \left(\begin{matrix}
         d_1^\alpha d_1^\beta & 0 & 0 & 0\\
         0 & d_1^\alpha d_2^\beta & 0 & 0 \\
         0 & 0 & d_2^\alpha d_1^\beta & 0 \\
         0 & 0 &0 & d_2^\alpha d_2^\beta
         \end{matrix}\right)
        \end{eqnarray}
\item The electron correlation can be obtained as
         \begin{eqnarray}
         \label{Eq:Ecorrelation-2el-2orb}
         D_{2} &=&
         \left(\begin{matrix}
         \sqrt{d_1^\alpha d_1^\beta} - d_1^\alpha d_1^\beta & 0 & 0 & -\sqrt[4]{d_1^\alpha d_1^\beta d_2^\alpha d_2^\beta} \\
         0 & - d_1^\alpha d_2^\beta & 0 & 0 \\
         0 & 0 & - d_2^\alpha d_1^\beta & 0 \\
         -\sqrt[4]{d_1^\alpha d_1^\beta d_2^\alpha d_2^\beta} & 0 &0 & \sqrt{d_2^\alpha d_2^\beta} - d_2^\alpha d_2^\beta
         \end{matrix}\right)
        \end{eqnarray}
\item An important consideration for electron correlation is that this is accounted for by orbital pair
         interactions. A particular important feature is that a pair of orbitals cannot contribute to the electron
         correlation if even 1 orbital is not correlated. That means that any energy lowering contribution in 
         Eq.(\ref{Eq:Ecorrelation-2el-2orb}) should vanish if an orbital involved has an occupation number
         of 0 or 1. As written most contributions do not do that. They all go to 0 if an orbital is empty, but
         they do not vanish if an orbital is fully occupied. Also note that this block of the 2-electron density 
         matrix represents the correlation between $\alpha$ and $\beta$ electrons.
\end{itemize}

\section{H2 in 4 orbitals}
\label{Subsect:h2-4orb}

\begin{itemize}
\item Like in the previous section we consider a basis set so that all natural orbitals are fixed by
         symmetry. In this case the basis set is chosen to contain an $s$ function and the $p_x$ function.
         The $p$-functions are also normalized but they have non-zero overlap on different centers. I.e. 
         $\langle \chi_A^{P_x}|\chi_B^{P_x}\rangle = P_{AB}$. 
\item With these basis functions the following natural orbitals can be constructed
         \begin{itemize}
         \item $\phi_1 = N_1 \left(\chi_A^S + \chi_B^S\right)$ where $N_1 = 1/\sqrt{2+2S_{AB}}$
         \item $\phi_2 = N_2 \left(\chi_A^S - \chi_B^S\right)$ where $N_2 = 1/\sqrt{2-2S_{AB}}$
         \item $\phi_3 = N_3 \left(\chi_A^{P_x} + \chi_B^{P_x}\right)$ where $N_3 = 1/\sqrt{2+2P_{AB}}$
         \item $\phi_4 = N_4 \left(\chi_A^{P_x} - \chi_B^{P_x}\right)$ where $N_4 = 1/\sqrt{2-2P_{AB}}$
         \end{itemize}
\item By symmetry the wave function is 
         $\Psi = c_{11}|\phi_1^\alpha\phi_1^\beta\rangle + c_{22}|\phi_2^\alpha\phi_2^\beta\rangle + c_{33}|\phi_3^\alpha\phi_3^\beta\rangle + c_{44}|\phi_4^\alpha\phi_4^\beta\rangle $
\item The resulting 1-electron density matrix is
         \begin{eqnarray}
         D_{1} &=&
         \left(\begin{matrix}
         c_{11}^2 & 0 & 0 & 0 \\
         0 & c_{22}^2 & 0 & 0 \\
         0 & 0 & c_{33}^2 & 0 \\
         0 & 0 & 0 & c_{44}^2
         \end{matrix}\right) \\
         &=&
         \left(\begin{matrix}
         d_{1} & 0 & 0 & 0 \\
         0 & d_{2} & 0 & 0 \\
         0 & 0 & d_{3} & 0 \\
         0 & 0 & 0 & d_{4} 
         \end{matrix}\right)
         \end{eqnarray}
\item The $\alpha$-$\beta$ block of the 2-electron density matrix becomes
         \setcounter{MaxMatrixCols}{16}
         \begin{eqnarray}
         \label{Eq:D2-2el-4orb}
         D_{2} &=&
         \begin{pmatrix}
         c_{11}^2 & 0 & 0 & 0 & 0 & c_{11}c_{22} & 0 & 0 & 0 & 0 & c_{11}c_{33} & 0 & 0 & 0 & 0 & c_{11}c_{44} \\
         0 & 0 & 0 & 0 &  0 & 0 & 0 & 0  & 0 & 0 & 0 & 0 &  0 & 0 & 0 & 0 \\
         0 & 0 & 0 & 0 &  0 & 0 & 0 & 0  & 0 & 0 & 0 & 0 &  0 & 0 & 0 & 0 \\
         0 & 0 & 0 & 0 &  0 & 0 & 0 & 0  & 0 & 0 & 0 & 0 &  0 & 0 & 0 & 0 \\
         0 & 0 & 0 & 0 &  0 & 0 & 0 & 0  & 0 & 0 & 0 & 0 &  0 & 0 & 0 & 0 \\
         c_{22}c_{11} & 0 & 0 & 0 & 0 & c_{22}^2 & 0 & 0 & 0 & 0 & c_{22}c_{33} & 0 & 0 & 0 & 0 & c_{22}c_{44} \\
         0 & 0 & 0 & 0 &  0 & 0 & 0 & 0  & 0 & 0 & 0 & 0 &  0 & 0 & 0 & 0 \\
         0 & 0 & 0 & 0 &  0 & 0 & 0 & 0  & 0 & 0 & 0 & 0 &  0 & 0 & 0 & 0 \\
         0 & 0 & 0 & 0 &  0 & 0 & 0 & 0  & 0 & 0 & 0 & 0 &  0 & 0 & 0 & 0 \\
         0 & 0 & 0 & 0 &  0 & 0 & 0 & 0  & 0 & 0 & 0 & 0 &  0 & 0 & 0 & 0 \\
         c_{33}c_{11} & 0 & 0 & 0 & 0 & c_{33}c_{22} & 0 & 0 & 0 & 0 & c_{33}^2 & 0 & 0 & 0 & 0 & c_{33}c_{44} \\
         0 & 0 & 0 & 0 &  0 & 0 & 0 & 0  & 0 & 0 & 0 & 0 &  0 & 0 & 0 & 0 \\
         0 & 0 & 0 & 0 &  0 & 0 & 0 & 0  & 0 & 0 & 0 & 0 &  0 & 0 & 0 & 0 \\
         0 & 0 & 0 & 0 &  0 & 0 & 0 & 0  & 0 & 0 & 0 & 0 &  0 & 0 & 0 & 0 \\
         0 & 0 & 0 & 0 &  0 & 0 & 0 & 0  & 0 & 0 & 0 & 0 &  0 & 0 & 0 & 0 \\
         c_{44}c_{11} & 0 & 0 & 0 & 0 & c_{44}c_{22} & 0 & 0 & 0 & 0 & c_{44}c_{33} & 0 & 0 & 0 & 0 & c_{44}^2 
         \end{pmatrix}
         \end{eqnarray}
\item As in Eq.(\ref{Eq:D2-2el-2orb}) we can assume that $c_{11} \approx 1$, i.e. the weak correlation limit,
         and therefore the other wave function coefficients have to be negative to maximally lower the energy. 
         However, if $c_{22}$ and $c_{33}$, for example, are negative then the term corresponding to
         $c_{22}c_{33}$ has to be positive. Using this information, rewriting Eq.(\ref{Eq:D2-2el-4orb}) in terms of 
         the 1-electron density matrix occupation numbers, and subtracting the independent particle 2-electron density
         matrix, the following density matrix for the $\alpha$-$\beta$ electron correlation is obtained
         \newcommand{\da}[1]{d_{#1}^\alpha}
         \newcommand{\db}[1]{d_{#1}^\beta}
         \tiny
         \begin{eqnarray}
         \label{Eq:Ecorrelation-2el-4orb}
         D_{2} &=&
         \begin{pmatrix}
         \sqrt{\da{1}\db{1}} -  \da{1}\db{1} & 0 & 0 & 0 & 0 & -\sqrt[4]{\da{1}\db{1}\da{2}\db{2}} & \ldots & -\sqrt[4]{\da{1}\db{1}\da{3}\db{3}} & \ldots & -\sqrt[4]{\da{1}\db{1}\da{4}\db{4}}  \\
         0 & -\da{1}\db{2} & 0 & 0 &  0 & 0 & \ldots & 0 & \ldots & 0 \\
         0 & 0 & -\da{1}\db{3} & 0 &  0 & 0 & \ldots & 0 & \ldots & 0 \\
         0 & 0 & 0 & -\da{1}\db{4} &  0 & 0 & \ldots & 0 & \ldots & 0 \\
         0 & 0 & 0 & 0 &  -\da{2}\db{1} & 0 & \ldots & 0 & \ldots & 0 \\
         -\sqrt[4]{\da{2}\db{2}\da{1}\db{1}}  & 0 & 0 & 0 & 0 & \sqrt{\da{2}\db{2}} -  \da{2}\db{2}  & \ldots & \sqrt[4]{\da{2}\db{2}\da{3}\db{3}} & \ldots & \sqrt[4]{\da{2}\db{2}\da{4}\db{4}} \\
         \vdots & \vdots & \vdots & \vdots &  \vdots & \vdots & \ldots & \vdots & \ldots & \vdots \\
         -\sqrt[4]{\da{3}\db{3}\da{1}\db{1}} & \vdots & \vdots & \vdots &  \vdots & \sqrt[4]{\da{3}\db{3}\da{2}\db{2}} & \ldots & \sqrt{\da{3}\db{3}} -  \da{3}\db{3}  & \ldots & \sqrt[4]{\da{3}\db{3}\da{4}\db{4}} \\
         \vdots & \vdots & \vdots & \vdots &  \vdots & \vdots & \ldots & \vdots & \ldots & \vdots \\
         -\sqrt[4]{\da{4}\db{4}\da{1}\db{1}} & 0 & 0 & 0 & 0 & \sqrt[4]{\da{4}\db{4}\da{2}\db{2}} & \ldots & \sqrt[4]{\da{4}\db{4}\da{3}\db{3}} & \ldots& \sqrt{\da{4}\db{4}} -  \da{4}\db{4} 
         \end{pmatrix}
         \end{eqnarray}
         \normalsize
\end{itemize}

\section{ triplet H2 in 4 orbitals}

\begin{itemize}
\item To see how electron correlation plays out among same spin electrons consider H2 in 
         the triplet state. As a minimal basis for correlation 4 orbitals are needed and the same
         basis set and orbitals as in subsection~\ref{Subsect:h2-4orb} are used.
\item Due to symmetry the wave function is of $B_{1u}$ irrep. The orbitals are $A_g$, $B_{1u}$, $B_{3u}$, and $B_{2g}$ symmetry
         which means there are only 2 determinants of $B_{1u}$ symmetry. Hence the wave function becomes:
         \begin{table}
         \begin{tabular}{ccccccccc}
         $D_{2h}$ & $E$ & $C_2(x)$ & $C_2(y)$ & $C_2(z)$ & $i$ & $\sigma(xy)$ & $\sigma(xz)$ & $\sigma(yz)$ \\
         \hline
         $A_g$      & +1 & +1 & +1 & +1 & +1 & +1 & +1 & +1 \\
         $B_{1u}$  & +1 & +1 & -1 & -1 & -1 & -1 & +1 & +1 \\
         $B_{3u}$  & +1 & -1 & -1 & +1 & -1 & +1 & +1 & -1 \\
         $B_{2g}$  & +1 & -1 & +1 & -1 & +1 & -1 & +1 & -1
         \end{tabular}
         \caption{$D_{2h}$ character table excerpt}
         \end{table}
         \begin{eqnarray}
         \Psi &=& c_{12}|\phi^\alpha_1\phi^\alpha_2\rangle + c_{34}|\phi^\alpha_3\phi^\alpha_4\rangle 
        \end{eqnarray}
\item The resulting 1-electron density matrix is
         \begin{eqnarray}
         D_{1} &=&
         \left(\begin{matrix}
         c_{12}^2 & 0 & 0 & 0 \\
         0 & c_{12}^2 & 0 & 0 \\
         0 & 0 & c_{34}^2 & 0 \\
         0 & 0 & 0 & c_{34}^2
         \end{matrix}\right) \\
         &=&
         \left(\begin{matrix}
         d_{1} & 0 & 0 & 0 \\
         0 & d_{2} & 0 & 0 \\
         0 & 0 & d_{3} & 0 \\
         0 & 0 & 0 & d_{4} 
         \end{matrix}\right)
         \end{eqnarray}
         where also $d_1 = d_2$ and $d_3 = d_4$.
\item In the case of the $\alpha$-$\beta$ block of the 2-electron density matrix this block can be
         expressed in terms of a Cartesian product of 1-electron density matrices. In the case under
         consideration here the 2-electron density matrix has to be constructed from the tilde 
         functions. In this case there are 2 electrons so 2 orthonormal tilde functions are needed.
        
         \begin{tabular}{c|cc}
               & $\tilde{\phi}_1$ & $\tilde{\phi}_2$ \\
         \hline
         $\phi_1$ & $\sqrt{d_1}$ & $0$ \\
         $\phi_2$ & $0$               & $\sqrt{d_2}$ \\
         $\phi_3$ & $\sqrt{d_3}$ & $0$ \\
         $\phi_4$ & $0$               & $\sqrt{d_4}$
         \end{tabular}
         
         Because of the particular structure of these tilde functions the exchange term in the
         2-electron occupation number is identical zero. The remaining term is 
         $\frac{1}{2}\left(\tilde{\phi}_1^*(\epsilon_1)\tilde{\phi}_1(\epsilon_1)\tilde{\phi}_2^*(\epsilon_2)\tilde{\phi}_2(\epsilon_2)+\tilde{\phi}_2^*(\epsilon_1)\tilde{\phi}_2(\epsilon_1)\tilde{\phi}_1^*(\epsilon_2)\tilde{\phi}_1(\epsilon_2)\right)$, giving
         \begin{eqnarray}
         \label{Eq:D2aa-1-electron}
         D_2 &=&
         \begin{pmatrix}
         \frac{1}{2}d_1 d_2 & 0 & 0 & 0 & 0 & 0 & 0 & 0 & 0 & 0 & 0 & 0 \\
         0 & 0 & 0 & 0 & 0 & 0 & 0 & 0 & 0 & 0 & 0 & 0 \\
         0 & 0 & \frac{1}{2}d_1 d_4 & 0 & 0 & 0 & 0 & 0 & 0 & 0 & 0 & 0 \\
         0 & 0 & 0 & \frac{1}{2}d_2 d_1 & 0 & 0 & 0 & 0 & 0 & 0 & 0 & 0 \\
         0 & 0 & 0 & 0 & \frac{1}{2}d_2 d_3 & 0 & 0 & 0 & 0 & 0 & 0 & 0 \\
         0 & 0 & 0 & 0 & 0 & 0 & 0 & 0 & 0 & 0 & 0 & 0 \\
         0 & 0 & 0 & 0 & 0 & 0 & 0 & 0 & 0 & 0 & 0 & 0 \\
         0 & 0 & 0 & 0 & 0 & 0 & 0 & \frac{1}{2}d_3 d_2 & 0 & 0 & 0 & 0 \\
         0 & 0 & 0 & 0 & 0 & 0 & 0 & 0 & \frac{1}{2}d_3 d_4 & 0 & 0 & 0  \\
         0 & 0 & 0 & 0 & 0 & 0 & 0 & 0 & 0 & \frac{1}{2}d_4 d_1 & 0 & 0  \\
         0 & 0 & 0 & 0 & 0 & 0 & 0 & 0 & 0 & 0 & 0 & 0  \\
         0 & 0 & 0 & 0 & 0 & 0 & 0 & 0 & 0 & 0 & 0 & \frac{1}{2}d_4 d_3  \\
         \end{pmatrix}
         \end{eqnarray}
         Note that the basis of this matrix is $|\phi^\alpha_1\phi^\alpha_2\rangle$,
         $|\phi^\alpha_1\phi^\alpha_3\rangle$, $|\phi^\alpha_1\phi^\alpha_4\rangle$,
         $|\phi^\alpha_2\phi^\alpha_1\rangle$, $|\phi^\alpha_2\phi^\alpha_3\rangle$,
         $|\phi^\alpha_2\phi^\alpha_4\rangle$, $|\phi^\alpha_3\phi^\alpha_1\rangle$,
         $|\phi^\alpha_3\phi^\alpha_2\rangle$, $|\phi^\alpha_3\phi^\alpha_4\rangle$,
         $|\phi^\alpha_4\phi^\alpha_1\rangle$, $|\phi^\alpha_4\phi^\alpha_2\rangle$,
         and $|\phi^\alpha_4\phi^\alpha_3\rangle$, everything being $0$ because of
         the antisymmetry of the wave function.
\item The $\alpha$-$\alpha$-block of the 2-electron density matrix is
         \begin{eqnarray}
         D_2 &=&
         \begin{pmatrix}
         c_{12}^*c_{12} & 0 & 0 & 0 & 0 & 0 & 0 & 0 & 0 & 0 & 0 & c_{12}^*c_{34} \\
         0 & 0 & 0 & 0 & 0 & 0 & 0 & 0 & 0 & 0 & 0 & 0 \\
         0 & 0 & 0 & 0 & 0 & 0 & 0 & 0 & 0 & 0 & 0 & 0 \\
         0 & 0 & 0 & 0 & 0 & 0 & 0 & 0 & 0 & 0 & 0 & 0 \\
         0 & 0 & 0 & 0 & 0 & 0 & 0 & 0 & 0 & 0 & 0 & 0 \\
         0 & 0 & 0 & 0 & 0 & 0 & 0 & 0 & 0 & 0 & 0 & 0 \\
         0 & 0 & 0 & 0 & 0 & 0 & 0 & 0 & 0 & 0 & 0 & 0 \\
         0 & 0 & 0 & 0 & 0 & 0 & 0 & 0 & 0 & 0 & 0 & 0 \\
         0 & 0 & 0 & 0 & 0 & 0 & 0 & 0 & 0 & 0 & 0 & 0  \\
         0 & 0 & 0 & 0 & 0 & 0 & 0 & 0 & 0 & 0 & 0 & 0  \\
         0 & 0 & 0 & 0 & 0 & 0 & 0 & 0 & 0 & 0 & 0 & 0  \\
         c_{34}^*c_{12} & 0 & 0 & 0 & 0 & 0 & 0 & 0 & 0 & 0 & 0 &  c_{34}^*c_{34} \\
         \end{pmatrix} \\
         \label{Eq:D2aa-wfn}
         &=&
         \begin{pmatrix}
         \sqrt{d_{1}d_{2}} & 0 & 0 & 0 & 0 & 0 & 0 & 0 & 0 & 0 & 0 & -\sqrt[4]{d_{1}d_{2}d_{3}d_{4}} \\
         0 & 0 & 0 & 0 & 0 & 0 & 0 & 0 & 0 & 0 & 0 & 0 \\
         0 & 0 & 0 & 0 & 0 & 0 & 0 & 0 & 0 & 0 & 0 & 0 \\
         0 & 0 & 0 & 0 & 0 & 0 & 0 & 0 & 0 & 0 & 0 & 0 \\
         0 & 0 & 0 & 0 & 0 & 0 & 0 & 0 & 0 & 0 & 0 & 0 \\
         0 & 0 & 0 & 0 & 0 & 0 & 0 & 0 & 0 & 0 & 0 & 0 \\
         0 & 0 & 0 & 0 & 0 & 0 & 0 & 0 & 0 & 0 & 0 & 0 \\
         0 & 0 & 0 & 0 & 0 & 0 & 0 & 0 & 0 & 0 & 0 & 0 \\
         0 & 0 & 0 & 0 & 0 & 0 & 0 & 0 & 0 & 0 & 0 & 0 \\
         0 & 0 & 0 & 0 & 0 & 0 & 0 & 0 & 0 & 0 & 0 & 0 \\
         0 & 0 & 0 & 0 & 0 & 0 & 0 & 0 & 0 & 0 & 0 & 0 \\
         -\sqrt[4]{d_{3}d_{4}d_{1}d_{2}} & 0 & 0 & 0 & 0 & 0 & 0 & 0 & 0 & 0 & 0 & \sqrt{d_{3}d_{4}} \\
         \end{pmatrix} 
         \end{eqnarray}
\item Combining Eq.(\ref{Eq:D2aa-1-electron}) and Eq.(\ref{Eq:D2aa-wfn}) the following 2-electron
         density matrix block we have for the 2-electron correlation density matrix
         \tiny
         \begin{eqnarray}
         \label{Eq:D2aa-correlation}
         D_2 &=&
         \begin{pmatrix}
         \sqrt{d_{1}d_{2}}-\frac{1}{2}d_1 d_2 & 0 & 0 & 0 & 0 & 0 & 0 & 0 & 0 & 0 & 0 & -\sqrt[4]{d_{1}d_{2}d_{3}d_{4}} \\
         0 & 0 & 0 & 0 & 0 & 0 & 0 & 0 & 0 & 0 & 0 & 0 \\
         0 & 0 & -\frac{1}{2}d_1 d_4 & 0 & 0 & 0 & 0 & 0 & 0 & 0 & 0 & 0 \\
         0 & 0 & 0 & -\frac{1}{2}d_2 d_1 & 0 & 0 & 0 & 0 & 0 & 0 & 0 & 0 \\
         0 & 0 & 0 & 0 & -\frac{1}{2}d_2 d_3 & 0 & 0 & 0 & 0 & 0 & 0 & 0 \\
         0 & 0 & 0 & 0 & 0 & 0 & 0 & 0 & 0 & 0 & 0 & 0 \\
         0 & 0 & 0 & 0 & 0 & 0 & 0 & 0 & 0 & 0 & 0 & 0 \\
         0 & 0 & 0 & 0 & 0 & 0 & 0 & -\frac{1}{2}d_3 d_2 & 0 & 0 & 0 & 0 \\
         0 & 0 & 0 & 0 & 0 & 0 & 0 & 0 & -\frac{1}{2}d_3 d_4 & 0 & 0 & 0  \\
         0 & 0 & 0 & 0 & 0 & 0 & 0 & 0 & 0 & -\frac{1}{2}d_4 d_1 & 0 & 0  \\
         0 & 0 & 0 & 0 & 0 & 0 & 0 & 0 & 0 & 0 & 0 & 0  \\
         -\sqrt[4]{d_{3}d_{4}d_{1}d_{2}} & 0 & 0 & 0 & 0 & 0 & 0 & 0 & 0 & 0 & 0 & \sqrt{d_{3}d_{4}}-\frac{1}{2}d_4 d_3  \\
         \end{pmatrix}
         \end{eqnarray}
         \normalsize
         Note that in this equation normally the occupation numbers of the 2-electron density matrix should appear
         in a form that cannot be cast in terms of the 1-electron density matrices. In this specific case it happens that
         it can be cast in that form. Also note that because of the constraint on the orbital pairs due to the
         anti-symmetry there are far fewer off-diagonal contributions relative to the basis size than in the case of
         $\alpha$-$\beta$ block.
\end{itemize}

\section{H4 in 4 orbitals}

\begin{itemize}
\item In this section a chemical system with more electrons than 2 is considered. 
         The system under consideration is H4 in a rectangular configuration. 
         The point group of this system is D2h and the 4 orbitals have the irreps $A_g$,
         $B_{2u}$, $B_{3u}$, and $B_{1g}$.
         The atom coordinates are of the form
         \begin{verbatim}
         H_A  -X -Y  0
         H_B  -X  Y  0
         H_C   X  Y  0
         H_D   X -Y  0
         \end{verbatim}
         where $Y > X$.
         The natural orbitals have the following nodal structure
         \begin{itemize}
         \item $\phi_1 = N_1\left(\chi_A^S + \chi_B^S + \chi_C^S + \chi_D^S\right)$
         \item $\phi_2 = N_2\left(\chi_A^S - \chi_B^S - \chi_C^S + \chi_D^S\right)$
         \item $\phi_3 = N_3\left(\chi_A^S + \chi_B^S - \chi_C^S - \chi_D^S\right)$
         \item $\phi_4 = N_4\left(\chi_A^S - \chi_B^S + \chi_C^S - \chi_D^S\right)$
         \end{itemize}
\item The wave function expanded in natural orbitals is
         \begin{eqnarray}
         \Psi &=& c_{1122}|\phi_1^\alpha\phi_1^\beta\phi_2^\alpha\phi_2^\beta\rangle +
                        c_{1133}|\phi_1^\alpha\phi_1^\beta\phi_3^\alpha\phi_3^\beta\rangle +
                        c_{1144}|\phi_1^\alpha\phi_1^\beta\phi_4^\alpha\phi_4^\beta\rangle + \nonumber\\
                 &&  c_{2233}|\phi_2^\alpha\phi_2^\beta\phi_3^\alpha\phi_3^\beta\rangle +
                        c_{2244}|\phi_2^\alpha\phi_2^\beta\phi_4^\alpha\phi_4^\beta\rangle +
                        c_{3344}|\phi_3^\alpha\phi_3^\beta\phi_4^\alpha\phi_4^\beta\rangle + \nonumber\\
                 &&  c_{1234}|\phi_1^\alpha\phi_2^\beta\phi_3^\alpha\phi_4^\beta\rangle +
                        c_{2143}|\phi_2^\alpha\phi_1^\beta\phi_4^\alpha\phi_3^\beta\rangle + 
                        c_{1243}|\phi_1^\alpha\phi_2^\beta\phi_4^\alpha\phi_3^\beta\rangle + \nonumber\\
                 &&  c_{2134}|\phi_2^\alpha\phi_1^\beta\phi_3^\alpha\phi_4^\beta\rangle + 
                        c_{1324}|\phi_1^\alpha\phi_3^\beta\phi_2^\alpha\phi_4^\beta\rangle +
                        c_{3142}|\phi_3^\alpha\phi_1^\beta\phi_4^\alpha\phi_2^\beta\rangle
         \end{eqnarray}
\item The $\alpha$ 1-electron density matrix is
         \begin{eqnarray}
         D_1 &=&
         \begin{pmatrix}
         \begin{array}{l}
         c_{1122}^2+ \\
         c_{1133}^2+ \\
         c_{1144}^2+ \\
         c_{1234}^2+ \\
         c_{1243}^2+ \\
         c_{1324}^2 
         \end{array}
         & 0 & 0 & 0 \\
         0 & 
         \begin{array}{l}
         c_{1122}^2+ \\
         c_{2233}^2+ \\
         c_{2244}^2+ \\
         c_{2143}^2+ \\
         c_{2134}^2+ \\
         c_{1324}^2
         \end{array}
         & 0 & 0 \\
         0 & 0 &
         \begin{array}{l}
         c_{1133}^2+ \\
         c_{2233}^2+ \\
         c_{3344}^2+ \\
         c_{1234}^2+ \\
         c_{2134}^2+ \\
         c_{3142}^2
         \end{array}
         & 0 \\
         0 & 0 & 0 &
         \begin{array}{l}
         c_{1144}^2+ \\
         c_{2244}^2+ \\
         c_{3344}^2+ \\
         c_{2143}^2+ \\
         c_{1243}^2+ \\
         c_{3142}^2
         \end{array} \\
         \end{pmatrix}
         \end{eqnarray}
\item The $\alpha$-$\beta$ 2-electron density matrix is
         \begin{eqnarray}
         D_2 &=&
         \begin{pmatrix}
         % 11,11
         \begin{array}{l}
         c_{1122}^2+ \\
         c_{1133}^2+ \\
         c_{1144}^2
         \end{array} &
         0 & 0 & 0 &
         % 11,21
         0 & 0 & 0 & 0 &
         % 11,31
         0 & 0 & 0 & 0 &
         % 11,41
         0 & 0 & 0 & 0 \\
         % 12,11
         0 & 0 & 0 & 0 &
         % 12,21
         \begin{array}{l}
         c_{1234}^*c_{2134}+ \\
         c_{1243}^*c_{2143}
         \end{array} &
         0 & 0 & 0 &
         % 12,31
         0 & 0 & 0 & 0 &
         % 12,41
         0 & 0 & 0 & 0 \\
         \end{pmatrix}
         \end{eqnarray}
\end{itemize}


\end{document}
