\documentclass[pra,nofootinbib]{revtex4-1}

\usepackage{hyperref}
%\usepackage{amsmath}
\usepackage[version=3]{mhchem}
\usepackage{xcolor}
\usepackage{graphicx}
\allowdisplaybreaks

\DeclareMathOperator{\tr}{Tr}
\newcommand{\eria}[4]{\left\langle #1^\alpha #2^\beta \right.\left| #3^\alpha #4^\beta \right\rangle}
\newcommand{\erib}[4]{\left\langle #1^\alpha #2^\alpha \right.\left|| #3^\alpha #4^\alpha \right\rangle}
\newcommand{\eris}[4]{\left\langle #1^\sigma #2^{\sigma'} \right.\left| #3^\sigma #4^{\sigma'} \right\rangle}
\newcommand{\reria}[4]{\left( #1^\alpha #3^\alpha \right.\left| #2^\beta #4^\beta \right)}
\newcommand{\Xs}[2]{\chi^{#1}_{#2}(r_{#2})}
\newcommand{\Xa}[1]{\Xs{\alpha}{#1}}
\newcommand{\Xb}[1]{\Xs{\beta}{#1}}
\newcommand{\XXa}[2]{|\Xs{\alpha}{#1}\Xs{\alpha}{#2}|}
\newcommand{\XXb}[2]{|\Xs{\beta}{#1}\Xs{\beta}{#2}|}
\newcommand{\Cs}[2]{C^{#1}_{#2}}
\newcommand{\Ca}[1]{\Cs{\alpha}{#1}}
\newcommand{\Cb}[1]{\Cs{\beta}{#1}}
\newcommand{\C}[2]{C_{{#1},{#2}}}
\newcommand{\Ctwo}[2]{C_{{#1}{#2}}}
\newcommand{\nel}{{n_{e}}}
\newcommand{\nela}{{n_{e}^\alpha}}
 \newcommand{\dlabel}[1]{\text{#1}\label{#1}}
%\newcommand{\dlabel}[1]{\label{#1}}

\begin{document}

\title{A double Hilbert space approach to natural orbital functional models}

\author{Hubertus J. J. van Dam}
\affiliation{Brookhaven National Laboratory, Upton, NY 11973-5000}
\date{February 25, 2020}

\begin{abstract}
\end{abstract}

\maketitle
\tableofcontents

\footnote{This paper is dedicated to the memory of my PhD advisor Dr. Joop H. van Lenthe.
Particularly relevant to this work is that in his lectures Joop would point out every year that the wave function
itself does not have a physical interpretation. Only the square of the wave function can be interpreted
as a probability density. The students were told that this would be a question in the exam paper, and the
question would appear in all past exam papers the students were given as practice papers. Joop insisted
that this question would remain on all exam papers until the time that all students would answer it 
correctly. In practice, only about 80\% of the students ever answered this question correctly, and the question
stood until Joop's retirement.

In this paper we exploit the fact that the wave function does not have a direct physical interpretation
for the freedom it offers to choose alternative formulations, as long as whatever form is chosen sensible
probability densities can be generated.}

\section{Introduction}

In a recent paper~\cite{van_Dam_2016} we proposed a form of effective
one-electron wave function. We showed that this wave function can generate any
arbitrary N-representable 1-electron density matrix even non-idempotent ones.
We also demonstrated that this approach can be used to formulate a density 
matrix functional model. 

Since then this model has been heavily critized by Piris and 
Pernal~\cite{Piris_2017}. Their criticism essentially questions the approach
by which we introduced additional degrees of freedom to enable the
formulation of N-representable density matrices with fractional occupation
numbers using only a single configuration wave function. In another twist, the
author was made aware of the fact that the approach we proposed actually was 
not new at all. In fact as early as 1963 it was recognized in field theories that 
expectation values of thermal ensembles could be formulated as expectation values
of wave functions by doubling the degrees of freedom~\cite{Araki_1963}. This idea 
was extended by Umezawa et al. to a model for finite temperature field theories under
the name of Thermo Field Dynamics (TFD) by 1975~\cite{TAKAHASHI_1996,Umezawa:1982} (the 
former paper is a reprint of~\cite{TAKAHASHI_1975} which is no longer available).
As orginally formulated, TFD is a general approach to finite temperature field
theories for systems including both bosons and fermions. However, this paper
focusses exclusively on electrons. 

In electronic structure applications, TFD has recently been used to
describe finite temperature effects at coupled cluster levels of
theory~\cite{Harsha_2019,Shushkov_2019}. For these applications an attractive aspect of TFD
is that it is a wave function based approach. Hence, formulating N-representable density
matrices should be straightforward. Whereas previously we (unwittingly)
exploited this property
to formulate N-representable 1-electron density matrices with fractional occupation
numbers~\cite{van_Dam_2016}, this
same property should enable the formulation of N-representable 2-electron density
matrices as well. Different from TFD the focus in this work is on electron
correlation instead of finite temperature effects. Nevertheless, independent
of the physics that generates fractional occupation numbers there is a clear
advantage to having an N-representable reference point. To exploit this fact we seek
to apply the mathematical approach to the wave function formulation of TFD as the basis
for a natural orbital functional model. 

To describe the approach outlined above this paper proceeds by explaining the relationship
between TFD and our earlier work. In an attempt to formulate an N-representable 2-electron
density matrix it became clear that our wave function formulation of~\cite{van_Dam_2016}
did not properly capture all required aspects of the anti-symmetry, the approach
only captured enough anti-symmetry to formulate N-representable 1-electron density
matrices. This is a flaw in our earlier work and not of TFD. To rectify this issue we
summarize the proper form of the wave function and derive the 1- and 2-electron density
matrices. As the 1-electron density matrix has the same form as in our previous paper we
refer for the proof of its N-representability to that work. The N-representability of 
the 2-electron density matrix is demonstrated. At that point a zero-temperature 
mean field energy expression is given that is N-representable even with fractional
occupation numbers. This mean field energy expression is subsequently compared to the 
full-CI energy for a simple model system, \ce{H2} in a minimal basis, to arrive at 
a correlation energy expression. Implementation details of this correlated electron
model are given, followed by representative examples.

\section{The thermal state of Thermo Field Dynamics in relation to N-representable 
         electron density matrices}

The foundational publication of TFD is~\cite{TAKAHASHI_1996} but to explain the relation
between TFD and the approach in this paper we refer to a more recent work by
Das~\cite{das2000topics} as the notation is easier to follow. In particular on section 4
is relevant to this discussion. In terms of 
an effective 1-electron model the eigenfunctions of the Hamiltonian are
simply the molecular orbitals. In an effective 1-electron model the molecular
orbitals are also the same as the natural orbitals. In~\cite{das2000topics} these
states are referred to as $|n\rangle$ or $|m\rangle$, in our prior paper~\cite{van_Dam_2016} 
these states are expressed as vectors $N$. In order to define a thermal vacuum state 
that produces the correct ensemble averages Das shows that this can be done by 
introducing a fictitious system that is identical to the orginal system, denoted as 
a tilde system. With this tilde system the Hilbert is space is doubled and the states
can be written as
\begin{eqnarray}
   |n,\tilde{m}\rangle &=& |n\rangle \otimes |\tilde{m}\rangle
\end{eqnarray}
This equation is the same as Eq.(8) in~\cite{van_Dam_2016} if we equate the tilde 
system of TFD with the correlation functions of our earlier paper. It then follows 
straightforwardly that Das's Eq.(44) is the same as van Dam's Eq.(16) if the density
matrix of the latter equation is traced with the matrix elements of the operator $O$.
Hence, it is clear that TFD and our approach share the same foundations for 
representing the state of a system. 

In our previous paper the states $|n,\tilde{m}\rangle$ were labeled $|G_s(r)\rangle$.
If multi-electron states were formulated as determinants of these states but with 
a modified way to translate $|G_s(r)\rangle\langle G_t(r')|$ into a density matrix
contribution then~\cite{van_Dam_2016} shows you can obtain an N-representable 
1-electron density matrix. However, this approach is insufficient 
to obtain N-representable higher order density matrices. In order to obtain
N-representable higher order density matrices one has to, independently, anti-symmetrize
over both the natural orbitals as well as the correlation functions. In TFD this issue
is addressed by expressing the Hamiltonian in second quantization. For the regular part
of the system the usual annihilation ($a$) and creation ($a^\dagger$) operators are used
with the corresponding anti-commutation relations
\begin{eqnarray}
  \delta_{ij} &=& [a_i,a_j^\dagger]_+ \\
  0 &=& [a_i,a_j]_+ = [a_i^\dagger,a_j^\dagger]_+
\end{eqnarray}
Likewise, annihilation ($\tilde{a}$) and creation ($\tilde{a}^\dagger$) operators for
the tilde system can be introduced that operate only on the correlation functions.
These operators obey the same anti-commutation relations
\begin{eqnarray}
  \delta_{ij} &=& [\tilde{a}_i,\tilde{a}_j^\dagger]_+ \\
  0 &=& [\tilde{a}_i,\tilde{a}_j]_+ = [\tilde{a}_i^\dagger,\tilde{a}_j^\dagger]_+
\end{eqnarray}
In addition, the regular annihilation and creation operators, and the tilde annihilation and
creation operators, are assumed to anti-commute. These properties of the regular and tilde
annihilation and creation operators effect the independent anti-symmetrization of the
regular and tilde functions, or in our terminology the natural orbitals and correlation
functions.
In TFD these operators are used to define thermal annihilation and creation operators of the
form
\begin{eqnarray}
  a(\beta)         &=& cos\theta(\beta)a         - sin\theta(\beta)\tilde{a}^\dagger \\
  \tilde{a}(\beta) &=& cos\theta(\beta)\tilde{a} + sin\theta(\beta)a\dagger 
\end{eqnarray}
where
\begin{eqnarray}
  cos\theta(\beta) &=& \frac{1}{\sqrt{1+e^{-\beta\omega}}} \\
  sin\theta(\beta) &=& \frac{e^{-\beta\omega/2}}{\sqrt{1+e^{-\beta\omega}}} \\
  \beta            &=& \frac{1}{k_B T}
\end{eqnarray}
Omega in turn, is a positive number appearing in the Hamiltonian
\begin{eqnarray}
  H         &=& \omega a^\dagger a \\
  \tilde{H} &=& \omega \tilde{a}^\dagger \tilde{a}
\end{eqnarray}
A consequence of this approach is that annihilating a particle in the thermal vacuum is
equivalent to creating a particle in the tilde system. This leads to be the interpretation 
that the tilde states are particle states of the heat bath.

The explanation above provides a description of systems at finite temperature, but here
the focus is on electron correlation. In this context the formalism needs to be applied to
isolated sytems of fermions at zero temperature. In this situation the interpretation
that the tilde states are particle states of the heat bath clearly does not apply as
there is no heat bath. Also at zero temperature the regular system and the tilde system
decouple leading to $a(\beta) = a$ and $\tilde{a}(\beta) = \tilde{a}$. Even though
the context is different it should still be possible to use the mathematical foundations 
of TFD and apply it to systems of interacting electrons at zero temperature but potentially
with electron correlation. The critical feature in this case is the double Hilbert
space approach that enables generating N-representable density matrices
with fractional occupation numbers. In TFD the fractional occupation numbers are generated
by thermal effects, here these occupation numbers are driven by correlation effects.

To outline how the independent anti-symmetry over the correlation functions and the
natural orbitals generates N-representable density matrices an updated recipe for 
their generation is given.
As stated above, we refer to the regular orbitals with $N=N_i(r)$ where $i$ is an 
orbital index and $r$ represents the spatial coordinates. The correlation functions
are referred to as $C=C_s(\epsilon)$ where $s$ is a function index and $\epsilon$ is
a fictitious coordinate. In a many electron setting these functions need to operate
on the coordinates of specific electrons and one arrives at a wave function of the form
\begin{eqnarray}
   \Psi(\epsilon_1,\ldots,\epsilon_\nel,r_1,\ldots,r_n)
   &=& \Psi(\epsilon)\otimes\Psi(r) \\
   &=& |C_1(\epsilon_1)\ldots C_s(\epsilon_s)\ldots C_t(\epsilon_t)\ldots C_\nel(\epsilon_\nel)|
   \otimes |N_1(r_1)\ldots N_i(r_i)\ldots N_j(r_j)\ldots N_n(r_n)|
   \label{Eq:Psi}
\end{eqnarray}
where the notation $|\ldots|$ refers to the usual Slater determinant. In addition all 
1-electron states are assumed to be spin-states of the same spin. The extension to
wave functions with 1-electron states of different spin is trivial.
Also $\nel$ is the number of electrons.
Incidentally, $n$ is chosen as the number of orbitals to represent the problem in.

Obviously, this wave function seems strange in that the $\Psi(r)$ factor has more 
coordinates than the $\Psi(\epsilon)$ factor. However, this is essentially a short-hand
notation where the $\epsilon$-coordinates and $C(\epsilon)$ functions corresponding to
unoccupied states have been omitted. One could include these states and their 
correlation function occupation numbers in the derivations but other than requiring
more writing the results would be no different than the abridged derivation given here.

Note that in Eq.~\ref{Eq:Psi} the factor $\Psi(r)$ has as many coordinates as there 
are natural orbitals in which the problem is expressed. 
It will be shown that introducing such a high dimensional coordinate
system is a convenient tool to introduce all natural orbitals into the density matrices.
In fact, as every natural orbital appears exactly once in $\Psi(r)$ the 1- and 2-electron 
density matrices generated from it will be unit density matrices. This result is very convenient
as a starting point for generating more general (non-unit) density matrices.

Second, every correlation function represents exactly one electron hence $C_s(\epsilon)C^*_s(\epsilon)$
represents the probability density of finding an electron at coordinate $\epsilon$.
It will be shown that to obtain N-representable density
matrices the coordinates $\epsilon$ can be integrated out until the right order density matrix of 
the $\Psi(\epsilon)$ part is obtained. In a final step the remaining $\epsilon$ coordinates can be integrated
out by first discretizing the coordinate to obtain points $\epsilon_i$, associating every discretization
point $\epsilon_i$ with a natural orbital index $i$, and subsequently summing over the corresponding
index $i$.

Clearly, there is reason for unease with such an unusual approach. However, remember that the 
wave function itself is not an observable. It is merely a mathematical tool to formulate 
physically sensible probability densities that can be used to express observable quantities.
As unusual as this wave function may seem it should be judged by its usefulness in 
expressing observable quantities.

Physically sensible probabilities densities are directly related to N-representable
density matrices. To demonstrate how a wave function of the form of Eq.~\ref{Eq:Psi}
leads to N-representable density matrices remember that the basis is a set of
orthonormal 1-electron functions, hence
\begin{eqnarray}
   \delta_{ij} &=& \langle N_i(r)        | N_j(r)        \rangle \\
   \delta_{st} &=& \langle C_s(\epsilon) | C_t(\epsilon) \rangle
\end{eqnarray}
As outlined above, the wavefunction defined in Eq.\ref{Eq:Psi} provides an opportunity
to consider the spatial components, generated from $\Psi(r)$, and the
natural orbital occupation numbers,
generated from $\Psi(\epsilon)$, of the density matrices separately. 
Considering the $\Psi(r)$ part the 1-electron and 2-electron 
density matrices of the natural orbitals become
\begin{eqnarray}
  \sum_{i=1}^{n} |\psi_i(r_1)\rangle \langle\psi_i(r'_1)|
  &=& n \int\Psi(r_1,\ldots,r_n)\Psi^*(r'_1,\ldots,r_n) d(r_2,\ldots,r_n) \\
  &=& \sum_{i=1}^{n} N_i(r_1) \hat{1} N_i^*(r'_1) \dlabel{Eq:D1:r} \\
  \sum_{i,j=1}^{n} |\psi_i(r_1)\psi_j(r_2)\rangle \langle\psi_i(r'_1)\psi_j(r'_2)| 
  &=& \frac{n(n-1)}{2}\int\Psi(r_1,r_2,\ldots,r_n)\Psi^*(r'_1,r'_2,\ldots,r_n) d(r_3,\ldots,r_n) \\
  &=&+\frac{1}{2}\sum_{i,j=1}^{n} N_i(r_1)N_j(r_2) \hat{1} N_i^*(r'_1)N_j^*(r'_2) \nonumber \\
  &&- \frac{1}{2}\sum_{i,j=1}^{n} N_i(r_1)N_j(r_2) \hat{1} N_j^*(r'_1)N_i^*(r'_2) \nonumber \\
  &&- \frac{1}{2}\sum_{i,j=1}^{n} N_j(r_1)N_i(r_2) \hat{1} N_i^*(r'_1)N_j^*(r'_2) \nonumber \\
  &&+ \frac{1}{2}\sum_{i,j=1}^{n} N_j(r_1)N_i(r_2) \hat{1} N_j^*(r'_1)N_i^*(r'_2) \dlabel{Eq:D2:r}
\end{eqnarray}
The equations~\ref{Eq:D1:r} and~\ref{Eq:D2:r} yield identity operators in the space
of anti-symmetrized density matrices.

The next step is to modulate the identity operators~\ref{Eq:D1:r} and~\ref{Eq:D2:r} by including the
factors stemming from the correlation functions that act as 1- and 2-electron occupation numbers.
\begin{eqnarray}
  \sum_{s=1}^\nel |\psi_s(\epsilon_1)\rangle \langle\psi_s(\epsilon'_1)|
  &=& \nel\int\Psi(\epsilon_1,\ldots,\epsilon_n)\Psi^*(\epsilon'_1,\ldots,\epsilon_n)
      d(\epsilon_2,\ldots,\epsilon_n) \\
  &=& \sum_{s=1}^\nel C_s(\epsilon_1) C_s^*(\epsilon'_1) \dlabel{Eq:D1:e} \\
  \sum_{s,t=1}^\nel |\psi_s(\epsilon_1)\psi_t(\epsilon_2)\rangle \langle\psi_s(\epsilon'_1)\psi_t(\epsilon'_2)| 
  &=& \frac{\nel(\nel-1)}{2}\int\Psi(\epsilon_1,\epsilon_2,\ldots,\epsilon_\nel)
      \Psi^*(\epsilon'_1,\epsilon'_2,\ldots,\epsilon_\nel) d(\epsilon_3,\ldots,\epsilon_\nel) \\
  &=&+\frac{1}{2}\sum_{s,t=1}^{\nel} C_s(\epsilon_1)C_t(\epsilon_2)
                                     C_s^*(\epsilon'_1)C_t^*(\epsilon'_2) \nonumber \\
  &&- \frac{1}{2}\sum_{s,t=1}^{\nel} C_s(\epsilon_1)C_t(\epsilon_2)
                                     C_t^*(\epsilon'_1)C_s^*(\epsilon'_2) \nonumber \\
  &&- \frac{1}{2}\sum_{s,t=1}^{\nel} C_t(\epsilon_1)C_s(\epsilon_2)
                                     C_s^*(\epsilon'_1)C_t^*(\epsilon'_2) \nonumber \\
  &&+ \frac{1}{2}\sum_{s,t=1}^{\nel} C_t(\epsilon_1)C_s(\epsilon_2)
                                     C_t^*(\epsilon'_1)C_s^*(\epsilon'_2) \dlabel{Eq:D2:e}
\end{eqnarray}
To obtain the diagonal $d_i$ of the 1-electron density matrix from Eq.~\ref{Eq:D1:e} we discretize
the coordinate $\epsilon$ as
\begin{eqnarray}
   d_i &=& \sum_{s=1}^\nel C_s(\epsilon_i) C_s^*(\epsilon_i) \\
       &=& \sum_{s=1}^\nel C_{is}C^*_{is} \dlabel{Eq:D1:di}
\end{eqnarray}
for all $i \in [1,n]$. The equating of $\epsilon_1$ and $\epsilon'_1$ comes from the contraction
of these coordinates as they are not free coordinates of the density matrix (only the spatial
coordinates are).

Similarly the diagonal $d_{ij}$ of the 2-electron density matrix can be obtained from Eq.~\ref{Eq:D2:e}
by discretizing the coordinates $\epsilon_1$ and $\epsilon_2$. 
\begin{eqnarray}
   d_{ij}
   &=&+\frac{1}{2}\sum_{s,t=1}^{\nel} C_s(\epsilon_i)C_t(\epsilon_j) 
                                      C_s^*(\epsilon_i)C_t^*(\epsilon_j) \nonumber \\
   &&- \frac{1}{2}\sum_{s,t=1}^{\nel} C_s(\epsilon_i)C_t(\epsilon_j) 
                                      C_t^*(\epsilon_i)C_s^*(\epsilon_j) \nonumber \\
   &&- \frac{1}{2}\sum_{s,t=1}^{\nel} C_t(\epsilon_i)C_s(\epsilon_j) 
                                      C_s^*(\epsilon_i)C_t^*(\epsilon_j) \nonumber \\
   &&+ \frac{1}{2}\sum_{s,t=1}^{\nel} C_t(\epsilon_i)C_s(\epsilon_j) 
                                      C_t^*(\epsilon_i)C_s^*(\epsilon_j) \\
   &=&+\frac{1}{2}\sum_{s,t=1}^{\nel} C_{is}C_{jt} C_{is}^*C_{jt}^* \nonumber \\
   &&- \frac{1}{2}\sum_{s,t=1}^{\nel} C_{is}C_{jt} C_{it}^*C_{js}^* \nonumber \\
   &&- \frac{1}{2}\sum_{s,t=1}^{\nel} C_{it}C_{js} C_{is}^*C_{jt}^* \nonumber \\
   &&+ \frac{1}{2}\sum_{s,t=1}^{\nel} C_{it}C_{js} C_{it}^*C_{js}^* \\
   &=&+\frac{1}{2}\sum_{s,t=1}^{\nel} C_{is}C_{is}^* C_{jt}C_{jt}^* \nonumber \\
   &&- \frac{1}{2}\sum_{s,t=1}^{\nel} C_{is}C_{it}^* C_{jt}C_{js}^* \nonumber \\
   &&- \frac{1}{2}\sum_{s,t=1}^{\nel} C_{it}C_{is}^* C_{js}C_{jt}^* \nonumber \\
   &&+ \frac{1}{2}\sum_{s,t=1}^{\nel} C_{it}C_{it}^* C_{js}C_{js}^* \dlabel{Eq:D2:dij}
\end{eqnarray}
for all $i,j \in [1,n]$.

The full 1-electron density matrix can then be obtained by multiplying $\hat{1}$ in Eq.~\ref{Eq:D1:r}
with $d_i$ from Eq.~\ref{Eq:D1:di} to obtain
\begin{eqnarray}
  D(r_1,r'_1) 
  &=& \sum_{i=1}^{n}\sum_{s=1}^\nel
      |\psi_i(r_1),\psi_s(\epsilon_i)\rangle \langle\psi_s(\epsilon_i),\psi_i(r'_1)| \\
  &=& \sum_{i=1}^{n} N_i(r_1) d_i N_i^*(r'_1) \dlabel{Eq:D1:di:r}
\end{eqnarray}
Likewise the full 2-electron density matrix can be obtained
by multiplying $\hat{1}$ in Eq.~\ref{Eq:D2:r} with $d_{ij}$ from Eq.~\ref{Eq:D2:dij} to obtain
\begin{eqnarray}
  \Gamma(r_1,r_2,r'_1,r'_2)
  &&= \sum_{i,j=1}^{n}\sum_{s,t=1}^\nel
      |\psi_i(r_1)\psi_j(r_2),\psi_s(\epsilon_i)\psi_t(\epsilon_j)\rangle
      \langle\psi_s(\epsilon_i)\psi_t(\epsilon_j),\psi_i(r'_1)\psi_j(r'_2)| 
      \\
  &&= \frac{1}{2}\sum_{i,j=1}^{n} N_i(r_1)N_j(r_2) d_{ij} N_i^*(r'_1)N_j^*(r'_2) \nonumber \\
  &&- \frac{1}{2}\sum_{i,j=1}^{n} N_i(r_1)N_j(r_2) d_{ij} N_j^*(r'_1)N_i^*(r'_2) \nonumber \\
  &&- \frac{1}{2}\sum_{i,j=1}^{n} N_j(r_1)N_i(r_2) d_{ij} N_i^*(r'_1)N_j^*(r'_2) \nonumber \\
  &&+ \frac{1}{2}\sum_{i,j=1}^{n} N_j(r_1)N_i(r_2) d_{ij} N_j^*(r'_1)N_i^*(r'_2) \dlabel{Eq:D2:dij:r}
\end{eqnarray}
To obtain the final expressions the natural orbitals need to be expanded in a basis.
In general this basis may involve non-orthogonal basis functions, such as Gaussians.
In such a case the orthonormality condition for the natural orbitals becomes
\begin{eqnarray}
  \delta_{ij} &=& \sum_{a,c=1}^{n} N_{ai}S_{ac} N_{cj} \\
  S_{ac} &=& \int \chi_a(r) \chi^*_c(r) \mathrm{d}r
\end{eqnarray}
Likewise, the 1-electron energy term associated with Eq.~\ref{Eq:D1:di:r} becomes
\begin{eqnarray}
  E^{(1)}
  &=& \sum_{i=1}^{n} \sum_{a,c=1}^{n} h_{ac} N_{ai} d_i N_{ci}^* \dlabel{Eq:E1:di:r} \\
  h_{ac}
  &=& \int \chi_a(r) \hat{H}_{1} \chi^*_c(r) \mathrm{d}r
\end{eqnarray}
and the 2-electron energy term associated with Eq.~\ref{Eq:D2:dij:r} becomes
\begin{eqnarray}
  E^{(2)}
  &=& \sum_{a,b,c,d=1}^{n} \langle a b | c d \rangle \Gamma_{abcd} \dlabel{Eq:E2:dij:r} \\
  \Gamma_{abcd}
  &&= \frac{1}{2}\sum_{i,j=1}^{n} N_{ai}N_{bj} d_{ij} N_{ci}^*N_{dj}^* \nonumber \\
  &&- \frac{1}{2}\sum_{i,j=1}^{n} N_{ai}N_{bj} d_{ij} N_{cj}^*N_{di}^* \nonumber \\
  &&- \frac{1}{2}\sum_{i,j=1}^{n} N_{aj}N_{bi} d_{ij} N_{ci}^*N_{dj}^* \nonumber \\
  &&+ \frac{1}{2}\sum_{i,j=1}^{n} N_{aj}N_{bi} d_{ij} N_{cj}^*N_{di}^* \dlabel{Eq:E2:dij:basis} \\
  \langle a b | c d \rangle
  &=& \int\int \chi_a(r_1) \chi^*_c(r_1) \frac{1}{r_{12}} \chi_b(r_2) \chi^*_d(r_2) 
      \mathrm{d}r_1 \mathrm{d}r_2
\end{eqnarray}

That the 1-electron density matrix obtained is N-representable was already shown in~\cite{van_Dam_2016}.
In the next section the N-representability of the 2-electron density matrix will be shown.

\section{The N-representability of the 2-electron density matrix in a double Hilbert space}

As stated above, the N-representability of the 1-electron density matrix as given by
Eq.~\ref{Eq:D1:di:r} was shown already in~\cite{van_Dam_2016}. In that work the emphasis 
was on 1-electron density matrices and the 2-electron density matrix was not given any
consideration. In this work the 1-electron density matrix was rederived but also the
2-electron density matrix has been derived resulting in Eq.~\ref{Eq:D2:dij:r}. However,
the derivation is of the 2-electron density matrix is based on an unconventional approach
based on the foundations of TFD. Hence, prudence requires to verify that the result actually
meets the conditions that N-representability imposes. 

Note that the 2-electron density matrix separates into different blocks when the system
is described with spin-free Hamiltonians. The $\alpha\beta$ and $\beta\alpha$ blocks 
can be represented as outer products of the corresponding 1-electron density matrices. As a
result these parts are trivially N-representable as long as the individual 1-electron
density matrices are themselves N-representable. The $\alpha\alpha$ and $\beta\beta$ blocks of 
the 2-electron density matrix are more complicated as the exchange operator comes into
play. Hence this section will focus on showing the N-representability of these blocks.
In fact, the focus is on the $\alpha\alpha$ block as the analysis for the $\beta\beta$ 
block is the same. 

For the $\alpha\alpha$-block of the 2-electron density matrix to be N-representable
it must satisfy the following conditions:
\begin{enumerate}
\item The $D$ condition states that the matrix must be non-negative (all eigenvalues must be $\ge 0$),
      or equivalently
      the pair-probability function for any pair of natural orbitals must be non-negative~\cite{Coleman_1963}
      \dlabel{Cond:N2repres:1:D};
\item The $Q$ condition states that the density matrix of hole states must be non-negative
      \dlabel{Cond:N2repres:1:Q};
\item The $G$ condition states that the density matrix of particle-hole pairs must be
      non-negative~\cite{Garrod_1964}
      \dlabel{Cond:N2repres:1:G};
\item The trace of the matrix must sum to the number of $\alpha$-electron pairs, i.e.
      $n_e^\alpha(n_e^\alpha-1)/2$ where $n_e^\alpha$ is the number of $\alpha$-electrons
      \dlabel{Cond:N2repres:2};
\item The 2-electron density matrix must be anti-symmetric, i.e. 
      $\Gamma_{abcd} = -\Gamma_{bacd} = -\Gamma_{abdc} = \Gamma_{badc} = \Gamma_{cdab} = -\Gamma_{cdba} = -\Gamma_{dcab} = \Gamma_{dcba}$
      \dlabel{Cond:N2repres:3};
\item The 1-electron density matrix must be obtainable from the 2-electron density matrix
      by integrating out the coordinates of one of the electrons
      \dlabel{Cond:N2repres:4}.
\end{enumerate}

\subsection{Non-negativity}
\dlabel{Sect:NonNegative}

Physically it is important that probability densities are non-negative. 
For pair probability densities to be non-negative the 2-electron density 
must be non-negative. 
In this subsection we prove that the 2-electron density matrix of Eq.~\ref{Eq:D2:dij:r}
satisfies the $D$ condition~\ref{Cond:N2repres:1:D}. 
In addition we show that the resulting pair probability densities are non-negative as well.
Furthermore, we will show that because of the structure of the equations it also follows
that if condition~\ref{Cond:N2repres:1:D} is satisfied, then the
$Q$ condition~\ref{Cond:N2repres:1:Q} and the $G$ condition~\ref{Cond:N2repres:1:G}
are satisfied as well. As the pair probability density is 
constructed as an ensemble the proof proceeds in two parts. First it is 
proven that the weights $d_{ij}$ (the eigenvalues of the 2-electron density matrix)
are all non-negative. Next it shown that
the individual natural orbital pair states generate non-negative probability densities.
As the density matrix is expressed in the basis where it is diagonal it follows
from these two parts that the pair probability density overall is non-negative.

The fact that the weights of Eq.~\ref{Eq:D2:dij} are non-negative can be
proven by making use of the Cauchy-Schwarz inequality which states that
\begin{eqnarray}
  \left|\sum_{i=1}^n x_i y_i^*\right|^2
  &\leq& \left(\sum_{j=1}^n|x_j|^2\right)\left(\sum_{k=1}^n|y_k|^2\right)
  \dlabel{Eq:CauchySchwarz}
\end{eqnarray}
For simplicity we prove that the first term in Eq.~\ref{Eq:D2:dij} is
larger than the second term for every combination of $i$ and $j$.
I.e. we have to show that
\begin{eqnarray}
  \sum_{s,t=1}^{n_e^\alpha} C_{is}C^*_{it}C_{jt}C^*_{js} &\leq&
  \sum_{s,t=1}^{n_e^\alpha} C_{is}C^*_{is}C_{jt}C^*_{jt}
\end{eqnarray}
where $s$ and $t$ are drawn from the $1$ to $n_e^\alpha$ occupied correlation 
functions.
By trivially rearranging the factors and the summations this expression can be
rewritten as
\begin{eqnarray}
  \left(\sum_{s=1}^{n_e^\alpha}C_{is}C^*_{js}\right)
  \left(\sum_{t=1}^{n_e^\alpha}C_{jt}C^*_{it}\right) &\leq&
  \left(\sum_{s=1}^{n_e^\alpha}C_{is}C^*_{is}\right)
  \left(\sum_{t=1}^{n_e^\alpha}C_{jt}C^*_{jt}\right)
  \dlabel{Eq:positivity:dij}
\end{eqnarray}
which is the same as Eq.~\ref{Eq:CauchySchwarz} if the vector $y$ is equated to
the $j$-th row of $C$ and vector $x$ is equated to the $i$-th row of $C$.
Furthermore as the first and last terms in Eq.~\ref{Eq:D2:dij} are obviously
non-negative it immediately follows that $d_{ij}$ are non-negative.

% Since $C$ is a unitary matrix it is known that 
% $\sum_{s=1}^{n_b} C_{is}C^*_{js} = \delta_{ij}$. From this it follows that
% the maximum of $d_{ij}$ is obtained if $s$ and $t$ are summed over all 
% correlation functions. In this case the first term of Eq.~\ref{Eq:D2:dij}
% becomes 1 and the second term vanishes, hence it follows that
% $d_{ij} \le 1$. 

To prove the second part, again the Cauchy-Schwarz inequality will be used.
However, first the non-negativity of the one electron density is considered.
The one electron probability density is the norm of a natural orbital at 
any point in space, i.e.
\begin{eqnarray}
  \rho(r_1)
  &=& \phi_i(r_1)\phi_i^*(r_1) \\
  &\ge& 0
\end{eqnarray}
The non-negativity is guaranteed by taking the norm of the natural orbital.

For pair probability densities consider
\begin{eqnarray}
  \rho_{ij}(r_1,r_2)
  &=& \phi_i(r_1)\phi^*_i(r_1)\phi_j(r_2)\phi^*_j(r_2)
     +\phi_j(r_1)\phi^*_j(r_1)\phi_i(r_2)\phi^*_i(r_2) \nonumber \\
  && -\phi_i(r_1)\phi^*_j(r_1)\phi_j(r_2)\phi^*_i(r_2)
     -\phi_j(r_1)\phi^*_i(r_1)\phi_i(r_2)\phi^*_j(r_2) 
\end{eqnarray}
The Cauchy-Schwarz inequality assumes a sum over a number of terms but nothing
precludes the number of terms being 1. From that it immediately follows that
\begin{eqnarray}
  \phi_i(r_1)\phi^*_i(r_1)\phi_j(r_2)\phi^*_j(r_2) 
  +\phi_j(r_1)\phi^*_j(r_1)\phi_i(r_2)\phi^*_i(r_2)
  &\ge& \phi_i(r_1)\phi^*_j(r_1)\phi_j(r_2)\phi^*_i(r_2)
       +\phi_j(r_1)\phi^*_i(r_1)\phi_i(r_2)\phi^*_j(r_2) 
\end{eqnarray}
And therefore that
\begin{eqnarray}
   \rho_{ij}(r_1,r_2) &\ge& 0
   \dlabel{Eq:positivity:rij}
\end{eqnarray}
for any pair of natural orbitals $i$ and $j$.

As Eq.~\ref{Eq:positivity:dij} proves that the 2-electron density matrix
is non-negative, and Eq.~\ref{Eq:positivity:rij} proves that every orbital
pair probability density is non-negative it follows that the total 
electron pair probability density
\begin{eqnarray}
   \rho(r_1,r_2) 
   &=& \sum_{ij}d_{ij}\rho_{ij}(r_1,r_2) \\
   &\ge& 0
\end{eqnarray}
which implies that the relevant non-negativity conditions are met.

In the above the $D$ condition was proven, and the proof was based on including
only the occupied correlations functions in the occupation numbers of the 2-electron
density matrix. However, the natural orbital for occupied states are the same as
the natural orbitals for the hole states. From that it follows that to show the 
non-negativity of the 2-electron density matrix of the hole states, i.e. the $Q$ 
N-representability condition~\ref{Cond:N2repres:1:Q}, we just need to replace
the occupied $s$ and $t$ correlation functions with unoccupied $s$ and $t$
correlation functions. I.e. instead of drawing $s$ and $t$ from the correlation
functions $1$ to $n_e^\alpha$, we draw $s$ and $t$ from the correlation functions
$n_e^\alpha+1$ to $n_b$. Now note that the form of the expression Eq.~\ref{Eq:D2:dij}
does not change irrespective of whether $s$ or $t$ are occupied or unoccupied
correlation functions. Therefore, if Eq.~\ref{Eq:D2:dij} is non-negative when 
$s$ and $t$ are occupied correlation functions then Eq.~\ref{Eq:D2:dij} is also
non-negative if one of $s$ or $t$, or both $s$ and $t$ are unoccupied correlation
functions. This proves that the $Q$ and $G$
conditions~\ref{Cond:N2repres:1:Q} and~\ref{Cond:N2repres:1:G} hold as well.

\subsection{Trace}
\dlabel{Sect:Trace}

In this subsection condition~\ref{Cond:N2repres:2} is considered which amounts
to showing that integrating both the positions of electron 1 and 2 out generates
the correct number of electron pairs, i.e. $n_e^\alpha(n_e^\alpha-1)/2$.

Again, the proof proceeds in two stages starting from Eq.~\ref{Eq:D2:dij:r}.
First it is shown that each factor consisting of just the natural orbital
parts contributes $1-\delta_{ij}$ which implies that the trace amounts to the
sum $\sum_{ij,i \ne j}d_{ij}$. Next, it is shown that 
$\sum_{ij,i \ne j}d_{ij}=n_e^\alpha(n_e^\alpha-1)/2$.

The natural orbital factor of Eq.~\ref{Eq:D2:dij:r} is
\begin{eqnarray}
  F_{ij}(r_1,r_2,r'_1,r'_2)
  &&= \frac{1}{2} N_i(r_1)N_j(r_2) N_i^*(r'_1)N_j^*(r'_2) \nonumber \\
  &&- \frac{1}{2} N_i(r_1)N_j(r_2) N_j^*(r'_1)N_i^*(r'_2) \nonumber \\
  &&- \frac{1}{2} N_j(r_1)N_i(r_2) N_i^*(r'_1)N_j^*(r'_2) \nonumber \\
  &&+ \frac{1}{2} N_j(r_1)N_i(r_2) N_j^*(r'_1)N_i^*(r'_2)
\end{eqnarray}
Given that 
\begin{eqnarray}
  \delta_{ij} &=& \int N_i(r) N_j^*(r)\mathrm{d}r
\end{eqnarray}
it follows that 
\begin{eqnarray}
  \int\int F_{ij}(r_1,r_2,r_1,r_2)\mathrm{d}r_1\mathrm{d}r_2
  &&= \frac{1}{2} \int N_i(r_1)N_i^*(r_1)\mathrm{d}r_1 \int N_j(r_2)N_j^*(r_2)\mathrm{d}r_2 \nonumber \\
  &&- \frac{1}{2} \int N_i(r_1)N_j^*(r_1)\mathrm{d}r_1 \int N_j(r_2)N_i^*(r_2)\mathrm{d}r_2 \nonumber \\
  &&- \frac{1}{2} \int N_j(r_1)N_i^*(r_1)\mathrm{d}r_1 \int N_i(r_2)N_j^*(r_2)\mathrm{d}r_2 \nonumber \\
  &&+ \frac{1}{2} \int N_j(r_1)N_j^*(r_1)\mathrm{d}r_1 \int N_i(r_2)N_i^*(r_2)\mathrm{d}r_2 \\
  &&= \frac{1}{2} 1 \cdot 1 \nonumber \\
  &&- \frac{1}{2} \delta_{ij} \cdot \delta_{ij} \nonumber \\
  &&- \frac{1}{2} \delta_{ij} \cdot \delta_{ij} \nonumber \\
  &&+ \frac{1}{2} 1 \cdot 1  \\
  &&= 1-\delta_{ij}
\end{eqnarray}
From this we have that the trace of the 2-electron density matrix is
$\sum_{ij,i < j}d_{ij}$. Also note from Eq.~\ref{Eq:D2:dij} that $d_{ii} = 0$
and therefore the trace of the 2-electron density matrix is
$\frac{1}{2}\sum_{ij,i \ne j}d_{ij} = \frac{1}{2}\sum_{ij}d_{ij}$. This sum is given by
\begin{eqnarray}
   \frac{1}{2}\sum_{ij}d_{ij}
   &=&+ \frac{1}{4}\sum_{ij}\sum_{s,t=1}^{\nela} C_{is}C_{is}^* C_{jt}C_{jt}^* \nonumber \\
   &&-  \frac{1}{4}\sum_{ij}\sum_{s,t=1}^{\nela} C_{is}C_{it}^* C_{jt}C_{js}^* \nonumber \\
   &&-  \frac{1}{4}\sum_{ij}\sum_{s,t=1}^{\nela} C_{it}C_{is}^* C_{js}C_{jt}^* \nonumber \\
   &&+  \frac{1}{4}\sum_{ij}\sum_{s,t=1}^{\nela} C_{it}C_{it}^* C_{js}C_{js}^* \\
   &=&+ \frac{1}{4}\left(\nela\right)^2 \nonumber\\
   &&-  \frac{1}{4}\sum_{s,t=1}^{\nela}\left(\delta_{st}\right)^2  \nonumber \\
   &&-  \frac{1}{4}\sum_{s,t=1}^{\nela}\left(\delta_{st}\right)^2  \nonumber \\
   &&+  \frac{1}{4}\left(\nela\right)^2 \\
   &=&+ \frac{1}{2}\left(\nela\right)^2 \nonumber\\
   &&-  \frac{1}{2}\nela \\
   &=&  \frac{\nela(\nela-1)}{2}
\end{eqnarray}
From this it is clear that the trace of the 2-electron density matrix equals the number
of electron pairs.

\subsection{Permutation symmetry}
\dlabel{Sect:Perm}

The 2-electron density matrix needs to have a permutation symmetry that is
consistent with the anti-symmetry of exchanging electrons.
This permutation symmetry can be expressed in terms of the labels on 
the 2-electron density matrix as
$\Gamma_{abcd} = -\Gamma_{bacd} = -\Gamma_{abdc} = \Gamma_{badc}$,
it can also be expressed in terms of the electron coordinates as
$\Gamma(r_1,r_2,r'_1,r'_2) = -\Gamma(r_1,r_2,r'_2,r'_1) = -\Gamma(r_2,r_1,r'_1,r'_2) = \Gamma(r_2,r_1,r'_2,r'_1)$.
Either way this permutation symmetry requires considering the natural orbital
factors only. In fact, as the basis function labels are introduced
by the discretization of the natural orbitals it would seem that the more
general case is considering the interchange of the electron coordinates. 
Considering that $\Gamma(r_1,r_2,r'_1,r'_2)$ is given by Eq.~\ref{Eq:D2:dij:r} one
obtains that
\begin{eqnarray}
   \Gamma(r_1,r_2,r'_2,r'_1)
   &=&+ \frac{1}{2}\sum_{i,j=1}^{n} N_i(r_1)N_j(r_2) d_{ij} N_i^*(r'_2)N_j^*(r'_1) \nonumber \\
   &&-  \frac{1}{2}\sum_{i,j=1}^{n} N_i(r_1)N_j(r_2) d_{ij} N_j^*(r'_2)N_i^*(r'_1) \nonumber \\
   &&-  \frac{1}{2}\sum_{i,j=1}^{n} N_j(r_1)N_i(r_2) d_{ij} N_i^*(r'_2)N_j^*(r'_1) \nonumber \\
   &&+  \frac{1}{2}\sum_{i,j=1}^{n} N_j(r_1)N_i(r_2) d_{ij} N_j^*(r'_2)N_i^*(r'_1) \\
   &=&+ \frac{1}{2}\sum_{i,j=1}^{n} N_i(r_1)N_j(r_2) d_{ij} N_j^*(r'_1)N_i^*(r'_2) \nonumber \\
   &&-  \frac{1}{2}\sum_{i,j=1}^{n} N_i(r_1)N_j(r_2) d_{ij} N_i^*(r'_1)N_j^*(r'_2) \nonumber \\
   &&-  \frac{1}{2}\sum_{i,j=1}^{n} N_j(r_1)N_i(r_2) d_{ij} N_j^*(r'_1)N_i^*(r'_2) \nonumber \\
   &&+  \frac{1}{2}\sum_{i,j=1}^{n} N_j(r_1)N_i(r_2) d_{ij} N_i^*(r'_1)N_j^*(r'_2) \\
   &=&- \frac{1}{2}\sum_{i,j=1}^{n} N_i(r_1)N_j(r_2) d_{ij} N_i^*(r'_1)N_j^*(r'_2) \nonumber \\
   &&+  \frac{1}{2}\sum_{i,j=1}^{n} N_i(r_1)N_j(r_2) d_{ij} N_j^*(r'_1)N_i^*(r'_2) \nonumber \\
   &&+  \frac{1}{2}\sum_{i,j=1}^{n} N_j(r_1)N_i(r_2) d_{ij} N_i^*(r'_1)N_j^*(r'_2) \nonumber \\
   &&-  \frac{1}{2}\sum_{i,j=1}^{n} N_j(r_1)N_i(r_2) d_{ij} N_j^*(r'_1)N_i^*(r'_2) \\
   &=& -\Gamma(r_1,r_2,r'_1,r'_2)
\end{eqnarray}
Verifying that $\Gamma(r_2,r_1,r'_1,r'_2) = -\Gamma(r_1,r_2,r'_1,r'_2)$ proceeds the same way.
Finally, 
$\Gamma(r_2,r_1,r'_2,r'_1) = -\Gamma(r_1,r_2,r'_2,r'_1)= -(-\Gamma(r_1,r_2,r'_1,r'_2)) = \Gamma(r_1,r_2,r'_1,r'_2)$
which concludes the proof.

\subsection{Obtaining the 1-electron density matrix from the 2-electron density matrix}
\dlabel{Sect:Integrate}

In this section it will be shown that
Eq.~\ref{Eq:D1:di:r} can be obtained from Eq.~\ref{Eq:D2:dij:r} by integrating the coordinates
of electron 2 out. As the $\alpha\beta$ part of this issue is trivial this section focusses
on the relation between the $\alpha$-spin 1-electron density matrix and the $\alpha\alpha$ part of
the 2-electron density matrix by assuming that the contains only $\alpha$ electrons. 
Hence, it is to be shown that
\begin{eqnarray}
   D(r_1,r'_1) = \frac{2}{\nela-1}\int\Gamma(r_1,r_2,r'_1,r_2)\mathrm{d}r_2
\end{eqnarray}
The right-hand-side of this equation can be written out as
\begin{eqnarray}
  \frac{2}{\nela-1}\int\Gamma(r_1,r_2,r'_1,r_2)\mathrm{d}r_2 \nonumber \\
  &&= \frac{1}{\nela-1}\sum_{i,j=1}^{n}\int N_i(r_1)N_j(r_2) d_{ij} N_i^*(r'_1)N_j^*(r_2)\mathrm{d}r_2 \nonumber \\
  &&- \frac{1}{\nela-1}\sum_{i,j=1}^{n}\int N_i(r_1)N_j(r_2) d_{ij} N_j^*(r'_1)N_i^*(r_2)\mathrm{d}r_2 \nonumber \\
  &&- \frac{1}{\nela-1}\sum_{i,j=1}^{n}\int N_j(r_1)N_i(r_2) d_{ij} N_i^*(r'_1)N_j^*(r_2)\mathrm{d}r_2 \nonumber \\
  &&+ \frac{1}{\nela-1}\sum_{i,j=1}^{n}\int N_j(r_1)N_i(r_2) d_{ij} N_j^*(r'_1)N_i^*(r_2)\mathrm{d}r_2 
\end{eqnarray}
Because of the orthonormality of the natural orbitals it follows
\begin{eqnarray}
  \frac{2}{\nela-1}\int\Gamma(r_1,r_2,r'_1,r_2)\mathrm{d}r_2 \nonumber \\
  &&= \frac{1}{\nela-1}\sum_{i,j=1}^{n}     N_i(r_1)         d_{ij} N_i^*(r'_1) \cdot 1                \nonumber \\
  &&- \frac{1}{\nela-1}\sum_{i,j=1}^{n}     N_i(r_1)         d_{ij} N_j^*(r'_1) \cdot \delta_{ij}      \nonumber \\
  &&- \frac{1}{\nela-1}\sum_{i,j=1}^{n}     N_j(r_1)         d_{ij} N_i^*(r'_1) \cdot \delta_{ij}      \nonumber \\
  &&+ \frac{1}{\nela-1}\sum_{i,j=1}^{n}     N_j(r_1)         d_{ij} N_j^*(r'_1) \cdot 1                
\end{eqnarray}
As $d_{ij} = 0$ if $i=j$ and $\delta_{ij} = 0$ if $i \ne j$ it is clear that the second and third terms are 
both $0$. Furthermore, as a result of summing over all combinations of $i$ and $j$ the first and fourth terms
are the same. Hence,
\begin{eqnarray}
  \frac{2}{\nela-1}\int\Gamma(r_1,r_2,r'_1,r_2)\mathrm{d}r_2 \nonumber \\
  &&= \frac{2}{\nela-1}\sum_{i,j=1}^{n} N_i(r_1) d_{ij} N_i^*(r'_1) 
\end{eqnarray}
Substituting Eq.~\ref{Eq:D2:dij} for $d_{ij}$ gives
\begin{eqnarray}
  \frac{2}{\nela-1}\int\Gamma(r_1,r_2,r'_1,r_2)\mathrm{d}r_2 \nonumber \\
  &&= \frac{1}{\nela-1}\sum_{s,t=1}^{\nela}\sum_{i,j=1}^{n}
      N_i(r_1) N_i^*(r'_1) C_{is}C_{is}^* C_{jt}C_{jt}^* \nonumber \\
  &&- \frac{1}{\nela-1}\sum_{s,t=1}^{\nela}\sum_{i,j=1}^{n}
      N_i(r_1) N_i^*(r'_1) C_{is}C_{it}^* C_{jt}C_{js}^* \nonumber \\
  &&- \frac{1}{\nela-1}\sum_{s,t=1}^{\nela}\sum_{i,j=1}^{n}
      N_i(r_1) N_i^*(r'_1) C_{it}C_{is}^* C_{js}C_{jt}^* \nonumber \\
  &&+ \frac{1}{\nela-1}\sum_{s,t=1}^{\nela}\sum_{i,j=1}^{n}
      N_i(r_1) N_i^*(r'_1) C_{it}C_{it}^* C_{js}C_{js}^* 
\end{eqnarray}
Because of the orthonormality of the correlation functions the following substitutions 
can be made
\begin{eqnarray}
  \sum_{t=1}^{\nela}\sum_{j}C_{jt}C^*_{jt} 
  &=& \nela \\
  \sum_{t=1}^{\nela}\sum_{j}C_{jt}C^*_{js}
  &=& \delta_{st}
\end{eqnarray}
to give
\begin{eqnarray}
  \frac{2}{\nela-1}\int\Gamma(r_1,r_2,r'_1,r_2)\mathrm{d}r_2 \nonumber \\
  &&= \frac{1}{\nela-1}\sum_{s=1}^{\nela}\sum_{i=1}^{n}
      N_i(r_1) N_i^*(r'_1) C_{is}C_{is}^* \nela \nonumber \\
  &&- \frac{1}{\nela-1}\sum_{s,t=1}^{\nela}\sum_{i=1}^{n}
      N_i(r_1) N_i^*(r'_1) C_{is}C_{it}^* \delta_{ts} \nonumber \\
  &&- \frac{1}{\nela-1}\sum_{s,t=1}^{\nela}\sum_{i=1}^{n}
      N_i(r_1) N_i^*(r'_1) C_{it}C_{is}^* \delta_{st} \nonumber \\
  &&+ \frac{1}{\nela-1}\sum_{t=1}^{\nela}\sum_{i=1}^{n}
      N_i(r_1) N_i^*(r'_1) C_{it}C_{it}^* \nela \\
  &&= \frac{1}{\nela-1}\sum_{s=1}^{\nela}\sum_{i=1}^{n}
      N_i(r_1) N_i^*(r'_1) C_{is}C_{is}^* \nela \nonumber \\
  &&- \frac{1}{\nela-1}\sum_{s=1}^{\nela}\sum_{i=1}^{n}
      N_i(r_1) N_i^*(r'_1) C_{is}C_{is}^*  \nonumber \\
  &&- \frac{1}{\nela-1}\sum_{s=1}^{\nela}\sum_{i=1}^{n}
      N_i(r_1) N_i^*(r'_1) C_{is}C_{is}^*  \nonumber \\
  &&+ \frac{1}{\nela-1}\sum_{t=1}^{\nela}\sum_{i=1}^{n}
      N_i(r_1) N_i^*(r'_1) C_{it}C_{it}^* \nela \\
  &&= \frac{2}{\nela-1}\sum_{s=1}^{\nela}\sum_{i=1}^{n}
      N_i(r_1) N_i^*(r'_1) C_{is}C_{is}^* (\nela-1) \\
  &&= 2\sum_{s=1}^{\nela}\sum_{i=1}^{n}
      N_i(r_1) N_i^*(r'_1) C_{is}C_{is}^* \\
  &&= 2D(r_1,r'_1)
\end{eqnarray}
{\bf check factor 2} 
This concludes the proof.

\subsection{Summary}

The proofs the N-representability of the 1-electron density matrix given
in~\cite{van_Dam_2016} and those of the 2-electron density matrix given here 
show that Eqs.~\ref{Eq:D1:di:r} and~\ref{Eq:D2:dij:r} do offer N-representable
density matrices even if the recipe to generate those is a bit unusual.

Also remember that the form of the 1-electron density matrix admits the exact
1-electron density matrix even of a correlated system. The 2-electron density
matrix expression admits fractional natural orbital pair occupation numbers,
but it does not represent correlated states (eigen functions of a correlated
2-electron density matrix would be geminals, i.e. linear combinations of natural
orbital pairs). 

Hence, the result is an effective 1-electron model that is N-representable
even in the presence of fractionally occupied natural orbitals. However,
as this model does not capture electron correlation to drive fractional
occupation numbers, the next step has to be deriving some expression
for the electron correlation energy.

\subsection{Electron correlation by comparing the Full-CI and WFN1 energy expressions for \ce{H2} in
            a minimal basis set}
\dlabel{sect:FCI:H2:min}

The model that has been derived so far generates N-representable energy expressions even
in the presence of fractional occupation numbers. However, without a description of 
electron correlation effects there are no effects that drive taking advantage of this
feature. Of course the energy expression derived here is rather different from the 
more conventional energy expressions that can be obtained of conventional configuration
interaction (CI) methods. A straightforward approach to investigating what correlation 
effects in the model formulated here might look like is to compare energy expressions
of Full-CI~\cite{Ross_1952} (or exact diagonalization) to the energy expressions of the model 
formulated here for a given molecular system.

A typical system that comes to mind for such an analysis is the hydrogen molecule in a 
minimal basis set. This is a system consisting of only two electrons represented in a basis
set of two spatial orbitals. For a system this simple the exact energy expressions for both
Full-CI as well as the model proposed here can easily be formulated. By comparing the 
resulting energy expressions it should be possible to establish what the form of the
correlation energy for this system is.

First of all in this simple representation there are only two orbitals which are fixed by
symmetry. These orbitals are the bonding and the anti-bonding orbitals, and assuming that
the hydrogen atoms are labeled $A$ and $B$ these orbitals are
\begin{eqnarray}
   \phi_1(r) &=& (\chi_A(r)+\chi_B(r))/\sqrt{2} \\
   \phi_2(r) &=& (\chi_A(r)-\chi_B(r))/\sqrt{2} 
\end{eqnarray}
In terms of these orbitals we can write down the Full-CI wave function as a sum of
Slater determinants. Although, because we have only 1 $\alpha$-electron and 1 $\beta$-electron
it is actually sufficient to just list the Hartree products.
\begin{eqnarray}
   \Psi &=& C_{11}\phi^\alpha_1(r_1)\phi^\beta_1(r_2)
         +  C_{12}\phi^\alpha_1(r_1)\phi^\beta_2(r_2)
         +  C_{21}\phi^\alpha_2(r_1)\phi^\beta_1(r_2)
         +  C_{22}\phi^\alpha_2(r_1)\phi^\beta_2(r_2)
\end{eqnarray}
By symmetry we have that $C_{12} = C_{21} = 0$. Furthermore because the wave function must be
normalized we have
\begin{eqnarray}
   1 &=& C_{11}^2+C_{22}^2
\end{eqnarray}
Hence $C_{22} = \pm \sqrt{1-C_{11}^2}$. Ultimately the 2-electron contribution to the total
energy can be written as $\tr(R,D)$ where $R$ is the matrix of 2-electron repulsion integrals,
$D$ is the 2-electron density matrix.
The electron repulsion matrix is
\begin{eqnarray}
   R &=& 
   \begin{pmatrix}
   \eria{1}{1}{1}{1} & \eria{1}{1}{1}{2} & \eria{1}{1}{2}{1} & \eria{1}{1}{2}{2} \\
   \eria{1}{2}{1}{1} & \eria{1}{2}{1}{2} & \eria{1}{2}{2}{1} & \eria{1}{2}{2}{2} \\
   \eria{2}{1}{1}{1} & \eria{2}{1}{1}{2} & \eria{2}{1}{2}{1} & \eria{2}{1}{2}{2} \\
   \eria{2}{2}{1}{1} & \eria{2}{2}{1}{2} & \eria{2}{2}{2}{1} & \eria{2}{2}{2}{2}
   \end{pmatrix} 
\end{eqnarray}
where $\eria{i}{j}{k}{l} = \int\int \phi_i^\alpha(r_1)\phi_j^\beta(r_2)\frac{1}{r_{12}}{\phi_k^\alpha}^*(r_1){\phi_l^\beta}^*(r_2)\mathrm{d}r_1\mathrm{d}r_2$.
Of course, because we are dealing with a 2-electron closed shell system electron 1 has $\alpha$-spin and
electron 2 has $\beta$-spin. I.e. $\langle i^\alpha j^\beta|k^\alpha l^\beta \rangle = 
(i^\alpha k^\alpha |j^\beta l^\beta)$.
The corresponding Full-CI 2-electron density matrix
\begin{eqnarray}
   D &=& 
   \begin{pmatrix}
   C_{11}^2     & 0 & 0 & C_{11}C^*_{22} \\
   0            & 0 & 0 & 0            \\
   0            & 0 & 0 & 0            \\
   C_{22}C^*_{11} & 0 & 0 & C_{22}^2
   \end{pmatrix}
   \dlabel{eq:fci:h2}
\end{eqnarray}
is obtained from $\Psi(r_1,r_2)\Psi^*(r'_1,r'_2)$.
By contrast, within the WFN1 approach the 2-electron density matrix, in this case, is constructed from the
1-electron density matrices. The 1-electron density matrix is constructed from the correlation functions.
The correlation functions are given by
\begin{eqnarray}
   \psi^\sigma(r_1)
        &=& c^\sigma_{1}\phi^\sigma_1(r_1)
         +  c^\sigma_{2}\phi^\sigma_2(r_1)
\end{eqnarray}
The corresponding 1-electron density matrix is
\begin{eqnarray}
   d &=& 
   \begin{pmatrix}
   \left.c_{1}^\sigma\right.^2 & 0                           \\
   0                           & \left.c_{2}^\sigma\right.^2
   \end{pmatrix} \\
   &=& 
   \begin{pmatrix}
   p_{1}^\sigma & 0            \\
   0            & p_{2}^\sigma
   \end{pmatrix}
\end{eqnarray}
and the 2-electron density matrix becomes
\begin{eqnarray}
  \Gamma &=&
  \begin{pmatrix}
  p_{1}^\alpha p_{1}^\beta & 0 & 0 & 0 \\
  0 & p_{1}^\alpha p_{2}^\beta & 0 & 0 \\
  0 & 0 & p_{2}^\alpha p_{1}^\beta & 0 \\
  0 & 0 & 0 & p_{2}^\alpha p_{2}^\beta 
  \end{pmatrix}
\end{eqnarray}
.
Comparing the 2-electron interaction as
\begin{eqnarray}
   T &=& E_{FullCI}-E_{WFN1} \\
   E_{FullCI} &=& E_{WFN1}+T
\end{eqnarray}
Then we have
\begin{eqnarray}
   E_{FullCI} &=& \eria{1}{1}{1}{1} C_{11}^2 + \eria{1}{1}{2}{2} C_{11}C_{22}^*
               +  \eria{2}{2}{1}{1} C_{22}C_{11}^* + \eria{2}{2}{2}{2} C_{22}^2 \\
   E_{WFN1}   &=& \eria{1}{1}{1}{1} p_1^\alpha p_1^\beta + \eria{1}{2}{1}{2} p_1^\alpha p_2^\beta
               +  \eria{2}{1}{2}{1} p_2^\alpha p_1^\beta + \eria{2}{2}{2}{2} p_2^\alpha p_2^\beta \\
   T
   &=& \eria{1}{1}{1}{1} (C_{11}^2-p_1^\alpha p_1^\beta)
    +  \eria{2}{2}{2}{2} (C_{22}^2-p_2^\alpha p_2^\beta) \nonumber \\
   &&+ \eria{1}{1}{2}{2} C_{11}C_{22}^*
    +  \eria{2}{2}{1}{1} C_{22}C_{11}^*       \nonumber \\
   &&- \eria{1}{2}{1}{2} p_1^\alpha p_2^\beta
    -  \eria{2}{1}{2}{1} p_2^\alpha p_1^\beta 
   \dlabel{Eq:T:subtract}
\end{eqnarray}
The Full-CI coefficients can equated to
\begin{eqnarray}
C_{11}^2 &=& \sqrt{p_1^\alpha p_1^\beta} 
             \dlabel{Eq:C2:11}
\end{eqnarray}
Based on this we also have that
\begin{eqnarray}
C_{11} &=& \pm\sqrt[4]{p_1^\alpha p_1^\beta} 
           \dlabel{Eq:C1:11}
\end{eqnarray}
Obviously in Eq.~\ref{Eq:T:subtract} we have to choose the sign of $C_{11}C^*_{22}$ and 
$C_{22}C^*_{11}$ so as to minimize
the Full-CI energy. If we now insert Eqs.~\ref{Eq:C2:11} and~\ref{Eq:C1:11} into Eq.~\ref{Eq:T:subtract} we have
\begin{eqnarray}
   T
   &=& \eria{1}{1}{1}{1} (\sqrt{p_1^\alpha p_1^\beta}-p_1^\alpha p_1^\beta)
    +  \eria{2}{2}{2}{2} (\sqrt{p_2^\alpha p_2^\beta}-p_2^\alpha p_2^\beta) \nonumber \\
   &&- \eria{1}{1}{2}{2} \sqrt[4]{p_1^\alpha p_1^\beta p_2^\alpha p_2^\beta}
    -  \eria{2}{2}{1}{1} \sqrt[4]{p_2^\alpha p_2^\beta p_1^\alpha p_1^\beta} \nonumber \\
   &&- \eria{1}{2}{1}{2} p_1^\alpha p_2^\beta
    -  \eria{2}{1}{2}{1} p_2^\alpha p_1^\beta 
   \dlabel{Eq:T:subtract:2}
\end{eqnarray}
In the following the terms in Eq.~\ref{Eq:T:subtract:2} are numbered 1 to 6 according to the 2-electron
integral involved.
Next we need to test what happens if one of the orbitals is not correlated. 
The fact that an
orbital is not correlated is evident from the occupation number being either $0$ or $1$. The argument is simply
that if in a pair of orbitals at least one is not correlated then the other orbital cannot be correlated
to it (even if the other orbital is correlated). Even if the other orbital is correlated it cannot be
correlated with an uncorrelated orbital, instead it has to be correlated to something else.
In other words, if the occupation of a particular orbital is $0$ or $1$ then $T$ of Eq.~\ref{Eq:T:subtract:2}
should be $0$. As written, all terms in Eq.~\ref{Eq:T:subtract:2} violate this condition. However, in the case
of a 2-electron system represented in 2-spatial orbitals there are additional conditions that constrain
the energy expression of Eq.~\ref{Eq:T:subtract:2} to avoid such problems. These constraints are not manifest
in Eq.~\ref{Eq:T:subtract:2}, however, but they can be explicitly introduced.
To address these issues there are two important relationships that hold for the closed shell 
2-electron molecules in a two orbital basis:
\begin{eqnarray}
  p_i^\sigma &=& 1-p_j^\sigma  \dlabel{Eq:constraint:2orb} \\
  p_i^\sigma &=& p_i^{\sigma'} \dlabel{Eq:constraint:closedshell}
\end{eqnarray}
Using these relationships one can replace the following quantity as
\begin{eqnarray}
  p_1^\alpha p_2^\beta 
  &=& \sqrt{p_1^\alpha p_1^\alpha p_2^\beta p_2^\beta} \\
  &=& \sqrt{p_1^\alpha(1-p_2^\alpha)p_2^\beta(1-p_1^\beta)} \\
  &=& \sqrt{p_1^\alpha(1-p_2^\beta)p_2^\beta(1-p_1^\alpha)} \\
  &=& \sqrt{p_1^\alpha(1-p_1^\alpha)p_2^\beta(1-p_2^\beta)}
\end{eqnarray}
where we have first used the fact that in a 2-orbital system with 1 electron
per spin the occupation numbers of the orbitals must add up to 1, in the second
step we have used the fact that in closed shell system the $\alpha$- and $\beta$-occupation
numbers are the same.
Likewise
\begin{eqnarray}
  p_1^\beta p_2^\alpha 
  &=& \sqrt{p_1^\beta(1-p_1^\beta)p_2^\alpha(1-p_2^\alpha)}
\end{eqnarray}
and
\begin{eqnarray}
  \sqrt[4]{p_1^\alpha p_1^\beta p_2^\alpha p_2^\beta}
  &=& \sqrt[4]{p_1^\alpha (1-p_2^\beta)(1-p_1^\alpha)p_2^\beta} \\
  &=& \sqrt[4]{p_1^\alpha (1-p_1^\alpha) p_2^\beta (1-p_2^\beta)} \\
  &=& \sqrt[4]{p_1^\beta (1-p_1^\beta) p_2^\alpha (1-p_2^\alpha)}
\end{eqnarray}
Using the relationships above in the expression for $T$ we may obtain
\begin{eqnarray}
   T
   &=& \eria{1}{1}{1}{1} \left(\sqrt{p_1^\alpha p_1^\beta}-p_1^\alpha p_1^\beta\right)
    +  \eria{2}{2}{2}{2} \left(\sqrt{p_2^\alpha p_2^\beta}-p_2^\alpha p_2^\beta\right) \nonumber \\
   &&- \eria{1}{1}{2}{2} \sqrt[4]{p_1^\alpha (1-p_1^\alpha) p_2^\beta (1-p_2^\beta)}
    -  \eria{2}{2}{1}{1} \sqrt[4]{p_2^\alpha (1-p_2^\alpha) p_1^\beta (1-p_1^\beta)} \nonumber \\
   &&- \eria{1}{2}{1}{2} \sqrt{p_1^\alpha (1-p_1^\alpha) p_2^\beta (1-p_2^\beta)}
    -  \eria{2}{1}{2}{1} \sqrt{p_2^\alpha (1-p_2^\alpha) p_1^\beta (1-p_1^\beta)}
   \dlabel{Eq:T:subtract:3}
\end{eqnarray}
Clearly, if either $p_1^\sigma$ or $p_2^\sigma$ is either $0$ or $1$ all terms $T_3$ to $T_6$ in
Eq.~\ref{Eq:T:subtract:3} are $0$. The only two terms that cannot be addressed in this way are the
first two terms of Eq.~\ref{Eq:T:subtract:3}. However, both these terms are non-negative. As the 
energy will be optimized by minimization it is expected that this will reduce the inappropriate
correlation behavior as much as possible by driving the occupation numbers to suitable values.

Note that the substitutions introduced above are exact, but only for closed shell systems with 
2-electrons in 2 spatial orbitals. In all other cases Eqs.~\ref{Eq:constraint:2orb} 
and~\ref{Eq:constraint:closedshell} do not hold. Hence, Eq.~\ref{Eq:T:subtract:3} is approximate
for all other systems, and its suitability for other systems needs to be tested.

%An important question related to extending this approximation to system with more
%electrons or more basis functions is how to do this. In the system studied thus
%far we have assumed only 2 spatial orbitals. In any realistic system there will likely
%be more orbitals. The fact that thus far we have only considered 2 orbital systems
%imposes a balance between the different terms in $T$. For example, in a 2 orbital system
%there are equally many $ii$ as $ij$ terms. For a system with more than two orbitals the
%number of $ij$ terms goes as $N^2$ whereas the number of $ii$ terms goes as $N$. 

To answer the questions of how to apply Eq.~\ref{Eq:T:subtract:3} more generally it is convenient
to split the expression into separate terms. 
First of all there are the diagonal terms $ii$:
\begin{eqnarray}
   T_{ii}       &=& \eria{i}{i}{i}{i} \left(\sqrt{p_i^\alpha p_i^\beta}-p_i^\alpha p_i^\beta\right)
                    \dlabel{Eq:Tii} \\
   T_{ii}       &=& T^{(2)}_{ii}-T^{(1)}_{ii}  \\
   T^{(1)}_{ii} &=& \eria{i}{i}{i}{i}p_i^\alpha p_i^\beta        \dlabel{Eq:Tii:1} \\
   T^{(2)}_{ii} &=& \eria{i}{i}{i}{i}\sqrt{p_i^\alpha p_i^\beta} \dlabel{Eq:Tii:2}
\end{eqnarray}
then there are the $ij$ terms where $i \ne j$:
\begin{eqnarray}
   T_{ij}       &=& -\eria{i}{i}{j}{j} \sqrt[4]{p_i^\alpha(1-p_i^\alpha) p_j^\beta(1-p_j^\beta)} \nonumber \\
                &-& \eria{i}{j}{i}{j} \sqrt{p_i^\alpha(1-p_i^\alpha) p_j^\beta(1-p_j^\beta)}
                    \dlabel{Eq:Tij} \\
   T_{ij}       &=& -T^{(4)}_{ij}-T^{(2)}_{ij} \\
   T^{(4)}_{ij} &=& \eria{i}{i}{j}{j}\sqrt[4]{p_i^\alpha(1-p_i^\alpha) p_j^\beta(1-p_j^\beta)} \dlabel{Eq:Tij:4} \\
   T^{(2)}_{ij} &=& \eria{i}{j}{i}{j}\sqrt{p_i^\alpha(1-p_i^\alpha) p_j^\beta(1-p_j^\beta)}    \dlabel{Eq:Tij:2}
\end{eqnarray}
The terms $T^{(1)}_{ii}$, $T^{(2)}_{ii}$, $T^{(2)}_{ij}$, and $T^{(4)}_{ij}$ are named based on the order of
the root involved as well as the pair of indeces.
The total $T$ can be written as
\begin{eqnarray}
   T &=& \sum_{i,j} T_{ij}
\end{eqnarray}
where $T_{ij}$ is given by either Eq.~\ref{Eq:Tii} or Eq.~\ref{Eq:Tij} depending on the values of $i$ and $j$.

As \ce{H2} has only 1 $\alpha$ and 1 $\beta$ electron it is clear that this can provide information
about the inter-spin electron correlation. Nothing can be learned from this system about the correlation
effects between electrons with the same spin. This might be resolved by looking at systems with more 
electrons such as have been used in the coupled cluster theory~\cite{Kowalski_1998,Surj_n_2001} investigating
the solutions of \ce{H4} and \ce{H8}.

\subsection{Electron correlation by comparing the Full-CI and WFN1 energy expressions for \ce{H2}
            in a double zeta basis set}

In the previous section it was shown that the 2 electrons in 2 orbitals can be solved exactly.
However, the resulting energy expression contains $T_{ij}$ terms which grow like $O(N^2)$ with
the size of the basis set. This creates an imbalance with the $T_{ii}$ which grow like $O(N)$.
A simple renormalization to address this problem failed as it resulted in a non-convex energy
expression, that generates as many different solutions as there are virtual orbitals. Clearly
such an approach is useless. To try and find a different way forward \ce{H2} in a double zeta
basis set (i.e. 4 orbitals instead of 2) is considered in this section.

To facilitate the comparison the equations will be expressed in terms of the Full-CI natural
orbitals. The symmetry of the system allows for bonding and anti-bonding orbitals as in
\begin{eqnarray}
   \phi_1(r) &=& (\chi_A^1(r)+\chi_B^1(r))/\sqrt{2} \\
   \phi_2(r) &=& (\chi_A^1(r)-\chi_B^1(r))/\sqrt{2} \\
   \phi_3(r) &=& (\chi_A^2(r)+\chi_B^2(r))/\sqrt{2} \\
   \phi_4(r) &=& (\chi_A^2(r)-\chi_B^2(r))/\sqrt{2} \\
\end{eqnarray}
In terms of these orbitals the wave function can be given as
\begin{eqnarray}
  \Psi &=& C_{11}\phi^\alpha_1(r_1)\phi^\beta_1(r_2)
        +  C_{13}\phi^\alpha_1(r_1)\phi^\beta_3(r_2)
        +  C_{31}\phi^\alpha_3(r_1)\phi^\beta_1(r_2)
        +  C_{33}\phi^\alpha_3(r_1)\phi^\beta_3(r_2) \nonumber \\
       &&+ C_{22}\phi^\alpha_2(r_1)\phi^\beta_2(r_2)
        +  C_{24}\phi^\alpha_2(r_1)\phi^\beta_4(r_2)
        +  C_{42}\phi^\alpha_4(r_1)\phi^\beta_2(r_2)
        +  C_{44}\phi^\alpha_4(r_1)\phi^\beta_4(r_2)
\end{eqnarray}
all other coefficients being zero by symmetry.
Because the wave function must be normalized we have
\begin{eqnarray}
  1 &=& C_{11}^2 + C_{13}^2 + C_{31}^2 + C_{33}^2 
     +  C_{22}^2 + C_{24}^2 + C_{42}^2 + C_{44}^2
\end{eqnarray}
Like in the previous section the 2-electron repulsion matrix can be given as
\begin{eqnarray}
  R &=& 
  \begin{pmatrix}
  \eria{1}{1}{1}{1} & \eria{1}{1}{1}{2} & \eria{1}{1}{1}{3} & \eria{1}{1}{1}{4} & \ldots & \eria{1}{1}{4}{4} \\
  \eria{1}{2}{1}{1} & \eria{1}{2}{1}{2} & \eria{1}{2}{1}{3} & \eria{1}{2}{1}{4} & \ldots & \eria{1}{2}{4}{4} \\
  \eria{1}{3}{1}{1} & \eria{1}{3}{1}{2} & \eria{1}{3}{1}{3} & \eria{1}{3}{1}{4} & \ldots & \eria{1}{3}{4}{4} \\
  \eria{1}{4}{1}{1} & \eria{1}{4}{1}{2} & \eria{1}{4}{1}{3} & \eria{1}{4}{1}{4} & \ldots & \eria{1}{4}{4}{4} \\
  \vdots            & \vdots            & \vdots            & \vdots            & \ddots & \vdots            \\
  \eria{4}{4}{1}{1} & \eria{4}{4}{1}{2} & \eria{4}{4}{1}{3} & \eria{4}{4}{1}{4} & \ldots & \eria{4}{4}{4}{4} \\
  \end{pmatrix}
\end{eqnarray}
The corresponding Full-CI 2-electron density matrix is given by
\setcounter{MaxMatrixCols}{20}
\tiny
\begin{eqnarray}
   D &=&
   \begin{pmatrix}
    \Ctwo{1}{1}^2 &
    0 &
    \Ctwo{1}{1}\Ctwo{1}{3}^{*} &
    0 &
    0 &
    \Ctwo{1}{1}\Ctwo{2}{2}^{*} &
    0 &
    \Ctwo{1}{1}\Ctwo{2}{4}^{*} &
    \Ctwo{1}{1}\Ctwo{3}{1}^{*} &
    0 &
    \Ctwo{1}{1}\Ctwo{3}{3}^{*} &
    0 &
    0 &
    \Ctwo{1}{1}\Ctwo{4}{2}^{*} &
    0 &
    \Ctwo{1}{1}\Ctwo{4}{4}^{*} \\
    0 &
    0 &
    0 &
    0 &
    0 &
    0 &
    0 &
    0 &
    0 &
    0 &
    0 &
    0 &
    0 &
    0 &
    0 &
    0 \\
    \Ctwo{1}{3}\Ctwo{1}{1}^{*} &
    0 &
    \Ctwo{1}{3}^2 &
    0 &
    0 &
    \Ctwo{1}{3}\Ctwo{2}{2}^{*} &
    0 &
    \Ctwo{1}{3}\Ctwo{2}{4}^{*} &
    \Ctwo{1}{3}\Ctwo{3}{1}^{*} &
    0 &
    \Ctwo{1}{3}\Ctwo{3}{3}^{*} &
    0 &
    0 &
    \Ctwo{1}{3}\Ctwo{4}{2}^{*} &
    0 &
    \Ctwo{1}{3}\Ctwo{4}{4}^{*} \\
    0 &
    0 &
    0 &
    0 &
    0 &
    0 &
    0 &
    0 &
    0 &
    0 &
    0 &
    0 &
    0 &
    0 &
    0 &
    0 \\
    0 &
    0 &
    0 &
    0 &
    0 &
    0 &
    0 &
    0 &
    0 &
    0 &
    0 &
    0 &
    0 &
    0 &
    0 &
    0 \\
    \Ctwo{2}{2}\Ctwo{1}{1}^{*} &
    0 &
    \Ctwo{2}{2}\Ctwo{1}{3}^{*} &
    0 &
    0 &
    \Ctwo{2}{2}^2 &
    0 &
    \Ctwo{2}{2}\Ctwo{2}{4}^{*} &
    \Ctwo{2}{2}\Ctwo{3}{1}^{*} &
    0 &
    \Ctwo{2}{2}\Ctwo{3}{3}^{*} &
    0 &
    0 &
    \Ctwo{2}{2}\Ctwo{4}{2}^{*} &
    0 &
    \Ctwo{2}{2}\Ctwo{4}{4}^{*} \\
    0 &
    0 &
    0 &
    0 &
    0 &
    0 &
    0 &
    0 &
    0 &
    0 &
    0 &
    0 &
    0 &
    0 &
    0 &
    0 \\
    \Ctwo{2}{4}\Ctwo{1}{1}^{*} &
    0 &
    \Ctwo{2}{4}\Ctwo{1}{3}^{*} &
    0 &
    0 &
    \Ctwo{2}{4}\Ctwo{2}{2}^{*} &
    0 &
    \Ctwo{2}{4}^2 &
    \Ctwo{2}{4}\Ctwo{3}{1}^{*} &
    0 &
    \Ctwo{2}{4}\Ctwo{3}{3}^{*} &
    0 &
    0 &
    \Ctwo{2}{4}\Ctwo{4}{2}^{*} &
    0 &
    \Ctwo{2}{4}\Ctwo{4}{4}^{*} \\
    \Ctwo{3}{1}\Ctwo{1}{1}^{*} &
    0 &
    \Ctwo{3}{1}\Ctwo{1}{3}^{*} &
    0 &
    0 &
    \Ctwo{3}{1}\Ctwo{2}{2}^{*} &
    0 &
    \Ctwo{3}{1}\Ctwo{2}{4}^{*} &
    \Ctwo{3}{1}^2 &
    0 &
    \Ctwo{3}{1}\Ctwo{3}{3}^{*} &
    0 &
    0 &
    \Ctwo{3}{1}\Ctwo{4}{2}^{*} &
    0 &
    \Ctwo{3}{1}\Ctwo{4}{4}^{*} \\
    0 &
    0 &
    0 &
    0 &
    0 &
    0 &
    0 &
    0 &
    0 &
    0 &
    0 &
    0 &
    0 &
    0 &
    0 &
    0 \\
    \Ctwo{3}{3}\Ctwo{1}{1}^{*} &
    0 &
    \Ctwo{3}{3}\Ctwo{1}{3}^{*} &
    0 &
    0 &
    \Ctwo{3}{3}\Ctwo{2}{2}^{*} &
    0 &
    \Ctwo{3}{3}\Ctwo{2}{4}^{*} &
    \Ctwo{3}{3}\Ctwo{3}{1}^{*} &
    0 &
    \Ctwo{3}{3}^2 &
    0 &
    0 &
    \Ctwo{3}{3}\Ctwo{4}{2}^{*} &
    0 &
    \Ctwo{3}{3}\Ctwo{4}{4}^{*} \\
    0 &
    0 &
    0 &
    0 &
    0 &
    0 &
    0 &
    0 &
    0 &
    0 &
    0 &
    0 &
    0 &
    0 &
    0 &
    0 \\
    0 &
    0 &
    0 &
    0 &
    0 &
    0 &
    0 &
    0 &
    0 &
    0 &
    0 &
    0 &
    0 &
    0 &
    0 &
    0 \\
    \Ctwo{4}{2}\Ctwo{1}{1}^{*} &
    0 &
    \Ctwo{4}{2}\Ctwo{1}{3}^{*} &
    0 &
    0 &
    \Ctwo{4}{2}\Ctwo{2}{2}^{*} &
    0 &
    \Ctwo{4}{2}\Ctwo{2}{4}^{*} &
    \Ctwo{4}{2}\Ctwo{3}{1}^{*} &
    0 &
    \Ctwo{4}{2}\Ctwo{3}{3}^{*} &
    0 &
    0 &
    \Ctwo{4}{2}^2 &
    0 &
    \Ctwo{4}{2}\Ctwo{4}{4}^{*} \\
    0 &
    0 &
    0 &
    0 &
    0 &
    0 &
    0 &
    0 &
    0 &
    0 &
    0 &
    0 &
    0 &
    0 &
    0 &
    0 \\
    \Ctwo{4}{4}\Ctwo{1}{1}^{*} &
    0 &
    \Ctwo{4}{4}\Ctwo{1}{3}^{*} &
    0 &
    0 &
    \Ctwo{4}{4}\Ctwo{2}{2}^{*} &
    0 &
    \Ctwo{4}{4}\Ctwo{2}{4}^{*} &
    \Ctwo{4}{4}\Ctwo{3}{1}^{*} &
    0 &
    \Ctwo{4}{4}\Ctwo{3}{3}^{*} &
    0 &
    0 &
    \Ctwo{4}{4}\Ctwo{4}{2}^{*} &
    0 &
    \Ctwo{4}{4}^2 \\
   \end{pmatrix}
\end{eqnarray}
\normalsize
Considering the correlation functions we have
\begin{eqnarray}
   \psi^\sigma(r_1)
        &=& c^\sigma_{1}\phi^\sigma_1(r_1)
         +  c^\sigma_{2}\phi^\sigma_2(r_1)
         +  c^\sigma_{3}\phi^\sigma_3(r_1)
         +  c^\sigma_{4}\phi^\sigma_4(r_1)
\end{eqnarray}
The corresponding 1-electron density matrix is
\begin{eqnarray}
   d &=&
   \begin{pmatrix}
   \left.c_{1}^\sigma\right.^2 & 0                           & 0                           & 0 \\
   0                           & \left.c_{2}^\sigma\right.^2 & 0                           & 0 \\
   0                           & 0                           & \left.c_{3}^\sigma\right.^2 & 0 \\
   0                           & 0                           & 0                           & \left.c_{4}^\sigma\right.^2
   \end{pmatrix} \\
   &=&
   \begin{pmatrix}
   p_{1}^\sigma & 0            & 0            & 0 \\
   0            & p_{2}^\sigma & 0            & 0 \\
   0            & 0            & p_{3}^\sigma & 0 \\
   0            & 0            & 0            & p_{4}^\sigma
   \end{pmatrix}
\end{eqnarray}
and the 2-electron density matrix becomes
\begin{eqnarray}
  \Gamma &=&
  \begin{pmatrix}
  p_{1}^\alpha p_{1}^\beta & 0 & 0 & 0 & 0 & 0 & 0 & 0 & 0 & 0 & 0 & 0 & 0 & 0 & 0 & 0 \\
  0 & p_{1}^\alpha p_{2}^\beta & 0 & 0 & 0 & 0 & 0 & 0 & 0 & 0 & 0 & 0 & 0 & 0 & 0 & 0 \\
  0 & 0 & p_{1}^\alpha p_{3}^\beta & 0 & 0 & 0 & 0 & 0 & 0 & 0 & 0 & 0 & 0 & 0 & 0 & 0 \\
  0 & 0 & 0 & p_{1}^\alpha p_{4}^\beta & 0 & 0 & 0 & 0 & 0 & 0 & 0 & 0 & 0 & 0 & 0 & 0 \\
  0 & 0 & 0 & 0 & p_{2}^\alpha p_{1}^\beta & 0 & 0 & 0 & 0 & 0 & 0 & 0 & 0 & 0 & 0 & 0 \\
  0 & 0 & 0 & 0 & 0 & p_{2}^\alpha p_{2}^\beta & 0 & 0 & 0 & 0 & 0 & 0 & 0 & 0 & 0 & 0 \\
  0 & 0 & 0 & 0 & 0 & 0 & p_{2}^\alpha p_{3}^\beta & 0 & 0 & 0 & 0 & 0 & 0 & 0 & 0 & 0 \\
  0 & 0 & 0 & 0 & 0 & 0 & 0 & p_{2}^\alpha p_{4}^\beta & 0 & 0 & 0 & 0 & 0 & 0 & 0 & 0 \\
  0 & 0 & 0 & 0 & 0 & 0 & 0 & 0 & p_{3}^\alpha p_{1}^\beta & 0 & 0 & 0 & 0 & 0 & 0 & 0 \\
  0 & 0 & 0 & 0 & 0 & 0 & 0 & 0 & 0 & p_{3}^\alpha p_{2}^\beta & 0 & 0 & 0 & 0 & 0 & 0 \\
  0 & 0 & 0 & 0 & 0 & 0 & 0 & 0 & 0 & 0 & p_{3}^\alpha p_{3}^\beta & 0 & 0 & 0 & 0 & 0 \\
  0 & 0 & 0 & 0 & 0 & 0 & 0 & 0 & 0 & 0 & 0 & p_{3}^\alpha p_{4}^\beta & 0 & 0 & 0 & 0 \\
  0 & 0 & 0 & 0 & 0 & 0 & 0 & 0 & 0 & 0 & 0 & 0 & p_{4}^\alpha p_{1}^\beta & 0 & 0 & 0 \\
  0 & 0 & 0 & 0 & 0 & 0 & 0 & 0 & 0 & 0 & 0 & 0 & 0 & p_{4}^\alpha p_{2}^\beta & 0 & 0 \\
  0 & 0 & 0 & 0 & 0 & 0 & 0 & 0 & 0 & 0 & 0 & 0 & 0 & 0 & p_{4}^\alpha p_{3}^\beta & 0 \\
  0 & 0 & 0 & 0 & 0 & 0 & 0 & 0 & 0 & 0 & 0 & 0 & 0 & 0 & 0 & p_{4}^\alpha p_{4}^\beta
  \end{pmatrix}
\end{eqnarray}

\subsection{Electron correlation by comparing the Full-CI and WFN1 energy expressions for triplet \ce{H2}
            in a double zeta basis set}

In Section~\ref{sect:FCI:H2:min} it was shown that the 2 electrons in 2 orbitals can be solved exactly.
However, the resulting energy expression only describes the $\alpha$-$\beta$-electron correlation. It 
does not describe electron correlation between same spin electrons. Considering triplet \ce{H2} in 
a double zeta basis set a system of 1-electron in 2 orbitals is obtained (after accounting for symmetry)
that like the singlet system in a minimal basis should allow for an exact solution.

To facilitate the comparison the equations will be expressed in terms of the Hartree-Fock ground state
orbitals. The symmetry of the system allows for bonding and anti-bonding orbitals as in
\begin{eqnarray}
   \phi_1(r) &=& (\chi_A^1(r)+\chi_B^1(r))/\sqrt{2} \\
   \phi_2(r) &=& (\chi_A^1(r)-\chi_B^1(r))/\sqrt{2} \\
   \phi_3(r) &=& (\chi_A^2(r)+\chi_B^2(r))/\sqrt{2} \\
   \phi_4(r) &=& (\chi_A^2(r)-\chi_B^2(r))/\sqrt{2} \\
\end{eqnarray}
where orbitals $\phi_1$ and $\phi_3$ are of the same irreducible representation, likewise
orbitals $\phi_2$ and $\phi_4$ are of the same irreducible representation.
In terms of these orbitals the wave function of the triplet state can be given as
\begin{eqnarray}
  \Psi &=& C_{12}|\phi^\alpha_1(r_1)\phi^\alpha_2(r_2)|
        +  C_{14}|\phi^\alpha_1(r_1)\phi^\alpha_4(r_2)|
        +  C_{32}|\phi^\alpha_3(r_1)\phi^\alpha_2(r_2)|
        +  C_{34}|\phi^\alpha_3(r_1)\phi^\alpha_4(r_2)|
\end{eqnarray}
all other coefficients being zero by symmetry.
In addition because of the Brillouin theorem~\cite{Surj_n_1989} the singly excited states do not interact
with the Hartree-Fock state and hence $C_{14}$ and $C_{32}$ are zero.
Because the wave function must be normalized we have
\begin{eqnarray}
  1 &=& C_{12}^2 + C_{34}^2
\end{eqnarray}
Like in the previous section the 2-electron repulsion matrix can be given as
\begin{eqnarray}
  R &=& 
  \begin{pmatrix}
  \erib{1}{2}{1}{2} & \erib{1}{2}{1}{4} & \erib{1}{2}{3}{2} & \erib{1}{2}{3}{4} \\
  \erib{1}{4}{1}{2} & \erib{1}{4}{1}{4} & \erib{1}{4}{3}{2} & \erib{1}{4}{3}{4} \\
  \erib{3}{2}{1}{2} & \erib{3}{2}{1}{4} & \erib{3}{2}{3}{2} & \erib{3}{2}{3}{4} \\
  \erib{3}{4}{1}{2} & \erib{3}{4}{1}{4} & \erib{3}{4}{3}{2} & \erib{3}{4}{3}{4} 
  \end{pmatrix}
\end{eqnarray}
where $||$ indicates that the integral is anti-symmetrized. 
The corresponding Full-CI 2-electron density matrix
\begin{eqnarray}
   D &=& 
   \begin{pmatrix}
   C_{12}^2     & 0 & 0 & C_{12}C^*_{34} \\
   0            & 0 & 0 & 0            \\
   0            & 0 & 0 & 0            \\
   C_{34}C^*_{12} & 0 & 0 & C_{34}^2
   \end{pmatrix}
   \dlabel{eq:fci:3h2}
\end{eqnarray}
is obtained from $\Psi(r_1,r_2)\Psi^*(r'_1,r'_2)$.

The correlation functions are given by
\begin{eqnarray}
   \psi_s^\sigma(r_1)
        &=& c^\sigma_{1s}\phi^\sigma_1(r_1)
         +  c^\sigma_{2s}\phi^\sigma_2(r_1)
         +  c^\sigma_{3s}\phi^\sigma_3(r_1)
         +  c^\sigma_{4s}\phi^\sigma_4(r_1)
\end{eqnarray}
From the correlation functions the 1-electron density matrix is
\begin{eqnarray}
   d &=&
   \begin{pmatrix}
   \left.c_{11}^\sigma\right.^2+\left.c_{12}^\sigma\right.^2 & 0 & 0 & 0 \\
   0 & \left.c_{21}^\sigma\right.^2+\left.c_{22}^\sigma\right.^2 & 0 & 0 \\
   0 & 0 & \left.c_{31}^\sigma\right.^2+\left.c_{32}^\sigma\right.^2 & 0 \\
   0 & 0 & 0 & \left.c_{41}^\sigma\right.^2+\left.c_{42}^\sigma\right.^2 
   \end{pmatrix} \\
   &=&
   \begin{pmatrix}
   p_{1}^\sigma & 0            & 0            & 0                \\
   0            & p_{2}^\sigma & 0            & 0                \\
   0            & 0            & p_{3}^\sigma & 0                \\
   0            & 0            & 0            & p_{4}^\sigma & 0 
   \end{pmatrix}
\end{eqnarray}
The corresponding 2-electron density matrix becomes
\begin{eqnarray}
  \Gamma &=&
  \begin{pmatrix}
  \left.c_{11}^\sigma\right.^2 \left.c_{22}^\sigma\right.^2 + \left.c_{12}^\sigma\right.^2\left.c_{21}^\sigma\right.^2
  -c_{11}^\sigma c_{12}^\sigma c_{22}^\sigma c_{21}^\sigma - c_{12}^\sigma c_{11}^\sigma c_{21}^\sigma c_{22}^\sigma
   & 0 & 0 & 0 \\
  0 & p_{1}^\alpha p_{2}^\beta & 0 & 0 \\
  0 & 0 & p_{2}^\alpha p_{1}^\beta & 0 \\
  0 & 0 & 0 & p_{2}^\alpha p_{2}^\beta
  \end{pmatrix}
\end{eqnarray}
Comparing the 2-electron interaction as
\begin{eqnarray}
  T = E_{FullCI} - E_{WFN1}
\end{eqnarray}
Then we have
\begin{eqnarray}
  E_{FullCI} &=& \erib{1}{2}{1}{2}C_{12}^2 +
                 \erib{1}{2}{3}{4}C_{12}C_{34}^* +
                 \erib{3}{4}{1}{2}C_{34}C_{12}^* +
                 \erib{3}{4}{3}{4}C_{34}^2 \\
  E_{WFN1}
  &=& 
\end{eqnarray}

\subsection{Electron correlation by comparing the Full-CI and WFN1 energy expressions for \ce{H4}}

In the previous section \ce{H2} in a minimal basis set was consider and it was shown 
that the exact total energy can be obtained with a suitable expression for the 
correlation energy. This expression can be obtained by simply comparing the
Full-CI energy and the WFN1 energy. However, this expression has a number of limitations:
1. It is valid only for 2 electrons in 2 orbitals; 2. It does not give any insight in
the correlation effects between electrons of the same spin. In particular the first
limitation is rather severe. Attempts to apply this energy expression to systems with larger
basis sets immediately led to energies that are far too low (there is an imbalance between
the $T_{ii}$ terms and the $T_{ij}$ simply because there are $O(N)$ of the first and $O(N^2)$
of the second). A renormalization of the $T_{ij}$ terms led to a variety of artifact: 
underestimation of the correlation effects with constant scaling factors; a non-convex energy
expression with adaptive scaling factor (and hence multiple ground state approximations). 

In a bid to resolve these issues a similar comparison is attempted here but for \ce{H4} in
a minimal basis set. In this system the hydrogen atoms are place on a straight line at equal
bond distances. Assuming the hydrogen atoms are labeled $A$, $B$, $C$, and $D$ then orbitals
are fixed by symmetry to be
\begin{eqnarray}
  \phi_1(r) &=& (\chi_A(r)+\chi_B(r)+\chi_C(r)+\chi_D(r))/2 \\
  \phi_2(r) &=& (\chi_A(r)+\chi_B(r)-\chi_C(r)-\chi_D(r))/2 \\
  \phi_3(r) &=& (\chi_A(r)-\chi_B(r)-\chi_C(r)+\chi_D(r))/2 \\
  \phi_4(r) &=& (\chi_A(r)-\chi_B(r)+\chi_C(r)-\chi_D(r))/2
\end{eqnarray}
Note that $\phi_1$ and $\phi_3$ have the same symmetry, as do $\phi_2$ and $\phi_4$.
The $\alpha$-spin part of the wave function is 
\begin{eqnarray}
  \Psi^\alpha &=& \Ca{12}|\Xa{1}\Xa{2}|+\Ca{13}|\Xa{1}\Xa{3}|+\Ca{14}|\Xa{1}\Xa{4}| \nonumber \\
              &+& \Ca{23}|\Xa{2}\Xa{3}|+\Ca{24}|\Xa{2}\Xa{4}|+\Ca{34}|\Xa{3}\Xa{4}|
\end{eqnarray}
the $\beta$-spin part of the wave function is similar. This leaves for the total 
wave functions
\begin{eqnarray}
  \Psi &=& \C{12}{12}\XXa{1}{2}\XXb{1}{2}
        +  \C{12}{14}\XXa{1}{2}\XXb{1}{4} \nonumber \\
       &+& \C{12}{23}\XXa{1}{2}\XXb{2}{3}
        +  \C{12}{34}\XXa{1}{2}\XXb{3}{4} \nonumber \\
       &+& \C{13}{13}\XXa{1}{3}\XXb{1}{3}
        +  \C{13}{24}\XXa{1}{3}\XXb{2}{4} \nonumber \\
       &+& \C{14}{12}\XXa{1}{4}\XXb{1}{2}
        +  \C{14}{14}\XXa{1}{4}\XXb{1}{4} \nonumber \\
       &+& \C{14}{23}\XXa{1}{4}\XXb{2}{3}
        +  \C{14}{34}\XXa{1}{4}\XXb{3}{4} \nonumber \\
       &+& \C{23}{12}\XXa{2}{3}\XXb{1}{2}
        +  \C{23}{14}\XXa{2}{3}\XXb{1}{4} \nonumber \\
       &+& \C{23}{23}\XXa{2}{3}\XXb{2}{3}
        +  \C{23}{34}\XXa{2}{3}\XXb{3}{4} \nonumber \\
       &+& \C{24}{13}\XXa{2}{4}\XXb{1}{3}
        +  \C{24}{24}\XXa{2}{4}\XXb{2}{4} \nonumber \\
       &+& \C{34}{12}\XXa{3}{4}\XXb{1}{2}
        +  \C{34}{14}\XXa{3}{4}\XXb{1}{4} \nonumber \\
       &+& \C{34}{23}\XXa{3}{4}\XXb{2}{3}
        +  \C{34}{34}\XXa{3}{4}\XXb{3}{4} 
\end{eqnarray}
Next we need to consider 3 different density matrices: 
\begin{itemize}
\item the $\alpha\alpha$ part of the 2-electron density matrix
\item the $\alpha\beta$ part of the 2-electron density matrix
\item the $\alpha$ part of the 1-electron density matrix
\end{itemize}
The $\alpha$ 1-electron density matrix is
\begin{eqnarray}
   \left(\begin{array}{cccc}
     \begin{array}{c}
     + \C{12}{12}^2 + \C{12}{14}^2 \\
     + \C{12}{23}^2 + \C{12}{34}^2 \\
     + \C{13}{13}^2 + \C{13}{24}^2 \\
     + \C{14}{12}^2 + \C{14}{14}^2 \\
     + \C{14}{23}^2 + \C{14}{34}^2 
     \end{array} &
     0 &
     \begin{array}{c}
     - \C{12}{12}\C{23}{12} \\
     - \C{12}{14}\C{23}{14} \\
     - \C{12}{23}\C{23}{23} \\
     - \C{12}{34}\C{23}{34} \\
     + \C{14}{12}\C{34}{12} \\
     + \C{14}{14}\C{34}{14} \\
     + \C{14}{23}\C{34}{23} \\
     + \C{14}{34}\C{34}{34} 
     \end{array} &
     0 \\
     0 &
     \begin{array}{c}
     + \C{12}{12}^2 + \C{12}{14}^2 \\
     + \C{12}{23}^2 + \C{12}{34}^2 \\
     + \C{23}{12}^2 + \C{23}{14}^2 \\
     + \C{23}{23}^2 + \C{23}{34}^2 \\
     + \C{24}{13}^2 + \C{24}{24}^2
     \end{array} &
     0 &
     \begin{array}{c}
     + \C{12}{12}\C{14}{12} \\
     + \C{12}{14}\C{14}{14} \\
     + \C{12}{23}\C{14}{23} \\
     + \C{12}{34}\C{14}{34} \\
     - \C{23}{12}\C{34}{12} \\
     - \C{23}{14}\C{34}{14} \\
     - \C{23}{23}\C{34}{23} \\
     - \C{23}{34}\C{34}{34} 
     \end{array} \\
     \begin{array}{c}
     - \C{23}{12}\C{12}{12} \\
     - \C{23}{14}\C{12}{14} \\
     - \C{23}{23}\C{12}{23} \\
     - \C{23}{34}\C{12}{34} \\
     + \C{34}{12}\C{14}{12} \\
     + \C{34}{14}\C{14}{14} \\
     + \C{34}{23}\C{14}{23} \\
     + \C{34}{34}\C{14}{34} 
     \end{array} &
     0 &
     \begin{array}{c}
     + \C{13}{13}^2 + \C{13}{24}^2 \\
     + \C{23}{12}^2 + \C{23}{14}^2 \\
     + \C{23}{23}^2 + \C{23}{34}^2 \\
     + \C{34}{12}^2 + \C{34}{14}^2 \\
     + \C{34}{23}^2 + \C{34}{34}^2
     \end{array} &
     0 \\
     0 &
     \begin{array}{c}
     + \C{14}{12}\C{12}{12} \\
     + \C{14}{14}\C{12}{14} \\
     + \C{14}{23}\C{12}{23} \\
     + \C{14}{34}\C{12}{34} \\
     - \C{34}{12}\C{23}{12} \\
     - \C{34}{14}\C{23}{14} \\
     - \C{34}{23}\C{23}{23} \\
     - \C{34}{34}\C{23}{34}
     \end{array} &
     0 &
     \begin{array}{c}
     + \C{14}{12}^2 + \C{14}{14}^2 \\
     + \C{14}{23}^2 + \C{14}{34}^2 \\
     + \C{24}{13}^2 + \C{24}{24}^2 \\
     + \C{34}{12}^2 + \C{34}{14}^2 \\
     + \C{34}{23}^2 + \C{34}{34}^2
     \end{array} 
   \end{array}\right)
\end{eqnarray}
Note that as these equations are expressed in the natural orbital basis
the off diagonal elements of the 1-electron density matrix have to be zero.

The $\alpha\alpha$ block of the 2-electron density matrix is
\begin{eqnarray}
   \left(\begin{array}{cccccc}
     \begin{array}{c}
     + \C{12}{12}^2 \\
     + \C{12}{14}^2 \\
     + \C{12}{23}^2 \\
     + \C{12}{34}^2
     \end{array} &
     0 &
     \begin{array}{c}
     + \C{12}{12}\C{14}{12} \\
     + \C{12}{14}\C{14}{14} \\
     + \C{12}{23}\C{14}{23} \\
     + \C{12}{34}\C{14}{34} 
     \end{array} &
     \begin{array}{c}
     + \C{12}{12}\C{23}{12} \\
     + \C{12}{14}\C{23}{14} \\
     + \C{12}{23}\C{23}{23} \\
     + \C{12}{34}\C{23}{34}
     \end{array} &
     0 &
     \begin{array}{c}
     + \C{12}{12}\C{34}{12} \\
     + \C{12}{14}\C{34}{14} \\
     + \C{12}{23}\C{34}{23} \\
     + \C{12}{34}\C{34}{34} 
     \end{array} \\
     0 &
     \begin{array}{c}
     + \C{13}{13}^2 \\
     + \C{13}{24}^2
     \end{array} &
     0 &
     0 &
     \begin{array}{c}
     + \C{13}{13}\C{24}{13} \\
     + \C{13}{24}\C{24}{24}
     \end{array} &
     0 \\
     \begin{array}{c}
     + \C{14}{12}\C{12}{12} \\
     + \C{14}{14}\C{12}{14} \\
     + \C{14}{23}\C{12}{23} \\
     + \C{14}{34}\C{12}{34} 
     \end{array} &
     0 &
     \begin{array}{c}
     + \C{14}{12}^2 \\
     + \C{14}{14}^2 \\
     + \C{14}{23}^2 \\
     + \C{14}{34}^2 
     \end{array} &
     \begin{array}{c}
     + \C{14}{12}\C{23}{12} \\
     + \C{14}{14}\C{23}{14} \\
     + \C{14}{23}\C{23}{23} \\
     + \C{14}{34}\C{23}{34} 
     \end{array} &
     0 &
     \begin{array}{c}
     + \C{14}{12}\C{34}{12} \\
     + \C{14}{14}\C{34}{14} \\
     + \C{14}{23}\C{34}{23} \\
     + \C{14}{34}\C{34}{34}
     \end{array} \\
     \begin{array}{c}
     + \C{23}{12}\C{12}{12} \\
     + \C{23}{14}\C{12}{14} \\
     + \C{23}{23}\C{12}{23} \\
     + \C{23}{34}\C{12}{34} 
     \end{array} &
     0 &
     \begin{array}{c}
     + \C{23}{12}\C{14}{12} \\
     + \C{23}{14}\C{14}{14} \\
     + \C{23}{23}\C{14}{23} \\
     + \C{23}{34}\C{14}{34}
     \end{array} &
     \begin{array}{c}
     + \C{23}{12}^2 \\
     + \C{23}{14}^2 \\
     + \C{23}{23}^2 \\
     + \C{23}{34}^2
     \end{array} &
     0 &
     \begin{array}{c}
     + \C{23}{12}\C{34}{12} \\
     + \C{23}{14}\C{34}{14} \\
     + \C{23}{23}\C{34}{23} \\
     + \C{23}{34}\C{34}{34}
     \end{array} \\
     0 &
     \begin{array}{c}
     + \C{24}{13}\C{13}{13} \\
     + \C{24}{24}\C{13}{24}
     \end{array} &
     0 &
     0 &
     \begin{array}{c}
     + \C{24}{13}^2 \\
     + \C{24}{24}^2
     \end{array} &
     0 \\
     \begin{array}{c}
     + \C{34}{12}\C{12}{12} \\
     + \C{34}{14}\C{12}{14} \\
     + \C{34}{23}\C{12}{23} \\
     + \C{34}{34}\C{12}{34}
     \end{array} &
     0 &
     \begin{array}{c}
     + \C{34}{12}\C{14}{12} \\
     + \C{34}{14}\C{14}{14} \\
     + \C{34}{23}\C{14}{23} \\
     + \C{34}{34}\C{14}{34} 
     \end{array} &
     \begin{array}{c}
     + \C{34}{12}\C{23}{12} \\
     + \C{34}{14}\C{23}{14} \\
     + \C{34}{23}\C{23}{23} \\
     + \C{34}{34}\C{23}{34} 
     \end{array} &
     0 &
     \begin{array}{c}
     + \C{34}{12}^2 \\
     + \C{34}{14}^2 \\
     + \C{34}{23}^2 \\
     + \C{34}{34}^2 
     \end{array} 
   \end{array}\right)
\end{eqnarray}

The $\alpha\beta$ block of the 2-electron density matrix is
\noindent
%\scalebox{0.125}{0.125}{%
\tiny
%\begingroup\makeatletter\def\f@size{1}\check@mathfonts
%\def\maketag@@@#1{\hbox{\m@th\large\normalfont#1}}%
\begin{eqnarray}
   \left(\begin{array}{cccccccccccccccc}
     \begin{array}{c}
     + \C{12}{12}^2 \\
     + \C{12}{14}^2 \\
     + \C{13}{13}^2 \\
     + \C{14}{12}^2 \\
     + \C{14}{14}^2
     \end{array} &
     0 &
     \begin{array}{c}
     - \C{12}{12}\C{12}{23} \\
     + \C{12}{14}\C{12}{34} \\
     - \C{14}{12}\C{14}{23} \\
     + \C{14}{14}\C{14}{34}
     \end{array} &
     0 &
     0 &
     \begin{array}{c}
     + \C{13}{13}\C{23}{23} \\
     + \C{14}{14}\C{24}{24} \\
     \end{array} &
     0 &
     \begin{array}{c}
     - \C{13}{13}\C{23}{34} \\
     - \C{14}{12}\C{24}{24}
     \end{array} &
     \begin{array}{c}
     - \C{12}{12}\C{23}{12} \\
     - \C{12}{14}\C{23}{14} \\
     + \C{14}{12}\C{34}{12} \\
     + \C{14}{14}\C{34}{14}
     \end{array} &
     0 &
     \begin{array}{c}
     + \C{12}{12}\C{23}{23} \\
     - \C{12}{14}\C{23}{34} \\
     - \C{14}{12}\C{34}{23} \\
     + \C{14}{14}\C{34}{34}
     \end{array} &
     0 &
     0 &
     \begin{array}{c}
     - \C{12}{14}\C{24}{24} \\
     - \C{13}{13}\C{34}{23} \\
     \end{array} &
     0 &
     \begin{array}{c}
     + \C{12}{12}\C{24}{24} \\
     + \C{13}{13}\C{34}{34}
     \end{array} \\
     0 &
     \begin{array}{c}
     + \C{12}{12}^2 \\
     + \C{12}{23}^2 \\
     + \C{13}{24}^2 \\
     + \C{14}{12}^2 \\
     + \C{14}{23}^2
     \end{array} &
     0 &
     \begin{array}{c}
     + \C{12}{12}\C{12}{14} \\
     - \C{12}{23}\C{12}{34} \\
     + \C{14}{12}\C{14}{14} \\
     - \C{14}{23}\C{14}{34}
     \end{array} &
     \begin{array}{c}
     + \C{13}{24}\C{23}{14} \\
     + \C{14}{23}\C{24}{13} 
     \end{array} &
     0 &
     \begin{array}{c}
     + \C{13}{24}\C{23}{34} \\
     + \C{14}{12}\C{24}{13} 
     \end{array} &
     0 &
     0 &
     \begin{array}{c}
     - \C{12}{12}\C{23}{12} \\
     - \C{12}{23}\C{23}{23} \\
     + \C{14}{12}\C{34}{12} \\
     + \C{14}{23}\C{34}{23}
     \end{array} &
     0 &
     \begin{array}{c}
     - \C{12}{12}\C{23}{14} \\
     + \C{12}{23}\C{23}{34} \\
     + \C{14}{12}\C{34}{14} \\
     - \C{14}{23}\C{34}{34}
     \end{array} &
     \begin{array}{c}
     - \C{12}{23}\C{24}{13} \\
     - \C{13}{24}\C{34}{14}
     \end{array} &
     0 &
     \begin{array}{c}
     - \C{12}{12}\C{24}{13} \\
     - \C{13}{24}\C{34}{34}
     \end{array} &
     0 \\
     \begin{array}{c}
     - \C{12}{23}\C{12}{12} \\
     + \C{12}{34}\C{12}{14} \\
     - \C{14}{23}\C{14}{12} \\
     + \C{14}{34}\C{14}{14}
     \end{array} &
     0 &
     \begin{array}{c}
     + \C{12}{23}^2 \\
     + \C{12}{34}^2 \\
     + \C{13}{13}^2 \\
     + \C{14}{23}^2 \\
     + \C{14}{34}^2
     \end{array} &
     0 &
     0 &
     \begin{array}{c}
     + \C{13}{13}\C{23}{12} \\
     + \C{14}{34}\C{24}{24}
     \end{array} &
     0 &
     \begin{array}{c}
     + \C{13}{13}\C{23}{14} \\
     + \C{14}{23}\C{24}{24}
     \end{array} &
     \begin{array}{c}
     + \C{12}{23}\C{23}{12} \\
     - \C{12}{34}\C{23}{14} \\
     - \C{14}{23}\C{34}{12} \\
     + \C{14}{34}\C{34}{14}
     \end{array} &
     0 &
     \begin{array}{c}
     - \C{12}{23}\C{23}{23} \\
     - \C{12}{34}\C{23}{34} \\
     + \C{14}{23}\C{34}{23} \\
     + \C{14}{34}\C{34}{34}
     \end{array} &
     0 &
     0 &
     \begin{array}{c}
     - \C{12}{34}\C{24}{24} \\
     - \C{13}{13}\C{34}{12}
     \end{array} &
     0 &
     \begin{array}{c}
     - \C{12}{23}\C{24}{24} \\
     - \C{13}{13}\C{34}{14}
     \end{array} \\
     0 &
     \begin{array}{c}
     + \C{12}{14}\C{12}{12} \\
     - \C{12}{34}\C{12}{23} \\
     + \C{14}{14}\C{14}{12} \\
     - \C{14}{34}\C{14}{23}
     \end{array} &
     0 &
     \begin{array}{c}
     + \C{12}{14}^2 \\
     + \C{12}{34}^2 \\
     + \C{13}{24}^2 \\
     + \C{14}{14}^2 \\
     + \C{14}{34}^2
     \end{array} &
     \begin{array}{c}
     - \C{13}{24}\C{23}{12} \\
     - \C{14}{34}\C{24}{13}
     \end{array} &
     0 &
     \begin{array}{c}
     + \C{13}{24}\C{23}{23} \\
     + \C{14}{14}\C{24}{13}
     \end{array} &
     0 &
     0 &
     \begin{array}{c}
     - \C{12}{14}\C{23}{12} \\
     + \C{12}{34}\C{23}{23} \\
     + \C{14}{14}\C{34}{12} \\
     - \C{14}{34}\C{34}{23}
     \end{array} &
     0 &
     \begin{array}{c}
     - \C{12}{14}\C{23}{14} \\
     - \C{12}{34}\C{23}{34} \\
     + \C{14}{14}\C{34}{14} \\
     + \C{14}{34}\C{34}{34}
     \end{array} &
     \begin{array}{c}
     + \C{12}{34}\C{24}{13} \\
     + \C{13}{24}\C{34}{12}
     \end{array} &
     0 &
     \begin{array}{c}
     - \C{12}{14}\C{24}{13} \\
     - \C{13}{24}\C{34}{23}
     \end{array} &
     0 \\
     0 &
     \begin{array}{c}
     + \C{23}{14}\C{13}{24} \\
     + \C{24}{13}\C{14}{23}
     \end{array} &
     0 &
     \begin{array}{c}
     - \C{23}{12}\C{13}{24} \\
     - \C{24}{13}\C{14}{34}
     \end{array} &
     \begin{array}{c}
     + \C{12}{12}^2 \\
     + \C{12}{14}^2 \\
     + \C{23}{12}^2 \\
     + \C{23}{14}^2 \\
     + \C{24}{13}^2
     \end{array} &
     0 &
     \begin{array}{c}
     - \C{12}{12}\C{12}{23} \\
     + \C{12}{14}\C{12}{34} \\
     - \C{23}{12}\C{23}{23} \\
     + \C{23}{14}\C{23}{34}
     \end{array} &
     0 &
     0 &
     \begin{array}{c}
     + \C{12}{14}\C{13}{24} \\
     + \C{24}{13}\C{34}{23}
     \end{array} &
     0 &
     \begin{array}{c}
     - \C{12}{12}\C{13}{24} \\
     - \C{24}{13}\C{34}{34}
     \end{array} &
     \begin{array}{c}
     + \C{12}{12}\C{14}{12} \\
     + \C{12}{14}\C{14}{14} \\
     - \C{23}{12}\C{34}{12} \\
     - \C{23}{14}\C{34}{14}
     \end{array} &
     0 &
     \begin{array}{c}
     - \C{12}{12}\C{14}{23} \\
     + \C{12}{14}\C{14}{34} \\
     + \C{23}{12}\C{34}{23} \\
     - \C{23}{14}\C{34}{34}
     \end{array} &
     0 \\
     \begin{array}{c}
     + \C{23}{23}\C{13}{13} \\
     + \C{24}{24}\C{14}{14}
     \end{array} &
     0 &
     \begin{array}{c}
     + \C{23}{12}\C{13}{13} \\
     + \C{24}{24}\C{14}{34}
     \end{array} &
     0 &
     0 &
     \begin{array}{c}
     + \C{12}{12}^2 \\
     + \C{12}{23}^2 \\
     + \C{23}{12}^2 \\
     + \C{23}{23}^2 \\
     + \C{24}{24}^2
     \end{array} &
     0 &
     \begin{array}{c}
     + \C{12}{12}\C{12}{14} \\
     - \C{12}{23}\C{12}{34} \\
     + \C{23}{12}\C{23}{14} \\
     - \C{23}{23}\C{23}{34}
     \end{array} &
     \begin{array}{c}
     + \C{12}{23}\C{13}{13} \\
     + \C{24}{24}\C{34}{14}
     \end{array} &
     0 &
     \begin{array}{c}
     + \C{12}{12}\C{13}{13} \\
     + \C{24}{24}\C{34}{34}
     \end{array} &
     0 &
     0 &
     \begin{array}{c}
     + \C{12}{12}\C{14}{12} \\
     + \C{12}{23}\C{14}{23} \\
     - \C{23}{12}\C{34}{12} \\
     - \C{23}{23}\C{34}{23}
     \end{array} &
     0 &
     \begin{array}{c}
     + \C{12}{12}\C{14}{14} \\
     - \C{12}{23}\C{14}{34} \\
     - \C{23}{12}\C{34}{14} \\
     + \C{23}{23}\C{34}{34}
     \end{array} \\
     0 &
     \begin{array}{c}
     + \C{23}{34}\C{13}{24} \\
     + \C{24}{13}\C{14}{12}
     \end{array} &
     0 &
     \begin{array}{c}
     + \C{23}{23}\C{13}{24} \\
     + \C{24}{13}\C{14}{14}
     \end{array} &
     \begin{array}{c}
     - \C{12}{23}\C{12}{12} \\
     + \C{12}{34}\C{12}{14} \\
     - \C{23}{23}\C{23}{12} \\
     + \C{23}{34}\C{23}{14}
     \end{array} &
     0 &
     \begin{array}{c}
     + \C{12}{23}^2 \\
     + \C{12}{34}^2 \\
     + \C{23}{23}^2 \\
     + \C{23}{34}^2 \\
     + \C{24}{13}^2
     \end{array} &
     0 &
     0 &
     \begin{array}{c}
     + \C{12}{34}\C{13}{24} \\
     + \C{24}{13}\C{34}{12}
     \end{array} &
     0 &
     \begin{array}{c}
     + \C{12}{23}\C{13}{24} \\
     + \C{24}{13}\C{34}{14}
     \end{array} &
     \begin{array}{c}
     - \C{12}{23}\C{14}{12} \\
     + \C{12}{34}\C{14}{14} \\
     + \C{23}{23}\C{34}{12} \\
     - \C{23}{34}\C{34}{14}
     \end{array} &
     0 &
     \begin{array}{c}
     + \C{12}{23}\C{14}{23} \\
     + \C{12}{34}\C{14}{34} \\
     - \C{23}{23}\C{34}{23} \\
     - \C{23}{34}\C{34}{34}
     \end{array} &
     0 \\
     \begin{array}{c}
     - \C{23}{34}\C{13}{13} \\
     - \C{24}{24}\C{14}{12}
     \end{array} &
     0 &
     \begin{array}{c}
     + \C{23}{14}\C{13}{13} \\
     + \C{24}{24}\C{14}{23}
     \end{array} &
     0 &
     0 &
     \begin{array}{c}
     + \C{12}{14}\C{12}{12} \\
     - \C{12}{34}\C{12}{23} \\
     + \C{23}{14}\C{23}{12} \\
     - \C{23}{34}\C{23}{23}
     \end{array} &
     0 &
     \begin{array}{c}
     + \C{12}{14}^2 \\
     + \C{12}{34}^2 \\
     + \C{23}{14}^2 \\
     + \C{23}{34}^2 \\
     + \C{24}{24}^2
     \end{array} &
     \begin{array}{c}
     - \C{12}{34}\C{13}{13} \\
     - \C{24}{24}\C{34}{12}
     \end{array} &
     0 &
     \begin{array}{c}
     + \C{12}{14}\C{13}{13} \\
     + \C{24}{24}\C{34}{23}
     \end{array} &
     0 &
     0 &
     \begin{array}{c}
     + \C{12}{14}\C{14}{12} \\
     - \C{12}{34}\C{14}{23} \\
     - \C{23}{14}\C{34}{12} \\
     + \C{23}{34}\C{34}{23}
     \end{array} &
     0 &
     \begin{array}{c}
     + \C{12}{14}\C{14}{14} \\
     + \C{12}{34}\C{14}{34} \\
     - \C{23}{14}\C{34}{14} \\
     - \C{23}{34}\C{34}{34} 
     \end{array} \\
     \begin{array}{c}
     - \C{23}{12}\C{12}{12} \\
     - \C{23}{14}\C{12}{14} \\
     + \C{34}{12}\C{14}{12} \\
     + \C{34}{14}\C{14}{14} 
     \end{array} &
     0 &
     \begin{array}{c}
     + \C{23}{12}\C{12}{23} \\
     - \C{23}{14}\C{12}{34} \\
     - \C{34}{12}\C{14}{23} \\
     + \C{34}{14}\C{14}{34}
     \end{array} &
     0 &
     0 &
     \begin{array}{c}
     + \C{13}{13}\C{12}{23} \\
     + \C{34}{14}\C{24}{24}
     \end{array} &
     0 &
     \begin{array}{c}
     - \C{13}{13}\C{12}{34} \\
     - \C{34}{12}\C{24}{24}
     \end{array} &
     \begin{array}{c}
     + \C{13}{13}^2 \\
     + \C{23}{12}^2 \\
     + \C{23}{14}^2 \\
     + \C{34}{12}^2 \\
     + \C{34}{14}^2
     \end{array} &
     0 &
     \begin{array}{c}
     - \C{23}{12}\C{23}{23} \\
     + \C{23}{14}\C{23}{34} \\
     - \C{34}{12}\C{34}{23} \\
     + \C{34}{14}\C{34}{34}
     \end{array} &
     0 &
     0 &
     \begin{array}{c}
     + \C{13}{13}\C{14}{23} \\
     + \C{23}{14}\C{24}{24}
     \end{array} &
     0 &
     \begin{array}{c}
     - \C{13}{13}\C{14}{34} \\
     - \C{23}{12}\C{24}{24} 
     \end{array} \\
     0 &
     \begin{array}{c}
     - \C{23}{12}\C{12}{12} \\
     - \C{23}{23}\C{12}{23} \\
     + \C{34}{12}\C{14}{12} \\
     + \C{34}{23}\C{14}{23}
     \end{array} &
     0 &
     \begin{array}{c}
     - \C{23}{12}\C{12}{14} \\
     + \C{23}{23}\C{12}{34} \\
     + \C{34}{12}\C{14}{14} \\
     - \C{34}{23}\C{14}{34}
     \end{array} &
     \begin{array}{c}
     + \C{13}{24}\C{12}{14} \\
     + \C{34}{23}\C{24}{13}
     \end{array} &
     0 &
     \begin{array}{c}
     + \C{13}{24}\C{12}{34} \\
     + \C{34}{12}\C{24}{13}
     \end{array} &
     0 &
     0 &
     \begin{array}{c}
     + \C{13}{24}^2 \\
     + \C{23}{12}^2 \\
     + \C{23}{23}^2 \\
     + \C{34}{12}^2 \\
     + \C{34}{23}^2
     \end{array} &
     0 &
     \begin{array}{c}
     + \C{23}{12}\C{23}{14} \\
     - \C{23}{23}\C{23}{34} \\
     + \C{34}{12}\C{34}{14} \\
     - \C{34}{23}\C{34}{34}
     \end{array} &
     \begin{array}{c}
     + \C{13}{24}\C{14}{14} \\
     + \C{23}{23}\C{24}{13}
     \end{array} &
     0 &
     \begin{array}{c}
     + \C{13}{24}\C{14}{34} \\
     + \C{23}{12}\C{24}{13}
     \end{array} &
     0 \\
     \begin{array}{c}
     + \C{23}{23}\C{12}{12} \\
     - \C{23}{34}\C{12}{14} \\
     - \C{34}{23}\C{14}{12} \\
     + \C{34}{34}\C{14}{14}
     \end{array} &
     0 &
     \begin{array}{c}
     - \C{23}{23}\C{12}{23} \\
     - \C{23}{34}\C{12}{34} \\
     + \C{34}{23}\C{14}{23} \\
     + \C{34}{34}\C{14}{34}
     \end{array} &
     0 &
     0 &
     \begin{array}{c}
     + \C{13}{13}\C{12}{12} \\
     + \C{34}{34}\C{24}{24}
     \end{array} &
     0 &
     \begin{array}{c}
     + \C{13}{13}\C{12}{14} \\
     + \C{34}{23}\C{24}{24} 
     \end{array} &
     \begin{array}{c}
     - \C{23}{23}\C{23}{12} \\
     + \C{23}{34}\C{23}{14} \\
     - \C{34}{23}\C{34}{12} \\
     + \C{34}{34}\C{34}{14}
     \end{array} &
     0 &
     \begin{array}{c}
     + \C{13}{13}^2 \\
     + \C{23}{23}^2 \\
     + \C{23}{34}^2 \\
     + \C{34}{23}^2 \\
     + \C{34}{34}^2
     \end{array} &
     0 &
     0 &
     \begin{array}{c}
     + \C{13}{13}\C{14}{12} \\
     + \C{23}{34}\C{24}{24}
     \end{array} &
     0 &
     \begin{array}{c}
     + \C{13}{13}\C{14}{14} \\
     + \C{23}{23}\C{24}{24}
     \end{array} \\
     0 &
     \begin{array}{c}
     - \C{23}{14}\C{12}{12} \\
     + \C{23}{34}\C{12}{23} \\
     + \C{34}{14}\C{14}{12} \\
     - \C{34}{34}\C{14}{23} 
     \end{array} &
     0 &
     \begin{array}{c}
     - \C{23}{14}\C{12}{14} \\
     - \C{23}{34}\C{12}{34} \\
     + \C{34}{14}\C{14}{14} \\
     + \C{34}{34}\C{14}{34} 
     \end{array} &
     \begin{array}{c}
     - \C{13}{24}\C{12}{12} \\
     - \C{34}{34}\C{24}{13} 
     \end{array} &
     0 &
     \begin{array}{c}
     + \C{13}{24}\C{12}{23} \\
     + \C{34}{14}\C{24}{13} 
     \end{array} &
     0 &
     0 &
     \begin{array}{c}
     + \C{23}{14}\C{23}{12} \\
     - \C{23}{34}\C{23}{23} \\
     + \C{34}{14}\C{34}{12} \\
     - \C{34}{34}\C{34}{23} 
     \end{array} &
     0 &
     \begin{array}{c}
     + \C{13}{24}^2 \\
     + \C{23}{14}^2 \\
     + \C{23}{34}^2 \\
     + \C{34}{14}^2 \\
     + \C{34}{34}^2 
     \end{array} &
     \begin{array}{c}
     - \C{13}{24}\C{14}{12} \\
     - \C{23}{34}\C{24}{13} 
     \end{array} &
     0 &
     \begin{array}{c}
     + \C{13}{24}\C{14}{23} \\
     + \C{23}{14}\C{24}{13} 
     \end{array} &
     0 \\
     0 &
     \begin{array}{c}
     - \C{24}{13}\C{12}{23} \\
     - \C{34}{14}\C{13}{24} 
     \end{array} &
     0 &
     \begin{array}{c}
     + \C{24}{13}\C{12}{34} \\
     + \C{34}{12}\C{13}{24} 
     \end{array} &
     \begin{array}{c}
     + \C{14}{12}\C{12}{12} \\
     + \C{14}{14}\C{12}{14} \\
     - \C{34}{12}\C{23}{12} \\
     - \C{34}{14}\C{23}{14} 
     \end{array} &
     0 &
     \begin{array}{c}
     - \C{14}{12}\C{12}{23} \\
     + \C{14}{14}\C{12}{34} \\
     + \C{34}{12}\C{23}{23} \\
     - \C{34}{14}\C{23}{34} 
     \end{array} &
     0 &
     0 &
     \begin{array}{c}
     + \C{14}{14}\C{13}{24} \\
     + \C{24}{13}\C{23}{23} 
     \end{array} &
     0 &
     \begin{array}{c}
     - \C{14}{12}\C{13}{24} \\
     - \C{24}{13}\C{23}{34} 
     \end{array} &
     \begin{array}{c}
     + \C{14}{12}^2 \\
     + \C{14}{14}^2 \\
     + \C{24}{13}^2 \\
     + \C{34}{12}^2 \\
     + \C{34}{14}^2 
     \end{array} &
     0 &
     \begin{array}{c}
     - \C{14}{12}\C{14}{23} \\
     + \C{14}{14}\C{14}{34} \\
     - \C{34}{12}\C{34}{23} \\
     + \C{34}{14}\C{34}{34} 
     \end{array} &
     0 \\
     \begin{array}{c}
     - \C{24}{24}\C{12}{14} \\
     - \C{34}{23}\C{13}{13} 
     \end{array} &
     0 &
     \begin{array}{c}
     - \C{24}{24}\C{12}{34} \\
     - \C{34}{12}\C{13}{13} 
     \end{array} &
     0 &
     0 &
     \begin{array}{c}
     + \C{14}{12}\C{12}{12} \\
     + \C{14}{23}\C{12}{23} \\
     - \C{34}{12}\C{23}{12} \\
     - \C{34}{23}\C{23}{23} 
     \end{array} &
     0 &
     \begin{array}{c}
     + \C{14}{12}\C{12}{14} \\
     - \C{14}{23}\C{12}{34} \\
     - \C{34}{12}\C{23}{14} \\
     + \C{34}{23}\C{23}{34} 
     \end{array} &
     \begin{array}{c}
     + \C{14}{23}\C{13}{13} \\
     + \C{24}{24}\C{23}{14} 
     \end{array} &
     0 &
     \begin{array}{c}
     + \C{14}{12}\C{13}{13} \\
     + \C{24}{24}\C{23}{34} 
     \end{array} &
     0 &
     0 &
     \begin{array}{c}
     + \C{14}{12}^2 \\
     + \C{14}{23}^2 \\
     + \C{24}{24}^2 \\
     + \C{34}{12}^2 \\
     + \C{34}{23}^2 
     \end{array} &
     0 &
     \begin{array}{c}
     + \C{14}{12}\C{14}{14} \\
     - \C{14}{23}\C{14}{34} \\
     + \C{34}{12}\C{34}{14} \\
     - \C{34}{23}\C{34}{34} 
     \end{array} \\
     0 &
     \begin{array}{c}
     - \C{24}{13}\C{12}{12} \\
     - \C{34}{34}\C{13}{24} 
     \end{array} &
     0 &
     \begin{array}{c}
     - \C{24}{13}\C{12}{14} \\
     - \C{34}{23}\C{13}{24} 
     \end{array} &
     \begin{array}{c}
     - \C{14}{23}\C{12}{12} \\
     + \C{14}{34}\C{12}{14} \\
     + \C{34}{23}\C{23}{12} \\
     - \C{34}{34}\C{23}{14} 
     \end{array} &
     0 &
     \begin{array}{c}
     + \C{14}{23}\C{12}{23} \\
     + \C{14}{34}\C{12}{34} \\
     - \C{34}{23}\C{23}{23} \\
     - \C{34}{34}\C{23}{34} 
     \end{array} &
     0 &
     0 &
     \begin{array}{c}
     + \C{14}{34}\C{13}{24} \\
     + \C{24}{13}\C{23}{12} 
     \end{array} &
     0 &
     \begin{array}{c}
     + \C{14}{23}\C{13}{24} \\
     + \C{24}{13}\C{23}{14} 
     \end{array} &
     \begin{array}{c}
     - \C{14}{23}\C{14}{12} \\
     + \C{14}{34}\C{14}{14} \\
     - \C{34}{23}\C{34}{12} \\
     + \C{34}{34}\C{34}{14} 
     \end{array} &
     0 &
     \begin{array}{c}
     + \C{14}{23}^2 \\
     + \C{14}{34}^2 \\
     + \C{24}{13}^2 \\
     + \C{34}{23}^2 \\
     + \C{34}{34}^2 
     \end{array} &
     0 \\
     \begin{array}{c}
     + \C{24}{24}\C{12}{12} \\
     + \C{34}{34}\C{13}{13} 
     \end{array} &
     0 &
     \begin{array}{c}
     - \C{24}{24}\C{12}{23} \\
     - \C{34}{14}\C{13}{13} 
     \end{array} &
     0 &
     0 &
     \begin{array}{c}
     + \C{14}{14}\C{12}{12} \\
     - \C{14}{34}\C{12}{23} \\
     - \C{34}{14}\C{23}{12} \\
     + \C{34}{34}\C{23}{23} 
     \end{array} &
     0 &
     \begin{array}{c}
     + \C{14}{14}\C{12}{14} \\
     + \C{14}{34}\C{12}{34} \\
     - \C{34}{14}\C{23}{14} \\
     - \C{34}{34}\C{23}{34} 
     \end{array} &
     \begin{array}{c}
     - \C{14}{34}\C{13}{13} \\
     - \C{24}{24}\C{23}{12} 
     \end{array} &
     0 &
     \begin{array}{c}
     + \C{14}{14}\C{13}{13} \\
     + \C{24}{24}\C{23}{23} 
     \end{array} &
     0 &
     0 &
     \begin{array}{c}
     + \C{14}{14}\C{14}{12} \\
     - \C{14}{34}\C{14}{23} \\
     + \C{34}{14}\C{34}{12} \\
     - \C{34}{34}\C{34}{23} 
     \end{array} &
     0 &
     \begin{array}{c}
     + \C{14}{14}^2 \\
     + \C{14}{34}^2 \\
     + \C{24}{24}^2 \\
     + \C{34}{14}^2 \\
     + \C{34}{34}^2 
     \end{array}  
   \end{array}\right)
\end{eqnarray}
%} %end_scalebox
\normalsize
%\endgroup

\section{Minimizing the total energy}

The correlation functions and the natural orbitals may be optimized by
starting from an energy expression with appropriate Lagrangians and 
minimizing it. An approach similar to that of our previous
paper~\cite{van_Dam_2016} is used. There is a difference because the energy
expression in this work cannot be represented entirely in terms of 1-electron
density matrices as the 2-electron term is orbital dependent. This means
that the trick of multiplying the Lagrangian for the natural orbitals with
the occupation numbers to enable factoring them out cannot be used here.

The problem to consider is minimizing
\begin{eqnarray}
  L &=&
  E(N^\alpha,C^\alpha;N^\beta,C^\beta)
  + \sum_{\sigma=\{\alpha,\beta\}}\sum_{i,j=1}^{n_b}\lambda^{N^\sigma}_{ij}
    \left(I_{ij}-\sum_{a,b=1}^{n_b}N^{\sigma*}_{ai}S_{ab}N^\sigma_{bj}\right)
    \nonumber \\
  &&+ \sum_{\sigma=\{\alpha,\beta\}}\sum_{r,s=1}^{n_b}\lambda^{C^\sigma}_{rs}
      \left(I_{rs}-\sum_{i,j=1}^{n_b}C^{\sigma*}_{ir}I_{ij}C^\sigma_{js}\right) 
\end{eqnarray}
\begin{equation}
  \min_{N^\alpha,C^\alpha,\lambda^{N^\alpha}_{ij},\lambda^{C^\alpha}_{rs}} L
\end{equation}
Before continuing the energy expression $E(N^\alpha,C^\alpha;N^\beta,C^\beta)$
must be clarified. In general the energy expression may be written as
\begin{eqnarray}
  E(N^\alpha,C^\alpha;N^\beta,C^\beta)
  &=& \sum_{\sigma,\sigma'=\{\alpha,\beta\}}
      \left(\sum_{a,b=1}^{n_b}\frac{1}{2}h_{ab}D^\sigma_{ab}
   +  \sum_{a,b,c,d=1}^{n_b}\frac{1}{2}(ab|cd)\Gamma_{abcd}^{\sigma\sigma'}
      \right) 
      \dlabel{Eq:ENC}
\end{eqnarray}
Note that the two-electron terms may be reformulated as
\begin{eqnarray}
   W &=& \sum_{a,b,c,d=1}^{n_b}(ab|cd)\Gamma_{abcd} \\
     &=& \sum_{a,b,c,d=1}^{n_b}(ab|cd)
         \sum_{\sigma,\sigma'=\{\alpha,\beta\}}\Gamma^{\sigma\sigma'}_{abcd}
         \dlabel{Eq:ENC2} \\
     &=& \sum_{\sigma,\sigma'=\{\alpha,\beta\}}W^{\sigma,\sigma'}
\end{eqnarray}
In the subsequent derivations the derivatives w.r.t. the $\alpha$-spin channel
will be given. The equations for the $\beta$-spin channel can be obtained by
interchanging the the $\alpha$ and $\beta$ spin labels.

For the orthogonality of the natural orbitals we obtain from
$\partial L/\partial\lambda^{C\alpha}_{rs} = 0$
\begin{eqnarray}
  \sum_{i,j=1}^{n_b}C^{\alpha*}_{ir}I_{ij}C^\alpha_{js} &=& I_{rs}
\end{eqnarray}
Likewise for the natural orbitals
$\partial L/\partial\lambda^{N\alpha}_{ij} = 0$ gives
\begin{eqnarray}
  \sum_{a,b=1}^{n_b}N^{\alpha*}_{ai}S_{ab}N^\alpha_{bj} &=& I_{ij}
\end{eqnarray}

\subsection{Derivatives with respect to the natural orbital coefficients}

\subsubsection{1-electron terms}

Considering derivatives w.r.t. the natural orbitals there are only two relevant
quantities. Those are the 1-electron energy terms of Eq.~\ref{Eq:ENC} and the
2-electron terms.

Differentiating the 1-electron energy terms we obtain
\begin{eqnarray}
  \frac{\partial}{\partial N^{*}_{ek}}\sum_{a,b=1}^{n_b}h_{ab}D_{ab}
  &=& \frac{\partial}{\partial N^{*}_{ek}}
      \sum_{a,b,i=1}^{n_b}h_{ab}N_{ai}d_iN^*_{bi} \\
  &=& \sum_{a,b,i=1}^{n_b}h_{ab}N_{ai}d_i\delta_{eb}\delta_{ki} \\
  &=& \sum_{a,i=1}^{n_b}h_{ae}N_{ai}d_i\delta_{ki} \\
  &=& \sum_{a=1}^{n_b}h_{ae}N_{ak}d_k
\end{eqnarray}
This derivative may be transformed into the natural orbital basis to give
\begin{eqnarray}
  \sum_{a,e=1}^{n_b}h_{ae}N_{ak}d_k N^*_{el}
  &=& d_k h_{kl} \\
  &=& M^{(1)}_{kl}
      \dlabel{Eq:M1:kl}
\end{eqnarray}
The result is a matrix where the rows are scaled by the occupation numbers
of the natural orbitals.

\subsubsection{2-electron terms}

The derivatives of the $\Gamma^{\sigma\sigma'}_{abcd}$ terms of
Eq.~\ref{Eq:ENC2} with respect to the natural orbitals are similar to the
1-electron terms because they only involve Coulomb interactions and no exchange
interactions. The derivatives of the $\Gamma^{\sigma\sigma}_{abcd}$ terms
are more involved and will be considered here in detail. Considering the
derivatives with respect to the natural orbitals the electron pair occupation
numbers do not have to considered in detail. Hence, those occupation numbers 
are simply represented as $d_{ij}$.

Recall, from Eq.~\ref{Eq:E2:dij:basis} that the basis function coefficients
$a$ and $c$ are associated with electron 1, and the coefficients $b$ and $d$
are associated with electron 2. Choosing that $N_{ek}^*$ is a coefficient of
electron 1,
consider
\begin{eqnarray}
   \frac{\partial}{\partial N^*_{ek}}W^{\sigma\sigma}
   &=& \sum_{a,b,c,d=1}^{n_b}(ab|cd)\frac{\partial}{\partial N^*_{ek}}
       \Gamma_{abcd}^{\sigma\sigma} \\
   &=& \sum_{a,b,c,d=1}^{n_b}(ab|cd)\frac{\partial}{\partial N^*_{ek}}
       \sum_{i,j=1}^{n_b}\left[
       N_{ai}N_{ci}N_{bj}N_{dj} d_{ij}
      +N_{aj}N_{cj}N_{bi}N_{di} d_{ij}\right. \nonumber \\
   && \left.-N_{ai}N_{cj}N_{bj}N_{di} d_{ij}
      -N_{aj}N_{ci}N_{bi}N_{dj} d_{ij}\right] \\
   &=& \sum_{a,b,c,d=1}^{n_b}(ab|cd)
       \sum_{i,j=1}^{n_b}\left[
       N_{ai}\delta_{ec}\delta_{ik}N_{bj}N_{dj} d_{ij}
      +N_{aj}\delta_{ec}\delta_{jk}N_{bi}N_{di} d_{ij}\right. \nonumber \\
   && \left.-N_{ai}\delta_{ec}\delta_{jk}N_{bj}N_{di} d_{ij}
      -N_{aj}\delta_{ec}\delta_{ik}N_{bi}N_{dj} d_{ij}\right] \\
   &=& \sum_{a,b,d=1}^{n_b}(ab|ed)
       \sum_{i,j=1}^{n_b}\left[
       N_{ak}\delta_{ik}N_{bj}N_{dj} d_{kj}
      +N_{ak}\delta_{jk}N_{bi}N_{di} d_{ik}\right. \nonumber \\
   && \left.-N_{ai}\delta_{jk}N_{bk}N_{di} d_{ik}
      -N_{aj}\delta_{ik}N_{bk}N_{dj} d_{kj}\right] \\
   &=& \sum_{a,b,d=1}^{n_b}(ab|ed)
       \sum_{j=1}^{n_b}\left[
       2 N_{ak}N_{bj}N_{dj} d_{kj}
      -2 N_{aj}N_{bk}N_{dj} d_{kj}\right] \\
   &=& \sum_{a,b,d=1}^{n_b}(ab|ed)
       \sum_{j=1}^{n_b}\left[
       2 N_{ak}N_{bj}N_{dj} d_{kj}
      -2 N_{ak}N_{bj}N_{dk} d_{kj}\right]
\end{eqnarray}
This equation can be transformed into the natural orbital basis to generate
\begin{eqnarray}
   M^{(2)}_{kl}
   &=& \sum_{a,b,e,d=1}^{n_b}(ab|ed)
       \sum_{j=1}^{n_b}\left[
       2 N_{ak}N^*_{el}N_{bj}N_{dj} d_{kj}
      -2 N_{ak}N^*_{el}N_{bj}N_{dk} d_{kj}\right]
       \dlabel{Eq:M2:kl}
\end{eqnarray}

\subsubsection{electron correlation terms}

To evaluate the derivatives of the correlation energy with respect to the natural
orbital coefficients the 2-electron integrals in the natural orbital basis have
to be considered. To focus on those integrals, first rewrite the occupation number
based factors as
\begin{eqnarray}
   P_{ii}       &=& P^{(2)}_{ii}-P^{(1)}_{ii}                                 \dlabel{Eq:Pii}  \\
   P^{(1)}_{ii} &=& p^\alpha_i p^\beta_i                                      \dlabel{Eq:P1ii} \\
   P^{(2)}_{ii} &=& \sqrt{p^\alpha_i p^\beta_i}                               \dlabel{Eq:P2ii} \\
   P^{(2)}_{ij} &=& \sqrt{p^\alpha_i(1-p^\alpha_i) p^\beta_j(1-p^\beta_j)}    \dlabel{Eq:P2ij} \\
   P^{(4)}_{ij} &=& \sqrt[4]{p^\alpha_i(1-p^\alpha_i) p^\beta_j(1-p^\beta_j)} \dlabel{Eq:P4ij}
\end{eqnarray}
With the expressions above the correlation energy terms can be rephrased as
\begin{eqnarray}
   T_{ii}       &=& \eria{i}{i}{i}{i}P_{ii}       \\
   T^{(2)}_{ij} &=& \eria{i}{j}{i}{j}P^{(2)}_{ij} \\
   T^{(4)}_{ij} &=& \eria{i}{i}{j}{j}P^{(4)}_{ij}
\end{eqnarray}
Taking the derivatives with respect to the natural orbital coefficient $N^{\alpha*}_{ek}$ we have
\begin{eqnarray}
   \frac{\partial T_{ii}}{\partial N^{\alpha*}_{ek}}
   &=& \sum_c \eria{i}{i}{c}{i}P_{ii}
       \frac{\partial N^{\alpha*}_{ci}}{\partial N^{\alpha*}_{ek}} \\
   &=& \sum_c \eria{i}{i}{c}{i}P_{ii}\delta_{ce}\delta_{ik} \\
   &=& \eria{k}{k}{e}{k}P_{kk}
\end{eqnarray}
Likewise, differentiating $T^{(2)}_{ij}$ gives
\begin{eqnarray}
   \frac{\partial T^{(2)}_{ij}}{\partial N^{\alpha*}_{ek}}
   &=& \sum_c\eria{i}{j}{c}{j}P^{(2)}_{ij}
       \frac{\partial N^{\alpha*}_{ci}}{\partial N^{\alpha*}_{ek}} \\
   &=& \sum_c\eria{i}{j}{c}{j}P^{(2)}_{ij}\delta_{ce}\delta_{ik} \\
   &=& \eria{k}{j}{e}{j}P^{(2)}_{kj}
\end{eqnarray}
Finally, differentiating $T^{(4)}_{ij}$ gives
\begin{eqnarray}
   \frac{\partial T^{(4)}_{ij}}{\partial N^{\alpha*}_{ek}}
   &=& \sum_c\eria{i}{i}{c}{j}P^{(4)}_{ij}
       \frac{\partial N^{\alpha*}_{cj}}{\partial N^{\alpha*}_{ek}} \\
   &=& \sum_c\eria{i}{i}{c}{j}P^{(4)}_{ij}\delta_{ce}\delta_{jk} \\
   &=& \eria{i}{i}{e}{k}P^{(4)}_{ik}
\end{eqnarray}
Note that ultimately we want Fock matrices in the natural orbitals basis
for the natural orbital optimization. For this reason we transform the remaining
atomic orbital index to natural orbital index $l$ as in
\begin{eqnarray}
   M^{(ii)}_{kl}
   &=& \sum_e\eria{k}{k}{e}{k}P_{kk}N^{\alpha*}_{el}
       \dlabel{Eq:Mii:kl} \\
   M^{(2,ij)}_{kl}
   &=& \sum_e\eria{k}{j}{e}{j}P^{(2)}_{kj}N^{\alpha*}_{el}
       \dlabel{Eq:M2ij:kl} \\
   M^{(4,ij)}_{kl}
   &=& \sum_e\eria{i}{i}{e}{k}P^{(4)}_{ik}N^{\alpha*}_{el}
       \dlabel{Eq:M4ij:kl} 
\end{eqnarray}
Note that for the full Fock matrices of Eq.~\ref{Eq:M2ij:kl} and~\ref{Eq:M4ij:kl}
we still need to sum over $j \ne k$ and $i \ne k$ respectively.

\subsubsection{Combining the natural orbital terms}

In the previous section various derivative terms where derived. Here all the
terms are brought together to arrive at the full Fock matrix to generate the
natural orbital rotations from. The only term that has not been discussed in the
previous sections is the 2-electron term stemming from the $\alpha\beta$ 
interaction. This term is in form the same as the 1-electron term from
Eq.~\ref{Eq:M1:kl} except that operator $h_{kl}$ is obtained from contracting
the 2-electron integrals with the 1-electron density matrix of the opposing 
spin
\begin{eqnarray}
   M^{\sigma\sigma'}_{kl}
   &=& \sum_{bd}d^{\sigma}_k(k^{\sigma}b|l^{\sigma}d)D^{\sigma'}_{bd}
       \dlabel{Eq:Mssp:kl}
\end{eqnarray}
Combining the terms of Eqs.~\ref{Eq:M1:kl}, \ref{Eq:M2:kl}, \ref{Eq:Mii:kl},
\ref{Eq:M2ij:kl}, \ref{Eq:M4ij:kl} and~\ref{Eq:Mssp:kl} obtains
\begin{eqnarray}
  M_{kl} 
  &=& M^{(1)}_{kl} + M^{(2)}_{kl} + M^{(ii)}_{kl} +
      \sum_j M^{(2,ij)}_{kl} + \sum_i M^{(4,ij)}_{kl} + M^{\sigma\sigma'}_{kl}
\end{eqnarray}
Note that $M_{kl}$ is not Hermitian but this can be achieved by constructing
the final Fock matrix as
\begin{eqnarray}
  F^{N}_{kl} &=& \frac{1}{2}\left(M_{kl}+(M^\dagger)_{kl}\right)
                 \dlabel{Eq:FN:kl}
\end{eqnarray}
The natural orbitals can be updated with the rotation that is determined
by diagonalizing $F^{N}_{kl}$.

\subsection{Derivatives with respect to the correlation function coefficients}

\subsubsection{1-electron terms}

Next considering the derivatives w.r.t. to the correlation functions there are
two terms or factors to consider. They are the 1-electron energy terms, and the
2-electron terms. Considering the 1-electron energy terms we
have
\begin{eqnarray}
  \frac{\partial}{\partial C^{*}_{kt}}\sum_{a,b=1}^{n_b}h_{ab}D_{ab}
  &=& \frac{\partial}{\partial C^{*}_{kt}}\sum_{a,b,i=1}^{n_b}\sum_{r=1}^{n_e}
      h_{ab}N_{ai}C_{ir}C^*_{ir}N^*_{bi} \\
  &=& \sum_{a,b,i=1}^{n_b}\sum_{r=1}^{n_e}
      h_{ab}N_{ai}C_{ir}N^*_{bi}\delta_{ik}\delta_{rt} \\
  &=& \sum_{a,b=1}^{n_b}
      h_{ab}N_{ak}C_{kt}N^*_{bk}
\end{eqnarray}
This equation can be expressed in terms of a Fock-matrix in the natural orbital 
basis as
\begin{eqnarray}
  F^{C(1)}_{jk}
  &=& \delta_{jk}\sum_{a,b=1}^{n_b}h_{ab}N_{ak}N^*_{bk}\\
  \frac{\partial}{\partial C^{*}_{kt}}\sum_{a,b=1}^{n_b}h_{ab}D_{ab}
  &=& F^{C(1)}_{jk}C_{kt}
\end{eqnarray}
This 1-electron Fock-matrix transformed into the correlation function basis
determines the rotation among the correlation functions. This transformed
Fock-matrix 1-electron term is
\begin{eqnarray}
  F^{C(1)}_{st}
  &=& \sum_{j,k=1}^{n_b} C_{js}F^{C^1}_{jk}C^*_{kt}
      \dlabel{Eq:FC1:st}
\end{eqnarray}

\subsubsection{2-electron terms}

For the 2-electron energy contributions focus on the occupation numbers
and their dependence on the correlation functions. This can be formulated
more compactly by integrating the basis function indeces out.
The resulting expression can them be 
combined with the derivative of the correlation function part. The natural
orbital dependent part can be written as
\begin{eqnarray}
   f_{ij} 
   &= \sum_{a,b,c,d=1}^{n_b}(ab|cd)&\left[
       N_{ai}N_{ci}N_{bj}N_{dj}+N_{aj}N_{cj}N_{bi}N_{di}
       \right. \nonumber \\
   &&  \left.
      -N_{ai}N_{cj}N_{bj}N_{di}-N_{aj}N_{ci}N_{bi}N_{dj}
       \right]
\end{eqnarray}
Note that, similar to the 1-electron case, $f_{ij}$ is also diagonal in the
natural orbital basis. However, $f_{ij}$ is an electron pair quantity and
therefore an $N^2$x$N^2$ diagonal matrix. To compute the contribution
to the correlation function Fock-matrix the quantity needs to be reduced
to a $N$x$N$ matrix. For this the derivative of the electron-pair occupation
numbers are considered, assuming that $C^*_{ku}$ are coefficients of electron 1, 
as the coefficients indexed with $i$.
\begin{eqnarray}
   \frac{\partial}{\partial C^*_{ku}}d_{ij}
   &=& \sum_{s,t=1}^{n_e}
       C_{is}\delta_{ik}\delta_{su}C_{jt}C_{jt}
      -C_{is}\delta_{ik}\delta_{tu}C_{jt}C_{js} \\
   &=& \sum_{s,t=1}^{n_e}
       C_{is}\delta_{ik}\delta_{su}C_{jt}C_{jt}
      -C_{it}\delta_{ik}\delta_{su}C_{js}C_{jt} \\
   &=& \sum_{t=1}^{n_e}
       C_{ku}\delta_{ik}C_{jt}C_{jt}
      -C_{kt}\delta_{ik}C_{ju}C_{jt}
\end{eqnarray}
To compute the 2-electron Fock-matrix terms in the correlation function basis
\begin{eqnarray}
  F^{C(2)}_{uv}
  &=& \sum_{k,i,j=1}^{n_b}f_{ij}C^*_{kv}\frac{\partial}{\partial C^*_{ku}}d_{ij}
      \\
  &=& \sum_{k,j=1}^{n_b}f_{kj}\sum_{t=1}^{n_e}
      C_{ku}C_{kv}C_{jt}C_{jt}
     -C_{kt}C_{kv}C_{ju}C_{jt}
      \dlabel{Eq:FC2:uv}
\end{eqnarray}
where electron 2 has been integrated out by summing over $t$.

\subsubsection{electron correlation terms}

In this section the focus is on differentiating the term of $T$ with respect to the 
correlation function coefficients. This means that the 2-electron integrals in the 
natural orbital basis can be considered as constants in these expressions. Hence,
only the derivatives of $P^{(1)}_{ii}$, $P^{(2)}_{ii}$, $P^{(2)}_{ij}$, and $P^{(4)}_{ij}$
of Eqs.~\ref{Eq:P1ii}, \ref{Eq:P2ii}, \ref{Eq:P2ij}, and~\ref{Eq:P4ij},
need to be considered here.
First of all the occupation number of a particular natural orbital is given by
\begin{eqnarray}
   p_i^\alpha &=& \sum_{s=1}^{n_e}C_{is}^\alpha C_{is}^{*\alpha}
\end{eqnarray}
Differentiating this with respect to $C_{kr}^{*\alpha}$ we get
\begin{eqnarray}
   \frac{\partial p_i^\alpha}{\partial C_{kr}^{*\alpha}}
   &=& \sum_{s=1}^{n_e}C_{is}^\alpha \delta_{ki}\delta_{rs} \\
   &=& \left\{
       \begin{array}{l}
         C_{kr}^\alpha, r \le n_e \\
         0, r > n_e
       \end{array}
       \right.
\end{eqnarray}
For the square root of the occupation number we have
\begin{eqnarray}
   \frac{\partial \sqrt{p_i^\alpha}}{\partial C_{kr}^{*\alpha}}
   &=& \frac{\partial \sqrt{p_i^\alpha}}{\partial p_i^\alpha}
       \frac{\partial p_i^\alpha}{\partial C_{kr}^{*\alpha}} \\
   &=& \frac{1}{2\sqrt{p_i^\alpha}}\frac{\partial p_i^\alpha}{\partial C_{kr}^{*\alpha}}
       \dlabel{Eq:d:p:2}
\end{eqnarray}
For the factors in the $T_{ij}^{(2)}$ terms we have
\begin{eqnarray}
   \frac{\partial \sqrt{p_i^\alpha(1-p_i^\alpha)}}{\partial C_{kr}^{*\alpha}}
   &=& \frac{\partial \sqrt{p_i^\alpha(1-p_i^\alpha)}}{\partial p_i^\alpha}
       \frac{\partial p_i^\alpha}{\partial C_{kr}^{*\alpha}} \\
   &=& \frac{(1-p_i^\alpha)-p_i^\alpha}{2\sqrt{p_i^\alpha(1-p_i^\alpha)}}
       \frac{\partial p_i^\alpha}{\partial C_{kr}^{*\alpha}} \\
   &=& \frac{1-2p_i^\alpha}{2\sqrt{p_i^\alpha(1-p_i^\alpha)}}
       \frac{\partial p_i^\alpha}{\partial C_{kr}^{*\alpha}}
       \dlabel{Eq:d:p1p:2}
\end{eqnarray}
For the factors in the $T_{ij}^{(4)}$ terms we have
\begin{eqnarray}
   \frac{\partial \sqrt[4]{p_i^\alpha(1-p_i^\alpha)}}{\partial C_{kr}^{*\alpha}}
   &=& \frac{\partial \sqrt[4]{p_i^\alpha(1-p_i^\alpha)}}{\partial p_i^\alpha}
       \frac{\partial p_i^\alpha}{\partial C_{kr}^{*\alpha}} \\
   &=& \frac{(1-p_i^\alpha)-p_i^\alpha}{4\sqrt[4]{p_i^\alpha(1-p_i^\alpha)}^3}
       \frac{\partial p_i^\alpha}{\partial C_{kr}^{*\alpha}} \\
   &=& \frac{1-2p_i^\alpha}{4\sqrt[4]{p_i^\alpha(1-p_i^\alpha)}^3}
       \frac{\partial p_i^\alpha}{\partial C_{kr}^{*\alpha}}
       \dlabel{Eq:d:p1p:4}
\end{eqnarray}
Combining these derivatives with the 2-electron integral parts obtains
for the $T_{ii}$ terms
\begin{eqnarray}
   \frac{\partial T_{ii}}{\partial C^{*\alpha}_{kr}}
   &=& \eria{i}{i}{i}{i}\left( 
           \frac{\partial P^{(2)}_{ii}}{\partial C^{*\alpha}_{kr}}
          -\frac{\partial P^{(1)}_{ii}}{\partial C^{*\alpha}_{kr}}
       \right) \\
   &=& \eria{i}{i}{i}{i}\left(
           \frac{1}{2}\sqrt{\frac{p^\beta_i}{p^\alpha_i}}
          -p^\beta_i
       \right) C^\alpha_{kr}\delta_{ki}
\end{eqnarray}
Summing over all $i$ and transforming into the correlation function
basis leads to
\begin{eqnarray}
   F^{C(ii)}_{rs}
   &=& \sum_{k}\eria{i}{i}{i}{i}\left(
           \frac{1}{2}\sqrt{\frac{p^\beta_i}{p^\alpha_i}}
          -p^\beta_i
       \right)C^\alpha_{kr}C^{*\alpha}_{ks}
       \dlabel{Eq:FC:ii}
\end{eqnarray}
For the $T^{(2)}_{ij}$ terms this obtains
\begin{eqnarray}
   \frac{\partial T^{(2)}_{ij}}{\partial C^{*\alpha}_{kr}}
   &=& \eria{i}{j}{i}{j}
       \frac{\sqrt{p_i^\alpha(1-p_i^\alpha)}}{\partial C^{*\alpha}_{kr}}
       \sqrt{p_j^\beta(1-p_j^\beta)} \\
   &=& \eria{i}{j}{i}{j}
       \frac{1-2p_i^\alpha}{2\sqrt{p_i^\alpha(1-p_i^\alpha)}}
       \sqrt{p_j^\beta(1-p_j^\beta)} 
       C^\alpha_{kr}\delta_{ki}
\end{eqnarray}
Summing over all $i$ and $j$ and transforming into the correlation
function basis leads to
\begin{eqnarray}
   F^{C(2,ij)}_{rs}
   &=& \sum_{kj}\eria{k}{j}{k}{j}
       \frac{1-2p_k^\alpha}{2\sqrt{p_k^\alpha(1-p_k^\alpha)}}
       \sqrt{p_j^\beta(1-p_j^\beta)} 
       C^\alpha_{kr}C^{*\alpha}_{ks}
       \dlabel{Eq:FC2:ij}
\end{eqnarray}
For the $T^{(4)}_{ij}$ terms this obtains
\begin{eqnarray}
   \frac{\partial T^{(4)}_{ij}}{\partial C^{*\alpha}_{kr}}
   &=& \eria{i}{i}{j}{j}
       \frac{\sqrt[4]{p_i^\alpha(1-p_i^\alpha)}}{\partial C^{*\alpha}_{kr}}
       \sqrt[4]{p_j^\beta(1-p_j^\beta)} \\
   &=& \eria{i}{i}{j}{j}
       \frac{1-2p_i^\alpha}{4\sqrt[4]{p_i^\alpha(1-p_i^\alpha)}^3}
       \sqrt[4]{p_j^\beta(1-p_j^\beta)} 
       C^\alpha_{kr}\delta_{ki}
\end{eqnarray}
Summing over all $i$ and $j$ and transforming into the correlation
function basis leads to
\begin{eqnarray}
   F^{C(4,ij)}_{rs}
   &=& \sum_{kj}\eria{k}{k}{j}{j}
       \frac{1-2p_k^\alpha}{4\sqrt[4]{p_k^\alpha(1-p_k^\alpha)}^3}
       \sqrt[4]{p_j^\beta(1-p_j^\beta)}
       C^\alpha_{kr}C^{*\alpha}_{ks}
       \dlabel{Eq:FC4:ij}
\end{eqnarray}


\subsubsection{Combining the correlation function terms}

In the previous subsections the various correlation function derivatives
were discussed, except the 2-electron term stemming from the $\alpha\beta$
interaction. The form of that term is similar to that of the 1-electron
interaction $h_{rs}$ except that it is formed by contracting the
2-electron integrals with the density matrix of the opposing spin
\begin{eqnarray}
   F^{C\sigma\sigma'}_{rs}
   &=& \sum_{k}\sum_{bd}\eria{k}{b}{k}{d}D^{\sigma'}_{bd}
       C^\sigma_{kr}C^{*\sigma}_{ks}
       \dlabel{Eq:FCssp}
\end{eqnarray}
Combining Eqs.~\ref{Eq:FC1:st}, \ref{Eq:FC2:uv}, \ref{Eq:FC:ii},
\ref{Eq:FC2:ij}, \ref{Eq:FC4:ij}, and~\ref{Eq:FCssp} gives
\begin{eqnarray}
   F^{C}_{uv}
   &=& F^{C(1)}_{uv} + F^{C(2)}_{uv} + F^{C\sigma\sigma'}_{uv} +
       F^{C(ii)}_{uv} + F^{C(2,ij)}_{uv} + F^{C(4,ij)}_{uv}
       \dlabel{Eq:FC:uv}
\end{eqnarray}
The correlation functions can be updated with the rotation obtained
from diagonalizing $F^{C}_{uv}$.

\section{Note on renormalization}

The $T^{(2)}_{ij}$ and $T^{(4)}_{ij}$ terms occur proportional to $N(N-1)$ whereas
the $T_{ii}$ terms are proportional to $N$. This imbalance causes the correlation
effects to be over estimated for larger basis sets (and larger numbers of electrons).
To correct for this the offending terms need to be scaled by $(N-1)_\mathrm{effective}$.
We propose to approximate
\begin{eqnarray}
   \frac{1}{(N-1)_\mathrm{effective}}
   &\approx&
   \frac{\sum_{i}\sqrt{p^\alpha_i p^\beta_i}-p^\alpha_i p^\beta_i}
        {\sum_{i,j\ne i}\sqrt{p^\alpha_i(1-p^\alpha_i)p^\beta_j(1-p^\beta_j)}}
   \dlabel{Eq:1Nm1:eff}
\end{eqnarray}
With this approximation the new $T$ terms $T'^{(2)}_{ij}$ and $T'^{(4)}_{ij}$
become
\begin{eqnarray}
   T'^{(2)}_{ij} &=& 
   T^{(2)}_{ij}
   \frac{\sum_{m}\sqrt{p^\alpha_m p^\beta_m}-p^\alpha_m p^\beta_m}
        {\sum_{m,n\ne m}\sqrt{p^\alpha_m(1-p^\alpha_m)p^\beta_n(1-p^\beta_n)}}
   \dlabel{Eq:Tp2:ij} \\
   T'^{(4)}_{ij} &=& 
   T^{(4)}_{ij}
   \frac{\sum_{m}\sqrt{p^\alpha_m p^\beta_m}-p^\alpha_m p^\beta_m}
        {\sum_{m,n\ne m}\sqrt{p^\alpha_m(1-p^\alpha_m)p^\beta_n(1-p^\beta_n)}}
   \dlabel{Eq:Tp4:ij}
\end{eqnarray}
With respect to the Fock matrices we get
\begin{eqnarray}
   \frac{\partial T'}{\partial C^{*\alpha}_{kr}}
   &=& \frac{\partial T}{\partial C^{*\alpha}_{kr}}
       \frac{\sum_{m}\sqrt{p^\alpha_m p^\beta_m}-p^\alpha_m p^\beta_m}
            {\sum_{m,n\ne m}\sqrt{p^\alpha_m(1-p^\alpha_m)p^\beta_n(1-p^\beta_n)}}
       \nonumber \\
   &+& T
       \frac{\sum_{m}\delta_{km}\left(\frac{1}{2}
             \sqrt{\frac{p^\beta_m}{p^\alpha_m}}-p^\beta_m\right)
             \frac{\partial p^\alpha_m}{\partial C^{*\alpha}_{kr}}}
            {\sum_{m,n\ne m}\sqrt{p^\alpha_m(1-p^\alpha_m)p^\beta_n(1-p^\beta_n)}}
       \nonumber \\
   &-& T
       \frac{\sum_{m}\sqrt{p^\alpha_m p^\beta_m}-p^\alpha_m p^\beta_m}
            {\left(\sum_{m,n\ne m}
                   \sqrt{p^\alpha_m(1-p^\alpha_m)p^\beta_n(1-p^\beta_n)}\right)^2}
       \frac{\partial\sum_{m,n\ne m}\sqrt{p^\alpha_m(1-p^\alpha_m)p^\beta_n(1-p^\beta_n)}}
            {\partial C^{*\alpha}_{kr}}
       \\
   &=& \frac{\partial T}{\partial C^{*\alpha}_{kr}}
       \frac{\sum_{m}\sqrt{p^\alpha_m p^\beta_m}-p^\alpha_m p^\beta_m}
            {\sum_{m,n\ne m}\sqrt{p^\alpha_m(1-p^\alpha_m)p^\beta_n(1-p^\beta_n)}}
       \nonumber \\
   &+& T
       \frac{\sum_{m}\delta_{km}\left(\frac{1}{2}
             \sqrt{\frac{p^\beta_m}{p^\alpha_m}}-p^\beta_m\right)
             \frac{\partial p^\alpha_m}{\partial C^{*\alpha}_{kr}}}
            {\sum_{m,n\ne m}\sqrt{p^\alpha_m(1-p^\alpha_m)p^\beta_n(1-p^\beta_n)}}
       \nonumber \\
   &-& T
       \frac{\sum_{m}\sqrt{p^\alpha_m p^\beta_m}-p^\alpha_m p^\beta_m}
            {\left(\sum_{m,n\ne m}
                   \sqrt{p^\alpha_m(1-p^\alpha_m)p^\beta_n(1-p^\beta_n)}\right)^2}
       \sum_{m,n\ne m}\delta_{mk}\frac{1}{2}
       \sqrt{\frac{p^\beta_n(1-p^\beta_n)}{p^\alpha_m(1-p^\alpha_m)}}
       \frac{\partial p^\alpha_m(1-p^\alpha_m)}
            {\partial C^{*\alpha}_{kr}}
       \\
   &=& \frac{\partial T}{\partial C^{*\alpha}_{kr}}
       \frac{\sum_{m}\sqrt{p^\alpha_m p^\beta_m}-p^\alpha_m p^\beta_m}
            {\sum_{m,n\ne m}\sqrt{p^\alpha_m(1-p^\alpha_m)p^\beta_n(1-p^\beta_n)}}
       \nonumber \\
   &+& T
       \frac{\sum_{m}\delta_{km}\left(\frac{1}{2}
             \sqrt{\frac{p^\beta_m}{p^\alpha_m}}-p^\beta_m\right)
             \frac{\partial p^\alpha_m}{\partial C^{*\alpha}_{kr}}}
            {\sum_{m,n\ne m}\sqrt{p^\alpha_m(1-p^\alpha_m)p^\beta_n(1-p^\beta_n)}}
       \nonumber \\
   &-& T
       \frac{\sum_{m}\sqrt{p^\alpha_m p^\beta_m}-p^\alpha_m p^\beta_m}
            {\left(\sum_{m,n\ne m}
                   \sqrt{p^\alpha_m(1-p^\alpha_m)p^\beta_n(1-p^\beta_n)}\right)^2}
       \sum_{m,n\ne m}\delta_{mk}\frac{1}{2}
       \sqrt{\frac{p^\beta_n(1-p^\beta_n)}{p^\alpha_m(1-p^\alpha_m)}}
       (1-2p^\alpha_m)\frac{\partial p^\alpha_m}
                           {\partial C^{*\alpha}_{kr}}
       \\
\end{eqnarray}

\section{Note on renormalization II}

The $T^{(2)}_{ij}$ and $T^{(4)}_{ij}$ terms occur proportional to $N(N-1)$ whereas
the $T_{ii}$ terms are proportional to $N$. This imbalance causes the correlation
effects to be over estimated for larger basis sets (and larger numbers of electrons).
To correct for this the offending terms need to be scaled, and the scale factor must
be $1$ for $N=2$. In the previous section we proposed a scale factor that is 
approximately $1/(N-1)$ in Eq.~\ref{Eq:1Nm1:eff}. However, when implemented the
correlation energy turned out to be too small. Alternatively we can write the
scale factor as
\begin{eqnarray}
   \left(\frac{2}{N}\right)_\mathrm{effective}
   &\approx&
   \frac{2\sum_{i}\sqrt{p^\alpha_i p^\beta_i}-p^\alpha_i p^\beta_i}
        { \sum_{i}\sqrt{p^\alpha_i p^\beta_i}-p^\alpha_i p^\beta_i
         +\sum_{i,j\ne i}\sqrt{p^\alpha_i(1-p^\alpha_i)p^\beta_j(1-p^\beta_j)}}
  \dlabel{Eq:2N:eff}
\end{eqnarray}
For large $N$ this expression approximates $\frac{2}{N}$ whereas Eq.~\ref{Eq:1Nm1:eff}
approximates $\frac{1}{N}$.

With this approximation the new $T$ terms $T'^{(2)}_{ij}$ and $T'^{(4)}_{ij}$
become
\begin{eqnarray}
   T'^{(2)}_{ij} &=& 
   T^{(2)}_{ij}
   \frac{2\sum_{m}\sqrt{p^\alpha_m p^\beta_m}-p^\alpha_m p^\beta_m}
        { \sum_{m}\sqrt{p^\alpha_m p^\beta_m}-p^\alpha_m p^\beta_m
         +\sum_{m,n\ne m}\sqrt{p^\alpha_m(1-p^\alpha_m)p^\beta_n(1-p^\beta_n)}}
   \dlabel{Eq:Tp2:ij:2} \\
   T'^{(4)}_{ij} &=& 
   T^{(4)}_{ij}
   \frac{2\sum_{m}\sqrt{p^\alpha_m p^\beta_m}-p^\alpha_m p^\beta_m}
        { \sum_{m}\sqrt{p^\alpha_m p^\beta_m}-p^\alpha_m p^\beta_m
         +\sum_{m,n\ne m}\sqrt{p^\alpha_m(1-p^\alpha_m)p^\beta_n(1-p^\beta_n)}}
   \dlabel{Eq:Tp4:ij:2}
\end{eqnarray}
With respect to the Fock matrices we get
\begin{eqnarray}
   \frac{\partial T'}{\partial C^{*\alpha}_{kr}}
   &=& \frac{\partial T}{\partial C^{*\alpha}_{kr}}
       \frac{2\sum_{m}\sqrt{p^\alpha_m p^\beta_m}-p^\alpha_m p^\beta_m}
            { \sum_{m}\left[\sqrt{p^\alpha_m p^\beta_m}-p^\alpha_m p^\beta_m\right]
             +\sum_{m,n\ne m}\sqrt{p^\alpha_m(1-p^\alpha_m)p^\beta_n(1-p^\beta_n)}}
       \nonumber \\
   &+& T
       \frac{2\sum_{m}\delta_{km}\left(\frac{1}{2}
             \sqrt{\frac{p^\beta_m}{p^\alpha_m}}-p^\beta_m\right)
             \frac{\partial p^\alpha_m}{\partial C^{*\alpha}_{kr}}}
            { \sum_{m}\left[\sqrt{p^\alpha_m p^\beta_m}-p^\alpha_m p^\beta_m\right]
             +\sum_{m,n\ne m}\sqrt{p^\alpha_m(1-p^\alpha_m)p^\beta_n(1-p^\beta_n)}}
       \nonumber \\
   &-& T
       \frac{2\sum_{m}\sqrt{p^\alpha_m p^\beta_m}-p^\alpha_m p^\beta_m}
            {\left(
                 \sum_{m}\left[\sqrt{p^\alpha_m p^\beta_m}-p^\alpha_m p^\beta_m\right]
                +\sum_{m,n\ne m}\sqrt{p^\alpha_m(1-p^\alpha_m)p^\beta_n(1-p^\beta_n)}
             \right)^2} \nonumber \\
   &&  \cdot\frac{\partial\sum_{m}
            \left[\sqrt{p^\alpha_m p^\beta_m}-p^\alpha_m p^\beta_m\right]
             +\sum_{m,n\ne m}\sqrt{p^\alpha_m(1-p^\alpha_m)p^\beta_n(1-p^\beta_n)}}
            {\partial C^{*\alpha}_{kr}}
       \\
   &=& \frac{\partial T}{\partial C^{*\alpha}_{kr}}
       \frac{2\sum_{m}\sqrt{p^\alpha_m p^\beta_m}-p^\alpha_m p^\beta_m}
            { \sum_{m}\left[\sqrt{p^\alpha_m p^\beta_m}-p^\alpha_m p^\beta_m\right]
             +\sum_{m,n\ne m}\sqrt{p^\alpha_m(1-p^\alpha_m)p^\beta_n(1-p^\beta_n)}}
       \nonumber \\
   &+& T
       \frac{2\sum_{m}\delta_{km}\left(\frac{1}{2}
             \sqrt{\frac{p^\beta_m}{p^\alpha_m}}-p^\beta_m\right)
             \frac{\partial p^\alpha_m}{\partial C^{*\alpha}_{kr}}}
            { \sum_{m}\left[\sqrt{p^\alpha_m p^\beta_m}-p^\alpha_m p^\beta_m\right]
             +\sum_{m,n\ne m}\sqrt{p^\alpha_m(1-p^\alpha_m)p^\beta_n(1-p^\beta_n)}}
       \nonumber \\
   &-& T
       \frac{2\sum_{m}\sqrt{p^\alpha_m p^\beta_m}-p^\alpha_m p^\beta_m}
            {\left(
                 \sum_{m}\left[\sqrt{p^\alpha_m p^\beta_m}-p^\alpha_m p^\beta_m\right]
                +\sum_{m,n\ne m}\sqrt{p^\alpha_m(1-p^\alpha_m)p^\beta_n(1-p^\beta_n)}
             \right)^2} \nonumber \\
   &&  \cdot\left(
       \sum_{m}\delta_{km}\left[\frac{1}{2}
       \sqrt{\frac{p^\beta_m}{p^\alpha_m}}-p^\beta_m\right]
       \frac{\partial p^\alpha_m}{\partial C^{*\alpha}_{kr}}
       +\sum_{m,n\ne m}\delta_{mk}\frac{1}{2}
       \sqrt{\frac{p^\beta_n(1-p^\beta_n)}{p^\alpha_m(1-p^\alpha_m)}}
       \frac{\partial p^\alpha_m(1-p^\alpha_m)}
            {\partial C^{*\alpha}_{kr}}
       \right) \\
   &=& \frac{\partial T}{\partial C^{*\alpha}_{kr}}
       \frac{2\sum_{m}\sqrt{p^\alpha_m p^\beta_m}-p^\alpha_m p^\beta_m}
            { \sum_{m}\left[\sqrt{p^\alpha_m p^\beta_m}-p^\alpha_m p^\beta_m\right]
             +\sum_{m,n\ne m}\sqrt{p^\alpha_m(1-p^\alpha_m)p^\beta_n(1-p^\beta_n)}}
       \nonumber \\
   &+& T
       \frac{2\sum_{m}\delta_{km}\left(\frac{1}{2}
             \sqrt{\frac{p^\beta_m}{p^\alpha_m}}-p^\beta_m\right)
             \frac{\partial p^\alpha_m}{\partial C^{*\alpha}_{kr}}}
            { \sum_{m}\left[\sqrt{p^\alpha_m p^\beta_m}-p^\alpha_m p^\beta_m\right]
             +\sum_{m,n\ne m}\sqrt{p^\alpha_m(1-p^\alpha_m)p^\beta_n(1-p^\beta_n)}}
       \nonumber \\
   &-& T
       \frac{2\sum_{m}\sqrt{p^\alpha_m p^\beta_m}-p^\alpha_m p^\beta_m}
            {\left(
                 \sum_{m}\left[\sqrt{p^\alpha_m p^\beta_m}-p^\alpha_m p^\beta_m\right]
                +\sum_{m,n\ne m}\sqrt{p^\alpha_m(1-p^\alpha_m)p^\beta_n(1-p^\beta_n)}
             \right)^2} \nonumber \\
   &&  \cdot\left(
       \sum_{m}\delta_{km}\left[\frac{1}{2}
       \sqrt{\frac{p^\beta_m}{p^\alpha_m}}-p^\beta_m\right]
       \frac{\partial p^\alpha_m}{\partial C^{*\alpha}_{kr}}
       +\sum_{m,n\ne m}\delta_{mk}\frac{1}{2}
       \sqrt{\frac{p^\beta_n(1-p^\beta_n)}{p^\alpha_m(1-p^\alpha_m)}}
       (1-2p^\alpha_m)
       \frac{\partial p^\alpha_m}{\partial C^{*\alpha}_{kr}}
       \right)
       \\
\end{eqnarray}

\section{Note on renormalization III}

The $T^{(2)}_{ij}$ and $T^{(4)}_{ij}$ terms occur proportional to $N(N-1)$ whereas
the $T_{ii}$ terms are proportional to $N$. This imbalance causes the correlation
effects to be over estimated for larger basis sets (and larger numbers of electrons).
To correct for this the offending terms need to be scaled, and the scale factor must
be $1$ for $N=2$. In the previous section we proposed a scale factor that is 
approximately $1/(N-1)$ in Eq.~\ref{Eq:1Nm1:eff}. However, when implemented the
correlation energy turned out to be too small. Alternatively we can write the
scale factor in a way that is approximately $2/N$ as in Eq.~\ref{Eq:2N:eff}.
However, the latter scale factor leads to correlation effects that are still
too large. However, as the overal behavior seems right in that both rescaling 
factors lead to a factor $1$ for $N=2$ we can define yet another rescaling
factor by interpolating between the two as in
\begin{eqnarray}
   R &=& (1-a)\frac{1}{N-1} + a\frac{2}{N}
\end{eqnarray}
To rederive the expressions, in particular for the derivatives, let's first re-express
these equations by defining some common terms. We introduce the terms $B_{ii}$ and
$B_{ij}$ as
\begin{eqnarray}
  B_{ii}
  &=& \sum_m \sqrt{p^\alpha_m p^\beta_m} - p^\alpha_m p^\beta_m 
      \dlabel{Eq:Bii} \\
  B_{ij}
  &=& \sum_{m,n \ne m} \sqrt{p^\alpha_m(1-p^\alpha_m)p^\beta_n(1-p^\beta_n)}
      \dlabel{Eq:Bij}
\end{eqnarray}
Next we write Eq.~\ref{Eq:1Nm1:eff} and~\ref{Eq:2N:eff} as
\begin{eqnarray}
  K_1 &=& \frac{B_{ii}}{B_{ij}}
          \dlabel{Eq:K1} \\
  K_2 &=& \frac{2B_{ii}}{B_{ii}+B_{ij}} 
          \dlabel{Eq:K2}
\end{eqnarray}
With these factors the rescaled $T$s become
\begin{eqnarray}
  T' &=& T\left((1-a) K_1 + a K_2\right) 
\end{eqnarray}
For the derivatives this gives
\begin{eqnarray}
  \frac{\partial T'}{\partial C^{*\alpha}_{kr}}
  &=& \frac{\partial T}{\partial C^{*\alpha}_{kr}}\left(a K_1 + (1-a) K_2\right) 
   +  T\left(a    \frac{\partial K_1}{\partial C^{*\alpha}_{kr}}
            +(1-a)\frac{\partial K_2}{\partial C^{*\alpha}_{kr}}\right)
\end{eqnarray}
The derivatives of the $K$ factors are
\begin{eqnarray}
  \frac{\partial K_1}{\partial C^{*\alpha}_{kr}}
  &=& \frac{1}{B_{ij}}\frac{\partial B_{ii}}{\partial C^{*\alpha}_{kr}}
   -  \frac{B_{ii}}{B_{ij}^2}\frac{\partial B_{ij}}{\partial C^{*\alpha}_{kr}} \\
  \frac{\partial K_2}{\partial C^{*\alpha}_{kr}}
  &=& \frac{2}{B_{ii}+B_{ij}}\frac{\partial B_{ii}}{\partial C^{*\alpha}_{kr}}
   -  \frac{2B_{ii}}{(B_{ii}+B_{ij})^2}\left(
         \frac{\partial B_{ii}}{\partial C^{*\alpha}_{kr}}
       + \frac{\partial B_{ij}}{\partial C^{*\alpha}_{kr}}
      \right) 
\end{eqnarray}
Differentiating the $B$ factors we have
\begin{eqnarray}
  \frac{\partial B_{ii}}{\partial C^{*\alpha}_{kr}}
  &=& \sum_m \delta_{km}\left(\frac{1}{2}\sqrt{\frac{p_m^\beta}{p_m^\alpha}}-p_m^\beta\right)
      \frac{\partial p_m^\alpha}{\partial C^{*\alpha}_{kr}} \\
  \frac{\partial B_{ij}}{\partial C^{*\alpha}_{kr}}
  &=& \sum_{m,n \ne m}\delta_{mk}\frac{1}{2}
      \sqrt{\frac{p_n^\beta(1-p_n^\beta)}{p_m^\alpha(1-p_m^\alpha)}}\left(1-2p_m^\alpha\right)
      \frac{\partial p_m^\alpha}{\partial C^{*\alpha}_{kr}}
\end{eqnarray}

Tests with this renormalization strategy are highlighting another problem. I have implemented
a Monte-Carlo optimizer (to test the gradients), and this showed different results in
different runs. Table~\ref{Table:MCRenorm} shows the results for 10 different runs. Each run
performed 1000 accepted Monte-Carlo steps, printed the results, and then restart from the
results and did another 1000 accepted Monte-Carlo steps. The second set of Monte-Carlo steps
are essentially a test on the convergence on the first set of Monte-Carlo step. The results in the
table show the calculation number, the energy after 1000 steps, after 2000 steps, and the final
occupation numbers.

\begin{table}
\begin{tabular}{|c|c|c|cccccccccc|}
\hline\hline 
ID & E$_{1000}$ & E$_{2000}$ & \multicolumn{10}{c}{occupation numbers} \\
\hline 
0 & -1.1617670620 & -1.1621052342 & 0.984420 & 0.000004 & 0.015296 & 0.000156 & 0.000031 & 0.000029 & 0.000031 & 0.000009 & 0.000009 & 0.000014 \\
1 & -1.1474268342 & -1.1474681905 & 0.996056 & 0.000002 & 0.000003 & 0.000013 & 0.003886 & 0.000007 & 0.000006 & 0.000002 & 0.000023 & 0.000002 \\
2 & -1.1618764561 & -1.1622929633 & 0.984318 & 0.000003 & 0.015390 & 0.000167 & 0.000030 & 0.000030 & 0.000029 & 0.000009 & 0.000010 & 0.000013 \\
3 & -1.1908897338 & -1.1909608606 & 0.955809 & 0.043794 & 0.000027 & 0.000036 & 0.000062 & 0.000062 & 0.000101 & 0.000039 & 0.000039 & 0.000032 \\
4 & -1.1909377927 & -1.1909959527 & 0.955701 & 0.043900 & 0.000026 & 0.000038 & 0.000062 & 0.000062 & 0.000103 & 0.000039 & 0.000038 & 0.000030 \\
5 & -1.1908782018 & -1.1909677076 & 0.955680 & 0.043920 & 0.000028 & 0.000039 & 0.000062 & 0.000063 & 0.000100 & 0.000039 & 0.000039 & 0.000030 \\
6 & -1.1908838730 & -1.1909667200 & 0.955842 & 0.043761 & 0.000026 & 0.000033 & 0.000062 & 0.000062 & 0.000103 & 0.000038 & 0.000039 & 0.000033 \\
7 & -1.1908938420 & -1.1909591106 & 0.955802 & 0.043801 & 0.000028 & 0.000036 & 0.000062 & 0.000062 & 0.000100 & 0.000039 & 0.000039 & 0.000032 \\
8 & -1.1586775499 & -1.1624595808 & 0.989206 & 0.000000 & 0.000012 & 0.010686 & 0.000013 & 0.000016 & 0.000035 & 0.000009 & 0.000010 & 0.000014 \\
9 & -1.1619564634 & -1.1622760992 & 0.984356 & 0.000004 & 0.015355 & 0.000164 & 0.000030 & 0.000030 & 0.000030 & 0.000009 & 0.000009 & 0.000013 \\
\hline\hline
\end{tabular}
\dlabel{Table:MCRenorm}
\caption{The results of 10 Monte-Carlo optimizations with $a=0$, showing the total energy after 1000 and 2000 
accepted Monte-Carlo steps, followed by the orbital occupation numbers of \ce{H2} in a 6-31G** basis
set. The Hartree-Fock energy is -1.131329765611, the Full-CI energy is -1.149414557325.}
\end{table}

The results clearly show that the renormalized energy expression is non-convex, it has multiple minima for
different approximations of the ground state. In fact, looking at the occupation numbers there is a highly
occupied orbital, a medium occupied orbital, and several low occupied orbitals. The different results are 
associated with different virtual orbitals having a medium occupation number. Hence, it is likely that
for every different virtual orbital there is a different minimum of the energy expression. On top of that
most renormalization factors over estimate the correlation energy, most by 0.04 Hartrees or more
than 24 kcal/mol. 

In conclusion this renormalization approach is useless (even at $a=0$ let alone $a>0$).

\section{Note on renormalization IV}

Given the total and utter failure of more advanced renormalization methods it would seem
that a much simpler approach might work better. The simplest approach to address the imbalance
between the $ii$ and the $ij$ terms is to address the difference in their basic scaling.
The $ii$ terms scale as $O(N)$ whereas the $ij$ terms scale as $O\left(N(N-1)\right)$. 
The simplest way to address this imbalance is to scale the $ij$ terms down by a factor
$\frac{1}{N-1}$. A problem with this approach is that if you create big basis sets not
every basis function might actually contribute. It is easy to add basis functions that 
do not do anything. Counting those spurious basis functions into the scale factor might
just bias the calculation.

However, as an attempt to incorporate the relevance of a basis function into the renormalization
failed badly this simple approach might have some advantages. Note that if orbitals don't 
actually get any significant occupation they don't contribute in any significant way to the
$ij$ part, however, they also don't contribute to the $ii$ terms either. So, maybe this
naive approach does actually work.

The results are given in Table~\ref{Table:MCRenorm:1overNm1} and clearly show that with
a constant scale factor there is a single solution for this problem. However, as the 
Hartree-Fock energy is $-1.131329765611$ and the Full-CI energy is $-1.149414557325$
the correlation energy is underestimated.

\begin{table}
\begin{tabular}{|c|c|c|cccccccccc|}
\hline\hline 
ID & E$_{1000}$ & E$_{2000}$ & \multicolumn{10}{c}{occupation numbers} \\
\hline 
0 & -1.1357532484 & -1.1358047293 & 0.998564 & 0.000556 & 0.000232 & 0.000210 & 0.000104 & 0.000104 & 0.000079 & 0.000058 & 0.000058 & 0.000034 \\
1 & -1.1357285553 & -1.1357840156 & 0.998558 & 0.000549 & 0.000236 & 0.000214 & 0.000104 & 0.000105 & 0.000081 & 0.000059 & 0.000059 & 0.000034 \\
2 & -1.1357689069 & -1.1358200801 & 0.998553 & 0.000566 & 0.000231 & 0.000211 & 0.000105 & 0.000103 & 0.000080 & 0.000058 & 0.000060 & 0.000033 \\
3 & -1.1357393732 & -1.1357940625 & 0.998573 & 0.000543 & 0.000233 & 0.000213 & 0.000103 & 0.000104 & 0.000081 & 0.000058 & 0.000059 & 0.000034 \\
4 & -1.1357691192 & -1.1358189473 & 0.998547 & 0.000565 & 0.000235 & 0.000213 & 0.000104 & 0.000105 & 0.000080 & 0.000059 & 0.000059 & 0.000034 \\
5 & -1.1357338144 & -1.1357867772 & 0.998567 & 0.000547 & 0.000231 & 0.000213 & 0.000105 & 0.000105 & 0.000081 & 0.000059 & 0.000058 & 0.000035 \\
6 & -1.1358057224 & -1.1358539970 & 0.998532 & 0.000586 & 0.000235 & 0.000206 & 0.000105 & 0.000104 & 0.000081 & 0.000059 & 0.000059 & 0.000034 \\
7 & -1.1357563659 & -1.1358076807 & 0.998550 & 0.000561 & 0.000234 & 0.000212 & 0.000105 & 0.000106 & 0.000081 & 0.000059 & 0.000059 & 0.000033 \\
8 & -1.1358067602 & -1.1358554599 & 0.998528 & 0.000588 & 0.000237 & 0.000208 & 0.000104 & 0.000105 & 0.000080 & 0.000058 & 0.000059 & 0.000033 \\
9 & -1.1358025170 & -1.1358521235 & 0.998530 & 0.000583 & 0.000237 & 0.000213 & 0.000104 & 0.000103 & 0.000080 & 0.000059 & 0.000059 & 0.000034 \\
\hline\hline
\end{tabular}
\dlabel{Table:MCRenorm:1overNm1}
\caption{The results of 10 Monte-Carlo optimizations with $a=0$, showing the total energy after 1000 and 2000 
accepted Monte-Carlo steps, followed by the orbital occupation numbers of \ce{H2} in a 6-31G** basis
set. The Hartree-Fock energy is -1.131329765611, the Full-CI energy is -1.149414557325.}
\end{table}

In Table~\ref{Table:MCRenorm:1overNm1} the results for the $\frac{1}{N-1}$ renormalization are given. As it
underestimates the correlation we can try $\frac{2}{N}$ as a renormalization instead. The corresponding 
result is given in Table~\ref{Table:MCRenorm:2overN}.

\begin{table}
\begin{tabular}{|c|c|c|cccccccccc|}
\hline\hline 
ID & E$_{1000}$ & E$_{2000}$ & \multicolumn{10}{c}{occupation numbers} \\
\hline 
0 & -1.1499571633 & -1.1503162250 & 0.992160 & 0.003352 & 0.001269 & 0.000941 & 0.000551 & 0.000548 & 0.000399 & 0.000305 & 0.000305 & 0.000170 \\
\hline\hline
\end{tabular}
\dlabel{Table:MCRenorm:2overN}
\caption{The results of 1 Monte-Carlo optimizations with $a=1$, showing the total energy after 1000 and 2000 
accepted Monte-Carlo steps, followed by the orbital occupation numbers of \ce{H2} in a 6-31G** basis
set. The Hartree-Fock energy is -1.131329765611, the Full-CI energy is -1.149414557325.}
\end{table}

\section{Note on optimization algorithm}

To minimize the energy expression we need to deal with very steep gradients near occupation 
numbers of $0$ and $1$. Attempts of using a conventional linesearch method failed as this
does not remove redundant rotations. A conventional repeat diagonalization method also
failed as the energy often converges upwards instead of downwards, often resulting in
the Hartree-Fock solution instead of the correlated solution. 

In another attempt we explore a dynamic damping approach, is essentially a combination
of linesearch and repeat diagonalization approach. In this approach we start with a
wavefunction $\Psi_i(N,C)$ with corresponding energy $E(\Psi_i)$ from which we derive
Fock matrices $F(\Psi_i) = \partial E(\Psi_i)/\partial N,C$. Subsequently, we damp the
Fock matrix $F'_{ij} = F_{ij}d_{i\ne j}$ meaning we scale the off diagonal elements by
a factor $d \in (0,\inf)$. Diagonalizing $F'$ updates the wavefunction and we obtain
a new energy. At this point we have $E(\Psi_i)$, $F(\Psi_i)$, and $E(\Psi_{i+1})$ where
$F$ contains information about the gradient. This is sufficient to formulate a second
order polynomial and find its minimum. The problem is that the energy expression is
not quadratic in $d$ by any means. The energy expression is closest to quadratic in
terms of the occupation numbers and the square of the natural orbitals. These quantities
are dependent on $d$ but in a highly non-linear fashion.

In order to deal with this issue we need to map $d$ to the coordinates in which
$E$ is quadratic. Find the minimum of $E$ in those coordinates, and then find $d$ 
that matches the minimum, and use this value as the damping factor.

Lets look at the transformation.
We start with a 2x2 Fock matrix:
\begin{eqnarray}
   F = \left(\begin{array}{cc}
       F_{11} & F_{12} \\
       F_{21} & F_{22}
       \end{array}\right)
\end{eqnarray}
Diagonalization leads to eigenvectors
\begin{eqnarray}
   V = \left(\begin{array}{cc}
       c      & s      \\
      -s      & c     
       \end{array}\right)
\end{eqnarray}
where $c = cos(\phi)$ and $s = sin(\phi)$. 
In addition we have from the Jacobi transformation that
\begin{eqnarray}
   \cot(2\phi) &=& \frac{F_{22} - F_{11}}{2 F_{12}}
\end{eqnarray}
and $E$ is a function of $s^2 = b$.
In the current context that means that $b = b(d)$ where 
\begin{eqnarray}
   b(d) &=& \sin\left(\frac{1}{2}\arctan\left(\frac{2F_{12}d}{F_{22}-F_{11}}\right)\right)^2
\end{eqnarray}
The inverse of this expression is
\begin{eqnarray}
   d    &=& \frac{F_{22}-F_{11}}{2F_{12}}\tan\left(2\arcsin\left(\sqrt{b}\right)\right)
\end{eqnarray}
In these equations the factor $\frac{F_{22} - F_{11}}{2 F_{12}}$ is Fock matrix dependent
and different for every orbital pair. Hence we are better off suppressing this factor and
replacing it with $1$. This gives:
\begin{eqnarray}
   b(d) &=& \sin\left(\frac{1}{2}\arctan\left(d\right)\right)^2 \\
   d    &=& \tan\left(2\arcsin\left(\sqrt{b}\right)\right)
\end{eqnarray}
As for the scale of the gradient we have
\begin{eqnarray}
   \frac{\partial E(b(d))}{\partial d} 
   &=& \frac{\partial E(b(d))}{\partial b}\frac{\partial b(d)}{\partial d} \\
   \frac{\partial b(d)}{\partial d} 
   &=& \frac{\cos(\arctan(d)/2)\sin(\arctan(d)/2)}{d^2+1}
\end{eqnarray}

\section{Note on the wavefunction}

In section \ref{sect:FCI:H2:min} we investigated the form of the correlation energy.
Obviously as we have a wavefunction that represents the exact 1-electron density
matrix, all the 1-electron energy terms are also exactly represented. Therefore, the
piece that is missing is the exact representation of the 2-electron energy. We would
be able to represent this exactly if we had the exact 2-electron density. In
section~\ref{sect:FCI:H2:min} this density matrix is given by Eq.~\ref{eq:fci:h2}.
If we try to represent Eq.~\ref{eq:fci:h2} directly in terms of the wavefunction then we
have that
\begin{eqnarray}
C_{11}C_{11}^* &=& c_1^\alpha c_1^\beta c_1^{\alpha*}c_1^{\beta*} \\
C_{22}C_{22}^* &=& c_2^\alpha c_2^\beta c_2^{\alpha*}c_2^{\beta*}
\end{eqnarray}
In other words $C_{ij} \approx c_i^\alpha c_j^\beta$. One remaining problem lies in the
terms like $C_{11}C_{22}^*$ and $C_{22}C_{11}^*$ as these might have to have to negative
values, as was found in Eq.~\ref{Eq:T:subtract:2}. If the $\alpha$-correlation functions
equal the $\beta$-correlation functions, as one would expect from a closed shell system,
then a product $C_{11}C_{22}^*$ can only be negative if the coefficients $c_i^\sigma$ are
complex valued. Because if $c_i^\alpha = c_i^\beta$ then $C_{ii} = c_i^\alpha c_i^\beta = c_i^2$.
Therefore, if $C_{ii} < 0$ or $C_{ii}^* < 0$ then $c_i$ has to be complex if not purely
imaginary.

The next aspect is the evaluation of the 2-electron density matrix. If we evaluate it
according to the ensemble approximation (like for the 1-electron density matrix), then it
does not represent any correlation effects. In particular any entanglement effects are
entirely absent. Hence, for a reasonable 2-electron density approximation we have to 
assume a different approach. In fact, Eq.~\ref{Eq:D2:dij} can be interpreted as
a way to approximate the CI coefficients of the wavefunction. Instead of representing
the 2-electron density matrix as an ensemble approximation of the wavefunction we
can evaluate the 2-electron part as the proper expectation value of the wavefunction. 
Essentially, this may seem inconsistent as the 1-electron density matrix is represented
using an ensemble representation. However, for the 1-electron density matrix an ensemble
representation can be exact, for the 2-electron density matrix that is never true!
A challenge with evaluating the 2-electron terms as a proper expectation value
is the cost of doing so. Previously we tried to express this in terms of the 
1-electron density matrix which allowed key quantities to be evaluated outside of the
the 4-fold nested loops over integrals. If we cannot do that within a wavefunction approach
then the cost would spiral to $O(N^6)$ (4-fold loop over integrals, and for every integral
consider all possible pair combinations of the correlation functions). This would put
the cost of the method at the same level as CCSD but likely yield less accurate results.
Anyway, for the $\alpha\beta$-electron interaction we have
\begin{eqnarray}
   d_{ijkl}^{\alpha\beta} \approx d_{ik}^\alpha d_{jl}^\beta \\
   d_{ij}^\sigma = \sum_{t=1}^{n_e^\sigma} C_{it}^\sigma \left(C_{jt}^\sigma\right)^*
\end{eqnarray}
For the $\sigma\sigma$ interaction we get
\begin{eqnarray}
   d_{ijkl}^{\sigma\sigma} \approx
   \sum_{s,t=1}^{n_e^\sigma} C_{is}^\sigma \left(C_{ks}^\sigma\right)^* 
                             C_{jt}^\sigma \left(C_{lt}^\sigma\right)^*
                           - C_{is}^\sigma \left(C_{kt}^\sigma\right)^* 
                             C_{jt}^\sigma \left(C_{ls}^\sigma\right)^*
\end{eqnarray}



\bibliographystyle{unsrt}
\bibliography{WFN1C,WFN1C_alt}

\end{document}
