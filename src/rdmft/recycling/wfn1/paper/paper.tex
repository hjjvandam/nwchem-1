\documentclass{article}
\usepackage{doi}
\usepackage{natbib}

\title{Notes on density matrix functional theory}
\author{Hubertus J. J. van Dam}

\begin{document}
\maketitle

Nguyen-Dang~\cite{Nguyen_Dang_1985} device a scheme to optimize the
ensemble and pure-state density matrices. Lude\tilde{n}a uses this paper to
attack the work by Gilbert claiming that his "proper use of N-representability
conditions" does not lead to degeneracies of natural orbitals. However, in the
paper they insist that the pure-state density matrix is idempotent which in
my view is incorrect. Hence the results are most likely invalid.

Yasuda~\cite{Yasuda_2001} discussed an approach involving density and hole
density matrices. It is shown that the correlation energy for the hole density
matrix and the density matrix have to be the same. He proposes a functional in
Eq.(22). However when examining practical results he uses full-CI 1-electron
density matrices and other full-CI derived quantities. It is not clear how
much the results depend on using data from full-CI but this does not seem like
a practical approach. The paper also points to the importance of Levy's
homogeneous scaling condition.

Nagy~\cite{Nagy_2002} discussed an approach based on disjoint electron pairs.
The paper is purely theoretical without any results. Hence the quality of the
approach is impossible to judge.

Kollmar~\cite{Kollmar_2003} discussed an approach with disjoint electron pairs.
This will not work in general as the real wavefunction should incorporate all
electron pairs. It does contain the normalization condition of the electron
pair functions.

Ayers~\cite{Ayers_2007} discusses N-representability conditions of density
functionals. The definition is essentially that an N-representable density
functional cannot produce an energy that is below the true variational 
minimum. This is obviously true but I am not sure how this can be enforced
for that I would need to understand this work in more detail.

Canc\`{e}s~\cite{Cance_s_2008} proposed a linesearch based algorithm to 
optimize the solutions for DMFT. This is pretty similar to the linesearch 
approach I tried except they don't use occupation functions of course. 
They also claim that most of the Baerends density matrix functionals are not
proper 1-matrix functionals as they are formulated in terms of occupation
numbers and natural orbitals but not explicitly in terms of density matrices.
I am a bit puzzled about this. If one puts the Hartree-Fock wavefunction into
the Schr\"{o}dinger equation the exchange terms also does not have a density
matrix in it. However we clearly manage to rearrange the definition to get it
into such a form (although this may only be true for idempotent density
matrices).

Rohr~\cite{Rohr_2010} discussed a scheme where a density and a density matrix
functional are combined. They use range separation whereby the density 
functional can be applied for the short range part and the density matrix
functional for the long range part. The reason for this approach is to avoid
double counting of correlation effects and to account for the fact that
the dynamic correlation is mostly short range and the static correlation long
range. Nevertheless the accuracy does not seem to be great. Also they compare
against semi-empirical results from 1974, how accurate can those be?

Mazziotti~\cite{Mazziotti_2012,Mazziotti_2012_a} claims to have solved the
density matrix N-representability conditions. He does so by building a 
hierarchy of N-representability conditions which couple different orders
of density matrices. The downside is that fullCI equivalent results can be
obtained only at a fullCI equivalent cost. Polynomial cost results can be
obtained only by introducing approximations that break the hierarchy.

Baldsiefen~\cite{Baldsiefen_2013} deviced an optimization scheme where he 
computes a matrix to diagonalize for the natural orbitals. However for the
occupation numbers he uses a Fermi-smearing approach. This requires a
temperature tensor which needs a sensible (but automated choice) of
temperature to get it to work. This is weird and unnecesarrily complicated.

Lude\tilde{n}a~\cite{Lude_a_2013} discusses at great length the
N-representability of density functionals (which seems a bit of a strange
concept). He also reiterates his objections against Gilbert's theorem although
it still isn't clear to me why he gets different results.

Lathiotakis~\cite{Lathiotakis_2014} claims that the orbital optimization does
not reduce to an iterative eigenvalue problem (as in DFT) and hence is 
expensive. This is a direct contradiction to my experience! Otherwise they 
implement some OEP like approach. However, this introduces division by
orbital energy differences which seems very dangerous to me as there are bound
to be degeneracies.

\section{Ideas}

The one thing that still baffles me about density matrices is that somehow they
have no knowledge of fundamental particles. Yes, there are occupation numbers
and even limits on occupation numbers, but the density matrix gives no
information about how those occupation numbers are added up from contributions
of single particles. For the 1-matrix this is not such a problem as the
occupation numbers are limited to \$0 \le n \le 1\$ but for 2-matrices the
admissible range is (much) larger. I.e. a pair occupation number of \$1.2\$ 
might consist of \$1\$ pair plus \$0.2\$ of another pair, or it might consist
of two \$0.6\$ contributions from two pairs. Within a density matrix 
formalism there is no way of knowing which one is right.

In wavefunctions the number of elementary particles is obvious as each particle
introduces a new set of coordinates. Hence I am thinking that maybe we need
N-representability conditions based on orbitals and gemininals (orbitals have
to be orthonormal due to the anti-symmetry, geminals have to be normalized). 
The orbitals and geminals should consist of occupation functions and natural
orbitals/geminals to be able to generate physically sensible electron
distributions while at the same time being able to impose conditions on 
individual 1-electron and electron pair states. Such an approach would
eliminate the N-representability conditions that refer to occupation numbers
and replace them with conditions on the wavefunctions.

\bibliographystyle{plain}
\bibliography{paper}
\end{document}
